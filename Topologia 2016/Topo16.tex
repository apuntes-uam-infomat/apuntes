\documentclass[palatino]{apuntes}

\title{Topologia 2016}
\author{Pablo Pérez Manso}
\date{16/17 C1}

% Paquetes adicionales

% --------------------

\begin{document}
\pagestyle{plain}
\maketitle

\tableofcontents
\newpage
% Contenido.

\begin{defn}[Topología]
$X \neq \varnothing$, conjunto. Sea $\tau \in \mathcal{P}(X)$. Si $\tau$ satisface:
\begin{enumerate}
\item $\varnothing \in \tau, X \in \tau$
\item Sea $\{A_\lambda\}_{\lambda \in \Lambda}$ con $ A_\lambda \in \tau, \forall \lambda \in \Lambda \implies  (\bigcup_{\lambda \in \Lambda} A_\lambda) \in \tau$
\item $A \in \tau, B \in \tau \implies A \cap B \in \tau$
\end{enumerate}
\end{defn}



\begin{defn}[Abierto]
Sea $K \subset X$, diremos que K es cerrado para $\tau$ si $C_XK\in\tau$.
\end{defn}
\begin{defn}[Cerrado]
Sea $K \subset X$, diremos que K es cerrado para $\tau$ si $C_XK\in\tau$.
\end{defn}

\begin{example}
	Si $\tau = \mathcal{P}(X) \implies \tau $ es una topología ya que cumple las tres propiedades de una topología. (ejercico)

	\begin{itemize}
		\item Para $x_0 \in X, G=\{x_0\} \in \tau$, luego G es abierto para $\tau$
		\item Para $x_0 \in X, C_xG=X\setminus\{x_0\} \in \tau$, luego $C_xG$ es abierto para $\tau$
	\end{itemize}
\end{example}

\begin{defn}[Topología discreta]
$\tau$ es una topología discreta si todos los subconjuntos son abiertos
\end{defn}


\begin{example}
$\tau = \{\varnothing, X\} \in \mathcal{P}(X) \implies \tau$ es una topología. Cumple las tres propiedades (ejercicio)
\end{example}


\begin{defn}[Topología trivial]
$\tau$ es una topología trivial si sólo tiene dos subconjuntos $\{\varnothing,X\}$
\end{defn}











































%% Apendices (ejercicios, examenes)
\appendix

\chapter{---}
% -*- root: ../Topo16.tex -*-


\printindex
\end{document}
