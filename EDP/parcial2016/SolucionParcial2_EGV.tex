\documentclass[bibnumbers, palatino]{apuntes}

\title{Ecuaciones en Derivadas Parciales}
\author{Elena Gutiérrez Viedma}
\date{15/16 C2}

% Paquetes adicionales
\usepackage{tikztools}
\usepackage{fancysprefs}
\usepackage{enumitem}
\usepackage{fastbuild}
\usepackage{titlesec}
\usepackage{xfrac}
% --------------------





\titleclass{\subsubsubsection}{straight}[\subsection]

\newcounter{subsubsubsection}[subsubsection]
\renewcommand\thesubsubsubsection{}
\renewcommand\theparagraph{\thesubsubsubsection.\arabic{paragraph}} % optional; useful if paragraphs are to be numbered

\titleformat{\subsubsubsection}
  {\normalfont\normalsize\bfseries}{\thesubsubsubsection}{1em}{}
\titlespacing*{\subsubsubsection}
{0pt}{3.25ex plus 1ex minus .2ex}{1.5ex plus .2ex}

\makeatletter
\renewcommand\paragraph{\@startsection{paragraph}{5}{\z@}%
  {3.25ex \@plus1ex \@minus.2ex}%
  {-1em}%
  {\normalfont\normalsize\bfseries}}
\renewcommand\subparagraph{\@startsection{subparagraph}{6}{\parindent}%
  {3.25ex \@plus1ex \@minus .2ex}%
  {-1em}%
  {\normalfont\normalsize\bfseries}}
\def\toclevel@subsubsubsection{4}
\def\toclevel@paragraph{5}
\def\toclevel@paragraph{6}
\def\l@subsubsubsection{\@dottedtocline{4}{7em}{4em}}
\def\l@paragraph{\@dottedtocline{5}{10em}{5em}}
\def\l@subparagraph{\@dottedtocline{6}{14em}{6em}}

\precompileTikz

\setcounter{secnumdepth}{4}
\setcounter{tocdepth}{4}

\bibliographystyle{plainnat}

\begin{document}
\pagestyle{plain}

% http://tex.stackexchange.com/a/14243
%\relpenalty=10000
%\binoppenalty=10000
% -*- root: ../EDP2016.tex -*-
\section{Parcial 2 de E.D.P.}

\begin{problem} Tenemos la siguiente ecuación de onda. \[ \begin{cases}
u_{tt} - u_{xx} = 0	& x \in (0,4),\; t > 0 \\
u(0,t) = 0 = u(4,t) & t ≥ 0 \\
u(x,0) = f(x)\\
u_t(x,0) = 0 	& x \in (0,4)
\end{cases} \]
donde $f(x)$ tiene la siguiente expresión:
\[ \begin{cases}
-(x-3)(x-1)	& x \in [1,3]\\
0 & x \in (0,1)\cup(3,4)\\
\end{cases} \]

Se pide:
\begin{itemize}
\item Valor máximo y mínimo de $u$ en $t=1$ y el punto $x$ en el que se alcanza.
\item Valor máximo y mínimo de $u$ en $t=2$ y el punto $x$ en el que se alcanza.
\item Valor de la energía en cualquier instante $t$.
\end{itemize}
\end{problem}


\end{document}
