% -*- root: ../EDP2016.tex -*-
\section{Exámenes}

\subsection{Parcial 3 - 2012/2013}


\begin{problem}[1] Sea $A$ el anillo $\set{1 < x^2 + y^2 < 4}$. Sea $u$ la solución del problema
\[
\begin{cases}
	- Δ u = 0 & \text{en } A \\
	u(x,y) = 1 & x^2 + y^2 = 1 \\
	u(x,y) = 0 & x^2 + y^2 = 4
\end{cases}
\]

\ppart Hallar los valores máximo y mínimo de $u$ en el conjunto $\adh{A}$.
\ppart Dado el trozo de cono $c(x,y) = 2 - \sqrt{x^2 + y^2}$ definido en $A$, demostrar que $u(x,y) ≤ c(x,y)$.

\solution

\spart

Tenemos que $u$ es armónica, así que podemos aplicar el \nref{prop:MaximoFuerte} para ver que el mínimo y máximo de $u$ se alcanzan en la frontera. Según los datos del problema, esos valores son respectivamente $0$ y $1$.

\spart

Aplicamos el principio de comparación. Si calculamos el laplaciano de $c(x,y)$, tenemos que \[ Δc(x,y) = \frac{-1}{\sqrt{x^2 + y^2}}\] luego $-Δc ≥ -Δu = 0$, así que entonces $c  ≥ u$.

\end{problem}


\begin{problem} Dado el problema de la ecuación de onda
\[
\begin{cases}
	u_{tt} - u_{xx} = 0 & x∈ (0,∞),\, t > 0\\
	u(x,0) = 0 \\
	u_t(x,0) = 0 \\
	u(0,t) = t^4
\end{cases}
\] se pide:

\ppart Hallar el soporte de la solución en el instante $t = 2$.
\ppart Hallar la solución $u(x,t)$.
\ppart Encontrar una solución $v(x,t)$ de la ecuación de ondas anterior con los mismos datos iniciales pero con condición de contorno unilateral $v_x(0,t) = t^4$.
\hint{Hay varias posibilidades para abordar este problema. Se puede hacer un cambio de variables que pase a una condición de contorno homogénea y buscar una extensión adecuada a todo el espacio en el problema resultante; o estudiar y resolver la ecuación que satisface la función $w(x,t) = v_x(x,t)$.}

\solution

\spart

\begin{wrapfigure}{R}{0.3\textwidth}
\vspace{-15pt}
\inputtikz{Ex3Ej2}
\vspace{-15pt}
\caption{Características para la ecuación de onda.}
\label{fig:Ex3Ej2}
\end{wrapfigure}

Este problema lo podemos ver a través de las características. Dado que la condición inicial es $0$, el valor que nos da la ecuación $u(0,t) = t^4$ se propaga a través de rectas de la forma $x - t = k$. En particular, en el instante $t = 2$ el soporte será $[0, 2]$.

\spart

Siguiendo con la \fref{fig:Ex3Ej2}, tendremos que la solución será la siguiente: \[ u(x,t) = \begin{cases}
(t - x)^4 & x ≤ t \\
0 & x > t
\end{cases}\]

\end{problem}
