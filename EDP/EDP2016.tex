\documentclass[bibnumbers, palatino]{apuntes}

\title{Ecuaciones en Derivadas Parciales}
\author{G. Guridi, E. Miravalls, G. Julián}
\date{15/16 C2}

% Paquetes adicionales
\usepackage{tikztools}
\usepackage{fancysprefs}
\usepackage{enumitem}
\usepackage{fastbuild}
% --------------------

\precompileTikz

\setcounter{tocdepth}{3}

\bibliographystyle{plainnat}

\begin{document}
\pagestyle{plain}

% http://tex.stackexchange.com/a/14243
%\relpenalty=10000
%\binoppenalty=10000

\begin{abstract}
Estos son los apuntes del curso de Ecuaciones en Derivadas Parciales (2016), del profesor Jesús García Azorero. Es recomendable también mirar los apuntes del año 2015 \cite{ApuntesEDPRual}, que son del mismo curso aunque con un temario organizado de forma distinta.
\end{abstract}

\maketitle

\tableofcontents
\newpage
% Contenido.

% -*- root: ../EDP2016.tex -*-
\chapter{Introducción}

\begin{figure}[hbtp]
\centering
\inputtikz{TransmisionCalor}
\caption{Esquema de la varilla (en negro) aislada y conectada a dos bloques de hielo (temperatura 0)}.
\label{fig:TransmisionCalor}
\end{figure}

El primer ejemplo que vamos a ver tiene que ver con la ecuación del calor y la solución de Fourier. Consideraremos una varilla de longitud $L$ aislada, y conectada con dos bloques de hielo. Llamaremos $u(x,t)$ a la temperatura en el punto $x$ en el instante $t$. Sabemos que en $t = 0$ tenemos una distribución inicial de temperatura, y en $t \to ∞$ tenemos que la temperatura será cero (no llega a fundir completamente los bloques de hielo).

Como notación, escribiremos las derivadas parciales con subíndices: $u_t \equiv \pd{u}{t}$, $u_{xx} = \pd[2]{u}{x}$, y similar.

Para estudiar el problema, vamos a ver que ocurre en el intervalo $[x, x+h]$. La temperatura, al final nos da una medida de la energía de los puntos. Si la sumamos en ese intervalo obtendremos una cantidad de calor acumulada en ese intervalo. Es decir, $\int_x^{x+h} u(s,t) \dif s$ es la cantidad de calor en el instante $t$ en $[x,x+h]$. Lo interesante es que esa cantidad de calor varía a lo largo del tiempo. Como la varilla está rodeada de aislante, la cantidad de calor sólo se puede estar perdiendo a través de los extremos.

Suponemos que podemos meter la derivada dentro de la integral (habría que demostrarlo), así que podemos decir que \(  \od{}{t} \int_x^{x+h} u(s,t) \dif s = \int_x^{x+h} u_t(s,t) \dif s \label{eq:intro_1}\)

La difusión de la temperatura será, según la ley de Newton, proporcional a la temperatura en el punto que estemos considerando\footnote{Como no somos físicos, la constante es 1.}. Si en el extremo derecho tenemos una derivada positiva (con respecto a x), estamos aportando calor al intervalo. Si tenemos derivada positiva en el extremo izquierdo, estamos perdiendo calor en el intervalo.

Tenemos entonces que ese intercambio de calor es proporcional a \[u_x(x+h, t) - u_x(x,t)\]

Por la regla de Barrow\footnote{Corolario del Teorema Fundamental del Cálculo} podemos decir que \[ u_x(x+h, t) - u_x(x,t) = \int_x^{x+h} u_{xx}(s,t) \dif s\]
Con esto, hemos desarrollado el lado derecho de la ecuación \ref{eq:intro_1}, y pasando el lado derecho restando, combinando las integrales\footnote{Los límites de integración son los mismos así que lo podemos hacer.}, y multiplicando por $\frac{1}{h}$ (ahora veremos por qué) tenemos que
\( \frac{1}{h} \int_{x}^{x+h} u_t(s,t) - u_{xx}(s,t) \dif s = 0 \label{eq:intro_2}\)

Si definimos $$F(x) = \int_{0}^{x} u_t(s,t) - u_{xx}(s,t) \dif s $$ entonces \ref{eq:intro_2}, por definición de cociente incremental, es:
\( \ref{eq:intro_2} = \frac{F(x+h) - F(x)}{h} = 0  \label{eq:intro_3}\)

y aplicando el límite $h \to 0$ y por el TFC\footnote{Teorema Fundamental del Cálculo}, obtenemos que \ref{eq:intro_3} tiende a:
\[ F'(x) = u_t(s,t) - u_{xx}(s,t) \]
Luego como \ref{eq:intro_3} era igual a 0, tenemos que,
\[ u_t(x,t) - u_{xx} (x,t) = 0, \forall x \in (0,1), \forall t > 0\]

El problema que tendremos que resolver es entonces que
$$u(0,t) = u(L,t) = 0 \text{ para } t > 0$$

¿Cómo resolver esto? Buscaremos que $u(x,t) = X(x) · T(t)$, es decir, que podamos separar las variables. En ese caso, tenemos que la ecuación pasa a ser \[ u_t - u_{xx} = 0 \implies XT' - X''T = 0 \implies \frac{X''}{X} = \frac{T'}{T} \]

La igualdad se debe cumplir $∀x ∈ (0,L)$ y $∀t > 0$. Tenemos a dos ecuaciones que dependen de dos variables distintas y que son iguales, luego debe existir un $λ ∈ ℝ$ tal que $\frac{X''}{X} = \frac{T'}{T} = λ$, y nos movemos al mundo de las ecuaciones ordinarias, llegando al sistema  \[ \begin{matrix} T' - λT = 0 \\ X'' - λX = 0 \end{matrix}\]

El problema es que no sabemos qué es λ, así que tenemos que irnos al resto de condiciones del problema. Sabíamos que en los extremos la temperatura es 0, así que tenemos que añadir esas condiciones $X(0) = 0 = X(L)$. Sin embargo, lo que estudiábamos era el problema de Cauchy, en el que teníamos valores sobre la función y su derivada en el mismo punto. Este es un problema de contorno que se queda fuera del temario de EDO \cite{ApuntesEDO}. Como tenemos una solución trivial ($X = 0$), tendremos que buscar para qué valores de λ no se cumple el teorema de existencia y unicidad.

Lo que sí sabemos resolver es ver qué tipo de soluciones tenemos para $X'' - λX = 0$. Para ello, resolvemos el polinomio característico. Para $λ = 0$, tenemos que $X(x) = a+bx$. Como $X(0) = 0 = a$, y $X(L) = bL = 0$, tenemos que $X = 0$ y no nos vale porque es una solución trivial.

Buscamos ahora ver qué pasa con $λ > 0$. Podemos decir perfectamente que $λ = μ^2$. En este caso, tendríamos que las soluciones son $X(x) = ae^{μx} + be^{-μx}$. Sin embargo, las condiciones de contorno nos dirán que $a = b = 0$.

Nos falta probar qué pasa con $λ < 0$. Igual que antes, ponemos $λ = -μ^2$. Repasando de nuevo el curso de EDO, tenemos que $X(x) = a \cos μx + b \sin μx$. Con las condiciones de contorno tenemos que $X(x) = b \sin μx$ y con la constante $μ = \frac{kπ}{L}$ para $k = 1,2,3,\dotsc$.

Las soluciones serán entonces de la forma \[ X_k = b_k \sin (\frac{kπ}{L} x)\] correspondientes a los autovalores $λ_k = -\left(\frac{kπ}{L}\right)^2$. % \lambda < 0

Por otra parte, miramos qué ocurre con $T$: resolviendo tenemos que $T_k(t) = c_k e^{λ_k t}$, así que las soluciones particulares de la ecuación del calor son \[ u_k(x,t) = a_k e^{-\left(\frac{kπ}{L}\right)^2t} \sin (\frac{kπ}{L} x) \]

Pero no hemos terminado el problema: nos falta meter el dato inicial $u(x,0) = f(x)$. Por ejemplo, si $f(x) = 5 \sin (\frac{2π}{L} x)$ entonces $u(x,t) = 5 e^{-\left(\frac{2π}{L}\right)^2t} \sin (\frac{2π}{L} x)$.

Como la ecuación es lineal, también podemos sacar la solución para funciones que sean sumas de senos; y utilizando la propiedad de que dadas dos soluciones particulares, si las sumamos también es solución, podemos llegar a la idea de Fourier.

¿Nos vale con sumas de senos? La respuesta es que sí: Fourier dice que cualquier función se puede escribir como suma infinita de senos, y por lo tanto siempre podemos encontrar la solución: \[ f(x) = \sum_{k=1}^∞ a_k \sin \frac{kπ}{L} x\]

La cuestión es que ahí nos encontramos con problemas:
\begin{itemize}
	\item La serie es alternada y no sabemos si converge.
	\item No sabemos cómo se calculan los $a_k$.
	\item No sabemos exactamente qué significa que una función sea igual a una serie infinita.
	\item No sabemos si la derivada de la función es igual a la derivada de la derecha.
\end{itemize}

Todos estos problemas son los que motivan el análisis funcional y las matemáticas del siglo XIX. Nosotros los veremos en este curso en el \fref{chap:EcuacionesSegundoOrden}.


% -*- root: ../EDP2016.tex -*-
\chapter{Ecuaciones de primer orden}

	\section{Planteamiento del problema}
	\label{sec:PlanteamientoPrimerOrden}

	\begin{figure}[thbp]
	\centering
	\inputtikz{CarrilCoches}
	\caption{Un modelo simplificado de cómo se mueven los coches en un carril.}
	\label{fig:CochesCarril}
	\end{figure}

	Vamos a empezar con un modelo sencillo. Observemos el fenómeno del embotellamiento fantasma, el cual tiene una ecuación bastante sencilla. Veamos cómo se generan esos atascos y cómo se disipan.

	Estudiaremos una función $u(x,t)$ que nos dará la densidad de coches en un punto $x$ en un momento $t$.

	Primero, observamos un modelo muy simple en el que solo hay un carril y los coches fluyen solo en una dirección, como en la \fref{fig:CochesCarril}. La variación en el número de coches de un intervalo será:
	\[ \od{}{t} \int^{x+h}_{x} u(s,t) \dif s = \int^{x+h}_{x} u_t(s,t) \dif s \]

	Esto equivale a al número de coches que entran por $x$ menos los que salen por $x + h$. Es decir, tenemos un flujo $q(u, x, t)$. Si suponemos que nos fijamos en un instante lo suficientemente pequeño en un tramo pequeño podemos asumir que depende solo de $u$. En ese caso, si $q > 0$ el flujo sería hacia la derecha.
	\( q(u(x,t)) - q(u(x+h,t)) = -\left[ q(u(x+h,t)) -  q(u(x,t)) \right] \eqexpl{Barrow} -\int^{x+h}_{x} [q(u)]_{x}
	\label{eq:BaseFlujo}\)

	Por lo tanto, tendremos sistemas del tipo:
	\begin{equation}
	\left\{
	\begin{array}{rl}
	u_t + [q(u)]_{x} =&\!\!\! 0 \\
	u(x, 0) =&\!\!\! F(x) \quad \quad \text{dato}
	\end{array}
	\right. \label{eq:ModeloAtasco}
	\end{equation}

	Ahora debemos probar distintas funciones $q$ y tratar de resolver el sistema. Eso será precisamente lo que haremos a lo largo de este capítulo.

	\section{Modelos básicos: flujo proporcional a la densidad}
	\label{sec:ModeloBasicoFlujoProporcional}

		\begin{figure}[hbtp]
		\centering
		\inputtikz{CochesModeloLineal}
		\caption{Con un modelo de flujo proporcional a la densidad, los coches simplemente se desplazan manteniendo la misma densidad.}
		\label{fig:CochesModeloLineal}
		\end{figure}

		Tomando $q(u) = cu$ en \eqref{eq:ModeloAtasco}, nuestro sistema se convierte en
		\begin{equation}
		\left\{
		\begin{array}{l}
		u_t + cu_x = 0 \\
		u(x,0) = F(x)
		\end{array}
		\right. \label{eq:ModeloBasico}
		\end{equation}

		Esta ecuación dice que el flujo es proporcional a la densidad. Implica que los coches se mueven con velocidad constante, a más velocidad más flujo y viceversa. Si en el instante inicial tenemos una función de la densidad respecto de las zonas del tramo, según pasa el tiempo la densidad se irá desplazando hacia la derecha manteniendo la misma forma (\fref{fig:CochesModeloLineal}).

		Podemos comprobar esta idea intuitiva. Si lo que hacemos es ``desplazar'' la densidad, entonces lo que tenemos es que $u(x,t) = F(x-ct)$. Vamos a comprobar que la ecuación \eqref{eq:ModeloBasico} sigue cumpliéndose con esta definición. Primero derivamos \[ u_x(x,t) = F'(x-ct) \qquad u_t(x,t) = F'(x-ct) · (-c) \] y luego sustituyendo nos queda directamente: \begin{align*}
			u_t + c·u_x &= 0 \\
			-c·F'(x-ct) + c·F'(x-ct) &= 0
		\end{align*}

		\subsection{Representación gráfica: Curvas características}

			Para entender mejor el problema lo que haremos será dibujarlo con un método que nos valdrá para muchos otros problemas. La idea es dibujar los ``conjuntos de nivel'' de la solución, y para eso usaremos el concepto \concept[Curvas\IS características]{curvas características}, curvas a lo lago de las cuales el dato (en este caso, la densidad) se mantiene constante.\footnote{Porque el problema es homogéneo.}

			En este problema, el dato inicial de la densidad se propaga con velocidad $ct$, luego lo que tendremos son rectas $x - ct = k$.

			Si lo hiciésemos sustituyendo en las ecuaciones nos saldría lo mismo: si $u(x,t)$ es constante, también lo es $F(x-ct)$ y esto en general sólo ocurrirá cuando $x - ct = k$. En cualquiera de los dos casos, el dibujo sería algo parecido al de la \fref{fig:ConjNivelCochesSimple}.

			\begin{figure}
			\centering
			\inputtikz{ConjNivelCochesSimple}
			\caption{Las rectas diagonales son las curvas características, y en cada región se mantiene la misma densidad que había en el primer momento $t = 0$.}
			\label{fig:ConjNivelCochesSimple}
			\end{figure}



	\subsection{Modelo con curvas dato alternativas}
		\label{sec:CurvaDatoRara}

		Imaginemos un río con una velocidad constante, en el que a partir de un punto realizamos un vertido. El río tiene velocidad $v$ y en ${x=0}$ realizaremos una contaminación $\beta(t)$. En ${t=0}$ consideraremos que está limpio. Usamos el mismo modelo que antes, pero con la peculiaridad de que la curva dato la estamos dando de otra forma: antes dábamos el dato inicial como lo que ocurría cuando ${t = 0}$. Aquí, damos una curva dato para $t > 0$ y $x > 0$.

		En este caso, las ecuaciones que tenemos son las siguientes:
		\begin{align*}
		u_t + vu_x &= 0 \\
		u(x,0) &= 0 \qquad x>0 \\
		u(0,t) &= \beta(t)\quad t>0
		\end{align*}

		Resolvemos el problema igual que en el caso anterior: buscamos los conjuntos de nivel, que nos volverán a salir rectas.

		\textbf{Buscamos $x(t)$ tal que $u(x(t),t)$ sea constante}

		Derivando en t la ecuación anterior obtenemos: $u_x x' + u_t = 0$ que junto con la ecuación $u_t + v u_x = 0$ nos da que $x' = v \Rightarrow x = x_0 + vt $.

		Las características saldrían \(x-vt = \text{Cte}(=x_0) \label{eq:rio_vcte}\)

		\begin{figure}[hbtp]
			\centering
			\inputtikz{ContaminacionRio}
			\caption{La zona roja indica contaminación, y la zona verde implica río limpio. Ambas están divididas por la recta $x - vt = 0$, que indica cuándo una zona del río se contamina. La curva dato está dada en las zonas marcadas en azul y naranja.}
			\label{fig:ContaminacionRio}
		\end{figure}

		Esas son nuestras rectas conjuntos de nivel, que en este caso indicarían el frente de la contaminación, el límite a partir del cual el río sigue limpio.

		$$u(x,t) =
			\begin{cases}
				0                      & (x-vt) > 0 \\
				\beta(t - \frac{x}{v}) & (x-vt) < 0
			\end{cases}
		$$

		El valor de $u(x,t)$ cuando $x-vt < 0$, se obtiene utilizando \ref{eq:rio_vcte} y tomando el punto de corte de la recta $x-c \cdot t=k$ que pasa por $(0,t^*)$:

		$$
		\begin{rcases}
			0 - vt^{*} = k \\
			x - vt = k
		\end{rcases}
		 \Rightarrow x-vt = -vt^{*} \iff t^* = \frac{x-vt}{-v} = t - \frac{x}{v}$$

		Por lo que
		$$u(x,t) = \beta(t^*) = \beta(t - \frac{x}{v}), \quad \text{ cuando } x - vt < 0$$

		De esto obtenemos una función de $x$ en función del tiempo que nos permite saber cuándo una parte del río se contamina.

	\subsection{Modelo con descomposición biológica}

	De momento, hemos considerado sólo modelos en los que el flujo que entraba en un segmento era igual al que salía por otro lado. Sin embargo, ¿qué ocurre cuando además tenemos algún tipo de descomposición externa? Veámoslo primero con un ejemplo, siguiendo el modelo anterior del río.

		Supongamos que existe descomposición biológica, de tal forma que en cada punto del río hay bacterias que ``quitan'' una parte de la contaminación proporcional (coeficiente γ) al flujo. En ese caso, el sistema de ecuaciones es \[
		\begin{cases}
		u_t + vu_x = -\gamma u \\
		u(x,0) = \text{Cte} \\
		u(0,t) = \beta
		\end{cases} \]

		Para resolverlo, podemos hacer un cambio de variable. Para ello, comenzamos multiplicando por $e^{\gamma t}$:
		$$u_t + \gamma u + vu_x = 0$$
		$$e^{\gamma t} u_t + e^{\gamma t} \gamma u + e^{ \gamma t} v u_x = 0 $$

		lo que es lo mismo
		\[ (e^{\gamma t}u)_t + v (e^{\gamma t} u)_x = 0 \]

		Y definimos la función $W$:
		$$W = e^{\gamma t}u$$


		Hay que comprobar cuál es el efecto del término de descomposición biológica ($-\gamma u$), y calcular $W(x,0)$ y $ W(0,t)$. Esto queda como ejercicio para el lector.


	\subsection{Modelo general para flujo lineal}
	\label{sec:ModeloGeneral}

		Vamos a tratar de unir todo lo que tenemos hasta ahora. En todos los casos (el del atasco y los dos del río) nuestras ecuaciones han sido de la forma \[
		\begin{cases}
		u_t + vu_x = f(x,t) \\
		u(x,0) = F(x) \\
		\end{cases} \]

		Cuando $f \not\equiv 0$ lo que nos ocurría es que la solución no es constante a lo largo de las características, aunque seguirán estando ahí y las podremos usar para hacernos una idea del problema.

		¿Cómo solucionamos el problema? Hay dos posibilidades.

		\subsubsection{Cambio de variable}

			La primera idea es hacer un cambio de variables para funcionar en una única variable, $z = x-vt$, que hace más sencillo observar la variación temporal. Sustituyendo la $x$, tenemos que \begin{align*}
			x &= z + vt\\
			W(z,t) &= u(z+vt, t)
			\end{align*}

			Para reconstruir $W$ (y por tanto la solución) lo que vamos a hacer es fijarnos en que la solución en un punto depende del dato inicial en ese punto y luego de cómo ha evolucionado a lo largo del tiempo. Es decir, \[ W(z,t) = W(z,0) + \int_{0}^t W_t(z,τ) \dif τ \]

			No parece que hayamos avanzado mucho, pero si calculamos $W_t$ \[
				W_t(z,t) = u_x(z + vt, t)v + u_t(z+vt, t) = f(z + vt, t)
			\] ahora podemos sustituir y esperar que nos salga algo interesante. De hecho, lo hace:
			\[ W(z,t) = W(z,0) + \int^{t}_{0} f(z+v\tau, \tau)\dif\tau \]

			Porque si ahora deshacemos el cambio de variable $z = x -vt$ nos queda que \begin{align}
			W(z,t) &= W(z,0) + \int^{t}_{0} f(z+v\tau, \tau)\dif\tau \nonumber \\
			u(z + vt, t) &= u(z,0) + \int^{t}_{0} f(z+v\tau, \tau) \dif \tau \nonumber \\
			u(x,t) &= u(x-vt,0)+ \int^{t}_{0} f(x-v(t-\tau),\tau) \dif \tau \label{eq:ModeloCombinado}
			\end{align} donde $u(x-vt,0) = F(x-vt)$.

			Esto es un milagro ya que tiene una fórmula explícita, que es poco común en EDOs y EDPs. Es posible encontrar algo mejor.

		\subsubsection{Principio de Duhamel}
		\label{sec:PrincipioDuhamel}

			Una segunda solución es aplicar el \concept{Principio\IS de Duhamel}, que se basa un poco en la misma idea que antes: que el valor de $u$ en un punto depende de cómo ha ido evolucionando a lo largo del tiempo, y a ese resultado se le puede sumar el dato inicial (la suma de soluciones es solución).

			Tendremos entonces que \[ u = \varphi + \psi \] lo que nos deja dos sistemas:

			\begin{minipage}{.5\linewidth}
				\[
				\begin{cases*}
					\varphi_t + v\varphi_x = 0\\
					\varphi(x,0) = F(x)
				\end{cases*}
				\]
			\end{minipage}
			\begin{minipage}{.5\linewidth}
				\[
				\begin{cases*}
					\psi_t + v\psi_x = f(x,t) \\
					\psi(x,0) = 0
				\end{cases*}
				\]
			\end{minipage}

			La comprobación de que ambos sistemas son equivalentes es trivial y se deja como ejercicio para el lector.

			El primer sistema ya lo hemos resuelto en la \fref{sec:ModeloBasicoFlujoProporcional}, y la solución era ${φ(x,t) = F(x - vt)}$. Para el segundo, aplicaremos la idea que hemos visto antes, que es ver que el valor de la solución en un punto dependerá de la aportación externa que ha habido en él a lo largo del tiempo. Lo malo es que no podemos simplemente integrar $f$, que sería la idea evidente.

			Lo que ocurre es que hay una función, que llamaremos ξ y que, por así decirlo, ``transporta'' las aportaciones externas. Dependerá de $x$ y de $t$, el momento en el espacio y tiempo en el que queremos saber la aportación externa, pero también de otro parámetro $s$, que será el ``momento'' en el que aparece la aportación externa que estamos considerando. Por ejemplo, cuando $t = s$ la aportación externa viene dada por $f(x,t)$, luego ha de ser $ξ(x,s; s) = f(x,s)$.

			Esta función tiene que cumplir la ecuación anterior $ξ_t + v ξ_x = 0$, luego el sistema que tenemos es \( \begin{cases}
			ξ_t + vξ_x = 0 \\
			ξ(x,s;s) = f(x,s)
			\end{cases} \label{eq:SistemaDuhamel} \)

			Como decíamos, la función ξ transporta las aportaciones externas, luego \( ξ(x,t; s) = f(x_0, s) \label{eq:DuhamelExpr1} \) donde $x_0$ es el punto donde ``surgió'' esta aportación externa, que dependerá del punto $x$ que consideremos y también del tiempo que ha pasado ($s-t$).

			Por suerte para nosotros, el sistema \eqref{eq:SistemaDuhamel} sabemos resolverlo: en él, las soluciones son constantes a lo largo de las rectas características. Y como $$ξ(x,t;s) = f(x_0, s) = ξ(x_0,s;s)$$ tenemos que $(x,t)$ y $(x_0, s)$ tienen que estar en la misma recta. Así podemos sacar $x_0$:
			\[
			\begin{rcases}
				x-vt = k\\
				x_0 - vs = k
			\end{rcases}
			 \implies x-vt = x_0 - vs \implies
			x_0 = x-v(t-s) \]

			Tal y como habíamos conjeturado al principio, ese punto inicial depende del que estamos considerando y de la diferencia de tiempo $s-t$. Sustituyendo ahora en \eqref{eq:DuhamelExpr1} tenemos la solución para $ξ$:
			\[ \xi(x,t) = f(x - v(t-s), s) \]

			Tenemos la solución correspondiente de un aporte instantáneo en el punto $t$. Por lo tanto la solución será la suma de todos los aportes instantáneos:
			\[  \psi(x,t) = \int^{t}_{0} f(x-v(t-s),s) \dif s \]

			Sólo falta sumar ambas soluciones para tener la solución final, que será \(
			u(x,t) = F(x-vt) + \int_0^t f(x-v(t-s),s) \dif s \label{eq:SolDuhamel} \)

			Aunque hemos hecho un montón de suposiciones que tendremos que comprobar en algún momento.

	\section{Modelos con flujo no lineal}
	\label{sec:ModeloTraficoRealista}

		\begin{wrapfigure}{R}{0.4\textwidth}
			\centering
			\vspace{-15pt}
			\inputtikz{parabola}
			\caption{El flujo no es lineal con respecto a la densidad: si hay muchos coches, acaban parándose.}
			\label{fig:parabola}
		\end{wrapfigure}

		En los ejemplos anteriores de tráfico no hemos tenido en cuenta que la velocidad de los coches no es totalmente proporcional a la densidad. Si ésta baja mucho, los coches llegan a pararse. Hay una densidad máxima donde el tráfico quedará totalmente estancado (los coches pegados).

		Una función más realista sería una parábola (\fref{fig:parabola}). Si no hay coches no pasa ninguno, pero a partir de un punto cuantos más coches, menos rápido van, y menos pasan. El flujo vendría dado entonces por la ecuación \[
		q(u) = Au (B-u) = ABu - Au^{2} \]

		Derivando en $x$, tenemos que \[ [q(u)]_x = (AB - 2Au) u_x\] y sustituyendo esto en nuestra ecuación genérica del modelo de un atasco \eqref{eq:ModeloAtasco} nos queda el siguiente sistema: \[ \begin{cases}
		u_t + (AB - 2Au) u_x = 0 \\ u(x,0) = F(x)\end{cases}\]

		La velocidad va a ser dependiente de la densidad, así que nos saldrán características cuya pendiente dependa del valor en el tiempo inicial. Si divergen no habrá problema, pero si convergen habrá puntos donde rectas de densidad 2 (por ejemplo) cortarán con rectas de densidad 1. Esto se entiende como coches que van más rápido que se encuentran con coches que van más lentos y se ven obligados a frenar.

		Para simplificar los cálculos de este modelo asumamos que $A = B = 1$. Por lo tanto ahora estudiaremos:
		\begin{equation*}
			\left\{
			\begin{array}{l}
				u_t + (1-2u)u_{x} = 0 \\
				u(x, 0) = F(x) \quad \quad \text{dato}
			\end{array}
			\right.
		\end{equation*}

		\begin{wrapfigure}{R}{0.4\textwidth}
			\centering
			\vspace{-15pt}
			\inputtikz{rectasDivergentes}
			\vspace{-15pt}
			\caption{La solución se propaga en rectas cuya pendiente depende del valor del dato inicial $F(x)$.}
			\label{fig:rectasDivergentes}
		\end{wrapfigure}

		Obtengamos las características, las curvas a lo largo de las cuales la solución es constante ${u(x,t) = k}$, luego derivando tenemos que ${u_x x' + u_t = 0}$. Viendo el sistema anterior, tiene que ser
		\[ x' = 1 - 2u \eqexpl{$u=k$} 1 - 2k \] y por lo tanto nuestras características serán
		\( x - (1-2k)t = x_0\quad,\quad k=F(x_0)  \label{eq:caracteristicas_realistas} \)

		Nos fijamos en un punto en el instante inicial. Con el valor de ese punto calculamos la pendiente. A lo largo de las características el valor del dato se mantiene. En este caso las características no son paralelas porque la pendiente es $1-2k$, es decir, depende de $k$, como en la \fref{fig:rectasDivergentes}.

		Para ver qué casos posibles hay (por ejemplo, ¿qué pasa cuando se cortan dos rectas características?) vamos a ver algunos ejemplos según el valor de $F(x)$.

		\begin{example}[Semáforo] \label{ejm:Semaforo}
			\begin{wrapfigure}[8]{L}{0.4\textwidth}
				\centering
				\vspace{-15pt}
				\inputtikz{semaforoCerrado}
				\vspace{-15pt}
				\caption{Semáforo cerrado inicialmente.}
				\label{fig:semaforoCerrado}
			\end{wrapfigure}

			El primer caso es sencillo: un semáforo en $t = 0$ está cerrado. Su ecuación será \[
			F(x) =
			\begin{cases}
				1 & x < 0 \\
				0 & x > 0
			\end{cases}
			\]

			Y en ${t>0}$ se abre y pueden pasar los coches.\\


			En la sección izquierda, $F = 1$ luego $x - (1 - 2·1)t = x + t = k \Rightarrow x = k -t $, con $k$ constante.

			En la derecha, con cuentas análogas, $F = 0 \Rightarrow x-t = k$.

			\begin{figure}[htbp]
				\centering
				\begin{subfigure}[b]{0.49\textwidth}
				\inputtikz{caracteristicasSemaforo}
				\vspace{-8pt}
				\caption{Rectas características.}
				\end{subfigure}
				\begin{subfigure}[b]{0.49\textwidth}
				\inputtikz{semaforoT}
				\caption{Un corte de la solución para $t = T$.}
				\end{subfigure}
				\caption{En el modelo del semáforo, como no tenemos la función definida en $x = 0$ hay una zona (la zona naranja) en la que no tenemos información de lo que ocurre.}
				\label{fig:caracteristicasSemaforo}
			\end{figure}

			Vemos que las características se abren y se separan desde $x=0$ hacia afuera. Por lo tanto, las características no nos dan información en la región entre las rectas características que parten del $0$, ya que ninguna característica pasa por ese espacio.

			Esto pasa porque no está definida la función en el $0$. Si lo estuviera, podríamos intentar sacar infinitas características desde el 0 que rellenaran la zona en la que no tenemos información.

		\end{example}

		\begin{example}[Frenazo]
			\begin{wrapfigure}[10]{R}{0.4\textwidth}
				\centering
				\inputtikz{modeloAtasco}
				\vspace{-10pt}
				\caption{Dato de densidad para $t=0$ en el modelo del atasco.}
				\label{fig:modeloAtasco}
			\end{wrapfigure}

			Imaginemos ahora dos zonas, una con densidad de coches baja y otra con densidad alta (se encuentran con un atasco). La función sería  \[
			F(x) =
			\begin{cases}
				1/5 & x < 0 \\
				2/3 & x > 0
			\end{cases}
			\]

			Si dibujamos las características con \eqref{eq:caracteristicas_realistas} nos saldrá algo como lo de la \fref{fig:ondaChoque}. Ahí vemos casos en los que cuando recorremos una recta característica con valor $\frac{1}{5}$ nos encontramos con otra de valor $\frac{2}{3}$. Momento en el cual se produce el frenazo.

			Fijando $T$ podemos mirar el valor de cada $x$ trazando su correspondiente característica. Ahí es donde nos encontraremos los puntos en los que las rectas cortan. La sucesión de puntos de corte en función de $t$ es la curva que nos interesa. Se corresponde a la función que dictamina cómo se van encendiendo las luces rojas de los coches según se van parando al llegar a un atasco.

			\begin{figure}[bhp]
				\centering
				\inputtikz{ondaChoque}
				\caption{Cuando las características se cortan aparece una {\bf onda de choque}.}
				\label{fig:ondaChoque}
			\end{figure}\index{Onda!de choque}

			\obs las características terminan cuando se cortan con la onda de choque.

		\end{example}

		\subsection{Soluciones generalizadas. Ecuación de Rankine-Hugoniot}

		Intentemos resolver la curva de puntos de corte. Volvemos a la ecuación \[ u_t +m [q(u)]_x = 0 \]

		Como tiene derivadas y las funciones tienen discontinuidades, no se cumplen las hipótesis de regularidad que asumimos al partir del planteamiento integral del problema. Por tanto, debemos volver al planteamiento original, y si lo resolvemos obtendremos \concept{Soluciones\IS débiles}. Esta ecuación venía de la siguiente integral: \[
			\frac{1}{h} \int^{x+h}_{x} u_t + [q(u)]_x = 0
		\] que a su vez venía de \eqref{eq:BaseFlujo}: \[
			\od{}{t} \int^{x+h}_{x} u = q(u(x))-q(u(x+h))
		\]

		\begin{figure}[tp]
			\centering
			\inputtikz{discontinuidadAtasco}
			\caption{Las funciones que tratamos en los dos ejemplos anteriores se pueden modelar de esta forma: $C^1$ a trozos con una única discontinuidad de salto.}
			\label{fig:discontinuidadAtasco}
		\end{figure}

		La función $u$ es $C^1$ a trozos con discontinuidad de salto en $x=s(t)$. Así que llegamos al problema:	\textbf{hallar $s(t)$}, que será la ecuación de la curva de choque.

		El problema de verdad lo tenemos en los intervalos que contienen a la discontinuidad. Consideraremos entonces el intervalo $[x_1, x_2]$ como en la \fref{fig:discontinuidadAtasco}, con $x = x_1$ y $x_2 = x + h$, y veremos qué ocurre cuando $h \to 0$. La primera tarea obvia es descomponer la integra en las dos partes continuas, de tal forma que nos queda \(
		\od{}{t} \left[ \int^{s(t)}_{x_1} u(x,t) \dif x + \int^{x^2}_{s(t)} u(x,t) \dif x \right] = q(u(x_1)) - q(u(x_2)) \label{eq:DescompIntervDiscontinuidad} \)

		Aquí nos encontramos con un \textit{show-stopper}: \[ \od{}{t} \int^{s(t)}_{x_1} u(x,t) \dif x \]

		¿Cómo derivamos con respecto a $t$ una integral de una función que depende de $t$ y cuyo límite de integración también depende de $t$? Para arreglar el problema, vamos a definir una nueva función: \[ G(z,t) = \int^{z}_{x_1} u(x,t) \dif x \]

		Esta función sí podemos derivarla: \[
		G_z(z,t) = u(z,t)
		\qquad
		G_t(z,t) = \int^{z}_{x_1} u_t(x,z) \dif x
		\]

		También podemos recuperar la integral original \[ \int^{s(t)}_{x_1} u(x,t) dx = G(s(t),t) \]

		Con esto podemos derivar, no sin antes introducir una pequeña notación para denotar si nos acercamos a la discontinuidad por la derecha o por la izquierda:
		\begin{align*}
		[u(s(t),t)]^{-} &= \lim_{x \to s(t)^{-}} u(x, t) \\
		[u(s(t),t)]^{+} &= \lim_{x \to s(t)^{+}} u(x, t)
		\end{align*}

		Ahora sí, derivamos:
		\begin{align*}
		\od{}{t} \int_{x^1}^{s(t)} u(x,t) \dif x
			&= \od{G(s(t), t)}{t} = \\
			&= s'(t) · G_z(s(t), t) + G_t(z,t) = \\
			&= s'(t) · \left[u(s(t), t)\right]^{-} + \int_{x_1}^{s(t)} u_t(x,t) \dif x
		\end{align*}

		Si realizamos lo mismo con la otra integral\footnote{Ejercicio para el lector.} de \eqref{eq:DescompIntervDiscontinuidad}, llegamos a esta \textbf{conclusión}:
		\begin{multline*}
		\left(\left[u(s(t),t)\right]^{-} - \left[u(s(t),t)\right]^{+}\right)· s'(t) + \int^{s(t)}_{x_1} u_t(x,t) \dif x + \int^{x^2}_{s(t)} u_t(x,t) \dif x = \\ = q(u(x_1,t)) - q(u(x_2,t))
		\end{multline*}

		Ahora sólo falta acercar $x_1$ y $x_2$ a la discontinuidad con $x_1 \to s(t)^{-}$, y $x_2 \to s(t)^{+}$; y ver qué pasa. Las integrales se irán a cero al hacer el intervalo de integración cada vez más pequeño y el \textbf{resultado} será el siguiente:

		\[ \left[ u(s(t),t)^{-} - u(s(t),t)^{+} \right] · s'(t) = q(u(s(t),t)^{-}) - q(u(s(t),t)^{+}) \] luego
		\( s'(t) = \frac{q(u(s(t),t)^{-}) - q(u(s(t),t)^{+})}{u(s(t),t)^{-} - u(s(t),t)^{+}} \label{eq:DerivadaOndaChoque} \)

		En otras palabras, que la variación de la curva de choque $s$ (llamada \concept{Ecuación\IS de Rankine - Hugoniot}) es el cociente entre el salto en el flujo $q$ y el salto en la densidad $u$. Esta curva es precisamente la que vimos en la figura \ref{fig:ondaChoque}.

		\noindent Vamos a obtener la fórmula explícita utilizando los datos dados en la figura \ref{fig:modeloAtasco}:

		El salto en $u$ es $\frac{2}{3} - \frac{1}{5} = \frac{7}{15}$

		Y en $q$ es $\frac{2}{3} (1-\frac{2}{3}) - \frac{1}{5}(1-\frac{1}{5}) = \frac{2}{9} - \frac{4}{25} = \frac{14}{225} $

		Finalmente, sustituimos y vemos que \[
		s'(t) = \frac{\text{Salto q}}{\text{Salto u}} = \frac{\frac{14}{225}}{\frac{7}{15}} = \frac{14.15}{7.225} \equiv \alpha \quad(>0) \]

		\begin{wrapfigure}[10]{R}{0.4\textwidth}
			\centering
			\vspace{-15pt}
			\inputtikz{OndaChoqueRPM}
			\vspace{-10pt}
			\caption{Con nuestro modelo, tendremos dos regiones $R^-$, $R^+$ y una onda de choque $s(t)$ que las separa.}
			\label{fig:OndaChoqueRPM}
		\end{wrapfigure}

		Como $s(0) = 0$, la ecuación de la curva será \[
		s(t) = \alpha t \]
		por lo que la onda de choque sigue $x = s(t) = \alpha t$.

			\noindent {\bf Resumiendo} un poco, lo que hemos logrado ha sido ver esto:

			\begin{itemize}[itemsep = 1pt]
				\item $ u \in C^1 \text{ en } R^{-} \text{ y en } R^{+}. $
				\item $u$ tiene una discontinuidad de salto a lo largo de la curva $x = s(t)$.
				\item La curva satisface la condición de Rankine-Hugoniot: $s' = \frac{\text{Salto q}}{\text{Salto u}}$.
			\end{itemize}

			Vamos a usarlo para resolver el ejemplo del semáforo de la página \pageref{ejm:Semaforo}, en el que habíamos llegado a que había una zona en la que no sabíamos qué ocurría.

			\begin{example}[Semáforo - Pendiente de resolución]
				El sistema que teníamos era el siguiente:
				\[
				\begin{cases} u_t + (1-2u)u_x = 0 \\
				u(x,0) = F(x)
				\end{cases} \]

				\obs Es importante recordar que el método de las curvas características exige que el lado derecho sea 0 (problema homogéneo). Si no, hay que aplicar el principio de Duhamel, el modelo con descomposición, etc.

				\begin{figure}[hbtp]
					\centering
					\inputtikz{caracteristicasSemaforo}
					\caption{Rectas características con una región en la que no sabemos qué ocurre.}
					\label{fig:caracteristicasSemaforo2}
				\end{figure}

				Las características son las curvas $(x(t),t)$ tales que $u(x(t),t) = k$, que ya habíamos calculado y quedaban $x - (1-2k)t = x_0$, donde $x_0$ era el punto de partida y $k = F(x_0)$ el valor que se propagaba. El dibujo al que llegábamos era el de la \fref{fig:caracteristicasSemaforo2}.

				\begin{wrapfigure}{L}{0.3\textwidth}
					\centering
					\inputtikz{FEpsilon}
					\caption{Usaremos esta aproximación para saltarnos la discontinuidad}
					\label{fig:FEpsilon}
				\end{wrapfigure}

				Para calcular la región intermedia usaremos esta \textbf{idea}: Una aproximación y paso al límite. Imaginemos que $F_\epsilon$ es continua y decreciente, como en la \fref{fig:FEpsilon}. Entonces en la región intermedia:
				\[ u_\epsilon (x,t) = F_\epsilon (x_\epsilon) \equiv k_\epsilon \]

				Las características serán,  $x - (1-2k_\epsilon) t = x_\epsilon$, con $F_\epsilon(x_\epsilon) = k_\epsilon$, y lo que querremos hallar será $\lim\limits_{\epsilon \rightarrow 0} u_\epsilon(x,t)$. Para eso, despejaremos $k_ε$ que es el valor que se propaga en esa región.
				\begin{gather*}
				\frac{x - x_\epsilon}{t} = 1-2k_\epsilon \\
				k_\epsilon = \frac{1}{2} \left(1 - \frac{x-x_\epsilon}{t}\right) \convs[][\epsilon][0] \frac{1}{2} \left(1 - \frac{x}{t} \right)
				\end{gather*}

				Con esto, llegamos a una fórmula para la densidad en la región intermedia:
				\[ u(x,t) = \frac{1}{2} \left(1 - \frac{x}{t} \right) \quad \text{ si } (x,t) \text{ está en la región intermedia. } \]

				\begin{figure}[hbtp]
					\centering
					\inputtikz{SolucionTFijo}
					\caption{Solución para un $T$ fijo.}
					\label{fig:SolucionTFijo}
				\end{figure}

				Así, llegaríamos a la siguiente expresión general para un tiempo $T$ fijo
				\( u(x,T) =
				\begin{cases}
					1 & x \leq -T \\
					\frac{1}{2} \left(1 - \frac{x}{T}\right) & -T < x < T \\
					0 & x \geq T
				\end{cases} \label{eq:SolucionRarefaccion}
				\)

				A esto se le llama la \concept{Onda\IS de rarefacción}. Este resultado no depende de la aproximación que tomemos de $F_\epsilon$, solo que sea continua ya que con el paso al límite la función desaparece.
			\end{example}


	\subsection{Cambio de variables. Ecuación de Burgers}
	\label{sec:EcuacionBurgers}

		Vamos a ver otro método para la resolución de este tipo de sistemas. Partimos de la ecuación
		\[ u_t + [q(u)]_x = 0 \quad (q \in C^2) \]

		La derivada del flujo con respecto a $x$ tendrá la forma $[q(u)]_x = V(u) u_x$, de tal forma que podemos convertir nuestra ecuación en \(
		u_t + V(u)u_x = 0 \label{eq:burgers1} \) sobre la que podemos realizar el siguiente cambio de variables
		\begin{align*}
		W &= V(u) \\
		W_t &= V'(u) \cdot u_t \\
		W_x &= V'(u) \cdot u_x
		\end{align*} y ver qué ecuación cumple $W$.

		¿Qué es lo que hemos hecho? Hemos pasado a escribir la ecuación en términos de una variable que es la ``velocidad''. Esta velocidad es la velocidad de $u$, la velocidad a la que se propaga la densidad. No tenemos que entenderla como la velocidad de los coches.

		Por ejemplo, imaginemos un mapa de tráfico en el que se ve la densidad del tráfico con colores. La $W$ es la velocidad a las que esas manchas de color se desplazan, no los coches en sí. Están relacionadas pero no son lo mismo.

		Para hallar el sistema que cumple $W$, lo que hacemos es multiplicar \eqref{eq:burgers1} por $V'(u)$ a ambos lados y ver qué nos sale:
		\begin{align}
		u_t + V(u)u_x &= 0 \nonumber \\
		u_t · V'(u) + V(u) u_x · V'(u) &= 0 \nonumber \\
		W_t + W · W_x &= 0 \label{eq:Burgers}
		\end{align}

		En este caso, la ecuación es una \concept{Ecuación\IS homogénea} ya que el término independiente es $0$. Si no fuese nulo, estaríamos ante una \concept{Ecuación\IS no homogénea}, y si el término independiente fuese $\epsilon W_{xx}$ su nombre sería una \concept{Ecuación\IS viscosa}.

		\begin{example}
			\[u_t + (1-2u) u_x = 0
			\rightarrow u(x,0) =
			\begin{cases}
				1 & x < 0 \\
				0 & x > 0
			\end{cases}
			\]

			\[W_t + W W_x = 0 \text{ con } W = 1 - 2u
			\rightarrow W(x,0) =
			\begin{cases}
				-1 & x < 0 \\
				1 & x > 0
			\end{cases}
			\]

			Obviamente, los coches no se mueven hacia atrás, lo que está ocurriendo es que la señal de arranque se está propagando hacia atrás.

		\end{example}


		\subsubsection{Flujo asociado a la ecuación de Burgers}

			Expresemos ahora $q$ en función de la ecuación de Burgers.
			\begin{gather}
			W_t + WW_x = W_t + [q_B(w)]_x = 0 \nonumber \\
			q_{B} = \frac{W^2}{2} \label{eq:FlujoBurgers}
			\end{gather}

		\subsubsection{Características en la ecuación de Burgers}

			\begin{figure}[hbtp]
				\centering
				\inputtikz{CaracteristicasBurger}
				\caption{Características en la ecuación de Burguers}
				\label{fig:CaracteristicasBurger}
			\end{figure}

			Como siempre, tomamos la solución constante \[ W(x(t),t) = k \] y, derivando,
			\[ W_t + W_x x' = 0 \Rightarrow x' = W = k \]

			Por lo que las características serán: \[ x-kt = x_0, \quad \text{ con }F(x_0) = k \] y con $W(x,0) = F(x)$ nuestro dato inicial.



		\newpage % FIXME

		\begin{example}[Zona central de atasco] $ $ % hack

			Vamos a estudiar un flujo de tráfico algo más complicado: coches que vienen rápido, se paran y luego siguen un poco más lento. Nuestro sistema es el siguiente:

			\begin{minipage}{\textwidth}
				\begin{wrapfigure}[8]{R}{0.4\textwidth}
					\centering
					\inputtikz{FTresVelocidades}
					\caption{Dato inicial de velocidad para este atasco.}
					\label{fig:FTresVelocidades}
				\end{wrapfigure}

				\[
					\begin{cases}
						u_t + u u_x = 0 \\
						u(x,0) = F(x)
					\end{cases}
				\]

				Nuestro dato inicial $F(x)$ estará definido como en la \fref{fig:FTresVelocidades}: \[
					F(x) = \begin{cases}
						2 & x ∈ [-2, 0) \\
						0 & x ∈ [0, 1) \\
						1 & x ∈ [1, 2)
					\end{cases}
				\]
			\end{minipage}

			$ $ % hack: fuerza linea en blanco para evitar que el párrafo se empotre contra la imagen

			Vemos que los que vengan de la izquierda con velocidad 2 se encontrarán con la zona de velocidad 0 en la que tendrán que parar, con lo que surgirá una onda de choque entre esas dos secciones. A su vez, entre la parte de velocidad 0 y la parte de velocidad 1 surgirá una onda de rarefacción ya que la parte con más velocidad ``huirá'' de la anterior. Esto ocurre para tiempos pequeños pero habrá que ver como evoluciona el sistema.

			Las características son:
			\[ x - kt = x_0 \text{ con } F(x_0) = k \]

			Va a ocurrir en algún punto que la onda de choque llegue hasta la zona delimitada por la onda de rarefacción. La onda de choque depende de los valores a ambos lados de ésta. Cuando ocurra este suceso la onda de choque estará definida por otra ecuación. Y si en algún momento llega a confluir con la primera característica de la sección de velocidad 1 obtendrá otra forma.


			Empecemos por \textbf{tiempos pequeños}. La onda de choque que parte de $x=0$ estará definida por la ecuación de Rankine-Hugoniot \eqref{eq:DerivadaOndaChoque}:
			\[ s_1' = \frac{\text{salto q}}{\text{salto u}} = \frac{\text{salto }\frac{u^2}{2}}{\text{salto }u} = \frac{\frac{4}{2} - 0}{2-0} = 1 \]

			Esa será la derivada de nuestra onda de choque, que al ser constante es la pendiente. La onda será:
			\[ s_1(t) = t \]

			Por otra parte, necesitamos calcular la onda de rarefacción del salto entre la zona de coches parados y la de coches que se mueven con velocidad $1$. Para eso  usamos el truco de tomar las aproximaciones por $F_ε$ continuas, y nos sale lo siguiente
			\begin{gather*}
			x-k_\epsilon t = x_\epsilon \\
			u_\epsilon (x,t) = k_\epsilon = F_\epsilon (x_\epsilon) \\
			x_\epsilon \convs[][ε][0] 1 \Rightarrow x - kt = 1
			\end{gather*}

			Con lo que obtenemos la onda de rarefación desde $x=1$, que será la zona gris en la \fref{fig:ModeloTresVelocidades}.
			\[ \frac{x-1}{t} = k = u(x,t) \]

			Con esto hemos obtenido qué pasa hasta $t = 1$. La onda de choque la hemos calculado con el salto, pero teníamos datos del lado derecho. Ahora la condición de salto cambia dependiendo del punto ya que dependerá del valor de la onda de rarefacción en ese punto.

			Fijémonos en el instante \textbf{t = 1}. Tenemos una nueva onda de choque $s_2$ que parte del punto ${(x=1, t=1)}$. El flujo, que podemos calcular según la ecuación \eqref{eq:FlujoBurgers}, será $q(u) = \frac{u^2}{2}$.

			Con esto ya podemos calcular los saltos:
			\begin{align*}
			u^{-} &= 2			 & u^{+} &= \frac{s_2-1}{t} \\
			q^{-} &= \frac{4}{2} & q^{+} &= \frac{(\frac{s_2-1}{t})^2}{2}
			\end{align*}

			Ahora usando de nuevo la ecuación de Rankine-Hugoniot, sacamos la ecuación de la segunda onda de choque:
			\[  s'_2 = \frac{\frac{1}{2}[4 - (\frac{s_2 - 1}{t})^2 ]}{2-(\frac{s_2 - 1}{t})} = \frac{\frac{1}{2}(2 + (\frac{s_2 - 1}{t}))}{1} = 1 + \frac{s_2}{2t} - \frac{1}{2t} \] con un dato inicial (el punto de partida de la onda) dado por \[
			s_2(1) = 1 \]

			Esto nos lleva a una ecuación diferencial de primer orden lineal \[ s_2' - \frac{s_2}{2t} = 1 - \frac{1}{2t} \] que podemos resolver multiplicando por un factor integrante: \[ \rho s_2' - \frac{\rho}{2t} s_2 = \rho · \left(1 - \frac{1}{2t}\right) \] esperando que el lado izquierdo sea $(ρs_2)'$ para poder resolverlo fácilmente.

			Continuamos resolviendo para comprobarlo. Primero sacamos el factor integrante: \[ \rho' = \frac{-\rho}{2t} \implies … \implies \rho(t) = \frac{1}{\sqrt{t}} \]

			Ahora calculamos la derivada de $(ρs_2)'$: \
			\[ \left(\frac{s_2}{\sqrt{t}}\right)' = \frac{1}{\sqrt{t}} \left( 1 - \frac{1}{2t}\right) = t^{-1/2} - \frac{1}{2} t^{-3/2} \]
			Integramos y operamos
			$$\frac{s_2}{\sqrt{t}} = \frac{t^{1/2}}{1/2} - \frac{1}{2} \frac{t^{-1/2}}{-1/2} = 2t^{1/2} + t^{-1/2} + C$$
			$$ s_2 = 2t + 1 + C \sqrt{t} $$
			$$ s(1) = 1 \implies C = -2 $$
			$$ x = s_2 (t) = 1 + 2t - 2 \sqrt{t} $$

			Y ahora pasamos a calcular $s_3$, la curva de choque al pasar la curva a tocar la zona roja. La curva de choque habrá atravesado completamente la zona de rarefacción y podremos calcularla por el salto otra vez al tener información de las características de la zonas iniciales $x>1$.


			Comenzamos igualando $s_2$ y aplicando el punto de corte con las características $x-1=t$:
			$$ 1+2t - 2 \sqrt{t} = 1 + t \implies t- 2 \sqrt{t} = 0$$
			Como $t\neq 0$, la solución debe ser $t=4$, luego tenemos que $x=5$ (sustituyendo en la característica).

			Para calcular el salto tenemos
			$$u^{-} = 2 ; u^{+} = 1; q = \frac{u^2}{2} \Rightarrow q^{-} = 2, q^{+} = \frac{1}{2}$$
			Luego Rankine-Hugoniot nos dice que
			$$s'_3 = \frac{2 - \frac{1}{2}}{2 - 1} = \frac{3}{2} \implies s_3 = \frac{3}{2} t + C$$

			Sustituyendo el dato en la ecuación
			$$s_3 (t = 4) = 5$$

			Obtenemos $C = -1$, y tenemos la curva $s_3$:

			$$s_3(t) = \frac{3}{2} t - 1$$


			\begin{figure}[hbtp]
				\centering
				\inputtikz{ModeloTresVelocidades}
				\caption{Resultado final del modelo de tres velocidades}
				\label{fig:ModeloTresVelocidades}
			\end{figure}


			% 2016/02/02

			\textbf{Ejercicio para el lector:} calcular los valores de
			$$ u(\frac{1}{2},t) $$
			$$ u(2,t)$$
			$$ u(6,t)$$


		\end{example}

		\newpage % FIXME
		\obs Con los siguientes datos

		$$u^{-} = 0, u^+ = 1$$
		$$q^{-} = 0, q^+ = \frac{1}{2}$$

		% http://tex.stackexchange.com/a/5770
		% http://tex.stackexchange.com/a/37288
		\begin{figure}[hbtp]
			\centering
			\begin{minipage}[t]{0.45\textwidth}
				\centering
				\inputtikz{ObsAccInf}
				\caption{Aceleración infinita: se salta de $v=0$ a $v=1$}
				\label{fig:obs-aceleracion-infinita}
			\end{minipage}\quad
			\begin{minipage}[t]{0.45\textwidth}
				\centering
				\inputtikz{ObsMuro}
				\caption{``Si no recuerdo mal, el sentido físico era coches estrellándose contra un muro. No es un caso que estemos especialmente interesados en estudiar, creo.''}
				\label{fig:obs-muro}
			\end{minipage}
		\end{figure}

		Tenemos
		\[
		\begin{rcases*}
			s'(t) = \frac{1/2}{1} = \frac{1}{2}\\
			s(0) = 1
		\end{rcases*} \rightarrow s(t) = \frac{1}{2}t + 1
		\]

		Fijándonos en entornos cercanos a $x=1$ tenemos una solución generalizada:

		$$u(x,t) =
		\begin{cases}
			0 & x < \frac{1}{2} t + 1 \\
			1 & x > \frac{1}{2} t + 1
		\end{cases}$$


		Hemos obtenido esta solución ya que se puede aplicar Rankine - Hugoniot. Pero esto ha ocurrido porque hemos rellenado la zona vacía entre las características con características paralelas a las existentes en la zona de más velocidad. Ver figura \ref{fig:obs-aceleracion-infinita}.

		Pero a nivel físico esto no tiene sentido ya que significaría coches acelerando instantáneamente. {\bf La condición de entropía} dice que eso no puede pasar. Por eso ese tipo de áreas se rellenen con una onda de rarefacción.

		El contrario (figura \ref{fig:obs-muro}), por supuesto, si que tiene más sentido físico.

		Para solucionar esta dualidad de soluciones se podría añadir un término de viscosidad, pero eso convertiría nuestro sistema en uno de segundo orden que todavía no podemos resolver.

		\newpage % FIXME

		\textbf{Ejercicio interesante}
			$$u_t + uu_x = 0$$
			$$u(x,0) = F(x) $$

			con $F(x)$ como se presenta en a figura \ref{fig:ejer-feb-2-F}, es decir
			\begin{figure}[hbtp]
			\begin{minipage}{0.45\textwidth}
				\centering
				\inputtikz{Ejer-feb-2-F}
				\caption{ }
				\label{fig:ejer-feb-2-F}
			\end{minipage}
			\begin{minipage}{0.45\textwidth}
				$ F(x) = \begin{cases}
					1   & x < 0\\
					1-x & 0 \leq x \leq 1\\
					0   & x < 1
				\end{cases}
				$
			\end{minipage}
			\end{figure}

			\begin{wrapfigure}{l}{0.45\textwidth}
				\vspace{-15pt}
				\inputtikz{Ejer-feb-2-XT}
				\caption{hay un choque en $(1,1)$}
				\label{fig:ejer-feb-2-XT} % p20
				\vspace{15pt}
			\end{wrapfigure}

			Sea un punto $(x,t)$ tal que $x<t<1$ tenemos que
			$$x - kt = x_0 \text{ con } F(x_0) = k $$
			$$u(x,t) = F(x_0) = k$$
			$$x_0 \in (0,1) \Rightarrow F(x_0) = 1 - x_0$$

			$ $ % hack para que la frase no se empotre contra la figura

			Tenemos que todas estas características pasan por $x = 1$, $t = 1$:
			\begin{gather*}
				x - (1-x_0) t = x_0 \\
				x-t = x_0 (1-t) \\
				x_0 = \frac{x-t}{1-t} \\
				K = F(x_0) = 1 - \frac{x-t}{1-t}
			\end{gather*}

			Todas estas características $s$ terminan colapsando en el punto $(x,t)$ y a partir de ese punto aparecerá una onda de choque. La onda de choque no tiene por qué aparecer en el instante inicial.

			Lo que vamos a ver según avanza el tiempo es que el escalón en $F(x)$ va a ir reduciéndose hasta ser completamente vertical y empezar la onda de choque. Lo que hay que hacer es pintar distintos escalones con $t$ fijos hasta encontrarlo. En la figura \ref{fig:ejer-feb-2-choque} se ve que aparece en $t=1$.

			\begin{figure}[hbtp]
				\centering
				\inputtikz{Ejer-feb-2-choque}
				\caption{La solución parecía buena al principio, pero degenera en un choque en $t=1$}
				\label{fig:ejer-feb-2-choque}
			\end{figure}

\clearpage % FIXME: puede crear una página en blanco en medio del documento
\section{Caso general: Problema de Cauchy}
	\label{sec:ProblemaCauchy}

	De momento hemos estudiado casos bastante específicos. De hecho, ya el planteamiento inicial (\fref{sec:PlanteamientoPrimerOrden}) era muy concreto, estudiando un problema que se podía entender como un flujo a lo largo del tiempo, con ese flujo dependiendo de la densidad. Ahora bien, como buenos matemáticos, nos interesaría encontrar un caso general para ecuaciones de primer orden.

	Lo primero es olvidarnos del ``tiempo'', y simplemente tomar dos variables genéricas $x,y$ para evitar confusiones. Así, nuestras ecuaciones serán de la forma \(
		a(x,y,u)u_x + b(x,y,u)u_y = c(x,y,u) \label{eq:PrimerOrdenGenerica}
	\)

	¿Y qué hacemos con el dato? Hasta ahora siempre lo habíamos dado a lo largo de la recta $t = 0$ (salvo en el ejemplo de la \fref{sec:CurvaDatoRara}). Sin embargo, es una condición demasiado restrictiva, así que en general tomaremos la solución del dato a lo largo de una curva genérica $(\alpha(s),\beta(s))$. Así, nuestro dato será
	\[ u(\alpha(s),\beta(s)) = \gamma(s) \]

	¿Cómo resolvemos este sistema? La idea será ``propagar'' la solución dada en la curva dato siguiendo la ecuación diferencial. En cierto modo es parecido a lo que hacíamos con las características, aunque esta vez necesitaremos más artillería matemática.

	Partiendo de que $u(x,y)$ es una solución a nuestro sistema, consideraremos su gráfica $S = \set{(x,y,z) ∈ ℝ^3 \tq z = u(x,y)}$. Esta gráfica no es más que el conjunto de nivel 0 de la aplicación \[ Φ(x,y,z) = u(x,y) - z\]

	Ahora vamos a por un truco/idea feliz: sabemos que el gradiente es perpendicular a los conjuntos de nivel, por lo que en particular $\grad Φ \perp S$. Ese gradiente es fácil de calcular: \[ \grad Φ = \left(u_x, u_y, -1\right)\]

	Y aquí es donde viene la segunda ocurrencia: darnos cuenta de que $\grad Φ$ y $(a, b, c)$ son dos vectores perpendiculares, donde $a$, $b$ y $c$ eran los coeficientes de nuestra ecuación \eqref{eq:PrimerOrdenGenerica}. La comprobación se hace con una simple cuenta, ya que simplemente volvemos a la ecuación de nuevo: \[ \pesc{(u_x, u_y, -1), (a, b, c)} = au_x + bu_y - c = 0 \]

	La conclusión es que el vector $(a(p),b(p),c(p))$ es tangente a nuestra superficie solución, con $p ∈ S$. En otras palabras, que podemos reconstruir una curva solución $\Gamma ⊂ S$ con $\Gamma' = (a,b,c)$ y
	$$\Gamma(s) = (x(0),y(0),z(0)) = (\alpha(s),\beta(s),\gamma(s))$$

	Así, podremos resolver la ecuación ordinaria obteniendo una curva de cada una de los datos iniciales. Una vez obtenidas estas curvas, si todo va bien, podremos ensamblarlas en una superficie parametrizable. Deberemos probar que esta superficie se puede escribir como una gráfica de una función, pero no nos valdrá con eso, tendremos que probar que se puede escribir como función de $x$ e $y$.

	De lo que aprendimos en EDO\footnote{jajajajaja} sabemos encontrar curvas con una tangente dada. ¿Pero podemos aplicarlo para obtener superficies?

	Antes de seguir con la formalización, lo que haremos será ver unas cuantas cuentas para entrar en situación.

	\subsection{Ejemplos y cuentas previos}

	Empecemos con un ejemplo:

	\begin{example}[Problema de Cauchy]
		Partimos del siguiente sistema:
		\begin{equation*}
			\left\{
			\begin{array}{l}
				xu_x - yu_y = u - y \\
				u(s^2,s) = s \quad (s > 0)
			\end{array}
			\right.
		\end{equation*}

		Interpretación geométrica. En cada $(x,y,z)$ el vector tangente debe ser $(\underbrace{x}_{a},\underbrace{-y}_{b},\underbrace{z-y}_{c})$

		Hallamos las \textbf{curvas solución:}
		\begin{equation*}
			\left.
			\begin{array}{rl}
				 x'(t) = x \\
				 y' = -y \\
				 z' = z-y
			\end{array}
			\right|
			\begin{array}{l}
				x = x_0 e^t \\
				y = y_0 e^{-t} \\
				z' -z = -y_0 e^{-t}
			\end{array}
		\end{equation*}

		Resolvemos usando el método del factor integrante y llegamos a \[ z = \frac{y_0}{2} e^{-t} + C e^t , \quad\text{ con } C = z_0 - \frac{y_0}{2} \] de tal forma que nuestra curva viene dada por la ecuación
		\begin{align*}
			x(t) &= x_0 e^t \\
			y(t) &= y_0 e^{-t} \\
			z(t) &= \frac{y_0}{2} e^{-t} + (z_0 - \frac{y_0}{2})e^t
		\end{align*} donde los valores iniciales vienen dados por la curva dato: \[ (x_0,y_0,z_0) = (s^2,s,s) \]

		Poniendo todo en función de $s$ y $t$, llegamos a
		\begin{align*}
		x(s,t) &= s^2 e^t \\
		y(s,t) &= se^{-t} \\
		z(s,t) &= \frac{s}{2}e^{-t} + \frac{s}{2}e^{t}
		\end{align*}
		que nos permite dar una definición para la superficie:
		\[
			\Phi(s,t) = \left(s^2e^t, se^{-t}, \frac{s}{2}(e^t + e^{-t}) \right )
		\]

		\newpage % FIXME
		Debemos comprobar si este objeto es realmente una superficie y si nos sirve, es decir, responder a las siguientes preguntas:

		\begin{itemize}[itemsep = 1pt]
			\item ¿$\Phi$ describe una superficie parametrizada?
			\item ¿Podemos despejar $z=u(x,y)$?
		\end{itemize}

		\textbf{Comenzamos a despejar}, porque tenemos suerte y se puede hacer explícitamente:

		$$xy = s^3; s=(xy)^{1/3}$$
		$$e^t = \frac{x}{s^2} = \frac{x}{(xy)^{2/3}} = \frac{x^{1/3}}{y^{2/3}}$$
		$$e^{-t} = \frac{y^{2/3}}{x^{1/3}}$$

		$$z = \frac{s}{2}(e^{t}+e^{-t}) = \frac{(xy)^{1/3}}{2} (\frac{x^{1/3}}{y^{2/3}}+\frac{y^{2/3}}{x^{1/3}}) = … = u(x,y) $$


		Puede haber problemas si $x=0$ o $y=0$, pero como nuestra curva dato es
		$$(s^2,s,s), \quad s > 0$$
		en un entorno de la misma tenemos perfectamente definida la solución.

		Como veremos más adelante, esta condición de compatibilidad nos la da el rango de la matriz Jacobiana de $\Phi$.

		\begin{figure}[hbtp]
			\centering
			\inputtikz{CurvaDatoProblemaCauchy}
			\caption{En un entorno del 0 tenemos problemas.}
			\label{fig:curva-dato-ProblemaCauchy}
		\end{figure}

	\end{example}

	\begin{example}[2]

		\begin{equation*}
			\begin{cases}
				yu_x - x u_y = 0 \\
				u(\alpha(s),\beta(s)) = \gamma(s) \\
				\text{Curva dato: } \Gamma \equiv (\alpha, \beta, \gamma)
			\end{cases}
		\end{equation*}

		De este sistema obtenemos un \concept{Sistema\IS característico}:
		\begin{equation*}
			\begin{cases}
				\frac{\dif x}{\dif t} = y \\
				\frac{\dif y}{\dif t} = -x \\
				\frac{\dif z}{\dif t} = 0
			\end{cases}
		\end{equation*}

		cuyas soluciones será las \concept[Curvas\IS características]{curvas características}:
		$(x(t),y(t),z(y))$; pero a veces solo nos interesarán sus {\bf proyecciones sobre el plano XY}: $(x(t),y(t))$.

		Para resolver \underline{este sistema concreto} y llegar a las soluciones explícitas (en función de $\alpha$, $\beta$ y $\gamma$), tenemos dos formas de hacerlo.
		La primera, requiere de bastantes cálculos utilizando el método de la exponencial de la matriz para resolver el sistema de EDOs:
		\[ \begin{pmatrix}
			x \\
			y
			\end{pmatrix} =
		\begin{pmatrix}
			 0 & 1 \\
			-1 & 0 \\
			\end{pmatrix}
			\begin{pmatrix}
			x \\
			y
			\end{pmatrix}\]

		Pero podemos utilizar el siguiente atajo observando que
		\begin{gather*}
			\dpd{^2 x}{t} = \dpd{y}{t} = -x \iff \dpd{^2 x}{t} + x = 0\\
			x = A \cos t + B \sin t\\
			y = \dpd{x}{t} = -A \sin t + B \cos t
		\end{gather*}
		Por tanto, utilizando que
		\begin{gather*}
			x(0) =\alpha(s)	\implies A = \alpha(s)\\
			y(0) = \beta(s) \implies B = \beta(s)
		\end{gather*}
		entonces
		$$x(t,s) = \alpha(s) \cos t + \beta (s) \sin t$$
		$$y(t,s) = -\alpha(s) \sin t + \beta (s) \cos t$$
		$$z(t,s) = \gamma(s)$$

		Combinando todas esas curvas características podremos ensamblar la superficie que buscamos. Ahora observemos los resultados con valores concretos de $\alpha$, $\beta$ y $\gamma$:

		$$(\alpha(s), \beta(s), \gamma(s)) = (s,s,s^2), \quad s>0$$
		$$\Phi(s,t) = (\underbrace{s(\cos t + \sin t)}_{x}, \underbrace{s(\cos t - \sin t)}_{y},s^2)$$

		$$x^2 + y^2 = s^2 (\cos^2 t + 2 \sin t \cos t + \sin^2 t) + s^2 (\cos^2 t - 2 \sin t \cos t + \sin^2 t) = 2s^2$$

		Si calculamos la proyección en el plano $(x,y)$ de las características lo que aparecen son circunferencias centradas en el origen. Pero eso nos plantea un problema. Si tenemos un dato inicial simple no tenemos problema, pero qué pasa si suponemos:
		$$\Gamma = (s, 0, \gamma(s))$$
		A lo largo de la característica la solución debería permanecer constante\footnote{recordemos que era un problema homogéneo
		} pero en este caso no pasaría si $\gamma(-s) \neq \gamma(s)$. Luego no podemos obtener soluciones globales y debemos conformarnos con soluciones locales.


	\end{example}

	\begin{example}{\bf 3}

		Resolvamos una \concept{Ecuación\IS de Burgers}
		\[
			\left\{
			\begin{array}{l}
				uu_x + u_y = 0 \\
				u(x,0) = F(x)
			\end{array}
			\right.
		\]

		\textbf{Sistema característico:}
		\[
			\left\{
			\begin{array}{l}
				\frac{\dif x}{\dif t} = z \\
				\frac{\dif y}{\dif t} = 1 \rightarrow y = t \\
				\frac{\dif z}{\dif t} = 0 \rightarrow z = F(s) \\
			\end{array}
			\right.
		\]

		$$(x,y,z) |_{t=0} = (s,0,F(s))$$
		Siendo un poco parcos en palabras, presentamos la solución:
		\[
			\left.
			\dpd{x}{t} = F(s) \rightarrow \\
			\begin{array}{l}
				x = F(s)t + C \\
				x |_{t=0} = s
			\end{array}
			\right\} \rightarrow C = s
		\]

		\textbf{Superficie solución:}
		$$\Phi(s,t) = (F(s) t + s, t , F(s))$$
		Si intentamos despejar
		$$x - F(s)t = s \implies x-F(s)y = s$$
		¿Cómo despejamos $s = s(x,y)$?

		\noindent¿Seremos capaces de encontrar $z = F(s(x,y)) = u(x,y)$?

		\noindent {\bf ¿Dónde se esconden los choques y las ondas de rarefacción?}

		\noindent¿Es posible que la solución dependa del método? Recordemos que aunque el dato sea $C^\infty$ pueden aparecer ondas de choque. La respuesta a todas estas preguntas es que estamos utilizando un teorema de existencia y unicidad de \underline{soluciones locales}.

		\noindent Por tanto, tenemos por delante un viaje en busca de los monstruos: soluciones no regulares... y los tremendos: ¿qué hacemos cuando la curva característica coincide con el dato?

	\end{example}


	Esto que hemos estado viendo se puede encontrar en \cite{salsaPDE} en el Capítulo 4.
% Lo que estamos viendo se puede encontrar en S. Salsa Partial Differential Equations in Action (Cap. 4)

% Clase 8/2/16.

\clearpage % FIXME: puede crear una página en blanco en medio del documento
\subsection{Formalización}

Vamos a tratar de formalizar lo que hemos visto hasta ahora. Tenemos una ecuación a resolver \[ a(x,y,u) u_x + b(x,y,u) u_y = c(x,y,u)\] con un dato \[ u(α(s), β(s)) = γ(s)\quad s∈[a,b]\] con $a,b,c,α,β,γ ∈ C^1$. El dato se puede tomar como dado a lo largo de una curva $Γ(s) = (α(s), β(s), γ(s))$ que ha de tener unas ciertas restricciones, principalmente que para cada valor de $x,y$ tenemos que tener un único valor de $z$ (no nos valen rectas verticales ni espirales, por ejemplo).

El método de resolución es el \concept{Método\IS de las características}. El primer paso es resolver el sistema característico en $t$ dado por
\begin{align*}
	\dpd{x}{t} &= a(x,y,z) \\
	\dpd{y}{t} &= b(x,y,z) \\
	\dpd{z}{t} &= c(x,y,z) \\
	\left. (x,y,z) \right|_{t=0} &= (α(s), β(s), γ(s))
\end{align*}

Este es un sistema de Ecuaciones diferenciales ordinarias, que sólo necesita que las funciones sean Lipschitz localmente. Esta condición viene gratis por ser $a,b,c ∈ C^1$, así que tenemos existencia y unicidad locales del sistema característico.

El siguiente paso es considerar la aplicación \[ Φ(s,t) = (x(s,t), y(s,t), z(s,t))\] con las funciones que hemos obtenido previamente resolviendo el sistema característico. La pregunta es si Φ es una \concept[Parametrización]{parametrización} y por lo tanto describe una superficie. Para ello necesitamos que la diferencial $\Dif Φ$ tenga rango 2 y sea un homeomorfismo sobre su imagen.

Dado que cuando nos alejamos del punto inicial tenemos monstruos, lo que buscaremos es que las dos condiciones se cumplan sólo para $t$ pequeño. Esto nos permite no tener que verificar la condición de homeomorfismo sobre su imagen, ya que se cumple directamente para $t$ pequeño por la estructura de la curva dato Γ, que no tiene autointersecciones.

Estudiamos ahora la diferencial de Φ para ver qué ocurre con la condición del rango: \[ \Dif Φ = \begin{pmatrix} \dpd{x}{s} & \dpd{y}{s} & \dpd{z}{s} \\ & & \\ \dpd{x}{t} & \dpd{y}{t} & \dpd{z}{t} \end{pmatrix} \]

Por continuidad, nos bastará ver que tiene rango máximo para $t = 0$, ya que nos dará directamente rango máximo en un entorno pequeño de $t$. Pero teniendo en cuenta el sistema característico, podemos simplificar y tenemos que \[ \eval[2]{\Dif Φ}_{t=0} = \begin{pmatrix}
α' & β' & γ'  \\
a(α, β, γ) & b(α,β,γ) & c(α,β,γ) \end{pmatrix}
 \] por lo que podemos evaluar el rango sin tener que resolver el sistema, ya que la diferencial sólo depende de los coeficientes y de la curva dato.

Una vez que sabemos que Φ define una superficie, necesitaremos despejar ${z = u(x,y)}$. Para eso necesitaremos el teorema de la función implícita \citep[Teorema II.5]{ApuntesAnalisisMat}, aunque en realidad sólo nos hace falta el teorema de la función inversa. Entonces, sólo tenemos que pedir que el determinante \( \left|\begin{matrix} a & b \\ α' & β' \end{matrix}\right| \label{eq:CondTransversalidad} \) sea distinto de $0$. Es decir, que sólo comprobando esta \concept{Condición\IS de transversalidad} podremos resolver el paso 2 (garantiza que Φ define una superficie) y el paso 3 (podemos despejar $z = u(x,y)$).

\begin{theorem} \label{thm:Transversalidad} Supongamos que $a,b,c,α,β,γ ∈ C^1$.

\noindent Si se verifica la condición de transversalidad \eqref{eq:CondTransversalidad} en todos los puntos de la curva dato, el problema tiene una solución $z = u(x,y)$ definida en \textbf{un entorno local} de cada punto, que se puede construir con las curvas características.
\end{theorem}

Sin embargo, esto no nos resuelve todos los problemas. Tenemos unicidad a través de este método, pero no sabemos si otro método nos puede dar otra solución distinta. Además, no sabemos qué pasa cuando la condición de transversalidad se estropea: ¿qué tipo de desastres aparecen?

\subsubsection{Cuando la condición de transversalidad no se cumple}
\label{sec:CondTransversalidadInvalida}

Suponemos que $s_0$ es un punto característico o singular, donde el determinante de \eqref{eq:CondTransversalidad} es cero. Aun así, podemos suponer que existe una solución $u ∈ C^1$ en un entorno de $Γ(s_0)$.

Si el determinante es $0$, entonces $aβ' - α'b = 0$, así que tenemos que \begin{align*}
aα'u_x + bα'u_y &= cα' \\
aα'u_x + bα'u_y &= aγ'
\end{align*} y por lo tanto $cα' = aγ'$, lo que nos lleva a que \[ \frac{a}{α'} = \frac{b}{β'} = \frac{c}{γ'} \]

Si aun así tenemos solución, la única posibilidad que nos queda para tener solución $C^1$ es que el rango de la matriz $\Dif Φ$ sea $1$. Alternativamente, si el rango de $\Dif Φ$ es 2, no existe ninguna solución $u ∈ C^1$ en un entorno del punto característico.

\subsubsection{¿Existen otras soluciones?}

\begin{figure}[hbtp]
\centering
\inputtikz{SolucionAlternativaCaracteristica}
\caption{Esquema de cómo resolver la situación cuando hay una solución alternativa (verde) a la definida por las curvas características (naranja).}
\label{fig:SolucionesAlternativasCaracteristica}
\end{figure}

Vamos a demostrar que cualquier posible solución que coincida con un punto en la característica, entonces coincide completamente con la característica. Definiremos \[ D(t) = z(t) - u(x(t), y(t))\] donde $u$ es nuestra solución alternativa y $z$ la curva característica. Su derivada será \[ D'(t) = z'(t) - u_x(x(t), y(t)) x'(t) - u_y(x(t), y(t)) y'(t)\] que haciendo cuentas tendremos que \begin{align*} D'(t) &= c(x, y, D(t) + u(x,y)) - u_x(x,y) \cdot a(x,y, D+u(x,y)) - u_y \cdot b(x,y, D + u(x,y)) \\
&= F(t, D(t))
\end{align*} simplificando un poco en el último paso, y sabiendo que $D(0) = 0$. En ese caso, volvemos a tener la expresión de la solución, así que el sistema \[ \begin{cases} D'= F(t,D) & \\ D(0) = 0 & \end{cases} \] tiene una solución única $D \equiv 0$ por el teorema de existencia y unicidad, luego $u$ contiene a la característica.


\subsubsection{Toda la curva dato es característica}

Antes (\fref{sec:CondTransversalidadInvalida}) hemos visto qué ocurre cuando la condición de transversalidad no se cumplía, pero sólo hemos tenido en cuenta el caso de un punto característico aislado. ¿Qué pasa si, por ejemplo, tenemos toda la curva dato con puntos característicos?

Podemos hacer un experimento mental: cogemos una curva transversal a la curva anterior (en la proyección) y que interseque con ella; y planteamos el mismo problema que antes pero con esta nueva curva dato. Ya que lo elegimos, nuestra curva hará que el determinante siempre sea distinto de 0, y tendremos una solución al nuevo sistema que también será válida para el sistema anterior, ya que la curva dato anterior será una característica de este nuevo sistema\footnote{\noteby{Guille}{No estoy 100\% seguro de esto.}}. La cuestión es que hay infinitas formas de coger esa nueva curva dato, así que tenemos infinitas posibles soluciones.

La conclusión de todo esto es que el problema no estaría bien propuesto porque sus soluciones no tendrían mucho sentido.

\subsection{Más ejemplos}

	\begin{example}[1]
		\[\left\{ \begin{array}{l} uu_x + u_y = 0 \\ u(x,0) = F(x) \end{array}\right.\]

		\[ \left.\begin{array}{r}
		a(x,y,z) = z \\
		b(x,y,z) = 1 \\
		c(x,y,z) = 0 \\
		\end{array} \right| \begin{array}{l}
		\alpha(s) = s \\
		\beta(s) = 0 \\
		\gamma(s) = F(s) \end{array}
		(x,y,z)|_{t=0} = (s,0,F(s))
		\]

		\textbf{Condición de transversalidad}

		\[\det \left.\begin{pmatrix}
		a & b \\
		\alpha' & \beta'
		\end{pmatrix} \right|_{t=0}  = \det \left. \begin{pmatrix}
		F(s) & 1 \\
		1 & 0
		\end{pmatrix} \right|_{t=0} = -1 \neq 0\]

		Por lo tanto tenemos existencia y unicidad de la solución. Vamos a calcularla:

		\textbf{Resolución del sistema característico}

		\[ (x,y,z) |_{t=0} = (s,0,F(s)) \]
		\begin{align*}
		\dpd{x}{t} &= z \\
		\dpd{y}{t} &= 1 \rightarrow y = t \\
		\dpd{z}{t} &= 0 \rightarrow z = F(s) \rightarrow x = F(s)\cdot t + s
		\end{align*}


		\textbf{Solución}
		\[\Phi(t,s) = (F(s)t+s, t, F(s))\]

		{\bf Proyección del dato y las características en el plano XY:}

		\begin{figure}[hbtp]
			\inputtikz{Ejemplo-02-09-proyXY}
			\caption{La pendiente de la característica azul dependerá de la $F$}
			\label{fig:Ejemplo-02-09-proyXY}
		\end{figure}

		\[
		\left.
		\begin{array}{r}
		x = F(s)t + s \\
		y = t
		\end{array}
		\right\} \quad x - F(s)\cdot y = s \iff x - ky = s \ \text{ con } \ k = F(s)
		\]
		Con esto vemos que la pendiente de las curvas características depende de la $F$. Esto hace que tengamos 2 casos
		\begin{itemize}
			\item Si la F es contínua, los choques se postergan.
			\item Si la F es discontínua, aparecen directamente.
		\end{itemize}


	\end{example}

	\textbf{Problema propuesto a continuación de lo anterior}

	Consideramos esta $F(s)$ que se muestra en la figura \ref{fig:Ejemplo-02-09-suave}:

	\[
	F(s) =
	\begin{cases}
	1 & s < 0 \\
	\cos^2 s & 0 \leq s \leq \frac{\pi}{2} \\
	0 & s > \frac{\pi}{2}
	\end{cases}
	\]

	aquí vamos a ver cómo los problemas empiezan cuando se forme una discontinuidad por culpa del avance de las características a diferencia de los ejemplos anteriores en los que ya había una discontinuidad en el dato inicial.

	\begin{figure}[hbtp]
		\centering
		\inputtikz{Ejemplo-02-09-suave}
		\caption{La $F$ propuesta. Nos valdría cualquier $F$ que se pudiese escribir como función de $x$ y que no se ponga vertical.}
		\label{fig:Ejemplo-02-09-suave}
	\end{figure}

	\begin{example}{\bf 2}

		\noindent Sea el problema:
		\[
		\begin{array}{l}
			u u_x + u_y = 1 \\
			u(x,0) = F(x)
		\end{array}
		\]

		Como es fácil comprobar que se cumple la condición de transversalidad, vamos a dar la superficie solución directamente:
		\[ \Phi(s,t) = (\frac{t^2}{2} + F(s) t + s, t, F(s) + t) \] proyectándolas después en XY: \[ x - F(s)y - \frac{y^2}{2} = s \] observamos que tenemos unas parábolas en y.

		Se deja como ejercicio para el diligente lector dibujar las características y compararlas con el ejemplo anterior.

	\end{example}

	\begin{example}{\bf3: ej8 hoja 1}

		Dado el problema
		\[(y+u)u_x + y u_y = x-y \]
		\[u(x,1) = 1 + x \]

		Vamos a darle a la maquinaria:

		\[ \left. \begin{array}{r}
		a(x,y,z) = y+z \\
		b(x,y,z) = y \\
		c(x,y,z) = x-y
		\end{array} \right| \begin{array}{l}
		\alpha(s) = s \\
		\beta(s) = 1 \\
		\gamma(s) = 1+s \end{array}
		(x,y,z)|_{t=0} = (s,1,1+s)
		\]

		Comprobamos la {\bf condición de transversalidad}:

		\[ \det \left.\begin{pmatrix}
		a  & b \\
		\alpha' & \beta' \end{pmatrix}\right|_{t=0} =
		\det \left.\begin{pmatrix}
		y+z  & y \\
		1 & 0 \end{pmatrix} \right|_{t=0} =
		\det \left.\begin{pmatrix}
		2+s  & 1 \\
		1 & 0 \end{pmatrix} \right|_{t=0} = -1 \neq 0 \]

		Por lo tanto se cumple la condición: tenemos existencia y unicidad de una solución $C^1$ local. Pasamos a resolver {\bf el sistema característico}:

		\begin{align*}
			 \dpd{x}{t} &= y+z  \\
			 \dpd{y}{t} &= y \\
			 \dpd{z}{t} &= x-y
		\end{align*}
		$$(x,y,z)|_{t=0} = (s,1,1+s)$$

		Llegados a este punto, podemos ver el sistema como en EDO:
		$$
		\begin{pmatrix}
			x \\
			y \\
			z
		\end{pmatrix}
		=
		\begin{pmatrix}
			0 & 1 & 1 \\
			0 & 1 & 0 \\
			1 & -1 & 0
		\end{pmatrix}
			\cdot
		\begin{pmatrix}
			x \\
			y \\
			z
		\end{pmatrix}
		$$
		Y aplicar el método de exponencial de una matriz. Pero en este caso, podemos resolverlo a mano:

		Si observamos la segunda ecuación y el dato, tenemos que $ y = e^t $.

% en la igualdad de las derivadas parciales, la segunda derivada de x es igual a la derivada de z + e^t y ahora mismo pone que es la segunda derivada de x es igual a la derivada de z

		Sustituyendo en el sistema:
		\[
		\begin{rcases*}
			 \dpd{x}{t} = e^t + z  \\
			 \dpd{z}{t} = x - e^t
		\end{rcases*}
		\rightarrow \dpd{^2 x}{^2 t} = e^t + \dpd{z}{t} = e^t + x - e^t = x
		\]
		Luego
		$$\dpd{^2 x}{^2 t} = x \rightarrow x = A \cdot e^t + B \cdot e^{-t}$$
		Finalmente, despejamos z en la ecuación de $\dpd{x}{t}$ y sustituimos:
		$$z = \dpd{x}{t} - e^t = (A \cdot e^t - B \cdot e^{-t}) - e^t = (A-1)e^t - B e^{-t}$$

		Aplicando el dato, obtenemos que $A=1+s$, $B=-1$; luego la solución queda así:

		\[ \Phi(s,t) = (\underbrace{(1+s)e^t-e^{-t}}_{x},\underbrace{e^t}_{y}, s\cdot e^t + e^{-t}) \]

		En este caso, además se puede despejar:

		\[  \left. \begin{array}{r}
		s = s(x,y) \\
		t = t(x,y)
		\end{array}
		\right\}\rightarrow \dots \rightarrow z = x + \frac{2}{y} - y \equiv u(x,y)
		\]

		Gracias a esto, podemos ver que la solución tiene un problema en un entorno de ${y = 0}$.

		\noindent {\bf ¿Habríamos podido verlo si no hubiésemos sabido despejar?} Probablemente.

		Hagamos la proyección de las características en el plano $XY$:

		\[
		\left.
		\begin{array}{l}
			x = (1+s)e^t - e^{-t} \\
			y = e^{t}
		\end{array}
		\right\}
		\Rightarrow x = (1+s) y - \frac{1}{y}
		\]

		\[ y = 1 \Rightarrow x = s \] Por encima de la recta $y=1$ tenemos curvas que son casi rectas, y por debajo hipérbolas.

		Usando como dato: \[ (x,y) |_{t=0} = (s,1) \]

		\begin{figure}[hbtp]
			\inputtikz{Ejemplo-3-feb-09-proy-XY}
			\caption{No tenemos datos para ver qué pasa en el semiplano inferior.}
			\label{fig:Ejemplo-3-feb-09-proy-XY}
		\end{figure}

		En el dibujo podemos ver cómo, a pesar de nuestras preocupaciones, las características nos indican que las soluciones que pasan por nuestro dato \underline{no pasan} por $y=0$.

	\end{example}


	Planteamos otro problema:
	\begin{example}{\bf 4: ej 10 hoja 1}

		\[
		\begin{rcases*}
			u^2u_x + u_y = 0, & x > 0 \\
			u(x,0) = \sqrt{x}
		\end{rcases*}
		 \]

		A falta de comprobación de las cuentas, la solución será \[u(x,y) = \sqrt{\frac{x}{y+1}} \]

		Hay que ver que la proyección de las características en el plano $XY$ y ver que son rectas que se cortan en el punto $(0,-1)$.

	\end{example}

	\begin{example}{\bf 5: ej 4 hoja 1}
		Vamos a hacer un pequeño comentario sobre este ejercicio:

		\[ V(\rho) =
		\left.
		\begin{array}{l}
			… \\
			… \\
			…
		\end{array}
		\right\} \Rightarrow \text{Flujo } q = v \cdot \rho
		\]
		\obs en este problema la velocidad es de las partículas, {\bf NO} la densidad.

		$$\rho_t + (q)_x = 0 \iff \rho_t + (v(\rho) \cdot \rho)_x = 0$$
		Aquí habría que hacer lo que los ingenieros llaman análisis dimensional de los datos:

		\begin{gather*}
		[y] = \frac{\text{\# coches}}{\text{tiempo}}\\
		[\rho] = \frac{\text{\# coches}}{\text{longitud}}\\
		[v] = \frac{\text{longitud}}{\text{tiempo}}
		\end{gather*}

		% algo más que añadir?


	\end{example}

	\begin{example}{\bf 6}

		Vamos a resolver otro ejemplo:
		\[
		\left\{
		\begin{array}{l}
			xu_x+ yu_y = -u\\
			u(\cos s, \sin s) = 1,\quad 0 \leq s \leq R
		\end{array}
		\right.
		\]

		\[ \left. \begin{array}{r}
		a(x,y,z) = x \\
		b(x,y,z) = y \\
		c(x,y,z) = -z
		\end{array} \right| \begin{array}{l}
		\alpha(s) = \cos s \\
		\beta(s) = \sin s \\
		\gamma(s) = 1 \end{array}
		\quad(x,y,z)|_{t=0} = (\cos s,\sin s,1)
		\]

		{\bf Condición de transversalidad:}

		\[\det \left. \begin{pmatrix}
			a & b \\
			\alpha' & \beta'
		\end{pmatrix} \right|_{t=0} =
		\det \left. \begin{pmatrix}
			a(\alpha(s), \beta(s), \gamma(s)) & b(\alpha(s), \beta(s), \gamma(s)) \\
			\alpha' & \beta'
		\end{pmatrix} \right|_{t=0} = \det \begin{pmatrix}
			\cos s & \sin s \\
			-\sin s & \cos s
		\end{pmatrix} = 1 \neq 0 \quad \forall s \]
		luego tenemos existencia y unicidad de una solución local $C^1$.

		Resolvamos el {\bf sistema característico}, que en este caso es extremadamente sencillo:
		\begin{align*}
			 \dpd{x}{t} &= x \rightarrow x(x,t) = \cos(s) e^t\\
			 \dpd{y}{t} &= y \rightarrow y(s,t) = \sin(s) e^t\\
			 \dpd{z}{t} &= -z \rightarrow z(s,t) = e^{-t}\\
			(x,y,z)|_{t=0} &= (\cos s,\sin s,1)
		\end{align*}
		Luego la {\bf superficie solución} es:
		\[\Phi(s,t) = (\cos(s) e^{t}, \sin(s) e^t, e^{-t})\]

		En este caso se alinean los astros, y podemos despejar explícitamente:

		Como
		\[
		\begin{rcases}
			x = \cos (s) e^t\\
			y = \sin (s) e^t
		\end{rcases}
		 \longrightarrow x^2 + y^2 = e^{2t} \]
		luego
		\[e^t = \sqrt{x^2 + y^2} \implies z = e^{-t} = \frac{1}{\sqrt{x^2 + y^2}} \]

		Por lo que la solución es:
		\[ u(x,y) = \frac{1}{\sqrt{x^2+y^2}}\]

		que tiene problemas en el $(0,0)$. Así que veamos su {\bf proyección en XY de dato y características}:

		\[
		\begin{rcases}
			x = \cos (s) e^t\\
			y = \sin (s) e^t
		\end{rcases}
		\implies y/x = \tan(s) \]

		Las soluciones son ${y = x \tan(s)}$, que son rectas que pasan por el $(0,0)$ con pendiente $\tan(s)$, como se puede apreciar en la figura \ref{fig:Proy-XY-ejemplo-6}.

		\begin{figure}[hbtp]
			\inputtikz{Proy-XY-ejemplo-6}
			\caption{Algunos ejemplos de características (azul) y del dato (colorado).}
			\label{fig:Proy-XY-ejemplo-6}
		\end{figure}

	\end{example}

	\begin{example}{\bf 7}

		Vamos a resolver otro ejemplo:
		\[
		\begin{cases}
			uu_x+ u_y = 1\\
			u(\frac{s^2}{4}, s) = \frac{s}{2}
		\end{cases}
		\]

		\begin{gather*}
			\left. \begin{array}{r}
			a(x,y,z) = z \quad\\
			b(x,y,z) = 1 \quad\\
			c(x,y,z) = 1 \quad
			\end{array} \right| \begin{array}{l}
			\alpha(s) = \frac{s^2}{4} \\
			\beta(s) = s \\
			\gamma(s) = \frac{s}{2} \end{array}\\
			(x,y,z)|_{t=0} = (\frac{s^2}{2}, s, \frac{s}{2})
		\end{gather*}


		Comprobamos la {\bf condición de transversalidad}:
		\[\det \left. \begin{pmatrix}
			a & b \\
			\alpha' & \beta'
		\end{pmatrix}\right|_{t=0} =
		\det \begin{pmatrix}
			\frac{s}{2} & 1 \\
			\frac{s}{2} & 1
		\end{pmatrix} = 0 \quad \forall s \]
		Como no se cumple, vamos a ver si podemos al menos tener solución:
		\[ \text{Rango}
		\left. \begin{pmatrix}
			a & b & c \\
			\alpha' & \beta' & \gamma'
		\end{pmatrix}
		\right|_{t=0} = \text{Rango}
		\left. \begin{pmatrix}
			\frac{s}{2} &1 & 1 \\
			\frac{s}{2} &1 & \frac{1}{2}
		\end{pmatrix} \right|_{t=0} = 2 \neq 1 \implies \not \exists \text{ solución } C^1
		\]

		Aunque sepamos que no existe solución, seguimos adelante y hacemos las cuentas, para ver qué ocurre:
		\[
		\begin{rcases}
			 \dpd{x}{t} = z  \\
			 \dpd{y}{t} = 1 \\
			 \dpd{z}{t} = 1
		\end{rcases}
		\Longrightarrow … \Longrightarrow
		\begin{cases}
			x = \frac{t^2}{2}+\frac{s}{2}t + \frac{s^2}{4} \\
			y = s+t \\
			z = \frac{s}{2}+t
		\end{cases}
		\]

		No podemos despejar la $z$, pero si podemos despejar la $x$. Esto significa que si vamos a tener superficie solución, que en la práctica no nos va a servir porque se dobla de forma que su plano tangente se pone vertical.

		Con un poco de destreza\footnote{$z^2$ es casi lo que buscamos, solo hay que conseguir sumarle $\frac{s^2}{4}$, y por suerte $y-z = \frac{s}{2}$} deberíamos obtener:
		\[
		x = \frac{1}{2} \{z^2 + (y-z)^2\}
		\]
		La superficie solución será el conjunto de nivel 0 de \[F(x,y,z)= x - \frac{1}{2} \{z^2 + (y-z)^2\} = 2x-\{z^2 + (y-z)^2\}\]

		Recordemos que el \concept[Gradiente]{gradiente} debe ser perpendicular al conjunto de nivel:
		\[\vec{n} = \nabla F = (2,-2y+2z, -4z+2y)\]

		Luego la solución de la curva dato es
		\[
		\begin{cases}
			x=\frac{s^2}{4} \\
			y = s \\
			z = \frac{s}{2}
		\end{cases}
		\implies \vec{n} = (2,-s,0)
		\]

		Este último 0 nos indica que que el vector normal es horizontal; luego el plano tangente es vertical, por lo que la derivada se hará infinito en algún punto. Luego $u(x,y) \notin C^1$.

		Además, podemos mirar {\bf las proyecciones en el plano XY}:
		\[x = \frac{1}{2} \{2z^2 + y^2 - 2yz\} = … = (\frac{y}{2} - z)^2 + \frac{y^2}{4} \geq \frac{y^2}{4}\]

		\begin{figure}[hbtp]
			\begin{minipage}[t]{0.45\textwidth}
				\inputtikz{Plano-tg-vertical}
				\caption{Idea: los planos tangentes a $\Phi$ en la curva dato se quedan verticales }
			\end{minipage}
			\quad %add desired spacing between images, e. g. ~, \quad, \qquad, \hfill etc
			\begin{minipage}[t]{0.45\textwidth}
				\inputtikz{Ejemplo7-ProyXY}
				\caption{Hemos sombreado la región donde caen las proyecciones.}
				\label{fig:Ejemplo7-ProyXY}
			\end{minipage}
		\end{figure}

		Con esto vemos de nuevo que la función de $z$ no puede ser expresada con $x$ e $y$, por lo que no vamos a ser capaces de encontrar una solución como la que buscábamos.


	\end{example}
	\begin{example}{\bf 8}

		Vamos a mirar el mismo problema, pero con distinto dato incial:
		\[
		\left\{
		\begin{array}{l}
			uu_x+ u_y = 1\\
			u(\frac{s^2}{2}, s) = s
		\end{array}
		\right.
		\]

		{\bf Transversalidad}
		\[ \det \left. \begin{pmatrix}
			a & b\\
			\alpha' & \beta'
		\end{pmatrix} \right|_{t=0} =
		\begin{pmatrix}
			s & 1\\
			s & 1
		\end{pmatrix} =
		0\quad \forall s\]
		Luego tenemos un problema: no podemos decir si tenemos solución ni garantizar su unicidad. Vamos a ver si podemos tener al menos solución $C^1$ mirando el rango:

		{\bf Rango}
		\[
		\left. \begin{pmatrix}
			a & b & c \\
			\alpha' & \beta' & \gamma'
		\end{pmatrix}
		\right|_{t=0} = \text{Rango}
		\begin{pmatrix}
			s &1 & 1 \\
			s &1 & 1
		\end{pmatrix} = 1
		\]
		Luego no podemos garantizar que haya solución $C^1$, pero esto nos dice que si la hay, puede no ser única.

		Resolvemos el {\bf sistema característico} y obtenemos la siguiente superficie solución:
		\[
		\Phi(s,t) = (\frac{1}{2}(s+t)^2, (s+t), (s+t))
		\]
		Que con el cambio e variables adecuado vemos que
		\[ \Phi(s,t) \eqexpl[\equiv]{$s+t = \xi$} \sigma(\xi) = (\frac{\xi^2}{2}, \xi, \xi) \equiv \text{Curva dato} \]
		Luego hemos dado con un caso en el cual la solución es nuestra propia curva dato: cualquier característica es la curva dato.

		\begin{figure}[hbtp]
			\inputtikz{Ejemplo-Dato-Y-Sol-Coinciden}
			\caption{Cualquier característica que dibujemos coincide con el dato inicial.}
			\label{fig:Ejemplo-Dato-Y-Sol-Coinciden}
		\end{figure}

		Tomamos un dato alternativo que satisfaga la transversalidad. Por ejemplo $u(x,0) = Cx$. Si se resuelve para estos datos podremos hasta despejar explícitamente y llegaremos a que:

		\[ z = \frac{x-\frac{y^2}{2}}{y + \frac{1}{C}}+y
		\]

		Aquí deberíamos dibujar las proyecciones de las características y vemos que, para $C$ fijo, todas se cortan en un punto.

		\begin{figure}[hbtp]
			\inputtikz{Ejemplo8-Caract-C1}
			\caption{Ejemplo, $C=1$. Las características se cortan en $x=\frac{1}{2 C^2}$, $y = \frac{-1}{C}$}
			\label{fig:Ejemplo8-Caract-C1}
		\end{figure}

	\end{example}

	\begin{example}{\bf 9}

		Vamos a mirar este otro problema:
		\[
		\left\{
		\begin{array}{l}
			u_x+ u_y = 1 - u\\
			u(x,x+x^2) = \sin(x) \quad (x>0)
		\end{array}
		\right.
		\]

		Comprobamos la {\bf condición de transversalidad}:
		\[\det \left. \begin{pmatrix}
			a & b \\
			\alpha' & \beta'
		\end{pmatrix} \right|_{t=0} = 2s \neq 0 \text{ si } s > 0 \]

		Luego tenemos existencia y unicidad de la solución si $s\neq0$.

		Pasamos a resolver el {\bf sistema característico}\footnote{la $z$ es una EDO lineal, que se resuelve con el factor integrante $\mu = e^t$} y la solución es:
		\[
		\begin{array}{l}
		x = s+t \\
		y = s^2 + s + t \\
		z = 1 - (1 - \sin(s) )e^{-t}
		\end{array}
		\]

		\begin{wrapfigure}{l}{4cm}
			\centering
			\inputtikz{Proy-XY-Ejemplo-9}
		\end{wrapfigure}

		Para hacer la {\bf proyección en el plano XY} hay que observar que $y = s^2 + x $, luego tenemos que son rectas con pendiente 1 y ordenada en el origen $s^2$.

		Despejando obtenemos que
		\[y-x = s^2 > 0 \implies y > x\]

		En el punto 0 no estaría definida la solución pero podemos observar que pasa en la región inferior a 0 de todas formas:

		\clearpage
		Tomando $ s\in \mathbb{R} $, vemos que las características cortan a la curva en dos puntos:

		\begin{figure}[hbtp]
			\centering
			\inputtikz{Proy-XY-Ejemplo-9-todo-R}
		\end{figure}

		Buscamos saber si hay algún entorno del 0 en la región izquierda en el cual la solución esté definida.

		¿Es esto compatible con que el sistema tenga solución? Veámoslo con un ejemplo:

		\textbf{Ej}
		\[g(x) = u(x,x+s^2)\]
		\[g'(x) = u_x (x,x+s^2) + u_y(x,x+s^2) = 1 - u(x,x+s^2) = 1 - g(x)\]
		\[g' = 1-g\]
		De lo que obtenemos que $(x,x+s^2) \rightarrow $ corta al dato en $x=s, x=-s$.

		Tendríamos dos datos:

		\[
		\begin{cases}
		g(s) = \sin(s) \\
		g(-s) = \sin(-s)
		\end{cases}
		\]

		Encontramos las curvas características y lo que antes era una EDP se convierte en una EDO a lo largo de la curva. Como alguna característica corta al dato en dos puntos, tenemos que comprobar si la EDO a lo largo de la curva es compatible con los dos datos a la vez. La idea es resolver la EDO con uno de los datos, y una vez resuelta y despejada la constante de integración comprobamos si es compatible con el segundo dato.

		Si la compatibilidad se cumpliera habría solución pero si no, solo se puede obtener solución en regiones aisladas. En este caso no lo vamos a comprobar pero no son compatibles.

	\end{example}

	\obs El profesor ha confeccionado un documento explicando mejor este ejemplo, que hemos incluido en el apéndice \ref{apx:JGA-cuasilinear}.


\chapter{Ecuaciones de segundo orden}
\label{chap:EcuacionesSegundoOrden}

\section{Método de separación de variables}

	Al empezar el curso ya vimos un ejemplo: la ecuación del calor en una dimensión con datos de contorno Dirichlet homogéneos.

	\begin{example}{1. Ecuación de calor con contorno Dirichlet}
		\[
		\begin{cases}
		u_t + u_xx = 0 & x \in (0,L), \quad t > 0 \\
		u(0,t) = u(L,t) = 0 & t > 0 \\
		u(x,0) = f(x) & x \in [0,L]
		\end{cases}
		\]

		La ecuación del medio es el dato de contorno de Dirichlet homogéneo, es decir, que especifica el dato en los extremos.

		Llegamos con separación de variables a que la solución del problema podía ser escrita como:

		\[ u(x,t) \eqexpl{?} \sum^{\infty}_{k=1} a_k e^{-(\frac{k\pi}{L})^2 t} \sin \left( \frac{k\pi}{L} x \right) \]
		donde
		\[ f(x) \eqexpl{?} \sum^{\infty}_{k=1} a_k \sin \left( \frac{k\pi}{L}x \right) \]
	\end{example}


	\begin{example}{2. Ecuación de calor con contorno de Neumann}

		\[
		\begin{cases}
		u_t + u_{xx} = 0 & x \in (0,L), t > 0 \\
		u(0,t) = u(L,t) = 0 \\
		u(x,0) = f(x) \in [0,L]
		\end{cases}
		\]

		Esta condición indica que no hay flujo de calor entre la varilla y cualquier punto fuera, incluidos los extremos. Esperamos que al final, cuando el tiempo tienda a infinito el calor se haya distribuido a lo largo de la varilla y la temperatura sea constante a lo largo de esta. El valor de esto probablemente sea el promedio.

		Empecemos con el método de separación de variables. Buscamos $u(x,t) = X(t) T(t)$ que sea solución de la ecuación con el contorno (el dato inicial se tratará después).


		\[
		\begin{array}{l}
			0 = u_t - u_xx = T' X - T X'' \\
			0 = u_x (0,t) = T(t) X'(0) \\
			0 = u_x (t,t) = T(t) X'(L)
		\end{array}
		\]

		De lo que obtenemos:

		\[ \frac{T'(t)}{T(t)} = \frac{X''(x)}{X'(x)} \quad \forall x, \forall t \]

		A esta proporción la podemos llamar $\lambda$:

		\[ \frac{T'}{T} = \frac{X''}{X'} = \lambda \in \mathbb{R} \]

		% Método general?
		Resolvemos la EDO en X:

		\[
		\left\{ \begin{array}{l}
		X'' = \lambda X \\
		X'(0) = X'(L) = 0
		\end{array} \right. \quad\quad \text{(problema de controno)}
		\]

		Veamos carios casos en función de $\lambda$:

		\begin{itemize}
			\item $\lambda = 0$

				Cuando $\lambda = 0 \Rightarrow X'' = 0$. Así que tenemos que $X'$ tiene que ser constante y $X$ lineal. Pero además los datos iniciales nos indican el valor de $X'$, al ser constante.

				\[ \left.
				\begin{array}{l}
					X(x) = a + bx \\
					\left.
					\begin{array}{r}
						X'(x) = b \\
						X'(0) = X'(L) = 0
					\end{array} \right\} \Rightarrow b = 0
				\end{array} \right\} \Rightarrow X \equiv a \]

				Tiene una solución no trivial que es $\lambda = 0, X=a_0$.

			\item $\lambda > 0$ con $\lambda = \mu^2$, $\mu \in \mathbb{R}$

				Lo cual nos lleva a una EDO de orden 2, que se resolvería con el polinomio característio.

				\[ \text{Las soluciones siguen } \left\{
				   \begin{array}{l}
				   	X(x) = a e^{\mu x} \\
				   	X'(x) = \mu (ae^{\mu x} - be^{-\mu x})
				   \end{array} \right.
				\]

				\[ \left. \begin{array}{l}
					0 = X'(0) \Rightarrow \mu(a - b) = 0 \\
					0 = X'(L) \Rightarrow \mu(a e^{\mu L} - b e^{-\mu L}) = 0
				\end{array} \right\}
					\Rightarrow … \Rightarrow a = b = 0
				\]


			\item $\lambda < 0$ con $\lambda = - \mu^2$

				Aquí volvemos a emplear el polinomio característico pero llegaremos a soluciones complejas.

			 	\[ \text{Solución} \left\{
				   \begin{array}{l}
				   	X(x) = a \cos(\mu x ) + b \sin( \mu x) \\
				   	X'(x) = -a \mu \sin(\mu x) + b \mu \cos(\mu x)
				   \end{array} \right.
				\]

			 	\[
			 		\begin{array}{l}
			 		0 = X'(0) = b \mu \\
			 		0 = X'(L)
			 		\end{array} \Rightarrow b = 0 \Rightarrow \left\{ \begin{array}{l}
			 			X(x) = + a \cos (\mu x ) \\
			 			X'(x) = -a \mu \sin (\mu x)
			 		\end{array} \right.
			 	\]

			 	De lo que obtenemos que

			 	\[0 = X'(L) = -a \mu \sin(\mu L) \Rightarrow \mu L = k \pi , \quad k = 1,2,…\]



		\end{itemize}

		Conclusión:

				\begin{align*}
					\lambda_0 = 0\quad & \quad X_0 = a_0 \\
					\lambda_k = - \left(\frac{k\pi}{L}\right)^2\quad & \quad X_k(x) = a_k \cos \frac{k \pi}{L}x
				\end{align*}

			 	EDO para T (para las $\lambda$ encontradas antes)

			 	\[\lambda_0 = 0 \Rightarrow T'_0 \equiv 0 \Rightarrow T_0 \equiv \lambda_0\]
			 	\[\lambda_k = - \left(\frac{k\pi}{L}\right)^2 \Rightarrow T'_k = \left(\frac{k\pi}{L}\right)^2 T_k \Rightarrow T_k (t) = \lambda_k e^{-\left(\frac{k\pi}{L}\right)^2 t} \]

			 	Soluciones particulares:

			 	\[u_0(x,t) = A_0, \quad u_k (x,t) = A_k e^{-\left(\frac{k \pi}{L} \right)^2 t} \cos \left( \left( \frac{k \pi}{L}\right) x \right) \]

			 	Dato inicial: $u(x,0) = f(x)$

			 	Idea: $u(x,t) \eqexpl{?} A_0 + \sum\limits_{k=1}^{\infty} A_k e^{- \left( \frac{k \pi}{L} \right)^2 t}  \cos \left( \frac{k \pi}{L} x \right)$

			 	Pero claro, no sabemos calcular $A_k$. ¿O como calculamos la convergencia? ¿Cómo calculamos las derivadas?


		\end{example}


		\begin{example}{3: Cuerda vibrante}

			Veamos una cuerda de guitarra en tensión. La guitarra está atada en los extremos y la altura sobre el eje horizontal es $u$.

			\begin{figure}[thbp]
			\centering
			\inputtikz{cuerdaGuitarra}
			\caption{}
			\label{fig:cuerdaGuitarra}
			\end{figure}


			\[  \begin{array}{l}
				u_{tt} - u_{xx} = 0 \quad \text{ 2º orden \quad 2 datos } \\
				u(0,t) = u(L,t) = 0 \quad \text{Dirichlet}\\
				u(x,0) = f(x) \\
				u_t(x,0) = g(x)
				\end{array}
			\]

			Por separación de variables. Buscamos un $u(x,t) = X(t) T(t)$, solución de la ecuación con el contorno:

			\[ 0 = u_{tt} - u_{xx} = T'' X - T X''\]
			\[ \frac{T''}{T} = \frac{X''}{X} = \lambda \in \mathbb{R}\]

			EDO para $X$:

			\[\begin{cases}
				X'' = \lambda X \\
				X(0) = X(L) = 0
			\end{cases}
			\]

			Vemos que ha cambiado respecto al sistema anterior en que la última ecuación ya no relaciona las derivadas de $X$ sino  $X$.

			\begin{itemize}
				\item $\lambda = 0$

					\[
					\left\{
					\begin{array}{l}
					X(x) = a + bx \\
					X(0) = 0 = X(L)
					\end{array}
					\right.
					\Rightarrow
					a = 0 = b
					\]

				\item $\lambda > 0$ con $\lambda = \mu^2$

					\[
					\left\{
					\begin{array}{l}
					X(x) = ae^{\mu x} + be^{-\mu x} \\
					X(0) = 0 = X(L)
					\end{array}
					\right.
					\Rightarrow … \Rightarrow
					a = b = 0
					\]

				\item $\lambda < 0$ con $\lambda = -\mu^2$

					\[
					\left\{
					\begin{array}{l}
					X(x) = a\cos(\mu x) + b\sin(\mu x) \\
					X(0) = 0 = X(L)
					\end{array}
					\right.
					\Rightarrow X(0) = a \Rightarrow X(x) = b \sin(\mu x)
					\]

					\[ \Rightarrow X(L) = 0 = b \sin (\mu L) \Rightarrow \mu = \frac{k \pi}{L}\]

			\end{itemize}

			Con lo que llegamos a las soluciones no triviales:

			\[\lambda_k = - (\frac{k\pi}{L})^2, \quad X_k(x) = b_k \sin \left(\frac{k\pi}{L} \right) x\]


			Una vez que resolvemos la EDO para $X$, la resolvemos para $T$:

			\[T'' = \lambda T\]

			Es similar a la X así que tenemos:

			\[T''_k = - (\frac{k\pi}{L})^2 T_k \Rightarrow T_k (t) = \alpha_k \cos\left( \frac{k \pi}{L} t \right) + \beta_k \sin \left( \frac{k \pi}{L}t \right)\]

			Con lo que llegamos a las soluciones particulares:

			\[u_k(x,t) = A_k \cos \left(\frac{k\pi}{L} t\right) \sin \left(\frac{k\pi}{L}x\right) + B_k \sin \left(\frac{k\pi}{L}t\right)  \sin \left(\frac{k\pi}{L}x\right) \]

			Idea: Buscar

			\[u(x,t) \eqexpl{?} \sum_{k=1}^{\infty} A_k \cos \left(\frac{k\pi}{L} t\right) \sin \left(\frac{k\pi}{L} x  \right)+ B_k \sin \left(\frac{k\pi}{L}t \right) \sin \left(\frac{k\pi}{L}  x \right)\]

			Datos iniciales:

			\[ f(x) = u(x,0) \eqexpl{?} \sum^{\infty}_{k=1} A_k \sin \left(\frac{k\pi}{L} x  \right)\]

			Suponiendo que derivada e integral conmutan:

			\[ u_t (x,t) \eqexpl{?} \sum_{k} - A_k \left(\frac{k\pi}{L} \right) \sin \left(\frac{k\pi}{L}t \right) \sin \left(\frac{k\pi}{L}x \right) + B_k \left(\frac{k\pi}{L} \right) \cos \left(\frac{k\pi}{L}t \right) \sin \left(\frac{k\pi}{L}x \right)
			\]

			\[g(x) = u_t(x,0) \eqexpl{?} \sum_k B_k  \left(\frac{k\pi}{L} \right) \sin \left(\frac{k\pi}{L}x \right)\]

		\end{example}


		\begin{example}{4: Ondas con condiciones periódicas}

			Estudiemos, por ejemplo, las olas en altamar. No tenemos un contorno fijo como antes, así que vamos a buscar soluciones que sean periódicas en los extremos. En este caso tendremos dos condiciones. Llamadas condiciones de periodicidad. Hemos puesto dos porque lo observamos en segundo orden:

			\[u(-L,t) = u(L,t), \forall t\]
			\[u_x(-L,t) = u_x(L,t), \forall t\]

			El problema nos queda así:

			\[  \begin{array}{l}
				u_tt - u_xx = 0 \quad x  \in (-L,L), t>0\\
				u(-L,t) = u(L,t), \forall t\\
				u_x(-L,t) = u_x(L,t), \forall t \\
				u(x,0) = f(x) \\
				u_t(x,0) = g(x)
				\end{array}
			\]

			Separamos variables:
			\[ \frac{T''}{T} = \frac{X''}{X} = \lambda \in \mathbb{R}\]

			EDO para $X$:

			\[\left\{\begin{array}{l}
				X'' = \lambda X \in (-L,L) \\
				X(-L) = X(L) \\
				X'(-L) = X'(L)
			\end{array}
			\right. \]

			% Revisado hasta aquí

			\begin{itemize}
				\item $\lambda = 0$ (RELLENAR)
				\item $\lambda > 0$ (RELLENAR)
				\item $\lambda < 0$ con $\lambda = -\mu^2$ (RELLENAR)

					\[ X (x) = a \cos \mu x + b \sin \mu x \]
			\end{itemize}

			\[ X(-L) = X(L)\]
			\[ X'(L) = X'(L)\]

			Ajustamos $\mu$ para que $X$ sea periódica con periodo $2L k \Rightarrow … \Rightarrow \mu = \frac{k \pi}{L}$. Donde $2L$ es la longitud del intervalo. Por lo que llegamos a las soluciones:

			\[
			\lambda_k = -(frac{k\pi}{L})^2\quad\quad X_k = a_k \cos frac{k\pi}{L} x + b_k \sin frac{k\pi}{L} x
			\]

			EDO para $T$:

			(RELLENAR)


			Tenemos soluciones particulares:

			\[u_k(x,t) = T_k(t) X_k(t) \quad (k=0,1,2,…)\]


			y soluciones en forma de serie:

			\[u(x,t) = \sum^{\infty}{k=0} u_k(x,t)\]



		\end{example}




















\section{Series de Fourier}

% -*- root: ../EDP2016.tex -*-
\chapter{Comportamiento cualitativo}

% clase 15/3/2016

En este capítulo vamos a ver:

\begin{itemize}

	\item Ondas (hiperbólica)
	\item Calor (parabólica)
	\item Laplace (elípticas)

\end{itemize}


\section{Ondas}

	Hemos visto ya el problema en dimensión 1: la cuerda vibrante.

	Hemos visto también el método de separación de variables que solo nos valdrá en un dominio acotado. Pero, ¿qué podemos hacer en dominios no acotados? ¿Hay más soluciones?.

	Después vimos la fórmula D'Alambert \eqref{eq:DALEMBERT}, que sí que nos sirve para dominios no acotados pero es difícil de interpretar en dominios acotados. Veamos un ejemplo de esta afirmación:

	\begin{example}

		\[\begin{cases}
			u_{tt} - u_{xx} = 0, x \in (0,L) t > 0 \\
			u(0,t) = u(L,t) = 0, t > 0\\
			u(x,0) = f(x) \\
			u_t(x,0) = 0
		\end{cases}\]

		Y tenemos la fórmula a la que llegamos:

		\[ u(x,t) = \frac{1}{2} \{f(x+t)+f(x-t)\}  \]

		Resulta que tenemos una fórmula muy útil pero ocurre que en un dominio acotado no podemos calcular $f$ cerca del borde al realizar $x+t$ o $x-t$. Tenemos que encontrar una manera de extender $f$ de manera que esté de acuerdo con el contorno. No es lo mismo que una cuerda esté sujeta y la onda se refleje de una manera de vuelta en la cuerda a que no esté sujeta.

		\begin{center}
			\begin{tikzpicture}
			\draw[-] (-1,0) -- (3,0);
			\draw[-] (0,-0.5) -- (0,2);


			\draw[dashed] (1.5,0) node [below] {$L$} -- (1.5,1.5);

			\draw[thick, blue] plot[smooth, tension=.9] coordinates{(0.7,0) (0.8,0.2) (1,0.4) (1.2,0.7) (1.5,0.9)};

			\end{tikzpicture}
		\end{center}

	\end{example}

	\subsection{Unicidad y conservación de la energía}

		Uno de los problemas más habituales al tratar las ecuaciones en derivadas parciales es saber si las soluciones son únicas o no. Partimos de nuestro problema
		\[ \begin{cases}
			u_{tt}-u_{xx} = 0\qquad x \in (0,L), t >0 \\
			u(x,0) = f(x) 	\\
			u_t(x,0) = g(x) \\
			\text{Datos de contorno Dirichlet, Neumann o periódicos}
		\end{cases} \]

		Supongamos que existen dos soluciones $u_1, u_{2}$. Entonces, tendremos que $u = u_1 - u_2$ será solución del sistema
		\( \begin{cases}
			u_{tt}-u_{xx} = 0\qquad x \in (0,L), t >0 \\
			u(x,0) = 0 	\\
			u_t(x,0) = 0 \\
			\text{Datos de contorno}
		\end{cases} \label{eq:Onda:ProblemaUnicidad} \)

		Para comprobar la unicidad, querremos saber si $u \equiv 0$ ($u_1$ y $u_2$ son iguales y la solución es única) o no.

		\begin{figure}[hbtp]
		\centering
		\inputtikz{EnergiaOnda}
		\caption{Una ilustración de las fuerzas que dependen de la velocidad ($\vf_c$) y de la posición ($\vf_p$) que llevan a la definición de energía cinética y potencial.}
		\label{fig:EnergiaOnda}
		\end{figure}

		Para ello, vamos a introducir la ``energía'' de la onda y vamos a ver si nos da algo interesante. En cada punto $x ∈ [0,L]$ vamos a tener dos ``fuentes'' de energía. Una dependerá de la velocidad que lleve un punto, que será la cinética. Mirando a la ecuación de onda, esa velocidad no es más que la derivada de la onda $u$ con respecto al tiempo: si en $t_0 + Δt$ la onda ha crecido entonces llevamos velocidad positiva.

		La otra fuente de energía será la potencial, que depende de la altura $u$ del punto que consideremos. Si nos fijamos de nuevo en la \fref{fig:EnergiaOnda}, de lo que depende es de la derivada con respecto a $x$ por razones que ahora mismo no tengo claras. En cualquier caso, haciendo un ejercicio de imaginación la expresión de la energía será entonces la suma de ambas a lo largo de toda la cuerda:
		\( E(t) = \frac{1}{2} \int_0^L (u_t)^2 + (u_x)^2 \dif x \label{eq:Onda:Energia} \)

		Podemos derivar esta ecuación con respecto al tiempo, ya que $u ∈ C^2$, y entonces tenemos que  \( E'(t) = \int_{0}^L u_t u_{tt} + u_x u_{xt} \dif x \label{eq:Onda:DerivadaEnergia} \)

		¿Cuánto vale esta derivada? Vamos a verlo calculando la integral de $u_x u_{xt}$:
		\begin{multline}
		\int_0^L \underbracket{u_x}_u \underbracket{u_{xt} \dif x}_{\dif v} = \eval[2]{u_x u_t}_{x= 0}^L - \int_0^L u_{xx} u_t \dif x = \\
		= u_x(L,t) u_t(L, t) - u_x(0,t) u_t(0,t) - \int_0^L u_t u_{xx} \dif x \label{eq:Onda:ContornosIntegral}
		\end{multline}

		Como se puede ver, aquí entran en juego las condiciones de contorno. Recordamos las tres posibilidades que tenemos:

		\begin{itemize}[itemsep = 0pt]
		\item \textbf{Neumann}: $u_x(L, t) = u_x(0,t) = 0$.
		\item \textbf{Dirichlet}: $u(0,t) = u(L, t) = 0 = u_t(0,t) = u_t(L, t)$.
		\item \textbf{Periódicas}: $u(L, t) = u(0,t)$, $u_x(L, t) = u_x(0,t)$, $u_t(L, t) = u_t(0,t)$.
		\end{itemize}

		En cualquiera de los tres casos, lo que vamos a tener va a ser lo mismo: que $u_x(L,t) u_t(L, t) - u_x(0,t) u_t(0,t) = 0$. Sustituyendo eso en \eqref{eq:Onda:ContornosIntegral}, lo que nos quedará será que \[ \int_0^L u_x u_{xt} \dif x = - \int_{0}^L u_t u_{xx} \dif x \], y a su vez esto nos permite resolver la ecuación para la derivada de la energía \eqref{eq:Onda:DerivadaEnergia}:
		\[ E'(t) = \int_0^L u_t \left(u_{tt} - u_{xx} \right) \dif x = 0\]
		simplemente fijándonos en que la ecuación de onda nos dice que $u_{tt} - u_{xx} = 0$.

		Finalmente, a lo que hemos llegado no es más que al tradicional principio de \textbf{conservación de la energía}. En este caso, la energía de la onda se conserva a lo largo del tiempo para la ecuación de onda homogénea, y $E(t) = E(0)$, donde
		\[ E(0) = \frac{1}{2} \int_0^L (u_t)^2 (x,0) + (u_x)^2 (x,0) \dif x = \frac{1}{2} \int_0^L g^2(x)+(f'(x))^2 \dif x \]

		\paragraph{Consecuencias} Una vez que tenemos esto, volvemos a las condiciones iniciales \eqref{eq:Onda:ProblemaUnicidad} del problema del que $u = u_1 - u_2$ era solución, en el que los datos iniciales eran $u(x,0) = u_t(x,0)$. Esto nos dice que $E(0) = 0$, y como la energía se conserva tenemos que $E(t) = 0\;∀t ∈ ℝ^+$. Equivalentemente, $0 = \int_0^L (u_t)^2 + (u_x)^2 \dif x$, luego $u_t = u_x = 0$ así que la solución es constante. Finalmente, como el dato inicial era $u(x,0) = 0$, la solución es $u \equiv 0$, luego tenemos \textbf{unicidad}.

		 	Veamos el caso Dirichlet, pero esto es válido para todos:

		 	\[ \begin{cases}
		 		u_{tt} - u_{xx} = F(x,t)\\
		 		u(0,t) = \alpha(t), u(L,t) = \beta{t} \\
		 		u(x,0) = f(x) \\
		 		u_t(x,0) = g(x)
		 	\end{cases}\]
		 	\[u_1,u_2 \in C^2 \text{ soluciones}\]

		 	En este problema no tenemos conservación, debido a $\alpha$, $\beta$ y $F(x,t)$ pero podemos transformarlo en un problema de $W$ en donde si tengamos esta propiedad:

		 	\[\begin{cases}
		 		W = u_1 - u_2 \\
		 		W_{tt} - W_{xx} = 0 \\
		 		W(0,t) = 0 = W(L,t) \\
		 		W(x,0) = 0 \\
		 		W_t(x,0)
		 	\end{cases}\]

		 	Por conservación de la energía tenemos:

		 	\[ \int_0^L (W_t)^2 + (W_x)^2 dx = 0 \Rightarrow W_t = W_x = 0 \Rightarrow W \equiv \text{cte.} \eqreason[\Rightarrow]{$W|_{t=0} = 0$} W(x,t) = 0 \forall x \forall t \Rightarrow u_1 = u_2 \]

		 	Por lo que hemos obtenido unicidad de las soluciones del problema inicial. Pero aunque sepamos que la solución es única, no sabemos si existe tal solución. Volvemos al problema inicial:

		 	\[ \begin{cases}
		 		u_{tt} - u_{xx} = F(x,t)\\
		 		u(0,t) = \alpha(t), u(L,t) = \beta{t} \\
		 		u(x,0) = f(x) \\
		 		u_t(x,0) = g(x)
		 	\end{cases}\]


		 	\[ C(x,t) \text{ tq. } C(0,t) = \alpha(t), C(L,t) = \beta{t}\]

		 	Por ejemplo: $C(x,t) = \alpha(t) + \frac{x}{L} (\beta(t)-\alpha(t))$

		 	Consideramos $v(x,t) = u(x,t) - c(x,t)$ (de modo que $v(0,t) = v(L,t) = 0$)

		 	\[\begin{cases}
		 		v_{tt} - v_{xx} = u_{tt} - c_{tt} - u_{xx} + c_{xx} = F - c_{tt} + c_{xx} = \gor{F} \\
		 		v(0,t) = v(L,t) = 0 \\
		 		v(x,0) = u(x,0) - c(x,0) = f(x) - c(x,0) = \gor{f}(x) \\
		 		u_t (x,0) = … = \gor{g}(x)
		 	\end{cases}
		 	\]

		 	Reescribimos nuestro sistema inicial con $v$ y obtenemos:

		 	\[ \begin{cases}
		 		v_{tt} - v_{xx} = \gor{F}(x,t)\\
		 		v(0,t) = 0, v(L,t) = 0 \\
		 		v(x,0) = \gor{f}(x) \\
		 		v_t(x,0) = \gor{g}(x)
		 	\end{cases}\]

		 	Tomando $v = H + W$ podemos separar nuestro problema en dos:

		 	\[ \begin{cases}
		 			H(x,t) \rightarrow
		 			\begin{cases}
				 		H_{tt} - H_{xx} = 0\\
				 		H(0,t) = H(L,t) = 0 \\
				 		H(x,0) = \gor{f}(x) \\
				 		H_t(x,0) = \gor{g}(x)
			 		\end{cases}\\
			 		W(x,t) \rightarrow
			 		\begin{cases}
				 		W_{tt} - W_{xx} = \gor{F}(x,t)\\
				 		W(0,t) = 0 = W(L,t) \\
				 		W(x,0) = 0 \\
				 		W_t(x,0) = 0
				 	\end{cases}
				 \end{cases}
			\]

			El sistema de la $H$ ya lo sabemos resolver por separación de variables, Fourier...

			El problema de la $W$ lo resolvemos aplicando Duhamel (método del impulso). \fbox{Fijamos $s>0$}:

			\[\begin{cases}
				\left.
				\begin{array}{l}
					\Phi_{tt} - \Phi_{xx} = 0 \\
					\Phi(0,t) = \Phi(L,t) = 0 \\
					\Phi(x,0) = 0 \\
				\end{array}
				\right| \text{ separación de variables, Fourier} \\
				\Phi_t(x,0) = \gor{F}(x,s)
			\end{cases}\]


			\textbf{Notación:} $ \Phi = \Phi(x,t,s) $

			Respuesta al impulso $\gor{F}(x,s)$ que actúa en $t=0$.

			\[ \Phi(x,t-s,s)\]

			Respuesta al impulso $\gor{F} (x,s)$ que actúa en $t-s = 0$ ($t=s$)

			Aplicando Duhamel:

			\[ W(x,t) = \int_0^t \Phi(x,t-s,s) ds \]

			\[ G(x,t,z) = \int_0^z \Phi(x,t-s,s) ds\]
			\[ G_t = \int_0^z \Phi_t (x,t-s,s) ds \]
			\[ G_z = \Phi(x,t-z,z) \]
			\[ W(x,t) = G(x,t,t)\]
			\[ W_t = G_t + G_z z_t = G_t(x,t,t) + G_z (x,t,t) \cdot 1\]

			\[ W_t(x,t) = \Phi(x,0,t)+ \int_0^t \Phi_t (x,t-s,s) ds\]
			\[ W_{tt} = \Phi_{s} (x,0,t) + \Phi_t(x,0,t) + \int_0^t \Phi_{tt} (x,t-s,s) ds  \]
			\[ W_{xx} = \int_0^L \phi_{xx} (x,t-s,s) ds \]
			\[ W_{tt} - W_{xx} = \Phi_{s}(x,0,t) + \Phi_t(x,0,t) + \int_0^L \phi_{tt} - \Phi_{xx} dx \]

			\[ W(0,t) = \int_0^t \underbrace{\Phi(0,t-s,s)}_{\equiv 0} ds = 0  \]
			\[ W(L,t) = \int_0^t \Phi(L,t-s,s) ds = 0 \]

			\[ W(x,0) = \int_0^0 \Phi(x,0-s,s) ds = 0 \]
			\[ W_t (x,0) = 0\]

			Nos hemos dejado el lado derecho de la ecuación, que es:

			\[\Phi_s (x,0,t) + \underbrace{\Phi_t(x,0,t)}_{\gor{F}(x,t)} \]

			Finalmente:

			\[ \Phi(x,0,s) = 0, \forall s \Rightarrow \Phi_s (x,0,s) = 0 \forall s  \]
			\[ \Phi_t(x,0,s) = \gor{F}(x,s), \forall s \]

			Hemos descompuesto el problema en cuatro problemas más pequeños. Y hemos resuelto cada uno, completándolo con este último. La suma de ellos será una solución del primero, y además sabemos que es única por el resultado obtenido antes.

			Ahora veamos que pasa con otras condiciones de contorno:

			\[ \begin{cases}
		 		u_{tt} - u_{xx} = F(x,t)\\
		 		u(0,t) = \alpha(t), u(L,t) = \beta(t) \\
		 		u(x,0) = f(x) \\
		 		u_t(x,0) = g(x)
		 	\end{cases}\]

		 	Lo cual cambiamos al problema:

		 	\[ \begin{cases}
		 		v_{tt} - v_{xx} = F(x,t)\\
		 		v(0,t) = \alpha(t), v(L,t) = \beta(t) \\
		 		v(x,0) = \gor{f}(x) \\
		 		v_t(x,0) = \gor{g}(x)
		 	\end{cases}\]

		 	Y llegamos igual que antes hasta el problema de $W$ ($W = u-v$):

		 	\[ \begin{cases}
		 		W_{tt} - W_{xx} = 0\\
		 		W(0,t) = 0 = W(L,t) \\
		 		W(x,0) = f - \gor{f} \\
		 		W_t(x,0) = g-  \gor{g}
		 	\end{cases}\]

		 	Por conservación de la energía:

		 	\[ \frac{1}{2} \int_0^L (W_t)^2 + (W_x)^2 dx = \frac{1}{2} \int_0^L (g-\gor{g})^2 + ((f-\gor{f})')^2 dx \]

		 	Hemos demostrado también dependencia continua de datos. Lo que quiere decir que datos pequeños nos llevan a energías pequeñas.

		 	Y por lo tanto acabamos diciendo que el problema de las ondas es un problema bien propuesto: tiene existencia, unicidad y dependencia contínua.


		 	% Clase gjulianm 29/3

		\subsection{Deducción de la ecuación de onda}

		\begin{figure}[hbtp]
		\centering
		\inputtikz{TensionCuerda}
		\caption{Esquema para la demostración, dejando sólo un trozo de cuerda y sustituyendo por la tensión.}
		\label{fig:TensionCuerda}
		\end{figure}

		La idea para la deducción de la ecuación de onda es la siguiente: quitar un cacho de cuerda y sustituirlo por una tensión $T(x,t)$, que representa el efecto del trozo de cuerda a la derecha del punto $x$ en el instante $t$.

		Las componentes horizontales están en equilibrio, luego tienen que anularse: \( T(x+h) \cos θ(x+h) - T(x) \cos θ(x) = 0 \label{eq:Onda:EquilibrioHorizontal} \)

		Las componentes verticales tienen que seguir la Ley de Newton: \( T(x + h) \sin θ(x+h) - T(x) \sin θ(x) = m · u_{tt} \label{eq:Onda:LeyNewton} \), donde $m$ es la masa que se calcula a partir de la densidad ρ (constante) y de la longitud de la curva: \[ m = ρ \int_{x}^{x+h} \sqrt{1+u_x^2} \dif s\]

		Para hacer la deducción de la ecuación, dividiremos entre $h$ en \eqref{eq:Onda:LeyNewton}, y haciendo tender $h \to 0$ tenemos la derivada: \[ \left(T(x) \sin θ(x)\right)_x = ρ\sqrt{1+u_x^2(x,t)} u_{tt} \]

		Haciendo el truco de escribir $\sin θ = \cos θ · \tan θ$, tenemos que $\tan θ = u_x$ y nos queda lo siguiente:  \[ \left(T(x) \cos θ(x) u_x \right)_x = ρ\sqrt{1+u_x^2(x,t)} u_{tt} \]

		Ahora derivamos y vemos qué ocurre: \[ \left(T(x) \cos θ(x) u_x \right)_x = \left(T(x) \cos θ(x)\right)_x u_x + T(x) \cos θ(x) u_{xx} \]

		Podemos simplificar viendo que \eqref{eq:Onda:EquilibrioHorizontal} nos decía que $T(x+h) \cos θ(x+h) - T(x) \cos θ(x) = 0$, luego $(T(x) \cos θ(x))_x = 0$. Además, por no sé qué tenemos que $T(x) \cos θ(x)$ es constante. % TODO

		Sólo nos falta quitarnos la raíz esa fea. Para ello hacemos la simplificación de tener oscilaciones pequeñas, de tal forma que $u_x$ es pequeño, $u_x^2$ es todavía más pequeño y $1 + u_x^2 \approx 1$. Así, nos queda la ecuación que ya conocemos: \[ u_{tt} - c^2 u_{xx} = 0\]

		\subsection{Ecuación de ondas en $ℝ$}

		Partimos de nuestra ecuación homogénea: \[ \begin{cases}
		u_{tt} - c^2 u_{xx} = 0 \\
		u(x,0) = f(x) \\
		u_t(x, 0) = g(x)
		\end{cases}\]

		La cuestión es que aquí no podemos aplicar las series de Fourier, porque no tenemos una longitud acotada: estamos estudiando la ecuación en todo $ℝ$. Lo que haremos será ver si aplicando la Fórmula de D'Alembert nos sale algo. Para eso, hacemos el siguiente cambio de variables: \begin{align*}
		ξ &= x + ct & ξ_x = 1,\,& ξ_t = c \\
		η &= x - ct & η_x = 1,\,& η_t = - c \\
		\end{align*}

		Haciendo las cuentas tenemos lo siguiente: \begin{align*}
		u_x &= u_ξ + u_η \\
		u_{xx} &= u_{ξξ} + 2 u_{ηξ} + u_{ηη} \\
		u_t &= c u_ξ - c u_η \\
		u_{tt} &= c^2 u_{ξξ} + 2c^2 u_{ηξ} + c^2 u_{ηη}
		\end{align*}

		La conclusión de todo esto es que la ecuación nos queda así: \[ 0 = u_{tt} - c^2 u_{xx} = -4c^2 u_{ηξ} \], de tal forma que la ecuación final nos queda muy simple: \[ u_{ηξ} = 0\]

		De ahí podemos sacar fácilmente que $u_η = α(η)$, una función que sólo depende de $η$. Por tanto, podemos sacar la fórmula para $u$: \[ u = \int α(η) \dif η + B(ξ) = A(η) + B(ξ) = A(x-ct) + B(x+ct)\]

		Si $u(x,0) = f(x)$, entonces $f(x) = A(x) + B(x)$, y si $u_t(x,0) = g(x)$ entonces $g(x) = -cA'(x) + cB'(x)$. El sistema resultante es el siguiente: \[ \begin{cases}
		A' + B' = f \\
		-A' + B' = \dfrac{1}{c} g
		\end{cases} \]

		Resolviéndolo, llegaremos de nuevo a la fórmula de D'Alembert: \( u(x,t) = \frac{1}{2} \left(f(x + ct) + f(x-ct)\right) + \frac{1}{2c} \int_{x-ct}^{x+ct} g(s) \dif s \label{eq:Onda:DAlembert} \)

		A partir de aquí podemos encontrar algunas consecuencias y estudiar la función de onda. Lo primero es ver que la única solución del problema en todo $ℝ$ viene dada por esta fórmula: no hemos introducido ni quitado nuevas soluciones.

		\begin{figure}[hbtp]
		\centering
		\inputtikz{ReprGraficaEcOnda}
		\caption{Representación gráfica y dominio de dependencia de la solución y de influencia de los datos.}
		\label{fig:ReprGraficaEcOnda}
		\end{figure}

		Lo siguiente es estudiar lo que se ve en la \fref{fig:ReprGraficaEcOnda}. $u(x,t)$ sólo dependerá de los valores de $f$ en los puntos $a,b$ y del valor de $g$ en el intervalo $(a,b)$. La zona sombreada en la figura será el dominio de influencia del intervalo $[a,b]$: es sólo ahí donde influyen los datos de $f,g$ de esos puntos. Por ejemplo, si en $t = 0$ $f = g \equiv 0$ fuera de $[a,b]$, entonces en $t = T$ la solución será cero fuera de $[a - cT, b + cT]$. En otras palabras, hay una velocidad de propagación finita de los datos.

		La aplicación de esto es la demostración de la conservación de la energía. Partíamos de una ecuación \[ E(t) = \frac{1}{2} \int_{-∞}^∞ u_x^2 \dif x + \frac{c^2}{2} \int_{-∞}^∞ u_t^2 \dif x\], y derivando y haciendo cuentas nos salía que \[ E'(t) = \int_{-∞}^∞ u_t(u_{tt} - c^2 u_{xx}) \dif x = 0\], luego la energía se conservaba.

		Ahora bien, cuando hacíamos eso en dominios acotados, lo que necesitamos era la hipótesis de los valores de contorno para poder demostrar que salía lo que tenía que salir. Aquí lo que necesitaremos simplemente es que $f(x)$ y $g(x)$ sean funciones de soporte compacto (son $0$ fuera de un intervalo $[-M, M]$), de tal forma que para tiempos finitos se cumple que en el instante $t$, $u(x,t)$ es $0$ fuera del intervalo $[-M - ct, M + ct]$ y por tanto los términos de borde en la integración por partes para sacar $E'(t)$ se anulan.

		Como nota, no podríamos pedir sólo que $u_x$ y $u_t$ fuesen integrables $L^2$, más que nada porque eso no implica que se vayan a $0$ en $∞$.

		La conservación de la energía nos dará unicidad de solución, también para el problema no homogéneo; y dependencia continua de los datos. Estos resultados habría que compararlos con el teorema en dominios acotados, que es donde lo demostramos en su día.

		\subsubsection{Aplicación a ecuaciones en dominios acotados}

		Todo esto que hemos hecho para dominios no acotados se aplicar de vuelta a dominios acotados y problemas con reflexiones. Tenemos una cuerda en el intervalo $[0, ∞)$ y nuestro sistema \[ \begin{cases}
		u_{tt} - c^2u_{xx} = 0 & x > 0,\, t ∈ ℝ \\
		u(0,t) = 0 & ∀ t ∈ ℝ \\
		u(x,0) = f(x) & x > 0 \\
		u_t(x,0) = g(x) & x > 0
		\end{cases}\]

		El primer método implicará una extensión de los datos a funciones $\tilde{f}, \tilde{g}$ a las que ya veremos qué valor dar cuando $x ≤ 0$, y sacaremos una solución $\tilde{u}$ por la fórmula de D'Alembert \eqref{eq:Onda:DAlembert}. En particular, querremos que $\tilde{u}(0,t) = 0\,∀t ∈ ℝ$. Sustituyendo en la fórmula, \[ 0 = \tilde{u}(0,t) = \frac{1}{2} \left(f(ct) + f(ct)\right) + \frac{1}{2c} \int_{-ct}^{ct} g(s) \dif s \]

		Para que eso sea 0, necesitamos que $\tilde{f}$ y $\tilde{g}$ sean impares. Así, nuestras extensiones serán \[
		\tilde{f}(x) = \begin{cases}
		f(x) & x > 0 \\
		-f(-x) & x ≤ 0
		\end{cases} \qquad
		\tilde{g}(x) = \begin{cases}
		g(x) & x > 0 \\
		-g(-x) & x ≤ 0
		\end{cases}
		\]

		De ahí sacaríamos la solución $\tilde{u}(x,t)$ y nuestra solución final sería $u(x,t) = \tilde{u}(x,t)$ restringida a $x > 0$.

		Sin embargo, tenemos un segundo método de resolución a través de una interpretación geométrica: gracias al método del paralelogramo podremos sacar el valor de la solución despejando (ver \fref{fig:AplicacionParalelogramo}), sabiendo que \( u(A) + u(C) = u(B) + u(D) \label{eq:Onda:Paralelogramo} \)

		\begin{figure}[hbtp]
		\inputtikz{AplicacionParalelogramo}
		\caption{Para poder obtener los valores cuando sólo sabemos parte de la solución (en este caso, el sombreado azul), podemos usar la regla del paralelogramo, colocando tres puntos en zonas donde sabemos el valor de la solución (uno en el dato inicial, otros dos en la solución calculada) y despejamos para el cuarto punto (en rojo).}
		\label{fig:AplicacionParalelogramo}
		\end{figure}

		Vamos a ver un ejemplo de este método, con la ecuación de siempre y datos iniciales $g \equiv 0$ y \[ f(x) = \begin{cases} 1 & x ∈ [1,2] \\ 0 & x ∈ [0,1) ∪ (2,∞) \end{cases} \]

		Sacando con la regla del paralelogramo los valores tendríamos algo como lo de la \fref{fig:OndaReflexion}, donde la onda rebotaría por la izquierda pero invertida. Además, en la zona donde la onda se anula, la energía se quedaría en la derivada $u_t$, no se nos pierde.

		\begin{figure}[hbtp]
		\centering
		\inputtikz{OndaReflexion}
		\caption{Propagación de la onda con un borde en la izquierda. En las zonas sin sombreado, la onda tiene amplitud nula.}
		\label{fig:OndaReflexion}
		\end{figure}


		Como ejercicios, habría que ver qué ocurre con las ondas en tiempos $t = 2, \frac{3}{4}, \frac{1}{3}$ con $c = 1$. También es interesante ver qué es lo que ocurre cuando $f \equiv 0$ y $g = \ind_{[1,2]}$: el dibujo cambiará de forma esencial.

		El último ejercicio sería ver qué ocurriría con dos bordes. Veríamos los rebotes y las zonas donde las dos ondas que se propagan se juntan.

		\subsection{Caso no homogéneo}
		\label{sec:Onda:NoHomogeneo}

		Nos vata saber qué es lo que ocurre cuando tenemos una fuerza externa, es decir, cuando nuestro sistema es \[ \begin{cases}
		u_{tt} - c^2 u_{xx} = F(x,t) & t ∈ ℝ, \, x ∈ ℝ \\
		u(x,0) = f(x) \\
		u_t(x,0) 0 g(x) \end{cases}\]

		Recuperando lo que habíamos visto en casos anteriores, lo que haremos será separar en dos sistemas \[ \begin{cases}
		v_{tt} - c^2 v_{xx} =0  & t ∈ ℝ, \, x ∈ ℝ \\
		v(x,0) = f(x) \\
		v_t(x,0) = g(x) \end{cases} \qquad \begin{cases}
		w_{tt} - c^2 w_{xx} = F(x,t) & t ∈ ℝ, \, x ∈ ℝ \\
		w(x,0) = 0 \\
		w_t(x,0) = 0) \end{cases}\]

		Resolviendo ambos sistemas, la solución final será la suma de soluciones. Para el primero usaríamos la fórmula de D'Alembert, y para el segundo usaríamos el método de Duhamel. Fijamos $t = τ > 0$, y entonces resolvemos el problema \[ \begin{cases}
		Φ_{tt} - c^2 Φ_{xx} = 0 \\
		Φ(x,0) = 0 \\
		Φ_t(x,0) = F(x,τ) \end{cases}\]

		\begin{wrapfigure}{R}[0.1\textwidth]{0.4\textwidth}
		\centering
		\inputtikz{IntegralOndaNoseque}
		\caption{La integral que hacemos es la del triángulo, con $f$ aportando en los puntos iniciales (verde), $g$ aportando en el intervalo naranja y $F$ en la zona sombreada.}
		\label{fig:Onda:Noseque}
		\end{wrapfigure}

		Así, el impulso $F(x,τ)$ en $t= 0$ sale como \[ Φ(x,t) = \frac{1}{2c} \int_{x-ct}^{x+ct} F(s,τ) \dif τ\], y para $t = τ$ tendríamos \[ Φ(x, t-τ) = \frac{1}{2c} \int_{x-c(t-τ)}^{x + c(t - τ)} F(s, τ) \dif s\]

		De esta forma, la solución para un punto dado es el impulso a lo largo del tiempo, esto es, \[ w(x,t) = \frac{1}{2c} \int_0^t Φ(x,t-τ) \dif τ = \frac{1}{2c} \int_0^t \int_{x-c(t-τ)}^{x + c(t - τ)} F(s, τ) \dif s \]

		Esta integral es en el fondo lo mismo que integrar en el triángulo de la \fref{fig:Onda:Noseque} con las aportaciones que se comentan. Además, la conservación de la energía nos daría la unicidad de la solución para este problema.

		Y ahora, dentro del mundo de cosas que no sé qué son (creo que un ejercicio, llevo mucho tiempo copiando), hay un ejercicio en el que una condición adicional es que $u_x(0,t) = 0$. Habría que hacer las extensiones, calculando $\tilde{u}$ y derivando para sacar las condiciones sobre las extensiones. Ahora bien, también se puede pasar a una condición de contorno tipo Dirichlet buscando la solución $v = u_x$, que nos quedaría el sistema de otra ecuación de onda \[ \begin{cases} v_{tt} - c^2v_{xx} = 0 \\ v(0,t) = 0 \\ v(x,0) = f'(x) \\ v_t(x,0) = g'(x) \end{cases}\] con condiciones de contorno tipo Dirichlet, que podemos resolver por la identidad del paralelogramo de antes.

		BOOOOOM.

		\subsection{Energía}

			Vamos a repasar dos de los problemas de la hoja 3:

			\begin{problem}

				\[\begin{cases}
					u_{tt} - c^2u_{xx} + u_t = 0, x \in (a,b)\\
					u(a,t) = u_x (b,t) = 0\\
					u(x,0) = f(x) \\
					u_t(x,0) = g(x)
				\end{cases}\]

				¿Tenemos unicidad?

				\solution

				Suponemos $u_1,u_2$ soluciones:

				\[ \begin{cases} W = u_1 - u2 \\
				W_{tt} - c^2 W_{xx} + W_t = 0 \\
				W(a,t) = W_x (b,t) = 0\\
				W(x,0) = 0\\
				W_t (x,0) = 0
				\end{cases}\]

				Método de la energía:
				\[ 0 = (W_{tt} - c^2 W_{xx} + W_t) W_t dx\]

				Si integramos seguimos teniendo 0:

				\[ 0 = \int_a^b (W_{tt} - c^2 W_{xx} + W_t) W_t dx\]
				\[ = \int_a^b W_{tt} W_t dx - c^2 \int^b_a W_{xx} W_t dx + \int_{a}^b W^2_t dx  \]

				Tenemos que $W_{tt}W_t \equiv (\frac{1}{2} (W_t)^2)_t $. Además, integrando por partes:

				\[ \int_a^b \underbrace{W_{xx}}_{dv} \underbrace{W_t}_{u} dx = W_x W_t |_a^b - \int_a^b \underbrace{W_x W_{tx}}_{(\frac{W_x^2}{2})_t} dx \]
				Observamos que
				\begin{itemize}
					\item $W_x (b,t) = 0$, $\forall t $
					\item $W(a,t) = 0$, $\forall t \implies W_t (a,t) = 0$
				\end{itemize}
				Luego $W_x W_t |_a^b = 0$

				{\bf Conclusión:}
				\[ E'(t) + \int_a^b W_t^2 dx = 0 \rightarrow \text{ la energía {\bf no} se conserva}\]
				\[ E'(t) = -\int_a^b W_t^2 dx \leq 0 \]

				\[ \left. \begin{array}{r}
					E \text{ decrece} \\
					E \geq 0 \\
					E(0) \eqreason{en w} 0 \end{array} \right\} \Rightarrow E \equiv 0 \Rightarrow W \equiv 0 \Rightarrow u_1 \equiv u_2 \]

			\end{problem}

			\begin{problem}

				\[\begin{cases}
					u_{tt} - c^2 u_{xx} + hu = F(x,t), x \in \real \\
					u, u_x, u_t \convs[x][±\infty] 0, \forall t\\
					\int_{-\infty}^\infty u_t^2 + c^2 u_x^2 + hu^2 dx < \infty (\forall t) \\
					u(x,0) = f(x) \\
					u_t(x,0) = g(x)
				\end{cases}\]

				¿Hay unicidad?

				\solution

				Suponiendo $u1,u_2$ soluciones:

				\[\begin{cases}
					W = u_1 - u_2 \\
					W_{tt} - c^2 W_{xx} + h W = 0
				\end{cases}\]

				Energía:
				\[ 0 = \int_{-\infty}^\infty (w_{tt} - c^2 W_{xx} + h W)W_t dx = … \]

				Integrando por partes en $\int\limits_{-\infty}^\infty W_{xx} W_{t}$, los términos de borde se anulan como en el caso anterior:

				\[ … = \int_{-\infty}^{\infty}  (\frac{W_t^2}{2})_t dx + c^2 \int_{-\infty}^\infty (\frac{W_x^2}{2})_t dx + h \int_{-\infty}^\infty (\frac{W^2}{2})_t dx \]

				Con lo que llegamos (usando $hWW_t = h (\frac{W^2}{2})_t$):

				\[ 0 = (\infty_{-\infty}^\infty  \frac{W_t^2}{2} + c^2 \frac{W_x^2}{2} + h \frac{W^2}{2} dx)_t\]

				(FALTA FINAL)


			\end{problem}


			Entremos un poco más en detalle con la energía y sus reflexiones en base a lo descrito anteriormente. Tomemos:

			\[\begin{cases}
				u_{tt} - u_{xx} = 0, x > 0 \\
				u(0,t) = 0 \forall t \\
				u(x,0) = f(x) \\
				u_t(x,0) = 0
			\end{cases}\]

			(FALTA TERMINAR EL DIBUJO)

			\begin{center}
			\begin{tikzpicture}
				\draw[->] (-2.1, 0) -- (4.2,0) node[right] {$x$};
				\draw[->] (0, -0.1) -- (0,3.2) node[above] {$t$};

				\node[vnlin, label = {below:$1$}] at (1,0) {};
				\node[vnlin, label = {below:$2$}] at (2,0) {};
				\node[vnlin, label = {below:$-1$}] at (-1,0) {};
				\node[vnlin, label = {below:$-2$}] at (-2,0) {};

				\draw[gray] (1,0) -- (4,3) ;
				\draw[gray] (2,0) -- (4,2) ;
				\draw[gray] (2,0) -- (0,2) -- (1,3);
				\draw[gray] (1,0) -- (0,1) -- (2,3);
				\draw[gray] (-1,0) -- (0,1);
				\draw[gray] (-2,0) -- (0,2);
				\draw[gray,dashed] (-2,0) -- (-3,1);
				\draw[gray,dashed] (-1,0) -- (-2,1);

				\fill[pattern = north east lines, pattern color = red] (1, -0.05) rectangle (2, 0.05);
				\fill[pattern = north east lines, pattern color = blue] (-2, -0.05) rectangle (-1, 0.05);
			\end{tikzpicture}
			\end{center}

			\[ u_x = \frac{1}{2} f'(x+t) (x+t)_x + \frac{1}{2} \gor{f}'(x-t)(x-t)_x = \frac{1}{2} \{f' + \gor{f}'\} \eqreason{$t = \frac{3}{2}$} 0 \]

			\[ u_t = \frac{1}{2} f'(x+t) (x+t)_t + \frac{1}{2} \gor{f}' (x-t)(x-t)_t = \frac{1}{2} \{ f'(x+t) - \gor{f} (x-t) \} \eqreason{$t = \frac{3}{2}$} f'  \]

			% No he copiado la explicación especial que dió sobre funciones no derivables TODO

			\obs Esta definición que viene ahora no entra y viene acompañada de un ejemplo que no se ha incluido aquí.

			\begin{defn}[Derivada débil]

				G es la derivada débil de $F$ si y solo si:

				\[ \int_{-\infty}^\infty G\Phi dx = -\int_{-\infty}^\infty F \Phi' dx, \forall \Phi \in C^1, \text{ sop. compacto} \]

				\begin{itemize}
					\item $F$ derivable (clásico) $\Rightarrow F'$ es su derivada débil.
					\item A veces la derivada débil existe aunque $F$ no sea derivable en el sentido clásico.

				\end{itemize}
			\end{defn}

			%Vamos entonces a ver que diferencias habría (en el caso de una cuerda de guitarra, entre darle un golpe a la)

			\[\begin{cases}
				u_{†t} -u_{xx} =0, x \in \real \\
				u(x,0) = f \\
				u_t(x,0) = g
			\end{cases}\]

			\[ u(x,t) = \frac{1}{2} \{ f(x+t) + f(x-t) \} + \frac{1}{2} \int_{x-t}^{x+t} g(s) ds \]

			Contemplamos varios casos:

			\begin{itemize}
				\item $f \neq 0, g = 0$
					(DIBUJO)

				\item $f=0, g \neq 0$
					(DIBUJO)

			\end{itemize}


	% Clase 5-4-2016

	\section{Laplaciano}

	Si nos fijamos en la ecuación del calor, para dimensión espacial $N$ nos queda que \[ u_t - (u_{x_1x_1} + u_{x_2x_2} + \dotsb + u_{x_N x_N})\]

	Para la ecuación de ondas, tenemos algo parecido:  \[ u_{tt} - (u_{x_1x_1} + u_{x_2x_2} + \dotsb + u_{x_N x_N})\]

	Ese último paréntesis es una operación en si misma, que llamaremos el \concept{Laplaciano}: \( Δu = \sum_{i=1}^N u_{x_i x_i} = \tr (\Dif^2 u) = \dv (\grad u) \label{eq:Laplaciano}\)

	Es especialmente interesante verlo como la divergencia del gradiente, para después poder aplicar en un futuro el teorema de Gauss para poder integrar.

	Una forma de ver la ecuación de ondas o del calor es como soluciones estacionarias de algo. Por ejemplo, si nos fijamos en el típico experimento de hacer vibrar una membrana con el sonido, estaríamos ante una ecuación \[ \begin{cases} u_{tt} - u_{xx} - u_{yy} \\ \restr{u}{∂Q} = 0 \\ u(x,0) = f(x) \\ u_t(x,0) = g(x) \end{cases} \] con $Q = (0,1) × (0,1)$.

	Podríamos aplicar entonces el método de separación de variables, buscando dos funciones cuyo producto sea la solución: \[ u(x,y,t) = Φ(x,y) T(t) \] de tal forma que las soluciones serían de la forma \[ \frac{T''}{T} = \frac{ΔΦ}{Φ} = λ \]

	La resolución en $Φ$ nos daría un problema de autovalores: \[ \begin{cases} ΔΦ = λ Φ \\ \restr{Φ}{∂Q} = 0 \end{cases} \], lo cual es un problema de autovalores. El resultado sería una sucesión de autovalores $λ_k$ y autofunciones $Φ_k$.

	Volviendo al caso de la membrana, si ponemos arena en la membrana se quedará con una forma, más concretamente en los nodos de la membrana: los puntos en los que no vibra.

	Nosotros no veremos el problema de resolver el laplaciano para dominios arbitrarios porque necesitamos mucho análisis funcional y en este curso no da tiempo. Nos vamos a dedicar a algo que no tengo claro qué es.

	\subsection{Funciones armónicas, subarmónicas y superarmónicas}

	No sé a qué viene esto.

	\begin{defn}[Función\IS armónica] Si $-Δu = 0$ en Ω, entonces $u$ es armónica en $Ω$. \end{defn}
	\begin{defn}[Función\IS subarmónica] Si $-Δu ≤ 0$ en Ω, entonces $u$ es subarmónica en $Ω$. \end{defn}

	\begin{defn}[Función\IS superarmónica] Si $-Δu ≥ 0$ en Ω, entonces $u$ es superarmónica en $Ω$. \end{defn}

	En dimensión $1$, las funciones armónicas son lineales, las subarmónicas convexas y las superarmónicas convexas.

	En dimensiones superiores, las cosas se complican un poco más. Por ejemplo, una función lineal sigue siendo armónica, pero $u(x,y) = x^2 - y^2$ también es armónica.

	Un ejemplo de función subarmónica es un paraboloide en dimensión $N$: \[ u(x_1, \dotsc, x_N) = \norm{\vx}^2 = x_1^2 + \dotsb + x_N^2 \] es subarmónica ya que $-Δu = -2N$.

	En dimensión $2$, la función \[ F(x,y) = \log (x^2 + y^2)\] tiene laplaciano $-ΔF = 0$ en $ℝ^2 \setminus \set{(0,0)}$, por lo que es armónica.

	En dimensión $N > 2$, la función \[ G(\vx) = \frac{1}{\norm{\vx}^{N-2}}\] también es armónica: $-ΔG = 0$ en $ℝ^N \setminus \set{\vec{0}}$.

	Estos dos ejemplos son llamadas \concept{Soluciones\IS fundamentales}, y ya veremos más tarde por qué se llaman así.

	Las funciones armónicas tienen unas propiedades básicas.

	\begin{prop} \label{prop:PropsFuncArmonicas} Sean $u,v$ funciones armónicas en $Ω$.
	\begin{enumerate}[itemsep = 0pt]
	\item La suma $u +v$ es armónica.
	\item El producto $uv$ en general no es armónico.
	\item La traslación $u(x-x_0)$ es armónica en $\set{x \tq x - x_0 ∈ Ω}$.
	\item El escalado $u(kx)$ es armónica en $\set{x \tq kx ∈ Ω}$.
	\item Dada una transformación lineal $\appl{A}{ℝ^N}{ℝ^N}$ con $\norm{Ax} = \norm{x}\;∀xx ∈ ℝ^N$ (isometría), entonces $u(Ax)$ es armónica en $\set{x \tq Ax ∈ Ω}$.
	\item \concept{Transformada\IS de Kelvin} La función \[ K(\vx) = \frac{1}{\norm{\vx}^{N-2}} u\left(\frac{\vx}{\norm{\vx}^2}\right) \] es armónica en $\set{ \vx \tq \frac{\vx}{\norm{\vx}^2} ∈ Ω}$.
	\end{enumerate}
	\end{prop}

	La transformada de Kelvin es especialmente importante, ya que lleva de puntos de dentro de una bola a puntos fuera, y permite estudiar lo que pasa en el centro de la bola (en el $0$) viendo lo que ocurre en el infinito. La cuenta es un poco infernal, y nosotros lo haremos sólo en un caso.

	\begin{proof}[Transformada de Kelvin - $u$ radial]
	Suponemos que $u$ es radial, esto es, que $u$ sólo depende de la norma del vector $\vx$. Diremos entonces que $u \equiv u(r)$ con $r = \norm{\vx}$.

	Vamos a estudiar la ``armonicidad'' de la función. Primero calculamos la derivada $r_i$: \begin{align*}
	r^2 &= x_1^2 + \dotsb + x_N^2 \\
	2rr_{x_i} &= 2x_i \\
	r_{x_i} &= \frac{x_i}{r}
	\end{align*}

	Con esto ya podemos calcular la segunda derivada de $u$: \begin{align*}
	u_{x_i} &= u_{r} r_{x_i} = u_{r} \frac{x_i}{r} \\
	u_{x_ix_i} &= \dotsb = \left(\frac{ur}{r}\right)_r \frac{x_i^2}{r} + \frac{u_r}{r}
	\end{align*}

	Así, el laplaciano queda que es igual a \[ Δu = \sum_{i=1}^N = \left(\frac{ur}{r}\right)_r · r + N\frac{u_r}{r} = \dotsb = u_{rr} + \frac{N-1}{r} u_r \]

	El ejercicio a cargo del lector consistiría en ver que dada la transformada de Kelvin \[ H(r) = \frac{1}{r^{N-2}} u\left(\frac{1}{r}\right)\] entonces \[ H_{rr} + \frac{N-1}{r} H_r = 0\]

	\end{proof}

	¿Cómo se escribiría la transformada de Laplace si dependiese también del ángulo? En ese caso, cada punto $\vx ∈ ℝ^N$ se escribiría como $\vx = r · ω$, con $r ∈ [0, ∞)$ y $ω ∈ \crc[N-1]$, de tal forma que el laplaciano sería \[ Δu(\vx) = u_{rr} + \frac{N-1}{r} u_r + \frac{1}{r^2} Δ_{\crc[N-1]} u \] , donde $Δ_{\crc[N-1]}$ sería el \concept{Operador\IS de Laplace-Beltrami} que es la restricción del laplaciano a la esfera. Esta sería la conexión de esto con la geometría y no vamos a ver nada más de esto.

	\subsection{Propiedades fundamentales: Principios del máximo y de la media}

	El principio del máximo nos dice lo siguiente: si tenemos una cuerda en equilibrio (una función armónica), no tenemos ni máximos ni mínimos fuera de la frontera. Si, por ejemplo, tenemos ondas en una membrana, entonces no está en equilibrio: está vibrando.

	Podemos mirarlo desde otro lado, tomando una función superarmónica con $-Δu = F ≥ 0$. Es decir, que hay un equilibrio en presencia de una fuerza externa $F$. Esta fuerza será positiva si actúa hacia arriba, y negativa si lo hace hacia abajo. El principio del máximo nos dirá que, respectivamente, sólo será posible una onda hacia arriba $u > 0$ o hacia abajo $u < 0$. (dibujo aquí.)

	\begin{prop}[Principio\IS del máximo débil] \label{prop:PrincipioMaximoDebil} Sea $Ω$ un dominio\footnote{Conjunto abierto y conexo} acotado en $ℝ^N$, y sea $u$ una función subarmónica en Ω y $u ∈ C^2(Ω) ∩ C(\adh{Ω})$. Entonces el máximo en el cierre es igual al máximo en la frontera: \[ \max_{\adh{Ω}} u = \max_{∂Ω} u \]
	\end{prop}

	El principio del máximo débil sí que nos permite casos en los que el máximo se alcanza en la frontera y también en el interior. Con artillería más fuerte veremos que estos casos, en la práctica, no son posibles; aunque para ello tendremos que esperar un poco.

	\begin{proof}

	\proofpart{$-Δu < 0$ en Ω (desigualdad estricta)}

	Haremos la demostración por reducción al absurdo: supongamos que $\max_{\adh{Ω}}$ se alcanza en $x_0 ∈ \mop{int} Ω$. En ese punto de máximo tiene que pasar dos cosas: que el gradiente se anula ($\grad u = \vec{0}$) y que el hessiano $\Dif^2 u$ tiene que ser semidefinida negativa. Esto es, los autovalores de $\Dif^2 u$ tienen que ser menores o iguales que $0$.

	En ese caso, la traza ha de ser menor o igual que cero, pero $\tr \Dif^2 u(x_0) = Δu(x_0) ≥ 0$, contradicción porque habíamos dicho que $-Δ u < 0$.

	\proofpart{$-Δu ≤ 0$ en Ω (desigualdad no estricta)}

	En este caso miraremos una función modificada $v_ε \approx u$, tal que $-Δv_ε < 0$. Tomaremos entonces \[ v_ε(\vx) = u(\vx) + ε\norm{\vx}^2\], de manera que \[ -Δv_ε (\vx) = -Δu(\vx) - ε Δ(\norm{\vx}^2) = -Δu(\vx)  - 2εN < 0 \]

	Aplicamos entonces el caso anterior a $v_ε$, y entonces hacemos tender $ε \to 0$. Esto queda como ejercicio para el lector. En el paso al límite, tendremos que aceptar la desigualdad no estricta, lo que nos da la debilidad de este principio.
	\end{proof}

	Aunque este principio lo hemos hecho sólo para funciones subarmónicas, para superarmónicas es sólo cambiar el signo. Para funciones armónicas, como son sub y super armónicas, tendremos que el mínimo y máximo se alcanzan en la frontera.

	Este principio será muy potente: nos permitirá demostrar unicidad, estimaciones, y más cosas que no he llegado a copiar.


	% Clase 6-4-2016
	\subsubsection{Aplicaciones}

	Vamos a ver cuatro tipos de aplicaciones: comparación, unicidad, estimación a priori y dependencia contínua.

	\begin{example}[1 - Comparación]
		Supongamos que tenemos una función superarmónica que es positiva en el borde:

		\[\begin{array}{r}
		-\Delta u ≥ 0 \text{ en } \Omega \\
		u|_{\delta \Omega} ≥ 0
		\end{array} \Rightarrow u ≥ 0 en \bar{\Omega} \]

		Entonces aplicamos el principio de comparación débil:

		\[\begin{array}{r}
		-\Delta u ≥ -\Delta v \text{ en } \Omega \\
		u|_{\delta \Omega} ≥ v|_{\delta \Omega}
		\end{array} \Rightarrow u \eqreason[≥]{Comparación débil} v en \bar{\Omega} \]

		Más tarde veremos un principio de comparación fuerte.
	\end{example}

	\begin{example}[2 - Unicidad]
		\[ \begin{cases}
			-\Delta u = F \text{ en } \Omega \\
			u|_{\delta \Omega} = g \leftarrow \text{ Dirichlet}
			\end{cases}
		\]

		Imaginamos $u_1$ y $u_2$ soluciones:

		\[\begin{array}{r}
		-\Delta u_1 = -\Delta u_2 \text{ en } \Omega \\
		u_1|_{\delta \Omega} = u_2|_{\delta \Omega}
		\end{array} \Rightarrow u_1 \equiv u_2 en \bar{\Omega} \]

		Tenemos una condición de unicidad pero que pasa con:

		\[ -\Delta u = F, \left. \frac{\delta u}{\delta m} \right|_{\delta \Omega} = g\]

		Es decir, condiciones Neumman.

	\end{example}

	\begin{example}[3 - estimación a priori]


		\[\begin{cases}
		-\Delta u  = F \text{ en } \Omega \text{ acotado } (\Omega \in B_R(0)) \\
		u|_{\delta \Omega} = g
		\end{cases}\]

		\[ m \leq F \leq M, \quad \gor{m} \leq g \leq \gor{M} \]

		Usamos la función auxiliar:

		\[ \Phi(\vx) = A - \underbrace{B}_{\frac{M}{2N}} \norm{\vx}^2 \]

		\[ - \Delta \Phi = … = 2NB \eqreason{$B = \frac{M}{2N}$} M \geq F = -\Delta u  \]

		Elegimos $A$ para tener $\Phi|_{\delta \Omega} \geq u |_{\delta\Omega}$

		\[ \Phi(x) = A - \frac{M}{2N} \norm{\vx}^2 \geq A - \frac{M}{2N} R^2 \eqreason{$A = \gor{M} + \frac{M}{2N} R^2 $} \gor{M} \geq g = u|_{\delta \Omega} \]

		\textbf{Conclusión}

		\[ \Phi(\vx) = \gor{M} + \frac{M}{2N} R^2 - \frac{M}{2N} \norm{\vx}^2 \]
		\[ \left\{ \begin{array}{l}
		-\Delta \Phi \geq - \Delta u \text{ en } \Omega \\
		\Phi |_{\delta \Omega} \geq u|_{\delta \Omega}

		\end{array} \right\} \Rightarrow \Phi \geq u \]

		\[ u(\vx) \leq \underbrace{\gor{M}}_{g} + \frac{M}{2N} (R^2 - \norm{\vx}^2), \forall \vx \in \Omega \]

	\end{example}

	\begin{example}[4 - dependencia contínua]

		(FALTA)

	\end{example}

	\obs Si tomamos la ecuación:

	\[ \begin{cases}
		-\Delta u = 0 \text{ en } \Omega \\
		u|_{\delta \Omega} = g
	\end{cases}\]

	Hay que interpretarlo como una solución estacionaria de una ecuación del calor.

	(DIBUJOS ESPINA DE LEBESGUE)

	\subsubsection{Consecuencias}

		\subsubsubsection{Propiedad de la medida}

		Caso $N = 2$:

		\[ B_R (0,0) \subset \Omega \]

		\[\begin{cases}
			-\Delta u = 0 \text{ en } B_R
			u|_{\delta B_R} = u
		\end{cases}\]

		Por el principio del máximo obtenemos unicidad:

		Con $R = 1$: \[ u(r, \theta) = \int_{0}^{2\pi} u(1,s) P_r(\theta-s) \dif{s} = \frac{1}{2\pi} \int_0^{2\pi} u(1,s) \frac{1-r^2}{1+r^2 - 2r\cos(\theta-s)} \dif{s} \]

		Con $R$ genérico: $\frac{r}{R}$:

		\[ \text{sol. en }B_R \equiv u(r, \theta) = \frac{1}{2\pi} \int_0^{2\pi} u(R,s) \frac{R^2-r^2}{R^2+r^2-2rR \cos(\theta-s)} \]

		\[  \underbrace{u (x = 0, y=0)}_{\text{CAR}} = \underbrace{u(r, \theta)}_{\text{POL}} |_{r=0} = \]

		\[ \frac{1}{2\pi} \int_0^{2\pi} \underbrace{u(R,s)}_{\text{POL}} ds =  \frac{1}{2\pi} \int_0^{2\pi} \underbrace{u(x=R\cos(s),y=R\sin(s))}_{\text{CAR}} ds\]

		(DIBUJO)

		Realizamos una traslación para obetner un resultado válido para cualquier $(a,b) \in \Omega$, y cualquiera $R$ tal que $B_R(a,b) \subset \Omega$.

		\[ u(a,b) = \frac{1}{2\pi R} \int_0^{2\pi} u(a+R\cos s, b R \sin s) R ds \]

		\[ C_R \equiv \{ (x-a)^2 + (y-b)^2 = R^2\}\]
		\[ \sigma_R (s) = (a + R \cos s , b + R \sin s), s \in [0,2\pi]\]
		\[ \sigma'_R (s) = (-R \sin s, R \cos s)\]
		\[ \norm{\sigma'_R} = R\]

		Obtenemos la propiedad de la media sobre circunferencias:

		\[ u(a,b) = \frac{1}{2\pi} \int_0^{2\pi} u(\sigma_R (s)) \cdot \norm{\sigma'} \]

		(FALTA Y LA FÓRMULA ANTERIOR ESTÁ INCOMPLETA)

		\obs La propiedad de la media sobre circunferencias $C_R$ $\Rightarrow$ Propiedad de la media sobre círculos $B_R$

		\begin{proof}
		\[ B_R (a,b) \subset \Omega\text{ , en particular: } B_{\rho}(a,b) \subset \Omega, \rho \in [0,R] \]

		\[ u(a,b) = \frac{1}{|C_\rho|} \int_{C_\rho}  u( \sigma_{\rho}(s) ) d\sigma_{\rho} = … = \frac{1}{2\pi\rho} \int_0^{2\pi} u(a + \rho\cos s, b+ \rho \sin s) \rho ds \]

		\[  \int_{0}^{R} \rho u (a,b) d \rho = \frac{1}{2\pi} \int_0^R \int_0^{2\pi} u(a + \rho \cos s , b + \rho \sin s) \rho ds d\rho \]

		\[ \frac{R^2}{2} u(a,b) = \frac{1}{2\pi}  \int\int_{B_R} u(x,y) dx dy \]

		\[ u(a,b) = \frac{1}{\pi R^2} \int\int_{B_R} u(x,y) dx dy = \frac{1}{|B_R|} \int\int_{B_R} u(x,y)dx dy \]

		\end{proof}

		% Clase 11-4-2016

		Investiguemos un poco más que pasa en $N > 2$. Realizaremos una prueba apoyada enlas identidades de Green.

		Empezamos con el \concept{Teorema\IS de Gauss}:

		\[ \int_{\delta \Omega} \gor{F} dx = \int_{\Omega} \dv \gor{F}dx = \int_{\delta \Omega} \pesc{\gor{F},\eqreason[\gor{n}]{normal exterior unitaria}} d\sigma \]

		\[ \Delta u = \dv (\Delta u)\]

		\begin{itemize}
			\item $\int\limits_{\Omega} \Delta u dx = \int\limits_{\delta\Omega} \pesc{\grad u, \gor{n}} dx = \int\limits_{\delta\Omega} \eqreason[\frac{\delta u}{\delta \gor{u}}]{Derivada de u en la dirección normal exterior} dx $

			\obs
			\begin{itemize}
				\item \[\left\{\begin{array}{l}
					-\Delta u = G \text{ en } \Omega \\
					\frac{\delta u}{\delta u}|_{\delta\Omega} = g

				\end{array} \right| \rightarrow - \int_{\Omega} G dx = \int_{\delta \Omega} g d \sigma (\text{Condición de compatibilidad entre datos}) \]

				\item $u$ armónica en $\Omega \Rightarrow \int_{\delta\omega} $ (falta)
			\end{itemize}

			\item \[\gor{F} = v \grad u = (v u_{x_1}, vu_{x_2}, … , vu_{x_n}) \]
			\[ \dv \gor{F} = … = \pesc{\grad v, \grad u} + v \Delta u \]

			\[ \int_{\Omega} \dv \gor{F} dx = \int_{\delta\Omega} \pesc{\gor{F},\gor{n}} d \sigma \]

			\[ \int_{\Omega} \pesc{\grad v, \grad u} dx + \int_{\Omega} \text{FALTA}\]

			\[ \int_{\Omega} v \Delta u dx = \int_{\delta \Omega} v \frac{\delta u}{\delta n} d \sigma - \int_{\Omega} \pesc{\grad v, \grad u} dx (\text{ Integración por partes}) \]

			\obs Supongamos $- \dif u = G$, tomamos $v=u$.

			\[ -\int_{\Omega} u G dx = \int_{\delta \Omega} u \frac{\delta u}{\delta n} d \sigma - \int_{\omega} \|\grad u\|^2 d \sigma \]

			\item Identidad de Greeen

				\[ \int_{\Omega} v \Delta u - u \delta v dx = \int_{\delta \Omega} v \frac{\delta u}{\delta n} - u \frac{\delta v}{\delta n} d \sigma  \]

				\textbf{Objetivo:} Probar la propiedad de la media (al menos para $N > 2$).

				Supongamos $-\Delta u = 0$ en $\Omega$

				Tomamos $v(\bar{x}) = \frac{1}{\|\bar{x} \|^{N-2}} $ (si $N = 2$, $v(\bar{x}) = \log(\|\bar{x}\|)$).

				\[ v(\bar{x}) = (x_1^2 + x_2^2 + … + x_N^2)^{\frac{2-N}{2}}  \]
				\[ V_{x_i} = \frac{2-N}{2} (x_1^2 + x_2^2 + … + x_N^2)^{\frac{2-N}{2} -1} 2x_i = (2-N) \|\bar{x}\|^{-N} x_i \]
				\[ \Delta v = (2-N) \frac{\bar{x}}{\|\bar{x}\|^N}\]
				\[-\Delta V = … = 0, \text{ en } \real^n-\{0\}\]

				Podríamos intentar resolver esto aplicando Gauss, pero no debemos ya que estamos en un domínio donde $v$, $\Delta v$ no son contínuas.

				\begin{proof}

					Vamos a centrarnos en un anillo de radio interior $\epsilon$ y radio exterior $R$. Por lo que aplicaremos Green en $B_{R}- B_{\epsilon}$.

					\[ \underbrace{ \int\limits_{B_R-B_\epsilon} v \Delta u - u \Delta v dx}_{= 0 (-\Delta u = -\Delta v = 0)} = \int\limits_{\delta(B_R-B_\epsilon)} v \frac{\delta u}{\delta u} - \frac{\delta v}{\delta u} d \sigma = (*) \]

					Ahora tenemos que ver cual es la frontera. En este caso se compone de dos anillos, el exterior y el interior. Y además tenemos que tener en cuenta cuales con los vectores normales exteriores. Para el disco exterior serán hacia radios más grandes, pero en el disco interior serán hacia radios más pequeños.

					Entonces tenemos que:

					\[(*) = \int_{\delta B_R} - \int_{\delta B_\epsilon} \Rightarrow  \int_{\delta B_R} = \int_{\delta B_\epsilon}\]

					\[ int_{\delta B_R} \underbrace{v}_{\frac{1}{R^{N-2}}} \frac{\delta u}{\delta u} - u \underbrace{\frac{\delta v}{\delta u}}_{\frac{(2-N)}{R^{N-1}}} d \sigma_R  = \frac{1}{R^{N-2}} \int_{\delta B_R} \frac{\delta u}{\delta u} d\sigma + \frac{N-2}{R^{N-1}} \int_{\delta B_R} B u d\sigma_R  \]

					(FALTA)

					\textbf{Conclusión}

					\[\frac{1}{R^{N-1}} \int_{\sigma B_R} u d\sigma_R = \frac{1}{\epsilon^{N-1}} \int_{\delta B_\epsilon} u d \sigma_\epsilon \]

					(FALTA UN MONTÓN QUE LLEGA A:)

					\[ m_\epsilon \leq \frac{1}{|\delta B_{\epsilon}} \int_{\delta B_\epsilon} u d \sigma_\epsilon \leq M_\epsilon  \]

					Por continuidad: \[M_\epsilon. m_\epsilon \convs[\epsilon][0] u(0) \]

				\end{proof}

				\textbf{Conclusión}

					El valor de la función armónica en el origen es:

					\[ u(0) = \frac{1}{|B_R|} \int_{\delta B_R} u(y) d \sigma_R \]

					\obs

					\[
						u \text{ armónica} \Rightarrow \begin{cases}
							-\Delta u = 0 \\
							\int_{\delta B} \frac{\delta u}{\delta u} d \sigma = \int_{B} \Delta u dx = 0
						\end{cases}
					\]

					\[
						u \text{ superarmónica} \Rightarrow \begin{cases}
							-\Delta u \geq 0 \\
							\int_{\delta B} \frac{\delta u}{\delta u} d \sigma = \int_{B} \Delta u dx \leq 0
						\end{cases}
					\]

					\[
						u \text{ subarmónica} \Rightarrow \begin{cases}
							-\Delta u \leq 0 \\
							\int_{\delta B} \frac{\delta u}{\delta u} d \sigma = \int_{B} \Delta u dx \geq 0
						\end{cases}
					\]

		\end{itemize}

		\begin{defn}[Desigualdad\IS para funciones sub/superarmónicas]

			\begin{itemize}
				\item $\Delta u \leq 0 \Rightarrow u(0) \leq \frac{1}{|\delta B_R|} \int_{\delta B_R} u d \sigma_R $
				\item $- \Delta u \geq 0 \Rightarrow u(0) \geq \frac{1}{|\delta B_R|} \int_{\delta B_R} u d \sigma_R $
			\end{itemize}

		\end{defn}


		% Clase 12-4-16

		\textbf{Resumen}

		Hemos visto una aplicación de la integración por partes con la identidad de Green: La propiedad de la media (en dimensión arbitraria): $-\Delta u=0$ en $\Omega \supset B_R(\bar{0})$: \[u(\bar{0}) = \frac{1}{|\delta B_R|} \int\limits_{\delta B_R} u(y)d\sigma_R \]

		Extensiones:
		\begin{itemize}
			\item Válido para cualquier $B_R(x_0) \subset \Omega$:
			\[ u(\bar{x_0}) = \frac{1}{|\delta B_R(\bar{x_0})|} \int\limits_{\delta B_R(\bar{x_0})} u(y)d\sigma_R \]

			\item Propiedad de la media para esferas $\Rightarrow$ propiedad de la media para bolas:
			\[ u(\bar{x_0}) = \frac{1}{|B_R(\bar{x_0})|} \int\limits_{B_R(\bar{x_0})} u(y)dy \]

			\item Válido (con desigualdad) para funciones subarmónicas o superarmónicas:
			\begin{itemize}
				\item $-\Delta u \geq 0 \rightarrow u(\bar{x_0}) \geq \frac{1}{|\delta B_R|} \int\limits_{\delta B_R(\bar{x_0})} u(y)d\sigma_R > \frac{1}{|B_R(\bar{x_0})|} \int\limits_{B_R(\bar{x_0})} u(y)dy$

				(FALTA)
			\end{itemize}
		\end{itemize}

		La segunda aplicación sería la Unicidad:

		\obs Hasta ahora tenemosunicidad para condiciones de Dirichlet, por el principio del máximo (débil).

		Veamos los casos dependiendo del contorno:

		\begin{itemize}
			\item Dirichlet:

				\[ \begin{cases}
					-\Delta u = F \text{ en } \Omega \\
					u |_{\delta \Omega} = g \\
					u_1,u_2 \text{ soluciones} \\
					W = u_1 - u_2
				\end{cases} \]

				\[ 0 = - \int_{\Omega} w \Delta w dx = - \left\{  \int_{\delta \Omega} \eqreason[w]{= 0} \frac{\delta w}{\delta u} d\sigma = \int_{\omega}|\grad w|^2 dx \right\}  \]

				(FALTA)

			\item Newman:

				\[ \left. \begin{array}{r}
					-\Delta u = F \text{ en } \Omega \\
					\frac{\delta u}{\delta u}|_{\delta \Omega} = g
					\end{array} \right\} … \left\{ \begin{array}{l}
						- \Delta w = 0 \\
						\frac{\delta w}{\delta u}|_{\delta \Omega} = 0
					\end{array}  \right.  \]

					\[ … \int_{\Omega} |\grad w|^2 dx = 0 \Rightarrow \grad w = 0 \Rightarrow W = \text{cte.}  \]

					Tenemos unicidad salvo constantes

			\item Mixtas:

				(No me da tiempo, salto a lo siguiente)

		\end{itemize}


		\begin{prop}[Principio\IS del máximo fuerte]

			\[ - \Delta u \geq 0 \text{ en } \Omega \]
			\[ u (\bar{x}) \geq \frac{1}{|\delta B_R(\bar{x})|} \int\limits_{\delta B_R (\bar{x})} u(\bar{y}) d \sigma_R \quad \forall x \in \Omega, \forall R : B_R(\bar{x}) \subset \Omega \]

		\end{prop}

		\begin{proof}

			Supongamos que sabemos que $u$ es continua y cumple la propiedad anterior. Intentemos probar que el mínimo se alcanza en la frontera. Intentemos demostrarlo por derucción al absurdo: Supongamos que existe $\bar{x_m} \in \text{int} \Omega$ tal que $u(\bar{x_m}) \leq (\bar{x_m}) \leq u(\bar{x}), \forall \bar{x} \in \bar{\omega}$

			Tomamos un $R$ tal que $B_R (\bar{x_m}) \subset \Omega$:

			\[ u(x_m) \geq \frac{1}{|\delta B_R|}  \int\limits_{\delta B_R} u(\bar{y}) d \sigma_R \]

			Si $u(y) \eqreason[>]{estricta} u(x_m) \text{ en } \rho \subset \delta B_R (x_m)$, entonces tendría $\frac{1}{|\delta B_R|} \int\limits_{\delta B_R} u(y) d \sigma_R > u(x_m) $ (u es contínua).

			Por tanto $u \equiv u(x_m)$ en todos los puntos de $\delta B_R(x_m)$, (para todo $R$ tal que $B_R \subset \Omega$). Es decir, $u \equiv u(x_m)$ en toda la bola $B_R$.

			Supongamos $\Omega$ conexo. Podemos conectar $x_m$ con $x_*$ mediante una curva contenida en $\Omega$

			(DIBUJAZO)

			Repetimos el argumento para $x_1$ y entonctramos otra bola en la que $u \equiv u(x_m)$. En un número finito de pasos podemos llegar hasta $x_*$. Como $\Omega$ está acotado es compacto, con conceptos de topología podemos asegurar que los radios de las circunferencias tienen un mínimo (distancia a la frontera) y nos aseguramos de que llegaremos a $x_*$.

			Por lo tanto $u$ es constante en $\Omega$.

		\end{proof}

		\begin{theorem}

			Dado $\Omega$ dominio abierto y conexo.

			\begin{itemize}
				\item $-\Delta u \geq 0$ en $\Omega$ entonces:
				\begin{itemize}
					\item $\min_{\bar{\Omega}} u = \min_{\delta \Omega} u$
					\item Además si $u(x_m) = \min_{\bar{\Omega}} u$ para algún $x_m \in \text{int} \Omega$ entonces $u \equiv$ cte.
				\end{itemize}

				\item $-\Delta u \leq 0$ en $\Omega$:
				\begin{itemize}
					\item $\max_{\bar{\Omega}} u = \max_{\delta \Omega} u$
					\item Además si $u(x_m) = \max_{\bar{\Omega}} u$ para algún $x_m \in \text{int} \Omega$ entonces $u \equiv$ cte.
				\end{itemize}

				\item $-\Delta u = 0$: Entonces se cumplen las dos condiciones anteriores.

			\end{itemize}

		\end{theorem}

		\begin{example}[Dominio no acotado]

			\[\begin{array}{l}
				-\Delta u = 0 \text{ en } \|x\| > 1 \\
				\lim_{\|x\| \to \infty} u(x) = 0 \\
				u|_{\|x\| = 1} = f(x), \text{ con } 1 < f(x) < 2
			\end{array}\]

			Nos hace falta saber que le pasa a la función en el fininito. El máximo será el máximo de $R=1$ y el mínimo cuando $R$ tiende a 2.

		\end{example}


		\begin{theorem}
			\[ \left. \begin{array}{l}
				u(\bar{x}) = \frac{1}{|B_R|} \int\limits_{B_R(\bar{x})} d\bar{y} \forall B_R(\bar{x}) \subset Omega \\
				u \text{ contínua }
			\end{array} \right\} \Rightarrow u \text{ armónica } \]

			(Se podría hacer también con el contorno pero sería equivalente)

		\end{theorem}

		\begin{proof}

			¿Como es posible que esa identidad indique que $u \in C^2$. Veámoslo:

			Sea $B_R \subset \Omega$. Consideramos:

			\[ \left. \begin{array}{l}
				- \Delta \Phi = 0 \text{ en } B_R \\
				\Phi |_{\delta B_R = u}
			\end{array} \right\} \Rightarrow \begin{cases}
				\text{ Sabemos resolver en } n = 2 \text{(Poisson)}\\
				\text{Asumimos que tiene solución para cualquier} N
			\end{cases}
			\]

			Queremos probar que $u \equiv \Phi$ en $B_R$, ya que $\Phi$ es armónica. Como es armónica cumple la propiedad de la media. $u$ también cumple la propiedad de la media, por lo que $\Phi - u$ también. Esto es suficiente para cumplir el principio del máximo y como $\Phi -u = 0$ en la frontera, $\Phi = u$.

		\end{proof}

		\textbf{Otros resultados}

		\begin{itemize}
			\item $u$ armónica $\Rightarrow u$ analítica
			\item Los ceros de una función armónica no pueden ser puntos aislados.
			\item Si $u$ es armónica, $u \equiv 0$ en $B_R$. Entonces $u \equiv 0$.
		\end{itemize}



		% Clase 18-4-2016

		Recordamos el teorema de Liouville:

		\begin{theorem}
			\[ \left. \begin{array}{r}
				-\Delta u = 0 \text{ en } \real^n \text{ con u acotada.} \\
				(\exists M \in \real \text{ tal que } - M \leq u(x) \leq M)
			\end{array} \right\} \Rightarrow u \equiv \text{ cte.} \]
		\end{theorem}

		\begin{theorem}[Teorema\IS de Harnack]
			\[ \left. \begin{array}{r}
				-\Delta u = 0  \\
				u \geq 0
			\end{array} \right\} \text{ en } \Omega\]

			\[ B_R(\bar{x_0}) \subset \Omega\]
			\[ \exists C_1(R), C_2(R) \text{ tales que:} \]
			\begin{itemize}
				\item $C_1(R) u(\bar{x_0}) \leq u(\bar{x} \leq C_2 (R) u(\bar{x_0}))$
				FALTA
			\end{itemize}

		\end{theorem}

		\begin{lemma}[Lemma\IS de Hopf]
			\[ \left. \begin{array}{l} \left. \begin{array}{r}
				-\Delta u = 0  \\
				u \geq 0
			\end{array} \right\} \text{ en } \Omega \\
			u(\bar{x_0}) = 0 \text{ para algún } x_0
			\end{array} \right\} \Rightarrow \text{ ALGO }
			\]
		\end{lemma}


		Antes de terminar de hablar de la ecuación de Laplace, volvamos a la integral de Dirichlet:
		\[ u_{tt} - u_{xx} = 0 \rightarrow \text{ Energía: } \frac{1}{2} \int_a^b u_t^2 + u_x^2 dx \]

		En dimensión $n$:
		\[ u_{tt}- \Delta u = 0 \rightarrow E \equiv \frac{1}{2}\int\limits_\Omega u^2_t | \nabla_x u | \]

		(FALTA OTRA VEZ)

		\[ - \Delta u = 0 \rightarrow \text{ energía potencial }\quad \int\limits_\Omega | \nabla u|^2 d\bar{x} \]

		Idea: La solución es la de energía mínima:

		\[ \begin{cases}
			-\Delta u = 0 \text{ en } \Omega \\
			u|_{\delta\Omega} = g
		\end{cases}\]
		\[ A = \{ \omega \in C^2(\Omega) / w |_{\delta \Omega} = g  \} \text{ (funciones admisibles)}\]
		$u$ solución $\Rightarrow$ $u \in A$

		Supongamos $u$ solución:
		Sea $v \in A$:

		\[ 0 = \int\limits_\Omega (-\Delta u)(u-v) d\bar{x} = - \int\limits_{\delta\Omega} \frac{\delta u}{\delta m} (u \eqreason[-]{$g-g=0$}v) + \int\limits_\Omega \pesc{\nabla u, \nabla (u-v)} d\bar{x} \]

		\[ 0 = \int_\Omega \|\nabla u\|^2 d\bar{x} - \int_\Omega \pesc{\nabla u, \nabla v} dx \leq \left( \int\Omega |\nabla u|^2 dx \right)^{1/2} \left( \int_\Omega |\nabla v|^2 \right)^1/2 \]

		(FALTA. Usa Cauchy Swartz aquí)

		\textbf{Conclusión:}

		\[ \int_\Omega | \nabla u|^2 dx \leq \left( \int_\Omega |\nabla u|^2 dx \right)^{1/2} \left( \int_\Omega |\nabla v|^2 dx \right)^{1/2} \]

		Es decir:

		\[ \left( \int_\Omega |\nabla u|^2 dx \right) \leq \left( \int_\Omega |\nabla v|^2 dx \right) \quad \forall v \in A \]

		Solución $\Rightarrow$ Energía mínima

		Para demostrar la implicación en el otro sentido:

		\[ E(\Phi) = \int_\Omega |\nabla \Phi|^2 dx \]

		\[ W \in A, \text{ tal que } E(w) \leq E(\Phi), \forall \Phi \in A\]
		\[ W \in A \Rightarrow W|_{\delta \Omega} = g\]

		Queremos probar que $-\Delta W = 0$ en $\Omega$.

		Sea $w+t\phi \in A, t \in \real$:
		\[\begin{cases}
			\phi \in C^2(\Omega)\\
			\Phi|_{\delta \Omega} = 0
	\end{cases}\]

		\[ E(w) \leq E(w + t \phi), \forall t\]

		Fijo $\phi$, definimos $g(t) = E(w + t \phi)$, como el mínimo en $t=0$.

		\[E= \int_\Omega | \nabla (w+t\phi)|^2 dx = \int\pesc{\nabla (w+t\phi), \nabla(w + t \phi)} dx\]
		\[ = … = \int_\Omega |\nabla w|^2 dx + t^2 \int_\Omega |\nabla \phi|^2 dx + 2t\int_\Omega \pesc{\nabla w, \nabla \phi} dx \]

		(FALTA UN MONTÓN)

		\textbf{Idea:} Minimizar $E$ en el conjunto $A$:

		\[ \begin{array}{l}
			E(\Phi) \geq 0, \forall \Phi \in A \\
			\{ E(\Phi); \Phi \in A\} \subset [0, \infty)\\
			\exists \alpha = \text{inf}\{E(\Phi) : \Phi \in A\}\\
			\alpha = \text{inf} \Rightarrow \alpha + \frac{1}{n} \text{ no cota inf.}\\
			\exists W_n \in A \text{ tal que } \alpha \leq E(W_n) < \alpha + \frac{1}{n} \\
			\text{ Encontramos } \{W_n\} \subset A\\
			E(W_n) \rightarrow \infty
		\end{array} \]

		\[ \text{¿} W_n \to W_0 \in A \text{?}\]
		\[ \text{¿} E(W_0) = \alpha \text{?}\]

		Entramos en el mundo de la convergencia de funciones. Veremos que $C^2$ no es suficiente.



		\subsubsection{Probabilidad}

		\[P(A \cup B) = 1 \quad \quad P(A \cap B) = 0\]
		\[P(C) = P(C|A) P(A) + P(C|B) P(B)\]

		(DIBUJO CAMPO PROBABILIDADES)

		$u(x) \equiv$ probabilidad de encontrar una puerta partiendo de $x$.

		(DIBUJO MOVIMIENTOS PROBABILIDAD)

		Por lo que $u(x,y) = u(x+h,y) \frac{1}{4} + u(x-h,y) \frac{1}{4} + u(x,y+h)\frac{1}{4} + u(x,t-h)\frac{1}{4} $
		\[ 0 = u(x+h,y) + u(x-h,y)+ u(x,y+h) + u(x,y-h) - 4u(x,y) \]
		\[ 0 = \underbrace{u(x+h,y) + u(x-h,y) -2u(x,y)}_{h^2} + \underbrace{u(x,y+h) + u(x,y-h) - 2u(x,y)}_{h^2}\]

		Cuando $h \to 0$:

		\[ 0 = \frac{\delta^2 u}{\delta x^2} + \frac{\delta^2 u}{\delta y^2} = \Delta u(x,y)\]

		De otra manera tenemos:

		\[ u(\bar{x}) = \frac{1}{|S_\epsilon|}\int_{S_\epsilon} u(x_\epsilon) d \sigma_{\epsilon}\]

		Lo cual es la propiedad de la media, que está relacionada con la probabilidad condicionada.

		\textbf{Tiempo de parada}

		\[ T(x,y) \equiv \text{ tiempo de parada partiendo de } x\]
		\[ T(x,y) = \frac{1}{4} \{ T(x+h,y) + T(x-h,y) + T(x,y+h) + T(x,y-h) \} + \tau(x,y,h)\]

		\[ … 0 = \frac{\delta^2\tau}{\delta x^2} + \frac{\delta^2 \tau}{\delta y^2} + \lim_{h \to 0} \frac{\tau(x,y,h)}{4h^2} \eqreason{Si $\tau = F(x,y) h^2$} \Delta T + \frac{F}{4} \]

		\[ \begin{cases}
			-\Delta T = F \\
			T |_{\delta \Omega} = 0
		\end{cases}\]






%% Apéndices (ejercicios, exámenes)
\appendix

% -*- root: ../EDP2016.tex -*-
\chapter{Resumen rápido}

\section{Ecuaciones de primer orden}

En general, para las ecuaciones de primer orden trabajaremos sobre un sistema \[ \begin{cases}
u_t + [q(u)]_x = f(x,t) \\
u(x,0) = F(x)
\end{cases}\], donde $f$ es la aportación externa, $q$ el flujo (que depende de la densidad) y $F(x)$ el dato inicial.

\subsection{Flujo lineal. Principio de Duhamel}

Para entender las soluciones trabajaremos con las curvas características, curvas en el espacio $x,t$ a lo largo de las cuales la solución se mantiene constante. Con un flujo lineal y sin aportación externa ($q(u) = cu$, $f \equiv 0$) las características son rectas de la forma $x - ct = x_0$, con $x_0$ el punto inicial (en $t = 0$) de esa recta.

Cuando la aportación externa no es nula, el enfoque es separar la solución en una parte que depende del dato incial y otra que depende de la aportación externa. El desarrollo está en la \fref{sec:ModeloGeneral}, pero la parte que nos interesa es la ecuación \eqref{eq:ModeloCombinado}, a la que se llega igualmente con el \nref{sec:PrincipioDuhamel} \[ u(x,t) = F(x-vt)+ \int^{t}_{0} f(x-v(t-\tau),\tau) \dif \tau \]

\subsection{Flujo más complicado}

Si el flujo se hace más complicado, como en la \fref{sec:ModeloTraficoRealista} que lo hacemos parabólico, la cosa se complica aunque no demasiado. Esta vez las curvas características serán de la forma \[ x - (1 - 2F(x_0)) t = x_0 \], donde $x_0$ es el punto inicial de la recta. Lo interesante es que la pendiente depende del valor inicial, y eso nos llevará a apariciones de ondas de choque o rarefacción cuando tengamos discontinuidades en el dato inicial.

Esa onda tendrá una ecuación $s(t)$ que se puede resolver a partir de su derivada \eqref{eq:DerivadaOndaChoque}: \[ s'(t) = \frac{q(u(s(t),t)^{-}) - q(u(s(t),t)^{+})}{u(s(t),t)^{-} - u(s(t),t)^{+}} \], que aunque parece muy liosa en realidad es simplemente el cociente entre el salto del flujo y el de la densidad cuando nos acercamos a la discontinuidad.

En el caso en el que tengamos no sólo ondas de choque sino también zonas en las que no hay características, podemos suponer la existencia de funciones $F_ε$ continuas que aproximan la función dato discontinua. Sacando las caraterísticas de ese dato y pasando al límite $ε \to 0$ podremos sacar las características en esa zona. Un ejemplo es el resultado de \eqref{eq:SolucionRarefaccion} para el modelo del semáforo.

\subsection{Ecuación de Burgers}

Otro método para resolver sistemas con flujo no lineal es el de la \textbf{Ecuación de Burgers}, desarrollado en la \fref{sec:EcuacionBurgers}. Consideramos que la derivada del flujo es $[q(u)]_x = V(u)u_x$, y entonces se realiza el cambio de variables $W = V(u)$. Esto nos permite entonces pasar de ecuaciones de la forma \[ W_t + WW_x = 0 \] a ecuaciones con un flujo $q_B = \frac{W^2}{2}$ asociado que sabemos cómo resolver con el método de las características. Las caracterstícas serán de nuevo $x - kt = x_0$ con $F(x_0) = k$.

\subsection{Caso general: Problema de Cauchy}

El desarrollo completo está en la \fref{sec:ProblemaCauchy}, pero la parte que nos interesa se resume rápido. Partimos de una ecuación \[ a(x,y,u) u_x + b(x,y,u) u_y = c(x,y,u) \], y tenemos un dato dado por una curva \[ Γ \equiv (α(s), β(s), γ(s)) \]

Si se cumple la condición de transversalidad \eqref{eq:CondTransversalidad}, esto es, que la siguiente matriz tenga determinante no nulo \[ \left|\begin{matrix} a & b \\ α' & β' \end{matrix}\right| ≠ 0 \],
entonces (\fref{thm:Transversalidad}) tenemos una solución única en un entorno local de cada punto que se puede construir con las curvas características. Estas curvas $(x,y,z)$ se podrán obtener resolviendo el siguiente sistema de EDOs: \begin{align*}
 x'(t) &= a(x(t), y(t), z(t)) \\
 y'(t) &= b(x(t), y(t), z(t)) \\
 z'(t) &= c(x(t), y(t), z(t)) \\
 (x(0), y(0), z(0)) &= (α(s), β(s), γ(s))
\end{align*}

Si la matriz tiene determinante nulo, la única posibilidad para tener una solución $C^1$ es que el rango de la matriz $\Dif Φ$ sea $1$ (ver \fref{sec:CondTransversalidadInvalida} para más detalles), en cuyo caso tendremos que hacer lo mismo que antes para encontrar la solución.

Si tenemos la mala suerte de que nos sale que toda la curva dato es característica, no hay solución y el problema está mal planteado.


\chapter{Ejercicios}

Últimamente nos llegan comentarios de gente que dice que "los apuntes están muy bien, pero me daba vergüenza deciros que el ejercicio $X.Y$ estaba mal". Gracias.

Pero si tú te beneficias de estos ejercicios y crees que están mal, por favor háznoslo saber por email o en persona.

El profesor no va a tener ningún reparo en ponernos un 0 en el examen.

% -*- root: ../EDP2016.tex -*-
\section{Hoja 1}


\begin{problem}[1]

	\ppart Utilizar el principio de Duhamel para resolver el problema
	\[ u_t + v u_x = f(x,t) \]
	con dato inicial $u(x,0) = 0$.

	Encontrar una solución explícita cuando $f(x,t) = e^{-t} \sin x$.

	\ppart Resolver el modelo de evolución de la contaminación en un río, incluyendo un término de depuración natural proporcional a la concentración que se rige por la ecuación:
	\[ u_t + vu_x = -\gamma u \]
	donde $\gamma$ es una constante positiva, con condiciones
	\[ u(x,0) = 0 \text{ si } x > 0, u(0,t) = \beta \text{ si } t > 0 \]

	\solution


\end{problem}











\begin{problem}[2] Resolver la ecuación de Burgers $u_t + uu_x = 0$ con dato inicial:
	\[ g(x) =
	\begin{cases}
		1 & x \leq 0 \\
		1-x & 0 < x < 1 \\
		0 & x \geq 1
	\end{cases} \]

	\solution

\end{problem}











\begin{problem}[3] Resolver la ecuación de Burgers $u_t + uu_x = 0$ con dato inicial:
	\[ g(x) =
	\begin{cases}
		0 & x < 0 \\
		1 & 0 < x < 1 \\
		0 & x > 1
	\end{cases} \]

	\solution


\end{problem}











\begin{problem}[4] Se considera la ecuación del transporte \[u_t + \frac{1}{u^2} u_x = 0\] con dato inicial \[u(x,0)=f(x) =
	\begin{cases}
	1 & x < 0 \\
	2 & x > 0
	\end{cases}\]
	
	Dibujar el diagrama $ X-T $ de la solución (características y valores de la solución). Dibujar las gráficas de las funciones $ u(1,t) $ y $ u(x,2) $.
	
\solution

\end{problem}









\begin{problem}[5]
	Resolver el problema de Cauchy $u_x + u_y = u^2, u(x,0) = 1$.

	\solution
\end{problem}











\begin{problem}[6]
	Resolver el problema de Cauchy $xu_x - yu_y = u - y, u(y^2,y)=y$. Estudiar si puede existir alguna solución definida en un entorno del origen.

	\solution
\end{problem}










\begin{problem}[7]
	Resolver la ecuación $u_x + 3y^{2/3} u_y = 2$ sujeta a la condición inicial $u(x,0) = φ(x)$.

	\solution

\end{problem}














\begin{problem}[8]
	Resolver la ecuación $(y + u)u_x + yu_y = x - y$ sujeta a la condición inicial $u(x,1) = 1 + x$.

	\solution
\end{problem}













\begin{problem}[9]
	Se considera la ecuación $yu_x - xu_y = 0 (y > 0)$. Para cada uno de los siguientes datos iniciales:

	\ppart $u(x,0) = x^2$
	\ppart $u(x,0) = x$
	\ppart $u(x,0) = x, x > 0$

	comprobar si el problema tiene solución. Si la tiene, encontrarla; si no la tiene, explicar por qué.

	\solution

\end{problem}











\begin{problem}[10]
	Resolver la ecuación $u_y + u^2 u_x = 0$ en $x > 0$ bajo la condición inicial $u(x,0) = \sqrt{x}$. ¿Cuál es el dominio de la existencia de la solución?

	\solution
\end{problem}













\begin{problem}[11]
	Resolver el problema de Cauchy $xu_x + yu_y = -u$, $ u(\cos s \sin s) = 1$, $ 0 \leq s \leq \pi$. ¿Está la solución definida en todas partes?

	\solution
\end{problem}








\begin{problem}[12]
	\ppart Encontrar una función $u = u(x,y)$ que resuelva el problema de Cauchy $(x + y^2)u_x + yu_y + (\frac{x}{y} - y) u = 1$, $u(x,1) = 0, x \in \real$.
	\ppart Comprobar si se cumple la condición de transversalidad.
	\ppart Dibujar las proyecciones sobre el plano $(x,y)$ de la condiciones inicial y de las características que emanan de los puntos $(2,1,0)$ y $(0,1,0)$.
	\ppart ¿Está la solución obtenida en $(a)$ definida en el origen $(x,y) = (0,0)$? ¿Contradice esto al teorema de la existencia-unicidad?

	\solution


\end{problem}



% -*- root: ../EDP2016.tex -*-
\section{Hoja 2}

\begin{problem} Determinar si se puede usar el método de separación de variables para cada una de las EDPs siguientes. En caso afirmativo, hallar las EDOs resultantes. En ningún caso se pide resolver.

\ppart $xu_{xx} + u_t = 0$.
\ppart $tu_{xx} + xu_t = 0$.
\ppart $u_{xx} + u_{xt} + u_t = 0$.
\ppart $(ρ(x) u_x)_x - r(x) u_{tt} = 0$.
\ppart $u_{xx} + (x + y) u_{yy} = 0$.

\solution

\spart Suponemos que $u = XT$, sustituimos y \[ 0 = xu_{xx} + u_t = xX''T + T'X \implies \frac{x · X''}{X} = \frac{-T'}{T} = λ \]

\spart \[ 0 = tu_{xx} + xu_t = t X'' T + x T' X \implies \frac{X''}{xX} = \frac{-T'}{tT} = λ\]

\spart \[ 0  = u_{xx} + u_{xt} + u_t = X''T + X'T' + XT' \implies \frac{X''}{X' + X} = \frac{T'}{T} \]

\spart \[ 0 = (ρ(x) u_x)_x - r(x) u_{tt} = ρ' X'T + ρ X''T - rX T'' \implies \frac{ρ'X' + ρX''}{rX} = \frac{T''}{T} \]

\spart \[ 0 = u_{xx} + (x + y) u_{yy} = X''Y + (x + y) X Y'' \] y esto no da mucha sensación de poderse separar.

\end{problem}

\begin{problem} Consideramos la ecuación del calor en dos dimensiones espaciales, $u_t = k Δ u$, donde $u = u(x,y,t)$ y $Δ = \pd[2]{}{x} + \pd[2]{}{y}$ es el laplaciano respecto a las variales espaciales $x,y$.

Consideramos una solución de la forma $u(x,y,t) = X(x) Y(y) T(t)$. Hallar las EDOs satisfechas por $X$, $Y$ y $T$.

\solution

\noindent Vamos a sustituir en la ecuación y supongamos $k\neq 0$: \[ 0 = u_t - k Δ u = XYT' - kT\left(X''Y + XY''\right) \implies \frac{T'}{T} = \frac{k(X''Y + XY'')}{XY} = λ \]
Luego hemos obtenido
\[
\begin{cases}
	T' = \lambda T\\
	k(X''Y +XY'') = \lambda XY
\end{cases}
\]
Operando en la segunda:
\begin{align*}
	kX''Y + kXY'' = \lambda XY &\iff kX''Y = X (\lambda Y - kY'')\\
	\frac{X''}{X} &= \frac{\lambda Y - kY''}{kY} = \mu \quad \mu \in \real\\
	&\begin{cases}
		X'' = \mu X\\
		Y'' = (\frac{\lambda}{k} - \mu) Y \quad (k \neq 0)
	\end{cases}
\end{align*}

\end{problem}

\begin{problem} El movimiento de una membrana circular está gobernado por la ecuación de ondas en dos dimensiones espaciales: \[ u_{tt} = c^2(u_{xx} + u_{yy})\qquad x^2 + y^2 ≤ R^2, \; t > 0 \]

\ppart Escribir la ecuación en coordenadas polares $(r,θ)$.
\ppart Consideramos una solución de la forma $u(r,θ,t) = R(r)Θ(θ)T(t)$. Encontrar las EDOs satisfechas por $R$, $Θ$ y $T$.

\solution

\end{problem}

\begin{problem}[4] Encontrar la solución del problema \[ \begin{cases}
u_{tt} = u_{xx} 	& 0 < x < π,\; t > 0 \\
u(0,t) = 0 = u(π,t) & t ≥ 0 \\
u(x,0) = \sin^3 x	& 0 ≤ x ≤ π \\
u_t(x,0) = \sin 2x 	& 0 ≤ x ≤ π
\end{cases} \]

\solution

\textbf{Nota de Azorero:}

Sabemos:

\[f(x) =  \frac{a_0}{2} +\sum_k a_k \cos (kx) + b_k \sin(kx)  \]

\[ x \in [-\pi,\pi]\]

\[ \gor{f} = \begin{cases}
	f & \text{ en }[0,\pi] \\
	\text{extensión } & \text{ en } [-R,0]
\end{cases} \]

\[a_k = \frac{1}{\pi}  \int^{\pi}_{-\pi} \underbrace{\gor{f}(\underbrace{\cos kx}_{\text{par}}) }_{\text{impar}} dx \qeq 0 \]

Solución: $\gor{f} = $ extensión impar de f.

\[ b_k = \frac{1}{\pi} \int_{-\pi}^{\pi} \underbrace{\underbrace{\gor{f}(x)}_{\text{impar}} \underbrace{\sin(kx)}_{\text{impar}}}_{\text{par}} dx = 2 \frac{1}{\pi} \int^\pi_{0} f(x) \sin(kx) dx \]

La idea es que nos quede un desarrollo en serie de senos, ya que esta es una ecuación de contorno Dirichlet. Hacemos separción de variables y tenemos que \[ T''X = X''T \implies \frac{T''}{T} = \frac{X''}{X} = λ \]

Si $λ = 0$, entonces $X'' = 0$, $X'$ constante y $X$ es polinomio de grado uno. Si $X(x) = a + bx$, como $X(0) = 0$, $a = 0$, y como $X(π) = 0$, $b = 0$ por lo que por aquí no llegamos a nada.

Si $λ > 0$, $λ = μ^2$, la solución será de la forma $X(x) = a e^{μx} + be^{-μx}$. Si $X(0) = 0$, entonces $a = -b$; y si $X(π) = 0$, entonces $0 = a(e^{μπ} - e^{-μπ})$, así que o bien $a = 0$ o bien $μ = 0$. Lo segundo es imposible, lo primero nos da una ecuación trivial.

Si $λ < 0$, $λ = -μ^2$, la solución será de la forma $X(x) = a \cos μx + b \sin μ x$. Si $X(0) = 0$, entonces $a = 0$. Si $X(π) = 0$, entonces $0 = b \sin μπ$ y por lo tanto $μ = k ∈ ℤ$.

Resolvemos ahora para $T$ sabiendo que $λ = λ_k = - k^2$, con $k ∈ ℤ$. Tenemos que \[ \frac{T''}{T} = -k^2 \], vamos a tener igualmente una solución de la forma $T(t) = a \cos k t + b \sin k t$. Si $X(x) T(0) = \sin^3 x$, entonces $T(0) = 1$ y $a = 1$. Si $u_t (x,0) = \sin 2x$, entonces $T'(0) = 1$. Derivando, \[ T'(t) = -a \sin k t + b \cos k t\], luego $b = 1$ y $T(t) = \sin k t + \cos k t $.

Para ajustar esto, tenemos que escribir $\sin^3 x$ y $\sin 2x$ como serie de cosenos. Usamos el truco de Azorero y tomamos la extensión impar de $f$, de tal forma que podrmeos calcular \[ b_k = \frac{2}{π} \int_0^π f(x) \sin kx \dif x \] siguiendo las ecuaciones de \eqref{eq:CoefsFourier}. Vamos a ello:
\begin{align*}
b_k &= \frac{2}{π} \int_0^π \underbracket{\sin^3 x}_{u} \underbracket{\sin k x \dif x}_{\dif v}
	= \frac{2}{π} \left(-k \sin^3 x \cos x + \int_{0}^π 3k \cos kx \sin^2 \cos x \right|_{x = 0}^π
\end{align*}

No me apetece seguir. Hágalo usted mismo.

\seprule

Otra posibilidad es usar la fórmula de D'Alembert \eqref{eq:DALEMBERT}, con $f(x) = \sin^3 x$ y $g(x) = \sin 2x$, de tal forma que \begin{multline*} u(x,t) = \frac{\sin^3 (x+t) + \sin^3(x-t)}{2} + \frac{1}{2} \int_{x-t}^{x+t} \sin 2s \dif s =  \\ =  \frac{\sin^3 (x+t) + \sin^3(x-t) - \cos 2(x+t) + \cos 2(x-t)}{2} \end{multline*}

\end{problem}

\begin{problem}[6] Resolver la siguiente ecuación: \[
\begin{cases}
u_t - u_{xx} = hu & x ∈ (0,π), t > 0 \\
u(0,t) = u(π,t) = 0 \\
u(x,0) = x(π-x)
\end{cases}\]

Usar el método de separación de variables. ¿Cuánto será $\lim_{t \to ∞} u(x,t)$?

\solution

Por searación de variables, buscamos $u(x,t) = X(x)·T(t)$, así que \begin{align*}
T'X-TX'' &= hXT \\
\frac{T'}{T} - \frac{X''}{X} &= h \\
\frac{T'}{T} &= \frac{X''}{X} + h = λ ∈ ℝ \\
\frac{T'}{T} - h &= \frac{X''}{X} = λ ∈ ℝ
\end{align*}

Además,la condición de contorno dice que $X(0) = X(π) = 0$. Podemos hacerlo de las dos formas que planteamos y resolviendo el sistema correspondiente. En nuestro caso resolveremos el siguiente sistema, aunque da lo mismo:
\begin{gather*}
\begin{cases}
X'' = λX \\
X(0) = X(L) = 0
\end{cases}
\\
\begin{cases} T' = (λ+h)T \end{cases} \end{gather*}

De nuevo haciendo las cuentas de toda la vida, tenemos que tener $λ_k = - k^2$ y las soluciones de la forma \[ X_k(x) = a_k \sin kx \], y con esto resolvemos en $T$ y tenemos \[ T_k(t) = α_k e^{-(k^2-h)t} \]

Las soluciones particulares serán entonces \[ u_k(x,t) = X_k(x) T_k(t) = A_k e^{-(k^2-h)t} \sin kx \] y la solución general de la forma \[ u(x,t) = \sum_{k=1}^{∞} A_k e^{-(k^2-h)t} \sin kx \], donde los $A_k$ serán los coeficientes del desarrollo en serie de senos del dato $x(π-x)$.

Una vez resuelto todo, lo que vemos es que la única diferencia con otros ejercicios es el coeficiente $h$ en el lado derecho de la ecuación. Así, si $h$ es positivo y muy grande podemos tener el problema de que esa suma no converja.

Si $h < 1$, todas las exponenciales tienen exponente negativo, así que $u(x,t) \convs[][t] 0$. Si $h = 1$, sólo la primera tiene exponente positivo, así que $u(x,t) \convs[][t] A_1 \sin x$. El problema ocurrirá cuando $h > 1$, el primer exponente es positivo y el límite no existe.

\end{problem}

\begin{problem}[11] Pruébese que \[ \lim_{n\to ∞} \int_0^π \log x \sin nx \dif x = 0\]

\solution

\end{problem}


\begin{problem}[9]

	\[
	\begin{array}{l}
		u_t - u_{xx} = 0, x \in (0,1), t>0
	\end{array}
	\]

	\solution

	El dato de contorno no es homogéneo ($\neq 0$). Intentaremos descomponer el problema. Encontrar una solución que sea suma de dos funciones $u=v+w$. COn la $v$ intentaremos ajustar el dato de contorno y con la $w$ hacer que el dato inicial se cumpla.

	\[ v_t - v_{xx} = 0 \]
	\[ v(0,t) = 0, v(1,t) = 1 \]

	Tenemos una solución trivial, estacionária que sería:

	\[ \left.\begin{array}{l}
		v''(x) = 0 \\
		v(0) = 0 \\
		v(1) = 1
	\end{array}\right\} \Rightarrow v(x) = x  \]

	Faltaría calcular la $w$:

	\[ \begin{array}{l}
		w_t - w_{xx} = 0 \\
		w(0,t) = 0 = w(1,t) \text{ dato de contorno homogéneo } \\
		w(x,0) = u(x,0) -v(x) = \Phi(x) - x
		\end{array}
	\]

	Y hemos llegado a un problema de los que ya sabemos resolver así que no tenemos problema. Se queda pendiente terminar el ejercicio.

\end{problem}

\begin{problem}[10]

	\[f(x) = e^x, x \in (-\pi,\pi)\]
	\[S_n f(x) = \text{N-ésima suma de Fourier de f}\]

	\[\text{¿} S_n f(\pi) \convs * \text{?}\]

	\solution

	\begin{center}
	\begin{tikzpicture}
	\draw[-] (-3,0) -- (3,0) node [below] {$\theta$};
	\draw[-] (0,-0.5) -- (0,2);

	\draw[-] (-1.5,0.1) -- (-1.5,-0.1) node[below] {$-\pi$};
	\draw[-] (1.5,0.1) -- (1.5,-0.1) node[below] {$\pi$};

	\draw[blue, thick] (-1.5,0.2) .. controls (0,0.6) .. (1.5,1.5) ;
	\draw[blue, thick] (-4.5,0.2) .. controls (-3,0.6) .. (-1.5,1.5) ;
	\draw[blue, thick] (1.5,0.2) .. controls (3,0.6) .. (4.5,1.5) ;

	\draw[dashed, green] (1.5,2) -- (1.5,0);

	\end{tikzpicture}
	\end{center}

	Representamos la extensión $2\pi$-periódica de la función. Y nos preguntan que pasa en el punto $\pi$. Tenemos dos teoremas de Dirichlet sobre convergencia puntual. En este caso usamos el segundo teorema de Dirichlet que nos dice:

	\[S_n f(\pi) \convs \frac{e^\pi+e^{-\pi}}{2}\]


\end{problem}





\bibliography{../Apuntes}{}
\printindex
\end{document}

