\chapter{Comportamiento cualitativo}

% clase 15/3/2016

En este capítulo vamos a ver:

\begin{itemize}

	\item ondas (hiperbólica)
	\item calor (parabólica)
	\item laplace (elípticas)

\end{itemize}


\section{Ondas}

	Hemos visto ya el problema en dimensión 1: la cuerda vibrante.

	Hemos visto también el método de separación de variables que solo nos valdrá en un dominion acotado. Pero ¿qué podemos hacer en dominios no acotados? y ¿Hay más soluciones?.

	Después vimos al fórmula D'Alambert \ref{eq:DALEMBERT}, que si que nos sirve para dominios no acotados pero es dificil de interpretar en dominios acotados. Veamos un ejemplo de esta afirmación:

	\begin{example}

		\[\begin{cases}
			u_{tt} - u_{xx} = 0, x \in (0,L) t > 0 \\
			u(0,t) = u(L,t) = 0, t > 0\\
			u(x,0) = f(x) \\
			u_t(x,0) = 0
		\end{cases}\]

		Y tenemos la fórmula a la que llegamos:

		\[ u(x,t) = \frac{1}{2} \{f(x+t)+f(x-t)\}  \]

		Resulta que tenemos una fórmula muy útil pero ocurre que en un dominio acotado no podemos calcular $f$ cerca del borde al realizar $x+t$ o $x-t$. Tenemos que encontrar una manera de extender $f$ de manera que esté de acuerdo con el contorno. No es lo mismo que una cuerda esté sujeta y la onda se refleje de una manera de vuelta en la cuerda a que no esté sujeta.

		\begin{center}
			\begin{tikzpicture}
			\draw[-] (-1,0) -- (3,0);
			\draw[-] (0,-0.5) -- (0,2);


			\draw[dashed] (1.5,0) node [below] {$L$} -- (1.5,1.5);

			\draw[thick, blue] plot[smooth, tension=.9] coordinates{(0.7,0) (0.8,0.2) (1,0.4) (1.2,0.7) (1.5,0.9)};

			\end{tikzpicture}
		\end{center}

	\end{example}


	\subsection{Unicidad}

		\[\begin{cases}
		   u_{tt}-u_{xx} = 0, x \in (0,L), t >0\\
		   \text{contorno} \begin{cases}
		   	\text{dirichlet}\\
		   	\text{neumann}\\
		   	\text{periódicas}
		   \end{cases}\\
		   \begin{cases}
		   u(x,0) = 0 \\
		   u_t (x,0) = 0
		   \end{cases}
		\end{cases}
		 \]


		 $u_1, u_{2}$ son soluciones. $u = u_1 - u_2$ es solución + controno + datos $\equiv$ 0. Queremos probar que $u\equiv0$


		 \[ \int_{0}^{L} u_t (u_{tt}- u_{xx}) dx = 0 \]

		 \[ \int_0^L \underbrace{u_t u_{tt}}_{\frac{1}{2}(u^2_t)_t} - u_t u_{xx} dx  \]

		 Tenemos operamos por partes:

		 \[ \int_0^L  \underbrace{u_t}_{u} \underbrace{u_{xx}}_{dv}  dx = \left. u_t u_x \right|_{x=0}^{L} - \int_0^L \underbrace{u_x u_{tx}}_{\frac{1}{2}(u_x^2)_t} dx \]

		 Y volviendo tenemos que:

		 \[  \int^{L}_{0} \frac{1}{2} (u_t^2)_t dx  \]

		 (FALTA)


		 \textbf{Conclusión}

		 \[ E(t) = \frac{1}{2} \int_{0}^L u_t^2(x,t) + u_x^2(x,t) dx \]

		 \[ E'(t) = 0 \forall t \quad \text{Conservación de energía total}\]


		 \[ E(t) = E(0) = \frac{1}{2} \int_0^L u_t^2 (x,0) + u_x^2 (x,0) dx = \frac{1}{2} \int_0^L g^2(x)+(f'(x))^2 dx \eqreason{$f=g=0$} = 0 \]

		 \[ E=0 \Rightarrow u_t = u_x = 0 \Rightarrow u \equiv \text{cte y } u(x,0) = 0 \Rightarrow u \equiv 0\]

		 Por lo tanto hemos demostrado que las dos soluciones son iguales y por lo tanto solo hay una solución.


