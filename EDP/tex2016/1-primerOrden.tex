\chapter{Ecuaciones de primer orden}

\section{Modelo de un atasco}

	Vamos a empezar con un modelo sencillo. Observemos el fenómeno del embotellamiento fantasma, el cual tiene una ecuación bastante sencilla. Veamos como se generan esos atascos y como se disipan.

	Estudiaremos una función que nos de la densidad de coches en un punto, en el tiempo.

	$$u(x,t) = \text{densidad}$$

	Observemos un modelo muy simple en el que solo hay un carril y los coches fluyen solo en una dirección.

	(DIBUO DE LA RECTA CON FLECHA HACIA LA DERECHA)

	La variación en el número de coches de un intervalo será:

	$$ \frac{d}{dt} \int^{x+h}_{x} u(s,t) ds = \int^{x+h}_{x} u_t(s,t) ds $$

	Esto equivale a al número de coches que entran por x menos los que salen por x + h. Es decir, tenemos un flujo:

	$$\text{Flujo: } q(u, x, t)$$

	Si suponemos que nos fijamos en un instante lo suficientemente pequeño en un tramo pequeño podemos asumir que depende solo de $u$. En ese caso, si $q > 0$ el flujo sería hacia la derecha.

	$$ q(u(x,t)) - q(u(x+h,t)) = -( q(u(x+h,t)) -  q(u(x,t)) ) \eqexpl{Barrow} -\int^{x+h}_{x} [q(u)]_{x} $$

	Por lo tanto tendremos sistemas del tipo:

	\begin{equation*}
	\left\{
	\begin{array}{l}
	u_t + [q(u)]_{x} = 0 \\
	u(x, 0) = F(x) \quad \quad \text{dato}
	\end{array}
	\right.
	\end{equation*}

	Ahora debemos probar distintas funciones q.


	\subsection{Modelo básico}

		Tomando $q(u) = cu$.

		\begin{equation*}
		\left\{
		\begin{array}{l}
		u_t + cu_x = 0 \\
		u(x,0) = F(x)
		\end{array}
		\right.
		\end{equation*}


		Esta ecuación dice que el flujo es proporcional a la densidad. Implica que los coches se mueven con velocidad constante, a más velocidad más flujo y viceversa. Si en el instante inicial tenemos una función de la densidad respecto de las zonas del tramo, según pasa el tiempo la densidad se irá desplazando hacia la derecha manteniendo la misma forma.

		(DIBUJO)

		\textbf{Comprobación:}

		$$ u(x,t) = F(x-ct) $$
		$$ u_x(x,t) = F'(x-ct)$$
		$$u_t(x,t) = F'(x-ct)*(-c)$$

		$$ \{ u_t + cu_x = … = 0 \quad\quad u(x,0) = F(x) \} $$

		\subsubsection{Representación gráfica}


			$$u(x,t) = \text{Cte.}$$
			$$F(x - ct) = \text{Cte.}$$

			Conclusión: $x-ct = k$.


			Las soluciones dependiendo de k se desplazan por el plano siguiendo una recta de inclinación $c$. Siguiendo una fórmula del tipo $x-ct = k$. Con esto se puede obtener como avanzan los datos (densidades) respecto de un instante inicial.

			Esto son los conjuntos de nivel de las soluciones.

			(DIBUJO)



	\subsection{Otro ejemplo}

		Imaginemos un río con una velocidad constante, en el que a partir de un punto realizamos un vertido. El río tiene velocidad $v$ y en $x=0$ realizaremos una contaminación $\beta(t)$. En $t=0$ consideraremos que está limpio.

		Tenemos las ecuaciones:

		$$u_t + vu_x = 0$$
		$$u(x,0) = 0 \quad (x>0) $$
		$$u(0,t) = \beta(t), t>0$$


		Resolvemos el problema igual que antes, buscamos los conjuntos de nivel que serán rectas como nos ha salido antes. Vamos a buscar esas rectas.

		\textbf{Buscamos $x(t)$ tal que $u(x(t),t)$ sea constante}

		Derivando en t: $u_x x' + u_t = 0$ que junto con la ecuación $u_t + v u_x = 0$ nos da que $x' = v \Rightarrow x = x_0 + vt $.

		Las características saldrían \(x-vt = \text{Cte}(=x_0) \label{eq:rio_vcte}\).

		Esas son nuestras rectas conjuntos de nivel, que en este caso indicarían el frente de la contaminación, el límite a partir del cual el río sigue limpio.

		$$u(x,t) =
			\begin{cases}
				0                      & (x-vt) > 0 \\
				\beta(t - \frac{x}{v}) & (x-vt) < 0
			\end{cases}
		$$

		\begin{figure}[hbtp]
			\centering
			\inputtikz{ContaminacionRio}
			\caption{}.
			\label{fig:ContaminacionRio}
		\end{figure}

		El valor de $u(x,t)$ cuando $x-vt < 0$, se obtiene utilizando \ref{eq:rio_vcte} y tomando el punto de corte de la recta $x-c \cdot t=k$ que pasa por $(0,t^*)$:

		$$
		\begin{rcases}
			0 - vt^{*} = k \\
			x - vt = k
		\end{rcases}
		 \Rightarrow x-vt = -vt^{*} \iff t^* = \frac{x-vt}{-v} = t - \frac{x}{v}$$

		Por lo que
		$$u(x,t) = \beta(t^*) = \beta(t - \frac{x}{v}), \quad \text{ cuando } x - vt < 0$$

		De esto obtenemos una función de x en función del tiempo que nos permite saber cuándo una parte del río se contamina.

	\subsection{Ejemplo más avanzado}

		Supongamos que existe descomposición biológica:

		$$u_t + vu_x = -\gamma u$$
		$$u(x,0) = \text{Cte}$$
		$$u(0,t) = \beta$$

		(Cambiamos de variable)

		$$u_t + \gamma_u + vu_x = 0$$
		$$e^{\gamma t} u_t + e^{\gamma t} u + v e^{ \gamma t} u_x = 0 $$

		lo que es lo mismo

		$$(e^{\gamma t}u)_t + v (e^{\gamma t} u)_x$$

		Y definimos la función $W$:

		$$W = e^{\gamma t}u$$.


		Hay que comprobar cuál es el efecto del término de descomposición biológica ($-\gamma u$). ¿ $W(x,0)$ , $ W(0,t)$ ?. Esto queda como ejercicio al lector.


	\subsection{Modelo con aportación externa}

		$$u_t + vu_x = f(x,t)$$
		$$u(x,0) = F(x)$$

		Ya al no ver constante, la solución no va a ser constante a lo largo de las características, que siguen existiendo. En el caso $f = 0$ éstas daban la solución, pero ahora van a influir.


		\subsubsection{Solución 1}

			Cambio de variables:
			$$z = x-vt$$

			Lo cual hace más sencillo observar la variación temporal. Continuamos el cambio:
			$$x = z + vt$$
			$$W(z,t) = u(z+vt, t)$$

			Entonces queremos estudiar la variación de $W$ respecto del tiempo:

			$$W_t(z,t) = u_x(z + vt, t)v + u_t(z+vt, t) = f(z + vt, t)$$

			Integramos:
			$$W(z,t) = W(0,t) + \int^{t}_{0} f(z+v\tau, \tau) d\tau $$

			Lo que se entiende como el dato inicial más la suma de las aportaciones de esta función $f$ en el tiempo pasado. Ahora deshacemos el cambio de variable:

			$$u(z + vt, t) = u(z,0) + \int^{t}_{0} f(z+v\tau, \tau) d\tau$$
			$$z = x-vt$$
			$$u(x,t) = u(x-vt,0)+ \int^{t}_{0} f(x-v(t-\tau),\tau) d\tau$$

			(donde $u(x-vt,0) = F(x-vt)$)

			Esto es un milagro ya que tiene una fórmula explícita, que es poco común en EDOs y EDPs.

		\subsubsection{Solución 2}

			Vamos a descomponer el sistema en 2. Dos soluciones en las que cada una va a ser solución de dos partes del problema que nos interesan. Una del dato inicial y otra del resto.
			$$u = \phi + \psi$$

			Por un lado:
			$$
			\begin{rcases}
				\phi_t + v\phi_x = 0 \\
				\phi(x,0) = F(x)
			\end{rcases}
			\rightarrow \phi(x,t) = F(x-vt)
			$$

			Y por otro:
			$$\psi_t + v\psi_x = f(x,t)$$
			$$\psi(x,0) = 0$$

			Este problema con \concept{DUHAMEL} (la fuente externa se interpreta como una fuente de datos iniciales, se cambia el cero de la segunda ecuación de las anteriores) se puede interpretar como que tenemos una función que transporta el sistema.

			$$\xi_t + v\xi_x = 0$$
			$$\xi(x,s) = f(x,s)$$

			Continuamos:
			\(\xi(x,t) = f(x^*,s) \label{eq:modelo_aportacion_externa_1}\)
			$$
			\begin{rcases}
				x-vt = k \\
				x^* - vs = k
			\end{rcases}
			 \rightarrow x-vt = x^* - vs
			$$
			\( x-v(t-s) = x^{*} \label{eq:modelo_aportacion_externa_2} \)

			Y sustituimos \ref{eq:modelo_aportacion_externa_2} en \ref{eq:modelo_aportacion_externa_1}
			$$\xi(x,t) = f(x - v(t-s), s)$$


			Tenemos la solución correspondiente de un aporte instantáneo en el punto $t$. Por lo tanto la solución será la suma de todos los aportes instantáneos:

			$$\Rightarrow \psi(x,t) = \int^{t}_{0} f(x-v(t-s),s) ds $$

			Aunque hemos hecho un montón de asunciones que tendremos que comprobar en algún momento.

	\subsection{Modelo de tráfico más realista}

		No hemos tenido en cuenta que la velocidad de los coches no es totalmente proporcional a la densidad. Si ésta baja mucho, los coches llegan a pararse. Hay una densidad máxima donde el tráfico quedará totalmente estancado (los coches pegados).

		Una función más realista sería una parábola. Si no hay coches no pasa ninguno, pero a partir de un punto cuantos más coches, menos rápido van, y menos pasan.

		$$ q(u) = Au (B-u) = ABu - Au^{2} $$

		Veamos su ecuación:
		$$u_t + [q(u)]_x = 0$$
		$$ … $$
		$$ u_t + (AB - 2Au) u_x = 0 $$

		La velocidad va a ser dependiente de la densidad.

		Nos saldrán características cuya pendiente dependa del valor en el tiempo inicial. Si divergen no habrá problema, pero si convergen, habrá puntos donde rectas de densidad 2 (por ejemplo) cortarán con rectas de densidad 1. Esto se entiende como coches que van más rápido que se encuentran con coches que van más lentos y se ven obligados a frenar.


