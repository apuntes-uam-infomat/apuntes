% -*- root: ../EDP2016.tex -*-
\section{Hoja 2}

\begin{problem}[1] Determinar si se puede usar el método de separación de variables para cada una de las EDPs siguientes; en caso afirmativo, hallar las EDOs resultantes (en ningún caso se pide resolver).

\ppart $xu_{xx} + u_t = 0$.
\ppart $tu_{xx} + xu_t = 0$.
\ppart $u_{xx} + u_{xt} + u_t = 0$.
\ppart $(ρ(x) u_x)_x - r(x) u_{tt} = 0$ (donde $ρ(x)$,$r(x)$ son dos funciones dadas).
\ppart $u_{xx} + (x + y) u_{yy} = 0$.

\solution

\end{problem}





\begin{problem}[2] Consideramos la ecuación del calor en dos dimensiones espaciales, $u_t = kΔu$, donde $u = u(x,y,t)$ y $Δ = \pd[2]{}{x} + \pd[2]{}{y}$ es el laplaciano respecto a las variales espaciales $x,y$.

Consideramos una solución de la forma $u(x,y,t) = X(x) Y(y) T(t)$. Hallar las EDOs satisfechas por $X$, $Y$ y $T$.

\solution

\end{problem}





\begin{problem}[3] El movimiento de una membrana circular está gobernado por la ecuación de ondas en dos dimensiones espaciales: \[ u_{tt} = c^2(u_{xx} + u_{yy})\qquad x^2 + y^2 ≤ R^2, \; t > 0 \]

\ppart Escribir la ecuación en coordenadas polares $(r,θ)$.
\ppart Consideramos una solución de la forma $u(r,θ,t) = R(r)Θ(θ)T(t)$. Encontrar las EDOs satisfechas por $R$, $Θ$ y $T$.

\solution

\end{problem}





\begin{problem}[4] Encontrar la solución del problema \[ \begin{cases}
u_{tt} = u_{xx} 	& 0 < x < π,\; t > 0 \\
u(0,t) = 0 = u(π,t) & t ≥ 0 \\
u(x,0) = \sin^3 x	& 0 ≤ x ≤ π \\
u_t(x,0) = \sin 2x 	& 0 ≤ x ≤ π
\end{cases} \]

\solution

\end{problem}





\begin{problem}[5]Resolver la ecuación del calor $ u_t=12u_{xx} $ en $ 0<x<\pi, t>0 $ con las siguientes condiciones: \[ \begin{cases}
u_x(0,t) = 0 = u_x(\pi, t) 	& t ≥ 0 \\
u_t(x,0) = 1 + \sin^4 x        & 0 ≤ x ≤ π
\end{cases} \]

Calcular $ \lim_{t \to ∞} u(x,t) $ para todo $ 0 < x < \pi $ y dar una interpretación física del resultado.

\solution

\end{problem}





\begin{problem}[6] Se considera el problema \[
\begin{cases}
u_t - u_{xx} - hu = 0 & x ∈ (0,π), t > 0 \\
u(0,t) = u(π,t) = 0 & t \geq 0 \\
u(x,0) = x(π-x) & 0 \leq x \leq \pi
\end{cases}\]

\ppart Resolver el problema utilizando el método de separación de variables
\ppart ¿Existe $ \lim_{t \to ∞}u(x,t) $ para todo $ 0 < x < \pi $ ? \hint{ distinguir los casos $ h<1 $, $ h=1 $ y $ h>1 $}.

\solution

\end{problem}





\begin{problem}[7] Consideramos la función $ \phi(x) = x^2 $ en $ 0 \leq x \leq 1 $.
\ppart Calcular su desarrollo de Fourier en serie de senos.
\ppart Calcular su desarrollo de Fourier en serie de cosenos.

\solution
\end{problem}





\begin{problem}[8] Calcular el desarrollo en serie de cosenos de la función $ \abs{\sin x} $ en el intervalo $ (-\pi, \pi) $. Supongamos que la serie converge a la función para cada $ x $ del intervalo. Utilizarla para calcular las sumas \[ \sum_{n=1}^{\infty} \frac{1}{4n^2-1} \quad y \quad \sum_{n=1}^{\infty} \frac{(-1)^n}{4n^2-1} \]


\solution
\end{problem}





\begin{problem}[9] Un hilo tiene longitud $ L=1 $ y coeficiente de difusión térmica $ k=1 $. Su temperatura satisface la ecuación del calor. Su extremo izquierdo se mantiene a temperatura 0 y el derecho a temperatura 1. Inicialmente (en $ t=0 $) la temperatura viene dada por \[ \phi(x) = \begin{cases}
	\frac{5x}{2}, & 0<x<\frac{2}{3} \\
	3-2x,         & \frac{2}{3} < x < 1
	\end{cases} \]

Calcular la solución. \hint{ Encontrar una solución estacionaria $ U(x) $ de la ecuación junto con las condiciones de contorno dadas, y resolver la ecuación del calor con datos de contorno de tipo Dirichlet y condición inicial $ u(x,0)=\phi(x)-U(x) $}

\solution

\end{problem}





\begin{problem}[10] Sea $ f(x) = e^x $ en $ [-\pi, \pi ] $. Extendemos $ f $ periódicamente a todo $ \real $. Determina la suma de su serie de Fourier en $ x = \pi. $


\solution

\end{problem}





\begin{problem}[11]Pruébese que $ \lim_{n\to ∞} \int_{0}^{\pi} \log x \sin (nx) dx = 0 $


\solution

\end{problem}





\begin{problem}[12] Sea $ f $ la función 10-periódica definida por \[f(x) = \begin{cases}
	0 & -5 < x < 0 \\
	3 & 0 < x < 5
\end{cases} \] Hallar sus coeficientes de Fourier. ¿Cómo habría de definirse $ f $ en los puntos $ x=-5 $, $ x=0 $ y $ x=5 $ par aque la serie de Fourier converja entodo punto?


\solution

\end{problem}





\begin{problem}[13]Demostrar que si $ u(r,\theta) $ es la solución del problema de Dirichlet para el disco $ r \leq 1 $ con dato de contorno periódico $ f \in C([-\pi,\pi]) $, se tiene que $ \min_{[-\pi,\pi]} f \leq u(r,\theta) \leq \max_{[-\pi,\pi]} f $


\solution

\end{problem}





\begin{problem}[14]  Sea $ u $ una función armóonica (es decir, $ Δu=0 $) en el disco $ D=\{x^2+y^2<4\} $ y tal que $ u(2\cos\theta, 2 \sin\theta)=3\sin(2\theta) +1 $. Encontrar el máximo valor de $ u $ en $ \overline{D} $ y el valor de $ u $ en el origen.


\solution

\end{problem}





\begin{problem}[15] Encontrar la función armónica en el círculo $ D = \{r<a\} $ con condición de frontera $ u=\sin^3\theta $ en la circunferencia $ r=a $.


\solution

\end{problem}





\begin{problem}[16] Probar que toda función armónica no negativa en el disco de rado $ a $ satisface la \textit{Desigualdad de Harnack}
	
\[ \frac{a-r}{a+r}u(0,0) \leq u(r,\theta) \leq \frac{a+r}{a-r} u(0,0)  \]


\solution

\end{problem}

