% -*- root: ../EDP2016.tex -*-
\section{Hoja 3}

\begin{problem}[1] Encontrar la solución de la ecuación de ondas $ u_{tt} = u_{xx} $ en $ x>0 $ con condición de contorno (unilateral) de tipo Dirichlet homogénea, $ u(0,t)=0 $, posición inicial $ u(x,0)=x^3+x^6 $ y velocidad inicial $ u_t(x,0)=\sin^2x $. ¿Es clásica (es decir, $ C^2 $)?

\solution

\end{problem}





\begin{problem}[2] Resolver el problema \[ \begin{cases}
	u_{tt} - u_{xx} = 0,    & x>0, t>0 \\
	u(0,t) = \frac{t}{1+t}, & t>0 \\
	u(x,0)=u_t(x,0)=0 ,     & x>0
\end{cases} \]

\solution

\end{problem}





\begin{problem}[3] Sea $ u $ solución de
\[ \begin{cases}
	 u_{tt} = u_{xx}, & 0<x<1 \\
	 u(x,0)=x^2(1-x) & \\
	 u_t(x,0) = (1-x)^2 & \\
	 u(0,t)=u(1,t)=0
\end{cases} 
\] Calcular $ u(\frac{2}{3},2) $ y $ u(\frac{1}{4},\frac{7}{2}) $


\solution

\end{problem}





\begin{problem}[4] Usar el método de energía para probar que el problema
\[ \begin{cases}
	u_{tt}+u_t-c^2u_{xx} =0,    & a<x<b, t>0\\
	u(a,t)=u_x(b,t)=0,          & t>0\\
	u(x,0)=f(x), u_t(x,0)=g(x), & a<x<b
\end{cases}
\]
tiene a lo sumo una solución

\solution

\end{problem}





\begin{problem}[5] Sea $ h $ una constante positiva. Usar el método de energía para probar el problema
	
\[ \begin{cases}
	u_{tt}-c^2u_{xx}+hu=F(x,t), & x\in\real, t>0 \\
	\lim_{x \to \pm ∞} u(x,t)=\lim_{x \to \pm ∞} u_x(x,t) = \lim_{x \to \pm ∞} u_t(x,t)=0, & t>0 \\
	\int_{∞}^{∞}(u_t^2+c^2u_x^2+hu^2)dx < ∞, & t\geq 0 \\
	U(x,0)=f(x), u_t(x,0)=g(x), & x\in\real
\end{cases}
\] tiene a lo sumo una solución.

\solution

\end{problem}





\begin{problem}[6] Obténgase por separación de variables la solución del problema \[ \begin{cases}
u_{xx}+u_{yy}=0, & (x,y)\in (0,\pi)\x(0,\pi) \\
u(0,y)=u(\pi,y)=u(x,\pi)=0 & \\
u(x,0)= \sin^2x
\end{cases}
\]

\solution

\end{problem}





\begin{problem}[7]Demostrar que el problema \[
\begin{cases}
-Δ u = F & \text{en} Ω \\
u = f & \text{en }C_1 ⊂ ∂C \\
\dpd{u}{\vn} + αu = g &\text{en } C_2 = ∂Ω \setminus C_1
\end{cases}\] con $α > 0$ y $Ω ⊂ ℝ^N$ dominio regular, tiene a lo sumo una solución.

\solution

\end{problem}





\begin{problem}[8] Probar la unicidad del problema de Dirichlet $ \Delta u=f $ en $ \Omega $, $ u=g $ en $ \partial\Omega $ mediante el método de energía. Es decir, dadas dos soluciones, $ u_1 $, $ u_2 $, multiplicar la ecuación que satisface su diferencia $ w \equiv U_1-u_2 $ por la misma función $ w $, integrar y usar el teorema de la divergencia.

\solution

\end{problem}





\begin{problem}[9] Sea $ u(x,y) $ la solución del problema \[ u_{xx}+u_{yy}=0 \quad (x,y) \in [-1,1]\x[-1,1] \]
con dato de contorno $ u(-1,y)=u(1,y)=u(x,-1)=0; u(x,1)=1 $. Hallar $ u(0,0) $.

\solution

\end{problem}





\begin{problem}[10] Probar que una solución de $ \Delta u - u^2=0 $ en un dominio $ \Omega $ no puede tomar su máximo en $ \Omega $, salvo que $ u \equiv 0 $.

\solution

\end{problem}





\begin{problem}[11] Sea $ \Omega = \{ x\in\real^N  : \abs{x}<1 \} $ y $ u\in C^2(\Omega)\cap C(\overline{\Omega}) $ solución del problema de Dirichlet $ \Delta = u^2+f(\abs{x}) $ si $ x\in\Omega, u=1 $ si $ x\in \partial\Omega $, donde $ f(\abs{x})\geq 0 $ es continuamente diferenciable. Calcular el máximo de $ u $ en $ \overline{\Omega} $ y demostrar que no depende de $ f $.
	
\solution

\end{problem}





\begin{problem}[12] Consideramos el problema

\[\begin{cases}
u_t - u_{xx} = 0,      & x \in (0,1), t > 0\\
u(0,t) = 0 = u(1,t),    & t>0\\
u(x,0) = 4x(1-x),       & 0<x<1
\end{cases}\]

\ppart Demostrar que $ 0<u(x,t)<1 $ para todo $ t<0 $ y $ 0<x<1 $.
\ppart Demostrar que $ u(x,t)=u(1-x,t) $ para todo $ t \geq 0 $ y $ 0 \leq x \leq 1 $
\ppart Usar el método de energía para probar que $ \int_{0}^{1} u^2dx$ es una función estrictamente decreciente de $ t $.

\solution

\end{problem}





\begin{problem}[13] El objetivo de este ejercicio es demostrar que el principio del máximo no es válido para la ecuación $ u_t=xu_{xx} $, porque tiene un coeficiente variable
	
\ppart Comprobar que $ u = -2xt - x^2 $ es una solución. Localizar su máximo en el rectángulo $ \{ x\in [-2,2], t\in [0,1] \} $.
\ppart ¿Dónde falla la prueba del principio del máximo visto en clase al aplicarla a esta ecuación?
	


\solution

\end{problem}





\begin{problem}[14] Consideramos la ecuación del calor en toda la recta con condición inicial $ u(x,0)=x^2 $.
\ppart Encontrar la solución como convolución del dato inicial con el núcleo del calor.
\ppart Encontrar la solución con el siguiente método especial: En primer lugar de mostrar que $ u_{xxx} $ satisface la ecuación del calor con dato inicial cero. Por lo tanto, por la unicidad de la solución para este problema, $ u_{xxx}=0 $. Integrando este resultado tres veces se obtiene que $ u(x,t)=A(t)x^2+B(t)x+C(t)$. Introduciendo esta expresión en el problema original, obtenemos un problema de EDOs que, una vez resuelto, proporciona A, B y C.

\solution

\end{problem}





\begin{problem}[15] Probar la unicidad del problema 
\[ \begin{cases}
u_t - u_{xx} = f(x,t),               & 0<x<L, t<0 \\
u_x(0,t)= g(t), \quad u_x(L,t)=h(t)  & t>0 \\
u(x,0)=\phi(x),                      & 0<x<L
\end{cases} \]
por el método de la energía

\solution

\end{problem}





\begin{problem}[16] Resolver la ecuación de difusión con disipación constante,
\[ \begin{cases}
u_t-ku_{xx}+bu=0,    & x\in \real, t>0 \\
u(x,0)=\phi(x)       & 0<x<L
\end{cases} \]
donde $ k,b>0 $ son constantes. \hint{Hacer el cambio de variables $ u(x,t)=e^{-bt}v(x,t) $}.

\solution

\end{problem}





\begin{problem}[17] Resolver la ecuación de difusión con disipación variable,
\[ \begin{cases}
u_t-ku_{xx}+bt^2u=0,     & x\in \real, t>0 \\
u(x,0)=\phi(x),          & x\in \real
\end{cases} \]
donde $ k,b>0 $ son constantes. Indicación: Las soluciones de la EDO $ w_t+bt^2w $ son de la forma $ Ce^{\frac{-bt^3}{3}} $. Esto sugiere hacer el cambio de variables $ u(x,t)=e^{\frac{-bt^3}{3}}v(x,t) $

\solution

\end{problem}





\begin{problem}[18] Resolver la ecuación de difusión con convección \[ \begin{cases}
u_t - a^2 u_{xx} + Vu_x = 0 & x ∈ ℝ,\, t > 0 \\
u(x,0) = φ(x) & x ∈ ℝ
\end{cases} \] donde $a,V$ son constantes. \hint{Pasar a coordenadas móviles $(y, τ)$ haciendo el cambio de variables $y = x - Vt$, $τ = t$.}
\solution

\end{problem}





\begin{problem}[19] Obtener una fórmula para la solución del problema de Neumann en la semirrecta 
\[ \begin{cases}
u_t-ku_{xx}=0,      & x>0, t>0 \\
u_x(0,t)=0,         & t>0 \\
u(x,0)=\phi(x),     & x>0
\end{cases} \]

\solution

\end{problem}





\begin{problem}[20] Sea $ u $ la solución del problema
\[ \begin{cases}
u_t-u_{xx}=0,      & 0<x<\pi, t>0 \\
u(0,t)=0=u(\pi,t),  & t>0 \\
u(x,0)=\sin^2x,     & 0<x<\pi
\end{cases} \] Demostrar que $ 0 \leq u(x,t) \leq e^{-t}\sin x $

\solution

\end{problem}





\begin{problem}[21] Obtener una fórmula para la solución de la ecuación de difusión no homogénea en la semirrecta con datos Dirichlet no homogéneos,
\[ \begin{cases}
u_t-ku_{xx}=f(x,t),      & x>0, t>0 \\
u(0,t)=h(t),             & t>0 \\
u(x,0)=\phi(x),          & x>0
\end{cases} \] \hint{Hacer un cambio de variables que transforme el problema en uno con condiciones de contorno homogéneas y después extender los datos a todo el espacio de forma adecuada.}

\solution

\end{problem}