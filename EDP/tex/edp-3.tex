\section{Funciones armónicas}
Esta sección tratará de funciones armónicas. En lo que sigue se denotará con $\Omega$ a un dominio de $\mathbb{R}^n$ y $u$ una función $u$ definida como sigue
$$u:\Omega \in \mathbb{R}^n \longrightarrow \mathbb{R}$$

\begin{definition}{Función armónica}
Se dice que $u$ es \textbf{armónica} $\iff u\in C^2(\Omega)$ y $\Delta u = 0$ en $\Omega$.

\noindent Se dice que $u$ es \textbf{subarmónica} $\iff u\in C^2(\Omega)$ y $\Delta u \ge 0$ en $\Omega$.

\noindent Se dice que $u$ es \textbf{superarmónica} $\iff u\in C^2(\Omega)$ y $\Delta u \le 0$ en $\Omega$.
\end{definition}

\subsection{Propiedad del valor de la media}
\begin{mathresult}{Propiedad de la media de las funciones armónicas}
Sea $u$ una función armónica y $B(x,R)$ una bola de centro $x$ y radio $R$. Si $\overline{B(y,R)}\in \Omega$, se tiene que 
$$u(x) = \frac{1}{\omega_nR^{n-1}}\int_{\delta B_R} u(x)dS = \frac{n}{\omega_n R^n} \int_{B_R}u(x)dV$$
donde $\omega_n$ es el área de la bola unidad de dimensión $n$. Es decir, estos tres valores son iguales:
\begin{itemize}
\item El valor de la función en el centro de la bola
\item El valor medio de la función en la superficie de la bola.
\item El valor medio en el interior de la bola.
\end{itemize}

Para funciones \textbf{subarmónicas}, se tiene que el valor en el centro de la bola es \textbf{menor} que el promedio de la función.

Para funciones \textbf{superarmónicas}, se tiene que el valor en el centro de la bola es \textbf{mayor} que el promedio de la función.
\end{mathresult}
\begin{proof}
Vamos a realizar la prueba para funciones \textbf{subarmónicas}. En principio, sabemos que 
$$\Delta u = div(\nabla u) = \nabla\cdot\nabla u = \nabla^2 u = \sum_{i=1}^n \frac{d^2u}{d^2x_i}$$
Vamos a definir $0<\rho<R$. Dado que $u$ es subarmónica ($\Delta u \ge 0$) tenemos que 
$$0 \le \int_{B_\rho}\Delta u(x) dV$$.

Utilizando el teorema de la divergencia de Gauss se tiene que 
$$\int_{B_\rho} div(\nabla u)dV = \int_{\delta B_\rho}<\nabla u(\xi), \nu>dS_\xi$$
o lo que es lo mismo
$$\int_{B_\rho}\Delta u(x) dV = \int_{\delta B_\rho}\frac{du}{d\nu}(\xi)dS_\xi$$
donde $\nu$ es la normal exterior a $B$.
Aplicamos el siguiente cambio de variables de $\xi$ a $w$
\begin{equation*}
\left\{
\begin{array}{l}
\xi = x+\rho w\\
w = \frac{\xi-x}{\rho}\\
||w|| = 1\\
\end{array}
\right.
\end{equation*}
Se llega a que
$$\int_{\delta B_\rho}\frac{du}{d\nu}(\xi)dS_\xi = \rho^{n-1}\int_{\delta B_1}\frac{du}{d\nu}(x+\rho w)dS_w$$
donde $B_1 = \{x\in\mathbb{R}^n: ||x|| = x_1^2+\hdots+x_n^2 < 1\}$

\noindent Ahora bién, dado que $$\frac{du}{d\nu}(x+\rho w) = <\nabla u(x+\rho w), \nu>$$ y en $B_1$, $\nu$ es equivalente al vector posición $w$. Se tiene que $$<\nabla u(x+\rho w), \nu> = <\nabla u(x+\rho w), w>$$
Luego $$\frac{du}{d\nu}(x+\rho w) = \frac{du}{d\rho}(x+\rho w)$$
Entonces
$$ \rho^{n-1}\int_{\delta B_1}\frac{du}{d\nu}(x+\rho w)dS_w  =  \rho^{n-1}\frac{d}{d\rho}\int_{\delta B_1}u(x+\rho w)dS_w$$
Si deshacemos el cambio de variable
$$\rho^{n-1}\frac{d}{d\rho}\int_{\delta B_1}u(x+\rho w)dS_w = \rho^{n-1}\frac{d}{d\rho}\left(\frac{1}{\rho^{n-1}}\int_{\delta B_\rho}u(\xi)dS_\xi\right)$$
Dado que $u$ es continua, se tiene que, si $\rho\to0$
$$\frac{1}{\rho^{n-1}}\int_{\delta B_\rho}u(\xi)dS_\xi\longrightarrow u(x)$$
Además, al ser este funcional no decreciente
\begin{equation}\label{eq:mediasubarm}
\frac{1}{\rho^{n-1}}\int_{\delta B_\rho}u(\xi)dS_\xi \le \frac{1}{R^{n-1}}\int_{\delta B_R}u(\xi)dS_\xi
\end{equation}
Por otro lado, el volumen de la bola $n-$dimensional es
$$Vol(B_1) = \omega_n = \frac{2\pi^{n/2}}{n\Gamma(\frac{n}{2})}$$
Como $$\int_{B_R}dx = R^n\int_{B_1}dy$$
se tiene $$Vol(B_R) = R^n\omega_n$$
El área de la bola $n-$dimensional es
$$Area(\delta B_1) = n\omega_n$$
luego
$$Area(\delta B_R) = n\omega_nR^{n-1}$$
Dividiendo \eqref{eq:mediasubarm} por $n\omega_n$ se tiene 
$$\frac{1}{n\omega_n\rho^{n-1}}\int_{\delta B_\rho}u(\xi)dS_\xi \le \frac{1}{n\omega_nR^{n-1}}\int_{\delta B_R}u(\xi)dS_\xi$$
Luego para toda bola
$$u(x) \le \frac{1}{n\omega_n\rho^{n-1}}\int_{\delta B_\rho}u(\xi)dS_\xi$$
Es decir
$$n\omega_n\rho^{n-1}u(x) \le \int_{\delta B_\rho}u(\xi)dS_\xi$$
Integrando con respecto a $\rho$ entre $0$ y $R$:
$$\omega_n\left.\rho^n\right|_0^R u(x) \le \int_0^Rd\rho\int_{\delta B_\rho}u(\xi)dS_\xi = \int_{B_R}u(x)dx$$
\end{proof}


\begin{mathresult}{Principio fuerte del máximo para funciones subarmónicas}
Sea $u(x)$ una función armónica en $\Omega$. Si $u$ alcanza su máximo en $\Omega$, entonces $u$ es constante en $\Omega$.
\end{mathresult}
\begin{proof}
Sea $M=\sup_\Omega u$ y $\Omega_M = \{x\in\Omega: u(x) = M\}$. Está claro que $\Omega_M$ es cerrado relativo a $\Omega$. Vamos a demostrar que $\Omega_M$ es también un abierto relativo a $\Omega$. De esta forma, como $\Omega$ es un dominio, al ser conexo, los únicos subconjuntos que son simultáneamente abiertos y cerrados son el vacío y el propio $\Omega$. Si $u$ tiene un máximo en $x_0$, tendríamos que $x_0\in\Omega_M\neq\emptyset$. Luego, $\Omega_M = \Omega$.

Sea $z\in\Omega_M$. Tenemos
$$0=(u(z) - M) = \frac{1}{|B_R|}\int_{B_R}(u(x)-M)dx = \frac{1}{|B_R|}\int_{B_R} 0dx = 0$$
Es decir, tenemos que $u(x)-M = 0$ en toda la bola. Por lo que para todo $z\in\Omega_M$ podemos encontrar una bola $B$ abierta tal que $B\in\Omega_M$, luego $\Omega_M$ es abierto.
\end{proof}

\subsection{Solución del problema de Dirichlet para el disco unidad}
\noindent De aquí en adelante, se denotará
$$\mathbb{D} = \{(x,y)\in \mathbb{R}^2: x^2+y^2 < 1\}$$
Dada $f\in\delta\mathbb{D}$, continua, $2\pi-$periódica y definida como sigue:
$$f:\mathbb{R}\longrightarrow\mathbb{R}$$
se quiere encontrar $u\in C^2(\mathbb{D})\cap C(\mathbb{\overline{\mathbb{D}}})$ que satisfaga
\begin{equation*}
\left\{
\begin{array}{l l}
-\Delta u = 0 & \text{en } \mathbb{D}\\
u=f & \text{en } \delta\mathbb{D}
\end{array}
\right.
\end{equation*}
Las soluciones que buscamos tienen la forma
$$u(x,y) = R(r)T(\theta)$$
con $r=||(x,y)||=\sqrt{x^2+y^2}$
Dado que la función $u$, es armónica, $u$ verifica $\Delta u = 0$, o lo que es lo mismo
$$R''T+\frac{1}{r}R'T+\frac{1}{r^2}RT'' = 0$$
Separando las variables, se obtiene
$$\frac{r^2R''+rR'}{R}=\frac{-T''}{T}$$
Observamos que tenemos una igualdad de dos funciones que dependen de variables distintas. El primer término depende sólamente de $r$, mientras que el segundo depende de $\theta$. Esto implica que dichos términos han de ser iguales a una constante $\lambda$, luego se tiene que
\begin{align}\label{eq:dirich1}
r^2R''+rR'-\lambda r = 0\\\label{eq:dirich2}
T''+\lambda T = 0
\end{align}
Tenemos un sistema de dos EDOs independientes, la primera es la ecuación de Cauchy-Euler, y la segunda es sencilla de resolver.
Las posibles soluciones para ambas EDOs son:
\begin{equation*}
\eqref{eq:dirich1}\ R(r) = \left\{
\begin{array}{l l}
1,log(r) & \lambda=0\\
r^{\sqrt{\lambda}}, r^{-\sqrt{\lambda}} & \lambda > 0\\
\text{soluciones complejas} & \lambda < 0
\end{array}
\right.
\end{equation*}
\begin{equation*}
\eqref{eq:dirich2}\ T(\theta) = \left\{
\begin{array}{l l}
1,\theta & \lambda=0\\
cos(\sqrt{\lambda}\theta), sin(\sqrt{\lambda}\theta) & \lambda > 0\\
cosh(\sqrt{-\lambda\theta}), sinh(\sqrt{-\lambda\theta}) & \lambda < 0
\end{array}
\right.
\end{equation*}
No se puede suponer que para todos los valores de $\lambda$ y para toda elección de $R$ y $T$, la fórmula $u(x,y) = R(r)T(\theta)$ va a definir una función armónica en un dominio $\Omega$. Esto es sólo verdad si se define una función $C^2(\Omega)$. Si $\Omega$ contiene curvas que encierran al origen, entonces la función $u$ será $2\pi-$periódica. En este caso $\Omega=\mathbb{D}$, por lo que $u$ ha de ser $2\pi-$periódica.

\noindent Eliminamos las siguientes soluciones por no ser $2\pi-$periódicas.
\begin{equation*}
\eqref{eq:dirich2}\ T(\theta) = \left\{
\begin{array}{l l}
1,\color{red}{\theta} & \lambda=0\\
cos(\sqrt{\lambda}\theta), sin(\sqrt{\lambda}\theta) & \lambda > 0\\
\color{red}{cosh(\sqrt{-\lambda\theta})}, \color{red}{sinh(\sqrt{-\lambda\theta})} & \lambda < 0
\end{array}
\right.
\end{equation*}
Para que
\begin{equation*}
\left.
\begin{array}{l l}
cos(\sqrt{\lambda}(\theta+2\pi)) = cos(\sqrt{\lambda}(\theta))\\
sin(\sqrt{\lambda}(\theta+2\pi)) = sin(\sqrt{\lambda}(\theta))\\
\end{array}
\right\} \iff \lambda = n^2
\end{equation*}
con $n=1,2,3,\hdots$

\noindent Eliminamos las siguientes soluciones dado que $(0,0)\in\mathbb{D}$
\begin{equation*}
\eqref{eq:dirich1}\ R(r) = \left\{
\begin{array}{l l}
1,\color{red}{log(r)} & \lambda=0\\
r^{\sqrt{\lambda}}, \color{red}{r^{-\sqrt{\lambda}}} & \lambda > 0\\
\end{array}
\right.
\end{equation*}
Se tiene entonces que $u$ es combinación lineal de las soluciones anteriores (tomamos la constante como $\frac{a_0}{2}$ por comodidad):
$$u_N(x,y) = \frac{a_0}{2}+\sum_{n=1}^Nr^n(a_ncos(n\theta)+b_nsin(n\theta))$$
La condición de frontera del problema de Dirichlet pide que $u=f$ en $\delta\Omega$. Por tanto
$$U_N(1,\theta) = f(\theta)$$
Si $f$ se puede escribir como combinación lineal de senos y cosenos, entonces el problema tiene solución y como ya se ha visto, dicha solución es única.
Supongamos que
$$f(\theta) = \frac{a_0}{2} + \sum_{n=1}^\infty (a_ncos(n\theta)+b_nsin(n\theta))$$
Se puede observar (ver figura \ref{fig:int-period}) que
\begin{equation*}
\int_0^{2\pi} cos(n\theta)d\theta = \int_0^{2\pi} sin(n\theta)d\theta=0
\end{equation*}
\begin{figure}[ht]
	\centering
	\begin{subfigure}{.5\textwidth}
	\centering
	\begin{tikzpicture}
	    \fill[fill=lavenderblue, samples=400, scale=0.5] (0,0) -- plot [domain=0:2*pi] (\x,{sin(4*(\x r))}) -- (2*pi,0) -- cycle;
    	\draw plot[domain=0:2*pi, samples=400, scale=0.5] (\x,{sin(4*(\x r))});
	    \draw[->] (-0.3,0) -- (pi+0.3,0);
	    \draw[->] (0,-0.3) -- (0,1.3);
	\end{tikzpicture}
	\caption{$sin(n\theta)$}
	\end{subfigure}%
	%
	\begin{subfigure}{.5\textwidth}
	\centering
	\begin{tikzpicture}
	    \fill[fill=lavenderblue, samples=400, scale=0.5] (0,0) -- plot [domain=0:2*pi] (\x,{cos(4*(\x r))}) -- (2*pi,0) -- cycle;
    	\draw plot[domain=0:2*pi, samples=400, scale=0.5] (\x,{cos(4*(\x r))});
	    \draw[->] (-0.3,0) -- (pi+0.3,0);
	    \draw[->] (0,-0.3) -- (0,1.3);
	\end{tikzpicture}
	\caption{$cos(n\theta)$}
	\end{subfigure}
	\caption{Integrales periódicas}
	\label{fig:int-period}
\end{figure}
Luego
$$\int_0^{2\pi} f(\theta)d\theta = \frac{a_0}{2}2\pi+0$$
obteniéndose así
$$\frac{a_0}{2} = \frac{1}{2\pi}\int_0^{\pi}f(\theta)d\theta$$
Ahora vamos a ver los valores que toman $a_n$ y $b_n$ para cada valor de $n$.
\begin{itemize}
\item Valor de $a_n$
\begin{align*}
\int_0^{2\pi}f(\theta)cos(m\theta)d\theta = & \frac{a_0}{2}\underbrace{\int_0^{2\pi}cos(m\theta)}_{=0} +\\
+ & \sum_{n=1}^\infty\left(a_n\underbrace{\int_0^{2\pi}cos(n\theta)cos(m\theta)d\theta}_{=\pi}\right) +\\
+ & \sum_{n=1}^\infty\left(b_n\underbrace{\int_0^{2\pi}sin(n\theta)cos(m\theta)d\theta}_{=0}\right)
\end{align*}
Luego
$$a_m = \frac{1}{\pi}\int_0^{2\pi}f(\theta)cos(m\theta)d\theta$$
\item Valor de $b_n$
\begin{align*}
\int_0^{2\pi}f(\theta)sin(m\theta)d\theta = & \frac{a_0}{2}\underbrace{\int_0^{2\pi}sin(m\theta)}_{=0} +\\
+ & \sum_{n=1}^\infty\left(a_n\underbrace{\int_0^{2\pi}cos(n\theta)sin(m\theta)d\theta}_{=0}\right) +\\
+ & \sum_{n=1}^\infty\left(b_n\underbrace{\int_0^{2\pi}sin(n\theta)sin(m\theta)d\theta}_{=\pi}\right)
\end{align*}
Luego
$$b_m = \frac{1}{\pi}\int_0^{2\pi}f(\theta)sin(m\theta)d\theta$$
\end{itemize}
Tenemos entonces que $u$ es de la forma
\begin{align*}
u(r,\theta) = & \frac{a_0}{2}+\sum_{n=1}^\infty r^n\left(a_ncos(n\theta)+b_n sin(n\theta)\right) =\\
= & \frac{1}{2\pi}\int_0^{2\pi}f(\theta)d\theta+\\
+ & \sum_{n=1}^\infty r^n\left(\frac{1}{\pi}\int_0^{2\pi}f(\varphi)cos(n\varphi)d\varphi cos(n\theta)\right) + \\
+ & \sum_{n=1}^\infty r^n\left(\frac{1}{\pi}\int_0^{2\pi}f(\varphi)sin(n\varphi)d\varphi sin(n\theta)\right) = \\
= & \frac{1}{\pi}\int_0^{2\pi}f(\varphi)\left\{\frac{1}{2}+\sum_{n=1}^\infty r^n\left(cos(n\varphi)cos(n\theta)+sin(n\varphi)sin(n\theta)\right)\right\}d\varphi = \\
= & \frac{1}{\pi}\int_0^{2\pi}f(\varphi)\left\{\frac{1}{2}+\sum_{n=1}^\infty r^n\left(cos(n(\theta-\varphi))\right)\right\}d\varphi 
\end{align*}
En definitiva:
$$u(r,\theta) = \int_0^{2\pi}f(\varphi)P(r,\theta-\varphi)d\varphi$$
donde
$$P(r,\varphi)=\frac{1}{\pi}\left\{\frac{1}{2}+\sum_{n=1}^\infty r^n cos(n\varphi)\right\}$$
Vamos a calcular el valor de $P(r, \varphi)$. Sea $z=re^{i\varphi}$. Se tiene que $z^n=r^ne^{in\varphi}$. Si $|z|<1$, se tiene que
$$\sum_{n=1}^\infty z^n = \frac{z}{1-z}$$
Así que 
$$\sum_{n=1}^\infty z^n + \frac{1}{2} = \frac{1+z}{2(1-z)}$$
Entonces
$$P(r,\varphi) = \frac{1}{2\pi}Re\left(\frac{1+z}{1-z}\right)$$
Vamos a simplificar este valor, multiplicando el numerador y el denominador por el conjugado de $1-z$.
$$\frac{1+z}{1-z}\cdot\frac{1-\overline{z}}{1-\overline{z}} = \frac{1-r^2+(z-\overline{z})}{|1-\overline{z}|^2} = \frac{1-r^2-2iIm(z)}{1+r^2-2Re(z)}$$
De donde se obtiene la parte real de $\frac{1+z}{1-z}$
$$P(r,\varphi) = \frac{1}{2\pi}\cdot\frac{1-r^2}{1-2rcos(\varphi)+r^2}$$
\begin{mathresult}{Solución del problema de Dirichlet}
La solución del problema de Dirichlet en el disco unidad es de la forma $$u(r,\theta) = \int_0^{2\pi}f(\varphi)P(r,\theta-\varphi)d\varphi$$
donde $P(r, \varphi)$ es el núcleo de Poisson.
$$P(r,\varphi) = \frac{1}{2\pi}\cdot\frac{1-r^2}{1-2rcos(\varphi)+r^2}$$
\end{mathresult}
\subsubsection{Núcleo de Poisson}
Como ya se ha visto, el \textbf{núcleo de Poisson} tiene la forma $$P(r,\varphi) = \frac{1}{2\pi}\cdot\frac{1-r^2}{1-2rcos(\varphi)+r^2}$$
Veamos unas pocas propiedades de esta función
\begin{itemize}
\item \textbf{Definición}

\begin{equation*}
\text{El núcleo de Poisson está definido en}
\left\{
\begin{array}{l}
0\le r\le 1\\
\varphi \neq 0
\end{array}
\right.
\text{ver figura \ref{fig:def-poisson}.}
\end{equation*}

\begin{figure}[ht]
\centering
\begin{tikzpicture}[scale=1.5]
	\draw[->] (-1.5,0) -- (1.5,0) node[right] {$x$};
	\draw[->] (0,-1.5) -- (0,1.5) node[above] {$y$};
	\draw [fill, lavenderblue] (0,0) circle [radius=1];
	\draw [] (0,0) circle [radius=1];
	\draw [fill, white] (1,0) circle [radius=0.07];
	\draw [] (1,0) circle [radius=0.07];
	\node (0,0) {$\mathbb{D}$};
\end{tikzpicture}
\caption{Dominio del núcleo de Poisson}
\label{fig:def-poisson}
\end{figure}

\item $P(r, \varphi)$ \textbf{es armónica}
Dado que $u(r,\theta)$ es armónica y 
$$u(r,\theta)=\int_0^{2\pi}f(\varphi)P(r,\theta-\varphi)d\varphi$$
Se tiene que 
$$0 = \Delta u(r,\theta)=\int_0^{2\pi}f(\varphi)\Delta P(r,\theta-\varphi)d\varphi$$
Es decir
$$\Delta P(r,\theta) = 0$$

\item $\int_0^{2\pi}P(r,\varphi)d\varphi = 1$

Basta tomar $f(\theta) = 1$ y comprobar que $u=1$ es solución. Por tanto
$$u(r, \theta) = 1 = \int_0^{2\pi}1\cdot P(r,\theta-\varphi)d\varphi$$

\item $\int_{-\pi}^{\pi}P(r,\varphi)d\varphi = 1$

Dado que las integrales necesarias para calcular lo anterior tienen el mismo valor entre $-\pi$ y $\pi$, basta repetir la prueba anterior.
\end{itemize}

\begin{theorem}

Si $f$ es continua en $\theta_0$, entonces
$$\lim_{(r,\theta)\to(1,\theta_0)}=f(\theta_0)$$
\end{theorem}
\begin{proof}
Vamos a calcular
$$u(r,\theta)-f(\theta_0)$$
Dado que $$\int_{-\pi}^{\pi}P(r,\tilde{\varphi})d\tilde{\varphi} = 1$$
Se tiene que 
$$u(r,\theta)-f(\theta_0)\cdot 1 = u(r,\theta)-f(\theta_0)\int_{-\pi}^{\pi}P(r,\tilde{\varphi})d\tilde{\varphi}$$
Si realizamos el siguiente cambio de variable:
\begin{equation*}
\left.
\begin{array}{l}
\tilde{\varphi} = \theta+\varphi\\
d\tilde{\varphi} = d\varphi
\end{array}
\right\}
\end{equation*}
entonces
$$u(r,\theta)-f(\theta_0)\int_{-\pi}^{\pi}P(r,\tilde{\varphi})d\tilde{\varphi}=\frac{1-r^2}{2\pi}\int_{-\pi}^{\pi}\frac{f(\theta_0+\varphi)-f(\theta_0)}{1-2rcos(\theta-\theta_0-\varphi)+r^2}$$
Podemos separar lo anterior en tres integrales
$$u(r,\theta)-f(\theta_0)=\frac{1-r^2}{2\pi}\left(\int_{-\pi}^{-\delta}\int_{-\delta}^{\delta}\int_{\delta}^{\pi}\right)\left(\frac{f(\theta_0+\varphi)-f(\theta_0)}{1-2rcos(\theta-\theta_0-\varphi)+r^2}\right)$$
La continuidad de $f$ nos dice que dado un $\varepsilon>0, \exists\delta$ tal que
$$|\varphi|<\delta \implies |f(\theta+\varphi)-f(\theta_0)| < \frac{\varepsilon}{2}$$
Vamos a calcular la integral del medio:
\begin{align*}
\frac{1-r^2}{2\pi}\int_{-\delta}^{\delta}\left(\frac{f(\theta_0+\varphi)-f(\theta_0)}{1-2rcos(\theta-\theta_0-\varphi)+r^2}\right) & \le \\
\frac{1-r^2}{2\pi}\int_{-\delta}^{\delta}\left(\frac{\varepsilon/2}{1-2rcos(\theta-\theta_0-\varphi)+r^2}\right) & =
\frac{\varepsilon}{2}\int_{-\delta}^{\delta}\left(P(r,\varphi)d\varphi\right) = \frac{\varepsilon}{2}\\
\end{align*}
Manteniendo el valor de $\delta$ fijo, si $|\varphi| > \delta$ y $|\theta-\theta_0|<\frac{\delta}{2}$, se tiene que
$$|\theta-\theta_0-\varphi| \ge |\varphi|-|\theta-\theta_0|>\delta-\frac{\delta}{2} = \frac{\delta}{2}$$
Calculamos las otras dos integrales, tomando $M=\max_{\delta\mathbb{D}}f$
\begin{align*}
\frac{1-r^2}{2\pi}\left(\int_{-\pi}^{-\delta}\int_{-\delta}^{\delta}\int_{\delta}^{\pi}\right)\left(\frac{f(\theta_0+\varphi)-f(\theta_0)}{1-2rcos(\theta-\theta_0-\varphi)+r^2}\right)\le \\
\le \frac{2M}{2\pi}\cdot\frac{1-r^2}{1-2rcos(\theta-\theta_0-\varphi)+r^2}\cdot 2(\pi-\delta)\to 0
\end{align*}
si $r\to 1$.
Tenemos por tanto que
$$u(r,\theta)-f(\theta_0) \le \frac{\varepsilon}{2} + \frac{\varepsilon}{2} = \varepsilon$$
\end{proof}