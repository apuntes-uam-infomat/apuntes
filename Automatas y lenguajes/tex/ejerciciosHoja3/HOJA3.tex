\documentclass[nochap]{apuntes}
\title{Autómatas y Lenguajes: Hoja 1}
\author{Pedro Valero}
\date{}

% Paquetes adicionales
\usepackage{tikztools}
\usepackage{fastbuild}
\usetikzlibrary{arrows}

\begin{document}
\pagestyle{plain}

%%%%%%%%%%%%%%%%%%%%%%%%%%%%%%%%%%%%%%%%%%%%%%%
%%%
%%% 		Problema 1
%%%
%%%%%%%%%%%%%%%%%%%%%%%%%%%%%%%%%%%%%%%%%%%%%%%
\begin{problem}
Sea la siguiente gramática:
\begin{itemize}
\item S ::= A
\item S ::= B
\item A ::= cA+b
\item A ::= a
\item B ::= cB+a
\item B :== b
\end{itemize}
 Calcula los conjuntos primero y siguiente para cada símbolo no terminal
\solution
\textbf{Primeros:}

\begin{itemize}
\item Primero(A) =
\item Primero(B) =
\item Primero(C) =
\item Primero(D) =
\end{itemize}

\textbf{Sigueintes}

\begin{itemize}
\item siguiente(A) =
\item siguiente(B) =
\item siguiente(C) =
\item siguiente(D) =
\end{itemize}
\end{problem}

%%%%%%%%%%%%%%%%%%%%%%%%%%%%%%%%%%%%%%%%%%%%%%%
%%%
%%% 		Problema 2
%%%
%%%%%%%%%%%%%%%%%%%%%%%%%%%%%%%%%%%%%%%%%%%%%%%
\begin{problem}
Sea la siguiente gramática LR(0):
\begin{itemize}
\item E ::= T
\item E ::= E + T
\item T ::= i
\item T ::= (E)
\end{itemize}
Calcula el cierre de la configuración inicial: E' ::= .ES
\solution
TODO
\end{problem}

%%%%%%%%%%%%%%%%%%%%%%%%%%%%%%%%%%%%%%%%%%%%%%%
%%%
%%% 		Problema 3
%%%
%%%%%%%%%%%%%%%%%%%%%%%%%%%%%%%%%%%%%%%%%%%%%%%
\begin{problem}
Sea la siguiente gramática LR(0):
\begin{itemize}
\item E ::= T
\item E ::= E + T
\item T ::= i
\item T ::= (E)
\end{itemize}

\ppart Calcula el cierre de la configuración E ::= (.L)

\ppart Calcula el estado al que se llega desde el estado anterior tras conocer el símbolo no terminal L.
\solution
TODO
\end{problem}

%%%%%%%%%%%%%%%%%%%%%%%%%%%%%%%%%%%%%%%%%%%%%%%
%%%
%%% 		Problema 4
%%%
%%%%%%%%%%%%%%%%%%%%%%%%%%%%%%%%%%%%%%%%%%%%%%%
\begin{problem}
Sea la siguiente gramática

\begin{itemize}
\item D ::= iPSn
\item P ::= :n
\item S ::= λ
\item S ::= n
\end{itemize}

\ppart Dibuja el diagrama de estados del analizador LR(0) para dicha gramática
\ppart Calcula la tabla de análisis para el analizador LR(0)
\ppart Indica justificadamente si la gramática es LR(0). Indica justificadamente si es SLR(1)
\solution
TODO
\end{problem}

%%%%%%%%%%%%%%%%%%%%%%%%%%%%%%%%%%%%%%%%%%%%%%%
%%%
%%% 		Problema 5
%%%
%%%%%%%%%%%%%%%%%%%%%%%%%%%%%%%%%%%%%%%%%%%%%%%
\begin{problem}
Sea la siguiente gramática:
\begin{itemize}
\item S ::= bLd
\item L ::= E;L
\item L ::= λ
\item E ::= i=c
\item E ::= b
\end{itemize}
\ppart Dibuja el diagrama de estados del analizador LR(0) para dicha gramática
\ppart Calcula la tabla de análisis para el analizador LR(0)
\ppart Indica justificadamente si es SLR(1)

\solution
TODO
\end{problem}

%%%%%%%%%%%%%%%%%%%%%%%%%%%%%%%%%%%%%%%%%%%%%%%
%%%
%%% 		Problema 6
%%%
%%%%%%%%%%%%%%%%%%%%%%%%%%%%%%%%%%%%%%%%%%%%%%%
\begin{problem}
Calcula los conjuntos primero y siguiente de todos los símbolos no terminales de las gramáticas siguientes

\ppart
\begin{itemize}
\item X ::= Ye
\item X ::= eYZf
\item Y ::= g
\item Y ::= Yg
\item Z ::= h
\end{itemize}

\ppart
\begin{itemize}
\item Q ::= fXY
\item X ::= cQ
\item X ::= λ
\item Y ::= iQ
\item Y ::= λ
\end{itemize}

\ppart
\begin{itemize}
\item A ::= BXB
\item X ::= .
\item X ::= .
\item X ::= e
\item B ::= 0B
\item B ::= 1B
\item B ::= λ
\end{itemize}

\solution
TODO
\end{problem}

%%%%%%%%%%%%%%%%%%%%%%%%%%%%%%%%%%%%%%%%%%%%%%%
%%%
%%% 		Problema 7
%%%
%%%%%%%%%%%%%%%%%%%%%%%%%%%%%%%%%%%%%%%%%%%%%%%
\begin{problem}
Cacula los símbolos de adelanto para el cierre de las siguientes reglas y gramáticas:
\ppart Cierre de E' ::= .E $\{ \$ \}$ para la siguiente gramática
\begin{itemize}
\item E' ::= E
\item E ::= T
\item E ::= E+T
\item T ::= i
\item T ::= (E)
\end{itemize}

\ppart Cierre de S' ::= .s $\{ \$ \}$ para la siguiente gramática
\begin{itemize}
\item S' ::= S
\item S ::= L=R
\item S ::= R
\item L ::= *R
\item L ::= i
\item R ::= L
\end{itemize}

\ppart Cierre de E ::= (.E) $\{ \$ \}$ para la siguiente gramática
\begin{itemize}
\item E ::= (L)
\item E ::= a
\item L ::= L,E
\item L ::= E
\end{itemize}
\solution
TODO
\end{problem}

%%%%%%%%%%%%%%%%%%%%%%%%%%%%%%%%%%%%%%%%%%%%%%%
%%%
%%% 		Problema 8
%%%
%%%%%%%%%%%%%%%%%%%%%%%%%%%%%%%%%%%%%%%%%%%%%%%
\begin{problem}
Sea la siguiente gramática independiente del contexto:
\begin{itemize}
\item S ::= aSb
\item S ::= ab
\end{itemize}
\ppart Dibuja el diagrama de estados del analizador LR(1) para dicha gramática
\ppart Calcula la tabla de análisis para el analizador LR(1)
\ppart Usa la tabla de análisis para analizar la sentencia $aabb$
\ppart Dibuja esquemáticamente el diagrama de estados LALR(1). Es suficiente con indicar los nombres de los estados, especificando cuáles son la unión de otros en LR(1).
\solution
TODO
\end{problem}

%%%%%%%%%%%%%%%%%%%%%%%%%%%%%%%%%%%%%%%%%%%%%%%
%%%
%%% 		Problema 9
%%%
%%%%%%%%%%%%%%%%%%%%%%%%%%%%%%%%%%%%%%%%%%%%%%%
\begin{problem}
Sea la siguiente gramática independiente del contexto:
\begin{itemize}
\item S ::= XX
\item X ::= aX
\item X ::= b
\end{itemize}
\ppart Dibuja el diagrama de estados del analizador LR(1) para dicha gramática
\ppart Calcula la tabla de análisis para el analizador LR(1)
\ppart Usa la tabla de análisis para analizar la sentencia $aabb$
\ppart Dibuja esquemáticamente el diagrama de estados LALR(1). Es suficiente con indicar los nombres de los estados, especificando cuáles son la unión de otros en LR(1).
\solution
TODO
\end{problem}

%%%%%%%%%%%%%%%%%%%%%%%%%%%%%%%%%%%%%%%%%%%%%%%
%%%
%%% 		Problema 10
%%%
%%%%%%%%%%%%%%%%%%%%%%%%%%%%%%%%%%%%%%%%%%%%%%%
\begin{problem}
Sea la siguiente gramática

\begin{itemize}
\item D ::= iPSn
\item P ::= :n
\item S ::= λ
\item S ::= n
\end{itemize}

\ppart Dibuja el diagrama de estados del analizador LR(1) para dicha gramática
\ppart Calcula la tabla de análisis para el analizador LR(1)
\ppart Indica justificadamente si la gramática es LR(1). Indica justificadamente si es SLR(1)
\solution
TODO
\end{problem}


\end{document}
