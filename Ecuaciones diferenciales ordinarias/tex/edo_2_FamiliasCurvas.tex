\section{Familias de curvas}

Sabemos que, dada una EDO de primer orden de la forma $F(x,y(x)) = y'$, el valor de la función $F$ nos indica la pendiente de la recta tangente en cada punto $(x, y(x))$.
Veamos algunos ejemplos:

\img{img/slope1.png}{Campo de pendientes $y^\prime = y$}{slope1}{0.55}

\begin{example}
Sea la EDO $y^\prime = y$\\
Vemos que en las rectas de la forma $y=c$ tenemos que $y=y^\prime = c$, por tanto la recta tangente a la solución en todos los puntos de la recta $y=c$ tiene pendiente $c$.

Podemos entonces calcular el campo de pendientes de la EDO (Ver \textbf{Figura \ref{fig:slope1}}).

Tomando un punto de partida, la curva solución tiene como recta tangente en cada punto las mostradas en el campo de pendientes.

En este caso sabemos que la solución a esa ecuación es de la forma $ae^{x+b}$ con $a$ y $b$ constantes.

Tomando un punto en el eje de abscisas, la recta tangente tiene la dirección del eje de abscisas, por tanto la solución partiendo de un punto del eje $x$ es el propio eje. Vemos que si escogemos un punto que no pertenezca a dicho eje es imposible llegar a ``tocar'' el eje, pues en caso contrario no existiría unicidad en la solución.
\end{example}

\img{img/slope2.png}{Campo de pendientes $y^\prime=x+y$}{slope2}{0.55}

\begin{example}
Analicemos la EDO $y^\prime = x+y$\\
En este caso, en las rectas de la forma $x+y=c$ la recta tangente a la curva solución tiene pendiente $c$.
En la \textbf{Figura \ref{fig:slope2}} se muestra el campo de pendientes de esta EDO.

Para hallar la solución a la ecuación observamos que $(x+y)^\prime = x+y+1$. Si denotamos $t=x+y$ tenemos la ecuación $t^\prime=t+1$ y despejando $\frac{t^\prime}{t+1} = 1$

Tras integrar ambos términos obtendremos la solución.
Para obtener todas las soluciones no hay que olvidar que $\int \frac{1}{x} = ln\abs{x}$
\end{example}

\img{img/slope3.png}{Campo de pendientes $y^\prime = x^2+y$}{slope3}{0.55}

\begin{example}
Sea la EDO $y^\prime = x^2+y$\\
Las curvas en las que la pendiente de la recta tangente a la solución se mantiene constante son de la forma $x^2+y=c$, es decir, parábolas. En la \textbf{Figura \ref{fig:slope3}} puede observarse el campo de pendientes de la EDO.

Nos gustaría estudiar dónde están los puntos de inflexión de las soluciones. Sabemos que los puntos de inflexión aparecen cuando $y^{\prime\prime} = 0$
Vemos entonces que
$$y^{\prime\prime} =  2x-y^\prime = 2x-x^2+y \implies y^{\prime\prime} = 0 \iff y = x^2-2x=x(x-2)$$
Los puntos de inflexión están en la parábola $y=x(x-2)$.
Veamos cómo solucionar esta EDO.

$$y^\prime + y = x^2$$
$$e^xy^\prime + e^xy = e^xx^2$$
$$(e^xy)^\prime = e^xx^2$$
$$e^xy = \int x^2e^xdx + C$$
$$y = e^{-x}(\int x^2e^xdx+C)$$
\end{example}

En estos ejemplos hemos visto curvas que donde la pendiente de la recta tangente a las soluciones es constante. Esto nos lleva a la siguiente definición:

\begin{definition}\name{Isoclinas}
Curvas en las que la pendiente de la recta tangente a las soluciones es constante.
En general, para una EDO de orden 1, las isoclinas vienen dadas por $F(x,y) = c$
\end{definition}

Veamos ahora un ejemplo en el que no hay unicidad de soluciones:


\img{img/slope4.png}{Campo de pendientes $y^\prime = \sqrt{1-y^2}$}{slope4}{0.55}
\begin{example}
Sea la EDO $y^\prime = \sqrt{1-y^2}$ y el dato $y(x_0)=C$.

Vemos que si $\abs{C} \gt 1 \implies$ No hay solucion. Si $C = 1 \implies y = 1$, si $C = -1 \implies y = -1$. En la \textbf{Figura \ref{fig:slope4}} se puede observar el campo de pendientes de la EDO.

Vemos que $y=sin(x)$ es solución porque
$$\derivative{x}sin(x)=cos(x)=\sqrt{1-sin^2(x)}$$
pero sólo es válida esta solución si $x\in[\frac{-\pi}{2}, \frac{\pi}{2}]$.

Vemos que no hay unicidad de solución, tomando el dato $y(\frac{-\pi}{2}) = -1:$
\begin{itemize}
\item \textit{Sol1: } $y=-1$
\item \textit{Sol2: }
$
  y=
  \left\lbrace
  \begin{array}{l}
     -1 \text{ si } x \lt \frac{-\pi}{2} \\
     sin(x) \text{ si } x\in[\frac{-\pi}{2}, \frac{\pi}{2}] \\
     1 \text{ si } x \gt \frac{-\pi}{2} \\
  \end{array}
  \right.
$

Podemos observar que la derivada de la raíz cuadrada se ``va a infinito'' en el $0$. Podríamos pensar que por esta razón no tenemos asegurada la unicidad de la solución.
\end{itemize}
\end{example}

Hasta ahora hemos visto que dada una EDO, el conjunto de soluciones nos proporciona una familia de curvas.
Vamos a analizar el siguiente problema:
\begin{itemize}
\item Dada una familia de curvas uniparamétrica, encontrar la ecuación diferencial ordinaria que satisfacen.
\end{itemize}

Veamos unos ejemplos:

\img{img/familia-circunferencia1.png}{Familia de curvas $x^2+y^2=R^2$}{familia-circunferencia1}{0.55}
\begin{example}
Tenemos la familia de curvas $x^2+y^2=R^2$, (Ver \textbf{Figura \ref{fig:familia-circunferencia1}}).

Derivando obtenemos $2x+2yy^\prime = 0$ y simplificando $x+yy^\prime = 0$.
\end{example}

\img{img/familia-circunferencia2.png}{Familia de curvas $x^2+y^2=2Cx$}{familia-circunferencia2}{0.55}
\begin{example}
Tenemos la familia de curvas $x^2+y^2=2Cx$, (Ver \textbf{Figura \ref{fig:familia-circunferencia2}}).

Derivando obtenemos $2x+2yy^\prime = 2C \iff x+yy^\prime = C$

Tenemos el sistema
$
  \left\lbrace
  \begin{array}{l}
  	 x+yy^\prime = C \\
     x^2+y^2=2Cx \\
  \end{array}
  \right.
$

Sustituyendo $C$ en la segunda ecuación tenemos $x^2+y^2 = 2(x + yy^\prime)x$.
\end{example}

\img{img/familia-parabola.png}{Familia de rectas tangentes a $f(x)=\frac{x^2}{4}$}{familia-parabola}{0.55}
\begin{example}
Vamos a obtener la familia de rectas tangentes a la parábola $f(x)=\frac{x^2}{4}$, (Ver \textbf{Figura \ref{fig:familia-parabola}}).

Tenemos que $f^\prime(x) =  \frac{x}{2}$ y por tanto, la recta tangente en un punto $a$ es $y=f(a)+f^\prime(a)(x-a)$. Por tanto tenemos que la familia de rectas tangentes a la parábola viene dada por $$y-\frac{a}{2}x + \frac{a^2}{4} = 0$$
Hemos construido una familia de rectas tangentes a una parábola. Se dice que la parábola es la \textbf{envolvente} de la familia de rectas obtenida.
\end{example}

\subsection{Envolvente de una familia de curvas}

\begin{definition}\name{Envolvente}
Una curva envolvente es aquella que ``toca'' a todas las curvas de una familia y es tangente en los puntos de contacto.
\end{definition}

\img{img/envolvente.png}{Envolvente de una familia de curvas}{envolvente}{0.55}

Vamos a analizar como calcular la curva envolvente a una familia de curvas.
Como ya hemos visto, en general una familia de curvas viene dada por una expresión de la forma $F(x,y,c) = 0$. Llamemos $Q=(x_q, y_q)$ al punto de contacto entre la curva y su envolvente.

En la \textbf{Figura \ref{fig:envolvente}} podemos observar que sumando una pequeña perturbación $\delta$ a la constante $c$ obtenemos otra curva de la familia, la cual interseca con la anterior en un punto $P_\delta = (x_\delta, y_\delta)$. Notamos que $(x_\delta, y_\delta) \to (x_q, y_q)$ cuando $\delta \to 0$.

Tenemos entonces que
$
  \left\lbrace
  \begin{array}{l}
     F(x_\delta, y_\delta, c) = 0 \\
     F(x_\delta, y_\delta, c+\delta) = 0  \\
  \end{array}
  \right.
$

Restando ambas ecuaciones tenemos $F(x_\delta, y_\delta, c+\delta) - F(x_\delta, y_\delta, c) = 0$. Dividiendo por $\delta$ y tomando límites tenemos que \[ \lim_{\delta\to 0} \frac{F(x_\delta, y_\delta, c+\delta) - F(x_\delta, y_\delta, c)}{\delta}=0 \] obteniendo así que $\derivative{c}F(x,y,c) = 0$.

Una vez hecho esto obtenemos un método para hallar la curva envolvente a una familia de curvas:

\begin{method}[para hallar la envolvente]
Para hallar la envolvente a una familia de curvas basta con eliminar $c$ del sistema de ecuaciones
\[
  \left\lbrace
  \begin{array}{l}
     F(x, y, c) = 0 \\
     \derivative{c}F(x, y, c) = 0  \\
  \end{array}
  \right.
\]
donde $F(x,y,c) = 0$ define la familia de curvas de la cual queremos hallar la envolvente.
\end{method}

Pongamos en práctica lo aprendido con un ejemplo:

\img{img/canon.png}{Envolvente a una familia de parábolas}{canon}{0.55}

\begin{example}
Supongamos que tenemos un cañón antiaéreo en el origen de coordenadas que dispara un proyectil con una velocidad inicial $V$. El ángulo $\alpha$ en el que dispara el cañón es variable. Sabemos que la curva que describe el proyectil es una parábola. El objetivo de este problema es hallar la zona en la que un avión podría volar sin ser alcanzado por un proyectil. Es sencillo darse cuenta de que la zona de peligro viene descrita por la que queda bajo la curva envolvente a la familia de parábolas que pueden describir los proyectiles lanzados (Ver \textbf{Figura \ref{fig:canon}}).

En primer lugar hallaremos la familia de curvas, tenemos en principio:
\begin{itemize}
\item Movimiento horizontal: $x(t) = V\cos(\alpha)t$
\item Movimiento vertical: $y(t) = V\sin(\alpha)t-\dfrac{g}{2}t^2$, donde $t$ es el tiempo y $g$ es la aceleración de la gravedad.
\item $y(t_{max}) = 0 \implies t_{max} = \frac{2Vsin(\alpha)}{g}$
\item Obtenemos así la ecuación de la \textbf{posición}:

$$\sigma(t) = (x(t), y(t)) = (Vcos(\alpha)t, t(Vsin(\alpha)-\frac{g}{2}t)$$

donde $t\in[0, t_{max}]$
\item Despejando t e igualando términos se obtiene la ecuación de la \textbf{trayectoria}:

$$tan(\alpha)x-\frac{g}{2V^2cos^2(\alpha)}x^2-y = 0$$
Usando que $1+tan^2(\alpha) = \frac{1}{cos^2(\alpha)}$ simplificamos la ecuación anterior:

$$tan(\alpha)x - \frac{g(tan^2(\alpha)+1)}{2V^2}x^2-y = 0$$
\item Hemos obtenido la familia de parábolas $F(x,y,\alpha) = 0$ donde
$$F(x,y, \alpha) = tan(\alpha)x - \frac{g(tan^2(\alpha)+1)}{2V^2}x^2-y$$
\item Para facilitar los cálculos llamamos $c=tan(\alpha)$ y calculamos
$$\derivative{c}F(x,y,c) = x-\frac{g}{V^2}cx^2$$
\item Notamos que $\derivative{c}F(x,y,c) = 0 \iff c=\frac{V^2}{gx}$
\item Sustituyendo y simplificando llegamos a que la envolvente a nuestra familia de parábolas es
$$y=\frac{V^2}{2g}-\frac{g}{2V^2}x^2$$
\end{itemize}
\end{example}

\subsection{Familias de curvas ortogonales}
Dada una familia de curvas $A$, buscamos otra familia $B$ tal que si una curva de $A$ interseca con una curva de $B$, en el punto de intersección son ortogonales.

Motivación:
\begin{itemize}
\item Sea $S\equiv$ Superficie dada por una gráfica $z=g(x,y)$.

La trayectoria de una gota de agua que se desliza sobre S $\equiv$ Familia ortogonal a los conjuntos de nivel de $g$.
\end{itemize}

\img{img/familias-ortogonales.png}{Curvas ortogonales}{familias-ortogonales}{0.7}

Sea $\alpha(x) = (x, y(x))$ una curva de $A$, tenemos que el vector tangente es $\alpha^\prime(x) = (1, y^\prime)$. Sea $\beta(x) = (x, \tilde{y})$ una curva de $B$, tenemos que el vector tangente es $\beta^\prime(x) = (1, \tilde{y}^\prime(x))$. (Ver \textbf{Figura \ref{fig:familias-ortogonales}})

Como condición de ortogonalidad tenemos $\dotproduct{\alpha^\prime(x)}{\beta^\prime(x)}=0$ de donde obtenemos que $\tilde{y}^\prime = \frac{-1}{y^\prime}$

Si la familia $A$ viene dada por la EDO $y^\prime(x) = f(x,y)$, la familia $B$ vendrá dada por la EDO $\tilde{y}^\prime = \frac{-1}{y^\prime(x)} = \frac{-1}{f(x,y(x))}$. Como en el punto de intersección $y(x) = \tilde{y}(x)$ tenemos que la EDO que define a la familia $B$ es $$\frac{-1}{\tilde{y}^\prime(x)} = f(x, \tilde{y}(x))$$

De aquí obtenemos un método para obtener la familia de curvas ortogonales a una familia definida como $f(x,y,c) = 0$.

\begin{method}[para hallar la familia ortogonal (coordenadas cartesianas)]
Dada la familia $A$ de curvas definida por $f(x,y,c) = 0$
\begin{itemize}
\item Hallar la EDO que define a la familia, que tendrá la forma $y^\prime(x) = g(x,y)$.
\item Calcular la EDO que define la familia $B$ de curvas ortogonales, que tendrá la forma $$\tilde{y}^\prime(x) = \frac{-1}{g(x, \tilde{y}(x))}$$
\item Para ello basta con sustituir $y^\prime$ en la EDO de $A$ por $\frac{-1}{\tilde{y}^\prime}$ e $y$ por $\tilde{y}$. Obteniendo así la EDO de $B$.
\item Resolver la EDO que define a $B$ obteniendo así la expresión para la familia de curvas ortogonales.
\end{itemize}
\end{method}

Pongamos el método en práctica con un par de ejemplos:

\img{img/gota-agua.png}{Familia de curvas $A = \set{xy = c \st c\in R}$ y su familia ortogonal}{gota-agua}{0.55}

\begin{example}
Sea la familia de curvas $A = \set{xy = c \st c\in \R}$ (Ver \textbf{Figura \ref{fig:gota-agua}}).
Derivando implícitamente obtenemos la EDO asociada a $A$:
$$y^\prime(x) = \frac{-y(x)}{x}$$
Calculamos la EDO asociada a la familia ortogonal:
$$\tilde{y}^\prime=(\frac{x}{\tilde{y}})$$
Despejando términos tenemos: $\tilde{y}\tilde{y}^\prime = x$ y resolviendo la ecuación:
$$\frac{\tilde{y}^2}{2}-\frac{x^2}{2} = C$$
que es la expresión para la familia ortogonal a $A$.
\end{example}

Veamos el segundo ejemplo:

\img{img/auto-ortogonal.png}{Familia de curvas auto-ortogonal}{auto-ortogonal}{0.55}

\begin{example}
Tenemos $A=\set{y^2-cx=\frac{c^2}{4} \st c\in \R}$.
Obtenemos la EDO correspondiente derivando implícitamente, tenemos el sistema:

\begin{center}
$
  \left\lbrace
  \begin{array}{l}
     2yy^\prime = c \\
     y^2-cx = \frac{c^2}{4} \\
  \end{array}
  \right.
$
$\implies y-2y^\prime x = y(y^\prime)^2$
\end{center}
Obtenemos la EDO de la familia ortogonal: $\tilde{y}-2(\frac{-1}{\tilde{y}^\prime})x=\tilde{y}(\frac{-1}{\tilde{y}^\prime})^2$.

Simplificando: $$\tilde{y}-2\tilde{y}^\prime x = \tilde{y}(\tilde{y}^\prime)^2$$ que es igual que la ecuación de la familia $A$.

Como resultado obtenemos que la familia ortogonal a $A$ es ella misma. Es decir, si dos curvas de la familia $A$ se intersecan, serán ortogonales. Esto es lo que se conoce como \textbf{familia auto-ortogonal}\index{Familia de curvas!auto-ortogonal}. (Ver \textbf{Figura \ref{fig:auto-ortogonal}}).
\end{example}

En el caso de que nos proporcionen una familia $A$ de curvas en coordenadas polares, procederemos del mismo modo para encontrar la familia  $B$ ortogonal a $A$.

Dada una curva de $A$ de la forma $\alpha(\theta) = (r(\theta)cos(\theta), r(\theta)sin(\theta))$. Llamaremos $\beta(\theta) = (\tilde{r}(\theta)cos(\theta), \tilde{r}(\theta)sin(\theta))$ a una curva de la familia $B$ que interseque con $\alpha$.

Como condición de ortogonalidad tenemos que $\dotproduct{\alpha^\prime(\theta)}{\beta^\prime(\theta)} = 0$.
Tras unas pocas operaciones llegamos a que $$\dotproduct{\alpha^\prime(\theta)}{\beta^\prime(\theta)} = 0 \iff r^\prime \tilde{r}^\prime = -r\tilde{r}$$ y como en la intersección de $\alpha$ y $\beta$ sabemos que $r = \tilde{r}$ concluimos con que $$\dotproduct{\alpha^\prime(\theta)}{\beta^\prime(\theta)} = 0 \iff \tilde{r}^\prime = \frac{-r^2}{r^\prime}$$ de donde obtenemos el siguiente método para obtener la familia de curvas ortogonal a una familia de curvas dada en coordenadas polares:

\begin{method}[para hallar la familia ortogonal (coordenadas polares)]
Sea la familia $A$ de curvas dada por $\alpha(\theta, c)$
\begin{itemize}
\item Hallar la EDO que define a la familia $A$.
\item Calcular la EDO que define la familia $B$ de curvas ortogonales, que tendrá la forma $$ \tilde{r}^\prime = \frac{-r^2}{r^\prime}$$
\item Basta con sustituir $r^\prime$ en la EDO de $A$ por $\frac{-\tilde{r}^2}{\tilde{r}^\prime}$ y $r$ por $\tilde{r}$. Obteniendo así la EDO de la familia ortogonal $B$.
\item Resolver la EDO que define a $B$ obteniendo así la expresión para la familia de curvas ortogonales.
\end{itemize}
\end{method}

Pongamos en práctica el método con un ejemplo:

\img{img/cardioide.png}{Familia de cardioides y su familia ortogonal}{cardioide}{0.55}

\begin{example}
Sea $A = \set{r=c(1+cos(\theta)) \st c \in \R^+}$ una familia de cardioides.

Derivando implícitamente obtenemos la EDO correspondiente, tenemos el sistema:
\begin{center}
$
\left\lbrace
  \begin{array}{l}
     c = \frac{-r^\prime}{sin(\theta)} \\
     r = c(1+cos(\theta))  \\
  \end{array}
\right.
$
$\implies sin(\theta)r = -r^\prime(1+cos(\theta))$
\end{center}
Hallamos ahora la EDO de la familia ortogonal:
$$\tilde{r}^\prime = \frac{1+cos(\theta)}{sin(\theta)}\tilde{r}$$
$$\frac{\tilde{r}^\prime}{\tilde{r}} = \frac{1+cos(\theta)}{sin(\theta)} = \frac{(1+cos(\theta))(1-cos^2(\theta))}{sin(\theta)(1-cos^2(\theta))} = \frac{sin(\theta)}{1-cos(\theta)}$$

Integrando ambos términos llegamos a la solución:
$$r^2 = e^C(1-cos(\theta))$$
que es la familia de cardioides simétricas a las cardioides de la familia $A$ respecto del eje de ordenadas. (Ver \textbf{Figura \ref{fig:cardioide}}).
\end{example}

