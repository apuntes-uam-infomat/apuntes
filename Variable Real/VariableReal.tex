\documentclass[nochap,palatino]{apuntes}

\usepackage{fancysprefs}

\title{Variable Real}
\author{Guillermo Julián Moreno}
\date{15/16 C1}

% Paquetes adicionales

% --------------------

\begin{document}
\pagestyle{plain}
\maketitle

\tableofcontents
\newpage
% Contenido.

\section{Introducción. Repaso de teoría de la integral y la medida}

La teoría moderna de la integración sigue lo que se dice los tres principios de J.E. Littlewood (1885 - 1977), que dan de cierta forma la idea detrás de la integración según Lebesgue.

\begin{enumerate}
\item Cada conjunto (medible) es \textit{casi} una suma finita de intervalos.
\item Cada función absolutamente integrable es \textit{casi} continua.
\item Cada sucesión convergente puntualmente de funciones absolutamente integrable es \textit{casi} uniformemente convergente.
\end{enumerate}

Por ejemplo, uno de los problemas que se pretende resolver con la integral de Lebesgue es el de la convergencia de funciones. Con la integral de Riemann, para que se pueda intercambiar el límite con la integral (esto es, que $\int \lim f_n = \lim \int f_n$), la hipótesis necesaria es que $f_n$ converja uniformemente. La integral de Lebesgue si nos permitirá dar una noción más completa de convergencia (tercer principio) que nos permita intercambiar fácilmente la integral con el límite.

La teoría de Lebesgue también nos permite saltarnos problemas como el planteado por la \concept[Función\IS de Dirichlet]{función\IS de Dirichlet}, dada por \[ \mathbb{D}(x) = \begin{cases} 1 & x ∈ ℚ \\ 0 & x ∈ ℝ \setminus ℚ \end{cases} \]

$\mathbb{D}$ no es continua y por lo tanto no es integrable Riemann. La cuestión es que es \textit{casi} continua (segundo principio), así que con Lebesgue podemos quitarnos los problemas que aparecen en los puntos $x ∈ ℚ$ e integrarla sin problemas.

En resumen, en todos los casos lo que tendremos es que, para todo ε, existe un conjunto de puntos de medida ε (los ``puntos malos'') de tal forma que nos lo podemos quitar y tendremos una función convergente, continua o integrable.

Este curso consistirá en la búsqueda y manejo de esos ``puntos malos'': veremos si podremos integrar en ellos, derivarlos y qué ocurre con las funciones en esos puntos.

Para trabajar con estas ideas desarrollaremos el concepto de las medidas que, por así decirlo, serán ``funciones generalizadas''. Por ejemplo, la \concept[Delta\IS de Dirac]{delta\IS de Dirac}, una ``función'' que vale cero en todo punto menos en $x = 0$, donde tiene un valor no concreto, se define en realidad como una medida dada por \[ δ_0 (A) = \begin{cases} 1 & 0 ∈ A \\ 0 & 0 ∉ A \end{cases} \] de tal forma que $\int_{ℝ} f(x) \dif δ = f(0)$.

\subsection{Conjuntos, espacios y funciones medibles}

A lo largo del curso trabajaremos con varias convenciones. Trabajaremos sobre la tres-tupla $(X, \mathcal{X}, μ)$, lo que llamaremos un espacio de medida. $X$ será el conjunto sobre el que mediremos. Normalmente será un subconjunto de $ℝ^n$, pero para el caso nos da igual. $\mathcal{X}$ será la σ-álgebra, el conjunto de subconjuntos medibles de $X$\footnote{Ver apuntes de TIM de Pedro.}. Por último, μ será la medida que usemos para integrar, usando la integral de Lebesgue.

¿Cuál es la diferencia entre las integrales de Riemann y de Lebesgue? Lo que se suele decir es que el primero mide en vertical y el segundo en horizontal. Una analogía es verlo como contar billetes. Por ejemplo, tenemos un billete de 50, uno de 20, dos de 10 y dos de 5. Riemann los contaría según le salen de la cartera, mientras que Lebesgue los agruparía primero en conjuntos de billetes iguales y luego contaría.

Mirando funciones, lo que hace Riemann es montar rectángulos en el eje $X$, mientras que Lebesgue los monta en el eje $Y$. De esta forma, la integral de Lebesgue es más estable, mientras que la de Riemann ve mejor las oscilaciones, por así decirlo. Un ejemplo curioso es el de la siguiente integral, \( \int_0^∞ \frac{\sin x}{x} \dif x \label{eq:IntRiemmanNoLeb} \) que es Riemann integrable pero no Lebesgue porque el valor absoluto no es integrable.

Para seguir trabajando con las medidas y poder integrar, necesitaremos saber qué funciones podemos medir y cuáles no.

\begin{defn}[Función\IS medible] Dados dos espacios de medida $(X, \mathcal{X}, μ),\,(Y, \mathcal{Y},ν)$, se dice que una función \stdf es medible si y sólo si $\inv{f}(E) ∈ \mathcal{X}\; ∀E ∈ \mathcal{Y}$.
\end{defn}

También necesitaremos saber en qué σ-álgebra estamos trabajando. Nosotros usaremos normalmente la σ-álgebra de Borel:

\begin{defn}[{σ}-álgebra\IS de Borel] Denotada por \borel, es la σ-álgebra más que pequeña tal que los abiertos son medibles.\end{defn}

Otra $\salgb$ importante es la \concept{{σ}-álgebra\IS de Lebesgue}, denotada por $\lebg (ℝ^n)$. La relación entre esta y la de Borel es que la de Borel está contenida en la de Lebesgue: $\borel \subsetneq \lebg $. De hecho, $\lebg  \setminus \borel = \mathcal{N}$, que es el conjunto de conjuntos de medida Lebesgue 0. Por ejemplo, las uniones numerables tienen medida cero, como los racionales. Este es el gran avance de Lebesgue: identificar los conjuntos de medida 0 para poder integrar más funciones.

Algo que nos interesará es ver la composición de funciones.

\begin{prop}

\begin{enumerate}
\item La composición de funciones \borel-medibles es \borel-medible.
\item La composición de funciones \lebg-medibles no es, en general, \lebg-medible.
\item La composición de funciones \lebg-medibles con una de ellas continua sí es \lebg-medible.
\end{enumerate}
\end{prop}

\subsection{Medidas positivas}

\begin{defn}[Medida\IS positiva]\footnote{Positivas implicará a partir de ahora mayor \textit{o igual} que cero.} Una medida positiva (no negativa) en un espacio medible \meds es una función
\begin{align*}
\appl{μ}{\mathcal{X}&}{[0,+∞)} \\
E &\longmapsto μ(E) ≥ 0
\end{align*} y tal que
\begin{enumerate}
\item $μ(∅) = 0$.
\item $μ\left(\bigcup_{n=1}^∞ E_n\right) = \sum_{n=1}^∞ μ(E_n)$ con $E_n$ disjuntos dos a dos.
\end{enumerate}
\end{defn}

Hay dos nombres cocretos que nos interesarán. Si $μ(X) < ∞$, entonces se dice que μ es una \concept[Medida\IS finita]{medida\IS finita}. Si $μ(X) = 1$, será una \concept[Medida\IS infinita]{medida\IS de probabilidad}.

Un ejemplo de una medida que no es una función es la delta de Dirac que habíamos visto antes, que de hecho es una medida de probabilidad.

\begin{prop}[Propiedades de la medida] Dado un espacio medible \meas:

\begin{enumerate}
\item \textbf{Monotonía}: Si $E⊂F$, entonces $μ(E) ≤ μ(F)$.
\item \textbf{Sub-aditividad}: $μ\left(\bigcup_{n=1}^∞ E_n\right) ≤ \sum_{n=1}^∞ μ(E_n)$ (sólo es igual si los conjuntos son disjuntos).
\item \textbf{Convergencia monótona}: Dada una sucesión de conjuntos $E_1 ⊂ E_2 ⊂ \dotsb$, entonces $μ\left(\bigcup_{n=1}^∞ E_n\right) = \lim μ(E_n)$.
\item \textbf{Convergencia dominada}: Dada una sucesión decreciente de conjuntos $E_1 ⊃ E_2 ⊃ \dotsb$, entonces $μ\left(\bigcap E_n\right) = \lim μ(E_n)$.
\end{enumerate}
\end{prop}

\subsection{Teorema de Littlewood}

Al principio de esta sección veíamos los principios de Littlewood. Estos se traducen en teoremas para los que ya tenemos suficiente teoría para enunciarlos. El primer teorema, obviamente, corresponde al primer principio de Littlewood.

\begin{theorem} Para todo $ε>0$ y para todo $E∈\mathcal{X}$ con $μ(E) < ∞$, existe un $F∈\algbA⊂ \mathcal{X}$ tal que \[ μ\left((E\setminus F) ∪ (F \setminus E)\right) < ε \], con $\algbA$ la familia de conjuntos simples (intervalos, cuadrados o cubos según la dimensión, que son los que sabemos medir).
\end{theorem}

Lo que este teorema nos dice es que cualquier conjunto medible se puede aproximar tan bien como queramos por conjuntos simples, que es la idea del primer principio de Littlewood.

\subsection{Integración}

Dado \meas un espacio de medida, definiremos la integral de la siguientes formas. Primero iremos a lo más simple, que son (atención al nombre) las funciones simples:

\begin{defn}[Función\IS simple] Se dice que $\appl{f}{X}{[0, +∞)}$ es simple si $f$ es de la forma \[ f(x) = \sum a_i \ind_{E_i}(x) \] para $\set{E_i} ⊆ \mathcal{X}$.\end{defn}

Para funciones simples $\appl{f}{X}{[0,+∞)}$ como \[ \int_X f\ \dif μ = \sum_{i=1}^n a_i μ\set{\inv{f}(a_i)} \] Como notación, hay que tener en cuenta que \[ \set{\inv{f}(a_i)} \equiv \set{x∈X\tq f(x)=a_i} \]

Para funciones más complicadas usaremos, como siempre, límites. Dada $\appl{f}{X}{[0, +∞)}$ medible, definiremos su integral como \[\int_X f \dif μ = \sup_{\substack{0≤g≤f \\ g \text{ simple}}} \int_X g \dif μ \]

Una observación que necesitaremos para poder usar esa definición es la siguiente proposición:

\begin{prop} Dada una familia de funciones $\appl{f_n}{X}{(-∞,+∞)}$ medibles, entonces $\sup f_n$ y $\inf f_n$ son medibles.\end{prop}

Con esto, ya podemos dar la definición de función integrable, que es bastante sencilla:

\begin{defn}[Función\IS integrable] Se dice que $f$ es integrable si y sólo si $\int f \dif μ < ∞$.\end{defn}

Ahora sólo nos falta aprender a integrar funciones negativas. Para eso definiremos las dos partes de una función: \begin{align*}
f_+ &= \max\set{f,0} \\
f_- &= \max\set{-f, 0}\end{align*}, de tal forma que $f = f_+ - f_-$, $\abs{f}= f_+ + f_-$ y entonces su integral es \[ \int_X f \dif μ = \int_X f_+ \dif μ - \int_X f_- \dif μ \]

Análogamente podemos hacer la integral de funciones complejas descomponiendo \[ f = (\Re f)_+ - (\Re f)_- + \imath (\Im f)_+ - \imath (\Im f)_- \] e integrando igual que antes.

Vamos a ver ahora las propiedades de las integrales.

\begin{prop} Dada $\appl{f}{X}{[-∞,+∞]}$, entonces

\begin{enumerate}
\item $f$ es integrable si y sólo si $\abs{f}$ es integrable\footnote{Esto en Riemann no es cierto, ver el contraejemplo de la \fref{eq:IntRiemmanNoLeb}.}, esto es, que $\int_X \abs{f} \dif μ < ∞$. En este caso decimos $f ∈ \lebg^1\meas \equiv \lebg^1 (X) \equiv \lebg_μ^1 (X) \equiv \lebg^1$, según lo que demos por supuesto en cada caso, donde $\lebg^1$ es el espacio vectorial de las funciones absolutamente integrable.
\item La aplicación que lleva cada función a su integral $f\longmapsto \int_X f \dif μ$ es lineal. Es fácil ver que $αf + βg = α \int_X f \dif μ + β \int_X g\dif μ$.
\item \concept{Desigualdad\IS triangular en $\lebg^1$}. Dada $\appl{f}{X}{ℂ}$ absolutamente integrable, entonces $\abs{\int f \dif μ} ≤ \int \abs{f}\dif μ$.
\item \textbf{Comparación}. Si $0 ≤ f ≤ g$, entonces $0 ≤ \int f \dif μ ≤ \int g \dif μ$. Además, si $f ≥ 0$ entonces $\int f = 0$ si y sólo si $f = 0$ en casi todo punto\footnote{$f = 0$ en casi todo punto si $∃N ⊂ X$ tal que $μ(N) = 0$ con $f ≠ 0$ en $N$.}. En el otro sentido, si $\int\abs{f} = 0$ entonces $f \equiv 0$ en casi todo punto.
\item Una función integrable se puede modificar todo lo que se quiera en un conjunto de medida nula mantieniendo su integral constante. Esta propiedad será un problema para definir los espacios de funciones integrables, como veremos más tarde.
\item $f ∈ \lebg^1_μ(X;ℂ)$ entonces $f$ es finita en casi todo punto.
\item La integral de Lebesgue ``generaliza'' la de Riemman, en el sentido de que todas las fórmulas como integración por partes, jacobianos y cambios de coordenadas son las mismas.
\end{enumerate}
\end{prop}

\begin{proof}
\begin{enumerate}
\item % TODO.
\end{enumerate}
\end{proof}

Ahora vamos con varios teoremas sobre integración.

\begin{theorem}[Teorema\IS de Fubini-Tonelli] Dada $\appl{f}{X×Y}{ℂ}$ medible, entonces \[ \iint\limits_{X×Y} f \dif μ \dif ν = \int_X \underbrace{\left(\int_Y f(x,y) \dif ν(y)\right)}_{F_1(x)} \dif μ(x) = \int_Y \underbrace{\left(\int_X f(x,y) \dif μ(x)\right)}_{F_2(y)} \dif ν(y) \] con $F_1$, $F_2$ medibles.

Concretamente, Tonelli nos da la medibilidad para funciones no negativas y Fubini la condición de integrabilidad.
\end{theorem}

Otro teorema, que se corresponde con el tercer principio de Littlewood.

\begin{theorem}[Teorema\IS de Egorov] Dada $\appl{f}{X}{ℂ}$ medible y $f_n \convs f$ en casi todo punto en $A⊆X$, entonces $f_n \convs f$ casi uniformemente en $A$. Esto es, converge uniformemente salvo en un conjunto de medida cero. Más formalmente, $∀ε>0$ existe un $N_ε⊆A$ tal que $f_n \convs f$ en $A\setminus N_ε$ y $0 ≤ μ(N_ε) < ε$.
\end{theorem}

La prueba de este teorema es sencilla y bonita:

\begin{proof} Podemos asumir que $f_n \convs f$ en todo punto\footnote{TODO: Entender por qué.}. Entonces, para $k,n ∈ ℕ$ definimos \[ E_n(k) = \sum_{m=n}^∞ \set{x∈X \tq \abs{f_m(x) - f(x)} > \frac{1}{k}} \], es decir, el conjunto de puntos en los que la función se aleja del límite.
Para todo $k$ fijo, es fácil ver que $E_n(k) ⊇ E_{n+1}(k) ⊇ \dotsb $ y además $\bigcap_{n=1}^∞ E_n(k) = ∅$\footnote{Esto también queda TODO.}. Entonces, por convergencia dominada de las medidas, sabemos que $\lim_{n\to \infty} μ(E_n(k)) = μ\left(\bigcap E_n\right) = μ(∅) = 0$.

Ahora la prueba se acaba de forma bastante sencilla: dado $ε > 0$ y $k∈ℕ$, elegimos $m_k$ tal que $μ(E_n(k)) ≤ \frac{ε}{2^k}$. Entonces, definimos $E = \bigcup_{k=1}^∞ E_{n_k}(k)$, y entonces $μ(E) ≤ \sum_{n≥1} μ(E_{n_k}) ≤ \sum_{k≥1} \frac{ε}{2^k} ≤ ε$.

Así, si $n > n_k$ y $x ∉ E$, tendremos que $\abs{f_n(x) - f(x)} < \frac{1}{k}$.

En definitiva, lo que hemos hecho ha sido coger todos los puntos donde va a converger y hacer algo más que me he perdido completamente.
\end{proof}

\begin{theorem}[Teorema\IS de Lusin] Dada $\appl{f}{X}{ℂ}$ medible con $μ(X) < ∞$, entonces $f$ es casi continua, que como viene siendo hasta ahora significa que $f$ es continua salvo en un conjunto de medida cero. Esto es, que $∀ε>0$ existe un $N_ε⊆X$ tal que $f$ es continua en $X\setminus N_ε$ y $μ(N_ε) < ε$.
\end{theorem}

Para la siguiente semana, tenemos que entender qué son las coordenadas polares en $ℝ^n$. Entender \cite{folland99} proposiciones 2.53, 2.54, ver por qué \[ \int_{ℝ^N} e^{-a\abs{x}^2} = \left(\frac{π}{a}\right)^{N/2} \] para $a ≥ 0$.

También tenemos que ver si es cierto que $\int_{ℝ^ℕ}\frac{1}{\abs{x}^α} \dif x = ∞$ para todo $α∈ℝ$ (spoiler: sí).

Comprobar que $\int_{ℝ^N} \frac{1}{(1 + \abs{x})^α} < ∞$ para cierto $α$ encontrarlo (pone algo de cambio a polares) y luego que $\int_{B_1(0)} \frac{1}{\abs{x}^{β}} < ∞$ y $\int_{B_1(0)} \frac{1}{(1-\abs{x})^α}$.

\section{Teoría de la integral y la medida}

\section{Espacios de Banach y Hilbert. Espacios $L^p$}

\section{SOC en espacio de Hilbert. Series de Fourier}

\section{Transformada de Fourier}

\section{Introducción a la teoría espectral}

\section{La ecuación del calor}

%% Apéndices (ejercicios, exámenes)
\appendix

\chapter{Ejercicios}
% -*- root: ../VariableReal.tex -*-
\section{Hoja 1}

\begin{problem} Sea \meas un espacio de medida.

\ppart Demostrar que $\appl{f}{Ω ⊆ X}{ℝ}$ es medible si y sólo si el conjunto $\set{x ∈ Ω \tq f(x) > r}$ es medible $∀r ∈ ℝ$.
\ppart Demostrar que si la sucesión de funciones $\appl{f_n}{Ω}{ℝ}$ es medible para todo $n$ también lo son las funciones $\sup f_n,\ \inf f_n,\ \limsup f_n$ y $\liminf f_n$.
\ppart Demostrar que si $\appl{f_n}{Ω}{ℝ}$ son medibles y $f_n(x) \convs f(x)$ puntualmente, entonces $f$ es medible.
\solution

\spart

Si $f$ es medible, entonces $∀E ⊆ ℝ$ medible se tiene que $\inv{f}(E)$ es medible, y en particular esto valdrá para los conjuntos de la forma $(r, +∞)$.

Para demostrar la implicación al otro lado, vemos ciertas propiedades de la inversa: sabemos que $μ(\inv{f}(E^c)) = μ(X) - μ(\inv{f}(E))$, que $\inv{f}(E ∩ F) = μ(E)∩μ(F)$ y que $\inv{f}(E∪F) = \inv{f}(E) ∪ \inv{f}(F)$. Así, podemos construir cualquier subconjunto de $ℝ$ a partir de intervalos de la forma $(r, +∞)$, que sabemos que son medibles. Esas operaciones son compatibles con la inversa y entonces podremos descomponerlo todo en conjuntos medibles\footnote{Esto merece algo más de explicación pero bueno.}.

\spart\label{ej:H1:SupremosMedibles} \textit{Nota: esto es una demostración extendida de la \fref{prop:SupremoInfimoMedibles}.}

Para demostrarlo vamos a usar el apartado anterior.

En el caso del supremo, sabemos que $\appl{\sup f_n}{Ω}{ℝ}$ será medible si y sólo si el conjunto \[ \set{x∈Ω \tq \sup_{n∈ℕ} f_n (x) > r } = \bigcup_{n ∈ ℕ} \set{x ∈ Ω \tq f_n(x) > r} \] es medible para todo $r ∈ ℝ$, que lo es porque se puede descomponer como unión de conjuntos medibles. En el caso del ínfimo lo tenemos análogamente pero haciendo la intersección de esos conjuntos: \[ \set{x ∈ Ω \tq \inf_{n∈ℕ} f_n(x) > r} = \bigcap_{n∈ℕ} \set{x∈Ω \tq f_n(x) > r } \]

\spart

Si $f_n \convs f$ puntualmente, eso significa que $\limsup f_n = \liminf f_n = f$, luego por lo que hemos visto en el apartado anterior es efectivamente medible.


\end{problem}

\begin{problem} Probar que si $\appl{f}{ℝ}{ℂ}$ es continua y $f(x) = 0$ en casi todo punto con respecto a la medida de Lebesgue, entonces $f(x) = 0\; ∀x ∈ ℝ$.

\solution

Si $f$ es nula en casi todo punto con respecto a la medida de Lebegue, entonces sólo podemos tener puntos aislados que difieran de 0. Sin embargo, si esos puntos existen no se cumpliría la definición de continuidad.

Más formalmente y usando la definición topológica de continuidad\footnote{No me apetece pensar continuidad en los complejos.} supongamos que existe $a ∈ ℝ$ tal que $c = f(a) ≠ 0$. Dado que $f$ es nula en casi todo punto, $c$ tiene que ser punto aislado y por lo tanto existe un $r ∈ ℝ$ tal que $0 ∉ \bola_r(c) ⊆ ℂ$, es decir, existe un entorno alrededor de $c$ que no contiene a $0$. Sin embargo, si cogemos un entorno $E = (a - ε, a + ε)$, tenemos que $f(E)$ es nulo en casi todo punto, por lo que $f(E) \nsubseteq \bola_r(c)$ y por lo tanto $f$ no puede ser continua. Contradicción, así que $f(x) = 0\; ∀x∈ℝ$.

\end{problem}

\begin{problem} \label{ej:H1:ConvMonotonaSeries} Sea $\set{f_n}$ una sucesión de funciones medibles con $f_n ≥ 0$. Usar el \nlref{thm:ConvMonotona} para demostrar que \[ \int_Ω\left(\sum_{n=1}^∞ f_n\right) \dif μ = \sum_{n=1}^∞ \int_Ω f_n \dif μ \]
\solution

Lo cierto es que esto es el \nlref{thm:ConvMonotonaSeries}, así que no hay que complicarse mucho la vida. Simplemente hay que definir la sucesión \[ S_N = \sum_{n=1}^N f_n \] que es monótona creciente por ser $f_n ≥ 0$, luego podemos usar el \fref{thm:ConvMonotona}. Lo que tenemos es por lo tanto que  \begin{align*} \int_Ω\left(\sum_{n=1}^∞ f_n\right) \dif μ &= \int_Ω \lim_{N\to \infty} S_N \dif μ \eqexpl{\ref{thm:ConvMonotona}} \lim_{N\to ∞} \int_Ω S_N \dif μ = \\ &= \lim_{N\to ∞} \int_Ω \sum_{n=1}^N f_n \dif μ \eqreasonup{Suma finita} \lim_{N\to ∞} \sum_{n=1}^N \int_Ω f_n \dif μ = \sum_{n=1}^∞ \int_Ω f_n \dif μ \end{align*}

\end{problem}

\begin{problem}[5] Probar la siguiente generalización de la \nref{prop:DesigualdadMarkov}. Sea $\appl{g}{ℝ}{[0, +∞]}$ una función medible y creciente en la imagen de $\appl{f}{\meas}{ℝ}$. Probar que \[ μ\left(\set{x ∈ X \tq f(x) > t}\right) ≤ \frac{1}{g(t)} \int_X g(f(x)) \dif μ(x) \]

Como consecuencia, probar que si $f_n \convs f$ en $\espL$, entonces $f_n \convs f$ en medida.
\solution

La desigualdad de Markov nos dice que, dada una función medible $f$, entonces \[ μ\left(\set{x∈X \tq \abs{f(x)} > ε }\right) ≤ \frac{1}{ε} \int_X \abs{f} \dif μ \]

Sea $E = \set{x ∈ X \tq f(x) > t}$. Podemos usar esto para descomponer la integral: \[ \frac{1}{g(t)} \int_X g(f(x)) \dif μ = \frac{1}{g(t)} \int_{X \setminus E} g(f(x)) \dif μ + \frac{1}{g(t)} \int_E g(f(x)) \dif μ\]

Como $g$ es creciente en la imagen de $f$, tenemos que $∀a ∈ f(E),\, g(a) > g(t)$, así que podemos simplificar en la segunda integral y nos queda que \begin{multline*}\frac{1}{g(t)} \int_{X \setminus E} g(f(x)) \dif μ + \frac{1}{g(t)} \int_E g(f(x)) \dif μ ≥ \frac{1}{g(t)} \int_{X \setminus E} g(f(x)) \dif μ + \frac{1}{g(t)} \int_E g(t) \dif μ = \\ = \int_{X \setminus E} g(f(x)) \dif μ + μ(E) ≥ μ(E) \end{multline*} y queda probada la desigualdad.

Ahora tenemos que usar esto para demostrar que si $f_n \convs f$ en $\espL$, entonces $f_n \convs f$ en medida o, en otras palabras, que
\[ \int_X \abs{f_n - f} \dif μ \to 0 \implies ∀ε > 0 \quad μ\left(\set{x∈X \tq \abs{f_n(x) - f(x)} > ε}\right) \to 0 \]

% TODO: Acabar esto.

\end{problem}

\section{Hoja 2}

\begin{problem} Sea μ una medida con signo sobre el espacio medible \meds. Definimos \index{Variación!total de una medida}la variación total de una medida como $\abs{μ} = μ_+ + μ_-$, siendo $μ_+, μ_-$ las partes positiva y negativa de una medida respectivamente, dadas por la descomposición de Jordan. Decimos que μ es finita si y sólo si $\abs{μ}$ lo es.

\ppart Probar la siguiente caracterización de la variación total\footnote{En los ejercicios aparece como partición de $X$ pero no le veo mucho sentido a eso.}: \[ \abs{μ} (E) = \sup \set{ \sum_{n=1}^∞ \abs{μ(E_i)} \tq \set{E_n} \text{ es una partición de } E} \]

Demostrar además que \begin{align*} μ_+(E) &= \sup \set{μ(F) \tq F ⊆ E, F ∈ \algb{X}}  \\  μ_-(E) &= - \inf \set{μ(F) \tq F ⊆ E, F ∈ \algb{X}} \end{align*}

Verificar que $μ = μ_+ - μ_-$ y que $\abs{μ} = μ_+ + μ_-$.

\ppart Probar que $\abs{μ}$ es la medida minimal tal que $-\abs{μ} ≤ μ ≤ \abs{μ}$.
\ppart Probar que $E ∈ \algb{X}$ es μ-nulo si y sólo si es $\abs{μ}$-nulo.
\ppart Probar que $\abs{μ_1 + μ_2} ≤ \abs{μ_1} + \abs{μ_2}$.
\ppart Probar que μ es finita si y sólo si $μ_+, μ_-$ lo son.
\ppart Sean $μ_1, μ_2$ medidas con signo. Probar que \[ μ_1 \perp μ_2 \iff μ_1 \perp \abs{μ_2} \iff μ_1\perp μ_{2,+} \iff μ_1 \perp μ_{2,-} \]
\solution

\spart Según el \nref{thm:DescompHahn} y el \nref{thm:DescompJordan}, siempre tendremos una partición de $X = X_+ ∪ X_-$ con medidas asociadas $μ_+, μ_-$ singulares. Al considerar todas las posibles particiones de $E_i$, siempre habrá alguna tal que los $E_i$ sean subconjuntos o bien de $X_+$ o de $X_-$ (es decir, que o bien $E_i ∩ X_+ = ∅$ o $E_i ∩ X_- = ∅$).

Si nos encontramos con una partición de este tipo, tendremos que $\abs{μ(E_i)} = μ_+(E_i)$ o que $\abs{μ(E_i)} = μ_-$, y por lo tanto la suma sobre todos los elementos de la partición será igual a $\abs{μ}(E)$. Sólo nos faltaría demostrar que esta partición es la que tiene valor máximo.

Por suerte, eso lo podemos ver muy fácilmente. Sea $\set{E^*_n}$ otra partición de $E$ que no cumpla la ``exclusividad'', es decir, que existan conjuntos cuyas intersecciones con $X_+$ \textbf{y} con $X_-$ sean no vacías. En ese caso, podemos considerar que cada elemento de la partición se puede descomponer $E^*_i = E_j ∪ E_k$ con $E_j, E_k$ parte de la partición $\set{E_n}$ y con\footnote{Si esto no ocurre porque no tenemos conjuntos suficientemente pequeños, siempre podemos montarnos otra partición que tendrá la misma medida de la suma.} $E_j ⊆ X_+$, $E_k ⊆ X_-$, de tal forma que \[ \abs{μ(E_i^*)} = \abs{μ(E_j) + μ(E_k)} = \abs{μ_+(E_j) - μ_-(E_k)} ≤ μ_+(E_j) + μ_-(E_k) \] donde para algunos $E_i$ la desigualdad será estricta (los que intersequen con $X_+$ y con $X_-$) y para otros habrá igualdad (más que nada sólo intersecarán con $X_+$ ó con $X_-$ y $E_j$ o $E_k$ serán vacíos). En cualquier caso, la suma sobre la partición siempre será menor que la suma sobre la partición ``exclusiva'' $\set{E_n}$.

Con esta demostración, es fácil ver que las definiciones de $μ_+, μ_-$ en base a supremo e ínfimo de subconjuntos de $E$ son válidas. Sólo tenemos que darnos cuenta de que $μ(F)$ valdrá el máximo cuando $F = E ∩ X_+$, y el mínimo (negativo, por eso el cambio de signo) cuando $F = E ∩ X_-$.

\spart

\spart

No veo muy bien cómo hacerlo, más que nada porque si $μ_+(E) = μ_-(E)$ se nos puede estropear la cosa.

\spart

\spart Bastante trivial.

\spart Hay dos tipos de implicaciones: o las que se ven fácilmente o las que me da la sensación de que no funcionan, como lo de $μ_1 \perp μ_{2,+} \iff μ_1 \perp μ_{2,-}$.

\end{problem}

\begin{problem}[6] Sea $\appl{μ}{\mathcal{B}(ℝ)}{[-∞,+∞]}$ una medida con signo sobre los borelianos de $ℝ$. Definimos la función $g(x) = μ([0,x])$. Probar que

\ppart $μ$ es medida continua si y sólo si $g$ es continua.
\ppart $μ$ es medida absolutamente continua si y sólo si $g$ es una función absolutamente continua.
\solution

Vamos con la definición de función absolutamente continua primero.

\begin{defn}[Función\IS absolutamente continua] \label{def:FuncAbsCont} $\appl{f}{ℝ}{ℝ}$ lo es si y sólo si para todo $ε>0$ existe un $δ>0$ tal que \[ \sum_{i=1}^k \abs{f(y_i) - f(x_i)} < ε\] para cualquier familia de intervalos disjuntos $[x_1, y_1], \dotsc, [x_n, y_n]$ tales que \[ \sum_{i=1}^k \abs{y_i - x_i} < δ \]

La continuidad absoluta implica continuidad uniforme (donde la δ no depende del punto que se coja), que a su vez implica continuidad normal (donde δ depende del ε pero también del punto $x_0$ que se valore).

La continuidad absoluta es la clase de funciones más grande tal que el teorema fundamental del cálculo integral con la integral de Lebesgue y la derivada clásica es cierto.
\end{defn}

\spart

Si $g$ es continua en $x_0$, entonces para todo $ε > 0$ existe un $δ > 0$ tal que, si $\abs{x_0 - x} < δ$, entonces $\abs{g(x_0) - g(x)} < ε$. Transformando esa última ecuación y suponiendo que $x_0 < x$, tenemos que $\abs{μ([0, x_0]) - μ([0,x])} = \abs{μ([x_0,x)} < ε$. Igualmente, si tomamos $x_1 > x$ con $\abs{x_1 - x} < δ$, tendremos que $\abs{μ([x, x_1))} < ε$. Usando la desigualdad triangular y operando un poquillo nos quedará que $\abs{μ([x_0, x_1))} < 2ε$.

Podemos hacer ε tan pequeño como queramos, de tal forma que $μ([x_0, x_1)) \to 0$ con $x ∈ [x_0, x_1)$ siempre, y por convergencia dominada tendremos que $μ(\set{x}) = 0$.

Creo que además podemos recorrer esta demostración al revés para sacar la implicación al otro lado.

\spart


\end{problem}

\begin{problem}[8] Si $f ∈ \espL$, con $f \not\equiv 0$, probar que

\ppart Existen $c, R > 0$ tales que la función maximal $Mf(x) ≥ c \abs{x}^{-N}$ para $\abs{x} > R$.
\ppart Como consecuencia, cuando $t > 0$ es pequeño, existe $c'$ tal que \[ m\left(\set{x ∈ ℝ^N \tq Mf(x) ≥ t}\right) ≥ \frac{c'}{t} \]
\ppart Probar que, en general, es falsa la desigualdad \[ \norm{Mf}_{\espL} ≤ C \norm{f}_{\espL} \]

Indicación: describir la función maximal de $f(x) = \ind_{(-1, 1)} (x)$.
\solution

\spart Cogemos $α = \inf Mf(x)$, que está bien definido ($Mf(x) ≥ \abs{f(x)} ≥ 0$) y además no es 0 (es el supremo de los valores medios para $r > 0$, sólo puede ser $0$ si $f \equiv 0$ y ya hemos dicho por hipótesis que eso no ocurre). Por otra parte, $c \abs{x}^{-N} < c R^{-N}$ si $\abs{x} >R$, así que sólo tenemos que coger $c,R$ que cumplan que $α ≥ cR^{-N}$.

\spart Es cierto, sólo hay que aplicar el \nref{thm:Maximal}. No sé, eso sí, como hacerlo como consecuencia de lo otro.

\spart Cuando $x ∈ (-1, 1)$, el valor de la función maximal es $1$. Cuando está fuera, el valor máximo se alcanzará en el primer momento en el que la bola $\bola_r(x)$ cubra $(-1,1)$ (es decir, la bola de menor radio que cubre ese intervalo). Esa bola será el intervalo $(-1, x + (x + 1))$ (si $x < 0$ es análogo), esto es, la bola de radio $x + 1$. Así, tendremos que \[ Mf(x) = \fint\limits_{(-1, 2x + 1)}\ind_{(-1, 1)} (t) \dif t = \frac{m((-1, 1))}{m((-1, 2x + 1))} = \frac{1}{\abs{x}} \quad x ∉ (-1,1) \]

Si ahora hacemos la integral de $Mf$ nos quedará que \[ \norm{Mf}_{\espL} = \int_{-∞}^∞ \abs{Mf(x)} \dif x = 2 \int_{1}^∞ \frac{1}{x} \dif x + \int_{-1}^1 \dif x \] que no converge por el $\frac{1}{x}$, luego la desigualdad no se cumple.

\end{problem}

\begin{problem} A partir del \nref{thm:Maximal} y usando la \nlref{thm:DesigualdadJensen}, probar que si $p ∈ [1,∞)$ existe $C < ∞$ tal que para toda $f ∈ \espLp(ℝ)$ y para todo $λ>0$ se cumple que \[ m\left(\set{x ∈ ℝ \tq Mf(x) > λ}\right) ≤ \frac{C}{λ^p} \int_ℝ \abs{f(y)}^p \dif y \] donde $M$ es la \nlref{def:FuncMaximalHL}.

\solution
\end{problem}

\begin{problem}[13] Probar el siguiente teorema que relaciona la derivada de Radon-Nikodim con el \nref{thm:DifLebesgue}. Sea μ una medida de Borel Regular y sea $μ = μ_S + μ_{AC}$ con $μ_{AC} = f \dif m$.  Entonces, para casi todo punto $x ∈ ℝ^n$, se tiene que \[ \lim_{r \to 0} \frac{μ(E_r)}{m(E_r)} = f(x) \] para toda familia $\set{E_r}$ que tiende suavemente (\fref{def:EncogerBien}) a $x$.

\solution

Siguiendo con la descomposición, tenemos que $μ(E_r) = μ_S(E_r) + μ_{AC} (E_r)$. Tenemos que ver que la parte singular nos la podemos quitar.

Como $E_r$ es una familia que tiende suavemente a $x$, entonces $m(E_r) > α m(\bola_r(x))$ con $α > 0$. En concreto, esto nos dice que $m(E_r) ≠ 0$. Como $μ_S \perp m$, la intersección de sus soportes tiene medida $0$, luego $μ(E_r) = 0$.

Ahora sólo tenemos que ir a por la parte singular continua, que como se tiene que $μ_{AC} = f \dif m$ con $f$ integrable, entonces nos queda que \[ μ_{AC} (E_r)  = \int_{E_r} f \dif m \]

Dividiendo esto por $m(E_r)$, nos queda el valor medio de $f$, así que finalmente tenemos \[ \lim_{r \to 0} \frac{μ(E_r)}{m(E_r)} = \frac{1}{m(E_r)} \int_{E_r} f \dif m = f(x)\] por el \nlref{thm:DifLebesgue}.
\end{problem}

\begin{problem}
Probar que si $f ∈ \espLloc$ y es continua en $x_0 ∈ ℝ^N$, entonces $x_0 ∈ L_f$. Es decir, que cada punto de continuidad es un \nlref{def:PuntoLebesgue}.
\solution

Recordamos que $x_0$ es un punto de Lebesgue de $f$ si y sólo si \[ \lim_{r \to 0} \fint_{\bola_r(x_0)} f \dif m = f(x_0) \]

Si $f$ es continua en $x_0$, entonces $∀ε > 0$ existe un $δ > 0$ tal que, si $\abs{x-x_0} < δ$, entonces $\abs{f(x) - f(x_0)} < ε$. Lo que hacemos entonces es coger bolas de radio $δ \to 0$ e integrar $f(x) - f(x_0)$:
\[ \fint_{\bola_δ(x_0)} \abs{f(x) - f(x_0)} \dif x ≤ \fint_{\bola_δ(x_0)} ε \dif x = ε \]

Si esta integral se va a cero con el valor absoluto cuando $ε \to 0$ (y por lo tanto $δ \to 0$ igualmente), entonces está claro que $\fint_{\bola_δ(x_0)} f(x) - f(x_0) \dif x \to 0$, y podemos sacar el $f(x_0)$ de ahí para quedarnos con la igualdad que teníamos que demostrar.
\end{problem}

\section{Hoja 3}

\begin{problem}[2] Usar la \nref{thm:DesigualdadJensen} para demostrar las siguientes desigualdades:

\ppart Desigualdad entre media aritmética y geométrica. Para todo $1 ≤ n ∈ ℕ$ y todo $x_i ∈ [0, +∞)$, \[ (x_1 \dotsb x_n)^{\frac{1}{n}} ≤ \frac{x_1 + \dotsb + x_n}{n} \]

\ppart Desigualdad entre media aritmética y geométrica generalizada: para todo $1 ≤ n ∈ ℕ$ y todo $α_i,x_i ∈ [0, +∞)$ con $α_1 + \dots + α_n = 1$, se tiene que \[ \prod x_i^{α_i} ≤ \sum α_i x_i \]

\solution

\spart Una función que nos permite pasar de productos a sumas es el logaritmo, así que vamos a ver qué pasa si tomamos $φ = \log$ en un espacio discreto. Por ser formales, diremos que $X = ℤ_n$, con $f(z) = x_z$ y la medida dada como $μ(z) = 1\;∀z ∈ ℤ_n$, de tal forma que  $\int f = \sum x_i$ y $μ(ℤ_n) = n$.

En este caso, φ no es convexa sino cóncava. Lo bueno es que entonces $-φ$ es convexa, así que podemos multiplicar por $-1$ en ambos lados de la desigualdad y se nos invierte. Así, lo que tenemos es que \[ \log\left(\frac{1}{n} \sum x_i\right) ≥ \frac{1}{n} \sum \log x_i  = \frac{1}{n} \log \prod x_i = \log \left(\prod x_i\right)^{\frac{1}{n}} \] y quitando los logaritmos ya tenemos lo que nos pedían.

\spart Aquí tenemos que cambiar la medida, y decir que $μ(z) = α_z$, lo que nos queda que $μ(ℤ_n) = 1$, que en el fondo nos va a simplificar algo las cosas. Aplicando de nuevo la desigualdad de Jensen con $φ = \log$, tenemos que \begin{align*}
\log\left(\int f \dif μ\right) &≥ \int \log f \dif μ \\
\log\left(\sum α_i x_i\right) &≥ \sum α_i \log x_i = \log\left(\prod x_i^{α_i}\right)
\end{align*}, que de nuevo se nos queda en lo que nos pedían demostrar si quitamos logaritmos.

\end{problem}

\begin{problem} Probar la \nlref{prop:DesHolder} como consecuencia de la \nlref{thm:DesigualdadJensen}.

\solution

La desigualdad de Hölder nos dice que, siendo $1 ≤ p ≤ + ∞$ y $q = \frac{p}{p - 1} ≥ 1$ de tal forma que $\frac{1}{p} + \frac{1}{q} = 1$, tenemos que \[ \int_X \abs{fg} \dif μ ≤ \norm{f}_p \norm{g}_q \]

Para demostrarla, vamos a hacer un truco que consiste en montar una medida artificial. Sea \[ ν = \frac{f^p}{\norm{f}_p^p} \] una medida, cuya principal ventaja es que $ν(X) = \int_X \frac{f^p}{\norm{f}_p^p} \dif μ = 1$ y por lo tanto $\fint \dif ν\equiv \int \dif ν$.

Sea $φ(z) = z^q$, convexa. Podemos aplicar ahora la desigualdad de Jensen a una función $h$ que ya veremos luego cómo montamos. La desigualdad nos dice que \begin{align*}
\left(\int h \dif ν\right)^q &≤ \int h^q \dif ν \\
\left(\int \frac{hf^p}{\norm{f}_p^p} \dif μ\right)^q &≤ \int \frac{h^qf^p}{\norm{f}_p^p} \dif μ
\end{align*}

Querríamos ahora que $hf^p$ fuese $fg$, así que tomamos $h = \frac{g}{f^{p-1}}$, que además nos da que $h^q f^p = \frac{g^qf^p}{f^{q(p-1)}} = \frac{g^qf^p}{f^p} = g^q$. Sustituyendo, tenemos que
\begin{align*}
\left(\int \frac{hf^p}{\norm{f}_p^p} \dif μ\right)^q &≤ \int \frac{h^qf^p}{\norm{f}_p^p} \dif μ \\
\frac{1}{\norm{f}_p^{pq}} \left(\int fg \dif μ\right)^q &≤ \frac{1}{\norm{f}_p^p} \int g^q \dif μ \\
\int fg \dif μ &≤ \norm{f}_p^{\frac{p(q-1)}{q}} \norm{g}_q \\
\int fg \dif μ &≤ \norm{f}_p \norm{g}_q
\end{align*}, que es la desigualdad de Hölder ya que $p\frac{q - 1}{q} = p\frac{\frac{p}{p-1} - \frac{p - 1}{p-1}}{\frac{p}{p - 1}} = \frac{\frac{1}{p-1}}{\frac{1}{p-1}}$ = 1.
\end{problem}


\begin{problem} Probar que la desigualdad de Hölder es una igualdad si y sólo si $\abs{f}^p = λ\abs{g}^{q}$ CTP para alguna constante $λ > 0$, con $q$ el exponente conjugado de Hölder.

\solution

Si $\abs{f}^p = λ\abs{g}^{q}$, entonces \[ \norm{f}_p = \left(\int_X \abs{f}^p \right)^{\frac{1}{p}} = \left(\int_X λ \abs{g}^q \right)^{\frac{1}{p}} = λ^\frac{1}{p} \norm{g}_q^{\frac{q}{p}} \], y como $\frac{q}{p} + 1 = \frac{q + p}{p} = q$, entonces $\norm{fg}_q = λ^{\frac{1}{p}} \norm{g}_q^q$.

Por otra parte \[ \int_X \abs{fg} \dif μ = \int_X \abs{λ^\frac{1}{p} g^\frac{q}{p} g} \dif μ  = λ^{\frac{1}{p}} \int_X g^q \dif μ = λ^{\frac{1}{p}} \norm{g}_q^q \], así que en este sentido queda demostrado.

Supongamos ahora que $\norm{fg}_1 = \norm{f}_p \norm{g}_q$. Entonces, no sé cómo demostrarlo.

\end{problem}

\begin{problem}[5] Probar por inducción las siguientes generalizaciones de la \nlref{prop:DesHolder}.

\ppart Dados $1 < p_1, p_2, \dotsc, p_n < ∞$ con $f_i ∈ L^{p_i}$ para $i = 1,2, \dotsc, n$, entonces se cumple que \[ \abs{\int \prod f_i \dif μ} ≤ \prod \norm{f_i}_{p_i} \] cuando $\sum\frac{1}{p_i} = 1$.
\solution

\spart Sabemos que para $n = 1,2$ se cumple. Supongamos que se cumple para $n - 1$ y vamos a demostrar que se cumple para $n$. Consideramos $g = \prod_{i=1}^{n-1} f_i$, con $\sum_{i=1}^{n-1}\frac{1}{p_i} = 1$, que sabemos que cumple que $\norm{g}_1 ≤ \prod \norm{f_i}_{p_i}$, luego en concreto $g ∈ L^1$.

\end{problem}


\bibliography{../Apuntes.bib}
\printindex
\end{document}
