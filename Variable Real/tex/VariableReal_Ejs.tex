% -*- root: ../VariableReal.tex -*-
\section{Hoja 1}

\begin{problem} Sea \meas un espacio de medida.

\ppart Demostrar que $\appl{f}{Ω ⊆ X}{ℝ}$ es medible si y sólo si el conjunto $\set{x ∈ Ω \tq f(x) > r}$ es medible $∀r ∈ ℝ$.
\ppart
\ppart
\solution

\spart

Si $f$ es medible, entonces $∀E ⊆ Y$ se tiene que $\inv{f}(E)$ es medible, y en particular esto valdrá para los conjuntos de la forma $(r, +∞)$. Para demostrar la implicación al otro lado, vemos ciertas propiedades de la inversa: sabemos que $μ(\inv{f}(E^c)) = μ(X) - μ(\inv{f}(E))$, que $\inv{f}(E ∩ F) = μ(E)∩μ(F)$ y que $\inv{f}(E∪F) = \inv{f}(E) ∪ \inv{f}(F)$. Así, podemos construir cualquier subconjunto de $ℝ$ a partir de intervalos de la forma $(r, +∞)$, que sabemos que son medibles. Esas operaciones son compatibles con la inversa y entonces podremos descomponerlo todo en conjuntos medibles\footnote{Esto merece algo más de explicación pero bueno.}.

\spart

\spart
\end{problem}
