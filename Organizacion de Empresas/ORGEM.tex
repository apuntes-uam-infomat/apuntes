\documentclass[nochap,palatino,notitlepage]{apuntes}

\title{Organización de Empresas Tecnológicas}
\author{Guillermo Julián \and Víctor de Juan}
\date{15/16 C1}

% Paquetes adicionales
\usepackage{fancysprefs}
\usepackage{booktabs}
\usepackage{multirow}

\begin{document}
\pagestyle{plain}
\maketitle

\begin{abstract}
Estos apuntes pretenden ser un resumen concreto y más o menos completo de la asignatura de Organización de Empresas Tecnológicas, con (esperemos) pocas abreviaturas.
\end{abstract}

\tableofcontents
\newpage
\section{Empresa}

\subsection{Inversión y bolsa}

La inversión en bolsa parte de un hecho fundamental, que es que las empresas tienen propietarios, en plural, y cada uno de ellos tendrá un porcentaje de propiedad. Normalmente, el porcentaje de propiedad se divide en unidades llamadas \concept[Acción]{acciones} o participaciones.

Esa propiedad puede dar lugar a diversos derechos, como por ejemplo el de recibir una parte de los beneficios de la empresa (\concept{Dividendo}) o a poder participar en la toma de decisiones de la empresa a través del \concept{Consejo\IS de accionistas}.

Monetariamente, esas participaciones tienen un valor económico, un valor que será proporcional a la valoración de la empresa. Además, esas participaciones se pueden vender y comprar. La bolsa es el mercado donde las participaciones o acciones de empresas se venden y se compran.

Como en cualquier acción de compra y venta, el precio fluctúa por la ley de la oferta y la demanda, y esas fluctuaciones de precio permiten ganar dinero (y, por supuesto, perderlo) en la bolsa a través de operaciones.

\subsubsection{Operaciones en bolsa}

La operación más sencilla que uno puede hacer en bolsa es comprar una acción, esperar a que su valor suba porque más gente quiera comprar participaciones de la empresa (por ejemplo, si esa empresa va bien y los inversores esperan que siga así) y venderla para recoger los beneficios. Ahora bien, las cosas pueden complicarse un poco más, y de hecho la terminología cambia. A partir de ahora, consideraremos una \concept{Posición} como una inversión abierta sobre las participaciones (o \textit{stock}) de una empresa.

Es obvio que tener acciones de una empresa es una posición, de hecho se llama \concept{Posición\IS larga}\footnote{Traducido, probablemente mal, del inglés \textit{long position}.}. Cuando el valor de una empresa sube en bolsa, podemos \concept{Cerrar\IS posiciones}, esto es, vender nuestras acciones para recoger el beneficio.

Ahora bien, eso sólo nos permite ganar dinero cuando la bolsa sube, y si hemos visto las noticias sabremos que los inversores también ganan dinero cuando la bolsa baja. Para eso se usa lo que se llama una \concept{Posición\IS corta}. Hay diferentes formas de mantener este tipo de posiciones, en cualquier caso basadas en prometer acciones a un tercero por un cierto valor $P$: si la valoración de la empresa baja por debajo de ese valor, podremos comprar las  acciones a $P - ε$ y sacar un beneficio de ε.

Las posiciones en corto se pueden lograr usando préstamos: pedimos un préstamo de $n$ acciones, en ese momento valoradas a $P$ cada una. Inmediatamente, vendemos a otro inversor esas $n$ acciones al mismo precio $P$. En este momento tenemos $Pn$ en nuestra cuenta, pero debemos $n$ acciones al prestamista. Cuando el valor de la empresa baje a $P-ε$, nosotros compraremos las acciones a ese valor y se las devolveremos al prestamista\footnote{El préstamo es en acciones, no en dinero, así que hemos saldado nuestra deuda.}. Así, al venderlas al principio ganamos $Pn$, y luego las compramos por $(P-ε)n$: nuestro beneficio será de $εn$.

También se pueden usar \concept{Contratos\IS de opciones}\footnote{De nuevo probablemente mal traducido del inglés \textit{options contract}.}. Estos contratos se venden por un precio determinado, y dan derecho al comprador a la venta o a la compra de un cierto \textit{stock} de una empresa por un precio fijo $P$ en un rango de fechas determinado. Así, podemos comprar a un inversor un contrato de venta que nos da derecho a venderle $n$ acciones a $P$ cada una dentro de unos cuantos días\footnote{O el período que sea.}. Si el precio de la acción baja a $P - ε$ cuando pase ese tiempo, podremos comprar las acciones por $P - ε$ y venderlas por $P$, sacando $ε$ de beneficio por acción.

Por supuesto, esto nos puede salir mal y podemos perder dinero si el \textit{stock} sube. De hecho, si no pudiese salirnos mal no podríamos hacer este negocio con nadie porque nadie querría tirar el dinero así.

Otro tipo de operaciones financieras son los llamados \concept{Contrato\IS por diferencia (CFD)}, que necesitan de un inversor (nosotros) y un proveedor con dinero que será nuestra ``marioneta'', por así decirlo, a cambio de un módico precio. Nosotros indicamos al proveedor que queremos que compre $n$ acciones de una empresa, que ahora mismo están a precio\footnote{Habitualmente, el proveedor dice el precio $P_b$ al que él compra las acciones, que suele estar cercano a $P$ pero que no tiene por qué ser igual.} $P$. Para que haga esa operación, tendremos que pagarle un porcentaje del coste $m$. Cuando el precio de la acción suba, nosotros daremos la instrucción al proveedor de que venda esas acciones a\footnote{Al igual que con el precio de compra, el precio de venta del proveedor no tiene por qué ser el mismo que el de cotización, pero para el caso nos da lo mismo.} $P + ε$. El beneficio de la operación, $nε$, va para nosotros.

La ventaja de estas operaciones es que habiendo invertido $m·nP$ con CFDs hemos conseguido un beneficios similar al que obtendríamos invirtiendo $nP$ en el mercado normal, así que nuestras ganancias se multiplican. La parte mala es que por un lado hay comisiones por todas partes: para comprar las CFDs hay una comisión, otra por mantenerlas de un día para otro, otra por retirar el dinero... Y, por supuesto, si la valoración de la empresa baja, las pérdidas también nos las comemos íntegras nosotros, así que nos podemos encontrar con que debemos al proveedor incluso más de lo que habíamos invertido.

\subsection{Indicadores financieros}
\label{sec:IndicadoresFinancieros}

Para evaluar la situación de una empresa y valorar si nos conviene invertir o no, se pueden definicir ciertos indicadores que nos den una imagen de cómo está funcionando en bolsa esa empresa.

\begin{itemize}
\item \concept{Price\IS to earnings}, es el cociente de la valoración de la empresa entre los beneficios obtenidos. El valor adecuado se suele considerar entre 10 y 17, por encima puede indicar sobrevaloración (muchos beneficios para lo poco que cuesta la acción) y, análogamente, por debajo indicará infravaloración.
\item \concept{Price\IS to sales}, lo de antes pero dividiendo entre las ventas de la empresa.
\item \concept{Price\IS to book}, lo de antes pero con el \textit{book value}\index{Book!value}, que es valor de los activos tangibles (esto es, los que tienen una valoración clara, nada de patentes, copyrights o similares) menos el de los pasivos.
\item \concept{Beta}, es la relación entre el movimiento del mercado y el de la empresa: si $β = 1$, entonces eso indica que la empresa crece con el mercado. Si $β=0$, las fluctuaciones del mercado no influyen en la valoración de la empresa. El ``mercado'' suele ser un índice de empresas, como el S\&P 500.
\item \concept{Alpha}, más comúnmente \textit{weighted alpha}, parece ser una media del crecimiento a lo largo de un año dando más peso a las variaciones más cercanas al momento actual. Si es positivo, indica que la acción ha crecido a lo largo del tiempo.
\item \concept{Yield}, es el cociente de dividendos anuales entre el precio de acción.
\item \concept{Return\IS on equity}, cociente de beneficios netos entre el valor líquido de la empresa\footnote{El valor líquido es los activos menos los pasivos o, en otras palabras, la valoración bursátil más lo que la empresa tenga ahorrado.}
\item \concept{Return\IS on assets}, lo mismo de antes pero dividido entre el total de activos.
\item \concept{Earnings\IS per share (EPS)}, los beneficios de la empresa dividido entre el número total de acciones.
\end{itemize}

\section{Administración}

\section{Análisis de estados financieros}

El análisis de estados financieros de una empresa implica, como su nombre indica, analizar y diseccionar el estado del dinero en una empresa. Se podrían distinguir dos ámbitos de contabilidad: por un lado, la financiera, que considera la empresa como un todo y analiza su relación con el exterior. Por otro lado, podríamos tener la contabilidad de costes que analiza cómo funciona la empresa por dentro.

\subsection{Contabilidad: Balance, PyG y caja}

\begin{figure}[hbtp]
\centering
\includegraphics[width=\textwidth]{img/Balance_PyG_Caja.png}
\caption{Un ejemplo de modelo de balance, PyG y caja. Vía \href{http://maestremiranda.com/techdir/wp-content/uploads/2015/10/EF0.-Bal_PYG_Caja.pdf}{maestremiranda.com}}
\label{fig:BalancePyGCaja}
\end{figure}

Las empresas españolas están obligadas a presentar las cuentas en el registro mercantil según el \href{https://www.boe.es/buscar/doc.php?id=BOE-A-2007-13023}{Plan General Contable de 2007}, corregido en la \href{https://www.boe.es/boe/dias/2009/02/10/pdfs/BOE-A-2009-2276.pdf}{Orden JUS/206/2009 del 28 de enero}.

\subsubsection{Balance}

Hay tres formas de afrontar la contabilidad de la empresa. La primera es la que surge al mirar el patrimonio de una empresa, que es donde vemos el activo $A$ y el pasivo $P$ (lo que se tiene y lo que se debe) y el patrimonio neto $N$. Este último a veces se denomina pasivo no exigible. En cualquier caso, las tres cantidades están relacionadas por la siguiente fórmula infinitamente complicada: \[ A - P = N\]

La \fref{fig:BalancePyGCaja} contiene una muestra de qué cuenta dentro de activos y qué dentro de pasivos. La diferencia entre activo o pasivo corriente y no corriente es sencilla: lo corriente es lo volátil, lo que está a corto plazo.

La \fref{tab:Balance} tiene un modelo abreviado del balance del Plan General Contable de 2007, relleno con los datos del siguiente ejemplo:

\begin{itemize}
\item En tesorería (cuentas corrientes) dispone de 260 (\textit{Efectivo y otros activos líquidos}).
\item Durante el ejercicio ha obtenido 10 de beneficio (\textit{PyG}).
\item Se sabe que hay una póliza de préstamo bancario por 230 (\textit{Deuda M/L}).
\item La aportación inicial de los socios fue de 100 (\textit{Capital social}).
\item A los suministradores todavía les debe 10 (letra a pagar con vencimiento 30 días) (\textit{Acreedores comerciales / proveedores}).
\item Los productos en inventario se valoran en 20 (\textit{Existencias}).
\item El activo fijo (mobiliario y ordenadores) asciende a 20 (\textit{Inmovilizado material}).
\item La cifra acumulada de beneficios retenidos es de 50 (\textit{Reservas}).
\item De las ventas realizadas le faltan por cobrar 100 (un cliente debe por mercancía no pagada) (\textit{Deudores comerciales}).
\end{itemize}

\begin{table}[hbtp]
\begin{minipage}{\textwidth}
\footnotesize
\centering
\begin{tabular}{l|c|c}
\textbf{Concepto} & \textbf{Debe} & \textbf{Haber} \\ \toprule
\multicolumn{3}{c}{\textsc{Activo} - \textbf{A) Activo no corriente}} \\ \midrule
I. Inmovilizado intangible & & - \\
II. Inmovilizado material & & 20 \\
III. Inversiones inmobiliarias & & - \\
IV. Inversiones en empresas del grupo y asociadas a largo plazo & & - \\
V. Inversiones financieras a largo plazo & & - \\
VI. Activos por impuesto diferido\footnote{Lo que el Estado nos debe de impuestos.} & & - \\
\textbf{Total activo no corriente} & & \textbf{20} \\ \midrule
\multicolumn{3}{c}{\textsc{Activo} - \textbf{B) Activo corriente}} \\ \midrule
I. Activos no corrientes para la venta & & - \\
II. Existencias & & 20 \\
III. Deudores comerciales y otras cuentas a cobrar & & 100 \\
IV. Inversiones en empresas del grupo y asociadas a corto plazo & & - \\
V. Inversiones financieras a corto plazo & & - \\
VI. Periodificaciones a corto plazo\footnote{Pagos hechos pero no devengados. Por ejemplo si se paga una póliza de seguros de dos años, la parte correspondiente al año siguiente debería ir en este epígrafe para compensar.} & & - \\
VII. Efectivo y otros activos líquidos equivalentes & & 260 \\
\textbf{Total activo corriente} & & \textbf{380} \\ \midrule
\multicolumn{3}{c}{\textsc{Patrimonio neto} - \textbf{A-1) Fondos propios}} \\ \midrule
I. Capital suscrito & & 100 \\
II. Prima de emisión\footnote{La diferencia entre el valor nominal de las acciones y el que se obtiene por ellas.} & & - \\
III. Reservas\footnote{Hay una obligación legal de mantenerla dotándola con cargo a los beneficios hasta que alcance el 20\% del capital social.} & & - \\
IV. Acciones y participaciones en patrimonio propias & & - \\
V. Resultados de ejercicios anteriores & - & 50 \\
VI. Otras aportaciones de socios & & - \\
VII. Resultado del ejercicio (PyG) & & 10 \\
VIII. Dividendo a cuenta & - &\\
IX. Otros instrumentos de patrimonio neto & - & - \\
\textbf{Total fondos propios} & - &  \textbf{160} \\
\textbf{A-2) Ajustes por cambio de valor} & - & - \\
\textbf{A-3) Subvenciones, donaciones y legados} & & - \\ \midrule
\multicolumn{3}{c}{\textsc{Pasivo} - \textbf{B) Pasivo no corriente}} \\ \midrule
I. Provisiones a largo plazo\footnote{Las provisiones son partidas que tendremos que pagar (sueldos, impuestos) más tarde, con importe o fecha no concretos.} & - &  \\
II. Deudas a largo plazo & 230 & \\
III. Deudas con empresas del grupo y asociadas a largo plazo & - &  \\
IV. Pasivos por impuesto diferido & - &  \\
V. Periodificaciones a largo plazo &  & \\
\textbf{Total pasivo no corriente} & \textbf{230} &  \\ \midrule
\multicolumn{3}{c}{\textsc{Pasivo} - \textbf{B) Pasivo corriente}} \\ \midrule
I. Pasivos vinculados con activos no corrientes mantenidos para la venta & - & \\
II. Provisiones a corto plazo & - & \\
III. Deudas a corto plazo  & - & \\
IV. Dedudas con empresas del grupo y asociadas a corto plazo & - & \\
V. Acreedores comerciales y otras cuentas a pagar & 10 & \\
VI. Periodificaciones a corto plazo & - & \\
\textbf{Total pasivo corriente} & \textbf{10} &  \\ \midrule
\end{tabular}

\caption{Modelo abreviado de balance relleno con el ejemplo anterior, según el Plan General Contable 2007. Se puede ver que $P = 240$, $A = 400$, $N = 160$ así que todo cuadra.}
\label{tab:Balance}
\end{minipage}
\end{table}

\subsubsection{Pérdidas y ganancias (PyG)}

Mientras que el balance da una foto instantánea el destado de una empresa, la cuenta de pérdidas y ganancias (PyG) indica cómo varía ese estado a lo largo del tiempo. Es importante recalcar que aquí se usa el criterio de \concept{Devengo}, por el cual los ingresos y gastos computan cuando nos comprometemos a ellos y no cuando el dinero de verdad cambia de manos (por ejemplo, una factura emitida en noviembre va en las cuentas de noviembre aunque nos transfieran el dinero en diciembre).

De nuevo en la \fref{fig:BalancePyGCaja} se puede ver qué hay en cada cosa, aunque de nuevo vamos a hacer un ejemplo. Durante un ejercicio se producen los siguientes hechos:

\begin{itemize}
\item Los sueldos y salarios se elevan a 150.
\item Los intereses de la deuda ascienden a 210.
\item Los servicios generales suman 30.
\item El importe de la cifra de negocio es de 500.
\item Se han dotado 20 para recuperar activos fijos.
\item La tasa impositiva es de un 35\%.
\item Los aprovisionamientos se cifran en 70.
\item El valor del stock final supera el inicial en 60
\end{itemize}

Los resultados se pueden ver en la \fref{tab:PyG}.

\begin{table}[hbtp]
\centering
\begin{minipage}{\textwidth}
\footnotesize
\begin{tabular}{l|c|c}
\textbf{Concepto} & \textbf{Debe} & \textbf{Haber} \\ \toprule
1. Importe neto de la cifra de negocios & & 500 \\
2. Variación de existencias de productos terminados y en curso de fabricación & - & 60 \\
3. Trabajos realizados por la empresa para su activo\footnote{Por ejemplo, I+D.} & - & \\
4. Aprovisionamientos & 70 &  \\
5. Otros ingresos de explotación\footnote{Subvenciones, donaciones y legados que financien activos o gastos de la explotación.} & & - \\
6. Gastos de personal & 150 & \\
7. Otros gastos de explotación & 30 & \\
8. Amortización del inmovilizado & - & \\
9. Imputación de subvenciones de inmovilizado no financiero\footnote{Subvenciones, legados y donaciones que financien el activo pasivo (p.e., te regalan una casa) o cancelaciones de deudas.} & - & - \\
10. Exceso de provisiones & - & \\
11. Deterioro y resultado por enajenaciones\footnote{Ventas, que los economistas hablan raro.} de inmovilizado & - & \\ \midrule
\textbf{A) Resultado de explotación} (BAAAIT) ($\sum_1^{11}$) & \textbf{250} & \textbf{560} \\ \midrule
12. Ingresos financieros & & - \\
13. Gastos financieros & 210 & \\
14. Variación de valor razonable en instrumentos financieros & - & - \\
15. Diferencias de cambio\footnote{De divisas, supongo.} & - & - \\
16. Deterioro y resultados por enajenaciones de instrumentos financieros & - & - \\ \midrule
\textbf{B) Resultado financiero} (BAAT) ($\sum_{12}^{16}$) & \textbf{210} & \textbf{0} \\ \midrule
\textbf{C) Resultado antes de impuestos} (BAT) ($A + B$) & \textbf{460} & \textbf{560} \\ \midrule
17. Impuestos sobre beneficios & 35 & \\ \midrule
\textbf{D) Resultado del ejercicio} ($C + 17$) & \textbf{495} & \textbf{560} \\ \bottomrule
\end{tabular}
\caption{Tabla de pérdidas y ganancias: el resultado final es de ganancias de 65. \href{http://www.plangeneralcontable.com/?tit=guia-del-pgc-de-pymes&name=GeTia&contentId=man_pgcpym&manPage=26}{Aquí hay} una explicación algo más detallada de los conceptos. Los \texttt{/BA+I?T/} son abreviaciones \textit{made in FMM} que no consigo saber qué son.}
\label{tab:PyG}
\end{minipage}
\end{table}

Un concepto relevante en el PyG es la \concept{Amortización}, que permite ajustar las cuentas teniendo en cuenta la depreciación de los activos. Por ejemplo, si compramos un coche tenemos que tener en cuenta que su precio disminuye con el tiempo. La amortización implica restarle al valor de ese activo un cierto porcentaje cada año.

% TODO: DEfinición de PYME.

El balance de pérdidas y ganancias es importante para Hacienda por el \concept{Impuesto\IS de sociedades}: es un impuesto sobre los beneficios de las empresas. En España está regulado por la Ley 27/2014, de 27 de noviembre, y es del 30\% para grandes empresas y 25\% para PYMEs. En País Vasco y Navarra, que cuentan con el concierto económico y por lo tanto Hacienda propia, los gobiernos autonómicos pueden cambiar los impuestos. Navarra mantiene los tipos del resto de España, mientras que el País Vasco \href{http://www.ogasun.ejgv.euskadi.eus/r51-341/es/contenidos/informacion/6901/es_2316/es_12215.html}{impone tipos reducidos} del 28 \% y 24 \% para grandes empresas respectivamente.

\subsubsection{Caja}

La caja es el estado más sencillo de los tres que hemos visto: refleja el flujo de dinero real en la empresa, pagos y cobros. No vamos a poner un ejemplo porque es trivial y ya llevo muchas tablas. Lo más sutil es que podemos dividir los flujos en operacionales (relacionados con la explotación, como ventas o compras a proveedores) y no operacionales (por ejemplo, ampliaciones de capital, operaciones financieras, ventas de activos no corrientes o dividendos).

\subsection{Circuito contable integrado}

Uno puede simplificar las cuentas en un circuito contable integrado, cuya única mención en Internet es \href{http://maestremiranda.com/techdir/wp-content/uploads/2015/10/CircuitoContable.pdf}{en maestremiranda.com}. Sencillo en cualquier caso.

\subsection{Análisis de ratios}

En la \fref{sec:IndicadoresFinancieros} veíamos algunos indicadores financieros. Para analizar el estado de una empresa usaremos algunos de estos, como el ROE (Return on Equity o rentabilidad financiera, beneficios entre recursos propios) o ROI (Return on investments, rentabilidad económica resultado de explotación entre los activos). La \fref{tab:Ratios} tiene undesglose y ejemplos de todos los ratios. Simplemente hay que hacer una pequeña mención, que es que la rentabilidad económica es igual al margen por la rotación (inmediato al ver las fórmulas).

\begin{table}[hbtp]
\centering
\footnotesize
\begin{tabular}{p{3.5cm}|c|c}
\multicolumn{3}{c}{\textsc{Información}} \\ \toprule
\textbf{Concepto} & \textbf{Emp. A} & \textbf{Emp. B} \\ \toprule
\multicolumn{3}{l}{\textit{Balance}} \\ \midrule
Activo no corriente\footnote{O inmovilizado.} $A_P$ & 1000 & 1000 \\
AC\footnote{Activo corriente, $A_C$ sumando los tres epígrafes.} - Stocks ($A_S$) & 1100 & 50 \\
AC - Clientes & 850 & 50 \\
AC - Tesorería & 50 & 1900 \\
\textbf{Total activo} ($A$) & \textbf{3000} & \textbf{3000} \\ \midrule
Recursos propios\footnote{Patrimonio neto.} ($N$) & 400 & 2600 \\
Recursos ajenos\footnote{Pasivos.} a medio/largo plazo ($P_L$) & 600 & 100 \\
Recursos ajenos a corto plazo ($P_C$) & 2000 & 300 \\
\textbf{Total pasivo} & \textbf{3000} & \textbf{3000} \\ \midrule
\multicolumn{3}{l}{\textit{PyG}} \\ \midrule
Ingresos $(I)$ & 60000 & 3750 \\
- Gastos operativos & 59250 & 3000 \\
= BAIT\footnote{O BAAAIT, no sé.} & 750 & 750 \\
- Intereses ($F$) & 234 & 36 \\
= BAT & 516 & 714 \\
- Impuestos 35 \% & 181 & 250 \\
= Resultado ($B$) & 335 & 464
\end{tabular}
~
\begin{tabular}{p{2.5cm}|c|c|c}
\multicolumn{4}{c}{\textsc{Ratios}} \\ \toprule
\textbf{Ratio} & \textbf{Fórmula} & \textbf{Emp. A} & \textbf{Emp. B} \\ \toprule
\multicolumn{4}{l}{\textit{Rentabilidad}} \\ \midrule
Financiera & $\frac{B}{N}$ & 83.7 \% & 17.8 \% \\
Económica & $\frac{BAIT}{A}$ & 25 \% & 25 \% \\
- Margen & $\frac{BAIT}{I}$ & 1.2 \% & 20 \% \\
- Rotación & $\frac{I}{A}$ & 20 & $1.25$ \\ \midrule
\multicolumn{4}{l}{\textit{Solvencia}} \\ \midrule
Activo comprometido & $\frac{P_L + P_C}{A}$ & 86.6 \% & 13.3\% \\
Cobertura inmovilizado & $\frac{N}{A_P}$ & 0.4 & 2.5 \\ \midrule
\multicolumn{4}{l}{\textit{Liquidez}} \\ \midrule
Ratio de circulante & $\frac{A_C}{P_C}$ & 1 & 6.6 \\
Test ácido & $\frac{A_C - A_S}{P_C}$ & 0.45 & 6.5 \\ \midrule
Coste de la deuda & $\frac{F}{P_L + P_C}$ & 9 \% & 9 \% \\
\end{tabular}
\caption{Análisis de estado financiero de dos empresas con ratios.}
\label{tab:Ratios}
\end{table}

\section{Viabilidad de negocios}
\section{Financiación}
\section{Evaluación de proyectos de inversión}
\section{Mercado y márketing}
\section{Recursos humanos}
\printindex
\end{document}
