\documentclass{apuntes}
\author{Guillermo Julián Moreno}
\date{Septiembre 2013}
\title{Manual paquete extendido \LaTeX}

\begin{document}

\pagestyle{plain}
\maketitle

\section{Comandos}

Comandos definidos en el paquete que sirven tanto para ir más rápido como para luego modificarlos sin tener que cambiar el código en los documentos por cada ocurrencia.

\subsection{Lógica}
Todos los comandos funcionan en modo matemático únicamente.
\begin{itemize}
\item \verb|\dimplies| $\dimplies$
\item \verb|\implies| $\implies$
\item \verb|\Or| $\Or$
\item \verb|\y| $\y$
\item \verb|\tq| $\tq$ (tal que)
\end{itemize}

\subsection{Cálculo}

\begin{itemize}
\item \verb|\deriv{f}{x}| $\deriv{f}{x}$
\item \verb|\dpa{f}{x}| $\dpa{f}{x}$
\item \verb|\rot| $\rot$ (rotacional)
\item \verb|\dv| $\dv$ (divergencia)
\item \verb|\img| $\img$ (imagen)
\end{itemize}

\subsection{Operaciones}
\begin{itemize}
\item \verb|\qeq| $\qeq$
\item \verb|\r{A}| $\r{A}$ Un circulito encima de la letra. A saber para qué sirve.
\item \verb|\comb{1}{2}| $\comb{1}{2}$ Números combinatorios.
\item \verb|\abs{-2}| $\abs{-2}$
\item \verb|\inv{f}| $\inv{f}$
\item \verb|\conj{x}|, \verb|\avg{x}|, \verb|\no{x}| $\avg{x}$ Barrita encima.
\item \verb|\card{A}| $\card{A}$ Cardinal.
\item \verb|\floor{x}| $\floor{x}$ Parte entera.
\item \verb|\pesc{x, y}| $\pesc{x, y}$ Producto escalar.
\item \verb|\md{x}| $\md{x}$ módulo.
\item \verb|\argmin_a| $\argmin_a$
\item \verb|\argmax_a| $\argmax_a$
\end{itemize}

\subsection{Conjuntos y relaciones}
\begin{itemize}
\item \verb|\x| $\x$  Producto cartesiano.
\item \verb|\appl{f}{X}{Y}| $\appl{f}{X}{Y}$ Aplicación
\item \verb|\uexists| $\uexists$ Existe y es único.
\item \verb|\sint| $\sint$ Integral típica.
\item \verb|\stdf| \stdf Aplicación típica (no hay que meterla en modo matemático)
\item \verb|\rel| $\rel$ Operador de relación.
\item \verb|\parts{A}| $\parts{A}$ Partes del conjunto.
\end{itemize}

El paquete también incluye abreviaciones para conjuntos usuales: $\nat, \ent, \rac, \real, \cplex$ con los comandos \verb|\nat|, \verb|\ent|, \verb|\rac|, \verb|\real|, \verb|\cplex| respectivamente.

\subsection{Vectores}

Hay varios vectores definidos para no tener que andar escribiendo. Los comandos son simplemente \verb|\vK|, donde K es la letra del vector. Por ejemplo, \verb|\vx| o \verb|va|.

Vectores definidos: $\vx, \vy, \vf, \va, \vu, \vn, \vv$.

\section{Entornos}
\subsection{Teorema y definición}

Los teoremas del paquete heredan de los teoremas de \texttt{amsthm}, pero además añaden un índice automático de teoremas que se puede imprimir en el documento con \verb|\printtheorems|. También se cargan automáticamente en el glosario de términos. El uso normal es el siguiente

\begin{verbatim}
\begin{theorem}[Teorema de las gallinas cluecas]
Teorema.
\end{theorem}
\end{verbatim}

\begin{theorem}[Teorema de las gallinas cluecas]
Teorema.
\end{theorem}

El título se carga automáticamente como una entrada para el índice. Si, por lo que sea, queremos separar los términos (por ejemplo, para que en este caso aparezca primero \textit{Teorema} y luego, debajo, \textit{de las gallinas cluecas}, ved la última página), no podemos usar la exclamación del comando \verb|\index{}| porque aparecería en el título. En su lugar, podemos usar el comando \verb|\IS|, que actúa como una separación para el índice pero que no aparece en el título.

\begin{verbatim}
\begin{theorem}[Teorema\IS de las gallinas cluecas]
Teorema.
\end{theorem}
\end{verbatim}

\begin{theorem}[Teorema\IS de las gallinas cluecas]
Teorema.
\end{theorem}

Si, por lo que sea, queremos especificar nosotros el índice, podemos añadir un segundo argumento.

\begin{verbatim}
\begin{theorem}[Teorema de las gallinas cluecas][Mi entrada]
Teorema.
\end{theorem}
\end{verbatim}

\begin{theorem}[Teorema\IS de las gallinas cluecas][Mi entrada]
Teorema.
\end{theorem}

Por otra parte tenemos el entorno \texttt{defn} para definiciones, que funciona exactamente igual que \texttt{theorem}:

\begin{verbatim}
\begin{defn}[Título de la definición][(opcional) entrada para el índice]
\end{defn}
\end{verbatim}

\begin{defn}[Título de la definición][(opcional) entrada para el índice]
\end{defn}

\subsection{Auxiliares para los teoremas}

Hay definidos varios entornos similares a teoremas. Todos ellos se pueden titular poniendo entre corchetes el título.

\begin{verbatim}
\begin{lemma}[Lema de la patata]
\end{lemma}
\end{verbatim}

\begin{lemma}[Lema de la patata]
\end{lemma}

\begin{verbatim}
\begin{corol}
\end{corol}
\end{verbatim}

\begin{corol}
\end{corol}

\begin{verbatim}
\begin{prop}
\end{prop}
\end{verbatim}

\begin{prop}
\end{prop}

\begin{verbatim}
\begin{axiom}
\end{axiom}
\end{verbatim}

\begin{axiom}
\end{axiom}

\begin{verbatim}
\begin{proof}
Probado queda.

Al final agrega un cuadradito de QED.
\end{proof}
\end{verbatim}

\begin{proof}
Probado queda. 

Al final agrega un cuadradito de QED.
\end{proof}

\begin{verbatim}
\begin{op}{Nombre de operación}
x = 3 + 1
\end{op}
\end{verbatim}

\begin{op}{Nombre de operación}
x = 3 + \deriv{f}{x} 
\end{op}

El entorno \texttt{op} incluye el modo matemático directamente, y el nombre de operación es obligatorio.
 
\section{Imágenes}

Hay dos comandos para poner fácilmente imágenes. El principal es \texttt{easyimgw}

\begin{verbatim}
\easyimgw{Patata.jpg}{Leyenda}{lblEtiqueta}{0.3}
\end{verbatim}

El último argumento es la anchura de la imagen expresada como proporción de la anchura del texto. $0.3$ significa que ocupa un $30\%$ de la anchura del texto, por ejemplo.

\easyimgw{Patata.jpg}{Leyenda}{lblEtiqueta}{0.3}

También está el comando \verb|\easyimg|, el uso es el mismo salvo que sólo necesita tres argumentos: la anchura se omite y se toma el valor por defecto del $80\%$ de anchura del texto. 
 
 \section{Clase apuntes}
 
 También hay un archivo llamado \texttt{apuntes.cls}, que provee la clase \textit{apuntes}. Básicamente, lo único que hace es cambiar la fuente, ajustar la geometría e incluir el paquete \texttt{exmath}. De esta forma, lo único que hay que hacer para usar todo el paquete y clase es cambiar la clase del documento. Es decir, que la primera línea sea 
 
 \begin{verbatim}
 \documentclass{apuntes}
 \end{verbatim}
 
 Además, la clase genera el título y la cabecera, sólo tenéis que configurar la fecha, título y autor con los siguientes comandos, que deben ir después de la definición de \texttt{documentclass}.
 
 \begin{verbatim}
 \author{Autor}
\date{Fecha}
\title{Título del documento}
 \end{verbatim}
 
\section{Consejos generales de \LaTeX}

\begin{itemize}
\item ¿Símbolos matemáticos demasiado pequeños? Por ejemplo, $\sum_{i=0}^n$. Usa \verb|\displaystyle|, así: \verb|$\displaystyle\sum_{i=0}^n$| que queda mejor: $\displaystyle\sum_{i=0}^n$.
\item Si quieres poner un código LaTeX en el texto, hazlo con \verb|\verb$comandoquesea$| o \verb|\begin{verbatim}código que sea.\end{verbatim}|.
\item El entorno \texttt{wrapfigure} permite crear figuras rodeadas por texto, que queda bien cuando no ocupa  todo el ancho del texto.
\item Usa etiquetas \verb|\label{nombreEtiqueta}| cuando quieras referenciar otras partes del documento con \verb|\ref{nombreEtiqueta}}|. Por ejemplo, si pones una etiqueta debajo de un comando de sección, \verb|\ref{nombreEtiqueta}}| mostrará el número de esa sección. También puedes hacerlo en figuras, tablas y listados de código. Además, tienes el comando \verb|\pageref{nombreEtiqueta}| si lo que quieres mostrar es el número de página donde está la etiqueta.
\end{itemize} 

\subsection{Entornos matemáticos}

El paquete \texttt{amsmath} viene con un buen número de entornos matemáticos para organizar ecuaciones. Lo escribiría aquí, pero son muchos así que te vas a la documentación de \texttt{amsmath}\footnote{\url{ftp://ftp.ams.org/pub/tex/doc/amsmath/amsldoc.pdf}}, sección 3, y ahí lo explica todo muy bien. Los más interesantes: \texttt{align} y \texttt{gather}. Las versiones con asterisco (\texttt{align*}, \texttt{gather*}) no te ponen los números en las ecuaciones.

\newpage
\printindex

\end{document}