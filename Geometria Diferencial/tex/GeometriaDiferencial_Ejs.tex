% -*- root: ../GeometriaDiferencial.tex -*-
\section{Diferentiable Manifolds}

\begin{problem}[1]
Show with details that the real projective space is a differentiable manifold

\solution
\textcolor{blue}{Este ejercicio está prácticamente resuelto en teoría pero lo repetimos aquí con más detalle}

Recordamos que, por definición, un conjunto $X$ es una variedad diferenciable si cumple:
\begin{enumerate}
\item $X$ es un espacio topológico con topología conexa y Hausdorff\footnote{Ver magníficos apuntes de Topología I.}.
\item $U_i ⊂ X$ es una familia numerable de abiertos con aplicaciones $\appl{Φ_i}{U_i}{ℝ^n}$, homomorfismos sobre sus imágenes. El par $(U_i, Φ_i)$ es una carta coordenada.
\item Para todo par de de cartas coordenadas $U_i, U_j$ de la estructura diferencial\footnote{Una variedad queda unívocamente definida por un atlas maximal. Un atlas maximal es un atlas que contiene a todos los atlas que son compatibles con él. Dos atlas son compatibles si son equivalentes} y sus correspondientes homomorfismos $Φ_i, Φ_j$, la función $ \inv{Φ_i} ○ Φ_j $ es difeomorfismo en el entorno en el que está definida (en $U_i ∩ U_j$).
\end{enumerate}

Por otro lado tenemos el plano proyectivo definido como:
\[ \projp^2 ≝ \quot{ℝ^3 \setminus \set{0}}{\sim} \]
donde $\sim$ es una relación de equivalencia definida de la siguiente forma: dados $e, e' ∈ ℝ^3$, están relacionados $e \sim e'$ si y sólo si $∃λ ≠ 0$ tal que $λe = e'$.

Primero hay que dotar de una topología a este espacio. Para ello vamos a definir unos conjuntos:

\[U_i = \{(x_1,x_2,x_3) : x_i ≠ 0\}\]

Esta construcción permite decir que $\projp^2(ℝ) = \displaystyle\bigcup_{i=1}^3 U_i$. En cada conjunto se puede construir el homomorfismo $\appl{Φ_i}{U_i}{ℝ^2}$ llevando $[x_0, x_1, x_2]$ a $\left[\frac{x_0}{x_i},\frac{x_1}{x_i},\frac{x_2}{x_i}\right]$. Dado que una de las coordenadas va a ser $1$ siempre, podemos quitarla y entonces será equivalente a $ℝ^2$.

Definimos la topología del plano proyectivo como:

\[ A⊂U_i, A∈\topl \dimplies Φ_i(A) ∈\topl_{ℝ^n}\]

es decir, un subconjunto de $U_i$ es abierto si y sólo si su imagen por $Φ_i$ es un abierto en $ℝ^n$ con la topología habitual. No es de este curso comprobar que esto realmente define una topología ni que esta topología sea la topología con más abiertos tal que la aplicación $\appl{π}{\real^3 \setminus \set{0}}{\projp^2}$ es continua\footnote{$π$ es la combinación apropiada de $Φ_i$ para que sea continua. Es como la proyección en el plano proyectivo utilizado todas las $Φ_i$}.

Esta topología es conexa y Hausdorff, y también compacta.

Al tener una topología conexa y Hausdorff, el plano proyectivo es una variedad topológica.

Tendríamos que comprobar la tercera propiedad de variedad diferenciable. Para ello vemos que el plano proyectivo tratado como variedad tiene tres cartas, dadas por $(U_i,Φ_i)$.

Si tomamos dos cartas $(U_i, Φ_i), \ (U_j, Φ_j)$ vemos que la intersección de los $U$ es:
\[W = U_i \cap U_j= \{(x_1,x_2,x_3): \ x_i \neq 0 \ x_j \neq 0\}\]

Veamos como se comporta la aplicación $ \inv{Φ_i} ○ Φ_j $  sobre $W$:
\[\inv{Φ_i} ○ Φ_j (W)= \inv{Φ_i} \left(\left\{ \left(\frac{x_1}{x_j}, \frac{x_2}{x_j},\frac{x_3}{x_j}\right)\right\}\right) = \left\{ \left(\frac{x_i \cdot x_1}{x_j}, \frac{x_i \cdot x_2}{x_j},\frac{x_i \cdot x_3}{x_j}\right)\right\} \text{ con } i,j \in \{1,2,3\}\]

Puesto que partimos de la premisa de que $x_i, x_j \neq 0$ tenemos que la función es claramente diferenciable\footnote{cada coordenada es un polinomio de varias variables con un posible denominador no nulo} y puede comprobarse fácilmente que es biyectiva.

La inversa de esta función sería $ Φ_i ○ \inv{Φ_j}$ que se comportaría de la siguiente forma:
\[Φ_i ○ \inv{Φ_j}\left( \{(y_1, y_2, y_3)\}\right) =Φ_i  \left( \{y_1x_j, y_2x_j, y_3x_j\}\right) = \left\{ \left(\frac{y_1x_j}{x_i}, \frac{y_2x_j}{x_i}, \frac{y_3x_j}{x_i} \right)\right\}\]

Esta función sólo estará definida en aquellos puntos en que $x_i\neq$ por lo que resulta ser también diferenciable y vuelve a ser sencilla la comprobación de que la función es biyectiva por lo que queda claro que es un difeomorfismo y por tanto \textbf{el plano proyectivo es una variedad diferencial}

\end{problem}

\begin{problem}[3]
Let $\appl{γ}{M}{N}$ be a differentiable map. Show that the definition of the differential $\appl{dγ_p}{\tgs_p M}{\tgs_{γ(p)} N}$ ofγ at $p$ does not depend on the choice of the curve and that $dγ_p$ is a linear map.

\solution
\textcolor{blue}{Hecho por Pedro. No fiarse al 100\%}

Por definición tenemos
\[dγ_p(v) \in \tgs_{α(p)} N = (γ\circ α)'(0)\]
siendo
\[\appl{α}{(-ε, ε)}{M} \text{ con } α(0)=p, α'(0)=v\]

Sabemos que
\[(γ\circ α)'(0) = α'(0)(γ'(α(0)) = v \cdot γ'(p)\]

sin importar la curva $α$ tomada siempre y cuando cumpla las condiciones pedidas.

Ahora es sencillo ver que es lineal pues tenemos que es una aplicación que, dado un vector $v$ nos lleva a ese mismo vector multiplicado por un valor constante por lo que es lineal de manera trivial.
\end{problem}

\begin{problem}[4]
Let $\appl{γ}{M}{N}$ be an inmersion and let $p$ be a point in $M$. Show that there exists a neighborhood $V \subset M$ of $p$ such that the restriction $γ|_V$ is an embedding (This means that every inmersion is locally an embedding).

\solution
\textcolor{blue}{Hecho por Pedro. No fiarse al 100\%}

Para empezar recordemos lo que era una \textbf{inmersión}: Sean $M$ y $N$ variedades diferenciables. Se dice que una función diferenciable $\appl{F}{M}{N}$ es una \textbf{inmersión} en $p$ si la diferencial $\appl{F_{*p}}{\tgs_pM}{\tgs_{F(p)}N}$ entre los espacios tangentes es inyectiva. Si $F$ es una inmersión para todo punto $p\in M$ decimos que es una \textbf{inmersión}.

Recordemos también lo que era un \textbf{embedding}: Si una \textbf{inmersión} es un homeomorfismo sobre su imagen, con la topología inducida en esta por $N$, decimos que es un \textbf{embedding}

Puesto que toda función diferenciable es continua podemos tomar un entorno de $γ(p)$ y calcular su preimagen, que será un entorno de $p$. La restricción de γ a este entorno será inyectiva, por ser inyectiva γ en general y será sobreyectiva por construcción por lo que será biyectiva. Así podemos garantizar que tendrá inversa diferenciable y por tanto se trata de un homeomorfismo sobre su imagen lo que implica que γ es un \textbf{embedding}
\end{problem}
\newpage
\begin{problem}[6]
Consider the cylinder $C=\{(x,y,z) \in \real^3 \tq x^2+y^2=1\}$ and indentify the point $(x,y,z)$ with $(-x,-y,-z)$. Show that the quotient space of $C$ by this equivalence relation can be given a differentiable structure. (infinite Möbius band)
\solution

\textcolor{blue}{Hecho por Pedro. No fiarse al 100\%}

Primero mostramos un ejemplo encontrado por internet que resuelve un problema muy similar a este.

\begin{center}
\includegraphics[keepaspectratio=true,width=\linewidth]{img/ejemplo_6.png}
\end{center}

Para dotar a un conjunto de estructura diferencial basta con encontrar un atlas para el cual se cumpla la definición de estructura diferencial sobre el conjunto dado.

Tomamos 8 cartas de la siguiente forma:
\begin{enumerate}
\item
\[\appl{\phi_1}{\{(x,y,z) \tq x\in [0,1), y \in [0,1), z \geq 0\}}{\real^3}\]
siendo $\phi_1=Id$
\item
\[\appl{\phi_2}{\{(x,y,z) \tq x\in (-1,0], y \in [0,1), z \geq 0\}}{\real^3}\]
siendo $\phi_2((x,y,z)=(-x,y,z)$
\item
\[\appl{\phi_3}{\{(x,y,z) \tq x\in [0,1), y \in (-1,0], z \geq 0\}}{\real^3}\]
siendo $\phi_3((x,y,z)=(x,-y,z)$
\item
\[\appl{\phi_4}{\{(x,y,z) \tq x\in [0,1), y \in [0,1), z < 0\}}{\real^3}\]
siendo $\phi_4((x,y,z)=(x,y,-z)$
\item
\[\appl{\phi_5}{\{(x,y,z) \tq x\in (-1,0], y \in (-1,0], z \geq 0\}}{\real^3}\]
siendo $\phi_5((x,y,z)=(-x,-y,z)$
\item
\[\appl{\phi_6}{\{(x,y,z) \tq x\in (-1,0], y \in [0,1), z < 0\}}{\real^3}\]
siendo $\phi_6((x,y,z)=(-x,y,-z)$
\item
\[\appl{\phi_7}{\{(x,y,z) \tq x\in [0,1), y \in (-1,0], z < 0\}}{\real^3}\]
siendo $\phi_7((x,y,z)=(x,-y,-z)$
\item
\[\appl{\phi_8}{\{(x,y,z) \tq x\in (-1,0], y \in (-1,0], z < 0\}}{\real^3}\]
siendo $\phi_8((x,y,z)=(-x,-y,-z)$

\end{enumerate}

La combinación de una de estas funciones con la inversa de otra no implicará más que cambios de signo sobre las variables de tal forma que
\[\phi_i\circ \inv{\phi_j}(x,y,z)=(\pm x, \pm y, \pm z\]
y queda claro que estas aplicaciones son $C^{\infty}$ y sus inversas, que son de la misma forma, también. Conviene observar que el jacobiano vale siempre $\pm 1$
\end{problem}