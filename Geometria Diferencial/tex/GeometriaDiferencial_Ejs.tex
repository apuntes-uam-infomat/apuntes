% -*- root: ../GeometriaDiferencial.tex -*-
\section{Diferentiable Manifolds}

\begin{problem}[1]
Show with details that the real projective space is a differentiable manifold

\solution
\textcolor{blue}{Este ejercicio está prácticamente resuelto en teoría pero lo repetimos aquí con más detalle}

Recordamos que, por definición, un conjunto $X$ es una variedad diferenciable si cumple:
\begin{enumerate}
\item $X$ es un espacio topológico con topología conexa y Hausdorff\footnote{Ver magníficos apuntes de Topología I.}.
\item $U_i ⊂ X$ es una familia numerable de abiertos con aplicaciones $\appl{Φ_i}{U_i}{ℝ^n}$, homomorfismos sobre sus imágenes. El par $(U_i, Φ_i)$ es una carta coordenada.
\item Para todo par de de cartas coordenadas $U_i, U_j$ de la estructura diferencial\footnote{Una variedad queda unívocamente definida por un atlas maximal. Un atlas maximal es un atlas que contiene a todos los atlas que son compatibles con él. Dos atlas son compatibles si son equivalentes} y sus correspondientes homomorfismos $Φ_i, Φ_j$, la función $ \inv{Φ_i} ○ Φ_j $ es difeomorfismo en el entorno en el que está definida (en $U_i ∩ U_j$).
\end{enumerate}

Por otro lado tenemos el plano proyectivo definido como:
\[ \projp^2 ≝ \quot{ℝ^3 \setminus \set{0}}{\sim} \]
donde $\sim$ es una relación de equivalencia definida de la siguiente forma: dados $e, e' ∈ ℝ^3$, están relacionados $e \sim e'$ si y sólo si $∃λ ≠ 0$ tal que $λe = e'$.

Primero hay que dotar de una topología a este espacio. Para ello vamos a definir unos conjuntos:

\[U_i = \{(x_1,x_2,x_3) : x_i ≠ 0\}\]

Esta construcción permite decir que $\projp^2(ℝ) = \displaystyle\bigcup_{i=1}^3 U_i$. En cada conjunto se puede construir el homomorfismo $\appl{Φ_i}{U_i}{ℝ^2}$ llevando $[x_0, x_1, x_2]$ a $\left[\frac{x_0}{x_i},\frac{x_1}{x_i},\frac{x_2}{x_i}\right]$. Dado que una de las coordenadas va a ser $1$ siempre, podemos quitarla y entonces será equivalente a $ℝ^2$.

Definimos la topología del plano proyectivo como:

\[ A⊂U_i, A∈\topl \dimplies Φ_i(A) ∈\topl_{ℝ^n}\]

es decir, un subconjunto de $U_i$ es abierto si y sólo si su imagen por $Φ_i$ es un abierto en $ℝ^n$ con la topología habitual. No es de este curso comprobar que esto realmente define una topología ni que esta topología sea la topología con más abiertos tal que la aplicación $\appl{π}{\real^3 \setminus \set{0}}{\projp^2}$ es continua\footnote{$π$ es la combinación apropiada de $Φ_i$ para que sea continua. Es como la proyección en el plano proyectivo utilizado todas las $Φ_i$}.

Esta topología es conexa y Hausdorff, y también compacta.

Al tener una topología conexa y Hausdorff, el plano proyectivo es una variedad topológica.

Tendríamos que comprobar la tercera propiedad de variedad diferenciable. Para ello vemos que el plano proyectivo tratado como variedad tiene tres cartas, dadas por $(U_i,Φ_i)$.

Si tomamos dos cartas $(U_i, Φ_i), \ (U_j, Φ_j)$ vemos que la intersección de los $U$ es:
\[W = U_i \cap U_j= \{(x_1,x_2,x_3): \ x_i \neq 0 \ x_j \neq 0\}\]

Veamos como se comporta la aplicación $ \inv{Φ_i} ○ Φ_j $  sobre $W$:
\[\inv{Φ_i} ○ Φ_j (W)= \inv{Φ_i} \left(\left\{ \left(\frac{x_1}{x_j}, \frac{x_2}{x_j},\frac{x_3}{x_j}\right)\right\}\right) = \left\{ \left(\frac{x_i \cdot x_1}{x_j}, \frac{x_i \cdot x_2}{x_j},\frac{x_i \cdot x_3}{x_j}\right)\right\} \] con $i,j \in \set{1,2,3}$.

Puesto que partimos de la premisa de que $x_i, x_j \neq 0$ tenemos que la función es claramente diferenciable\footnote{cada coordenada es un polinomio de varias variables con un posible denominador no nulo} y puede comprobarse fácilmente que es biyectiva.

La inversa de esta función sería $ Φ_i ○ \inv{Φ_j}$ que se comportaría de la siguiente forma:
\[Φ_i ○ \inv{Φ_j}\left( \{(y_1, y_2, y_3)\}\right) =Φ_i  \left( \{y_1x_j, y_2x_j, y_3x_j\}\right) = \left\{ \left(\frac{y_1x_j}{x_i}, \frac{y_2x_j}{x_i}, \frac{y_3x_j}{x_i} \right)\right\}\]

Esta función sólo estará definida en aquellos puntos en que $x_i\neq$ por lo que resulta ser también diferenciable y vuelve a ser sencilla la comprobación de que la función es biyectiva por lo que queda claro que es un difeomorfismo y por tanto \textbf{el plano proyectivo es una variedad diferencial}

\end{problem}

\begin{problem}[3]
Let $\appl{γ}{M}{N}$ be a differentiable map. Show that the definition of the differential $\appl{dγ_p}{\tgs_p M}{\tgs_{γ(p)} N}$ ofγ at $p$ does not depend on the choice of the curve and that $dγ_p$ is a linear map.

\solution
\textcolor{blue}{Hecho por Pedro. No fiarse al 100\%}

Por definición tenemos
\[dγ_p(v) \in \tgs_{α(p)} N = (γ\circ α)'(0)\]
siendo
\[\appl{α}{(-ε, ε)}{M} \text{ con } α(0)=p, α'(0)=v\]

Sabemos que
\[(γ\circ α)'(0) = α'(0)(γ'(α(0)) = v \cdot γ'(p)\]

sin importar la curva $α$ tomada siempre y cuando cumpla las condiciones pedidas.

Ahora es sencillo ver que es lineal pues tenemos que es una aplicación que, dado un vector $v$ nos lleva a ese mismo vector multiplicado por un valor constante por lo que es lineal de manera trivial.
\end{problem}

\begin{problem}[4]
Let $\appl{γ}{M}{N}$ be an inmersion and let $p$ be a point in $M$. Show that there exists a neighborhood $V \subset M$ of $p$ such that the restriction $γ|_V$ is an embedding (This means that every inmersion is locally an embedding).

\solution
\textcolor{blue}{Hecho por Pedro. No fiarse al 100\%}

Para empezar recordemos lo que era una \textbf{inmersión}: Sean $M$ y $N$ variedades diferenciables. Se dice que una función diferenciable $\appl{F}{M}{N}$ es una \textbf{inmersión} en $p$ si la diferencial $\appl{F_{*p}}{\tgs_pM}{\tgs_{F(p)}N}$ entre los espacios tangentes es inyectiva. Si $F$ es una inmersión para todo punto $p\in M$ decimos que es una \textbf{inmersión}.

Recordemos también lo que era un \textbf{embedding}: Si una \textbf{inmersión} es un homeomorfismo sobre su imagen, con la topología inducida en esta por $N$, decimos que es un \textbf{embedding}

Puesto que toda función diferenciable es continua podemos tomar un entorno de $γ(p)$ y calcular su preimagen, que será un entorno de $p$. La restricción de γ a este entorno será inyectiva, por ser inyectiva γ en general y será sobreyectiva por construcción por lo que será biyectiva. Así podemos garantizar que tendrá inversa diferenciable y por tanto se trata de un homeomorfismo sobre su imagen lo que implica que γ es un \textbf{embedding}
\end{problem}
\newpage
\begin{problem}[6]
Consider the cylinder $C=\{(x,y,z) \in \real^3 \tq x^2+y^2=1\}$ and indentify the point $(x,y,z)$ with $(-x,-y,-z)$. Show that the quotient space of $C$ by this equivalence relation can be given a differentiable structure. (infinite Möbius band)
\solution

\textcolor{blue}{Hecho por Pedro. No fiarse al 100\%}

Primero mostramos un ejemplo encontrado por internet que resuelve un problema muy similar a este.

\begin{center}
\includegraphics[keepaspectratio=true,width=\linewidth]{img/ejemplo_6.png}
\end{center}

Para dotar a un conjunto de estructura diferencial basta con encontrar un atlas para el cual se cumpla la definición de estructura diferencial sobre el conjunto dado.

Tomamos 8 cartas de la siguiente forma:
\begin{enumerate}
\item
\[\appl{\phi_1}{\{(x,y,z) \tq x\in [0,1), y \in [0,1), z \geq 0\}}{\real^3}\]
siendo $\phi_1=Id$
\item
\[\appl{\phi_2}{\{(x,y,z) \tq x\in (-1,0], y \in [0,1), z \geq 0\}}{\real^3}\]
siendo $\phi_2((x,y,z)=(-x,y,z)$
\item
\[\appl{\phi_3}{\{(x,y,z) \tq x\in [0,1), y \in (-1,0], z \geq 0\}}{\real^3}\]
siendo $\phi_3((x,y,z)=(x,-y,z)$
\item
\[\appl{\phi_4}{\{(x,y,z) \tq x\in [0,1), y \in [0,1), z < 0\}}{\real^3}\]
siendo $\phi_4((x,y,z)=(x,y,-z)$
\item
\[\appl{\phi_5}{\{(x,y,z) \tq x\in (-1,0], y \in (-1,0], z \geq 0\}}{\real^3}\]
siendo $\phi_5((x,y,z)=(-x,-y,z)$
\item
\[\appl{\phi_6}{\{(x,y,z) \tq x\in (-1,0], y \in [0,1), z < 0\}}{\real^3}\]
siendo $\phi_6((x,y,z)=(-x,y,-z)$
\item
\[\appl{\phi_7}{\{(x,y,z) \tq x\in [0,1), y \in (-1,0], z < 0\}}{\real^3}\]
siendo $\phi_7((x,y,z)=(x,-y,-z)$
\item
\[\appl{\phi_8}{\{(x,y,z) \tq x\in (-1,0], y \in (-1,0], z < 0\}}{\real^3}\]
siendo $\phi_8((x,y,z)=(-x,-y,-z)$

\end{enumerate}

La combinación de una de estas funciones con la inversa de otra no implicará más que cambios de signo sobre las variables de tal forma que
\[\phi_i\circ \inv{\phi_j}(x,y,z)=(\pm x, \pm y, \pm z\]
y queda claro que estas aplicaciones son $C^{\infty}$ y sus inversas, que son de la misma forma, también. Conviene observar que el jacobiano vale siempre $\pm 1$
\end{problem}

\section{fasc-4-ejemplos}

\subsection{Variedades}
\begin{problem}[2]
Estudiar, siguiendo el modelo de $S^2$ la estructura de variedad diferenciable, con dos cartas, en $S^n$

\solution
\textcolor{blue}{Hecho por Pedro. No fiarse al 100\%}

Comenzamos considerando una esfera de radio 1 en $\real^n$ que tendrá la ecuación:
\[\sum_{i=1}^n x_i^2=1\]
y describiendo explícitamente el atlas de dos cartas dado por la proyección estereográfica.

Siguiendo el ejemplo de la hoja, consideramos la proyección tomando el polo norte $(1,0...0)$ y el plano $x_1=-1$. Posteriormente construiremos la segunda carta tomando el polo sur $(-1,0...0)$ y el plano $x_1=1$. Vamos a ello.

\textbf{Primera carta}

\begin{itemize}
\item Supongamos un punto $p$ cualquiera del plano $x_1=-1$:
\[p=(-1,x_2,...x_n)\]

Si construimos la recta que lo une con el polo norte y la intersecamos con la esfera $S^n$ obtenemos el punto intersección $q$.
\[q = \left(1-2t, x_2\cdot t,...,x_n\cdot t\right)\]
Si el punto es la intersección con la esfera, su módulo deberá ser 1. Utilizamos este dato para calcular $t$.

\[1+4t^2-4t+t^2\left(\sum_{i=2}^nx_i^2\right)=1 \implies 4t^2-4t+t^2\left(\sum_{i=2}^nx_i^2\right) = 0 \implies t=\frac{4}{\sum_{i=2}^nx_i^2+4}\]

Por tanto el punto de intersección es:
\[q=\left(1-\frac{8}{\sum_{i=2}^nx_i^2+4}, \frac{4x_2}{\sum_{i=2}^nx_i^2+4},...,\frac{4x_n}{\sum_{i=2}^nx_i^2+4} \right)\]

\item Al revés. Empezamos ahora con un punto
\[x\in S^n\tq x=(α_1...α_n) \text{ con } \sum_{i=1}^n α_i^2 = 1\]

Construimos ahora el vector que une este punto con el polo norte y lo intersecamos con el plano $x_1=-1$

La recta unión con el polo norte queda de la forma:
\[\left(1+t(α_1+1),α_2t,...,α_nt \right)\]
y forzamos la intersección de esta recta con el plano para conocer el punto
\[1+t(α_1+1)=-1 \implies t = \frac{-2}{α_1+1}\]
con lo que el punto sería:
\[\left( -1, \frac{-2α_2}{α_1-1}, ..., \frac{-2α_n}{α_1-1}\right)=\left( -1, \frac{2α_2}{1-α_1}, ..., \frac{2α_n}{1-α_1}\right)\footnote{En el ejmplo de la hoja creo que escriben $α_1$ en función de las otras coordenadas, pero no veo necesaria esta complicación}\]
\end{itemize}

\textbf{Segunda carta}
\begin{itemize}
\item Supongamos un punto $p$ cualquiera del plano $x_1=1$:
\[p=(1,x_2,...x_n)\]

Si construimos la recta que lo une con el polo sur y la intersecamos con la esfera $S^n$ obtenemos el punto intersección $q$.
\[q = \left(-1+2t, x_2\cdot t,...,x_n\cdot t\right)\]
Si el punto es la intersección con la esfera, su módulo deberá ser 1. Utilizamos este dato para calcular $t$.

\[1+4t^2-4t+t^2\left(\sum_{i=2}^nx_i^2\right)=1 \implies 4t^2-4t+t^2\left(\sum_{i=2}^nx_i^2\right) = 0 \implies t=\frac{4}{\sum_{i=2}^nx_i^2+4}\]

Por tanto el punto de intersección es:
\[q=\left(-1+\frac{8}{\sum_{i=2}^nx_i^2+4}, \frac{4x_2}{\sum_{i=2}^nx_i^2+4},...,\frac{4x_n}{\sum_{i=2}^nx_i^2+4} \right)\]

\item Al revés. Empezamos ahora con un punto
\[x\in S^n\tq x=(α_1...α_n) \text{ con } \sum_{i=1}^n α_i^2 = 1\]

Construimos ahora el vector que une este punto con el polo sur y lo intersecamos con el plano $x_1=1$

La recta unión con el polo sur queda de la forma:
\[\left(-1+t(α_1+1),α_2t,...,α_nt \right)\]
y forzamos la intersección de esta recta con el plano para conocer el punto
\[-1+t(α_1+1)=1 \implies t = \frac{2}{α_1+1}\]
con lo que el punto sería:
\[\left( 1, \frac{-2α_2}{α_1+1}, ..., \frac{-2α_n}{α_1+1}\right))\]
\end{itemize}

Estudiamos ahora un punto cualquiera del plano $(-1,...x_n)$. Si lo mandamos en la esfera por la primera proyección que hemos calculado, llegamos al punto
\[\left(1-\frac{8}{\sum_{i=2}^nx_i^2+4}, \frac{4x_2}{\sum_{i=2}^nx_i^2+4},...,\frac{4x_n}{\sum_{i=2}^nx_i^2+4} \right)\]

Ahora calculamos la imagen directa de este punto por la segunda proyección estereográfica, con lo que llegamos a:
\[\left( 1, \frac{-2\frac{4x_2}{\sum_{i=2}^nx_i^2+4}}{2-\frac{8}{\sum_{i=2}^nx_i^2+4}}, ..., \frac{-2\frac{4x_n}{\sum_{i=2}^nx_i^2+4}}{2-\frac{8}{\sum_{i=2}^nx_i^2+4}}\right)=\left( 1, \frac{-4x_2}{\sum_{i=2}^nx_i^2},...,\frac{-4x_n}{\sum_{i=2}^nx_i^2}\right)\]

y podemos ver que se trata de un difeomorfismo sobre su imagen

\textcolor{blue}{En algún punto he metido la gamba con los signos porque debería salirme todo positivo. Pero ya he currado mucho con este ejercicio. Una paja y a seguir.}

\end{problem}



\begin{problem}[3]
Parametrizamos los puntos del hemisferio norte de la esfera $x_2 > 0$,
excluyendo el ecuador ($x_2 = 0$), en la forma $(θ, τ, + \sqrt{1 − θ^2 − τ^2})$ con
$θ^2 + τ^2 < 1$. Esto convierte al hemisferio norte en una carta coordenada, y podemos cubrir la esfera con seis cartas de este tipo tomando
como ecuador cada una de las intersecciones de la esfera con los planos
coordenados. COMPROBAR que, efectivamente, se trata de un atlas.

DEMOSTRAR que los dos atlas son equivalentes, viendo que los cambios
de carta de una carta de uno de ellos a una carta del otro inducen difeomorfismos.

\solution
\end{problem}

\begin{problem}[4]
Demostrar que no existe ninguna variedad diferenciable compacta que se pueda recubrir con una única carta coordenada
\solution

\textcolor{blue}{Hecho por mi. No fiarse al 100\%}

Si una variedad se puede recubrir por una única carta será homeomorfa a un abierto de $\real^n$ y, consecuentemente, nunca podrá ser compacta.

Esta respuestá está basada en la proposición 1.28 del documento: \href{http://ocw.um.es/ciencias/geometria-y-topologia/material-de-clase-1/01-variedadesdiferenciablessubvariedades-v100901.pdf}{Variedades Diferenciales y Subvariedades.pdf}
\end{problem}

\begin{problem}[5]
Demostrar que toda variedad 1-dimensional compacta y conexa es difeomorfa a la circunferencia $S^2$
\solution

\textcolor{blue}{Hecho por mi. No fiarse al 100\%}

Para empezar es obvio que en caso de haber un difeomorfismo entre una variedad y la circunferencia, la variedad ha de ser compacta y conexa, pues así lo es la circunferencia.

Para resolver este ejercicio me baso en la proposición \href{https://books.google.es/books?id=CAOjRFAMJFUC&pg=PA131&lpg=PA131&dq=variedad+1-dimensional&source=bl&ots=MLkMP7HMyo&sig=aLLSSaYknqPZhPsn-5jJM2MwPAc&hl=es&sa=X&ei=DeMeVZSOEczZPdLkgfgJ&ved=0CCcQ6AEwAQ#v=onepage&q&f=false}{VI.1.6}. Básicamente la copio como un bellaco pero ahí dejo el link para el que quiera profundizar.

\textcolor{blue}{Justo en la versión que se puede consultar gratis en Google han quitado las dos páginas clave en que salía esto. El lunes pillo el libro en ciencias y lo completo}

\end{problem}

\begin{problem}[6]
Demostrar, como consecuencia del teorema de invarianza del dominio, que si $n \neq m$, no pueden ser homeomorfos $\real^n$ y $\real^m$. Tampoco es posible que sean homeomorfos un abierto de $\real^n$ con uno de $\real^m$
\solution

\textcolor{blue}{Hecho por Pedro. No fiarse al 100\%}

La solución se ha obtenido de este \href{http://www.cmat.edu.uy/~rpotrie/documentos/pdfs/invarianciadimension.pdf}{documentopdf}

\textcolor{blue}{Esta demostración se hace prácticamente usando los conocimiento del curso de Topología y no veo una forma clara de hacerlo con lo estudiado en este curso de Geometría Diferencial.}
\end{problem}

\begin{problem}[7]
Demostrar que, si $\appl{f}{M}{N}$ es una función continua y localmente inyectiva de variedades topologícas de dimensión $n$, entonces la imagen de todo abierto $U \subset M$ es un abierto de $N$. En particular, $f(M)\subset N$ debe ser un abierto, que puede ser toda la variedad $N$
\solution

\textcolor{blue}{Hecho por Pedro. No fiarse al 100\%}

Esto es un resultado de Topología que no voy a rehacer ya que no le veo mucha relación con lo que estamos estudiando en esta asignatura.

Por comentar algo relacionado con lo que estamos viendo, he viso en Wikipedia, y cito textualmente: ``El teorema de la invariancia del dominio establece que una función continua y localmente inyectiva entre dos variedades topológicas n-dimensionales debe ser abierta.''
\end{problem}

\begin{problem}[8]
Sea $S$ el conjunto de sucesiones $\appl{σ}{\nat}{\real}$ de números reales.

Definimos una topología en $S$ exigiendo que todas las funciones naturales de proyección
\[\appl{μ_{i_1,i_2,...,i_n}}{S}{\real^n}\]
sean continuas.\footnote{Estas funciones básicamente llevan la sucesión a un vector de $\real^n$ formado por $n$ elementos de la sucesión.}

Definimos una función $\appl{F}{S}{S}$ mediante
\[F(x_0,x_1,...,x_n,...)=(x_1,x_2,...,x_n,...)\]
Demostrar que la función $F$ es continua e inyectia, pero su imagen no es un abierto de $S$

\solution


\end{problem}

\begin{problem}[9]
Sea $U$ una bola unidad abierta en $\real^n$. Comprobar que la aplicación
\[f(\vx)=\frac{\vx}{\sqrt{1-\norm{\vx}^2}}\]
es un difeomorfismo de $U$ en todo $\real^n$
\solution

\yoP

Para comprobar que $f$ es un difeomorfismo debemos ver que es diferenciable, biyectiva y con inversa diferenciable. Vamos a ello.

La función, descompuesta en coordenadas, queda de la forma:
\[f(\vx)=\left(\frac{x_1}{\sqrt{1-\norm{\vx}}}, \frac{x_2}{\sqrt{1-\norm{\vx}}}, ..., \frac{x_n}{\sqrt{1-\norm{\vx}}}\right)\]

Vamos a derivarla:
\[\frac{\partial f}{\partial x_i}=\left(\frac{x_1x_i}{(\sqrt{1-\norm{\vx}})^3},\frac{x_2x_i}{(\sqrt{1-\norm{\vx}})^3},..\frac{1}{\sqrt{1-\norm{\vx}}}+\frac{x_i^2}{(\sqrt{1-\norm{\vx}})^3}.,\frac{x_nx_i}{(\sqrt{1-\norm{\vx}})^3}\right)\]

Podemos observar que la derivada existe (y por tanto la función es diferenciable) en todo punto con norma distinta de 1. Puesto que nos estamos restringiendo a $U$ que es la bola unidad \textbf{abierta} no hay problema con esto.

Podemos ver de manera sencilla que la función es inyectvia pues la derivada nunca se anula y podemos ver que es sobreyectiva por contrucción. Para cualquier punto de $\real^n$ que tomemos podemos escribirlo como $f(\vx)$\footnote{Esto queda claro al calcular la función inversa, cosa que haremos a continuación}.

Por último nos queda estudiar la inversa.

\[f(\vx)=\vy \implies (y_1,....y_n)=\left(\frac{x_1}{\sqrt{1-\norm{\vx}}}, \frac{x_2}{\sqrt{1-\norm{\vx}}}, ..., \frac{x_n}{\sqrt{1-\norm{\vx}}}\right)\]

A ojo de buen cubero podemos ver que la función inversa sería
\[f^{-1}(\vy)=\frac{\vy}{\sqrt{1+\norm{\vy}}}\]
cosa que podemos comprobar fácilmente.

Resulta también trivial la observación de que esta función inversa también es diferenciable, por lo que queda claro que $f$ es un difeomorfismo.

\end{problem}


\subsection{Campos}
\begin{problem}[1]
(\textbf{Grupos uniparamétricos de automorfismos}) En cada uno de los siguientes ejemplos se da una familia uniparamétrica de automorfismos de $\real^n$ y se pide que se compruebe que es un grupo y determine el campo asociado (\textbf{generador infinitesimal del grupo}). En todos los casos el grupo es un grupo de transformaciones lineales y el campo es un campo lineal (coeficientes de grado a lo más uno)

\ppart Grupo de traslaciones
\[τ_1(x_1,...,x_n)=(x_1+t,x_2,...,x_n)\]

\ppart Grupo de homotecias
\[τ_t(x_1,...,x_n)=(e^tx_1, e^tx_2,...,e^tx_n\]

\ppart Sea $A$ una matriz $n\times n$. Definimo un grupo uniparamétrico de automorfismos, asociado a la matriz $A$, mediante
\[τ_t(X)=e^{tA}X\]
¿Es el primer ejemplo un caso particular de este?

\ppart
Supongamos ahora $n=2$, y sea
\[ \left( \begin{array}{cc}

\cos(t) & -\sin(t) \\
\sin(t) & \cos(t)

\end{array} \right)\]
la matriz de una rotación de ángulo variable $t$ en el plano. Definimos un grupo uniparamétrico de automorfismos
\[τ_t(x_1,x_2)=A(t)\cdot {x_1 \choose x_2}\]

¿Es este ejemplo un caso particular del c)?

\solution
\yoP

Para demostrar que cada uno de estos conjuntos es un grupo debemos identificar la operación que define el grupo, demostrar que es asociativa y encontrar el elemento neutro o identidad y el inverso.

\spart
La operación suma se define de la siguiente forma:
\[(τ_a+τ_b)(x_1,...,x_n)=(x_1+a+b,x_2,...,x_n)\]

Es evidente que la operación es asociativa, el elemento neutro es $τ_0$ y el inverso de $τ_t$ es $τ_{-t}$

Queda claro que es un grupo.

Vamos a encontrar el campo asociado. La teoría de lo que vamos a hacer a continuación viene en las páginas 73-74 de \href{http://matematicas.unex.es/~ricarfr/EcDiferenciales/LibroEDLat.pdf}{documento.pdf}

Básicamente nos da un teorema y su demostración. El teorema dice:

Sea $X$ un grupo uniparamétrico local de clase $k$. Para cada $f\in C^{\infty}(U)$ y $p \in U$ definimos
\[(Df)(p)=\lim_{t \to 0}\frac{f[X(t,p)]-f(p)}{t}\]
entonces $D \in D_k(U)$ y lo llamaremos generador infinitesimal de X.

El ejemplo concreto que vamos a hacer (o uno muy similar) así como los siguientes vienen resuletos al final de la página 16 de \href{http://matematicas.unex.es/~ricarfr/EcDiferenciales/LibroEDLat.pdf}{documento.pdf}

Así para este ejemplo tendríamos la aplicación flujo $\appl{\phi}{\real^{n+1}}{\real^n}$ siendo $\phi(x_1,...,x_n,t)=(x_1+t,x_2,...,x_n)$.

\[X = \left(\frac{\partial \phi}{\partial t}\right)_{t=0} \frac{\partial}{\partial x_i}=\frac{\partial}{\partial x}\]


\textbf{Del último documento mencionado cabe destacar:}

\begin{defn}[Generador infinitesimal]
Sea $\{σ_t\}$ un grupo monoparamétrico de transformaciones de U. Llamamos \textbf{generador infinitesimal} de $\{σ_t\}$ al campo vectorial que asigna al punto $p$, el vector tangente a la curva $(-ε,ε)\to U$, $t \to σ_t(p)$

Consideremos $\appl{\phi}{\real \times U}{U}$ la aplicación flujo del grupo $\phi(t,p)=σ_t(p)$. entonces el generador infinitesimal puede considerarse
\[X = \sum_{i=1}^n \left( \frac{\partial \phi_i}{\partial t}\right)_{t=0}\frac{\partial}{\partial x_i}\]
\end{defn}

\end{problem}

\begin{problem}[2]
Sea $D$ el campo en $\real^2$ definido como
\[D=x \frac{\partial}{\partial x}+y \frac{\partial}{\partial y }\]

El campo está definido en tdo el plano pero

\ppart Una integral primera definida en todo el plano es necesariamente constante

\ppart Para cada punto, diferente al origen hay un entorno en el que hay una integral primera. Determnar todas esas integrales primeras no globales y los abiertos máximos en que están definidas.
\solution
\end{problem}

\begin{problem}[3]
Estudiar el campo en el plano definido como
\[D=(y+x(1-x^2-y^2))\frac{\partial}{\partial x} + (-x+y(1-x^2-y^2))\frac{\partial}{\partial y}\]
usando, si te parece conveniente, un cambio a coordenadas polares.
\solution

\end{problem}
