%
% Soluciones a los ejercicios de Probabilidad II.
%
% Curso 2014 - 2015 2º cuatrimestre
%

%%%%%%%%%%%%%%%%%%%%%%%%%%%%%%%%%%%%%%%%%%%%%%%%%%%%%%%%%%%%%%%%%%%%%%%%%%%%%%%
\newpage
\section{A petit example}


%%%%%%%%%%%%%%%%%%%%%%%%%%%%%%%%%%%%%%%%%%%%%%%%%%%%%%%%%%%%%%%%%%%%%%%%%%%%%%%
\begin{problem}[5]Determina si es verdadero o falso:

\ppart Si añadimos 7 a todos los datos de un conjunto, el primer cuartil aumenta en 7 unidades y el rango intercuartílico no cambia.

\ppart Si todos los datos de un conjunto se multiplican por -2, la desviación típica se dobla.

\ppart Si todos los datos de un conjunto se multiplican por 2, la varianza se dobla.

\ppart Al multiplicar por tres todos los datos de un conjunto, el coeficiente de asimetría no varía

\ppart Si el coeficiente de correlación entre dos variables vale -0.8, los valores por debajo del promedio de una variable están asociados con valores por debajo del promedio de la otra.

\ppart Si $\forall i\,y_i<x_i$ entonces el coeficiente de correlación es negativo.


\ppart Si cambiamos el signo de todos los datos de un conjunto, el coeficiente de asimetría también cambia de signo.

\ppart Al restar una unidad a cada dato de un conjunto, la desviación típica siempre disminuye.

\ppart Si a un conjunto de datos con media $\gx$ se le añade un nuevo dato que coincide con $\gx$, la
media no cambia y la desviación típica disminuye.

\solution 

\spart Falso. Añadir siete a todos los datos es una traslación, así que la distribución de los datos no cambia. El rango intercuartílico se mantiene y el cuantil también.

\spart Teniendo en cuenta que si multiplicamos todos los datos del conjunto por $-2$ la media también se multiplica por $-2$, y sustituyendo en la fórmula de la varianza:

\[ \sigma' = \sqrt{\frac{1}{n} \sum_{i=1}n (-2x_i)^2 - (-2\avg{x})^2} = \sqrt{\frac{1}{n} \sum_{i=1}4\left(n x_i^2 - \avg{x}^2\right)} = \sqrt{4\sigma^2} = 2\sigma \]

Por lo tanto, la desviación típica sí se dobla.

\spart Usando los cálculos del apartado anterior vemos que la varianza se multiplica por cuatro.

\spart Efectivamente: cambiar el signo haría una reflexión de los datos sobre el eje Y y la asimetría estaría orientada hacia el lado contrario. 

\spart  Teniendo en cuenta que si multiplicamos todos los datos del conjunto por $3$ la media también se multiplica por $3$

El coeficiente de asimetría se calcula:

\[\frac{1}{n} \sum_{i=1}^n (x_i-\gx)^3\]

Sustituyendo en la fórmula del coeficiente de asimetría

\[\frac{1}{n} \sum_{i=1}^n (3x_i - 3\gx)^3 = \frac{1}{n} \sum_{i=1}^n 3^3 (x_i-\gx)^3 = 27 \cdot \frac{1}{n} \sum_{i=1}^n (x-\gx)^3\]

Por lo tanto el coeficiente de asimetría sí varía.

\spart Falso. \[ \hat\sigma^2 = \frac{1}{n} \sum_{j=1}^n (y_j - \avg{y})^2 = \begin{cases} y_j = x_j - 1 \\ \avg{y} = \frac{1}{n} \sum_{j=1}^n (x_j - 1) = \frac{1}{n} ( \sum_{j=1}^n x_j ) - 1 = \avg{x} - 1 \end{cases} \]
\[ = \frac{1}{n} \sum_{j=1}^n (x_j - 1 - ( \avg{x} - 1))^2 = \frac{1}{n} \sum_{j=1}^n (x_j - \avg{x})^2  = \sigma^2 \]

\spart Falso. 2 variables pueden tener una correlación creciente aunque $y_i<x_i$.

\spart Falso. La desviación típica se mantiene (los datos siguen estando ``igual de separados'').

\spart Verdadero. Al hacer el cálculo de la media no varía (en la fórmula del ejercicio 2 se puede comprobar que si añadimos un $x_i=\gx$ el sumatorio de la derecha queda igual) y la desviación típica disminuye.

\end{problem}
