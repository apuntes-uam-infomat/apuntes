\documentclass{apuntes}

\usepackage{hyperref}

\usepackage{tikztools}
\usepackage{fastbuild}
\usepackage{tikz-3dplot}

\usepackage{tikz}
\usepackage{graphicx}
\usepackage{latexsym, amsfonts, amsmath, amssymb, amscd, epsfig,amsthm}
\input xy 
\xyoption{all} %%!!
\usetikzlibrary{calc, intersections}
\author{Alberto Parramon}
\date{2014/2015 2º cuatrimestre}

\renewcommand*{\arraystretch}{1.5}

\title{Probabilidad II}
\precompileTikz

\begin{document}

\pagestyle{plain}
\maketitle

\tableofcontents
\newpage
\chapter{Cositas}

%Los diagramas de Venn que aparecen los he hecho en la siguiente web:
%https://www.gliffy.com/go/html5/launch?app=1b5094b0-6042-11e2-bcfd-0800200c9a66

\section{Evaluación}
P = parcial 26 de Marzo.

F = final Mayo.

NOTA=$max(0.3P+0.7F,F)$ 

\section{El profe}
Jesús Munarriz

jesus.munariz@uam.es

Despacho 205 módulo 8.

Tutorías: L-X-J de 14:30 a 15:30 y también a otras horas mediante cita previa.

\section{Nociones básicas aleatorias}
\begin{itemize}
\item Leyes de Morgan y manejo de conjuntos (se utilizarán en muchas demostraciones):
\begin{enumerate}
\item $(\bigcup_{n=1}^{\infty}A_n)^c = \bigcap_{n=1}^{\infty}A_n^c$
\item $(\bigcap_{n=1}^{\infty}A_n)^c = \bigcup_{n=1}^{\infty}A_n^c$
\item $A \backslash B = A \cap B^c$

\end{enumerate}
\item Sobre funciones indicatrices:\\
$\ind_A(w)=1$ si $w \in A$\\
$\ind_A(w)=0$ si $w \in A^c$

\item $f_+ = max(f,0)$ y $f_- = max(-f,0)$. Y por tanto, $f=f_+ -f_-$ y $\abs{f}=f_+ +f_-$

\item Tma Fundamental del calculo: Dada una función f(x) continua en el intervalo [a,b] y sea F(x) cualquier función primitiva de f, es decir $F '(x) = f(x)$. Entonces:

\[
\int_a^b f(x)dx = F(b)-F(a)
\]
\end{itemize}

\chapter{Espacios de probabilidad}
\section{Formación de un espacio de probabilidad}
En primer lugar vamos a definir y estudiar los elementos por los que está formado un espacio de probabilidad:

\begin{defn}[Algebra de conjuntos]Sea $\Omega$ un espacio muestral (un conjunto), y sea $\algb{M}$ una colección de subconjuntos (eventos $w$) de $\Omega$. $\algb{M}$ es un álgebra si:
\begin{enumerate}
\item $\Omega \in \algb{M}$.
\item $A ∈ \algb{M}$ $\Rightarrow$ $A^c ∈ \algb{M}$. ($A^c = \Omega \setminus A = \{w \in \Omega : w \notin A\} $)
\item $A \in \algb{M}$ y $B \in \algb{M}$ $\Rightarrow$ $A \cup B \in \algb{M}$.  (la unión finita pertenece al álgebra)

\end{enumerate}
\end{defn}

Propiedades:
\begin{enumerate}
\item[3'] $A \in \algb{M}$ y $B \in \algb{M}$ $\Rightarrow$ $A \cap B \in \algb{M}$. (la intersección finita pertenece al álgebra)
\begin{proof}

$3 \Rightarrow 3')$ $A,B \in \algb{M} \stackrel{2}{\Leftrightarrow} A^c, B^c \in \algb{M} \stackrel{3}{\Rightarrow} A^c \cup B^c \in \algb{M} \Leftrightarrow (A \cap B)^c \in \algb{M} \stackrel{2}{\Leftrightarrow} A \cap B \algb{M}$

$3' \Rightarrow 3)$ $A,B \in \algb{M} \stackrel{2}{\Leftrightarrow} A^c, B^c \in \algb{M} \stackrel{3'}{\Rightarrow} A^c \cap B^c \in \algb{M} \Leftrightarrow (A \cup B)^c \in \algb{M} \stackrel{2}{\Leftrightarrow} A \cup B \in \algb{M}$
\end{proof}
\end{enumerate}

\begin{defn}[{σ}-álgebra]Sea $\Omega$ un espacio muestral (un conjunto), y sea $\algb{M}$ una colección de subconjuntos (eventos $w$) de $\Omega$. $\algb{M}$ es una $\salgb$ si:
\begin{enumerate}
\item $\Omega \in \algb{M}$.
\item $A ∈ \algb{M}$ $\Rightarrow$ $A^c ∈ \algb{M}$. ($A^c = \Omega \setminus A = \{w \in \Omega : w \notin A\} $)
\item $A_1, A_2,..., A_n \in \algb{M} \Rightarrow \bigcup_{n=0}^{\infty}A_n \in \algb{M}$ (la unión numerable pertenece a la $\salgb$).
\end{enumerate}
\end{defn}

\obs: $\algb{M}$ es una $\salgb$ si es un álgebra y además la unión numerable de elementos de $\algb{M}$ pertenece a $\algb{M}$. 

Propiedades:
\begin{enumerate}
\item[3']$A_1, A_2,..., A_n \in \algb{M} \Rightarrow \bigcap_{n=0}^{\infty}A_n \in \algb{M}$ (la intersección numerable pertenece a la $\salgb$.)
\begin{proof}
Se demuestra de la misma manera que para la intersección finita realizada anteriormente.
\end{proof}
\end{enumerate}


\begin{defn}[Función de probabilidad] P:$\algb{M} \rightarrow [0,1]$ es una función definida en ($\Omega,\algb{M}$). Siendo $\Omega$ un conjunto. 

Si $\algb{M}$ es un álgebra entonces P es finitamente aditiva y cumple:
\begin{enumerate}
\item $P(\Omega) = 1$.
\item $A,B ∈ \algb{M}$ y $A \cap B = \emptyset$ $\Rightarrow$ $P(A \cup B) = P(A) + P(B)$ (\textbf{aditividad finita}).
\end{enumerate}

Si $\algb{M}$ es una $\salgb$ entonces P es numerablemente aditiva y cumple:
\begin{enumerate}
\item $P(\Omega) = 1$.
\item $A_1,A_2,...,A_n ∈ \algb{M}$ y son disjuntos 2 a 2 $\Rightarrow$ $P(\bigcup_{i=0}^{\infty}) = \sum_{i=0}^{\infty}P(A_i)$ (\textbf{aditividad numerable}).
\end{enumerate}
\end{defn}

\obs No hay probabilidad numerablemente aditiva en $\mathbb{N}={0,1,2,...}$ que sea uniforme, es decir, que para cualquier $i,j \in \mathbb{N}, i \neq j, P({i}) = P({j})$. 

\begin{proof}
Lo probamos usando la propiedad de la aditividad numerable y cogiendo $\Omega = \mathbb{N}$
\begin{itemize}
\item Si $P({i})=0$ $\Rightarrow$ $1=P(\mathbb{N})= \sum_{i=0}^{\infty}P({i})= \sum_{i=0}^{\infty}0=0$  contradicción. 
\item Si $P({i})=k>0$ $\Rightarrow$ $1=P(\mathbb{N})= \sum_{i=0}^{\infty}P({i})=\sum_{i=0}^{\infty}k = \infty$  contradiccion. 
\end{itemize}
Sin embargo, sí existen probabilidades finitamente aditivas en $\mathbb{N}$ que satisfacen $P({i}) \neq 0$.
\end{proof}

Una vez estudiados sus elementos, podemos dar una definición de qué es un espacio de probabilidad:

\begin{defn}[Espacio de probabilidad] Es la tripla $(\Omega, \algb{M}, P)$. siendo $\Omega$ un conjunto, $\algb{M}$ una $\salgb$ y P una función de probabilidad.
\end{defn}

Propiedades:
\begin{enumerate}
\item $P(A^c) = 1-P(A)$
\begin{proof}
$1=P(\Omega)=P(A \cup A^c) = P(A) + P(A^c)$ 

Hemos usado la propiedad de la aditividad numerable de las funciones de probabilidad, por ser $A$ y $A^c$ disjuntos.
\end{proof} 
\item $P(A \cup B) = P(A) + P(B) - P(A \cap B)$
\begin{proof}

Por un lado (volvemos a usar la aditividad numerable):\\
$A \cup B = A \cup (B \backslash A) \Rightarrow P(A \cup B) = P(A \cup (B \backslash A)) = P(A) + P(B \backslash A)$\\

Por otro lado:\\
$P(B) = P(B \cap A) + P(B \cap A^c)$\\
$P(B \cap A^c) = P(B \backslash A) = P(B) - P(B \cap A)$ \\

Juntando los dos resultados obtenemos la expresión que queríamos demostrar.
\end{proof}

\item Continuidad inferior: Sean $A_1 \subset A_2 \subset A_3 \subset ...$ una sucesión creciente de conjuntos medibles, es decir, pertenecientes a $\algb{M}$ entonces:
\[ P(\bigcup_{n=1}^{\infty}A_n) = \lim_{n \rightarrow \infty} P(A_n)
\]


\begin{proof}
Vamos a usar la propiedad de aditividad numerable.

Definimos:

$D_1=A_1$\\
$D_2=A_2 \backslash A_1$\\
$D_{n+1}=A_{n+1} \backslash \bigcup_{n=1}^{\infty}A_n=A_{n+1} \backslash \bigcup_{n=1}^{\infty}D_n=A_{n+1} \backslash A_n$

Entonces: 
\[P(\lim_{n \rightarrow \infty}A_n)=P(\bigcup_{n=1}^{\infty}A_n)=P(\bigcup_{n=1}^{\infty}D_n)=\sum_{n=1}^{\infty}P(D_n)=\lim_{n \rightarrow \infty}\sum_{i=1}^{n}P(D_i)=\lim_{n \rightarrow \infty}P(\bigcup_{i=1}^{n}D_i) =
\]
\[
=\lim_{n \rightarrow \infty}P(A_n)
\]
\end{proof} 

\obs De esta propiedad podemos afirmar lo siguiente: \[ \bigcup_{n=1}^{\infty}A_n = \lim_{n \rightarrow \infty} A_n \Rightarrow P(\lim_{n \rightarrow \infty} A_n) = \lim_{n \rightarrow \infty} P(A_n)
\]
\obs Si $A_1 \subset A_2 \subset A_3 \subset...$ entonces tenemos una convergencia puntual:
\[
\lim_{n \Rightarrow \infty}\ind_{A_n}(w) = \ind_{\bigcup_{i=1}^{\infty}A_i}(w)
\]

\item Continuidad superior: Sean $A_1 \supset A_2 \supset A_3 \supset ...$ una sucesión decreciente de conjuntos medibles, es decir, pertenecientes a $\algb{M}$ entonces 
\[ P(\bigcap_{n=1}^{\infty}A_n) = \lim_{n \rightarrow \infty} P(A_n)
\]

\begin{proof}
\[P(\bigcap_{i=1}^{\infty}A_i) = 1-P\left((\bigcap_{i=1}^{\infty}A_i)^c\right)=1-P(\bigcup_{i=1}^{\infty}A^c) \stackrel{3)}{=} 1-\lim_{n \rightarrow \infty}P(A_n^c)=\lim_{n \rightarrow \infty}(1-P(A_n^c)) = 
\]
\[
\lim_{n \rightarrow \infty}P(A_n)
\]
\end{proof}

\obs Si $A_1 \supset A_2 \supset A_3 \supset...$ entonces tenemos una convergencia puntual:
\[
\lim_{n \Rightarrow \infty}\ind_{A_n}(w) = \ind_{\bigcap_{i=1}^{\infty}A_i}(w)
\]
\end{enumerate}


\begin{defn}[Espacio de medida]
 Es la tripla $(\Omega, \algb{M}, P)$. siendo $\Omega$ un conjunto, $\algb{M}$ una $\salgb$ y $\mu$ una función de medida.
\end{defn}

La diferencia con espacio de probabilidad, es que aquí definimos $\mu$ como una función: $\mu: X \rightarrow [0,\infty)$  (o incluyendo el $\infty$). En lugar de P que era una función: $P: X \rightarrow [0,1]$. Ambas trabajan con conjuntos pertencientes a $\algb{M}$, es decir, medibles.


\section{Conceptos de Probabilidad I}
Incorporamos al modelo nueva información relevante que condiciona los nuevos valores asignados.

\begin{defn}[Probabilidad condicionada]
Supongamos un conjunto $\Omega$ y dos conjuntos A y B pertenecientes a la $\salgb$ $\algb{M}$ de $\Omega$. Suponemos también que $P(B)>0$, y definimos una nueva función de probabilidad:

\[ P_B(A) = P(A | B) = \frac{P(A \cap B)}{P(B)}\]

$P_B$ es una probabilidad en B. $P_B(A)$ es la probabilidad de A condicionada a B. Dicho de otra forma, es la probabilidad de A sabiendo que se ha dado el suceso B.
\end{defn}

\begin{center}
\includegraphics[scale=0.75]{img/Dvenn1.png}
\end{center}

\begin{defn}[Regla del producto]
Sean $\{A_1, A_2,...,A_n\}$ eventos con $P(A_i)>0$ entonces

\[
P(\bigcap_{i=1}^{\infty}A_i) = P(A_1)P(A_2|A_1)P(A_3|A_1\cap A_2)...P(A_n|A_1\cap A_2\cap ... \cap A_{n-1})
\]
(Suponiendo que $P(A_n|A_1\cap A_2\cap ... \cap A_{n-1})>0$)
\begin{proof}
\[
P(A_1)P(A_2|A_1)P(A_3|A_1\cap A_2)...P(A_n|A_1\cap A_2\cap ... \cap A_{n-1})=
\]
\[
=P(A_1)\frac{P(A_2\cap A_1)}{P(A_1)}\frac{P(A_3\cap A_2\cap A_1)}{P(A_2\cap A_1)}...\frac{P(\bigcap_{i=1}^{n-1}A_i)}{P(\bigcap_{i=1}^{n-2}A_i)}\frac{P(\bigcap_{i=1}^{n}A_i)}{P(\bigcap_{i=1}^{n-1}A_i)} = P(\bigcap_{i=1}^{n}A_i)
\]
\end{proof}
\end{defn}
\begin{example}
Caja con 10 bolas blancas y 10 bolas negras. Se extrae 1 bola y sin devolverla a la caja se extra otra segunda. ¿Cuál es la probabiidad de que las dos sean blancas?

P(2 blancas)=$\frac{10}{20}\frac{9}{19}$
\end{example}

\begin{defn}[Regla de la probabilidad total]
Sea  $\{A_1, A_2,...,A_n\}$ una partición de $\Omega$ con $P(A_i)>0 \forall i=1,2,...,n$. Entonces, $\forall B \subset \Omega$ medible (perteneciente a $\algb{M}$):
\[
P(B)=\sum_{i=1}^{n}P(B\cap A_i)=\sum_{i=1}^{n}P(B|A_i)P(A_i)
\]
(Se obtiene de despejar de la formula de la probabilidad condicionada: $P(A|B)=\frac{P(A \cap B)}{P(B)}$)
\end{defn}

\begin{figure}[h]
\centering
\includegraphics[page=1,scale=0.745]{img/Dvenn2.png}
\caption{Ejemplo de partición con n=16}
\end{figure}

\begin{defn}[Teorema de Bayes]
Modeliza la noción de "causa-efecto", donde $A_1, A_2,...,A_n$ son posibles causas del efecto B:
\[
P(A_i|B)= \frac{P(A_i\cap B)}{P(B)}= \frac{P(B|A_i)P(A_i)}{\sum_{k=1}^{n}P(B|A_k)P(A_k)}
\]

(Combina la regla de la probabilidad condicionada con la regla de la probabilidad total)
\end{defn}

\begin{defn}[Independencia]
\[
A,B \text{ son independientes} \Leftrightarrow P(A \cap B)=P(A)P(B)
\]

\obs Si un suceso A es independiente de otro suceso B (de modo que B no proporciona información útil sobre A) entonces $P(A|B)=P(A)$.
\obs Dada una sucesión finita $\{A_i\}_{i=1}^{n}$ o infinita $\{A_i\}_{i=1}^\infty$ de eventos, decimos que estos son independientes si toda subsucesión $A_{i_1}, A_{i_2},..., A_{i_n}$ con $(2 \leq n < \infty)$ finita, saisface:
\[
P(\bigcap_{i=1}^nA_{i_j})=\prod_{i=1}^{n}(P(A_{i_j}))
\]

\obs Los conjuntos $A_1, A_2,...,A_n$ son independientes 2 a 2 si $\forall$ par \{i,j\} con $i \neq j$, tenemos que $A_i$ y $A_j$ son independientes.

\begin{example}
Supongamos que A es independiente de A. Entonces tenemos que: $P(A\cap A)=P(A)=P(A)P(A) \Leftrightarrow P(A)=0$ ó $P(A)=1$
\end{example}
\end{defn}

\begin{defn}[Norma y convergencia]
Sea $(X, \algb{M}, \mu)$ un espacio de medida. Para $0<p<\infty$, definimos $L^p=L^p(X,\algb{M},\mu)=\{f:X\rightarrow \mathbb{R}$ ó $\mathbb{C} | \int_X \abs{f}^p d\mu < \infty\}$

\begin{itemize}
\item Si $p \geq 1$, entonces $\norm{f}_p = (\int_X {\abs{f}^p d\mu)}^{\frac{1}{p}}$ es una norma.
\item Y decimos que una función converge en $L_p$:

\[
f_n \stackrel{L_p (n\rightarrow \infty)}{\rightarrow} g \Leftrightarrow \norm{f_n -g}_p  \rightarrow 0 \Leftrightarrow \int{\abs{f_n -g}^pd\mu} \rightarrow 0
\]
\end{itemize}
\end{defn}

\begin{example}
Sean A,B,C $\subset [0,1]$ con los borelianos (la $\salgb$ generada por los abiertos) y la medida de Lebesgue $\lambda$. Observamos que este espacio de medida coincide con un espacio de probabilidad con función de probabilidad uniforme P.

Es decir, tenemos 3 conjuntos que son uniones numerablos o complementarios de intervalos en $[0,1]$. Estos conjuntos no son independientes. Vamos a definir una nueva probabilidad de manera que tengamos independencia con conjuntos semejantes a estos:

\~{A} $= A\times[0,1]^2 \subset [0,1]^3$

\~{B} $= [0,1]\times B \times[0,1]$

\~{C} $= C \times[0,1]^2$

Por tanto \~{A}, \~{B} y \~{C} son independientes y \~{A} $\cap$ \~{B} = $\{x \in [0,1]^3, x=(x_1, x_2, x_3) | x_1 \in A, x_2 \in B, x_3 \in [0,1]\}$.

Y podemos definir \~{P}(\~{A}$\cap$ \~{B})=$P(A)P(B)P([0,1])$ (por definición de medida producto)

Siendo \~{P} en $[0,1]^3$ la probabilidad producto.

Además \~{P}(\~{A}$\cap$\~{C}) = \~{P}(\~{A})\~{P}(\~{C})=$P(A)P(C)$ etcetera etcetera...

Unos dibujitos aclaratorios, para hacerlo más fácil consideramos que estamos en $[0,1]^2$ y:

$A1 = A\times[0,1]$

$B1 = [0,1]\times B$

\begin{figure}[h]
\centering
\includegraphics[page=1,scale=0.545]{img/Dvenn3.png}
\caption{A = [0,1/2], B=[0,1/2]}
\end{figure}

\end{example}

\begin{defn}[Independencia de conjuntos respecto a otro conjunto] Sea $P(C)>0 \Rightarrow$ A y B son condicionalmente independientes con respecto a C si: 
\[
P(A\cap B|C)=P(A|C)P(B|C)
\]

\textcolor{red}{Esto me lo invento un poco yo:} Es decir, A y B son independientes entre ellos tomando como $\Omega$ el conjunto C.

Entonces si $P(B\cap C)>0$, se cumple que $P(A|B\cap C)=P(A|C)$
\end{defn}

\begin{defn} [Variable aleatoria]
Dado ($\Omega_1$,$\algb{M}$) y ($\Omega_2$,$\algb{B}$):
\[
X \text{es una variable aleatoria} \Leftrightarrow \forall B \in \algb{B}, X^{-1}(B) \in \algb{M}
\]

En este curso usaremos como $\Omega_2$ conjuntos como $\mathbb{R}$ o $\mathbb{R} \cup \{\pm \infty\}$ o $\mathbb{C}$. Si $\Omega_2 = \mathbb{R}^d$, decimos que la función medible $X: \Omega_1 \rightarrow \mathbb{R}^d$ es un vector medible.
\end{defn}
\obs Una variable aleatoria es una función medible.

\begin{defn}[Variable aleatoria discreta]
Una variable aleatoria X es discreta $\Leftrightarrow$ $P_X$ es discreta. Es decir, si toma valores en un conjunto numerable. Se caracteriza por tener una función de masa o de probabilidad, y una función de distribución.

Otras definiciones:

Recordemos que X es una función. $X: \Omega \rightarrow \mathbb{R}$. X es discreta si existe un conjunto numerable $x_1, x_2,...,x_n \in \mathbb{R}$ (siendo $x_i=X(w_i)$) tal que $P(\bigcup_{i=1}^{\infty}x_i=1)$.

$P_X$ es discreta si $P_X(\mathbb{R})=\sum_{i=1}^{\infty}P(X=x_i)$ (es decir, se puede expresar como un sumatorio numerable).
\end{defn}

\begin{defn}[Variable aleatoria continua]
Una variable aleatoria es continua si toma valores en un conjunto no numerable. Tiene asociada una función de densidad o de probabilidad, y una función de distribución.
\end{defn}

\begin{defn}[ley de X]
Dado el espacio de probabilidad ($\Omega, \algb{M}, P$) y la variable aleatoria $X: \Omega \rightarrow \mathbb{R}$, la ley de X es la probabilidad en $\mathbb{R}$ definida mediante:
\[
P_x(B)=P(X^{-1}(B)) \text{ } \forall B \in Borel(\mathbb{R})
\]

Notación: $P_x(B) = P(X^{-1}(B)) = P(\{w \in \Omega : X(w) \in B\})=P(X \in B)$

Notación: $Borel(\mathbb{R})$ es el conjunto de intervalos en $\mathbb{R}$, (los borelianos de toda la vida).


\end{defn}


\begin{example}
Supongamos un dado de 6 caras. Tomamos dos situaciones:
\begin{enumerate}
\item  $\Omega = \{1,2,3,4,5,6\}$ y $\algb{M}=\mathbb{P}(\Omega)$.

Entonces la función X se define:

$X: \Omega \rightarrow \mathbb{R}$

Todas las funciones X son medibles ya que tenemos $(\Omega,\algb{M})$ y $(\mathbb{R}, Borel(\mathbb{R}))$, y entonces $\forall A \in Borel(\mathbb{R})$ tenemos que $X^{-1}(Borel(\mathbb{R})) \in \algb{M}$

Una vez definidos cuales son los medibles en los conjuntos de salida $(\Omega,\algb{M})$ y de llegada $(\mathbb{R}, Borel(\mathbb{R}))$. Podemos definir la siguiente variable aleatoria X.

$X(w) = 1$ si $w=3$ 

$X(w) = 0$ si $w\neq3$

Siendo $w \in \Omega$, esta variable aleatoria es equivalente a $X(w) = \ind_{\{3\}}$.

De esta manera podemos ver que: $X^{-1}(\{1\}) = \{3\}$, $X^{-1}(\{0\}) = \{1,2,4,5,6\}$, $X^{-1}((1/2,\infty)) = \{3\}$, y $X^{-1}((-2,\infty)) = \Omega$ entre otros ejemplos.


\item $\Omega = \{1,2,3,4,5,6\}$ y $\algb{B} = \{\Omega, \emptyset, \{1,3,5\}, \{2,4,6\} \}$

Entonces la función X se define:

$X: \Omega \rightarrow \mathbb{R}$

Pero en este caso, las funciones medibles ($\algb{B}-medibles$) son aquellas que son constantes en $\{1,3,5\}$ y en $\{2,4,6\}$. Ya que por ejemplo si tengo una función del tipo: $X(\{1\})=1$, $X(\{2\})=2$, $X(\{3\})=3$, $X(\{4\})=4$, $X(\{5\})=5$, $X(\{6\})=6$. Y calculo $X^{-1}((1/2, 3/2))=\{1\}$, que no pertenece a $\algb{B}$ y no es medible, por tanto X no sería medible y no sería una variable aleatoria.
\end{enumerate}

\obs Dada una función $X:(\Omega, \algb{M}) \rightarrow (\mathbb{R}, Borel(\mathbb{R}))$, para comprobar que es una función medible, y por tanto que es una variable aleatoria (es decir, que $\forall B \in Borel(\mathbb{R}) \rightarrow X^{-1}(B) \in \algb{M}$), basta comprobarlo para cualquier clase que genere a los borelianos (a $Borel(\mathbb{R})$).

Puesto que $Borel(\mathbb{R})=\{(r,\infty): r\in \mathbb{Q}\}$, para ver que X es una variable aleatoria basta comprobar que $\forall r \in \mathbb{Q}, X^{-1}((r, \infty)) \in \algb{M}$.

\end{example}

\begin{defn}[Esperanza o media]
Sea un espacio de probabilidad $\{\Omega, \algb{M},P\}$, la media o esperanza de una variable aleatoria X es:
\[
E(X)=E_p(X)=\int_{\Omega}X(w)dP(w)
\]

Si la variable aleatoria es continua:

\[
E(X)=E_p(X)=\int_{\Omega}X(w)dP(w) = \int_{-\infty}^{\infty}x\cdot dP_X(x)= \int_{-\infty}^{\infty}x\cdot dF_X(x) = \int_{-\infty}^{\infty}x\cdot f_P(x) dx
\]

Donde $F_X(t)=P(X \leq t)$ es la función de distribución de X y $f(t)=F'_X(t)$ la de densidad.

Si es discreta:

\[
E(X)=\sum_{n=1}^{\infty}X(w_n)P(w_n)
\]


\obs \textbf{Breve explicación del concepto de integrar respecto a una medida}(Desde aquí hasta la definición de varianza hay una explicación, posiblemente obvia para muchos lectores,  de lo que es derivar con respecto a una medida, además esta escrita con mis palabras con lo que puede que os liéis, si es así, os recomiendo borrarla de la cabeza, aunque también puede servir) Derivar con respecto a P(w), es derivar con respecto a una medida, que tal y como hemos visto en TIM es equivalente a derivar la función de distribución de dicha medida e integrar con respecto a x. O lo que es lo mismo, multiplicar por la función de densidad. 

El concepto es natural en este caso, si se piensa que la integral es una suma infinita, y si derivas con respecto a una medida lo que quieres es obtener la medida de cada $w \in \Omega$. En este caso se suele hacer un cambio de variable, e integrar en $\mathbb{R}$ en lugar de en $\Omega$, por tanto ahora integras las x=X(w), y pones la función de densidad, que asigna a cada valor X(w) su probabilidad (sabemos que al hacer la integral estamos considerando que la variable aleatoria es continua, y no discreta, y que por tanto la probabilidad de un evento es igual a la probabilidad de un punto, que es 0 ya que el area bajo un punto es 0), que es lo equivalente a la función de probabilidad, de manera que queda:

\[
E(X)=E_p(X)=\int_{\Omega}X(w)dP(w) = \int_{\mathbb{R}}x\cdot dP_X(x)= \int_{\mathbb{R}}x\cdot dF(x) = \int_{\mathbb{R}}x\cdot f(x) dx
\]

Siendo $f(x)$ la función de densidad asociada a esa probabilidad (a esa función de distribución P), que aplicando el Tma. Fundamental del Calculo (ver capitulo 1):

\[
F(x) = \int_{-\infty}^{x}f(x)dx  \rightarrow \frac{dF(x)}{dx}= f(x) \rightarrow dF(x) = f(x)dx
\]

Por tanto, dada una integral con respecto a una medida $\mu$, primero obtenemos la función de distribución asociada a $\mu$ y posteriormente derivamos esa función de distribución obteniendo la función de densidad y la integral con respecto a x, que s´i sabemos resolver.

Cuando la variable aleatoria es discreta la esperanza se calculará como una suma finita por la medida (probabilidad) de cada evento $w \in \Omega$.

%\obs $P(a \leq w \leq b)= \int_{b}^{a} f(x)dx$, por tanto: $dP(a \leq w \leq b) = f(b)-f(a).
\end{defn}


\begin{defn}[Varianza]
Sea un espacio de probabilidad $\{\Omega, \algb{M},P\}$, la varianza de una una variable aleatoria X es:
\[
var(X)=E[(X-E(X))^2] = E(X^2)-E(X)^2
\]

\obs Si $X(w)=c$  $\forall w$, entonces $E(X)=c$
\obs E(E(X))=E(X)
\end{defn}

\begin{defn}[Formula del cambio de variable]
Sea $X: \Omega \rightarrow \mathbb{R}$ una variable aleatoria, y sea $g:\mathbb{R} \rightarrow \mathbb{R}$ una función de Borel. Entonces:

\[
\mathbb{E}(g(X))=\int_{\Omega}g(X(w))dP(w)=\int_{\mathbb{R}}g(x)dP_X(x)
\]

Recordatorio: $P_X(A)=P(X\in A)=P(X^{-1}(A))$
\end{defn}

Antes de demostrar esta fórmula vamos a recordar algunos conceptos de TIM que usaremos:

\begin{defn}[Integrar respecto a una medida una función indicatriz]
La integral con respecto a una medida de una función indicatriz evaluada sobre un subconjunto $E \in \Omega$ medible (perteneciente a la $\salgb$) es la medida de dicho subconjunto E:

\[\int_{\Omega} \ind_{E}d\mu = \mu(E)\]
\end{defn}

\begin{defn}[Función simple]
Combinación lineal finita de funciones indicatrices.

\[
s(x) = \sum_{i=1}^{n}(c_i \cdot \ind_{B_i}(x))
\]
\end{defn}

\begin{defn}[Integrar respecto a una medida una función simple]
Sea $(\Omega,\algb{M},\mu)$ un espacio de medida:

\[\int_{\Omega} s\mu = \sum_{i=1}^{n}(c_i \int \ind_{B_i} d\mu) = \sum_{i=1}^{n}(c_i \mu(B_i))\]
\end{defn}

\begin{defn}[Función $L^+$]
$f \in L^+ \Leftrightarrow f:X\rightarrow [0, \infty]$
\end{defn}

\begin{defn}[Teorema de aproximación de funciones simples]
Si $f:X \rightarrow [0, \infty]$ es una función medible, entonces existe una sucesión crecientes de funciones simples $s_n$, $0 \leq s_1 \leq s_2 \leq ... \leq f$ tal que $\forall x  \in X, s(x) \rightarrow f(x)$. Además la convergencia es uniforme sobre conjuntos en los que |f| es acotada.
\end{defn}

\begin{defn}[Teorema de la convergencia monótona: TCM]
Si $f_n$ es una sucesión creciente de funciones $L^+$ y $f(x)=sup_n\{f_n(x)\}=lim_nf_n(x)$ entonces:
\[
\int f = \lim_n \int f_n
\]
\end{defn}

\begin{proof}
\textbf{De la fórmula del cambio de variable} Esta demostración puede servir para aclarar algunos conceptos de TIM, lo hacemos despacio:

\begin{enumerate}
\item Primero vamos a ver que es cierto para funciones $g=\ind_{B}$, con $B \in Borel(\mathbb{R})$. De manera que por ser una función indicatriz se cumple que: $\ind_{B}(X(w))=1 \Leftrightarrow X(w) \in B$. 

En este caso tendríamos (Usamos la definición de integrar respecto a una medida una función indicatriz):
\[
\mathbb{E}(g(X(w))) = \mathbb{E}(g(X)) = \mathbb{E}(\ind_{B}(X))=\int_{\Omega}\ind_{B}(X(w))dP(w) =
\]
\[
= P(X(w)\in B) = P(X^{-1}(B))=P_X(B)=\int_{\Omega}\ind_{B}(x)dP_X(x)
\] 

Breve explicación: 
\begin{itemize}
\item Poner o no la 'w' al poner X(w) no es más que notación, se sobreentiende que esa w siempre esta ahí, ya que X es una variable aleatoria, y por tanto una función que depende de w.
\item Estamos integrando sobre un espacio $\Omega$, sin embargo la indicatriz esta evaluada sobre un espacio de Borel que no tiene por que pertenecer a $\Omega$. Por eso, el resultado de la integral no es P(B) que es lo que sería aplicando la definición anterior (si ocurriera que $B\in \Omega$). En este caso, el resultado sera la medida P de todos los $w \in \Omega$ que provocan que $X(w) \in B$.
\item La última igualdad sale por definición de integrar respecto a una medida una función indicatriz.
\end{itemize}
\item Si $g \geq 0$ (es decir, $g \in L^+$), entonces, por el teorema de la aproximación de funciones simples, existe una sucesión $s_n\nearrow g$ (converge a g en todo punto y la sucesión es monótona creciente). Así podemos escribir:

\[
\int g dP = \int \lim_{n \rightarrow \infty}s_n dP \stackrel{TCM}{=} \lim_{n \rightarrow \infty}\int s_n dP
\]

Y dicha integral la sabemos resolver aplicando la definición de integral sobre funciones simples que hemos visto anteriormente y aplicando el punto 1 de esta demostración.
\item Y el último caso sería una funcion $g:\mathbb{R} \rightarrow \mathbb{R}$, en cuyo caso escribimos $g=g_+-g_-$ (ver capitulo 1.3). Y nos queda $\int gdP = \int g_+dP -\int g_-dP$(por la linealidad de la integral).

Por tanto hemos expresado g como resta de dos funciones $g_+$ y $g_-$, ambas pertenecientes a $L^+$. Entonces podemos hallar $\int g$ aplicando el punto 2 de esta demostración. Debemos tener en cuenta que $\int g$ existirá si: $\int g_+dP < \infty$ ó $\int g_-dP < \infty$

\end{enumerate}
\end{proof}


\begin{example}
\begin{itemize}
\item $\mathbb{E}(X)$ no esá bien definida si $\int_{\mathbb{R}}X_+ dP_X=\infty$ y $\int_{\mathbb{R}}X_- dP_X=\infty$.

\item $\mathbb{E}(X^2)$ siempre esta bien definida ya que $(X^2)_-=0$.
\item Si $X \geq 0$, $\mathbb{E}(X)$ está bien definida.
\item Sea una variable aleatoria de Bernoulli de parámetro p. Entonces $P(X=1)=p$ y $P(X=0)=1-p$. Llamamos función de masa a $P(X=i)$.

$\mathbb{E}(X)$ está bien definida porque X(w) es siempre $\geq 0$. 

\[
\mathbb{E}(X)=0\cdot P(X=0)+1\cdot P(X=1)=0\cdot(1-p)+1\cdot p=p
\]
\[
\mathbb{E}(X^2)=0^2\cdot P(X=0)+1^2\cdot P(X=1)=0^2\cdot(1-p)+1^2\cdot p=p
\]
\[
\mathbb{V}(X)=\mathbb{E}(X^2)-\mathbb{E}(X)^2=p-p^2=p(1-p)
\]
\item Sea $S_n=X_1+X_2+...+X_n$, con $X_i=Bernoulli(p)$, entonces $S_n ~ Binomial=B(n,p)$.
\[
\mathbb{E}(S_n)=\mathbb{E}(\sum_{i=1}^{n}X_i)\stackrel{linealidad integral}{=}\sum_{i=1}^{n}(\mathbb{E}(X_i))=np
\]
\end{itemize}
\end{example}




\begin{defn}[medida absolutamente continua con respecto a otra $\mu << \nu$]
$\mu << \nu$ si $\mu(A)=0 \Rightarrow \nu(A)=0$

\begin{example}
Sea $f \geq 0$ medible y sea $\mu(A)=0$, entonces $\int_Afd\mu=0$ (se ve fácil con la fórmula del cambio de variable). Si defino $\nu(B)=\int_B fd\nu$, entonces tenemos que $\mu << \nu$.
\end{example}
\end{defn}

\begin{defn}[medida con signo]
Sea $f:\mathbb{R}\rightarrow \mathbb{R}$, entonces $\nu(B)=\int_B fd\nu$ es una medida con signo (y además $\mu << \nu$). Es una medida con signo ya que al ser una medida que depende de una función que va de $[-\infty, \infty]$, puede adquirir valores negativos.
\end{defn}

\begin{defn}
X es una variable aleatoria continua $\Leftrightarrow$ $P_x$ es absolutamente continua con respecto a la medida de Lebesgue ($\lambda$) (se escribe $P_X << \lambda$)
\end{defn}

\begin{theorem}[Teorema de Radon-Nikodyn]
Sea $\nu << \mu$ una medida con signo ($\mu \geq 0$, $\mu$ y $\nu$ son $\sfin$, y $\nu_+(X) < \infty$ ó $\nu_-(X) < \infty$). Entonces existe una funcion f medible tal que $f:X \rightarrow \mathbb{R}$ que cumple que $\forall B$ medible $\subset X$ se cumple $\nu(B)=\int_Bfd\mu$. Escribimos $f=\frac{d\nu}{d\mu}$, la derivada de Radon-Nikodyn. 
\end{theorem}

\begin{defn}[Medida\IS $\sfin$]\label{defSigmaFinita}
Dado un espacio de medida $(X, \algb{M}, µ)$, decimos que una medida es $\sfin$ si el conjunto $X$ puede expresarse como una unión de elementos de la $\salgb$ de medida finita. Es decir, si \[X=\bigcup_{n=1}^{\infty}E_n, \ E_n \in \algb{M} \text{ y }µ(E_n)< \infty\]
\end{defn}






%\begin{figure}[h]
%\centering
%\includegraphics[page=1,scale=0.745]{img/Dvenn2.png}
%\caption{Ejemplo de partición con n=16}
%\end{figure}

%\centerline{\includegraphics[page=1,scale=0.745]{img/Dvenn2.png}} % scale obtenido empíricamente para que quepa en la página

%\easyimg{img/Dvenn2.png}{El histograma es una aproximación de la función de densidad real en base a la muestra que hemos obtenido.}{lblDensidad}

%\easyimg{img/DensidadAHistograma.png}{El histograma es una aproximación de la función de densidad real en base a la muestra que hemos obtenido.}{lblDensidad}

%\centerline{\includegraphics[page=1,scale=0.745]{pdf/_Solucion_T1P1.pdf}} % scale obtenido empíricamente para que quepa en la página


%\includepdf[pages=2-]{pdf/_Solucion_T1P1.pdf}



\chapter{Hojas de Ejercicios}
%
% Soluciones a los ejercicios de Probabilidad II.
%
% Curso 2014 - 2015 2º cuatrimestre
%

%%%%%%%%%%%%%%%%%%%%%%%%%%%%%%%%%%%%%%%%%%%%%%%%%%%%%%%%%%%%%%%%%%%%%%%%%%%%%%%
\section{Hoja 1}

\textcolor{red}{A LA ESPERA DE SER ESTRICTAMENTE CORREGIDA. BASTANTE FIABLES TODOS MENOS EL 12  QUE ESTA MAL}

Se asume siempre que estamos trabajando en un espacio de probabilidad $(\Omega, \mathcal{A}, P)$, y que  $\mathcal{B}\subset \mathcal{A}$ es una sub-$\sigma$-\'algebra.

%%%%%%%%%%%%%%%%%%  PROBLEMA 1.1  %%%%%%%%%%%%%%%%%%%%%%%%%
\begin{problem}[1]De una urna con 10 bolas blancas y 10 bolas negras se extraen simultaneamente 3 bolas. 
Calcular la probabilidad de que exactamente dos de ellas sean blancas. Responder a la misma
pregunta si las bolas se extraen de manera sucesiva.
\solution

\begin{expla}
En este caso no hay reposición de las bolas extraídas. Por tanto no hay diferencia a la hora de calcular la probabilidad entre la extracción simultánea y la sucesiva.

Para resolver esos problemas, la forma más intuitiva es buscar la fracción:
\[
\frac{\text{Numero de casos favorables}}{\text{Numero de casos posibles}}
\]
\end{expla}
A = Extracción simultánea de 3 bolas blancas

b = blanca

n = negra
\[
P(A)=\underbrace{\frac{10}{20}}_{b}\underbrace{\frac{9}{19}}_{b}\underbrace{\frac{10}{18}}_{n}+\underbrace{\frac{10}{20}}_{b}\underbrace{\frac{10}{19}}_{n}\underbrace{\frac{9}{18}}_{b}+\underbrace{\frac{10}{20}}_{n}\underbrace{\frac{10}{19}}_{b}\underbrace{\frac{9}{18}}_{b} = 3\frac{900}{6840}=\frac{2700}{6840}=0.39
\]

\end{problem}
%%%%%%%%%%%%%%%%%%%%%%%%%%%%%%%%%%%%%%%%%%%%%%%%%%%%%%%%%%


%%%%%%%%%%%%%%%%%%  PROBLEMA 1.2  %%%%%%%%%%%%%%%%%%%%%%%%%
\begin{problem}[2]Disponemos de dos urnas,
$U_1$, que contiene 6 bolas azules y 8 bolas blancas, y
$U_2$,
 que contiene
3 bolas azules y 9 bolas blancas. Se sortea con un dado equilibrado de 4 caras la elecci\'on de una
urna, escogiendose
$U_1$,
si salen 1,
2 o 3, y
$U_2$,
si sale 4. Posteriormente se extrae al azar una bola de
esa urna.

\ppart ?` Cual es la probabilidad de que la bola extraida sea azul?
Sugerencia: usar la regla de la probabilidad total. Respuesta: $43/112$.

\ppart  Si la bola extraida resulta ser blanca  ?`cual es la probabilidad de que proceda de la
urna
$U_1$?

Sugerencia: usar  Bayes o el apartado anterior. Respuesta:
$16/23 = 1 -  43/112$.
\solution
\begin{expla}

$U_1 \rightarrow$ 6a, 8b  (1,2,3)

$U_2 \rightarrow$ 3a, 9b  (4)
\end{expla}
\spart
A = bola extraída azul

$U_1$ = Extraemos de la urna 1

$U_2$ = Extraemos de la urna 2
\[
P(A) = P(A\cap U_1)+P(A\cap U_2) = P(A|U_1)P(U_1)+P(A|U_2)P(U_2)=
\]
\[
=\frac{3}{4}\cdot\frac{6}{14}+\frac{1}{4}\cdot\frac{3}{12}=\frac{18}{56}+\frac{3}{48}=\frac{18}{56}+\frac{1}{16}=\frac{36}{112}+\frac{7}{112}=\frac{43}{112}
\]

\spart
B = bola extraida blanca.

$U_1$ = Extraemos de la urna 1.

\[
P(U_1|B)= \frac{P(U_1 \cap B)}{P(B)} = \frac{P(B|U_1)P(U_1)}{P(B|U_1)P(U_1)+P(B|U_2)P(U_2)}=
\]
\[
=\frac{\frac{8}{14}\cdot\frac{3}{4}}{\frac{8}{14}\cdot\frac{3}{4}+\frac{9}{12}\cdot\frac{1}{4}}=\frac{\frac{24}{56}}{\frac{24}{56}+\frac{9}{48}}=\frac{\frac{24}{56}}{\frac{48}{112}+\frac{21}{112}}=\frac{\frac{24}{56}}{\frac{69}{112}}=\frac{48}{69}=\frac{16}{23}
\]

\end{problem}

%%%%%%%%%%%%%%%%%%%%%%%%%%%%%%%%%%%%%%%%%%%%%%%%%%%%%%%%%%

%%%%%%%%%%%%%%%%%%  PROBLEMA 1.3  %%%%%%%%%%%%%%%%%%%%%%%%%
\begin{problem}[3]Enfermedades raras. Ning\'un test biol\'ogico es 100 $\%$ preciso. Supongamos que un test para
determinar si cierta infecci\'on se ha producido, da falsos positivos en un 1 $\%$ de los casos, y falsos
negativos en un 2 $\%$  de los casos. Si una de cada 100 000 personas entre la poblaci\'on general est\'a
infectada, determinar la probabilidad de que una persona escogida al azar est\'e infectada, sabiendo
que el test ha dado positivo.
\solution

\begin{expla}

\begin{center}
\includegraphics[scale=0.75]{img/Dvenn5.png}
\end{center}


\begin{itemize}

\item Las personas o están infectadas o no están infectadas. 1=P(A1)+P(A2)

\item Los tests o dan positivo o dan negativo: 1 = P(+)+P(-)

\item $A_1$ = persona infectada $\rightarrow P(A1)=10^{-5}$

\item $A_2$ = persona no infectada $\rightarrow P(A_2)=1-10^{-5}$

\item + = test positivo

\item - = test negativo

\item Falso positivo (probabilidad de que el test de positivo estando la persona sin infectar) = $1\% = P(+|A_2)=0.01$

\item $P(+|A_2)+P(-|A_2)=1 \rightarrow P(-|A_2)=0.99$

\item Falso negativo (probabilidad de que el test de negativo sabiendo que la persona esta infectada) = $2\% = P(-|A_1)=0.02$

\item $P(-|A_1)+P(+|A_1)=1 \rightarrow P(+|A_1)=0.98$
\end{itemize}

\end{expla}



\[
P(A_1|+)=\frac{P(A_1\cap +)}{P(+)}=\frac{P(+|A_1)P(A_1)}{P(+|A_1)P(A_1)+P(+|A_2)P(A_2)}=
\]
\[
=\frac{0.98\cdot10^{-5}}{0.98\cdot10^{-5}+0.01\cdot\frac{99999}{10^5}}=0.000979
\]

\end{problem}

%%%%%%%%%%%%%%%%%%%%%%%%%%%%%%%%%%%%%%%%%%%%%%%%%%%%%%%%%%

%%%%%%%%%%%%%%%%%%  PROBLEMA 1.4  %%%%%%%%%%%%%%%%%%%%%%%%%
\begin{problem}[4]Angel y Benito tienen sendas barajas espa\~nolas (40 cartas). Cada uno saca de su
baraja una carta al azar (es decir, con iguales probabilidades, e independientemente). Hallar:

\ppart La probabilidad de obtener al menos un as. 

\ppart La probabilidad de obtener dos cartas del mismo palo. 

\ppart La probabilidad de no obtener ning\'un as.

\ppart La probabilidad de no obtener ni una copa ni una espada.
\solution

\begin{expla}

\end{expla}

\spart
A = obtener al menos un AS

B = no obtener ningun AS
\[
P(A)=\frac{4}{40}\cdot\frac{4}{40}+\frac{4}{40}\cdot\frac{36}{40}+\frac{36}{40}\cdot\frac{4}{40} = 0.19
\]

Otra forma
\[
P(A)=1-P(B)=1-\frac{36}{40}\cdot\frac{36}{40} = 1 - 0.81 = 0.19
\]

\spart
C = dos cartas del mismo palo

Da igual de que palo sea la primera carta, la cosa es que la segunda sea del mismo.

Pensando de otra forma tenemos el siguiente espacio muestral:

$\Omega = \{(c, o),(c, e),(c, b),(c, c),(o, o),(o, c),(o, e),(o, b),(e, c),(e, o),(e, b),(e, e),(b, c),(b, e)\\,(b, o),(b, b)\}$

\[
p(C)=\frac{1}{4}
\]

\spart
B = no obtener ningun AS

\[
P(B)=\frac{36}{40}\cdot\frac{36}{40} = 0.81
\]

\spart
D = no obtener ni una copa ni una espada

\[
P(D) = \frac{20}{40}\cdot\frac{20}{40}=\frac{1}{4}
\]


\end{problem}

%%%%%%%%%%%%%%%%%%%%%%%%%%%%%%%%%%%%%%%%%%%%%%%%%%%%%%%%%%

%%%%%%%%%%%%%%%%%%  PROBLEMA 1.5  %%%%%%%%%%%%%%%%%%%%%%%%%
\begin{problem}[5] Ana y Bea eligen cada una un n\'umero al azar, entre 0 y 2. Sean $A, B, C, D,$ los siguientes
eventos: 

\  $A$: La diferencia entre ambos n\'umeros es al menos 1/3.

\ $B$:   Al menos uno de los n\'umeros es mayor que 1/3.

\ $C$: Los dos n\'umeros son iguales.

\ $D$: El n\'umero de Bea es mayor que 1/3.

Hallar $P(B)$, $P(C)$ y $P(A\cup D)$.
\solution

\begin{expla}
Suponemos que el número elegido es natural, es decir, pertenece al conjunto $\{0,1,2\}$.
\end{expla}

\spart
E = Ningún número es mayor que 1/3.

\[
P(B)=1 - P(E)= 1 - \frac{1}{3}\cdot\frac{1}{3} = \frac{8}{9}
\]

\spart

Da igual qué numero escojas el primero, el que importa es el segundo.

Pensando de otra forma tenemos el siguiente espacio muestral:

$\Omega=\{(0,0),(0,1),(0,2),(1,0),(1,1),(1,2),(2,0),(2,1),(2,2)\}$

\[
P(C)=\frac{1}{3}
\]

\spart
Dado el espacio muestral $\Omega$ definido en el apartado anterior, vemos que:

$P(A)=\frac{2}{3}$ 

Ya que los elementos que cumplen A son: \{(0,1),(0,2),(1,0),(1,2),(2,0),(2,1)\}

$P(D)=\frac{2}{3}$

Ya que los elementos que cumplen D son: \{(0,1),(0,2),(1,1),(1,2),(2,1),(2,2)\}

Por tanto, los elementos que cumplen $A\cup D$ serán la unión de los elementos que cumplen A y los que cumplen D.

\[
P(A \cup D) = \frac{8}{9}
\]

\end{problem}

%%%%%%%%%%%%%%%%%%%%%%%%%%%%%%%%%%%%%%%%%%%%%%%%%%%%%%%%%%

%%%%%%%%%%%%%%%%%%  PROBLEMA 1.6  %%%%%%%%%%%%%%%%%%%%%%%%%
\begin{problem}[6]Con 12 chicas y 4 chicos se forman al azar 4 grupos de 4 personas.
Calcular la probabididad de que haya un chico en cada grupo. Sugerencia:
usar la regla del producto.
 
\solution

\begin{expla}

$A_n$ = Un chico en el grupo n

B = Un chico en cada grupo
\end{expla}

$P(A_1)=4\cdot\frac{4}{16}\cdot\frac{12}{15}\cdot\frac{11}{14}\cdot\frac{10}{13}=0.4835$

$P(A_2|A_1)=4\cdot\frac{3}{12}\cdot\frac{9}{11}\cdot\frac{8}{10}\cdot\frac{7}{9}=0.509$

$P(A_3|A_1\cap A_2)=4\cdot\frac{2}{8}\cdot\frac{6}{7}\cdot\frac{5}{6}\cdot\frac{4}{5}=0.5714$

$P(A_4|A_1\cap A_2\cap A_3)=4\cdot\frac{1}{4}\cdot\frac{3}{3}\cdot\frac{2}{2}\cdot\frac{1}{1}=1$


\[
P(B)=\bigcap_{n=1}^4P(A_n)=P(A_1)P(A_2|A_1)P(A_3|A_2\cap A_1)P(A_4|A_1\cap A_2\cap A_3)=0.14
\]




\end{problem}

%%%%%%%%%%%%%%%%%%%%%%%%%%%%%%%%%%%%%%%%%%%%%%%%%%%%%%%%%%

%%%%%%%%%%%%%%%%%%  PROBLEMA 1.7  %%%%%%%%%%%%%%%%%%%%%%%%%
\begin{problem}[7]Benito tiene un dado  trucado, con 6 caras  numeradas del 1 al 6. 
La probabilidad de las
distintas caras es proporcional al n\'umero de puntos inscritos en
ellas. Hallar la probabilidad de que Benito obtenga con ese dado un n\'umero
par.
\solution

\begin{expla}

Teniendo en cuenta que la suma de los números del dado es 21

\end{expla}

\[
P(par)=P(2)+P(4)+P(6)=\frac{2}{21}+\frac{4}{21}+\frac{6}{21}=\frac{12}{21}=0.57
\]

\end{problem}

%%%%%%%%%%%%%%%%%%%%%%%%%%%%%%%%%%%%%%%%%%%%%%%%%%%%%%%%%%

%%%%%%%%%%%%%%%%%%  PROBLEMA 1.8  %%%%%%%%%%%%%%%%%%%%%%%%%
\begin{problem}[8] En el esquema que aparece a continuaci\'on, el agua fluye  desde $A$ hacia
$B$. Hay, como se indica en el dibujo, ocho compuertas.
Independientemente unas de otras, cada compuerta est\'{a} abierta con
probabilidad $p$, $0 <p <1$. Calcular la probabilidad de que el agua llegue
 de $A$ a $B$. Calcular dicha probabilidad cuando $p = 1/3$.
% Drawing generated by LaTeX-CAD 1.8a - requires latexcad.sty
% (c) 1996 John Leis leis@usq.edu.au
$$\xymatrix{    &   & \circ\ar @{.}[dr]!U||&  &   \circ  \ar @{.}[dr]!U|| &  &  \\
A \ar[r]  & \circ  \ar @{.}[ru]!U||   \ar @{.}[rd]!U||   & &  \circ \ar @{.}[ru]!U||  \ar @{.}[rd]!U||  & & \circ \ar[r]  & B\\  
  &   & \circ \ar @{.}[ru]!U||  &  &   \circ  \ar @{.}[ru]!U||   &  &  }$$ 

Respuestas: $p^8 - 4 p^6 + 4 p^4, 289/6561$.
\solution

\begin{expla}

Se entiende que en las intersecciones el agua va en todas las direcciones.

Llamamos C al punto intermedio (a la intersección entre los dos rombos).

Para llegar hasta el punto C, tenemos que considerar la existencia de 4 puertas: $P_1, P_2, P_3, P_4$, enumeradas de izquierda a derecha y de arriba hacia abajo. Por tanto, para que el agua llegue a C, deben estar abiertas al menos $P_1$ y $P_2$, o $P_3$ y $P_4$.

Si consideramos el siguiente espacio muestral:

$\Omega = \{(0000),(0001),(0010),(0011),(0100),(0101),...,(1110),(1111)\}$

Formado por 16 elementos, en los que un 1 en la posición n indica que la puerta $P_n$ está abierta, tenemos que con esas 16 combinaciones el agua NO llegaría a C en los elementos con 4 0's (1), los elementos con 3 0,s (4) y los elementos 1010, 0101, 1001, 0110 (4). Por tanto solo nos sirven 7 combinaciones. La de todo 1's, las de 3 1's (4), 1100 y 0011.

$P(1100)=P(0011)=p^2(1-p)^2$

$P(1110)=P(1101)=P(1011)=P(0111)=p^3(1-p)$

$P(1111)=p^4$

Por tanto la probabilidad de llegar a C desde A es:

\[
P(AC)=2p^2(1-p)^2+4p^3(1-p)+p^4
\]

La probabilidad de llegar al punto B, desde el punto C es exactamente la misma que la de ir desde A hasta C.
\end{expla}
Por tanto, según la regla del producto quedaría:
\[
P(AB)=P(AC)P(CB|AC)=P(AC)^2=(2p^2(1-p)^2+4p^3(1-p)+p^4)^2=
\]
\[
=(2p^2(p^2+1-2p)+4p^3-4p^4+p^4)^2 = (2p^4 + 2p^2-4p^3-3p^4+4p^3)^2 =
\]
\[
=(-p^4+2p^2)^2=p^8-4p^6+4p^4
\]

Para $p=\frac{1}{3}$, nos queda:

\[
P(AB)=\frac{1}{3^8}-4\cdot\frac{1}{3^6}+4\cdot\frac{1}{3^4}=\frac{289}{6561}
\]



\end{problem}

%%%%%%%%%%%%%%%%%%%%%%%%%%%%%%%%%%%%%%%%%%%%%%%%%%%%%%%%%%

%%%%%%%%%%%%%%%%%%  PROBLEMA 1.9  %%%%%%%%%%%%%%%%%%%%%%%%%
\begin{problem}[9]En una reuni\'on hay 25 personas. Calcular la
probabilidad de que celebren su cumplea\~{n}os el mismo d\'ia del
a\~{n}o al menos dos personas. Observación: con frecuencia es m\'as f\'acil
calcular intersecciones que uniones. Sugerencia: calcular la probabilidad
del evento complementario.
\solution


A = cumpleaños mismo día al menos dos personas.

$A^c$ = cumpleaños distinto día todas las personas.

\[
P(A)=1-P(A^c)=1-\frac{365}{365}\cdot\frac{364}{365}\cdot\frac{363}{365}\cdot...\cdot\frac{341}{365}=1-0.43=0.57
\]

\end{problem}

%%%%%%%%%%%%%%%%%%%%%%%%%%%%%%%%%%%%%%%%%%%%%%%%%%%%%%%%%%

%%%%%%%%%%%%%%%%%%  PROBLEMA 1.10  %%%%%%%%%%%%%%%%%%%%%%%%%
\begin{problem}[10]Inclusi\'on-Exclusi\'on: Probar que 
$
P(\cup_{i=1}^n A_i)= \sum_{k=1}^n \sum_{I\subset \{1, \dots, n\}, |I| = k} (-1)^{k-1}
P(\cap_{i\in I} A_i).
$
\solution

\begin{expla}

Tenemos que probar:

\[
P(\bigcup_{i=1}^nA_i)=\sum_{k=1}^{n}\sum_{I\subset \{1,...,n\}:|I|=k}(-1)^{k-1}P(\bigcap_{i\in I}A_i)
\]

Vamos a ver lo que significa para una colección de 3 subconjuntos $\{A_1,A_2,A_3\}$; de forma que sea fácil de ver. El término que puede llevar a la duda es el segundo sumatorio, sólo dice que escojamos todas las subcolecciones posibles de tamaño k dentro de nuestra colección $\{A_1,A_2,A_3\}$.

\begin{center}
\includegraphics[scale=0.75]{img/Dvenn4.png}
\end{center}

\[
P(A_1\cup A_2 \cup A_3)=\underbrace{(-1)^0P(A_1)+(-1)^0P(A_2)+(-1)^0P(A_3)}_{k=1}+
\]
\[
+\underbrace{(-1)^1P(A_1\cap A_2)+(-1)^1P(A_1\cap A_3)+(-1)^1P(A_2\cap A_3)}_{k=2}+\underbrace{(-1)^2P(A_1\cap A_2\cap A_3)}_{k=3}=
\]
\[
=\underbrace{P(A_1)+P(A_2)+P(A_3)}_{k=1}+\underbrace{(-P(A_1\cap A_2)-P(A_1\cap A_3)-P(A_2\cap A_3))}_{k=2}+\underbrace{P(A_1\cap A_2\cap A_3)}_{k=3}
\]

\end{expla}

\begin{itemize}
        \item Hipótesis:

        Suponemos que \[
        P(\bigcup_{i=1}^nA_i)=\sum_{k=1}^{n}\sum_{I\subset
\{1,...,n\}:|I|=k}(-1)^{k-1}P(\bigcap_{i\in I}A_i)
        \]
        \item Base de inducción: n=2

        Para n = 2 se tiene que $$P(A_1\cup A_2)=P(A_1)+P(A_2)-P(A_1\cap
A_2)$$ Vemos que esto es cierto para cualquier conjunto.

        \item Inducción:
        Supongamos la hipótesis cierta para n. Vamos a demostrar que es
válida para n+1.
        $${P\big(\bigcup_{i=1}^{n+1}A_i\big)=P\big((\bigcup_{i=1}^nA_i)\cup A_{n+1}\big)=P\big(\bigcup_{i=1}^nA_i\big)+P(A_{n+1})-P\big((\bigcup_{i=1}^nA_i)\cap
A_{n+1}\big)}$$
        A esto hemos llegado aplicando la fórmula para n = 2.

        Por hipótesis inductiva sabemos lo que vale $P\big(\bigcup_{i=1}^nA_i\big)$.
        Ahora escribimos $P\big((\bigcup_{i=1}^nA_i)\cap
A_{n+1}\big)=P\big(\bigcup_{i=1}^n(A_i\cap A_{n+1})\big)$

        $$P(\bigcup_{i=1}^n(A_i\cap A_{n+1}))=\sum_{1\leq i_1\leq
n}P(A_{i_1}\cap A_{n+1})\quad -\sum_{1\leq i_1<i_2\leq
n}P\big((A_{i_1}\cap A_{n+1})\cap (A_{i_2}\cap A_{n+1})\big)\quad
$$$$+\sum_{1\leq i_1<i_2<i_3\leq n}P\big((A_{i_1}\cap A_{n+1})\cap
(A_{i_2}\cap A_{n+1})\cap (A_{i_3}\cap
A_{n+1})\big)\;+\;\cdots$$$$+\;(-1)^{n-1}\sum_{1\leq
i_1<\cdots<i_n\leq n}P\big((A_{i_1}\cap A_{n+1})\cap \cdots \cap
(A_{i_n}\cap A_{n+1})\big)$$

        Además tenemos que:
        $$ P\big((A_{i_1}\cap A_{n+1})\cap(A_{i_2}\cap A_{n+1})\cap \cdots
\cap (A_{i_n}\cap A_{n+1})\big)=P(A_{i_1}\cap A_{i_2}\cap\cdots\cap
A_{i_n}\cap A_{n+1})$$
        Al final nos queda que :
        $$P(\bigcup_{i=1}^{n+1}A_i) = \sum_{k=1}^{n}\sum_{I\subset
\{1,...,n\}:|I|=k}(-1)^{k-1}P(\bigcap_{i\in I}A_i) + P(A_{n+1})$$$$ -
\sum_{k=1}^{n}\sum_{I\subset \{1,...,n\}:|I|=k}(-1)^{k-1}P(A_{i_1}
\cap A_{i_2} \cap .... \cap A_{i_k} \cap A_{n+1})$$
        Y esto es igual a:
        $$\sum_{k=1}^{n+1}\sum_{I\subset
\{1,...,n+1\}:|I|=k}(-1)^{k-1}P(\bigcap_{i\in I}A_i)$$
\end{itemize}

\end{problem}

%%%%%%%%%%%%%%%%%%%%%%%%%%%%%%%%%%%%%%%%%%%%%%%%%%%%%%%%%%

%%%%%%%%%%%%%%%%%%  PROBLEMA 1.11  %%%%%%%%%%%%%%%%%%%%%%%%%
\begin{problem}[11] Emparejamientos al azar: tenemos $n$ cartas, que colocamos al azar en $n$ sobres
(en vez de cuidadosamente poner cada carta en su sobre).
Calcular la probabilidad de que alguna carta est\'a en el sobre correcto
(es decir, al menos una carta). Estimar dicha probabilidad cuando
$n\to\infty$. Sugerencia: usar Inclusi\'on-Exclusi\'on.
Observar que la probabilidad de que todas las cartas de 1 a $k$ esten en el sobre correcto
es $(n-k)!/n!$. Respuesta en el l\'{\i}mite: $1 - e^{-1}$.
\solution

\begin{expla}

Para hacernos una idea del problema consideramos n=3 y el siguiente espacio de probabilidad:

$\Omega_3=\{(123),(132),(213),(231),(312),(321)\}$

Que representa todas las posibles combinaciones (numero de carta-numero sobre, es decir, 3!) que puede haber con 3 cartas y 3 sobres. 

Según esto la probabilidad de que la carta 1 este en el sobre 1 sería: $\frac{2}{6} = \frac{2}{3\cdot 2}=\frac{1}{3}$

Sea: $A_i$ = carta n-esima en el sobre correcto.

Se observa fácilmente que: $P(A_i) = \frac{(n-1)!}{n!}$, siendo n el número de cartas totales.

Sea A = todas las cartas en el sobre correcto. Simplemente hay que observar que si la primera carta esta en el sobre correcto, para poner la segunda en su sobre tenemos una carta y un sobre menos donde elegir.

\[
P(A)=P(\bigcap_{j=1}^nA_j)=P(A_1)P(A_2|A_1)P(A_3|A_2\cap A_1)...P(A_j|\bigcap_{i=1}^{n-1}A_i)=
\]
\[
\frac{(n-1)!}{n!}\cdot\frac{(n-2)!}{(n-1)!}\cdot\frac{(n-3)!}{(n-2)!}\cdot...\cdot\frac{(n-n+1)!}{(n-n+2)!}\cdot\frac{(n-n)!}{(n-n+1)!}=\frac{1}{n!}
\]

Sea $B_k$ = todas las cartas de la 1 a la k en el sobre correcto.

\[
P(B_k)=P(\bigcap_{n=1}^kA_n)=P(A_1)P(A_2|A_1)P(A_3|A_2\cap A_1)...P(A_n|\bigcap_{i=1}^{k-1}A_i)=
\]
\[
\frac{(n-1)!}{n!}\cdot\frac{(n-2)!}{(n-1)!}\cdot\frac{(n-3)!}{(n-2)!}\cdot...\cdot\frac{(n-k+1)!}{(n-k+2)!}\cdot\frac{(n-k)!}{(n-k+1)!}=\frac{(n-k)!}{n!}
\]

Esta probabilidad es la misma cojamos las k primeras cartas o k cartas en diferentes posiciones.
\end{expla}

Vamos a usar Inclusión-Exclusión:
\[
P(\bigcup_{i=1}^nA_i)=\sum_{k=1}^{n}\sum_{I\subset \{1,...,n\}:|I|=k}(-1)^{k-1}P(\bigcap_{i\in I}A_i)
\]

C = Al menos una carta en el sobre correcto: 

\[
P(C)=P(\bigcup_{i=1}^nA_i)=\sum_{k=1}^{n}\sum_{I\subset \{1,...,n\}:|I|=k}(-1)^{k-1}\frac{(n-k)!}{n!}=\sum_{k=1}^{n}\left( (-1)^{k-1}\frac{(n-k)!}{n!}\binom{n}{k}\right)=
\]
\[
= \sum_{k=1}^{n}\left( (-1)^{k-1}\frac{(n-k)!}{n!}\cdot \frac{n!}{(n-k)!\cdot k!}\right)= \sum_{k=1}^{n}\left( (-1)^{k-1}\frac{1}{k!}\right) \stackrel{n \rightarrow \infty}{\rightarrow} 1-e^{-1}
\]
Ahora vamos a demostrar que:

\[
\sum_{k=1}^{n}\left( (-1)^{k-1}\frac{1}{k!}\right) \stackrel{n \rightarrow \infty}{\rightarrow} 1-e^{-1}
\]

\begin{proof}
\[
\sum_{k=1}^{n}\left( (-1)^{k-1}\frac{1}{k!}\right) = 1 + \sum_{k=0}^{n}\left((-1)^{k-1}\frac{1}{k!}\right) = 1 - \sum_{k=0}^{n}\left((-1)^{k}\frac{1}{k!}\right)
\]

Por otro lado, aplicando Taylor, sabemos que:

\[
f(x)=e^x=f(0)+\frac{f'(0)}{1!}(x-0)+\frac{f''(0)}{2!}(x-0)^2...
\]

Y por tanto:

\[
f(-1)=e^{-1}=1-\frac{1}{1!}+\frac{1}{2!}-\frac{1}{3!}... = \sum_{k=0}^{n}\left((-1)^{k}\frac{1}{k!}\right)
\]
\end{proof}



\end{problem}

%%%%%%%%%%%%%%%%%%%%%%%%%%%%%%%%%%%%%%%%%%%%%%%%%%%%%%%%%%

%%%%%%%%%%%%%%%%%%  PROBLEMA 1.12  %%%%%%%%%%%%%%%%%%%%%%%%%
\begin{problem}[12] Media o esperanza. Con los datos del problema anterior, calcular el n\'umero esperado de
emparejamientos al azar, es decir, cuantas cartas esperamos que est\'an en el sobre correcto.
Comentario: este problema es muy f\'acil.
\solution

\begin{expla}
\[
\mathbb{E}(X) = \sum_{w \in \Omega} X(w)P(w)
\]

En nuestro caso $\Omega$ esta formado por n! eventos (tal y como vimos en el ejercicio anterior), X es una variable aleatoria que vale "m", siendo m el número de cartas que están bien emparejadas con su sobre.

Llamamos $P(B_k)$ a la probabilidad de k cartas bien emparejadas.
\end{expla}


\[
\mathbb{E}(X) = \sum_{w \in \Omega} X(w)P(w) = \sum_{k=0}^n kP(B_k) = \sum_{k=0}^n k\cdot\frac{(n-k)!}{n!}
\]


\end{problem}

%%%%%%%%%%%%%%%%%%%%%%%%%%%%%%%%%%%%%%%%%%%%%%%%%%%%%%%%%%

%%%%%%%%%%%%%%%%%%  PROBLEMA 1.13  %%%%%%%%%%%%%%%%%%%%%%%%%
\begin{problem}[13] Dado $C$ con $P(C) > 0$, decimos que $A$ y $B$ son condicionalmente independientes
con respecto a $C$ si $P(A\cap B|C) =P(A|C) P(B|C)$. Probar que si  $P(B\cap C) > 0$, 
$P(A\cap B|C) =P(A|C) P(B|C)$ es equivalente a $P(A|C) = P(A|B \cap C)$. 


\solution

\begin{expla}
Partimos de: $P(A\cap B|C)=P(A|C)P(B|C)$  

Despejamos $P(A|C)$ y operamos:
\end{expla}

\[
P(A|C)=\frac{P(A\cap B|C)}{P(B|C)}=\frac{\frac{P(A\cap B\cap C)}{P(C)}}{\frac{P(B \cap C)}{P(C)}}=\frac{P(A \cap B\cap C)}{P(B\cap C)}=P(A|B\cap C)
\]

\end{problem}

%%%%%%%%%%%%%%%%%%%%%%%%%%%%%%%%%%%%%%%%%%%%%%%%%%%%%%%%%%

%%%%%%%%%%%%%%%%%%  PROBLEMA 1.14  %%%%%%%%%%%%%%%%%%%%%%%%%
\begin{problem}[14] Estudiar para $ \alpha>0 $ la convergencia en
media cuadr\'atica (es decir, en $L^2$) de la sucesi\'on $\{X_n\}_{n=1}^\infty$, sabiendo que

\[  P(X_n=n)=\frac{1}{n^\alpha}, \hspace{5mm} P(X_n=0)=1-\frac{1}{n^\alpha}.
  \]
\solution

\begin{expla}
Dado un espacio de probabilidad $(\Omega, \algb{M}, P)$, y una variable aleatoria $X_n$, tenemos que $X_n$ es también una función medible que cumple:

\[
X_n \stackrel{L_2 (n\rightarrow \infty)}{\rightarrow} X \Leftrightarrow \norm{X_n -X}_2  \rightarrow 0 \Leftrightarrow \int{\abs{X_n -X}^2dP} \rightarrow 0
\]

Tenemos que cuando $n \rightarrow \infty$, se cumple que:
\begin{itemize}
\item $P(X_n = n) \rightarrow 0$
\item $P(X_n = 0) \rightarrow 1$
\end{itemize}

Así definimos X como una variable aleatoria que cumple:

\begin{itemize}
\item $P(X = n) = 0$
\item $P(X = 0) = 1$
\end{itemize}

\end{expla}

\[
\int{\abs{X_n -X}^2dP} = \int_{\Omega}{\abs{X_n(w) -X(w)}^2dP(w)}
\]

Como X toma el valor 0 con probabilidad 1, tenemos:

\[
\int_{\Omega}\abs{X_n(w) -X(w)}^2dP(w)=\int_{\Omega}\abs{X_n(w)}^2dP(w)=\int_{\Omega}(X_n(w))^2dP(w) =
\]
\[
=\mathbb{E}((X_n)^2)=n^2\cdot\frac{1}{n^{\alpha}}+0^2\cdot(1-\frac{1}{n^{\alpha}})=\frac{n^2}{n^{\alpha}} \stackrel{n \rightarrow \infty}{\rightarrow} 0 \text{ si } \alpha>2 
\]

Por tanto la sucesión $\{X_n\}_{n=1}^\infty$ converge cuadráticamente si $\alpha > 2$.



\end{problem}

%%%%%%%%%%%%%%%%%%%%%%%%%%%%%%%%%%%%%%%%%%%%%%%%%%%%%%%%%%





\newpage
\section{Hoja 2}

Se asume siempre que estamos trabajando en un espacio de probabilidad $(\Omega, \mathcal{A}, P)$,
y que  $\mathcal{B}\subset \mathcal{A}$ es una sub-$\sigma$-\'algebra.


%%%%%%%%%%%%%%%%%%  PROBLEMA 2.1  %%%%%%%%%%%%%%%%%%%%%%%%%
\begin{problem}[1]Sea $X$ una v.a. con distribuci\'on $N(0,1)$ (normal con media 0 y varianza 1). 
Calcular el tercer momento $E(X^3)$.
\solution

\begin{expla}
Sea una variable aleatoria X, a esta se le asigna una función de distribución ($F_X(t)$) y otra de densidad ($f_X(t)$) que cumplen:

\[
F_X(t)=\int_{-\infty}^{t}f_x(t)dt
\]

En el caso de distribución N(0,1) tenemos:
\[
f(x)=\frac{e^{-t^2/2}}{\sqrt{2\pi}}
\]

Sabiendo que la esperanza se calcula como:
\[
\mathbb{E}(X)=\int_{-\infty}^{\infty}t\cdot f(t)dt
\]

Escribimos:
\end{expla}

\[
\mathbb{E}(X^3)=\int_{-\infty}^{\infty}t^3\cdot \frac{e^{-t^2/2}}{\sqrt{2\pi}}dt = \frac{1}{\sqrt{2\pi}}\int_{-\infty}^{\infty}t^3\cdot e^{-t^2/2}dt
\]

Hacemos un cambio de variable y llamamos $w=\frac{t^2}{2}$ por tanto: $\frac{dw}{dt}=t$

\[
\frac{1}{\sqrt{2\pi}}\int t^3\cdot e^{-t^2/2}dt = \frac{1}{\sqrt{2\pi}}\int 2tw\cdot e^{-w}\frac{dw}{t} = \frac{2}{\sqrt{2\pi}}\int w\cdot e^{-w}dw
\]

Ahora vamos a integrar por partes

$u=w$

$du=1$

$v=-e^{-w}$

$dv=e^{-w}$

\[
\frac{2}{\sqrt{2\pi}}\int w\cdot e^{-w}dw = \frac{2}{\sqrt{2\pi}}\left( -we^{-w}+\int e^{-w}dw \right) =\frac{2}{\sqrt{2\pi}}\left( -we^{-w}-e^{-w} \right)=
\]
\[
= \frac{-2e^{-w}}{\sqrt{2\pi}}(w+1)
\]

Y nos queda que:
\[
\mathbb{E}(X^3)=\int_{-\infty}^{\infty}t^3\cdot \frac{e^{-t^2/2}}{\sqrt{2\pi}}dt = \left.\frac{-2e^{-t^2/2}}{\sqrt{2\pi}}(t^2/2+1)\right|_{-\infty}^{\infty}=0
\]




\end{problem}
%%%%%%%%%%%%%%%%%%%%%%%%%%%%%%%%%%%%%%%%%%%%%%%%%%%%%%%%%%


%%%%%%%%%%%%%%%%%%  PROBLEMA 2.2  %%%%%%%%%%%%%%%%%%%%%%%%%
\begin{problem}[2]Sea $\mathcal{B} \subset \mathcal{A}$ la sub-$\sigma$-\'algebra generada por una partici\'on
$\{A_1, \dots,A_n\}$ de $\Omega$, donde todos los conjuntos de la partici\'on tienen probabilidad positiva.
Dada una v.a. $X$, demostrar que si la variable aleatoria $T(X)$ satisface

1) para todo $B\in \mathcal{B}$, $\int_B T(X) dP = \int_B X dP$, y

2) $T(X)$ es  $\mathcal{B}$-medible, 

entonces cuando $w\in A_i$, $T(X)(w)= \frac{1}{P(A_i)}\int_{A_i} X dP$.

Probar que si  $T(X)(w)$ est\'a definida mediante  $T(X)(w):= \frac{1}{P(A_i)}\int_{A_i} X dP$ cuando  
 $w\in A_i$, entonces $T(X)$ satisface las condiciones 1) y 2) enunciadas arriba.

Este ejercicio demuestra que las condiciones 1) y 2) determinan de modo \'unico la esperanza condicional,
cuando la  sub-$\sigma$-\'algebra est\'a generada por una partici\'on (el resultado tambi\'en es cierto
para  sub-$\sigma$-\'algebras arbitrarias, pero entonces es necesario definir la esperanza condicional
mediante un procedimiento distinto).
\solution

\begin{expla}

Tenemos que demostrar la doble implicación, empezamos de izquierda a derecha y luego lo demostraremos de derecha a izquierda.
\end{expla}

\begin{itemize}
\item $\Rightarrow)$ Sabemos que:
\begin{enumerate}
\item $\forall B \in \algb{B}, \int_{B}T(X)dP = \int_{B}XdP$
\item $T(X)$ es $\algb{B}-medible$
\end{enumerate}

Tenemos que llegar a: $T(X)(w)= \frac{1}{P(A_i)}\int_{A_i} X dP$ con $w \in A_i$.

Partimos del punto 1, sabemos que esa propiedad se cumple $\forall B \in \algb{B}$, por tanto, se cumplirá para todos los elementos $A_i$ de la partición:
\[
\int_{A_i}T(X)dP = \int_{A_i}XdP  \Leftrightarrow  \frac{1}{P(A_i)}\cdot\int_{A_i}T(X)dP = \frac{1}{P(A_i)}\cdot\int_{A_i}XdP
\]

la parte derecha ya es idéntica a la parte derecha de la fórmula a la que queremos llegar, por tanto, sólo nos queda demostrar que: $\frac{1}{P(A_i)}\cdot\int_{A_i}T(X)dP = T(X)(w)$

Observamos que esto es cierto si la integral no depende de T(X) y se puede sacar como constante, ya que de darse eso tendríamos:

\[
\frac{1}{P(A_i)}\cdot\int_{A_i}T(X)dP = \frac{1}{P(A_i)}\cdot T(X)\cdot\int_{A_i}1dP = \frac{P(A_i) \cdot T(X)}{P(A_i)} = T(X)
\]

(Ver definición de integral sobre una medida)

Es aquí cuando usamos que T(X) es $\algb{B}-medible$. Eso quiere decir, que la antiimagen de un medible pertenece a $\algb{B}$. Como los elementos de $\algb{B}$ son los conjuntos $A_i$, quiere decir que T(X) (que es lo mismo que T(X(w))) tiene que tener valor constante para todos los elementos $w \in A_i$. 

¿Por qué?, bueno, piensa que en $A_2$ tenemos 2 elementos: $w_1$ y $w_2$, si $T(X(w_1))=2$ y $T(X(w_2))=4$, entonces, $T^{-1}(1,3)=w_1$ que es un elemento que NO pertenece a $\algb{B}$. Si no ha quedado claro con esta explicación, se puede ver algún ejemplo más tras la demostración de 'Ley de X' de estos apuntes.

Por tanto, definimos T(X)(w)=T(X(w)) y concluimos que T(X)(w) es constante para los w pertenecientes a cada partición $A_i$. Por tanto queda demostrada la implicación primera ya que:

\[
\frac{1}{P(A_i)}\cdot\int_{A_i}T(X)(w)dP = \frac{1}{P(A_i)}\cdot T(X)(w)\cdot\int_{A_i}1dP = \frac{P(A_i) \cdot T(X)(w)}{P(A_i)} =
\]
\[
=T(X)(w)  \text{ para todo } w \in A_i
\]

\item $\Leftarrow)$ Sabemos que:

$T(X)(w)= \frac{1}{P(A_i)}\int_{A_i} X dP$ con $w \in A_i$.

Tenemos que llegar a: 
\begin{enumerate}
\item $\forall B \in \algb{B}, \int_{B}T(X)dP = \int_{B}XdP$
\item $T(X)$ es $\algb{B}$-medible
\end{enumerate}

Empezamos demostrando 2: Lo hacemos por reducción al absurdo. Suponemos que no es $\algb{B}$-medible, entonces existe al menos un conjunto $A_j$ con al menos dos elementos $w_1$ y $w_2$ tal que $T(X(w_1))=c_1$ y $T(X(w_2))=c_2$ con $c_1 \neq c_2$. Por tanto:

\[
c_1=T(X)(w_1)=\frac{1}{P(A_j)}\int_{A_j} X dP \neq \frac{1}{P(A_j)}\int_{A_j} X dP=T(X)(w_2)=c_2
\]

Lo cual es contradictorio, porque como podemos ver, T(X)(w) es constante para elementos pertenecientes al mismo $A_i$, es decir, a la misma clase de equivalencia.

Ahora demostramos 1:

\[
T(X)(w)= \frac{1}{P(A_i)}\int_{A_i}X dP \Leftrightarrow T(X)(w)\cdot P(A_i)= \int_{A_i}X dP
\]

Como T(X) es $\algb{B}$-medible, entonces es constante para todos los w pertenecientes a una misma clase de equivalencia $A_i$, por tanto podemos introducir T(X) dentro de la integral quedando:
\[
\int_{A_i}T(X)dP = \int_{A_i}XdP
\]

Sea $B \in \algb{B}$ de modo que $B=A_{i_1} \cup A_{i_2} \cup ... \cup A_{i_k}$, entonces llegamos a la propiedad 1:

\[
\forall B \in \algb{B}, \int_{B}T(X)dP = \int_{A_{i_1}}T(X)dP + \int_{A_{i_2}}T(X)dP + ... + \int_{A_{i_k}}T(X)dP =
\]
\[
= \int_{A_{i_1}}XdP + \int_{A_{i_2}}XdP + ... + \int_{A_{i_k}}XdP = \int_{B}XdP
\]



\end{itemize}




\end{problem}

%%%%%%%%%%%%%%%%%%%%%%%%%%%%%%%%%%%%%%%%%%%%%%%%%%%%%%%%%%

%%%%%%%%%%%%%%%%%%  PROBLEMA 2.3  %%%%%%%%%%%%%%%%%%%%%%%%%
\begin{problem}[3]En un examen tipo test se plantean 5 preguntas para responder verdadero o falso. 
 Los puntos asignados a las preguntas V o F son: respuesta correcta,
1 punto, respuesta incorrecta,
- 1 punto, en blanco, 0 puntos. Valor m\'{\i}nimo del  problema: 0 puntos. Es decir, si la puntuaci\'on es
negativa se asigna un cero al problema. 

a)  Calcular la nota esperada de un alumno que responda a las 5 perguntas
de manera aleatoria, por ejemplo lanzando una moneda equilibrada 5 veces.

b) Sabiendo que el  alumno ha respondido correctamente a  la
primera pregunta, calcular la nota esperada.
\solution

\textcolor{red}{ESTA SOLUCIONADO EN EL EXAMEN FINAL DEL CURSO 2013/2014}

\begin{expla}
Se nos podría ocurrir plantear el problema de la siguiente manera:

Sea $X_i$ v.a. independientes que cumplen : $P(X_i(w)=1)=1/2=P(X_i(w)=-1) \forall i$

Por tanto, tenemos que:
\[
\mathbb{E}(X_i)=\frac{1}{2}\cdot(1)+\frac{1}{2}\cdot(-1)=0
\] 

Y si consideramos que i=5:
\[
\mathbb{E}\left(\sum_{i=1}^{5}X_i\right)=\sum_{i=1}^{5}\mathbb{E}(X_i)=0
\]

Pero esto no lo podemos plantear así ya que la nota mínima del problema es de 0. Y eso implica que las variables aleatorias $X_i$ no se pueden sumar ya que son dependientes y no expresarían la solución del problema.

Por tanto lo planteamos de otra manera: Sea el siguiente espacio muestral:

$\Omega = \{(00000), (00001), (00010),..., (11111)\}$, hay un total de $2^5=32$ combinaciones diferentes de posibles respuestas. Teniendo en cuenta que las preguntas no se dejan en blanco. Consideramos la variable aleatoria Z, asociada a ese espacio muestral.

\end{expla}

\spart
\begin{itemize}
\item 5 mal: nota del problema = 0
\item 1 bien y 4 mal: nota del problema = 0
\item 2 bien y 3 mal: nota del problema = 0
\item 3 bien y 2 mal: nota del problema = 1
\item 4 bien y 1 mal: nota del problema = 3
\item 5 bien: nota del problema = 5
\end{itemize}

\[
\mathbb{E}(Z)=\frac{\binom{5}{0}}{32}\cdot0+\frac{\binom{5}{1}}{32}\cdot0+\frac{\binom{5}{2}}{32}\cdot0+\frac{\binom{5}{3}}{32}\cdot1+\frac{\binom{5}{4}}{32}\cdot3+\frac{\binom{5}{5}}{32}\cdot5=
\]
\[
=\frac{\binom{5}{3}}{32}\cdot1+\frac{\binom{5}{4}}{32}\cdot3+\frac{\binom{5}{5}}{32}\cdot5=\frac{\frac{5!}{3!\cdot 2!}}{32}\cdot1+\frac{\frac{5!}{4!\cdot 1!}}{32}\cdot3+\frac{1}{32}\cdot5=\frac{10}{32}\cdot1+\frac{5}{32}\cdot3+\frac{1}{32}\cdot5=\frac{10}{32}+\frac{15}{32}+\frac{5}{32}=\frac{30}{32}=0.9375
\]

\spart
En esta ocasión:

A = posibles respuestas sabiendo que la primera respuesta es correcta =$\{(10000), (10001), (10010),..., (11111)\}$.


Estudiamos las posibles dentro del conjunto A, mirando los 4 digitos de mas a la derecha:
\begin{itemize}
\item 4 mal: nota del problema = 0
\item 1 bien y 3 mal: nota del problema = 0
\item 2 bien y 2 mal: nota del problema = 1
\item 3 bien y 1 mal: nota del problema = 3
\item 4 bien: nota del problema = 5
\end{itemize}

\[
\mathbb{E}(Z|A)=\frac{1}{P(A)}\left( \frac{\binom{4}{0}}{32}\cdot0+\frac{\binom{4}{1}}{32}\cdot0+\frac{\binom{4}{2}}{32}\cdot1+\frac{\binom{4}{3}}{32}\cdot3+\frac{\binom{4}{4}}{32}\cdot5 \right)=
\]

\[
= \frac{1}{\frac{1}{2}}\left( \frac{\binom{4}{2}}{32}\cdot1+\frac{\binom{4}{3}}{32}\cdot3+\frac{\binom{4}{4}}{32}\cdot5\right)=
2\cdot \left( \frac{\frac{4!}{2!\cdot 2!}}{32}\cdot1+\frac{\frac{4!}{3!\cdot 1!}}{32}\cdot3+\frac{1}{32}\cdot5 \right) =
\]
\[
= 2\cdot \left(\frac{6}{32}\cdot1+\frac{4}{32}\cdot3+\frac{1}{32}\cdot5 \right) = 2 \cdot \frac{23}{32}=\frac{46}{32}=1.4375
\]
%TODO se ha hecho por la cuenta la vieja


\end{problem}

%%%%%%%%%%%%%%%%%%%%%%%%%%%%%%%%%%%%%%%%%%%%%%%%%%%%%%%%%%

%%%%%%%%%%%%%%%%%%  PROBLEMA 2.4  %%%%%%%%%%%%%%%%%%%%%%%%%
\begin{problem}[4]Lanzamos un dado equilibrado de 4 caras dos veces. Sea $W$ la variable aleatoria que toma como valor
el m\'aximo de los dos lanzamientos. A continuaci\'on lanzamos una moneda equilibrada $W$ veces,  usando
$S$ para denotar el n\'umero de caras obtenido. Todos los lanzamientos son independientes.

a) Hallar $E(S|W)$.
 
b)  Hallar $E(S)$.
\solution

\begin{expla}
Por definición:

\[
E_{P_B}(X)=\frac{1}{P(B)}\int_{B}X(w)dP(w)=E(X|B)
\]

W = máximo de lanzar un dado de 4 caras 2 veces.

Consideramos el siguiente espacio muestral:

$\Omega=\{(1,1),(1,2),(1,3),(1,4),(2,1),(2,2),...,(4,1),(4,2),(4,3),(4,4)\}$

Donde $W(a,b)=max(a,b)$

Sea $P(X_i)=1/2$ probabilidad de que salga cara en el lanzamiento i-ésimo de una moneda. $X(w)=1$ si sale cara y 0 si sale cruz.

Por tanto: definimos $Y_n \sim Binomial(n,\frac{1}{2})$. $Y_n$ corresponde a las caras obtenidas al lanzar una moneda n veces.

\end{expla}


\spart
De manera intuitiva, vemos que el numero de caras que pueden salir, condicionando el numero de tiradas que vamos a hacer (W), es exactamente:

$\mathbb{E}(S|W)=P(cara)\cdot(\text{número de tiradas})=\frac{W}{2}$

Y que responde a una $Binomial(W,\frac{1}{2})$, quedando:

\[
\mathbb{E}(S|W)=\sum_{i=1}^{W}\mathbb{E}(X_i)=\mathbb{E}(Y_W)=\frac{W}{2}
\] 


SOLUCIÓN DEL PROFESOR:
$X_i=1$ si en el lanzamieno i-ésimo sale cara, $X_i=0$ si no. $P(X_i=1)=\frac{1}{2}$

$S_n=X_i + X_2 +...+X_n$

Definimos $S_w$ con $w=max{Y_1,Y_2}$ siendo $Y_1=$ primer lanzamiento e $Y_2=$segundo lanzamiento.

Sabemos que $P(Y_n=m)=\frac{1}{4}$ para $m=1,2,3,4$. Por tanto $P(Y_1=i_1,Y_2=i_2)=\frac{1}{16}$ para $1\leq i_1, i_2 \leq 4$

$P(W=1)=\frac{1}{16}$, $P(W=2)=\frac{2}{16}$, $P(W=3)=\frac{6}{16}$ y $P(W=4)=\frac{7}{16}$

Por tanto:

$$
\mathbb{E}(S|W)(w) =
  \left\lbrace
  \begin{array}{l}
     \mathbb{E}(S|W=1) = \frac{1}{2} \text{ si } w \in W^{-1}(1) \\
     \mathbb{E}(S|W=2) = 1 \text{ si } w \in W^{-1}(2) \\
     \mathbb{E}(S|W=3) = \frac{3}{2} \text{ si } w \in W^{-1}(3) \\
     \mathbb{E}(S|W=4) = 2 \text{ si } w \in W^{-1}(4) \\
  \end{array}
  \right.
$$


%Vamos a intentar aplicar la fórmula que tenemos para la definición de esperanza condicionada, nos quedaría:


%\[
%\mathbb{E}(S|W)=\frac{1}{P(W)}\sum_{w\in W}X(w)P(w)
%\]

%Habría que definir el espacio muestral S, que contemplaría el número de veces que se tira la moneda y las posibilidades de obtener un número de caras para esas veces, es bastante lioso, por lo tanto lo que hacemos es declarar una variable aleatoria $S_{i,w}$ que calcula la probabilidad de obtener cara en el lanzamiento i-ésimo, sabiendo que vamos a tirar w veces.

%\[
%= \frac{1}{P(W)}\sum_{w \in W}(1\cdot P(w=cara)\cdot P(w\in W)+0\cdot P(w=cruz)\cdot P(w\in W)) = 
%\]

%\[
%= \frac{1}{P(W)}\sum_{w \in W}1\cdot P(w=cara)\cdot P(w\in W)= \frac{1}{P(W)}\sum_{w \in W}1\cdot \frac{1}{2}\cdot P(W) =  W\cdot \frac{1}{2}
%\]

%Lo que hemos querido poner es que hay que mirar también la probabilidad de que se realizan W tiradas, ya que el sumatorio se realiza sobre elementos que pertenecen al espacio muestral donde se tiene en cuenta el número de tiradas que se realizan, y que viene determinado por W que es otra variable aleatoria.

\spart
$P(W=1)=\frac{1}{16}$, $P(W=2)=\frac{2}{16}$, $P(W=3)=\frac{6}{16}$ y $P(W=4)=\frac{7}{16}$

%TODO : este ejercicio esta resuelto sin aplicar ninguna fórmula dada en teoría.

\[
\mathbb{E}(S)=\mathbb{E}(\mathbb{E}(S|W))=\sum_{i=1}^{4}P(W=i)\cdot\mathbb{E}(S|W=i)=\sum_{i=1}^{4}\left( P(W=i)\cdot i \cdot \frac{1}{2}\right)= 
\]

\[
=\frac{1}{16}\cdot1\cdot\frac{1}{2}+\frac{2}{16}\cdot2\cdot\frac{1}{2}+\frac{6}{16}\cdot3\cdot\frac{1}{2}+\frac{7}{16}\cdot4\cdot\frac{1}{2}=\frac{1}{32}+\frac{1}{8}+\frac{18}{32}+\frac{28}{32}=\frac{51}{32}
\]
\end{problem}

%%%%%%%%%%%%%%%%%%%%%%%%%%%%%%%%%%%%%%%%%%%%%%%%%%%%%%%%%%


\newpage
\section{Hoja 3}

Salvo afirmaci\'on expresa en sentido
contrario se asume siempre que estamos trabajando en un espacio de probabilidad $(\Omega, \mathcal{A}, P)$,
que  $\mathcal{B}\subset \mathcal{A}$ es una sub-$\sigma$-\'algebra, que las funciones son medibles, etc..

Recordatorio: si $1\le p < \infty$, $\|f\|_p := \left(\int|f|^p\right)^{1/p}$, mientras que
$\|f\|_\infty$ denota el supremo esencial de $|f|$. De hecho, la definici\'on
 $\|f\|_p := \left(\int|f|^p\right)^{1/p}$ tiene sentido para cualquier $p > 0$ finito, pero puede demostrarse que si $p < 1$ esta expresi\'on no define una norma.


%%%%%%%%%%%%%%%%%%  PROBLEMA 3.1  %%%%%%%%%%%%%%%%%%%%%%%%%
\begin{problem}[1] La desigualdad aritm\'etico-geom\'etrica 
dice que la media  aritm\'etica de un conjunto de n\'umeros no negativos es al menos tan grande
como su media  geom\'etrica, es decir, si $a_1, \dots ,a_n > 0$ satisfacen  $a_1 + \cdots  + a_n = 1$,  y  
 $x_1, \dots ,x_n \ge 0$, entonces $\Pi_{i=1}^n x_i^{a_i} \le \sum_{i=1}^n a_i x_i$.
Demostrar dicha desigualdad. Sugerencia: usar la concavidad
de log, o la convexidad de exp.
\solution

\begin{expla}

\end{expla}

\end{problem}
%%%%%%%%%%%%%%%%%%%%%%%%%%%%%%%%%%%%%%%%%%%%%%%%%%%%%%%%%%


%%%%%%%%%%%%%%%%%%  PROBLEMA 3.2  %%%%%%%%%%%%%%%%%%%%%%%%%
\begin{problem}[2] Demostrar la siguiente desigualdad de Young: para $t, u \ge 0$, y $p,q > 1$ tales que
$1/p + 1/q =1$, tenemos $tu \le t^p/ p + u^q/ q$. Observaci\'on: \'esta es otra forma de
escribir la  desigualdad aritm\'etico-geom\'etrica para $n=2$, mediante un cambio obvio de variables.
\solution

\begin{expla}

\end{expla}

\end{problem}

%%%%%%%%%%%%%%%%%%%%%%%%%%%%%%%%%%%%%%%%%%%%%%%%%%%%%%%%%%

%%%%%%%%%%%%%%%%%%  PROBLEMA 3.3  %%%%%%%%%%%%%%%%%%%%%%%%%
\begin{problem}[3] Sean $p,q \ge 1$ exponentes conjugados, es decir, $p$ y $q$ satisfacen
$1/p + 1/q =1$ (si $p=1$, entonces $q = \infty$, y viceversa).
Dado un espacio de medida arbitrario $(\Omega, \mathcal{A}, \mu)$, demostrar la desigualdad de
H\"older: si $f\in L^p$ y $g\in L^q$, entonces $fg\in L^1$, y $\|fg\|_1 \le \|f\|_p\|g\|_q$.
Sugerencias: El caso $p=1, q=\infty$ sale directamente de las definiciones.
Para $p >1$, suponemos que  $\|f\|_p, \|g\|_q\ne 0$, y reemplazamos 
$f$ y $g$ con 
 $f/\|f\|_p$ y $g/\|g\|_q$.  Despu\'es usamos la desigualdad de Young (punto a
punto) e integramos.
\solution

\begin{expla}

\end{expla}

\end{problem}

%%%%%%%%%%%%%%%%%%%%%%%%%%%%%%%%%%%%%%%%%%%%%%%%%%%%%%%%%%

%%%%%%%%%%%%%%%%%%  PROBLEMA 3.4  %%%%%%%%%%%%%%%%%%%%%%%%%
\begin{problem}[4] Dada $f:\Omega\to \mathbb{R}$ en $L^p$,  $1 < p <  \infty$, escoger $g\in L^q$ tal que
$\int fg = \|fg\|_1 = \|f\|_p\|g\|_q$. Concluir que 
$\|f\|_p= \operatorname{sup}_{\{g\in L^q: \|g\|_q=1\}} \int fg $.
\solution

\begin{expla}

\end{expla}

\end{problem}

%%%%%%%%%%%%%%%%%%%%%%%%%%%%%%%%%%%%%%%%%%%%%%%%%%%%%%%%%%

%%%%%%%%%%%%%%%%%%  PROBLEMA 3.5  %%%%%%%%%%%%%%%%%%%%%%%%%
\begin{problem}[5] Usar $\|f\|_p= \operatorname{sup}_{\{g\in L^q: \|g\|_q=1\}} \int fg $ cuando $1 < p <  \infty$
para obtener la desigualdad triangular o de Minkowski: $\|f + g\|_p \le \|f\|_p + \|g\|_p$.
Probar tambi\'en los casos (bastante obvios) $p=1$  y  $p = \infty$.
\solution

\begin{expla}

\end{expla}

\end{problem}

%%%%%%%%%%%%%%%%%%%%%%%%%%%%%%%%%%%%%%%%%%%%%%%%%%%%%%%%%%

%%%%%%%%%%%%%%%%%%  PROBLEMA 3.6  %%%%%%%%%%%%%%%%%%%%%%%%%
\begin{problem}[6] Probar que $\|\cdot\|_p$ es una norma en $L^p$ (considerando que dos funciones
son iguales
cuando son iguales en casi todo punto).
\solution

\begin{expla}

\end{expla}

\end{problem}

%%%%%%%%%%%%%%%%%%%%%%%%%%%%%%%%%%%%%%%%%%%%%%%%%%%%%%%%%%

%%%%%%%%%%%%%%%%%%  PROBLEMA 3.7  %%%%%%%%%%%%%%%%%%%%%%%%%
\begin{problem}[7] Probar la desigualdad de Jensen: si $X:\Omega\to I\subset \mathbb{R}$, donde
$I$ es un intervalo en $\mathbb{R}$, y $X$ tiene media finita, entonces para toda funci\'on
convexa $g:I\to \mathbb{R}$, $g(EX) \le E(g(X))$. 

Sugerencia: n\'otese que al trabajar en un espacio de probabiildad, si $L(t) := a t + b$ es una recta, entonces conmuta con la integraci\'on,
es decir,
$L(\int X(\omega) dP (\omega)) = \int L (X(\omega)) dP (\omega)$. La desigualdad de Jensen
es consecuencia de esta observaci\'on, junto con el hecho de que las funciones convexas
est\'an por encima de todas sus rectas soporte.
\solution

\begin{expla}

\end{expla}

\end{problem}

%%%%%%%%%%%%%%%%%%%%%%%%%%%%%%%%%%%%%%%%%%%%%%%%%%%%%%%%%%

%%%%%%%%%%%%%%%%%%  PROBLEMA 3.8  %%%%%%%%%%%%%%%%%%%%%%%%%
\begin{problem}[8] Probar que si $0 < r\le s \le \infty$,
 entonces $\|f\|_{L^r(\Omega, \mathcal{A}, P)}\le \|f\|_{L^s(\Omega, \mathcal{A}, P)}$, luego $L^s(\Omega, \mathcal{A}, P) \subset L^r(\Omega, \mathcal{A}, P)$ (sugerencia, usar Jensen o H\"older). Demostrar que si el espacio tiene medida infinita, esta
inclusi\'on puede fallar.
\solution

\begin{expla}

\end{expla}

\end{problem}

%%%%%%%%%%%%%%%%%%%%%%%%%%%%%%%%%%%%%%%%%%%%%%%%%%%%%%%%%%



\newpage
\section{Hoja 4}

Salvo afirmaci\'on expresa en sentido
contrario se asume siempre que estamos trabajando en un espacio de probabilidad $(\Omega, \mathcal{A}, P)$,
que  $\mathcal{B}\subset \mathcal{A}$ es una sub-$\sigma$-\'algebra, que las funciones son medibles, etc..

Recordatorio: si $0 < p < \infty$, $\|f\|_p := \left(\int|f|^p\right)^{1/p}$, mientras que
$\|f\|_\infty$ denota el supremo esencial de $|f|$. 

%%%%%%%%%%%%%%%%%%  PROBLEMA 4.1  %%%%%%%%%%%%%%%%%%%%%%%%%
\begin{problem}[1] Demostrar que si $X := \{X_t\}_{t\in T}$  es una colecci\'on uniformemente integrable
de variables aleatorias, entonces
$\|X\|_1 := \sup_{t\in T} \|X_t\|_{1} < \infty$.
\solution

\begin{expla}

\end{expla}

\end{problem}
%%%%%%%%%%%%%%%%%%%%%%%%%%%%%%%%%%%%%%%%%%%%%%%%%%%%%%%%%%


%%%%%%%%%%%%%%%%%%  PROBLEMA 4.2  %%%%%%%%%%%%%%%%%%%%%%%%%
\begin{problem}[2] Sea $Y\in L^1$. Dada la filtraci\'on
$\{\mathcal{A}_n\}_{n=0}^{\infty}$, definimos
$X_n := E(Y|\mathcal{A}_n)$. 
Probar que $X := \{X_n\}_{n=0}^{\infty}$  es una martingala uniformemente integrable, 
adaptada a 
$\{\mathcal{A}_n\}_{n=0}^{\infty}$.  Decidir razonadamente a qu\'e converge.
\solution

\begin{expla}

\end{expla}

\end{problem}

%%%%%%%%%%%%%%%%%%%%%%%%%%%%%%%%%%%%%%%%%%%%%%%%%%%%%%%%%%

%%%%%%%%%%%%%%%%%%  PROBLEMA 4.3  %%%%%%%%%%%%%%%%%%%%%%%%%
\begin{problem}[3] Probar que si $X := \{X_n\}_{n=0}^{\infty}$  es una martingala adaptada a la filtraci\'on
$\{\mathcal{A}_n\}_{n=0}^{\infty}$, entonces para todo $n\ge 0$ tenemos 
que $\sigma (X_0, \dots ,X_n) \subset \mathcal{A}_n$, y $X$  es una martingala adaptada a  
$\sigma (X_0, \dots ,X_n)$. Aqu\'{\i} $\sigma (X_0, \dots ,X_n)$ denota la $\sigma$-\'agebra m\'as
peque\~{n}a que hace que todas las funciones $ X_0, \dots ,X_n $ sean medibles.
\solution

\begin{expla}

\end{expla}

\end{problem}

%%%%%%%%%%%%%%%%%%%%%%%%%%%%%%%%%%%%%%%%%%%%%%%%%%%%%%%%%%

%%%%%%%%%%%%%%%%%%  PROBLEMA 4.4  %%%%%%%%%%%%%%%%%%%%%%%%%
\begin{problem}[4] Probar que si $X := \{X_n\}_{n=0}^{\infty}$  es una submartingala, y 
$0  <  r\le s\le \infty$, entonces $\|X\|_r  \le \|X\|_s$, donde 
$\|X\|_p := \sup_{n\in \mathbb{N}} \|X_n\|_{p}$. 
\solution

\begin{expla}

\end{expla}

\end{problem}

%%%%%%%%%%%%%%%%%%%%%%%%%%%%%%%%%%%%%%%%%%%%%%%%%%%%%%%%%%


\newpage
\section{Hoja 5}

Salvo afirmaci\'on expresa en sentido
contrario se asume siempre que estamos trabajando en un espacio de probabilidad $(\Omega, \mathcal{A}, P)$,
que  $\mathcal{B}\subset \mathcal{A}$ es una sub-$\sigma$-\'algebra, que las funciones son medibles, etc..

Recordatorio: si $0 < p < \infty$, $\|f\|_p := \left(\int|f|^p\right)^{1/p}$, mientras que
$\|f\|_\infty$ denota el supremo esencial de $|f|$. 

%%%%%%%%%%%%%%%%%%  PROBLEMA 5.1  %%%%%%%%%%%%%%%%%%%%%%%%%
\begin{problem}[1] Los l\'{\i}mites superior e inferior de una sucesi\'on de conjuntos  $\{A_n\}_{n=1}^{\infty}  $ se
definen respectivamente como $\limsup_n A_n := \cap_{n\ge 1 } \cup_{k\ge n } A_k$ y 
$\liminf_n A_n := \cup_{n\ge 1 } \cap_{k\ge n } A_k$. Determinar que conjunto es m\'as grande.
Hallar la relaci\'on entre  $\limsup_n A_n$ y  $\limsup_n \mathbf{1}_{ A_n}$. Hacer lo mismo con los
 l\'{\i}mites inferiores.
\solution

\begin{expla}

\end{expla}

\end{problem}
%%%%%%%%%%%%%%%%%%%%%%%%%%%%%%%%%%%%%%%%%%%%%%%%%%%%%%%%%%


%%%%%%%%%%%%%%%%%%  PROBLEMA 5.2  %%%%%%%%%%%%%%%%%%%%%%%%%
\begin{problem}[2] Probar que si $\{X_n\}_{n=1}^{\infty}$  converge c. s. a $X$, entonces para todo $\epsilon > 0$,
$P(\limsup_n\{ |X_n - X| > \epsilon\}) = 0$.
\solution

\begin{expla}

\end{expla}

\end{problem}

%%%%%%%%%%%%%%%%%%%%%%%%%%%%%%%%%%%%%%%%%%%%%%%%%%%%%%%%%%

%%%%%%%%%%%%%%%%%%  PROBLEMA 5.3  %%%%%%%%%%%%%%%%%%%%%%%%%
\begin{problem}[3] Probar que si para todo $\epsilon > 0$,
$P(\limsup_n\{ |X_n - X| > \epsilon\}) = 0$, entonces  $\{X_n\}_{n=1}^{\infty}$  converge casi seguro a $X$.
Sugerencia: Tomar $\epsilon = 1/k$, $k$ natural, y usar el hecho de que la uni\'on numerable de conjuntos
de probabilidad cero tiene probabilidad cero.
\solution

\begin{expla}

\end{expla}

\end{problem}

%%%%%%%%%%%%%%%%%%%%%%%%%%%%%%%%%%%%%%%%%%%%%%%%%%%%%%%%%%

%%%%%%%%%%%%%%%%%%  PROBLEMA 5.4  %%%%%%%%%%%%%%%%%%%%%%%%%
\begin{problem}[4] Probar que la convergencia casi seguro implica la convergencia en probabilidad. Sugerencia: Usar alguno de los
problemas anteriores.
\solution

\begin{expla}

\end{expla}

\end{problem}

%%%%%%%%%%%%%%%%%%%%%%%%%%%%%%%%%%%%%%%%%%%%%%%%%%%%%%%%%%


\newpage
\section{Hoja 6}

Salvo afirmaci\'on expresa en sentido
contrario se asume siempre que estamos trabajando en un espacio de probabilidad $(\Omega, \mathcal{A}, P)$,
que  $\mathcal{B}\subset \mathcal{A}$ es una sub-$\sigma$-\'algebra, que las funciones son medibles, etc..

Recordatorio: si $0 < p < \infty$, $\|f\|_p := \left(\int|f|^p\right)^{1/p}$, mientras que
$\|f\|_\infty$ denota el supremo esencial de $|f|$. 


%%%%%%%%%%%%%%%%%%  PROBLEMA 6.1  %%%%%%%%%%%%%%%%%%%%%%%%%
\begin{problem}[1] Dar un ejemplo de dos variables incorreladas pero dependientes.

\solution

\begin{expla}

\end{expla}

\end{problem}
%%%%%%%%%%%%%%%%%%%%%%%%%%%%%%%%%%%%%%%%%%%%%%%%%%%%%%%%%%


%%%%%%%%%%%%%%%%%%  PROBLEMA 6.2  %%%%%%%%%%%%%%%%%%%%%%%%%
\begin{problem}[2] Decidir razonadamente si la independencia de los sucesos $A$ y $B$ es equivalente a la independencia de $A$ y $B^c$.
\solution

\begin{expla}

\end{expla}

\end{problem}

%%%%%%%%%%%%%%%%%%%%%%%%%%%%%%%%%%%%%%%%%%%%%%%%%%%%%%%%%%

%%%%%%%%%%%%%%%%%%  PROBLEMA 6.3  %%%%%%%%%%%%%%%%%%%%%%%%%
\begin{problem}[3] Hallar la relaci\'on entre la independencia de los sucesos $A$ y $B$, y la independencia de las variables aleatorias $\mathbf{1}_A$ y $\mathbf{1}_B$.
\solution

\begin{expla}

\end{expla}

\end{problem}

%%%%%%%%%%%%%%%%%%%%%%%%%%%%%%%%%%%%%%%%%%%%%%%%%%%%%%%%%%

%%%%%%%%%%%%%%%%%%  PROBLEMA 6.4  %%%%%%%%%%%%%%%%%%%%%%%%%
\begin{problem}[4] Probar que si $X_n\to  X$ en $L^p$, donde  $0 < p \le \infty$, entonces $X_n\to  X$ en probabilidad.
\solution

\begin{expla}

\end{expla}

\end{problem}

%%%%%%%%%%%%%%%%%%%%%%%%%%%%%%%%%%%%%%%%%%%%%%%%%%%%%%%%%%

%%%%%%%%%%%%%%%%%%  PROBLEMA 6.5  %%%%%%%%%%%%%%%%%%%%%%%%%
\begin{problem}[5] Probar que si $X_n\to  X$ en  probabilidad, entonces $X_n\to  X$ en distribuci\'on. Decimos que
 $X_n\to  X$ en distribuci\'on si para todo punto $x$ de continuidad de $F_X$, $\lim_n F_{X_n} (x) = F_X (x)$.
\solution

\begin{expla}

\end{expla}

\end{problem}

%%%%%%%%%%%%%%%%%%%%%%%%%%%%%%%%%%%%%%%%%%%%%%%%%%%%%%%%%%


\newpage
\section{Hoja 7}

Salvo afirmaci\'on expresa en sentido
contrario se asume siempre que estamos trabajando en un espacio de probabilidad $(\Omega, \mathcal{A}, P)$,
que  $\mathcal{B}\subset \mathcal{A}$ es una sub-$\sigma$-\'algebra, que las funciones son medibles, etc..


%%%%%%%%%%%%%%%%%%  PROBLEMA 7.1  %%%%%%%%%%%%%%%%%%%%%%%%%
\begin{problem}[1] Lanzamos una moneda lastrada, con probabilidad de sacar cara
igual a $3/5$. Si sale cara lanzamos un dado equilibrado con cuatro
caras numeradas del 1 al 4, y si sale cruz lanzamos un dado
equilibrado con seis caras numeradas del 1 al 6. Sea $Y$ el n\'umero
obtenido. Denotando $X=1$ si sale cara, $X=0$ si sale cruz, hallar
a) $E(Y|X)$, y  b) $E(Y)$.
\solution

\begin{expla}

\end{expla}

\end{problem}
%%%%%%%%%%%%%%%%%%%%%%%%%%%%%%%%%%%%%%%%%%%%%%%%%%%%%%%%%%


%%%%%%%%%%%%%%%%%%  PROBLEMA 7.2  %%%%%%%%%%%%%%%%%%%%%%%%%
\begin{problem}[2] Tenemos un dado equilibrado con 6 caras numeradas del 1 al 6. Lanzamos la  primera vez y 
apuntamos el n\'umero $x$ obtenido. A continuaci\'on, 
lanzamos el dado $x$ veces y sumamos los valores $y_i$ obtenidos: $s_x = y_1 + \cdots  + y_x$.
Hallar $E(X)$, $E(Y_i)$, y $E(S_X)$, donde $X$ es la variable aleatoria ``resultado del primer lanzamiento", $Y_i$ el resultado del lanzamiento
$ 1 + i$, y $S_X = \sum_{k = 1}^X Y_k$. Se asume que todas
las tiradas son independientes. 
Determinar la relaci\'on entre las tres medias.
\solution

\begin{expla}

\end{expla}

\end{problem}

%%%%%%%%%%%%%%%%%%%%%%%%%%%%%%%%%%%%%%%%%%%%%%%%%%%%%%%%%%

%%%%%%%%%%%%%%%%%%  PROBLEMA 7.3  %%%%%%%%%%%%%%%%%%%%%%%%%
\begin{problem}[3] a) Enunciar la Ley Fuerte de Los Grandes N\'umeros.

b) Consideramos $(0,1)$ con la medida de Lebesgue, Escribimos $w\in (0,1)$ usando
la expansi\'on decimal habitual. Definimos $X_n(w)$ como el n\'umero de sietes que aparecen
en las $n$ primeras posiciones de la expansi\'on decimal de $w$ (por ejemplo, $X_3(0.777) = 3,
 x_3(0.12345) = 0$.  Decidir razonadamente si $\lim_n n^{-1}X_n$ converge casi seguro, y
 en caso de respuesta afirmativa, hallar el l\'{\i}mite.
 
 c) Decidir razonadamente si 
 $$
 \lim_n \frac{X_n - 10^{-1}n}{n^{2/3}}
 $$ 
 converge casi seguro, y
 en caso de respuesta afirmativa, hallar el l\'{\i}mite.
\solution

\begin{expla}

\end{expla}

\end{problem}

%%%%%%%%%%%%%%%%%%%%%%%%%%%%%%%%%%%%%%%%%%%%%%%%%%%%%%%%%%

%%%%%%%%%%%%%%%%%%  PROBLEMA 7.4  %%%%%%%%%%%%%%%%%%%%%%%%%
\begin{problem}[4]  Calcular las funciones generatrices de probabilidad, de momentos y las funciones caracter\'{\i}sticas de $X$ cuando $X$ es
a) Bernoulli$(p)$,
b) Binomial$(n,p)$,
c) Poisson$(\lambda)$. Usar unicidad y la propiedad multiplicativa para conclu\'{\i}r que la suma de v.a. independientes
con distribuci\'on Poisson$(\lambda_i)$, $i = 1, \dots, n$, es Poisson$(\sum_1^n \lambda_i)$.
\solution

\begin{expla}

\end{expla}

\end{problem}

%%%%%%%%%%%%%%%%%%%%%%%%%%%%%%%%%%%%%%%%%%%%%%%%%%%%%%%%%%

%%%%%%%%%%%%%%%%%%  PROBLEMA 7.5  %%%%%%%%%%%%%%%%%%%%%%%%%
\begin{problem}[5] Calcular la funci\'on generatriz de momentos y la funci\'on caracter\'{\i}stica de $X\sim \operatorname{Uniforme}(0,1)$. 
\solution

\begin{expla}

\end{expla}

\end{problem}

%%%%%%%%%%%%%%%%%%%%%%%%%%%%%%%%%%%%%%%%%%%%%%%%%%%%%%%%%%

%%%%%%%%%%%%%%%%%%  PROBLEMA 7.6  %%%%%%%%%%%%%%%%%%%%%%%%%
\begin{problem}[6]Calcular la funci\'on generatriz de momentos y la funci\'on caracter\'{\i}stica de $Z\sim N(0,1)$. 
 
\solution

\begin{expla}

\end{expla}

\end{problem}

%%%%%%%%%%%%%%%%%%%%%%%%%%%%%%%%%%%%%%%%%%%%%%%%%%%%%%%%%%

%%%%%%%%%%%%%%%%%%  PROBLEMA 7.7  %%%%%%%%%%%%%%%%%%%%%%%%%
\begin{problem}[7]Probar que la funci\'on caracter\'{\i}stica de $X$ con media 0 y varianza $1$
satisface
$$
\phi_X(t) = 1 - \frac{1}{2} t^2   + o(t^2)
$$ 
cuando $t\to 0$. Sugerencia: usar la expansi\'on de Taylor de $e^{ir}$, donde $r$ es real.
\solution

\begin{expla}

\end{expla}

\end{problem}

%%%%%%%%%%%%%%%%%%%%%%%%%%%%%%%%%%%%%%%%%%%%%%%%%%%%%%%%%%

%%%%%%%%%%%%%%%%%%  PROBLEMA 7.8  %%%%%%%%%%%%%%%%%%%%%%%%%
\begin{problem}[8] Demostrar el TCL usando los ejercicios anteriores, unicidad y la propiedad multiplicativa de las funciones
caracter\'{\i}sticas, y el Teorema de Continuidad de L\'evy-Cram\'er.
\solution

\begin{expla}

\end{expla}

\end{problem}

%%%%%%%%%%%%%%%%%%%%%%%%%%%%%%%%%%%%%%%%%%%%%%%%%%%%%%%%%%

%%%%%%%%%%%%%%%%%%  PROBLEMA 7.9  %%%%%%%%%%%%%%%%%%%%%%%%%
\begin{problem}[9]Para $n=1, 2, 3, \dots$, sea  $\{X_{n,m}\}_{m=1}^{n}$  una configuraci\'on triangular de v.a., tales que las variables en cada fila son independientes, sea $S_{n} := \sum_{i=1}^n X_{n,i}$, y sean $0 < a < b < 1$. Supongamos que las
variables en la fila $n$ son Bernoulli($p_n$), es decir, $P(X_n = 1) = p_n$, $P(X_n = 0) = 1 - p_n$.
Usar el Teorema de Berry-Esseen para demostrar que si $a \le p_n \le b$ para todo $n$,
entonces $(S_n - n p_n)/\sqrt{n p_n (1 - p_n)}$ converge en distribuci\'on a $Z\sim N(0,1)$.
\solution

\begin{expla}

\end{expla}

\end{problem}

%%%%%%%%%%%%%%%%%%%%%%%%%%%%%%%%%%%%%%%%%%%%%%%%%%%%%%%%%%

%%%%%%%%%%%%%%%%%%  PROBLEMA 7.10  %%%%%%%%%%%%%%%%%%%%%%%%%
\begin{problem}[10]Probar que si $\{X_n\}_{n=0}^{\infty}$  es una sucesi\'on de v.a.i.i.d. en $L^2$, se cumple
la condici\'on de Lindeberg. Sugerencia, usar el Teorema de la Convergencia Dominada de
Lebesgue.
\solution

\begin{expla}

\end{expla}

\end{problem}

%%%%%%%%%%%%%%%%%%%%%%%%%%%%%%%%%%%%%%%%%%%%%%%%%%%%%%%%%%

%%%%%%%%%%%%%%%%%%  PROBLEMA 7.11  %%%%%%%%%%%%%%%%%%%%%%%%%
\begin{problem}[11] Para $n=1, 2, 3, \dots$, sea  $\{X_{n, m}\}_{m=1}^{n}$  una configuraci\'on triangular de v.a., tales que las variables en cada fila son independientes, y sea $S_{n} := \sum_{i=1}^n X_{n, i}$.

a) Enunciar la condici\'on de Lindeberg.

b) Suponiendo que las variables en la fila $n$ son Bernoulli($p_n$), con
$p_n = n^{- 1/2}$,
decidir razonadamente
si 
$(S_n - n p_n)/\sqrt{n p_n (1 - p_n)}$ converge en distribuci\'on a $Z\sim N(0,1)$.

c) Sea $s \in (0,1)$. Decidir razonadamente qu\'e sucede si en el apartado anterior, en
vez de  $p_n = n^{ - 1/2}$ tenemos $p_n = n^{ - s}$.

d) Decidir razonadamente qu\'e sucede si en el apartado b), en
vez de  $p_n = n^{- 1/2}$ tenemos $p_n = n^{ - 1}$.
Sugerencia: recordar la Ley de los N\'umeros Peque\~nos.
\solution

\begin{expla}

\end{expla}

\end{problem}

%%%%%%%%%%%%%%%%%%%%%%%%%%%%%%%%%%%%%%%%%%%%%%%%%%%%%%%%%%

%%%%%%%%%%%%%%%%%%  PROBLEMA 7.12  %%%%%%%%%%%%%%%%%%%%%%%%%
\begin{problem}[12] Probar que la condici\'on de Lyapunov implica la condici\'on de Lindeberg. 
La condici\'on de Lyapunov nos dice que existe un $\delta > 0$ tal que
$\lim_{n\to \infty} Lyap(n, \delta) = 0$, donde 
$$
Lyap(n, \delta) := \frac{1}{s_n^{2 + \delta}} \sum_{k=1}^n  E|X_k - \mu_k|^{2 + \delta} .
$$
\solution

\begin{expla}

\end{expla}

\end{problem}

%%%%%%%%%%%%%%%%%%%%%%%%%%%%%%%%%%%%%%%%%%%%%%%%%%%%%%%%%%





%%%%%%%%%%%%%%%%%%%%%%%%%%%%%%%%%%%%%%%%%%%%%%%%%%%%%%%%%%%%%%%%%%%%%%%%%%%%%%%



\chapter{Examenes}
%
% Soluciones a los examenes de Probabilidad II.
%
% Curso 2014 - 2015 2º cuatrimestre
%

%%%%%%%%%%%%%%%%%%%%%%%%%%%%%%%%%%%%%%%%%%%%%%%%%%%%%%%%%%%%%%%%%%%%%%%%%%%%%%%

Solución a los exámenes parcial y final del curso 2013-2014

\includepdf[pages={1},scale=1]{pdf/_Probabilidad2Parcial2014soluciones.pdf}

\includepdf[pages={1-8}, scale=1]{pdf/_SolnFinalMayo2014.pdf}



\end{document}

