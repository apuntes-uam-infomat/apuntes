\documentclass[nochap]{apuntes}

\usepackage{hyperref}

\usepackage{tikztools}
\usepackage{fastbuild}
\usepackage{tikz-3dplot}

\usepackage{tikz}
\usepackage{graphicx}
\usepackage{latexsym, amsfonts, amsmath, amssymb, amscd, epsfig,amsthm}
\usepackage{enumitem}
\usepackage{natbib}
\usepackage[nottoc]{tocbibind}
\usepackage{fancysprefs}

\usepackage{listings}
\lstset{
    language=R,
    basicstyle=\ttfamily
}

\usepackage{caption}


\definecolor{codegreen}{rgb}{0,0.6,0}
\definecolor{codegray}{rgb}{0.5,0.5,0.5}
\definecolor{codepurple}{rgb}{0.58,0,0.82}
\definecolor{backcolour}{rgb}{0.95,0.95,0.92}

\lstdefinestyle{mystyle}{
    backgroundcolor=\color{backcolour},
    commentstyle=\color{codegreen},
    keywordstyle=\color{magenta},
    numberstyle=\tiny\color{codegray},
    stringstyle=\color{codepurple},
    basicstyle=\footnotesize,
    breakatwhitespace=false,
    breaklines=true,
    captionpos=b,
    keepspaces=true,
    numbers=left,
    numbersep=5pt,
    showspaces=false,
    showstringspaces=false,
    showtabs=false,
    tabsize=2
}



\input xy
\xyoption{all} %%!!
\usetikzlibrary{calc, intersections}
\author{Víctor de Juan}
\date{2014/2015 2º cuatrimestre}

\renewcommand*{\arraystretch}{1.5}
\title{Estadística II}
\precompileTikz

\begin{document}


\pagestyle{plain}
\maketitle

\tableofcontents
\newpage

% -*- root: ../EstadisticaII.tex -*-
\section{Introducción}
Se presentan apuntes de Estadística II, tomados de la clase dada por José Berrendero.

El profesor nos facilita unas diapositivas a las que se harán referencia alguna vez.

\subsection{Vectores aleatorios}


Sea $X = (X_1, \dots , X_p)'$ un vector aleatorio p-dimensional.


\begin{defn}[Esperanza]
	Definimos la esperanza como el vector de medias, es decir:
	\[
	\mathbb{E}(X) = μ = (μ_1,\dots,μ_p)'
	\]
	donde $μ_i = \mathbb{E}(X_i)$.
\end{defn}


\paragraph{Propiedades:}
\begin{enumerate}
\item $\mathbb{E}(X+c) = \mathbb{E}(X)+c$.
\item Sea A una matriz cuadrada de dimensión $n\times p$ siendo $p$ la dimensión de $X$ \[\mathbb{E}(AX) = A\mathbb{E}(X)\]

\end{enumerate}



\begin{defn}[Covarianza]
\[\sigma_{i,j} = \cov{X_i,X_j} = \mathbb{E}\left((X_i-\mathbb{E}(X_i))(X_j-\mathbb{E}(X_j))\right) = \mathbb{E}(X_i X_j)-\mathbb{E}(X_i)\mathbb{E}(X_j)\]

Dos propiedades importantes de la covarianza son:

\begin{enumerate}
\item $\cov{X,X}= \var{X}$
\item $\cov{X,Y}=\cov{Y,X}$
\end{enumerate}

\end{defn}

Al tener varias variables, ya no podemos hablar de varianzas. Definimos el correspondiente p-dimensional de la varianza.


\begin{defn}[Matriz de covarianzas]
	Llamamos $\var{X} = Σ$ a la matriz de covarianzas, cuya posición $(i,j)$ es $σ_{ij} = \cov{X_i,X_j}$.


	Curiosidades:

	\begin{itemize}
		\item Por la definición de covarianza, la diagonal de esta matriz es el vector p-dimensional cuya entrada $i$ es la varianza de $X_i$.
		\item Es una matriz \textbf{simétrica} ya que $\cov{X_i,X_j} = \cov{X_j,X_i}$
	\end{itemize}

	Además: \[\var{X} = \mathbb{E}[(X-μ)(X-μ)'] = \mathbb{E}(XX')-μμ'\]

\end{defn}

Vamos a demostrar esta útlima afirmación.

\begin{proof}
\[
\var{X}=\mathbb{E}\left((X-\mu)(X-\mu)'\right) = \mathbb{E}(XX'- \mu X' - X \mu'+\mu \mu')=
\]
\[
\mathbb{E}(XX')-\mathbb{E}(\mu X')-\mathbb{E}(X\mu')+\mathbb{E}(\mu \mu')= \mathbb{E}(XX')-\mu \mathbb{E}(X')-\mu' \mathbb{E}(X)+\mu\mu'=
\]
\[
 \mathbb{E}(XX')-\mu \mu'-\mu' \mu+\mu \mu' = \mathbb{E}(XX')-\mu \mu'=\Sigma
\]
\end{proof}

\paragraph{Propiedades:}

Sea $X$ un vector aleatorio p-dimensional, $A$ una matriz $n\times p$ y $b∈ℝ^n$

\begin{enumerate}
\item $\var{AX+b} = \mathbb{E}\left[ A(X-\mu)(X-\mu)'A' \right]=A \Sigma A'$.
\begin{proof}
\[
\var{AX+b} = \mathbb{E}\left[ (AX+b-A\mu-b)(AX+b-A\mu-b)' \right] =
\]
\[
 =\mathbb{E}\left[ (AX-A\mu)(AX-A\mu)' \right] = \mathbb{E}\left[A(X-\mu)(X-\mu)'A'\right] = A\mathbb{E}\left[(X-\mu)(X-\mu)'\right]A' =
 \]
 \[
 =A \Sigma A'
\]
\end{proof}
\label{propiedades:esperanzaYvarianza}

\subitem Con esta propiedad podemos deducir más fácilmente expresiones como $\var{X_1 - X_2}$ de la siguiente manera:
\[\var{X_1 - X_2} = (1,-1) \begin{pmatrix}σ_1^2&σ_{12}\\ σ_{21} & σ_2^2\end{pmatrix} \begin{pmatrix}1\\-1\end{pmatrix} = σ_1^2+σ_2^2 - 2σ_{12}\]

\item Si recordamos, $\var{X} > 0$. La versión matricial dice $Σ$ es semidefinida positiva.
\begin{proof}
Sea $a_i\in ℝ$ y $X = (X_1,\dots X_n)$ un vector aleatorio.

\[
0 ≤ \var{\sum_{i=1}^p a_iX_i} = \var{a'X}
\]

Por la propiedad anterior, tenemos: \[ \var{a'X} = a'Σa\], con lo que $Σ$ tiene que ser semidefinida positiva.
\end{proof}

\subitem Si $Σ$ no es definida positiva, $\implies \exists a\in ℝ^p \tq a'Σa = 0 \implies V(a'X) = 0 \implies \exists c\in ℝ \tq P(a'X = c) = 1$. Si esto se da el vector $X$ toma valores con probabilidad 1 en un subespacio de dimensión inferior a p. En el caso de $p=2$, las variables se situarían sobre una recta.

\item Sea $\vec{Y}$ un vector aleatorio $\vec{Y}\inℝ^p$con matriz de covarianzas $Σ$, distribuido normal-multidimensionalmente. Sean $A\vec{Y}$ y $B\vec{Y}$ 2 combinaciones lineales, con $A\in ℝ^q\times ℝ^p$ y $B\in ℝ^r\times ℝ^p$. Entonces:
\label{propiedad:CovCombinacionLineal}
\[\cov{A\vec{Y},B\vec{Y}} = AΣB' = BΣA'\]

\begin{proof}
\[\begin{pmatrix}A\vec{Y}B\vec{Y}\end{pmatrix} = (A,B)' \vec{Y}\equiv N_{q+r}\left(\begin{pmatrix}Aµ\\Bµ\end{pmatrix} , \begin{pmatrix}A\\B\end{pmatrix}Σ\begin{pmatrix}A'&B'\end{pmatrix} \right)
\]

Desarrollando la matriz de covarianzas: \[
\begin{pmatrix}A\\B\end{pmatrix}Σ\begin{pmatrix}A'&B'\end{pmatrix} =
\begin{pmatrix}
AΣA' & AΣB' \\ BΣA' & BΣB'
\end{pmatrix}
\]

Y la covarianza es cualquiera de los elementos que no está en la diagonal (que como es simétrica, deberían ser iguales)
\end{proof}

\end{enumerate}


\subsection{Función característica}
La función característica de un vector aleatorio $X$ es:

\[
\phi_X(t)=\mathbb{E}(\exp^{it'X})\; t\in ℝ^p
\]

\paragraph{Propiedades:} Lo interesante de esta función (como pasaba en el caso unidimensional) es lo siguiente:

\begin{prop} Sean X e Y dos vectores aleatorios:
\[
\phi_X(t)=\phi_Y(t) \Leftrightarrow X \stackrel{d}{=} Y
\]

\end{prop}


\begin{prop}[Mecanismo de Cramer-Wold]
\[a'X \overset{d}{\equiv} a'Y, \; ∀a\in ℝ^p \dimplies X \overset{d}{=} Y\]
\end{prop}

\begin{proof}
\paragraph{$\implies$} es trivial.


\paragraph{$\impliedby$} se demuestra utilizando funciones características y tomando  $t = 1$.
\end{proof}

También se cumple que:

\[X_n \convs[d] X  \dimplies a'X_n \convs[d] a'X\; ∀a∈ℝ^n\]


\subsection{Normal multivariante}

Habiendo definido lo que es un vector aleatorio, vamos a definir la distribución normal multivariante, que aparecerá continuamente a lo largo del curso.

\begin{defn}[Normal p-dimensional]El vector aleatorio $X$ es normal p-dimensional  con vector de medias μ y vector de covarianzas Σ si tiene densidad dada por:
\[
f(x) = |Σ|^\frac{-1}{2}(2π)^{\frac{-p}{2}}exp\left\{ -\frac{1}{2}(x-μ)'Σ^{-1}(x-μ) \right\}\; x∈ℝ^p
\]
\label{def:Normal_multivariante}


\paragraph{Notación:} $X \equiv N_p(μ,Σ)$ significa: $X$ es normal p-dimensional con media μ y matriz de covarianzas Σ.
\end{defn}

\begin{prop}
Sea $\vec{X}\in ℝ^p$ un vector aleatorio.

\[\vec{X} \sim N_p(µ,Σ) \implies ∀\vec{a}\in \real^p, \vec{a}\vec{X}' \sim N_1(\vec{a}\vec{µ},\vec{a}Σ\vec{a}')\]
\end{prop}

\begin{proof}

Que la media y la varianza son esas, está calculado anteriormente en \ref{propiedades:esperanzaYvarianza}.

Lo de que sean una normal, \href{https://en.wikipedia.org/wiki/Normally_distributed_and_uncorrelated_does_not_imply_independent}{Wikipedia}

Corroborado por correo con José R.
\end{proof}


Vamos a profundizar en esta definición porque es clave, como veremos más adelante en la \fref{sec:incorrnotindep} entre otras.

Sean \[X_1 \equiv N(0,1)\;\; X_2 = N(0,2)\] ¿El vector $\vec{X}= (X_1,X_2)$ cumple $\vec{X}\sim N_2 (µ,Σ)$?

\begin{prop}
\label{prop:NormalidadConjuntaIncorrelacionIndependencia}
Sean $X_i \sim N(µ_i,σ_i)$
\begin{itemize}
	\item $X_i$ independientes \textbf{entonces} $\vec{X} = (X_1,...,X_n) \sim N_n(µ,Σ)$
	\subitem Al ser independientes son incorreladas y por tanto la matriz de covarianzas será diagonal.
	\item $\corr{X_i,X_j} = 0$ y $\vec{X} = (X_1,...,X_n) \sim N_n(µ,Σ)$ \textbf{entonces} son independientes.
\end{itemize}
\end{prop}


\begin{example}
Vamos a ver un ejemplo en dimensión 2 para ilustrar cómo reconocer si un conjunto de datos tiene una distribución normal.

Sean $μ = (0,0)'$ y $Σ = \begin{pmatrix} λ_1 & 0 \\ 0 & λ_2 \end{pmatrix}$

Vamos a ver sus conjuntos de nivel tomando:

\[(X_1,X_2) Σ^{-1} (X_1,X_2)' = cte \implies \frac{x_1^2}{λ_1} + \frac{x_2^2}{λ_2} = cte\]


Dependiendo de los valores de $λ_1,λ_2$ tendremos casos distintos.

\begin{itemize}
	\item Si $λ_1 = λ_2$, entonces tendremos circunferencias.
	\item $λ_1 ≠ λ_2$ entonces tendremos elipses.
	\subitem
	Estas elipses tendrán como eje mayor uno de los 2 ejes, ya que las variables son independientes ($\cov{X_1,X_2} = 0$).
	\subitem Si por el contrario, $Σ$ no fuera diagonal, entonces las variables no serían independientes y tendríamos una correlación entre las variables provocando que el eje mayor de la elipse fuera una recta que no corresponde con ninguno de los ejes.
\end{itemize}


Para más información consultar las transparencias de Berrendero, en las que hay un ejemplo.

\end{example}


\subsection{Incorreladas no implica independientes}
\label{sec:incorrnotindep}
\textcolor{gray}{Esta sección sale de \href{https://en.wikipedia.org/wiki/Normally_distributed_and_uncorrelated_does_not_imply_independent}{Wikipedia}}


\begin{defn}[Correlación]
Sean $X,Y$ dos variables aleatorias. Se define el coeficiente de correlación:
\[\rho_{X,Y}=\corr{X,Y}={\cov{X,Y} \over \sigma_X \sigma_Y}\]

\paragraph{Curiosidades:}

\begin{itemize}
	\item $\rho_{X,Y} ∈ [-1,1]$ Así, podemos comparar todas las correlaciones independientemente de las variables.
	\item 2 variables se llaman \concept{Variables incorreladas} si y sólo si su coeficiente de correlación es $0$.
	\item Obviamente, $\cov{} = 0 \implies \corr{} = 0$
\end{itemize}
\end{defn}


\paragraph{¿Incorrelación implica independencia?}

En general es un problema muy interesante y útil la independencia o no de variables y la correlación es algo fácil de calcular, pero \textbf{incorrelación \underline{NO} implica independencia}. Esto sólo ocurre en algunos casos.

\begin{itemize}
	\item Cuando las variables $X,Y$ son Bernoulli, entonces incorrelación si implica independencia.\footnote{Esto para el curso da igual, pero es interesante de saber}
	\item Si $\vec{X} = (X_1,X_2) \sim N_2(µ,Σ)$ y $\corr{X_1,X_2} = 0$, entonces $X_1$ es independiente de $X_2$.

	\subitem La condición de $\vec{X} \sim N_2(µ,Σ)$ es muy importante. Si sólo tuviéramos $X_i \sim N_1(µ,σ)$, entonces incorrelación \textbf{no} implica independencia.

	Es por ello que este comentario se sitúa después de la definición de normal multivariante (\fref{def:Normal_multivariante})
\end{itemize}

\begin{prop}
Sea $\vec{Y}$ un vector distribuido normalmente.

\textbf{Entonces:} un vector cualquiera $\vec{X}$ se distribuye normalmente si lo podemos escribir en la forma $A\vec{Y}$, para una matriz A.
\end{prop}

\begin{proof}
Nos lo creemos del correo electrónico.
\end{proof}



\subsection{Estandarización multivariante}


Al igual que en el caso unidimensional, nos interesaría poder transformar una normal de media $μ$ y varianza $Σ$ en una $N(0,1)$. A continuación vamos a ver ese proceso con una normal multivariante.


\begin{prop}[Estandarización multivariante] Si $X \equiv N_p(\mu, \Sigma)$ y definimos $Y = \Sigma^{-1/2}(X-\mu)$, entonces $Y_1,...,Y_p$ son i.i.d. N(0,1).\end{prop}

\begin{proof}
Sabemos por definición que:
\[
f_X(x)=\abs{\Sigma}^{-1/2}(2\pi)^{-p/2} exp \left( -\frac{1}{2}(x-\mu)' \right)
\]

Vamos a aplicar un cambio de variable en la fórmula de la densidad:

Despejando de $Y = h(X)= \Sigma^{-1/2}(X-\mu)$, obtenemos que $\Sigma^{1/2}Y+\mu=h^{-1}(Y)=X$.

Y ahora cogemos el Jacobiano de $h^{-1}(Y)=X$ que será $\Sigma^{1/2}$ ($\mu$ es una constante e Y es la variable).

También hay que considerar la exponencial de la fórmula de la densidad, ahi hacemos el cambió de variable de:

$$e^X \text{ por } e^{h^{-1}(Y)}=e^{\Sigma^{1/2}Y+\mu}$$

Y el Jacobiano sería $e^{\Sigma^{1/2}Y}$:


Por tanto nos quedaría:
\[
f(X) = f(h^{-1}(Y))·\abs{Jh(x)} = \abs{\Sigma}^{-1/2}(2 \pi)^{-p/2} \exp\left(-\frac{1}{2}(\Sigma^{-1/2}Y+\mu-\mu)'  \right) \exp\left( \Sigma^{1/2}Y \right) \Sigma^{1/2}  =
\]
\[
= \abs{\Sigma}^{-1/2}(2 \pi)^{-p/2} \exp\left(-\frac{1}{2}(\Sigma^{-1/2}Y)' \right) \exp\left( \Sigma^{1/2}Y \right) \abs{\Sigma}^{1/2} =
\]
\[
\abs{\Sigma}^{-1/2}(2 \pi)^{-p/2} \exp\left(-\frac{1}{2}(Y'\Sigma^{-1/2}\Sigma^{1/2}Y \right) \abs{\Sigma^{1/2}} = (2 \pi)^{-p/2} \exp\left(-\frac{1}{2}(Y'Y) \right)
\]
\end{proof}



Vamos a ver un ejemplo para profundizar en la distribución.
\begin{example}
Definimos el siguiente vector aleatorio: $X = (X_1,X_2,X_3)' \equiv N_3(\mu, \Sigma)$ con:

\[
\mu=
\left(
\begin{array}{c}
0\\
0\\
0
\end{array}
\right) \text{,       }
\Sigma=
\left(
\begin{array}{ccc}
7/2& 1/2& -1 \\
1/2& 1/2& 0 \\
-1& 0& 1/2
\end{array}
\right)
\]

\ppart Calcula las distribuciones marginales $X_i \equiv N(\mathbb{E}(X_i), \var{X_i})$:

$X_1\equiv N(0, 7/2)$

$X_2\equiv N(0, 1/2)$

$X_3\equiv N(0, 1/2)$

Para calcular estos valores solo hace falta mirar los datos que nos da el problema, el vector de medias $\mu$ y la matriz de covarianzas $\Sigma$:

\[
\Sigma=\left(
\begin{array}{ccc}
\var{X_1}& \sigma_{1,2}& \sigma_{1,3} \\
\sigma_{2,1}& \var{X_2}& \sigma_{2,3} \\
\sigma_{3,1}& \sigma_{3,2}& \var{X_3}
\end{array}
\right)
\]

\[
\mu=
\left(
\begin{array}{c}
\mathbb{E}(X_1)\\
\mathbb{E}(X_2)\\
\mathbb{E}(X_3)
\end{array}
\right)=
\left(
\begin{array}{c}
\mu_1\\
\mu_2\\
\mu_3
\end{array}
\right)
\]

\ppart Calcula la distribución del vector $(X_1,X_2)'$:

Este vector sigue una distribución normal que puede obtener de las matriz $\Sigma$ y el vector de medias $\mu$:
\[
\left(
\begin{array}{c}
X_1\\
X_2
\end{array}
\right)
\equiv N_2\left[
\left(
\begin{array}{c}
0\\
0
\end{array}
\right)
\text{, }
\left(
\begin{array}{cc}
7/2& 1/2 \\
1/2 & 1/2
\end{array}
\right)
\right]
\]

\ppart ¿Son $X_2$ y $X_3$ independientes?

Sí son independientes ya que la covarianza entre ambas variables es 0. La covarianza entre $X_2$ y $X_3$ es el elemento de la fila 3 y la columna 2 de la matriz de covarianzas $\Sigma$, (que al ser $\Sigma$ simétrica coincide con el elemento de la fila 2 y la columna 3).

\ppart ¿Es $X_3$ independiente del vector $(X_1, X_2)'$?

No son independientes ya que el vector de covarianzas entre ambas variables no es 0. Como en el caso anterior, tomamos como el elemento que ocipa la fila 3 y las columnas 1 y 2, es decir, el vector $(-1,0)$, que al no ser idénticamente nulo, concluimos que $X_3$ no es independiente del vector $(X_1,X_2)$


\ppart Calcula la  distribución de la variable aleatoria $(2X_1-X_2+3X_3)$.

Procedemos de la siguiente manera:

\[
(2X_1-X_2+3X_3)=(2,-1,3)\left(
\begin{array}{c}
X_1\\
X_2\\
X_3
\end{array}
\right)\equiv
N\left( 0,  \right)
\]

\end{example}



\subsection{Distribuciones condicionadas}

\begin{prop}

Sea $X=(X_1, X_2)$ con $X_1∈ℝ^p$ y $X_2∈ℝ^{p-q}$.

\begin{gather*}
µ = (µ_1, µ_2)\\
Σ = \left(\begin{array}{c|c} Σ_{11} & Σ_{12} \\\hline Σ_{21} & Σ_{22}
\end{array}\right)
\end{gather*}
\label{form::EspVarCondicionada}


\textbf{entonces: }  $X_2 | X_1 \sim N_{p-q}\left(µ_{2.1},Σ_{2.1}\right)$, donde

\begin{equation}
µ_{2.1} = µ_2 + Σ_{21}Σ_{11}^{-1}(X_1 - µ_1) = \esp{X_2|X_1}
\end{equation}

\begin{equation}
	Σ_{2.1} = Σ_{22} - Σ_{21}Σ_{11}^{-1} Σ_{12} = \var{X_2 | X_1}
\end{equation}

\end{prop}

\begin{proof}
Definimos $X_{2.1} = X_2 - Σ_{21}Σ_{11}^{-1}X_1$.

\[
\begin{pmatrix}
X_1\\
X_{2.1}
\end{pmatrix} =
\begin{pmatrix}
I &| &0\\
\hline
- Σ_{21}Σ_{11}^{-1}  &| &I
 \end{pmatrix}
\]

Como es una combinación lineal de $(X_1,X_2)'$, entonces $X_{2.1}$ es normal multivariante.

Vamos a calcular la media y la matriz de covarianzas de $X_{2.1}$

$X_{2-1} = N\left( µ_2-Σ_{21}Σ_{11}^{-1}µ_1 , \begin{pmatrix} Σ_{11} &|&0\\\hline 0&|&Σ_{2.1} \end{pmatrix} \right)$

Donde las covarianzas se calculan: $AΣA'$, siendo $A$ la matriz de la combinación lineal, es decir:

\[
A=\begin{pmatrix}
I &| &0\\
\hline
- Σ_{21}Σ_{11}^{-1}  &| &I
 \end{pmatrix}
\]



\paragraph{Conclusiones:}

\begin{itemize}
	\item $X_1$ es independiende de $X_{2.1}$
	\item $X_{2.1}$ es normal, con media y varianza calculadas anteriormente.
	\subitem $X_{2.1}|X_1$, al ser independientes, también se distribuye normalmente, con los mismos parámetros.
	\item Dado $X_1$, los vectores $X_{2.1}$ y $X_2$  difieren en el vector constante $Σ_{21}Σ_{11}^{-1}X_1 \implies X_2|X_1 = N\left( µ_{2.1}, Σ_{2.1} \right)$
\end{itemize}

\end{proof}

\begin{example}
Vamos a considerar $X_1, X_2$ como escalares, para entender la proposición. Este ejemplo le surgió a un investigador que quería predecir la estatura de los hijos en función de la de los padres (que no padres y madres, sólo padres).


\[
\begin{pmatrix}
X\\Y
\end{pmatrix} \equiv N_2\left( \begin{pmatrix} µ_x \\ µ_y \end{pmatrix}, \begin{pmatrix}
σ_x^2&σ_{xy}\\σ_{xy}&σ_y^2
\end{pmatrix} \right)
\]
Definimos \[\gor{Y} = \esp{Y|X} = µ_y + \frac{σ_{xy}}{σ_x^2}(x-µ_x)\]. La esperanza de la altura del hijo condicionada a la altura del padre será la media de las alturas de los hijos corregida por un factor en el que influye la diferencia de altura del padre con respecto a su media. Es de esperar que si Yao Ming tiene un hijo, sea más alto que la media.

El factor de corrección $\frac{σ_{xy}}{σ_x^2}$ es importante y no me he enerado bien de dónde sale.

Ahora vamos a calcular $\var{Y|X} = σ_{y}^2 - \frac{σ_{xy}^2}{σ_x^2} = σ_y^2 \left( 1- \rho^2\right)$ donde $\rho = \frac{σ_{xy}^2}{σ_x^2σ_y^2}$, el coeficiente de correlación.

Ha dicho algo así como \textit{La única relación que puede existir entre 2 variables normales es una relación lineal.}


Este coeficiente de correlación aparece también en la expresión de la esperanza. Vamos a verlo:

 \[\gor{Y} = µ_y + \frac{σ_{xy}}{σ_x^2}(x-µ_x) \dimplies \frac{\gor{Y}-µ_y}{σ_y} = \frac{σ_{xy}}{σ_xσ_y}\frac{x-µ_x}{σ_x}\]

 Es decir:

 \[
\frac{\gor{Y}-µ_y}{σ_y} = \rho \frac{x-µ_x}{σ_x}
 \]

Aplicado a la estatura de los hijos respecto de los padres, se interpreta como: ``Si un padre es muy alto, su hijo será alto pero no destacará tanto como el padre''. Este fenómeno lo definió como \concept{Regresión a la mediocridad}.

\end{example}

\begin{defn}[Homocedásticidad]\label{defn::Homocedasticidad}
Sea $X=(X_1 ,X_2)$ con $X_1∈ℝ^p$ y $X_2∈ℝ^{p-q}$. Entonces son vectores \textbf{homocedásticos} $\dimplies Σ_{2.1}$ es constante.

Ya veremos más adelante este concepto con mayor detalle.
\end{defn}


\begin{example}

Ahora vamos a ver un par de ejemplos numéricos:

Sea \[\begin{pmatrix}X,Y\end{pmatrix} \equiv N_2 \left( \begin{pmatrix}0,0\end{pmatrix}, \begin{pmatrix}10&3\\3&1\end{pmatrix} \right)\]

\paragraph{Distribución $Y|X$:}

Utilizando las fórmulas definidas en \ref{form::EspVarCondicionada} para $X_i$ unidimensionales:

\[\esp{Y|X} =  µ_{2-1} = 0 + 3·\frac{1}{10}(X-0) = \frac{3}{10}x\]
\[\var{Y|X} = Σ_{2.1} = 1-\frac{3}{10}·3 = \frac{1}{10}\]

\paragraph{Distribución $X|Y$:}

\[E(X|Y) = 3y\]
\[V(X|Y) = 1\]

Ambas son normales unidimensionales ya que $(X,Y)$ es normal multivariante.

\end{example}

\begin{example}
Sea \[\begin{pmatrix}X\\Y\end{pmatrix} \sim
N_2\left( \begin{pmatrix}1\\1\end{pmatrix}, \begin{pmatrix} 3&1\\1&2 \end{pmatrix}\right)\]

Queremos calcular la distribución de $(X+Y) | (X-Y) = 1$

Para ello, definimos 2 variables, $Z_1 = X+Y$ y $Z_2 = X-Y$, con lo que ahora tenemos que calcular $Z_2 | Z_1 = 1$

Lo primero es hallar la relación matricial entre $X,Y$ y $Z_i$

\begin{equation*}
	\begin{pmatrix}Z_1 \\ Z_2 \end{pmatrix} = \begin{pmatrix}X+Y\\X-Y\end{pmatrix} = \begin{pmatrix} 1 & 1\\1&-1 \end{pmatrix}\begin{pmatrix}X\\Y\end{pmatrix}
\end{equation*}

¿Cuáles son la esperanza y la matriz de covarianzas de el vector aleatorio $(Z_1,Z_2)$? Para ello necesitamos la matriz de la combinación lineal que ya tenemos:

\[
µ_Z = A·µ_{xy} = \begin{pmatrix}1&1\\1&-1\end{pmatrix} \begin{pmatrix}1\\1\end{pmatrix} = \begin{pmatrix} 2\\0 \end{pmatrix}
\]

\[
Σ_Z = AΣ_{xy}A' = \begin{pmatrix}1&1\\1&-1\end{pmatrix} \begin{pmatrix} 3&1\\1&2 \end{pmatrix}  \begin{pmatrix}1&1\\1&-1\end{pmatrix} = \begin{pmatrix}7&1\\1&3\end{pmatrix}
\]

Ahora ya podemos calcular la distribución como en el ejemplo anterior:

\begin{gather*}
	\esp{Z_1 | Z_2 = 1} = 2+1·\frac{1}{3}(1-0) = \frac{7}{3}\\
	\var{Z_1 | Z_2 = 1} = 7 - 1·\frac{1}{3} · 1 = \frac{20}{3}
\end{gather*}

En este caso, al ser homocedásticas (\ref{defn::Homocedasticidad}) entonces $\var{Z_1 | Z_2 = 1} = \var{Z_1 | Z_2 = n} ∀n∈ℕ$

\end{example}


\subsection{Formas cuadráticas bajo normalidad}

\begin{prop}[]
Sea $B$ una matriz simétrica e idempontente, $$Y\sim N_2(µ,σ^2I_n)$$ y \[µ'Bµ = 0 \text{  y  } p = Rg(B)\]

\textbf{Entonces: } \[\frac{Y'BY}{σ^2} \equiv \chi_p^2 \]
\end{prop}

\obs
\begin{itemize}
	\item La única matriz idempontente de rango completo es $I_n$
	\item $λ = 0,1 ∀λ$ autovalor de $B$.
	\begin{proof}
\[\left.\begin{array}{c} Bu = λu\\ Bu=B^2u = λBu = λ^2u \end{array}\right\} λu = λ^2u \implies λ=0,1\]
	\end{proof}

	\subitem Este último hecho permite calcular los grados de libertad de la distribución más fácilmente, ya que $p = Rg(B) = tr(B) = \#\{i \tq λ_i = 1\}$
\end{itemize}

\begin{prop}[Formas cuadráticas bajo normalidad]
\label{prop:tema1_pepino}
Sea $\vec{Y}\equiv = N_n (µ,σ^2I_n)$ y sean $A_{p\times n},B_{q\times n},C_{n\times n},D_{n\times n}$ con $C,D$ simétricas.

\textbf{Entonces:}
\begin{itemize}
	\item $AY$ y $BY$ son independientes $\dimplies$ $AB' = 0$
	\item $Y'CY$ e $Y'DY$ son independientes $\dimplies CD = 0$
	\item $AY$ e $Y'CY$ son independientes $\dimplies AC=0$
\end{itemize}

\end{prop}

\begin{lemma}[Lema de Fisher]
Sean $Y_1,..,Y_n \overset{iid}{\sim} N(µ,σ)$. Vamos a considerar el vector cuyas marginales son estas $Y \equiv (Y_1,...,Y_n) = N(µ1_n,σ^2I_n)$\footnote{$1_n = (1,1,...,1)$ n veces.}

\textbf{Entonces: } $\gor{Y}, S^2 = \frac{\sum (Y_i - \gor{Y})^2}{n-1}$ son independientes. Además,
\[\frac{(n-1)S^2}{σ^2} \equiv \chi^2_{n-1}\]

\end{lemma}

\begin{proof}
Lamentándolo mucho, la prueba será ignorada por el momento.
\end{proof}


\begin{theorem}[TCL Multivariante]
Sean $X_1,...,X_n$ vectores aleatorias independientes e idénticamente distribuidas (vec.a.i.i.d.) con $X_i \sim N(µ,Σ)$, con $Σ$ definida positiva.

\textbf{Entonces:}

\[\sqrt{n} Σ^{\frac{-1}{2}} (\gor{X_n} - µ) \convs[d] N(0,I) \dimplies \sqrt{n}(\gor{X_n}-µ) \convs[d] N_p(0,Σ)\]

La velocidad a la que $\gor{X_n}$ converge a $µ$ es del orden de $\frac{1}{\sqrt{n}}$

\end{theorem}

\begin{proof}
Lamentándolo mucho, la prueba será ignorada por el momento.
\end{proof}

% -*- root: ../EstadisticaII.tex -*-
\section{Contrastes no paramétricos}

Hipótesis no paramétrica: hipótesis que no se formula en términos de un número finito de parámetros.

\begin{enumerate}
\item Bondad de ajuste: A partir de una muestra $X_1,...,X_n \equiv F$ de variables aleatorias independientes idénticamente distribuidas, contrastar:
\begin{itemize}
\item $H_0: F=F_0$ donde $F_0$ es una distribución prefijada.
\item $H_0: F \in \{F_{\theta} : {\theta}\in H\}$, donde H es el espacio paramétrico.
\end{itemize}
\item Homogeneidad: Dados $X_1,...,X_n \equiv F$ y $Y_1,...,Y_n \equiv G$ de variables aleatorias independientes idénticamente distribuidas. Contrastar $H_0: F=G$.
\item Hipótesis de independencia: Dada $(X_1,Y_1),...,(X_n,Y_n) \equiv F$ de variables aleatorias independientes idénticamente distribuidas. Contrastar $H_0: X$ e $Y$ son independientes.
\end{enumerate}

\subsection{Contraste $\chi^2$ de bondad de ajuste}
Consideramos una distribución totalmente especificada bajo $H_0: X_1,...,X_n \equiv F$ de variables i.i.d.

$H_0: F=F_0$ es la hipótesis nula y queremos ver que F, que es la distribución obtenida con los datos verdaderos (obtenidos empíricamente) es igual a $F_0$ que es la distribución teórica.

Vamos a definir los pasos que tenemos que seguir para comprobar si $H_0$ es cierta:
\begin{enumerate}
\item Se definen k clases $A_1,...,A_k$. En el caso del dado, los valores de cada cara.
\item Se calculan las frecuencias observadas de datos en cada clase.
\subitem \[O_i = #\{j\tq X_j ∈ A_i\}\]
\[O_i \sim Bin\left(n,p_i = p_{H_0}(A_i)\right) \text{ es una variable aleatoria}\]
\item Se calculan las frecuencias esperadas si $H_0$ fuera cierta:
\subitem\[\esp_i = \esp_{H_0} = n·p_i \text{ al ser } O_i \text{ una binomial}\]
\item Se comparan $O_i$ con $E_i$ mediante el \concpet{estadístico de Pearson}, para comprobar si lo observado se parece a lo esperado. 

\[T = \sum_{i=1}^k \frac{(O_i - E_i)^2}{E_i}\]

\subitem Más adelante justificaremos porqué este estadístico es el utilizado. Además, el estadístico puede calcularse de otra manera:

\[
T = \sum \frac{O_i^2}{E_i} - n
\]
\item Se rechaza $H_0$ en la región crítica $R = \{ T>c\}$, donde $c$ depende del nivel de significación $α$.
\subitem $c$ se obtiene consultando en las tablas para $α = P_{H_0}(T>c)$
\end{enumerate}

Pero se nos presenta el siguiente problema, ¿Cuál es la distribución de $T$ bajo $H_0$?

\begin{theorem}[Distribución del estadístico de Pearson]
Bajo $H_0$

\[
T = \sum_{i=1}^k \frac{(O_i - E_i)^2}{E_i} \convs[d] \chi^2_{k-1}
\]
\end{theorem}

\begin{proof}
\textcolor{red}{Para otro momento}
\end{proof}

\begin{example}
Tiramos un dado 100 veces y obtenemos:

\begin{tabular}{|c|c|c|c|c|c|c|}
\hline
Resultados & 1 & 2 & 3 & 4 & 5 & 6 \\
\hline
Frecuencia & 10 & 20 & 20 & 10 & 15 & 25\\
\hline
\end{tabular}

Y consideramos $H_0: p_i=\frac{1}{6} \; \forall i=1,...,6$. Es decir que el dado no está trucado y cada cara tiene la misma probabilidad ($p_i$) de salir.

¿Es cierta la hipótesis, con un nivel de confianza/significación del 95\%?


Las clases son cada uno de los posibles resultados y las frecuencias observadas se encuentran en la tabla.

Vamos a calcular el estadístico $T$ \[T = \sum_{i=1}^6 \frac{O_i^2}{E_i} - n = \frac{6}{100} (\sum O_1^2) - 100 = ... = 11\]

Ahora, consultando las tablas buscamos el valor $\chi^2_{5;0.05} = 11.07 > T = 11$, entonces no estamos en la región crítica, por lo que no podemos rechazar la hipótesis.

Al ser valores muy próximos, observamos que el p-valor\footnote{el menor nivel de confianza para poder rechazar} del contraste tendrá que ser algo mayor que $0.05$.

\end{example}







\subsubsection{Hipótesis nula compuesta}

Vamos a estudiar el siguiente problema. Sea $X_1,...,X__n \overset{}{\sim}{iid}  F$ y una hipótesis compuesta: $H_0: F\in \{ F_{\theta}, \theta\in Ω \subset ℝ^n\}$. Esta hipótesis compuesta puede ser ``los datos se distribuyen normalmente, con media y varianza desconocidas''

Los pasos a seguir son: 
\begin{enumerate}
	\item Definir las clases $A_1,...,A_k$
	\item Calcular las frecuencias observadas $O_1,...,O_n$
	\item Estimamos $\thete$ por el método de máxima verosimilitud. Sea $\gor{\theta}$ el e.m.v.
	\subitem Pero para calcular las frecuencias esperadas, no tenemos una única normal. La idea intuitiva sería: hay unos parámetros que son los que mejor ajustan la distribución. ¿Cuál es la que mejor ajusta? La que tenga los parámetros estimados.
	\item Ahora ya podemos calcular las frecuencias esperadas:
	
	$E_i = n\gor{p_i}$ donde $\gor{p_i} = P_{\theta}(A_i)$, con $i=1,...,k$
	\item Ya podemos calcular el estadístico de Pearson:

	\[
	T= \sum_{i=1}^k \frac{(O_i - \gor{E_i})^2}{\gor{E_i}}
	\]

	\subitem ¿Qué distribución tiene este estadístico? Antes hemos visto que es una $\chi^2_{k-1}$ cuando se dan unas ciertas condiciones.

	En este caso, es de esperar que $T$ tienda a tomar valores menores que en el caso simple.

	Además, al estimar $r$ parámetros (las $r$ componenetes del vector $\theta$)) se introducen $r$ nuevas restricciones sobre el vector $(O_1,...,O_r)$.

Se puede probar, \footnote{bajo ciertas condiciones de regularidad que las distribuciones que conocemos cumplen y que son demasiado complicadas de enunciar} que:

\[
	\sum_{i=1}^k \frac{(O_i - \gor{E_i})^2}{\gor{E_i}} \convs[d] \chi^2_{k-1-r}
\]

	\item Se rechaza $H_0$ en  la región crítica \[R = \{ T > \chi^2_{k-1-r}\}\]
\end{enumerate}

\begin{example} 
\paragraph{Los bombardeos de Londres}

Los alemanes bombardeaban mucho a Londres durante la guerra mundial, y los ingleses quérían saber si los alemanes podían dirigir los misiles, o los impactos eran aleatorios.

Para ello, alguien hizo el estudio estadístico, para contrastar la hipótesis ``los impactos son aleatorios''.

Los impactos deberían seguir una poisson \footnote{ya que es el límite de una binomial, en la que consideramos los impactos como éxitos}. Para ello, dividió Londres en $n=576$ cuadrados, cada uno de ellos será la variable $X_i$ que debería seguir una Poisson.

La idea del contraste es: la Poisson que más se puede parecer es la que tenga de media el e.m.v. Si esa Poisson no se parece, entonces ninguna Poisson se puede parecer.



\begin{enumerate}
	\item[Clases] Los valores que toma la Poisson (recordamos que son número naturales). En este caso sólo se han definido 5, ya que la última es $>4$ \footnote{Se recomienda no definir más de 5 clases, para que la estimación no pierda demasiada información. }
	\item[Obs] $O_i = \{#j : X_j=i\}$, por ejemplo, las frecuencias observadas de la clase $0$, es decir, $O_0 = 229$, donde ese número es el número de las $n$ regiones de Londres en donde no cayó ningún misil.
	\item[e.m.v.] El e.m.v. de una Poisson es la media muestral, con lo que $\hat{λ} = \gor{x} = 0.9323$
	\item[Esp] Calculamos las frecuencias esperadas utilizando el parámetro estimado $$\hat{E_i} = n\hat{p_i} = 576 · e^{-0.9323}\frac{(0.9323)^i}{i!}$$
	\item[T] El estadístico $\chi^2$ de Pearson es: \[T = \sum_{i=1}^k \frac{(O_i - \gor{E_i})^2}{\gor{E_i}} = 1.01 \]

	Desde esta información, ya podemos hacernos una idea de si vamos a poder rechazar. ¿Por que? Al estimar un único parámetro $T \overset{iid}{\sim} \chi^{2}_{5-1-1}$, y además $\mathbb{E}(\chi^2_k) = k$, con lo que  debería habernos salido $T=3$. Al ser un vector bastante normal, podemos ver que tiene muy poca pinta de que vayamos a poder rechazar la hipótesis nula. Al $T$ estar por debajo de $3$ no estaremos en la región crítica.

\end{enumerate}


\obs Este ejemplo se encuentra también en las transparencias, donde podemos ver los valores y algunas gráficas explicativas. 
\end{example}


\subsection{Contrastes con $R$}
\paragraph{Hipótesis simple}
Con el siguiente código de $R$, podemos hacer el contraste de bondad de ajuste de una $\chi^2$ fácilmente. 

\begin{lstlisting}[style=mystyle]
obs = c(10,20,20,10,15,25)
ls.str(chisq.test(obs))
\end{lstlisting}

Si a \textit{chisq.test} no le damos más argumentos, supondrá hipótesis simple con equiprobabilidad de $p$. Podríamos darle otro argumento, y hacer lo siguiente para el ejemplo de los misiles:

\begin{lstlisting}[style=mystyle]
res = c(seq(0,4),7)
obs = c(229,211,93,35,7,1)
n = sum(obs)
lambda = sum(res*obs)/n
prob = dpois(res,lambda)
esp = n*prob
# Se agrupan las dos ultimas clases:
obs = c(obs[1:4],sum(obs[5:6]))
prob = c(p[1:4],1-sum(p[1:4]))
esp = c(esp[1:4],n-sum(esp[1:4]))
# Codigo para el grafico de barras:
matriz = rbind(p,obs/n)
rownames(matriz) = c('Frecuencias','Poisson')
barplot(matriz,beside=TRUE,names.arg=c(0:4),legend.text=TRUE,
col=c('lightgreen','orange'))
# Test chi 2
t = chisq.test(obs,p=prob)$statistic
pvalor = 1 - pchisq(t,3)
\end{lstlisting}


% -*- root: ../EstadisticaII.tex -*-

\chapter{Regresión}
El objetivo de la regresión es predecir una/s variable/s en función de la/s otra/s.


\section{Regresión lineal}

Observamos dos variables, X e Y , el objetivo es analizar la relación existente entre ambas, de forma que podamos predecir o aproximar el valor de la variable Y a partir del valor de la variable X.

\begin{itemize}
\item La variable Y se llama variable respuesta.
\item La variable X se llama variable regresora o explicativa.
\end{itemize}

Por ejemplo:
\begin{center}
\includegraphics[scale=0.5]{img/RentaVsFracaso.png}
\end{center}

Queremos predecir el fracaso escolar en función de la renta. La variable respuesta es el fracaso escolar, mientras que la variable regresora es la renta.

\subsection{Regresión lineal simple}

Frecuentemente existe una relación lineal entre las variables. En el caso del fracaso escolar,queremos construir una recta $Y_i = β_0 X_i + β_1\; i=1,...,n$ que minimice el error.

El problema es estimar los parámetros $β_0,β_1$. Una manera de hacer esto es:

\subsubsection{Recta de mínimos cuadrados}

\begin{defn}[Recta de mínimos cuadrados]
Estimando $β_i$ por $\hat{β_i}$ obtenemos: \[\hat{Y_i} = \hat{β}_0 + \hat{β}_1 x_i\]

La recta viene dada por los valores $\hat{β_0}, \hat{β_1}$ para los que se minimiza el error cuadrático, es decir:
\[\sum_{i=1}^n \left(Y_i - \hat{Y_i}\right)^2 =  \sum_{i=1}^n \left[ Y_i - (\hat{β_0} + \hat{β_1}x_i) \right]^2\]
\end{defn}

\begin{example}
\begin{center}
\includegraphics[scale = 0.6]{img/ejemploRectaRegresionLineal.png}
\end{center}
\end{example}

\paragraph{Cómo calcular la pendiente} de la recta de mínimos cuadrados.


Vamos a ver unas pocas maneras de calcular la recta de mínimos cuadrados.

\begin{itemize}

	\item El sistema habitual:

	\[ \hat{β_1} = \frac{\sum_{i=1}^n(x_i - \bar{x})(Y_i - \bar{Y})}{\sum_{i=1}^n (x_i - \bar{x})^2} = \frac{S_{xy}}{S_{xx}} \]
	Donde
		\[S_{xy} = \sum_{i=1}^n(x_i - \bar{x})(Y_i - \bar{Y}) \]
		\label{Ssubxx}
		Y, consecuentemente, como el avispado lector podrá imaginar
		\[S_{xx} = \sum_{i=1}^n (x_i - \bar{x})^2\]

		Es interesante darse cuenta que, siendo $S_x$ la cuasivarianza, tenemos $S_{xx} = (n-1)S_x$


	\subitem \[β_0 = \bar{Y} - \hat{β_1}\bar{x}\]

	\textbf{Entonces:}
	\[\text{recta} \equiv y - \bar{y} = \frac{S_{xy}}{S_{xx}}(x - \bar{x} ) \]

	\item Mínimos cuadrados como promedio de pendientes:
	\label{rmc::promediopendientes}
	\[
	\hat{β_1} = \frac{S_{xy}}{S_{xx}} = \sum_{i=1}^n \frac{(x_i - \bar{x})^2}{S_{xx}} \left( \frac{(Y_i - \bar{Y})}{x_i - \bar{x}} \right) = \sum_{i=1}^n ω_i \left( \frac{(Y_i - \bar{Y})}{x_i - \bar{x}} \right)
	\]

	Vemos que hemos ponderado la pendiente de cada recta que une cada punto con la media. Este peso es mayor cuanto mayor es la distancia \textbf{horizontal}.

	\item Mínimos cuadrados como promedio de respuestas:

	\[
	\hat{β_1} = \frac{\sum_{i=1}^n  (x_i - \bar{x}) (Y_i - \bar{Y})}{S_{xx}} \overset{(1)}{=} \sum_{i=1}^n \frac{x_i-\gor{x}}{S_{xx}} Y_i = \sum α_i Y_i
	\]

	$(1) \impliedby$ hemos utilizado una propiedad básica, importantísima y, a simple vista, poco (o nada) intuitiva:

	\begin{prop}
	Sea $\{x_i\},\{y_i\}$ datos de variables aleatorias.
	\[
		\sum (x_i - \gor{x}) (y_i -\gor{y}) = \sum(x_i -\gor{x})y_i = \sum (y_i - \gor{y})x_i
	\]

	\textbf{Importante:} sólo quitamos la media de una de las 2. No podemos hacer $\sum (x_i - \gor{x}) (y_i -\gor{y}) = \sum x_iy_i$, porque esto ya no es verdad.
	\end{prop}
	\begin{proof}
		\[\sum_{i=1}^n (x_i - \gor{x}) (y_i -\gor{y}) = \sum_{i=1}^n (x_i -\gor{x})y_i - \underbrace{\sum_{i=1}^n (x_i-\gor{x})\gor{y}}_{0}\]

		Vamos a ver por qué ese término es 0.
	\[\sum_{i=1}^n (x_i-\gor{x})\gor{y} \overset{(1)}{=} \left(\left(\sum_{i=1}^n x_i\right) - n\gor{x}\right)\frac{\sum y_i}{n}\]

	$(1)\to$ Estamos restando $n$ veces el  término $\gor{x}$ que no tiene índice del sumatorio, con lo que podemos sacarlo fuera.


	Aplicando la propiedad distributiva con el factor $\frac{1}{n}$, obtenemos:


	\[
		\left(\frac{\left(\sum x_i\right)}{n} - \frac{n\gor{x}}{n}\right)\sum_{i=1}^n y_i = (\gor{x} - \gor{x}) \sum y_i = 0
	\]

	\obs Pero... ¿y porqué $\sum(x_i -\gor{x})y_i ≠ 0$? ¿Cuál es el fallo de lo siguiente?

	\[
		\sum(x_i -\gor{x})y_i = \frac{\sum(x_i -\gor{x})y_i}{n}·n
	\]
	¿Y aplicamos el mismo razonamiento que antes?

	La respuesta es que, en este caso el factor $(x_i-\gor{x})$ está multiplicado por $y_i$ \textbf{dentro} del sumatorio, es decir:

	\[
	\sum_{i=1}^n (x_i - \gor{x}) (y_i -\gor{y}) = \sum_{i=1}^n \left[(x_i -\gor{x})y_i\right] - \sum_{i=1}^n \left[(x_i-\gor{x})\gor{y}\right] \]
	Y podemos sacar $\gor{y}$ del sumatorio, porque está multiplicando y no tiene índice del sumatorio.

	\[ \sum_{i=1}^n \left[(x_i -\gor{x})y_i\right] - \sum_{i=1}^n \left[(x_i-\gor{x})\right]\gor{y}
	\]
	\end{proof}
	
	\begin{prop}Propiedades de estos $α_i$

		\begin{enumerate}
			\item $\sum α_i = 0$
				\begin{proof}
					\[\sum α_i = \sum \frac{(x_i-\gor{x})}{S_{xx}} = 0 \impliedby \sum (x_i-\gor{x}) = 0\]
					\[\sum (x_i-\gor{x}) = \left(\sum x_i\right) - n\gor{x} = \left(\sum x_i\right) - n \frac{\sum x_i}{n} = 0\]
				\end{proof}
			\item $\sum α_ix_i = 1$
				\begin{proof} 
					\[ \sum α_ix_i = \sum\frac{(x_i-\gor{x})x_i}{S_{xx}} = \frac{1}{S_{xx}}\sum (x_i-\gor{x})(x_i-\gor{x}) = \frac{S_{xx}}{S_{xx}} = 1\]
				\end{proof}
			\item $\sum α_i^2 = \frac{1}{S_{xx}}$
				\begin{proof}
					\[\sum α_i^2 = \sum \frac{(x_i-\gor{x})(x_i-\gor{x})}{S_{xx}^2} = \sum \frac{(x_i-\gor{x})x_i}{S_{xx}^2} = \sum \frac{α_i x_i}{S_{xx}} = \frac{1}{S_{xx}} \sum α_ix_i \]
					Utilizando el anterior, tenemos $\sum α_i^2= \frac{1}{S_{xx}}$
				\end{proof}

		\end{enumerate}

	\end{prop}


\begin{defn}[Residuo]
En una recta de mínimos cuadrados: Sea $y_i = β_1x_i - β_0$ y sea $\hat{y}_i = \hat{β}_1x_i - \hat{β}_0$, llamamos residuo a $$e_i = y_i - \hat{y}_i$$

Los residuos cumplen:

\[
\sum_{i=1}^n e_i = 0
\]

Esto es intuitivo, ya que los errores se compensan y además es una buena propiedad.
\end{defn}



\begin{prop}
Sean $\{e_i\}$ una variable aleatoria que cumple \footnote{Se ha utilizado la $e$ porque es útil en cuanto a los residuos de la recta de mínimos cuadrados}:
\[\sum e_i = 0\]

Entonces:
\[\sum e_i x_i = 0 \implies \cov{e,x} = 0\]
\end{prop}

\begin{proof}
\[
\cov{e,x} = \esp{e}\esp{x} - \esp{e·x}
\]

Vamos a ver que los 2 sumandos son 0.

 $\esp{e}\esp{x} = 0 \impliedby \esp{e} \overset{?}{=} \gor{e} = 0$ 


Por otro lado:
\[ \sum (e_i - \vec{µ}) x_i = \sum (e_i - \vec{µ}) (x_i - \vx) \]


\[
\esp{e·x} = \sum e_ix_i = \sum e_ix_i - \vx \sum e_i = \sum e_i(x_i - \vx)
\]
\end{proof}


Esto tiene la siguiente explicación ``intuitiva'': La recta de mínimos cuadrados contiene toda la información lineal que $X$ puede dar sobre $Y$ (debido a que la covarianza entre los residuos y $X$ es 0).
\end{itemize}

\subsubsection{Fallos de la recta de mínimos cuadrados}

Vamos a ver un par de ejemplos ilustrativos:

\begin{example}[Sobre los datos atípicos]

Esta es una recta de mínimos cuadrados calculada para una nube de puntos a la que se ha añadido un punto atípico. Se ve una cierta tendencia de que la pendiente debería ser positiva, pero el dato atípico provoca un cambio brusco.
\begin{center}
%\includegraphics[scale=0.9]{img/rmc_atipico1.png}
\includegraphics[scale=0.9]{img/rmc_atipico2.png}
\end{center}

\end{example}

\begin{example}[Sobre la distancia horizontal]

¿Y da igual lo atípico que sea un dato? La respuesta es que no. Si el dato es muy atípico en la variable respuesta ($Y$), pero es muy \textit{típico} en la variable regresora, la recta no se desvía tanto. Vamos a verlo y después explicamos la razón.

Esta es la recta, en la que hemos ignorado los 3 datos que parecen ``atípicos''.
\begin{center}
\includegraphics[scale=0.9]{img/sobredistanciahorizontal.png}
\end{center}

Ahora calculamos las rectas teniendo en cuenta sólo uno de los puntos.

\begin{center}
\includegraphics[scale=0.4]{img/sobredistanciahorizontal1.png}
\includegraphics[scale=0.4]{img/sobredistanciahorizontal2.png}
\end{center}

Vemos que la recta azul no se desvía apenas de la original, mientras que la recta verde si se desvía un montón. ¿Esto a qué se debe? A que importa más la distancia horizontal de la media que la distancia vertical. Si vamos a la expresión de la recta de mínimos cuadrados como promedio de las pendientes \label{rmc::promediopendientes} vemos que hay un término $\frac{(x_i - \gor{x})}{S_{xx}}$ que hemos tomado como pesos para ponderar y en este caso, la distancia horizontal $(x_i - \gor{x})$ está multiplicando en el numerador.



\end{example}





\subsubsection{Introduciendo ``aleatoreidad'' para poder hacer IC}

Sea $\{ε_i\}$ siendo $ε_i \sim N(0,σ^2)$. Lo habitual es no saber cómo han sido generados los datos y es probable que no vayamos a conocer con exactitud absoluta la recta de mínimos cuadrados. Es por ello que suponemos el siguiente modelo para la variable respuesta:

\[
Y_i = β_1 x_i + β_0 + ε_i
\]


Tenemos que $\bar{y}_i \sim N$, ya que es una combinación lineal de variables normales \textbf{independientes} (como vimos en el Tema 1).


\begin{example}
Sea $σ=1, β_0 = 0$ y $β_1 = 1$.

Entonces el modelo es:

\[
Y_i = x_i + ε_i
\]

Fijamos $n=10$ y generamos las respuestas para $x_i = i$. Además, repetimos el experimento 6 veces y calculamos las rectas de mínimos cuadrados, obteniendo:

\begin{center}
\includegraphics[scale=0.6]{img/6ejemplosRegresion.png}
\end{center}

Vemos que obviamente las rectas no son las mismas. Esto se debe al $ε_i$ introducido. ¿Cuáles son los valores que toman $β_1$ y $β_0$? Habiendo repetido el experimento 1000 veces, obtenemos los siguientes histogramas:

\begin{center}
\includegraphics[scale=0.3]{img/1000vecesb0.png}
\includegraphics[scale=0.3]{img/1000vecesb1.png}
\end{center}

Vemos que no siempre es el mismo valor. Sabemos (por cómo hemos construido los datos) que $β_0 = 0$ y $β_1 = 1$, pero nuestra manera de calcularos (debido a $ε_i$) no siempre nos da el valor concreto.


\end{example}

El ejemplo anterior nos muestra que en realidad, estamos estimando $β_i$, aunque no nos guste y ahora tenemos que plantearnos ¿Cómo de buenos son nuestros estimadores? Tal vez son una mierda, o tal vez son insesgados.

Para ello, vemos que al haber añadido un error $\epsilon_i \sim N(0,σ^2)$, tenemos:

\[
Y_i = β_0 + β_1x + ε_i \implies Y_i \equiv N(β_0 + β_1X_i, σ^2)
\]


\subsubsection{Estimando $β_1$}

\begin{prop}
Nuestro estimador ``pendiente de la recta de mínimos cuadrados:'' $\hat{β_1}$  cumple

\[
\hat{β_1} \equiv N\left(β_1,\frac{σ^2}{S_{xx}} \right)
\]

\end{prop}

\begin{proof}
Él en clase lo ha hecho al revés. Muchos cálculos para llegar a la conclusión, pero aquí molamos más. En algún momento \textcolor{red}{revisará} alguien los apuntes y completará.

\begin{itemize}
	\item $\esp{\hat{β_1}} = β_1$
	\item $\var{\hat{β_1}} = ... = \displaystyle\frac{σ^2}{S_{xx}}$
\end{itemize}
\end{proof}

\subsubsection{Estimando $β_0$}

\begin{prop}
Nuestro estimador ``término independiente de la recta de mínimos cuadrados:'' $\hat{β_0}$  cumple

\[
\hat{β_0} = N\left(β_0 , σ^2 \left( \frac{1}{n} + \frac{\gor{x}^2}{S_{xx}}\right)  \right)
\]
\end{prop}

\begin{proof}
\begin{itemize}
	\item $\esp{\hat{β_0}} = β_0$
	\item
	$\var{\hat{β_0}} = \var{\gor{Y}} + \var{\hat{β_1}\gor{X}} - 2 \cov{\gor{Y},\hat{β_1}\gor{X}}$

 	\subitem Calculamos: $\cov{\gor{Y},\hat{β_1}\gor{X}}$ utilizando cosas del tema 1

 	\[
		\cov{\gor{Y},\hat{β_1}\gor{X}} = \cov{\frac{1'_n \gor{Y}}{n},α\gor{Y}} = \frac{1}{n}1'_nσ^2
 	\]
 	debido a que $α = 0$.

 	Ademas de ser incorrelados, son \textbf{independientes}. ¿Porqué? Porque conjuntamente son normales, es decir \[
 		\begin{pmatrix} \gor{Y} \\ \hat{β_1} \end{pmatrix} \equiv A\gor{Y} \equiv N_2
 	\]
\end{itemize}

\end{proof}


\textbf{Conclusiones:}
\begin{align*}
\gor{Y} &\text{ es independiente de } \hat{β_1}\\
\hat{β_1} &\equiv \left(β_1,\frac{σ^2}{S_{xx}}\right)\\
\hat{β_0} &\equiv \left(β_0,σ^2 \left( \frac{1}{n} + \frac{\gor{x}^2}{S_{xx}}\right)\right)
\end{align*}

¿Son estas las variables $\hat{β_1} $ y $\hat{β_0}$ normales una normal conjunta? Sí, \textbf{sí son una normal conjunta}. Una manera que tenemos de saber si es una normal conjunta es si son independientes, y en este caso no lo son.  Intuitivamente es fácil de ver. En una recta, si aumentamos la pendiente (y estamos en el primer cuadrante) entonces el término independiente disminuye. 

Esta dependencia tiene que aparecer. Vamos a estudiar la covarianza entre los estimadores:

\[
\cov{β_1,β_0} = \cov{\gor{Y} - \hat{β_1}\gor{x}, \hat{β_0}} = ... = -\gor{x}\frac{σ^2}{S_{xx}}
\]


Pero sabemos que sí son una normal bidimensional porque toda combinación lineal de nuestros parámetros de la recta es una variable aleatoria (la variable regresora $\hat{Y}$) normal.


\subsubsection{IC y Contrastes para $β_1$}
\label{subsubsec:ICparaB1}

Recordamos que \[ \hat{β}_1 \equiv N\left(β_1,\frac{σ^2}{S_{xx}}\right)\]

Podemos normalizar y buscar una cantidad pivotal (como hacíamos en estadística I)

\[
\frac{\hat{β_1} - β_1}{\frac{σ}{\sqrt{S_{xx}}}} \equiv N\left(0,1\right)
\]

Pero aquí nos encontramos con que necesitamos $σ$, la varianza de los errores. Esta varianza a menudo no es conocida (porque no sabemos con exactitud cuál es la recta verdadera) y tenemos que estimarla.

Para estimarla, parece razonable usar \[ \hat{σ} = S_R =\frac{\sum_{i=1}^n e_i^2}{n-2}\]

\begin{expla}
Recordamos que para que estimar la varianza, utilizamos (por el lema de Fisher) $n-1$ de denominador para que el estimador sea insesgado. Esto sale de que en la demostración, hay una matriz de rango $n-1$ ya que existe una restricción.

Siguiendo este razonamiento, en este caso tenemos 2 restricciones\footnote{$\sum e_i = 0$ y $\sum e_ix_i = 0$}, por lo que si lo demostráramos rigurosamente, aparecería una matriz de rango $n-2$ y por eso es el denominador. De esta manera, conseguimos un estimador insesgado.

\end{expla}

Además, $S_R$ se denomina \concept{Varianza\IS residual}

\begin{prop}
Una pequeña generalización del lema de Fisher:
\[
\frac{(n-2)S_{R}^2}{σ^2} \equiv \chi_{n-2}^2
\]

Además, es independiente de $\hat{β_1}$

\end{prop}



\begin{proof}
Esta proposición es un caso particular de un teorema que veremos más adelante.
\end{proof}


Llamamos 
\[ P_{R} = \frac{\hat{β_1}-β_1}{\frac{S_R}{\sqrt{S_{xx}}}}\]
\[ P_σ = \frac{\hat{β_1}-β_1}{\frac{σ}{\sqrt{S_{xx}}}}\]

Sabemos que $P_σ \sim N(0,1)$. Pero, al estimar ¿Se mantiene$P_R \sim N(0,1)$? 

Al estimar $σ$,  $P_{R}$ no tiene porqué ser exactamente $N(0,1)$. Si $n\to ∞$, entonces $S_R = σ$ y entonces $P_σ = P_R = N(0,1)$, pero para otros valores de $n≠∞$...

Por ello, nos vemos en la necesidad de manipular $P_R$ algebraicamente a ver si conocemos qué distribución tiene (que debería ser algo cercano a una normal, ya que estamos estimando $σ$ con un estimador insesgado. Tal vez las colas de la distribución son menos pesadas y podríamos esperar que fuera una $\mathcal{T}$)

\label{Cuentas:largas}

\[
\displaystyle\frac{\hat{β_1}-β_1}{\displaystyle\frac{S_R}{\sqrt{S_{xx}}}} = \displaystyle\frac{\hat{β_1}-β_1}{\displaystyle\frac{σ}{\sqrt{S_{xx}}}\frac{S_R}{σ}} = \left( \displaystyle\frac{\hat{β_1}-β_1}{\displaystyle\frac{σ}{\sqrt{S_{xx}}}} \right)\displaystyle\frac{1}{\displaystyle\frac{S_R}{σ}} = \displaystyle\frac{ \displaystyle\frac{\hat{β_1}-β_1}{\frac{σ}{\sqrt{S_{xx}}}} }{\displaystyle\frac{S_R}{σ}}
\]

En el numerador tenemos una $N(0,1)$ y en el denominador la raíz de una $\chi^2$ dividida por sus grados de libertad (por la proposición anterior). Esto es por definición una $\mathcal{T}$ (T-Student) con los mismos grados de libertad que la $\chi^2$, es decir $n-2$. (\href{https://en.wikipedia.org/wiki/Student%27s\_t-distribution#Characterization}{Wikipedia})


\begin{prop}
Podemos calcular el intervalo de confianza  para la pendiente de la recta, utilizando la fórmula de intervalo de confianza

\[
IC_{1-α}(β_1) \equiv \left[ \hat{β_1} \mp \mathcal{T}_{n-2,\frac{α}{2}}\frac{S_R}{\sqrt{S_{xx}}}\right] = \left[ \hat{β_1} \mp \mathcal{Z}\frac{σ}{\sqrt{S_{xx}}}\right] %\equiv \left[ \gor{Y} \mp \mathcal{T}_{n-1,\frac{α}{2}}\frac{S_R}{\sqrt{n}} \right]
\]
\end{prop}

\subsubsection{Contraste en R}

\label{example:R-output}
\begin{lstlisting}[style=mystyle]
# Ajusta el modelo
regresion = lm(Fracaso~Renta)
summary(regresion)

lm(formula = Fracaso ~ Renta)

Residuals:	Min		1Q			Median		3Q		Max
				-7.8717 -3.7421		0.5878	3.0368	11.5423
---
Coefficients:	Estimate Std.	Error 	t-value		Pr(>|t|)
(Intercept)		38.4944				3.6445	10.562		8.37e-10 ***
Renta 				-1.3467				0.2659	-5.065		5.14e-05 ***
---
Signif. codes: [...]
Residual standard error: 4.757 on 21 degrees of freedom
Multiple R-Squared: 0.5499,
Adjusted R-squared: 0.528
\end{lstlisting}


Aquí, la fila de \textit{intercept} es el término independiente y renta es la pendiente. Además, los p-valores son para el contraste $\hat{β_i} = 0$, dentro de la hipótesis $β_i \geq 0$. \footnote{Si queremos contrastar si es positivo, nos vamos al caso límite que lo separa y contrastamos eso}.

En este caso, el p-valor para $H_0: \hat{β_1}=$ es $5.14e-5$, con lo que no podemos rechazar la hipótesis.


\subsubsection{Predicciones}

Sea $(x_1,y_1),...,(x_n,y_n) \to y_i = β_0 + β_1x_i + ε_i$.

Dado una nueva observación $x_0$, tenemos 2 problemas para predecir:

\begin{itemize}
	\item \textbf{Inferencia sobre $m_0 \equiv \esp{y_0 | x_0} = β_0 + β_1x_0$}

	En este caso, $$\hat{m_0} = \hat{β_0} + \hat{β_1}x_0$$

	¿Cómo es este estimador?

	\[\esp{\hat{m_0}} = β_0 + β_1x_0 = m_0\]
	\[\var{\hat{m_0}} = ... = σ^2\left[\frac{1}{n} + \frac{(x_0-\bar{x})^2}{S_{xx}} \right] \]

	\subitem Intuitivamente, lo que significa el segundo sumando de la varianza es que ``cuanto más cerca esté $x_0$ de la media, mejor será la estimación''.

	\textbf{Conclusión:}

	\[
		\hat{m_0} \sim N\left( m_0, σ^2\left[\frac{1}{n} + \frac{(x_0-\bar{x})^2}{S_{xx}} \right]\right)
	\]



	\subitem \textbf{Intervalo de confianza} para $m_0$ utilizando la fórmula de intervalos de confianza:

	\[
IC_{1-α}(m_0) \equiv \left[ \hat{m_0} \pm \mathcal{T}_{n-2,\frac{α}{2}}S_R\sqrt{\frac{1}{n} + \frac{(x - \gor{x})^2}{S_{xx}}}\right]
\]

	\item \textbf{Predecir $Y_0$} usamos de nuevo:

	\[
\hat{Y_0} = \hat{β_0} + \hat{β_1}x \to Y_0 - Y \equiv N\left( 0, σ^2\left( 1 + \frac{1}{n}+  \frac{(x-\gor{x})^2}{S_{xx}}\right) \right)
	\]

	Donde la varianza ha sido calculada:

	\[
	\var{Y_0 - \hat{Y_0}} = \underbrace{\var{Y_0}}_{σ^2} - \var{\hat{Y_0}} + \underbrace{2 \cov{Y_0,\hat{Y_0}}}_{ = 0 \text{ (indep.) }} = σ^2 + σ^2\left( \frac{1}{n}+  \frac{(x-\gor{x})^2}{S_{xx}} \right)
	\]


	Este es un problema más complicado, ya que tenemos que tener en cuenta el término de error $ε_i$ y es por esto que aparece el 1 en la varianza. Tenemos que tener en cuenta la incertidumbre.

	Estandarizando y cambiando $σ$ por $S$, tenemos:

	\[
	\frac{Y_0 - \hat{Y_0}}{S_r \sqrt{1 + \frac{1}{n} + \frac{(x-\gor{x})^2}{S_{xx}}}} \equiv \mathcal{T}_{n-2}
	\]

	Ya que tenemos una normal estandarizada dividida por su .... que por definición, es una $\mathcal{T}$ de student.

	Ahora, vamos a construir el \concept{intervalo de predicción} (cambia ligeramente la interpretación)

	\[
1 - α = P\left\{ -\mathcal{T}_{n-2;\frac{α}{2}} < \frac{Y_0 - \hat{Y_0}}{...} < \mathcal{T}_{n-2;\frac{α}{2}}    \right\} = P \left\{ Y_0 \in \left[ \hat{Y_0} \pm \mathcal{T}_{n-2;\frac{α}{2}} S_R \sqrt{1+\frac{1}{n}+...} \right]  \right\}
	\]
\end{itemize}

Ahora vamos a hacer unos ejemplos numéricos.

\begin{example}Seguimos con el ejemplo de la renta.
\begin{problem}
\begin{center}
\begin{tabular}{c|c|c}
&media&desviación típica\\\hline
\% fracaso & 20.73 & 6.927\\
renta &13.19  & 3.814
\end{tabular}
La renta está expresada en miles de euros.
\end{center}


\ppart IC para $β_1$ de nivel $95\%$.
\ppart IC para \% de fracaso medio si la renta es de $14.000$ euros.


Es parte del enunciado la salida de ``R'' incluida en: \ref{example:R-output}

\solution
\spart

\[
IC_{1-α}(β_1) = \left[-1.3467 \mp \mathcal{T}_{21;0.025} · (0.2659)\right]
\]

Donde el $-1.3467$ es el estimador $\gor{β_1}$ que obtenemos de la salida de $R$. Lo mismo el $0.2659$, que es el error típico.

\spart
\[ \gor{Y_0} = 38.49 - (1.3467) · \underbrace{14}_{x_0} = 19.64\]

Siendo este el estimador, vamos a construir el intervalo de confianza. \footnote{Podría ser que nos pidieran el intervalo de predicción, pero en ese caso estarían pidiendo el intervalo de ...... para predecir. Además, nos están dando un $x_0$ que para predicción no lo tenemos}

\[
IC_{1-α}(m_0) = \left[19.64 \mp (2.06)(4.757)\sqrt{\frac{1}{23}+\frac{(14-13.19)^2}{S_{xx}}}\right]
\]
Donde $S_{xx} = 320.06$ y podemos calcularlo despejando de:

\[
E.T.(\gor{β_1}) = \sqrt{\frac{S_R^2}{S_{xx}}} \to \sqrt{S_{xx}} = \frac{4.757}{0.2659} \to S_{xx} = 320.06
\]
Donde $E.T.(\gor{β_1})$ es el error típico dado por \textit{R}. En este caso es $0.2659$y $S_R^2 = 4.757^2$

También podríamos utililzar $S_x = \frac{S_{xx}}{n-1}$, sabiendo que $S_x^2 = \frac{n}{n-1}σ^2$. Sabemos que $S_x = 3.814$ por ser la renta la variable explicativa, es decir, las $x$ de nuestro modelo de regresión.

\[
\frac{n}{n-1}\left(3.814\right)^2 = \frac{S_{xx}}{n-1} \to S_{xx} = 21·\left(3.814^2·\frac{22}{21}\right) = 320.03
\]
\end{problem}


\end{example}


\obs Todos estos cálculos y todas estas fórmulas se basan en muchas hipótesis (como que la distribución del error sigue una distribución normal). Pero podría ser que esto no ocurriera y estuviéramos suponiendo un modelo falso. Para ello, en estadística existe el \concept{Diagnóstico del modelo}. Este diagnóstico, consiste en comprobar si las hipótesis del modelo son \textbf{aceptables} para los datos disponibles. ¡Ojo! Aceptable... Puede haber muchos modelos aceptables para un mismo conjunto de datos.

Este diagnóstico se suele basar en el análisis de los residuos del modelo.

\begin{example}
	Vamos a ver a ojo unos cuantos ejemplos. Vamos a utilizar que $\corr{e,\gor{y}} = 0$ bajo el modelo (como calculamos anteriormente)

\begin{center}
\includegraphics[scale=0.5]{img/diagmodelo.png}
\end{center}

De estos 4 gráficos, el bueno es el primero, ya que los demás no cumplen alguno.
\end{example}

\begin{example}
Vamos a ver otro ejemplo, donde arriba están los datos y abajo los residuos. Mirando sólo la fila de arriba podríamos saber si nuestro modelo para la regresión se cumple o sino.


\begin{center}
\includegraphics[scale=0.5]{img/diagmodelo_2.png}
\end{center}

Vemos que el primero y el último si tienen este modelo como aceptable, ya que en los residuos no hay ningún patrón (y se cumple que la correlación es 0).

En el segundo, podríamos suponer que es bueno, pero al diagnosticar el modelo mirando los residuos, vemos que no. El diagnóstico del modelo \textbf{magnifica los errores}.

En el cuarto, vemos más claro que es heterocedástico y que no se cumple el modelo supuesto.
\end{example}

En regresión múltiple veremos que no podemos ver los datos, ya que son demasiadas variables, pero sí podemos estudiar los residuos como acabamos de hacer en los ejemplos anteriores.


\subsection{Regresión lineal múltiple}

El ejemplo que vamos a estudiar en regresión múltiple es el consumo de gasolina en EEUU intentando predecirlo a partir de unas cuantas variables. Las variables regresoras son:

\begin{center}
\begin{tabular}{cccccccc}
State&Drivers&FuelC&Income&Miles&MPC&Pop&Tax\\\hline
AL&3559897&2382507&23471&94440&12737.00&3451586&18.0\\
AK&472211&235400&30064&13628&7639.16&457728&8.0\\
AZ&3550367&2428430&25578&55245&9411.55&3907526&18.0
\end{tabular}
\end{center}


Estos son los datos que obtenemos de $R$:

\begin{lstlisting}[style=mystyle]
reg <- lm(FuelC ~ Drivers+Income+Miles+MPC+Tax, data=fuel2001)
summary(reg)
Call:
lm(formula = FuelC ~ Drivers + Income + Miles + MPC + Tax, data = fuel2001)
Coefficients:
Estimate	Std.Error	t	value	Pr(>|t|)
(Intercept)	-4.844e+05	8.102e+05	-0.598	0.552903
Drivers	6.144e-01	2.229e-02	27.560	<	2e-16
Income	7.526e+00	1.611e+01	0.467	0.642587
Miles	5.813e+00	1.587e+00	3.664	0.000652
MPC	4.643e+01	3.488e+01	1.331	0.189820
Tax	-2.114e+04	1.298e+04	-1.629	0.110298
---
Residual standard error: 394100 on 45 degrees of freedom
Multiple R-squared: 0.9808, Adjusted R-squared: 0.9787
F-statistic: 459.5 on 5 and 45 DF, p-value: < 2.2e-16
\end{lstlisting}

\subsubsection{Notación}


\begin{itemize}
	\item $n$ es el número de observaciones, en este caso, el número de estados.
	\item $k$ es el número de atributos.
	\item $ε_i \sim N(0,σ^2)$
	\item $n\geq k+2$: esta hípótesis  es muy necesaria.\footnote{En la estadística, habría que rehacer el modelo para cuando $k>n$. ¿Y cuándo $k>n$? ¿Cuándo puede ocurrir esto? Cada vez más hay más información para cada individuo. En estudios genéticos por ejemplo, que hay millones de genes pero no se pueden hacer el estudio con millones de personas... \textbf{LA MALDICIÓN DE LA DIMENSIONALIDAD} que decimos en Introducción previa a los Fundamentos Básicos del Aprendizaje Automático.\\ Una posible solución al problema es un algoritmo que filtre los atributos que son importantes.}
\end{itemize}

Regresión simple es un caso particular de múltiple, tomando $k=1$.

\subsubsection{Modelo}

Tenemos una muestra de $n$ observaciones de las variables $Y$ y $X_1,…,X_k$. Para la observación $i$, tenemos el vector $(Y_i,x_{i1},x_{i2},…,x_{ik})$.

El modelo de regresión lineal múltiple supone que:
\[Y_i=β_0+β_1x_{i1}+…+β_kx_{ik}+ε_i,\ i=1,...,n\]

donde las variables de error $ε_i$ verifican:
\begin{enumerate}
\item $ε_i$ tiene media cero, para todo $i$.
\item Var($ε_i$) = $σ^2$, para todo $i$ (homocedasticidad).
\item Son variables independientes.
\item Tienen distribución normal.
\item $n ≥ k + 2$ (hay más observaciones que parámetros).
\item Las variables $X_i$ son linealmente independientes entre sí (no hay colinealidad).
\end{enumerate}

Las 4 primeras hipótesis se pueden reexpresar así: las observaciones de la muestra son independientes entre sí con
\[Y_i \equiv N(β_0 +β_1x_{i1} +...+β_kx_{ik},σ),\ i=1,...,n\]

En forma matricial:

\[
	\begin{pmatrix}
		Y_1\\
		Y_2\\
		\vdots \\
		Y_n
	\end{pmatrix}
	=
	\begin{pmatrix}
		1 & x_{11} & … & x_{1k} \\
		1 & x_{21} & … & x_{2k} \\
		\vdots & \vdots &  & \vdots \\
		1 & x_{n1} & … & x_{nk}
	\end{pmatrix}
	\begin{pmatrix}
		β_1\\
		β_2\\
		\vdots \\
		β_n
	\end{pmatrix}
	+
	\begin{pmatrix}
		ε_1\\
		ε_2\\
		\vdots \\
		ε_n
	\end{pmatrix}
\]


De forma más compacta, el modelo equivale a:
\[Y =Xβ+ε,\ ε \equiv N_n(0,σ^2I_n) \iff Y \equiv N_n(Xβ,σ^2I_n)\]

\begin{defn}[Matriz de diseño]
	La matriz $X$ se conoce como matriz de diseño
\end{defn}

\subsubsection{Estimación de los parámetros del modelo}

La pregunta que lógicamente se nos viene a la cabeza en este momento es: ¿Cómo estimar $β$ a partir de $Y$ y $X$?. Para ello nos serviremos de la interpretación geométrica del modelo:

Sea $\mathcal{V} ⊂ R^n$ el subespacio vectorial generado por las columnas de la matriz de diseño $X$ ($\dim(\mathcal{V}) = k + 1$).
\[μ∈\mathcal{V} \iff ∃β∈R^{k+1} : μ=Xβ\]
El modelo equivale a suponer $Y \equiv N_n(μ, σ^2I_n)$, donde $μ ∈ \mathcal{V}$.

\newpage
\begin{figure}[hbtp]
	\centering
	\inputtikz{proyecta_V}
\end{figure}

Con esto, parece razonable estimar $µ$ mediante la proyección ortogonal de $Y$ sobre $\mathcal{V}$ para obtener $\hat{Y} = X\hat{β}$. Equivalentemente: $\norm{Y-X\hat{β}}^2 \leq \norm{Y-Xβ}^2, ∀β\in ℝ^{k+1}$

\begin{defn}[Estimador mínimos cuadrados]

	Al $\hat{β}$ que minimiza
	\[\norm{Y - Xβ}^2 = \sum_{i=1}^n (Y_i - β_0 - β_1x_{i1} - … - β_kx_{ik})^2\]
	se le denomina estimador de mínimos cuadrados.
\end{defn}

Veamos qué podemos sacar de lo visto hasta ahora para averiguar quién es exactamente $\hat{β}$:

Para que $\hat{Y}$ sea la proyección de $Y$ sobre $\mathcal{V}$ es necesario y suficiente que se satisfagan las ecuaciones normales:

\begin{defn}[Ecuaciones normales]
	Tomando $e = Y - \hat{Y}$ como el vector residuo, las ecuaciones normales se satisfacen si:
	\[X'(Y - \hat{Y})=0 \iff X'e = 0\]
\end{defn}

Que se satisfagan estas ecuaciones es equivalente a decir que la resta $Y - \hat{Y}$ da un vector perpendicular a la base $\mathcal{V}$. Sustituyendo $\hat{Y} = X\hat{β}$:

\[X'(Y - X\hat{β}) = 0 \iff X'Y = X'X\hat{β}\]

Ya que las filas de $X'$ (las columnas de $X$) son independientes, sabemos que $X'X$ tiene rango completo y por tanto es invertible. Y llegamos a la expresión para nuestro estimador de mínimos cuadrados $\hat{β}$:
\begin{equation}
	\boxed{\hat{β} = (X'X)^{-1}X'Y}
\end{equation}


\begin{obs}
	En regresión simple se tiene que:
	\[
		X\equiv
		\begin{pmatrix}
			1 & x_1\\
			\vdots & \vdots \\
			1 & x_n
		\end{pmatrix}
		\text{ y que: }
		X'X =
		\begin{pmatrix}
			n & \sum x_i \\
			\sum x_i & \sum x_i^2
		\end{pmatrix}
	\]
\end{obs}

Con lo visto hasta el momento sabemos que $\hat{Y} = X\hat{β} = X(X'X)^{-1}X'Y$, es decir, que nuestra $\hat{Y}$ está expresada como producto de $Y$ por una matriz que llamaremos:

\begin{defn}[Hat matrix]
	\[H = X(X'X)^{-1}X'\]
	se conoce como hat matrix, puesto que da $Y$ gorro: $\hat{Y} = HY$.
\end{defn}

La hat matrix $H$ tiene las siguientes \textbf{propiedades}:
\begin{itemize}
	\item Es matriz de proyección ortogonal sobre $\mathcal{V}$.
	\item Es simétrica e idempotente.
	\item Tiene rango $k+1$ (la dimensión del espacio $\mathcal{V}$ sobre el que proyecta).
\end{itemize}

\begin{obs}
	Podemos servirnos de la hat matrix para expresar el vector de residuos:
	\[e = Y - \hat{Y} = Y - HY = (I - H)Y\]
	Donde $(I-H)$ es una matriz simétrica e idempotente con rango $rg(I-H)=~n-~(k~+~1)$, que proyecta sobre el espacio ortogonal $\mathcal{V}^\perp$.
\end{obs}

Para acabar esta sección enumeramos algunas propiedades de los parámetros:
\begin{itemize}
\item $\hat{β}$ es el estimador de máxima verosimilitud (EMV) de $β$:
\[L(β,σ)= \left(\frac{1}{\sqrt{2π}σ}\right)^n exp\left\{-\frac{1}{2σ^2} \norm{Y−Xβ}^2 \right\}.\]

\item El EMV de $σ^2$ es:
\[\hat{σ}^2 = \frac{\norm{Y−Xβ}^2}{n} = \frac{\norm{e}^2}{n} = \frac{\sum_{i=1}^n e_i^2}{n}\]

\item El vector $\hat{β}$ tiene distribución $N_{k+1}(β, σ^2(X′X)^{−1})$ (en la siguiente sección se demuestra).
\end{itemize}

\subsubsection{Estudio de la varianza residual}
Un estimador insesgado de $σ^2$ es la varianza residual $S_R^2$ , es decir, la suma de los residuos al cuadrado, corregida por los grados de libertad apropiados (en este caso $n-\dim{(\mathcal{V})}$):
\[S_R^2 = \frac{\sum e_i^2}{n-(k+1)} = \frac{\norm{e}^2}{n-k-1} = \frac{\norm{Y - \hat{Y}}^2}{n-k-1}\]

Si nos fijamos en que:
\[\norm{Y - \hat{Y}}^2 = Y'(I-H)(I-H)Y = Y'(I-H)Y\]

y sabiendo que la matriz $I-H$ es simétrica e idempotente, si recordamos la proposición \ref{prop:tema1_pepino} y demostramos $μ(I-H)μ'=0$, podemos determinar que la distribución de $S_R^2$:

Dado que $μ∈\mathcal{V}$ y sabiendo que $μ=XB$:
\[(I-H)μ = (I-H)XB = 0\]
Ya que el vector $(I-H)$ proyecta sobre $\mathcal{V}^\perp$.

Así llegamos a que:
\[\frac{\norm{Y-\hat{Y}}^2}{σ^2} = \frac{Y'(I-H)Y}{σ^2}=\]
\begin{equation}
	\boxed{\frac{(n-k-1)S_R^2}{σ^2} \equiv χ_{n-k-1}^2}
\end{equation}


Además si tomamos esperanzas en ambos lados de la igualdad obtenemos:
\[\frac{(n-k-1)·\mathbb{E}(S_R^2)}{σ^2} = n-k-1\]
\begin{equation}
	\boxed{\mathbb{E}(S_R^2) = σ^2}
\end{equation}

Lo siguiente que haremos es mirar si $\hat{β}$ y $S_R^2$ \textbf{son independientes}. Esto es cierto dado que se cumple que $(X'X)^{-1}X' · (I-H)=0$:
\[(X'X)^{-1}X' · (I-H) = (X'X)^{-1}X' - (X'X)^{-1}X' · X(X'X)^{-1}X' =0\]

\begin{obs}
	El lema de Fisher \ref{lemma:fisher} no es más que el resultado de aplicar los resultados vistos en esta sección al caso particular del modelo:
	\[y_i = β_0 + ε_i \iff X=\begin{pmatrix}1\\ \vdots \\ 1\end{pmatrix}\]
	En este caso tenemos que $\dim{(V)}=1$
\end{obs}


\subsubsection{Distribución de $\hat{β}$ y contrastes de hipótesis individuales sobre los coeficientes $\hat{β}_i$}
\[\esp{\hat{β}} = (X'X)^{-1}X' \underbrace{Xβ}_{\esp{Y}} = β\]
\[\var{\hat{β}} = σ^2I_n · (X'X)^{-1}X' · X(X'X)^{-1} = σ^2(X'X)^{-1}\]

Como $\hat{β}$ es una combinación lineal de $Y$ (que sigue una distribución normal):

\[
\hat{β} \equiv N_{k+1}\left(β,σ^2(X'X)^{-1}\right)
\]

Y la regresión simple, es un caso particular de esta fórmula.

\textbf{Notação}: $q_{ij}\equiv$ entrada $i,j$ de la matriz $(X'X)^{-1}$

\paragraph{Consecuencias:}

\begin{itemize}
	\item ¿Cuál es la distribución marginal de $\hat{β_j}$ a partir de la que hemos visto de la conjunta? Como vimos en el tema 1, es también una normal, con el correspondiente valor del vector $β$ como media y el elemento $j,j$ de la diagonal.
	\[ \hat{β}_j \equiv N\left(β_j, σ^2q_{jj}\right)\]

	Ahora, podemos estandarizar:

	\[
	\frac{\hat{β_j}-β_j}{σ\sqrt{q_{jj}}} \equiv N(0,1)
	\]

	Y utilizando que $S_R$ es independiente de $σ$ y la definición de $t-$student tenemos:

	\[
	\frac{\hat{β_j}-β_j}{S_R\sqrt{q_{jj}}} \equiv \mathcal{T}_{n-k-1}
	\]

	¿Cuál es el intervalo de confianza?

	\[
		IC_{1-α}(β_j) \equiv \left[\hat{β_j}\mp \mathcal{T}_{n-k-1;\frac{α}{2}}\underbrace{S_R\sqrt{q_{jj}}}_{\text{Error típico de }β_j} \right]
	\]

	Y, como en regresión simple, estudiamos $H_0 : β_j = 0$:
	\[
		R = \left\{ \frac{\abs{\hat{β}_j}}{S_R\sqrt{q_{jj}}} > \mathcal{T}_{n-k-1;\frac{α}{2}} \right\}
	\]
\end{itemize}

En las traspas encontramos una salida de regresión múltiple de $R$. La columna \textit{estimate} es el vector $\hat{β}$, el p-valor es del contraste $H_0 : β_j = 0$.


\subsubsection{Combinaciones lineales de los coeficientes} Vamos a querer constrastar cosas del estilo ¿las 2 variables influyen igual?

Para ello, transformamos eso en $H_0: β_1 = β_2 \to H_0 : β_1 - β_2 = 0$, entonces tenemos una combinación lineal $a∈ℝ^{k+1}$, tal que $H_0:a'\hat{β} = 0$

Para poder hacer esto, lo primero ha sido estimar $\hat{β}$ y lo siguiente será saber su distribución.

\[a'\hat{β} = N\left(a'β,\underbrace{σ^2a'(X'X)^{-1}a}_{\left(E.T.(a'\hat{β})\right)^2}\right) \to \frac{a'\hat{β} - a'β}{S_R\sqrt{a'(X'X)^{-1}a}} \equiv \mathcal{T}_{n-k-1}\]

Y con esto, ya podemos construir el intervalo de confianza para una combinación lineal de los parámetros:

\[
IC_{1-α}(a'β) = \left[ a'\hat{β} \mp \mathcal{T}_{n-k-1;\frac{α}{2}}E.T.(a'\hat{β}) \right]
\]

La región de rechazo correspondiente es:

\[
R = \left\{ \frac{a'\hat{β}}{S_R\sqrt{σ^2a'(X'X)^{-1}a}} >  \mathcal{T}_{n-k-1;\frac{α}{2}} \right\} = \left\{ \frac{a'\hat{β}}{E.T.(\hat{β})} >  \mathcal{T}_{n-k-1;\frac{α}{2}} \right\}
\]

\paragraph{Ejercicio:} ¿Y si queremos hacer un contraste unilateral $a'β > 0$?

\subsubsection{Inferencia sobre subconjuntos de coeficientes}

Todos los contrastes hechos hasta ahora son muy fáciles porque se basan en una única normal. Nuestros contrastes han sido del tipo $β_i = 0$. En esta sección vamos a estudiar contrastes del tipo $H_0: β_1 = 0, β_3 = 0$. 

La primera aproximación puede ser calcular la región de confianza de $β_1$ y la de $β_3$ y tomar la intersección. Es decir:

\begin{center}
\includegraphics[scale=0.5]{img/confianzamultivariantemal.png}
\end{center}

 Pero al no ser independientes, no estamos teniendo en cuanto las correlaciones, la información que me da saber $β_1$ para saber $β_3$. 

\begin{center}
\includegraphics[scale=0.5]{img/confianzamultivariantebien.png}
\end{center}

\paragraph{Vamos a ello formalmente:}
\newcommand{\bpp}{β_{(p)}}
\newcommand{\hbpp}{\hat{β}_{(p)}}
\newcommand{\fpnk}{F_{p,n-k-1}}
Sea $β_{(p)}$ un conjunto de coeficientes de $β$. Sea $\hbpp$ el correspondiente subconjunto de estimadores.

Sea $Q_q$ la submatriz de $(X'X)^{-1}$ cuyas filas y columnas corresponden a las coordenadas del vector $\bpp$.

\[\hbpp \equiv N_p \left( \bpp, σ^2Q_p \right)\]

Si este es nuestro estimador, vamos a estandarizarlo (utilizando propiedades del tema 1).

\[
\frac{(\hbpp - \bpp)'Q_p^{-1}(\hbpp - \bpp)}{σ^2}\equiv \chi^2_p
\]

Tenemos el problema de que no conocemos $σ$ y tenemos que estimarlo. Para ello, vamos a seguir un proceso parecido a \ref{Cuentas:largas}. Para ello necesitamos:

\begin{defn}[Distribución $F_{n,m}$]

\[
F_{n,m} \equiv \frac{\chi^2_m / m}{\chi^2_n / n}
\]

Es muy habitual que aparezca la $F$ al comparar varianzas. 
\end{defn}

Sabiendo lo que es una $F$, vamos a estudiar qué ocurre al cambiar $σ$ por $S_R$

\[
\frac{(\hbpp - \bpp)'Q_p^{-1}(\hbpp - \bpp)}{S_R^2} = \frac{\frac{(\hbpp - \bpp)'Q_p^{-1}(\hbpp - \bpp)}{σ^2}}{\frac{S_R^2}{σ^2}}
\]

En el numerador tenemos una $\chi^2_p$, pero nos faltaría dividir por los grados de libertad para tener una $F$, entonces:

\[
\frac{\frac{(\hbpp - \bpp)'Q_p^{-1}(\hbpp - \bpp)}{pσ^2}}{\frac{S_R^2}{σ^2}} = \frac{\chi^2_p/p}{\chi^2_{n-k-1}/n-k-1} \equiv F_{p,n-k-1}
\]

\paragraph{Conclusión}

%TODO: esto está bien seguro
\[
\frac{(\hbpp - \bpp)'Q_p^{-1}(\hbpp - \bpp)}{pS_R^2} \equiv F_{p,n-k-1}
\]
\[
1-α = P\left( (\hbpp - \bpp)'\left(S_R^2Q_p\right)^{-1}(\hbpp - \bpp) \leq p F_{p,n-k-1}\right)
\]

Esto nos da un elipsoide de confianza, es decir, sabemos que caerá en la región del elipsoide con probabilidad $1-α$:

\begin{figure}[hbtp]
	\centering
	\inputtikz{elipsoide_confianza}
\end{figure}

\begin{example}
Vamos a querer contrastar si son iguales los coeficientes $β_2,β_3$ las variables ``Income'' y ``Miles''. La hipótesis es: $H_0 : β_2=β_3$ a nivel $α=0.05$

Aquí damos los datos necesarios para completar (en la realidad, los obtendremos de $R$:
\[
S_R^2Q_{(2)} = \begin{pmatrix} 259.45&7.89\\7.89&2.52 \end{pmatrix}
\]

Vamos a proceder al contraste. Construimos el estadístico para construir la región de rechazo:

\[
t = \frac{|\hat{β_2} - \hat{β_3}|}{E.T.(\hat{β_2}-\hat{β_3})} = \frac{1.725}{15.687} \not \ge t_{45;0.025}
\]

Por tanto no podemos rechazar la hipótesis nula de que $β_2=β_3$.

El error típico se ha calculado es:
\[\sqrt{\var{\hat{β_2}-\hat{β_3}}} = \sqrt{(1,-1) S_R^2Q_p (1,-1)} = 15.687\]
Y los 45 grados de libertad los obtenemos de $R$.

\end{example}


\subsection{Estimadores de mínimos cuadrados}
\subsection{Inferencia sobre los parámetros del modelo}
\subsection{Análisis de la varianza}
\subsection{Contrastes de hipótesis lineales}
\subsection{Modelo unifactorial}




\appendix
\chapter{Ejercicios}
% -*- root: ../EstadisticaII.tex -*-
\section{Hoja 1}

\begin{problem}[1]
Sea $Y = (Y_1,Y_2,Y_3)' ≡ N_3(µ,Σ)$, donde \[µ = (0,0,0)'\;
Σ =\begin{pmatrix}
1&0&0\\
0&2&−1\\
0&−1&2
\end{pmatrix}
\]


\ppart  Calcula la distribución del vector $(X_1,X_2)$, donde $X_1 = Y_1 + Y_3$ y $X_2 = Y_2 + Y_3$.
\ppart ¿Existe alguna combinación lineal de las variables aleatorias $Y_i$ que sea independiente de $X_1$?

\solution

\spart 
\[
\begin{pmatrix}X_1 \\ X_2 \end{pmatrix} = \begin{pmatrix} Y_1 + Y_3 \\ Y_2 + Y_3 \end{pmatrix} = \begin{pmatrix} 1&0&1\\0&1&1 \end{pmatrix} \begin{pmatrix} Y_1\\Y_2\\Y_3 \end{pmatrix} \equiv N_1\left( \begin{pmatrix}0\\0 \end{pmatrix},\begin{pmatrix}3&1\\1&2\end{pmatrix} \right)
\]

\spart 
\[
\begin{pmatrix} Ay\\By \end{pmatrix} = \begin{pmatrix} A\\B \end{pmatrix} Y \equiv N_{q+r} \left( \begin{pmatrix} Aμ\\Bμ \end{pmatrix},\begin{pmatrix} A\\B \end{pmatrix} Σ(A',B') \right)
\]

Entonces \[cov\left(a'y,(1,0,1)y\right) = (a_1,a_2,a_3) \begin{pmatrix} 1&0&0\\0&2&-1\\0&-1&2\end{pmatrix} \begin{pmatrix} 1&0&1 \end{pmatrix}\]

\end{problem}


\begin{problem}[2]

\solution

\end{problem}

\begin{problem}[3]


\solution

\end{problem}

\begin{problem}[5]

Calcula la distribución condicionada de $X$ dado $Y$ = $y$, y la de $Y$ dado $X$ = $x$.

\solution


\[
\begin{pmatrix}X\\Y \end{pmatrix} \equiv N_2\left(\begin{pmatrix}0\\0\end{pmatrix},\begin{pmatrix}1&-1\\-1&2\end{pmatrix}^{-1}\right)
\]

Aplicando las fórmulas vistas en teoría \ref{form::EspVarCondicionada}

\[
E(X|Y=y) = μ_y + Σ_{21}Σ_{11}^{-1}(X-μ_x) = 0 + \frac{1}{1}(y-0) = y
\]
\[
E(Y|X=x) = μ_x + Σ_{21}Σ_{11}^{-1}(Y-μ_y) = 0 + \frac{1}{2}(x-0) = \frac{x}{2}
\]

\end{problem}

\begin{problem}[7]
Sea $X = (X1,X2,X3)'$ un vector aleatorio con distribución normal tridimensional con vector de medias $(0,0,0)'$ y matriz de covarianzas
\[
Σ =
\begin{pmatrix}
1&2&−1\\
2&6&0\\
−1&0&4
\end{pmatrix}
\]


Definamos las v.a. $Y_1 = X_1 + X_3, Y_2 = 2X_1 − X_2 e Y_3 = 2X_3 − X_2$. Calcula la distribución de $Y_3$ dado que $Y_1=0$ e $Y_2=1$.

\solution

Lo primero es descubrir la matriz de la combinación lineal, esto es:
\[
\begin{pmatrix} Y_1\\Y_2\\Y_3\end{pmatrix} = \begin{pmatrix}1&0&1\\2&-1&0\\0&-1&2\end{pmatrix}\begin{pmatrix}X_1\\X_2\\X_3\end{pmatrix} \equiv N_3 \left( \begin{pmatrix}0\\0\\0\end{pmatrix}, \begin{pmatrix}3&-2&4\\-2&2&-2\\4&-2&22 \end{pmatrix} \right)
 \]

Llamamos \[A=\begin{pmatrix}3&-2&4\\-2&2&-2\\4&-2&22 \end{pmatrix}\]

¿De dónde sale esta matriz? Elena opina (y Jorge lo confirma) que $A = ΣBΣ'$, donde $B$ es la matriz de la combinación lineal, es decir: 
\[B=\begin{pmatrix}1&0&1\\2&-1&0\\0&-1&2\end{pmatrix}\]

\[
E(Y_3|y_1 = 0, y_2 = 1) = 0 + (4,-2) \begin{pmatrix} 3&-2\\-2&2\end{pmatrix}^{-1} \begin{pmatrix}0-0\\1-0\end{pmatrix} = ... = 1
\]

\[
V(Y_3|y_1=0,y_2=1) = 22 - (4,-2) \begin{pmatrix} 3&-2\\-2&2 \end{pmatrix}^{-1} \begin{pmatrix}4\\-2\end{pmatrix} = ... = 16
\]


Entonces, la distribución de $Y = (Y_1,Y_2,Y_3)' = N_3(1,16)$

\end{problem}

\section{Hoja 2}


\begin{problem}[1] Calcula la distribución exacta bajo la hipótesis nula del estadístico de Kolmogorov-Smirnov para muestras de tamaño 1.

\solution

La hipótesis sería $H_0 : F = F_0$ continua, con $X \sim F$

En este caso,

\[D=||F_1 - F_0||_{\inf} = (1) = \max\{F_0(x), 1 - F_0(x)\}\]

$(1)$ hay 2 posibles caminos. Al dibujar lo que nos dicen (una muestra de tamaño 1) podemos sacarlo por intuición. Sino, aplicamos la fórmula de los estadísticos.

Ahora calculamoms:

\[ P_{F_0}(D\leq x) = P_{F_0} = \left\{\max \{ ... \}\leq x\right\} = P_{F_0} = P_{F_0}\{ 1-x \leq F_0(x) \leq x \}\]

Resolvemos la desigualdad, aplicando que $F_0$ es una uniforme.

\[
P\{1-x \leq U \leq x\} = \left\{ \begin{array}{cc} 0 & x\leq \frac{1}{2} \\ 2x-1 & x\geq \frac{1}{2}\end{array} \right. \implies D \sim \mathcal{U}\left(\frac{1}{2},1\right)
\]

Ya que $1-x > x \dimplies x\le \frac{1}{2}$

\end{problem}
\begin{problem}[2] Se desea contrastar la hipótesis nula de que una única observación X procede de una distribución N(0,1). Si se utiliza para ello el contraste de Kolmogorov-Smirnov, determina para qué valores de X se rechaza la hipótesis nula a nivel α = 0,05.
\solution

Este ejercicio está muy relacionado con el primero. Es una aplicación al caso de la normal.


Mirando en la tabla, encontramos que para $α = 0.05$, entonces $d_α = 0.975$. Con esta inormación podemos construir la región crítica:
\[ R = \left\{\max\{\Phi(x), 1 - \Phi(x))\} > 0.975\right\} = \{\Phi(x) > 0.975\} \cup \{1 - \Phi(x) > 0.975\} =\]
\[ \{ X>\Phi^{-1}(0.975)\} \cup \{X < \Phi^{-1}(0.025)\}\]

Consultando las tablas, vemos que $\Phi^{-1}(0.975) = 1.96$ y por simetría, $\Phi^{-1}(0.025) = -1.96$

\[R = \{|X| > 1.96\}\]


\obs Es interesante saber que, al ser simétrica la normal, la interpretación gráfica es muy fácil. Si dividimos la normal en 3 intervalos, $(-∞ , -1.96) , (-1.96,1.96) , (1.96, ∞)$, el área encerrada en las colas es el nivel de significación, en este caso: \[\text{Area }\left((-∞ , -1.96)\cup (1.96, ∞)\right) = 0.05\]

\end{problem}
\begin{problem}[3] Da una demostración directa para el caso k = 2 de que la distribución del estadístico del contrast $\chi^2$ de bondad de ajuste converge a una distribución $\chi_1^2$ , es decir,
\[
T = \frac{(O1 − E1)^2}{E1} +
\frac{(O2 − E2)^2}{E2} \convs[d] \chi_1^2\]

\label{ej::2.3}

[Indicación: Hay que demostrar que $T = X^2_n$ , donde $X_n\convs[d] N(0,1)$. Para reducir los dos sumandos a uno, utilizar la relación existente entre O1, E1 y O2, E2.]
\solution

Si tenemos $n$ datos, vamos a construir la tabla de contingencia. Creo que consideramos una binomial porque, al sólo tener 2 clases, o eres de una o eres de la otra con una probabilidad $p$.

\begin{center}
\begin{tabular}{c|cc}
 & $A_1$ & $A_2$ \\\hline
 Obs & $n\gor{p}$ & $n(1-\gor{p})$\\
 Esp  & $np_0$ & $n(1-p_0)$\\
\end{tabular}
\end{center}

\[ T = \sum_{i=1}^2 \frac{(O_i - E_i)^2}{E_i} = \frac{n^2(\gor{p}-p_0)^2}{n} + \frac{n^2(\gor{p}-p_0)}{n(1-p_0)}  = ... \]
Simplificando, llegamos a:

\[
T = \left(\frac{|\gor{p}-p_0|}{\sqrt{\frac{p_0(1-p_0)}{n}}} \right)
\]

Está contando un montón de cosas interesantes que me estoy perdiendo.



Entre ellas, tenemos que $\sqrt{T} \convs[d]N(0.1)$ por el teorema central del límite ( es el caso particular para una binomial), con lo que $T\convs[d] \chi^2$. ¿Porqué 1 grado de libertad? Porque sólo estamos estimando 1 parámetro, el $\gor{p}$.

Esto responde también al problema 11. 

\end{problem}
\begin{problem}[4] El número de asesinatos cometidos en Nueva Jersey cada día de la semana durante el año 2003 se muestra en la tabla siguiente:

\begin{center}
\begin{tabular}{c|ccccccc}
Día & Lunes & Martes & Miércoles & Jueves & Viernes & Sábado & Domingo \\\hline
Frecuencia & 42 & 51 & 45 & 36 & 37 & 65 & 53
\end{tabular}
\end{center}

\ppart Contrasta a nivel α = 0,05, mediante un test $χ2$, la hipótesis nula de que la probabilidad de que se cometa un asesinato es la misma todos los días de la semana.

\ppart ¿Podría utilizarse el test de Kolmogorov-Smirnov para contrastar la misma hipótesis? Si tu
respuesta es afirmativa, explica cómo. Si es negativa, explica la razón.


\ppart Contrasta la hipótesis nula de que la probabilidad de que se cometa un asesinato es la misma desde el lunes hasta el viernes, y también es la misma los dos días del fin de semana (pero no es necesariamente igual en fin de semana que de lunes a viernes).

\solution

\spart $n = 329$, $E_i = \frac{329}{7}$ y $H_0 : p_i = \frac{1}{7}$

Calculamos el estadístico $T = ... = \sum_{i=1}^7 ... = ... = 13.32$

Por otro lado, $\chi^2_{6;0.05} = 12.59$, con lo que rechazamos la hipótesis.

\spart No podría utilizarse al tratarse de algo discreto y KS sólo sirve para continuas.

\spart

Tenemos la siguiente tabla:

\begin{center}
\begin{tabular}{c|ccccccc}
Día & Lunes & Martes & Miércoles & Jueves & Viernes & Sábado & Domingo \\\hline
Frecuencia & p & p & p & p & p & q & q
\end{tabular}
\end{center}

Con $5p + 2q = 1 \implies q = \frac{11-5p}{2}$

Entonces, tenemos \[ e.m.v.(p) =L(p)= p^{42+51+...+37} \left( \frac{11-5p}{2} \right)^{65+53} \]

Ahora, despejamos tomando $l(p) = ln(L(p)) = 211 ln(p) + 118ln\left(\frac{11-5p}{2}\right)$ y maximizamos:

\[
l'(p) = 0 \implies ... \left\{\begin{array}{c} \gor{p} = 0.128\\ \gor{q} = 0.179 \end{array}\right.
\]


Ahora construimos el estadístico:

\[
T = \sum_{i=1}^7 \frac{O_i^2}{\gor{E}_i^2} - n = ... = 5.4628
\]

Y comparamos con la $\chi^2$. ¿Cuántos grados de libertad? Si tenemos $7$ clases, siempre perdemos uno, con lo que serían 6. Sin embargo hemos estimado un parámetro, con lo que son $5$ grados de libertad. Entonces: $ c = \chi^2_{5;0.05} = 11.07$

Como $T < c$, no podemos rechazar la hipótesis.

\obs
Podríamos plantearnos contrastar que es uniforme de lunes a viernes ($H_1$) y otra uniforme distinta en fines de semana ($H_2$). Entonces tendríamos $H_0 : H_1 \cap H_2$, y construir la región $R = R_1 \cup R_2$. ¿Cuál es el problema de este camino?

El nivel de significación, ya que $P_{H_0}(R_1 \cup R_2) = P_{H_0}(R_1) + P_{H_0}(R_2) - P_{H_0}(R_1\cap R_2) = 2α - α^2 \sim 2α$. 

Podríamos tomar, chapucerillamente $α = \frac{α}{2}$ para que al final, $P_{H_0} ( R_1 \cup R_2) = α$. Aquí surge otro problema, que es que estamos despreciando la probabilidad de la intersección y tomándolo como independiente cuando no tiene porqué serlo. Es una aproximación ``buena'' que a veces se utiliza, pero pudiendo hacerlo bien... 

\end{problem}



\begin{problem}[5] Para estudiar el número de ejemplares de cierta especie en peligro de extinción que viven en un
bosque, se divide el mapa del bosque en nueve zonas y se cuenta el número de ejemplares de cada
zona. Se observa que 60 ejemplares viven en el bosque repartidos en las 9 zonas de la siguiente
forma:


\begin{center}
\begin{tabular}{|c|c|c|}
\hline
8&7&3 \\\hline
5&9&11 \\\hline
6&4&7 \\\hline
\end{tabular}
\end{center}

Mediante un contraste de hipótesis, analiza si estos datos aportan evidencia empírica de que los
animales tienen tendencia a ocupar unas zonas del bosque más que otras.

Tomamos $α = 0.01$
\solution

$T = 7.47$, $\chi^2_{8;0.001} = 20.09$

Aceptamos la hipótesis $H_0 : $ la especie se reparte uniformemente.

\end{problem}
\begin{problem}[6] Se ha desarrollado un modelo teórico para las diferentes clases de una variedad de moscas. Este
modelo nos dice que la mosca puede ser de tipo L con probabilidad p
2
, de tipo M con probabilidad
q
2 y de tipo N con probabilidad 2pq (p + q = 1). Para confirmar el modelo experimentalmente
tomamos una muestra de 100 moscas, obteniendo 10, 50 y 40, respectivamente.
\ppart
Hallar la estimación de máxima verosimilitud de p con los datos obtenidos.
\ppart
¿Se ajustan los datos al modelo teórico, al nivel de significación 0’05?
\solution

\doneby{Jorge}

\spart
Primero calculamos la función de verosimilitud para $p$:
\[L_n(p) = L_n(p) = \prod_{i=0}^n f(x_i;p) = (p^2)^{10} · (q^2)^{50} · (2pq)^{40}\]

El EMV lo obtendremos maximizando $\log L_n(p)$:
\[\log L_n(p) = 20 \log p + 100 \log q + 40 \log 2pq\]
\[\frac{\partial}{\partial p} \log L_n(p) = \frac{20}{p} - \frac{100}{1-p} + 40 \frac{2-4p}{2p(1-p)} = 0 \]

Maximizamos con $\hat{p}=\frac{3}{10} \implies \hat{q}=\frac{7}{10}$.

\spart
En este caso tomamos $H_0 \equiv P(X∈L)=p^2, P(X∈M)=q^2, P(X∈N)=2pq$

Usando el estado el contraste de bondad de ajuste de la $χ^2$, el estadístico de Pearson queda:
\[T = \sum_{i=1}^3 \frac{\left(O_i - \hat{E}_i\right)^2}{\hat{E}_i} = \sum_{i=1}^3 \frac{O_i^2}{\hat{E}_i} - n =\]
\[ = \frac{10^2}{p^2·100} + \frac{50^2}{(1-p)^2 · 100} + \frac{40^2}{2p(1-p)·100} - 100 ≈ 0.22\]

Puesto que en este caso $k=3$ y hemos estimado 1 parámetro ($p$), tenemos que $T$ se distribuye como una $χ^2_{3-1-1}$. En las tablas nos encontramos con que $χ^2_{1;0.05}=3.84 > T$ y no rechazamos $H_0$, es decir los datos se ajustan al modelo teórico.

\end{problem}
\begin{problem}[7]
\ppart
Aplica el test de Kolmogorov-Smirnov, al nivel 0.05, para contrastar si la muestra (3.5, 4, 5, 5.2, 6) procede de la $U(3,8)$.
\ppart
Aplica el test de Kolmogorov-Smirnov, al nivel 0.05, para contrastar la hipótesis de que la
muestra (0, 1.2, 3.6) procede de la distribución $N(µ~=~1;σ~=~5)$.
\solution
\doneby{Jorge}

\spart
La función de distribución de una $U(3,8)$ es:
\[
	F(x)=
	\begin{cases}
		0 & ,x<3 \\
		\frac{x-3}{5} & ,3≤x≤8 \\
		1 & ,x>8
	\end{cases}
\]

\begin{center}
	\begin{tabular}{ c | c | c | c | c }	
		$x_{(i)}$ & $\frac{i}{n}$ & $F_0(x_{(i)})$ & $D_n^{+}$ & $D_n^{-}$ \\ \hline
		3.5 & 0.2  & 0.1  & 0.1  & 0.1   \\ \hline
		4   & 0.4  & 0.2  & 0.2  & 0     \\ \hline
		5   & 0.6  & 0.4  & 0.2  & 0     \\ \hline
		5.2 & 0.8  & 0.44 & 0.36 & -0.16 \\ \hline
		6   & 1    & 0.6  & 0.4  & -0.2
	\end{tabular}
\end{center}

Tendremos por tanto que $D_n=\norm{F_n-F_0}_∞=0.4$. Si nos vamos a la tabla del contraste K-S vemos que $c=0.565$ para $α=0.05$.

Como $D_n<c$ \textbf{no rechazamos} la hipótesis nula de que las muestras vienen de la uniforme.

\spart
\begin{center}
	\begin{tabular}{ c | c | c | c | c }	
		$x_{(i)}$ & $\frac{i}{n}$ & $F_0(x_{(i)})$ & $D_n^{+}$ & $D_n^{-}$ \\ \hline
		0 & 0.3 & 0.42 & -0.12 & 0.42   \\ \hline
		1.2 & 0.6 & 0.52 & 0.08 & 0.22     \\ \hline
		3.6 & 1 & 0.7 & 0.3 & 0.1
	\end{tabular}
\end{center}

Tendremos por tanto que $D_n=\norm{F_n-F_0}_∞=0.42$. Si nos vamos a la tabla del contraste K-S vemos que $c=0.708$ para $α=0.05$.

Como $D_n<c$ \textbf{no rechazamos} la hipótesis nula de que las muestras vienen de la $N(1,5)$.

\end{problem}




\begin{problem}[8] Se ha clasificado una muestra aleatoria de 500 hogares de acuerdo con su situación en la ciudad
(Sur o Norte) y su nivel de renta (en miles de euros) con los siguientes resultados:
\begin{center}
	\begin{tabular}{c c c}
		\hline
		Renta & Sur & Norte\\ \hline
		0 a 10 & 42 & 53 \\
		10 a 20 & 55 & 90 \\
		20 a 30 & 47 & 88 \\
		más de 30 & 36 & 89\\ \hline
	\end{tabular}
\end{center}

\ppart
A partir de los datos anteriores, contrasta a nivel α = 0,05 la hipótesis nula de que en el sur los
hogares se distribuyen uniformemente en los cuatro intervalos de renta considerados.

\ppart
A partir de los datos anteriores, ¿podemos afirmar a nivel α = 0,05 que la renta de los hogares
es independiente de su situación en la ciudad?
\solution


\spart
Tenemos $H_0: p_i=\frac{1}{4}$ y usando el contraste de bondad de ajuste de la $χ^2$:
\[T = \sum_{i=1}^4 \frac{O_i^2}{E_i} - n_{\text{sur}} = \frac{42^2 + 55^2 + 47^2 + 36^2}{\frac{1}{4}·180} - 180 = 4.31\]

En las tablas encontramos que $χ^2_{k-1;α} = χ^2_{3;0.05} = 7.815$. Como $T<χ^2_{3;0.05}$, \textbf{no podemos rechazar} la hipótesis nula de que en el sur los hogares se distribuyen uniformemente en los cuatro intervalos de renta considerados.

\spart
Lo primero que haremos es estimar las probabilidades de que la v.a. caiga en cada una de las 6 clases que tenemos ($A_i$ serán los intervalos de renta y $B_i$ si el hogar es del norte o del sur):
\[p(x∈A_1) = \frac{42+53}{500} = 0.19\]
\[p(x∈A_2) = \frac{55+90}{500} = 0.29\]
\[p(x∈A_3) = \frac{47+88}{500} = 0.27\]
\[p(x∈A_4) = \frac{36+89}{500} = 0.25\]

\[p(x∈B_1) = \frac{42+55+47+36}{500} = 0.36\]
\[p(x∈B_2) = \frac{53+90+88+89}{500} = 0.64\]

Bajo la $H_0$ consideramos $A_i$ independiente de $B_i$, de modo que $p_{i,j} = p_i·p_j$ tal y como se muestra en la siguiente tabla:

\begin{center}
	\begin{tabular}{c | c}
		$p_{1,1} = 0.0684$ & $p_{1,2} = 0.1216$\\ \hline
		$p_{2,1} = 0.1044$ & $p_{2,2} = 0.1856$\\ \hline
		$p_{3,1} = 0.0972$ & $p_{3,2} = 0.1728$\\ \hline
		$p_{4,1} = 0.09$ & $p_{4,2} = 0.16$
	\end{tabular}
\end{center}

Sabiendo que $\hat{E}_{ij} = n·p_{i,j}$:

\begin{center}
	\begin{tabular}{c | c}
		$\hat{E}_{1,1} = 34.2$ & $\hat{E}_{1,2} = 60.8$\\ \hline
		$\hat{E}_{2,1} = 52.2$ & $\hat{E}_{2,2} = 92.8$\\ \hline
		$\hat{E}_{3,1} = 48.6$ & $\hat{E}_{3,2} = 86.4$\\ \hline
		$\hat{E}_{4,1} = 45$ & $\hat{E}_{4,2} = 80$
	\end{tabular}
\end{center}

\[T=\sum_{j=1}^2 \sum_{i=1}^4 \frac{O_{ij}^2}{\hat{E}_{ij}} - n = 8.39\]

Si nos vamos a las tablas vemos que $χ^2_{(k-1)(p-1); α} = χ^2_{3·1; 0.05} = 7.815 < T$ y por tanto \textbf{rechazamos la hipótesis nula} de que la renta de los hogares es independiente de su situación en la ciudad.


\end{problem}
\begin{problem}[9] A finales del siglo XIX el físico norteamericano Newbold descubrió que la proporción de datos
que empiezan por una cifra d, p(d), en listas de datos correspondientes a muchos fenómenos
naturales y demográficos es aproximadamente:
p(d) = log10
d + 1
d
!
, d = 1,2,...,9.
Por ejemplo, p(1) = log10 2 ≈ 0,301030 es la frecuencia relativa de datos que empiezan por 1. A raíz
de un artículo publicado en 1938 por Benford, la fórmula anterior se conoce como ley de Benford.
El fichero poblacion.RData incluye un fichero llamado poblaciones con la población total de los
municipios españoles, así como su población de hombres y de mujeres.
(a) Contrasta a nivel α = 0,05 la hipótesis nula de que la población total se ajusta a la ley de Benford.
(b) Repite el ejercicio pero considerando sólo los municipios de más de 1000 habitantes.
(c) Considera las poblaciones totales (de los municipios con 10 o más habitantes) y contrasta a nivel
α = 0,05 la hipótesis nula de que el primer dígito es independiente del segundo.
(Indicación: Puedes utilizar, si te sirven de ayuda, las funciones del fichero benford.R).
\solution

\end{problem}
\begin{problem}[10] Se ha llevado a cabo una encuesta a 100 hombres y 100 mujeres sobre su intención de voto. De
las 100 mujeres, 34 quieren votar al partido A y 66 al partido B. De los 100 hombres, 50 quieren
votar al partido A y 50 al partido B.
\ppart
Utiliza un contraste basado en la distribución $χ^2$ para determinar si con estos datos se puede
afirmar a nivel $α = 0,05$ que el sexo es independiente de la intención de voto.
\ppart
Determina el intervalo de valores de α para los que la hipótesis de independencia se puede
rechazar con el contraste del apartado anterior.
\solution

Este ejercicio ha caido en un examen.

\doneby{Jorge}

\spart
Procediendo como en el ejercicio anterior obtendremos que bajo la hipótesis nula de independencia:
\[p_{A,\text{mujer}} = p_{A, \text{hombre}} = 0.21\]
\[p_{B,\text{mujer}} = p_{B, \text{hombre}} = 0.29\]

Por tanto:
\[T=\sum_{j=1}^2 \sum_{i=1}^2 \frac{O_{ij}^2}{\hat{E}_{ij}} = 5.25\]

Si nos vamos a las tablas vemos que $χ^2_{(k-1)(p-1); α} = χ^2_{1; 0.05} = 3.841 < T$, y por tanto \textbf{rechazamos la hipótesis nula} de que el sexo es independiente de la intención de voto.

\paragraph*{En clase: } hemos contrastado homogeneidad (las intenciones de voto se distribuyen igual) en vez de independencia, pero viene a ser lo mismo.

\spart
El p-valor asociado a $T=5.25$ es $\left[1 - F_{χ^2_{1}}(5.25)\right] = 0.02$, por tanto para $α~∈~[0.02,1]$ rechazamos la hipótesis de independencia del apartado anterior.

Para calcular el p-valor, utilizamos que una $\chi^2_1$ es una normal al cuadrado, es decir:

\[p = P(X>5.25) = P ( Z^2 > 5.25) = P(|Z| > 2.29) = 0.022\]

siendo $Z\sim N(0,1)$



\end{problem}
\begin{problem}[11] Sea X1,...,Xn una muestra de una distribución Bin(1, p). Se desea contrastar H0 : p = p0. Para ello hay dos posibilidades: 

\ppart  Un contraste de proporciones basado en la región crítica
$R = \{|\gor{p} − p_0|\} > z\frac{α}{2} p p0(1 − p0)/n $
\ppart un contraste $χ2$ de bondad de ajuste con k = 2 clases. ¿Cuál es la relación entre ambos contrastes?
\solution

Consultar el ejercicio \ref{ej::2.3}.

\end{problem}
\begin{problem}[12] En un estudio de simulación se han generado 10000 muestras aleatorias de tamaño 10 de una
distribución $N(0,1)$. Para cada una de ellas se ha calculado con R el estadístico de Kolmogorov-Smirnov
para contrastar la hipótesis nula de que los datos proceden de una distribución normal
estándar, y el correspondiente p-valor.
\ppart
Determina un valor x tal que la proporción de estadísticos de Kolmogorov-Smirnov mayores
que x, entre los 10000 obtenidos, sea aproximadamente igual a 0.05. ¿Cuál es el valor teórico al que
se debe aproximar la proporción de p-valores menores que 0.1 entre los 10000 p-valores obtenidos?
\ppart
¿Cómo cambian los resultados del apartado anterior si en lugar de considerar la distribución
normal estándar se considera una distribución uniforme en el intervalo (0,1)?
\solution

\doneby{Jorge}

\spart
\begin{itemize}
	\item La $x$ que nos piden es $f_{D,α=0.05}$ ($f_D$ es la función de densidad del estadístico K-S). Si acudimos a la tabla vemos que para $n=10$ $x = f_{D,0.05} = 0.41$. Un poco más explicado el razonamiento:
	\[
		\underbrace{\frac{\#\{i : D_i > x\}}{10000}}_{P(D>x)} \simeq 0.05
	\]

	\item Precisamente el $10\%$ de los p-valores debería ser menor que $0.1$, ya que hacer un contraste nivel de significación $α=0.1$ significa que en el $10\%$ de los casos rechazamos la hipótesis nula, es decir, en le $10\%$ de los casos los p-valores son $<0.1$.

	Esto se debe al concepto de nivel de significación, ya que si el nivel de significación es $0.01$, entonces nos estamos equivocando en 1 de cada 100 contrastes que hagamos, es decir:

	\[
		\frac{\# \{ i : p^{(i)} < α\}}{10000} \simeq α
	\]
\end{itemize}

\spart
\begin{itemize}
	\item Al contrastar con una distribución $U(0,1)$ cabría esperar que las $1000$ $D_i$ tomaran valores más altos, pues la distancia entre $F_n$ (que se monta a partir de datos que vienen de una $N(0,1)$) y $F_0=F_{U(0,1)}$ sería más grande que al tomar como $F_0$ la de una $N(0,1)$. Por tanto \textbf{el valor $x$ debería ser mayor}.

	\item Por otra parte la proporción de p-valores menores que $0.1$ debería aumentar, ya que el test debería devolver p-valores más pequeños (pues debería de rechazar la hipótesis de que los datos vienen de una $U(0,1)$).
\end{itemize}

\paragraph{Solución de clase:}

Al tener muchas muchas muestras, las frecuencias deberían ser las probabilidades.

\end{problem}


%%%%%%%%%%%%
%% HOJA 3 %%
%%%%%%%%%%%%

\section{Hoja 3}
\begin{problem}[1] La Comunidad de Madrid evalúa anualmente a los alumnos de sexto de primaria de todos los colegios sobre varias materias. Con las notas obtenidas por los colegios en los años 2009 y 2010 (fuente: diario El País) se ha ajustado el modelo de regresión simple:
\[Nota2010 = β_0 + β_1Nota2009 + ε,\]
en el que se supone que la variable de error ε verifica las hipótesis habituales. Los resultados
obtenidos con R fueron los siguientes:\\[1em]

Coefficients:

\begin{tabular}{c | c | c | c | c}
	~ & Estimate & Std. Error & t value & Pr(>|t|) \\
	(Intercept) & 1.40698 & 0.18832 & 7.471 & 1.51e-13 \\
	nota09 & 0.61060  & 0.02817 & 21.676 & < 2e-16
\end{tabular}

 Residual standard error: 1.016 on 1220 degrees of freedom

 Multiple R-squared: 0.278,Adjusted R-squared: 0.2774

 F-statistic: 469.8 on 1 and 1220 DF,  p-value: < 2.2e-16 \\[1em]

También se sabe que en 2009 la nota media de todos los colegios fue 6,60 y la cuasidesviación típica fue 1,03 mientras que en 2010 la media y la cuasidesviación típica fueron 5,44 y 1,19, respectivamente.

\ppart ¿Se puede afirmar a nivel $α = 0,05$ que existe relación lineal entre la nota de 2009 y la de 2010? Calcula el coeficiente de correlación lineal entre las notas de ambos años.

\ppart Calcula un intervalo de confianza de nivel $95\%$ para el parámetro $β_1$ del modelo.

\ppart Calcula, a partir de los datos anteriores, un intervalo de confianza de nivel $95\%$ para la nota
media en 2010 de los colegios que obtuvieron un 7 en 2009.


\solution
\end{problem}



\chapter{Recordando}

\section{Estimador de máxima verosimilitud}
\label{sec:estimadorMaximaVerosimilitud}
En lo que sigue vamos a suponer que $\{X_n\}$ es una muestra formada por v.a.i.i.d. cuya distribución tiene una función de densidad o de masa $f(.;\theta_0)$ perteneciente a una familia de funciones $\{f(.;\theta) \tq \theta \in \Theta\}$. $\theta_0$ nos indica el valor real, y $\theta$ es un parámetro genérico.

Intuitivamente, lo que pensamos con este método es que la función de masa mide lo verosímil que es que salga un cierto parámetro. 

\begin{defn}[Función\IS de verosimilitud] También llamada \textit{likelihood function}. Dada una muestra fija $\{x_n\}$, se define como

\[ L_n(\theta;x_1,\dotsc,x_n) = L_n(\theta) = \prod_{i=1}^n f(x_i;\theta) \]
\end{defn}

\begin{defn}[Estimador\IS de máxima verosimilitud]\label{defEMV} También llamado EMV o MLE (\textit{maximum likelihood estimator}) es el argumento que maximiza la función de verosimilitud:

\[ \hat{\theta}_n = \hat{\theta}_n(x,\dotsc,x_n) = \argmax_{\theta\in\Theta} L_n(\theta;x_1,\dotsc,x_n) \]

cuando ese máximo está bien definido.
\end{defn}

Para evitar usar derivadas en un producto potencialmente muy largo, podemos maximizar el logaritmo de la verosimilitud, que es creciente y está bien definido porque la densidad es siempre mayor que cero, y los casos en los que sea cero no los estudiamos porque no ocurren (ocurren con probabilidad 0).




\end{document}
