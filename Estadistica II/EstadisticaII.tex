\documentclass[nochap]{apuntes}

\usepackage{hyperref}

\usepackage{tikztools}
\usepackage{fastbuild}
\usepackage{tikz-3dplot}

\usepackage{tikz}
\usepackage{graphicx}
\usepackage{latexsym, amsfonts, amsmath, amssymb, amscd, epsfig,amsthm}
\input xy
\xyoption{all} %%!!
\usetikzlibrary{calc, intersections}
\author{Alberto Parramón}
\date{2014/2015 2º cuatrimestre}

\renewcommand*{\arraystretch}{1.5}
\title{Estadística II}
\precompileTikz

\begin{document}

\pagestyle{plain}
\maketitle

\tableofcontents
\newpage

\section{Introducción}
Se presentan apuntes de Estadística II, tomados de la clase dada por José Berrendero.

El profesor nos facilita unas diapositivas, por tanto, se mostrarán las mismas y se explicaran con detalle.

\section{Distribución normal multivariante}

\includepdf[frame=true, noautoscale=true, delta=10 10, nup=1x2,pages={2-3}, scale=1]{pdf/_tema1.pdf}

\subsection{Esperanza, varianza y covarianza de variables aleatorias}
Dada una variable aleatoria definimos:
\begin{itemize}
\item Esperanza: $\mu = \mathbb{E}(X) = \int_{-\infty}^{\infty}x\cdot f_P(x) dx$

Propiedades:
\begin{enumerate}
\item $\mathbb{E}(aX) = a\mathbb{E}(X)$
\item $\mathbb{E}(X+Y) = \mathbb{E}(X)-\mathbb{E}(Y)$
\item $\mathbb{E}(X+c) = \mathbb{E}(X)+c$ (La esperanza de una constante es la propia constante)
\end{enumerate}
\item Varianza: $Var(X) = \mathbb{E}((X-\mathbb{E}(X))^2) =\mathbb{E}((X-\mu)^2) = \mathbb{E}(X^2)-\mu^2$

Propiedades:
\begin{enumerate}
\item $Var(X+b)=Var(X)$
\item $Var(aX)=a^2Var(X)$
\item $Var(X)\geq 0$
\end{enumerate}
\item Covarianza (entre dos variables aleatorias $X_i$, $X_j$): $\sigma_{i,j} = Cov(X_i,X_j) = \mathbb{E}\left((X_i-\mathbb{E}(X_i))(X_j-\mathbb{E}(X_j))\right) = \mathbb{E}(X_i X_j)-\mathbb{E}(X_i)\mathbb{E}(X_j)$

Dos propiedades importantes de la covarianza son:

\begin{enumerate}
\item Cov(X,X)= Var(X)
\item $Cov(X,Y)=Cov(Y,X)$
\end{enumerate}

\end{itemize}

\subsection{Esperanza, varianza y covarianza de vectores aleatorios}

Un vector aleatorio es un vector de variables aleatorias.

Notación: como durante el curso vamos a trabajar con vectores aleatorios, vamos a generalizar los símbolos que iremos usando:
\begin{itemize}
\item $X = (X_1, X_2,...,X_p)'$ será un vector de p variables aleatorias. Las variables aleatorias serán $X_1, X_2,...,X_p$. La comilla simple $'$ indica que $X$ es un vector columna.
\item $\mu$ será la esperanza del vector aleatorio X: $\mathbb{E}(X)$. Las esperanzas de cada variable aleatoria serán $\mu_1, \mu_2,...,\mu_p$.
\item Si A es una matriz, A' es su traspuesta
\end{itemize}

Por tanto, dado un vector de p variables aleatorias (vector aleatorio p-dimensional), tenemos un resultado parecido.

\begin{itemize}
\item Esperanza. Será un vector columna con las esperanzas de cada variable aleatoria.
\[
\mathbb{E}(X) = \mu = (\mu_1, \mu_2,..., \mu_p)'
\]

Donde cada $\mu_i = \mathbb{E}(X_i)$.

Ejemplo p=3:
\[
\mathbb{E}(X)=
\mathbb{E}\left[
\left(
\begin{array}{c}
X_1\\
X_2\\
X_3
\end{array}
\right)
\right]=
\left(
\begin{array}{c}
\mathbb{E}(X_1)\\
\mathbb{E}(X_2)\\
\mathbb{E}(X_3)
\end{array}
\right)=
\left(
\begin{array}{c}
\mu_1\\
\mu_2\\
\mu_3
\end{array}
\right)=
\mu
\]

Propiedades:
\begin{enumerate}
\item $\mathbb{E}(X+c) = \mathbb{E}(X)+c$. Como en el caso de variables aleatorias.
\item $\mathbb{E}(AX) = A\mathbb{E}(X)$. Donde A es una matriz de dimensión $pxp$ siendo p la dimensión de X.

Lo vemos para p=3:

\[
\mathbb{E}(AX)=
\mathbb{E}\left[
\left(
\begin{array}{ccc}
a_{1,1}& a_{1,2}& a_{1,3}\\
a_{2,1}& a_{2,2}& a_{2,3}\\
a_{3,1}& a_{3,2}& a_{3,3}
\end{array}
\right)
\left(
\begin{array}{c}
X_1\\
X_2\\
X_3
\end{array}
\right) \right]=
\mathbb{E}\left[ 
\left(
\begin{array}{c}
a_{1,1}X_1 - a_{1,2}X_2 - a_{1,3}X_3\\
a_{2,1}X_1 - a_{2,2}X_2 - a_{2,3}X_3\\
a_{3,1}X_1 - a_{3,2}X_2 - a_{3,3}X_3
\end{array}
\right)
\right]=
\]

\[
=\left(
\begin{array}{c}
a_{1,1}\mathbb{E}(X_1) - a_{1,2}\mathbb{E}(X_2) - a_{1,3}\mathbb{E}(X_3)\\
a_{2,1}\mathbb{E}(X_1) - a_{2,2}\mathbb{E}(X_2) - a_{2,3}\mathbb{E}(X_3)\\
a_{3,1}\mathbb{E}(X_1) - a_{3,2}\mathbb{E}(X_2) - a_{3,3}\mathbb{E}(X_3)
\end{array}
\right)=
\left(
\begin{array}{ccc}
a_{1,1}& a_{1,2}& a_{1,3}\\
a_{2,1}& a_{2,2}& a_{2,3}\\
a_{3,1}& a_{3,2}& a_{3,3}
\end{array}
\right)
\left(
\begin{array}{c}
\mathbb{E}(X_1)\\
\mathbb{E}(X_2)\\
\mathbb{E}(X_3)
\end{array}
\right)=
\]
\[
=A\mathbb{E}(X)
\]

\end{enumerate}

\item Varianza. La varianza va a ser una matriz, donde cada elemento va a ser la covarianza entre dos de las p variables aleatorias que conforman el vector. Será por tanto una matriz simétrica (ya que $\sigma_{i,j}=Cov(X_i,X_j)=Cov(X_j,X_i)=\sigma_{j,i}$). La matriz resultante será la llamada matriz de covarianzas $\Sigma$.
\[
Var(X)=\mathbb{E}\left((X-\mu)(X-\mu)'\right) = \mathbb{E}(XX')-\mu \mu'=\Sigma
\]

\begin{proof}
\[
Var(X)=\mathbb{E}\left((X-\mu)(X-\mu)'\right) = \mathbb{E}(XX'- \mu X' - X \mu'+\mu \mu')= 
\]
\[
\mathbb{E}(XX')-\mathbb{E}(\mu X')-\mathbb{E}(X\mu')+\mathbb{E}(\mu \mu')= \mathbb{E}(XX')-\mu \mathbb{E}(X')-\mu' \mathbb{E}(X)+\mu\mu'=
\]
\[
 \mathbb{E}(XX')-\mu \mu'-\mu' \mu+\mu \mu' = \mathbb{E}(XX')-\mu \mu'=\Sigma
\]
\end{proof}

Ejemplo p=3:

\[
Var(X)=
\mathbb{E}\left[
\left(
\begin{array}{c}
X_1-\mu_1\\
X_2-\mu_2\\
X_3-\mu_3
\end{array}
\right)
(X_1-\mu_1, X_2-\mu_2, X_3-\mu_3)\right]=
\left(
\begin{array}{ccc}
\sigma_{1,1}& \sigma_{1,2}& \sigma_{1,3} \\
\sigma_{2,1}& \sigma_{2,2}& \sigma_{2,3} \\
\sigma_{3,1}& \sigma_{3,2}& \sigma_{3,3}
\end{array}
\right)=
\]

\[
=\left(
\begin{array}{ccc}
Var(X_1)& \sigma_{1,2}& \sigma_{1,3} \\
\sigma_{2,1}& Var(X_2)& \sigma_{2,3} \\
\sigma_{3,1}& \sigma_{3,2}& Var(X_3)
\end{array}
\right) = \Sigma
\]

Donde se cumple que $\sigma_{1,2}=\sigma_{2,1}$, $\sigma_{1,3}=\sigma_{3,1}$ y $\sigma_{3,2}=\sigma_{2,3}$. Y por tanto $\Sigma$ es simétrica. 
\end{itemize}

Propiedades:
\begin{enumerate}
\item $Var(AX+b) = \mathbb{E}\left[ A(X-\mu)(X-\mu)'A' \right]=A \Sigma A'$.  Donde A es una matriz de dimensión $pxp$ siendo p la dimensión de X.
\begin{proof}
\[
Var(AX+b) = \mathbb{E}\left[ (AX+b-A\mu-b)(AX+b-A\mu-b)' \right] =
\]
\[
 =\mathbb{E}\left[ (AX-A\mu)(AX-A\mu)' \right] = \mathbb{E}\left[A(X-\mu)(X-\mu)'A'\right] = A\mathbb{E}\left[(X-\mu)(X-\mu)'\right]A' = 
 \]
 \[
 =A \Sigma A'
\]

\end{proof}
\end{enumerate}

\textcolor{red}{Mirar si tiene importancia lo de $\Sigma$ semidefinida positiva y tal}

\subsection{Función característica}
La función característica de un vector aleatorio X es:
\[
\phi_X(t)=\mathbb{E}(\exp^{it'X})
\]

Siendo X y t p-dimensionales.

\prop Mecanismo de Cramer-Wold...

Esta función caracteriza la distribución de X:
\prop Sean X e Y dos vectores aleatorios:
\[
\phi_X(t)=\phi_Y(t) \Leftrightarrow X \stackrel{d}{=} Y
\]

\textcolor{red}{Completar este apartado consultando a Elena}

\subsection{Matriz de covarianzas}
Como ya dijimos anteriormente la matriz de covarianzas $\Sigma$ define la varianza de un vector aleatorio y es simétrica. Por tanto podemos expresar $\Sigma$ de la siguiente forma:
\[
\Sigma = CDC^{-1}
\]

Siendo D una matriz diagonal.

\textcolor{red}{$C^{-1}=C'$ ya que las columnas de C son vectores ortogonales. OJO CUIDAO, que tienen que ser ortonormales...
Una matriz real A es ortogonal si y sólo si sus vectores filas o vectores columna son cada uno un conjunto ortonormal de vectores.}
Por tanto:
\[
\Sigma = CDC'  \text{ y } \Sigma^{-1} = CD^{-1}C'
\]

\textcolor{blue}{Caso particular:
\[
p=2 \text{ , } 
\mu=\left(
\begin{array}{c}
0\\
0
\end{array}
\right)
\text{ , }
\left(
\begin{array}{cc}
\lambda_1& 0 \\
0 & \lambda_2
\end{array}
\right)
\]
Tenemos:
\[
(X_1, X_2)
\left(
\begin{array}{cc}
\lambda_1& 0 \\
0 & \lambda_2
\end{array}
\right)
\left(
\begin{array}{c}
X_1\\
X_2
\end{array}
\right) = cte
\Rightarrow
\frac{X_1^2}{\lambda_1}+\frac{X_2^2}{\lambda_2}=cte
\]
}


\subsection{Estandarización multivariante}
\begin{defn}
Sea un vector aleatorio X, es normal p-dimensional con vector de medias $\mu$ y matriz de covarianzas $\Sigma$ (notación: $X\equiv N_p(\mu, \Sigma)$) si tiene densidad dada por:

\[
f(x)=\abs{\Sigma}^{-1/2}(2\pi)^{-p/2} exp \left( -\frac{1}{2}(x-\mu)' \right) 
\]
\end{defn}

\prop Si $X \equiv N_p(\mu, \Sigma)$ y definimos $Y = \Sigma^{-1/2}(X-\mu)$, entonces $Y_1,...,Y_p$ son i.i.d. N(0,1).

\begin{proof}
Sabemos por definición que:
\[
f_X(x)=\abs{\Sigma}^{-1/2}(2\pi)^{-p/2} exp \left( -\frac{1}{2}(x-\mu)' \right) 
\]

Vamos a aplicar un cambio de variable en la fórmula de la densidad:

Despejando de $Y = h(X)= \Sigma^{-1/2}(X-\mu)$, obtenemos que $\Sigma^{1/2}Y+\mu=h^{-1}(Y)=X$.

Y ahora cogemos el Jacobiano de $h^{-1}(Y)=X$ que será $\Sigma^{1/2}$ ($\mu$ es una constante e Y es la variable).

También hay que considerar la exponencial de la fórmula de la densidad, ahi hacemos el cambió de variable de:

$$e^X \text{por} e^{h^{-1}(Y)}=e^{\Sigma^{1/2}Y+\mu}$$ 

Y el Jacobiano sería $e^{\Sigma^{1/2}Y}$:
 

Por tanto nos quedaría:
\[
f(X) = f(h^{-1}(Y))*\abs{Jh(x)} = \abs{\Sigma}^{-1/2}(2 \pi)^{-p/2} \exp\left(-\frac{1}{2}(\Sigma^{-1/2}Y+\mu-\mu)'  \right) \exp\left( \Sigma^{1/2}Y \right) \Sigma^{1/2}  =
\]
\[
= \abs{\Sigma}^{-1/2}(2 \pi)^{-p/2} \exp\left(-\frac{1}{2}(\Sigma^{-1/2}Y)' \right) \exp\left( \Sigma^{1/2}Y \right) \abs{\Sigma}^{1/2} =
\]
\[
\abs{\Sigma}^{-1/2}(2 \pi)^{-p/2} \exp\left(-\frac{1}{2}(Y'\Sigma^{-1/2}\Sigma^{1/2}Y \right) \abs{\Sigma^{1/2}} = (2 \pi)^{-p/2} \exp\left(-\frac{1}{2}(Y'Y) \right) 
\]
\end{proof}


\subsection{Ejercicio 1}
Definimos el siguiente vector aleatorio: $X = (X_1,X_2,X_3)' \equiv N_3(\mu, \Sigma)$ con:

\[
\mu=
\left(
\begin{array}{c}
0\\
0\\
0
\end{array}
\right) \text{,       }
\Sigma=
\left(
\begin{array}{ccc}
7/2& 1/2& -1 \\
1/2& 1/2& 0 \\
-1& 0& 1/2
\end{array}
\right)
\]

\ppart Calcula las distribuciones marginales $X_i \equiv N(\mathbb{E}(X_i), Var(X_i))$:

$X_1\equiv N(0, 7/2)$

$X_2\equiv N(0, 1/2)$

$X_3\equiv N(0, 1/2)$

Para calcular estos valores solo hace falta mirar los datos que nos da el problema, el vector de medias $\mu$ y la matriz de covarianzas $\Sigma$:

\[
\Sigma=\left(
\begin{array}{ccc}
Var(X_1)& \sigma_{1,2}& \sigma_{1,3} \\
\sigma_{2,1}& Var(X_2)& \sigma_{2,3} \\
\sigma_{3,1}& \sigma_{3,2}& Var(X_3)
\end{array}
\right)
\]

\[
\mu=
\left(
\begin{array}{c}
\mathbb{E}(X_1)\\
\mathbb{E}(X_2)\\
\mathbb{E}(X_3)
\end{array}
\right)=
\left(
\begin{array}{c}
\mu_1\\
\mu_2\\
\mu_3
\end{array}
\right)
\]

\ppart Calcula la distribución del vector $(X_1,X_2)'$:

Este vector sigue una distribución normal que puede obtener de las matriz $\Sigma$ y el vector de medias $\mu$:
\[
\left(
\begin{array}{c}
X_1\\
X_2
\end{array}
\right)
\equiv N_2\left[
\left(
\begin{array}{c}
0\\
0
\end{array}
\right)
\text{, }
\left(
\begin{array}{cc}
7/2& 1/2 \\
1/2 & 1/2
\end{array}
\right)
\right] 
\]

\ppart ¿Son $X_2$ y $X_3$ independientes?

Sí son independientes ya que la covarianza entre ambas variables es 0. La covarianza entre $X_2$ y $X_3$ es el elemento de la fila 3 y la columna 2 de la matriz de covarianzas $\Sigma$, (que al ser $\Sigma$ simétrica coincide con el elemento de la fila 2 y la columna 3).

\ppart ¿Es $X_3$ independiente del vector $(X_1, X_2)'$?
???

\ppart Calcula la  distribución de la variable aleatoria $(2X_1-X_2+3X_3)$.

Procedemos de la siguiente manera:

\[
(2X_1-X_2+3X_3)=(2,-1,3)\left(
\begin{array}{c}
X_1\\
X_2\\
X_3
\end{array}
\right)\equiv
N\left( 0,  \right)
\]



\subsection{Distribuciones condicionadas}

\begin{prop}

Sea $X=(X_1|X_2)$ con $X_1∈ℝ^p$ y $X_2∈ℝ^{p-q}$. Consideramos las particiones correspondientes de $µ$ y de $\Sigma$.

\end{prop}

\begin{proof}
Definimos $X_{2.1} = X_2 - Σ_{21}Σ_{11}^{-1}X_1$.

\[
\begin{pmatrix}
X_1\\
X_{2.1}
\end{pmatrix} = 
\begin{pmatrix}
I &| &0\\
\hline
- Σ_{21}Σ_{11}^{-1}  &| &I
 \end{pmatrix}
\]

Como es una combinación lineal de $(X_1,X_{2.1})'$, entonces $X_{2.1}$ es normal multivariante.

Vamos a calcular la media y la matriz de covarianzas de $X_{2.1}$

$X_{2-1} = N\left( µ_2-Σ_{21}Σ_{11}^{-1}µ_1 , \begin{pmatrix} Σ_{11} &|&0\\\hline 0&|&Σ_{2.1} \end{pmatrix} \right)$

Donde las covarianzas se calculan: $AΣA'$, siendo $A$ la matriz de la combinación lineal, es decir:

\[
A=\begin{pmatrix}
I &| &0\\
\hline
- Σ_{21}Σ_{11}^{-1}  &| &I
 \end{pmatrix}
\]



\paragraph{Conclusiones:}

\begin{itemize}
	\item $X_1$ es independiende de $X_{2.1}$
	\item $X_{2.1}$ es normal, con media y varianza calculadas anteriormente.
	\subitem $X_{2.1}|X_1$, al ser independientes, también se distribuye normalmente, con los mismos parámetros.
	\item Dado $X_1$, los vectores $X_{2.1}$ y $X_2$  difieren en el vector constante $Σ_{21}Σ_{11}^{-1}X_1 \implies X_2|X_1 = N\left( µ_{2.1}, Σ_{2.1} \right)$
\end{itemize}

\end{proof}

\begin{example}
Vamos a considerar $X_1, X_2$ como escalares, para entender la proposición. Este ejemplo le surgió a un investigador que quería predecir la estatura de los hijos en función de la de los padres (que no padres y madres, sólo padres).


\[
\begin{pmatrix}
X\\Y
\end{pmatrix} \equiv N_2\left( \begin{pmatrix} µ_x \\ µ_y \end{pmatrix}, \begin{pmatrix}
σ_x^2&σ_{xy}\\σ_{xy}&σ_y^2
\end{pmatrix} \right)
\]
\label{form::EspVarCondicionada}
Definimos $\gor{Y} = E(Y|X) = µ_y + \frac{σ_{xy}}{σ_x^2}(x-µ_x)$. La esperanza de la altura del hijo condicionada a la altura del padre será la media de las alturas de los hijos corregida por un factor en el que influye la diferencia de altura del padre con respecto a su media. Es de esperar que si Yao Ming tiene un hijo, sea más alto que la media.

El factor de corrección $\frac{σ_{xy}}{σ_x^2}$ es importante y no me he enerado bien de dónde sale.

Ahora vamos a calcular $V(Y|X) = σ_{y}^2 - \frac{σ_{xy}^2}{σ_x^2} = σ_y^2 \left( 1- \rho^2\right)$ donde $\rho = \frac{σ_{xy}^2}{σ_x^2σ_y^2}$, el coeficiente de correlación.

Ha dicho algo así como \textbf{La única relación que puede existir entre 2 variables normales es una relación lineal.}


Este coeficiente de correlación aparece también en la expresión de la esperanza. Vamos a verlo:

 \[\gor{Y} = µ_y + \frac{σ_{xy}}{σ_x^2}(x-µ_x) \dimplies \frac{\gor{Y}-µ_y}{σ_y} = \frac{σ_{xy}}{σ_xσ_y}\frac{x-µ_x}{σ_x}\]

 Es decir:

 \[
\frac{\gor{Y}-µ_y}{σ_y} = \rho \frac{x-µ_x}{σ_x}
 \]

Aplicado a la estatura de los hijos respecto de los padres, se interpreta como: ``Si un padre es muy alto, su hijo será alto pero no destacará tanto como el padre''. Este fenómeno lo definió como \concept{Regresión a la mediocridad}. 

\end{example}

\begin{defn}[Homocedástico]
$Σ_{2.1}$ no depende de $X_1$.

Esto se da cuando $\begin{pmatrix}X_1,X_2\end{pmatrix}$ es normal multivariante. Si no fueran normal multivariante, serían heterocedásticas. \footnote{Un ejemplo sería $X_1$ la renta de una familia y $X_2$ los ahorros de la misma. Los datos no se distribuyen conjuntamente normal, con lo que la $Σ_{2.1}$ si depende de $X_1$. Ya veremos más adelante este concepto con mayor detalle.}
\end{defn}


\begin{example}

Ahora vamos a ver un par de ejemplos numéricos:

Sea \[\begin{pmatrix}X,Y\end{pmatrix} \equiv N_2 \left( \begin{pmatrix}0,0\end{pmatrix}, \begin{pmatrix}10&3\\3&1\end{pmatrix} \right)\]
 
\paragraph{Distribución $Y|X$:}

\[E(Y|X) = \frac{3}{10}x\]
\[V(Y|X) = \frac{1}{10}\]

\paragraph{Distribución $X|Y$:}

\[E(X|Y) = 3y\]
\[V(X|Y) = 1\]

Ambas son normales unidimensionales ya que $(X Y)$ es normal multivariante.

Sea \[\begin{pmatrix}X,Y\end{pmatrix} \equiv N_2 ...\]

Sea $Z_1 = X+Y$ y $Z_2 = X-Y$.

\[
\begin{pmatrix}Z_1//Z_2\end{pmatrix} = \begin{pmatrix}1&1\\1&-1\end{pmatrix}\begin{pmatrix}X\\Y\end{pmatrix} \implies \begin{pmatrix}Z_1\\Z_2\end{pmatrix} = N_2\left(\begin{pmatrix}2\\0\end{pmatrix},\begin{pmatrix}7&1\\1&3\end{pmatrix}\right)
\]

Ahora vamos a calcular lo que nos piden: $E(Z_1|Z_2=1)$.

\[E(Z_1|Z_2=1) = 2 + \frac{1}{3}(1-0) = \frac{7}{3}\]

Es importante destacar que la distribución no depende del valor concreto por ser homocedásticas .

\end{example}

\appendix
\chapter{Ejercicios}
% -*- root: ../EstadisticaII.tex -*-
\section{Hoja 1}

\begin{problem}[1]
Sea $Y = (Y_1,Y_2,Y_3)' ≡ N_3(µ,Σ)$, donde \[µ = (0,0,0)'\;
Σ =\begin{pmatrix}
1&0&0\\
0&2&−1\\
0&−1&2
\end{pmatrix}
\]


\ppart  Calcula la distribución del vector $(X_1,X_2)$, donde $X_1 = Y_1 + Y_3$ y $X_2 = Y_2 + Y_3$.
\ppart ¿Existe alguna combinación lineal de las variables aleatorias $Y_i$ que sea independiente de $X_1$?

\solution

\spart 
\[
\begin{pmatrix}X_1 \\ X_2 \end{pmatrix} = \begin{pmatrix} Y_1 + Y_3 \\ Y_2 + Y_3 \end{pmatrix} = \begin{pmatrix} 1&0&1\\0&1&1 \end{pmatrix} \begin{pmatrix} Y_1\\Y_2\\Y_3 \end{pmatrix} \equiv N_1\left( \begin{pmatrix}0\\0 \end{pmatrix},\begin{pmatrix}3&1\\1&2\end{pmatrix} \right)
\]

\spart 
\[
\begin{pmatrix} Ay\\By \end{pmatrix} = \begin{pmatrix} A\\B \end{pmatrix} Y \equiv N_{q+r} \left( \begin{pmatrix} Aμ\\Bμ \end{pmatrix},\begin{pmatrix} A\\B \end{pmatrix} Σ(A',B') \right)
\]

Entonces \[cov\left(a'y,(1,0,1)y\right) = (a_1,a_2,a_3) \begin{pmatrix} 1&0&0\\0&2&-1\\0&-1&2\end{pmatrix} \begin{pmatrix} 1&0&1 \end{pmatrix}\]

\end{problem}


\begin{problem}[2]

\solution

\end{problem}

\begin{problem}[3]


\solution

\end{problem}

\begin{problem}[5]

Calcula la distribución condicionada de $X$ dado $Y$ = $y$, y la de $Y$ dado $X$ = $x$.

\solution


\[
\begin{pmatrix}X\\Y \end{pmatrix} \equiv N_2\left(\begin{pmatrix}0\\0\end{pmatrix},\begin{pmatrix}1&-1\\-1&2\end{pmatrix}^{-1}\right)
\]

Aplicando las fórmulas vistas en teoría \ref{form::EspVarCondicionada}

\[
E(X|Y=y) = μ_y + Σ_{21}Σ_{11}^{-1}(X-μ_x) = 0 + \frac{1}{1}(y-0) = y
\]
\[
E(Y|X=x) = μ_x + Σ_{21}Σ_{11}^{-1}(Y-μ_y) = 0 + \frac{1}{2}(x-0) = \frac{x}{2}
\]

\end{problem}

\begin{problem}[7]
Sea $X = (X1,X2,X3)'$ un vector aleatorio con distribución normal tridimensional con vector de medias $(0,0,0)'$ y matriz de covarianzas
\[
Σ =
\begin{pmatrix}
1&2&−1\\
2&6&0\\
−1&0&4
\end{pmatrix}
\]


Definamos las v.a. $Y_1 = X_1 + X_3, Y_2 = 2X_1 − X_2 e Y_3 = 2X_3 − X_2$. Calcula la distribución de $Y_3$ dado que $Y_1=0$ e $Y_2=1$.

\solution

Lo primero es descubrir la matriz de la combinación lineal, esto es:
\[
\begin{pmatrix} Y_1\\Y_2\\Y_3\end{pmatrix} = \begin{pmatrix}1&0&1\\2&-1&0\\0&-1&2\end{pmatrix}\begin{pmatrix}X_1\\X_2\\X_3\end{pmatrix} \equiv N_3 \left( \begin{pmatrix}0\\0\\0\end{pmatrix}, \begin{pmatrix}3&-2&4\\-2&2&-2\\4&-2&22 \end{pmatrix} \right)
 \]

Llamamos \[A=\begin{pmatrix}3&-2&4\\-2&2&-2\\4&-2&22 \end{pmatrix}\]

¿De dónde sale esta matriz? Elena opina (y Jorge lo confirma) que $A = ΣBΣ'$, donde $B$ es la matriz de la combinación lineal, es decir: 
\[B=\begin{pmatrix}1&0&1\\2&-1&0\\0&-1&2\end{pmatrix}\]

\[
E(Y_3|y_1 = 0, y_2 = 1) = 0 + (4,-2) \begin{pmatrix} 3&-2\\-2&2\end{pmatrix}^{-1} \begin{pmatrix}0-0\\1-0\end{pmatrix} = ... = 1
\]

\[
V(Y_3|y_1=0,y_2=1) = 22 - (4,-2) \begin{pmatrix} 3&-2\\-2&2 \end{pmatrix}^{-1} \begin{pmatrix}4\\-2\end{pmatrix} = ... = 16
\]


Entonces, la distribución de $Y = (Y_1,Y_2,Y_3)' = N_3(1,16)$

\end{problem}

\section{Hoja 2}


\begin{problem}[1] Calcula la distribución exacta bajo la hipótesis nula del estadístico de Kolmogorov-Smirnov para muestras de tamaño 1.

\solution

La hipótesis sería $H_0 : F = F_0$ continua, con $X \sim F$

En este caso,

\[D=||F_1 - F_0||_{\inf} = (1) = \max\{F_0(x), 1 - F_0(x)\}\]

$(1)$ hay 2 posibles caminos. Al dibujar lo que nos dicen (una muestra de tamaño 1) podemos sacarlo por intuición. Sino, aplicamos la fórmula de los estadísticos.

Ahora calculamoms:

\[ P_{F_0}(D\leq x) = P_{F_0} = \left\{\max \{ ... \}\leq x\right\} = P_{F_0} = P_{F_0}\{ 1-x \leq F_0(x) \leq x \}\]

Resolvemos la desigualdad, aplicando que $F_0$ es una uniforme.

\[
P\{1-x \leq U \leq x\} = \left\{ \begin{array}{cc} 0 & x\leq \frac{1}{2} \\ 2x-1 & x\geq \frac{1}{2}\end{array} \right. \implies D \sim \mathcal{U}\left(\frac{1}{2},1\right)
\]

Ya que $1-x > x \dimplies x\le \frac{1}{2}$

\end{problem}
\begin{problem}[2] Se desea contrastar la hipótesis nula de que una única observación X procede de una distribución N(0,1). Si se utiliza para ello el contraste de Kolmogorov-Smirnov, determina para qué valores de X se rechaza la hipótesis nula a nivel α = 0,05.
\solution

Este ejercicio está muy relacionado con el primero. Es una aplicación al caso de la normal.


Mirando en la tabla, encontramos que para $α = 0.05$, entonces $d_α = 0.975$. Con esta inormación podemos construir la región crítica:
\[ R = \left\{\max\{\Phi(x), 1 - \Phi(x))\} > 0.975\right\} = \{\Phi(x) > 0.975\} \cup \{1 - \Phi(x) > 0.975\} =\]
\[ \{ X>\Phi^{-1}(0.975)\} \cup \{X < \Phi^{-1}(0.025)\}\]

Consultando las tablas, vemos que $\Phi^{-1}(0.975) = 1.96$ y por simetría, $\Phi^{-1}(0.025) = -1.96$

\[R = \{|X| > 1.96\}\]


\obs Es interesante saber que, al ser simétrica la normal, la interpretación gráfica es muy fácil. Si dividimos la normal en 3 intervalos, $(-∞ , -1.96) , (-1.96,1.96) , (1.96, ∞)$, el área encerrada en las colas es el nivel de significación, en este caso: \[\text{Area }\left((-∞ , -1.96)\cup (1.96, ∞)\right) = 0.05\]

\end{problem}
\begin{problem}[3] Da una demostración directa para el caso k = 2 de que la distribución del estadístico del contrast $\chi^2$ de bondad de ajuste converge a una distribución $\chi_1^2$ , es decir,
\[
T = \frac{(O1 − E1)^2}{E1} +
\frac{(O2 − E2)^2}{E2} \convs[d] \chi_1^2\]

\label{ej::2.3}

[Indicación: Hay que demostrar que $T = X^2_n$ , donde $X_n\convs[d] N(0,1)$. Para reducir los dos sumandos a uno, utilizar la relación existente entre O1, E1 y O2, E2.]
\solution

Si tenemos $n$ datos, vamos a construir la tabla de contingencia. Creo que consideramos una binomial porque, al sólo tener 2 clases, o eres de una o eres de la otra con una probabilidad $p$.

\begin{center}
\begin{tabular}{c|cc}
 & $A_1$ & $A_2$ \\\hline
 Obs & $n\gor{p}$ & $n(1-\gor{p})$\\
 Esp  & $np_0$ & $n(1-p_0)$\\
\end{tabular}
\end{center}

\[ T = \sum_{i=1}^2 \frac{(O_i - E_i)^2}{E_i} = \frac{n^2(\gor{p}-p_0)^2}{n} + \frac{n^2(\gor{p}-p_0)}{n(1-p_0)}  = ... \]
Simplificando, llegamos a:

\[
T = \left(\frac{|\gor{p}-p_0|}{\sqrt{\frac{p_0(1-p_0)}{n}}} \right)
\]

Está contando un montón de cosas interesantes que me estoy perdiendo.



Entre ellas, tenemos que $\sqrt{T} \convs[d]N(0.1)$ por el teorema central del límite ( es el caso particular para una binomial), con lo que $T\convs[d] \chi^2$. ¿Porqué 1 grado de libertad? Porque sólo estamos estimando 1 parámetro, el $\gor{p}$.

Esto responde también al problema 11. 

\end{problem}
\begin{problem}[4] El número de asesinatos cometidos en Nueva Jersey cada día de la semana durante el año 2003 se muestra en la tabla siguiente:

\begin{center}
\begin{tabular}{c|ccccccc}
Día & Lunes & Martes & Miércoles & Jueves & Viernes & Sábado & Domingo \\\hline
Frecuencia & 42 & 51 & 45 & 36 & 37 & 65 & 53
\end{tabular}
\end{center}

\ppart Contrasta a nivel α = 0,05, mediante un test $χ2$, la hipótesis nula de que la probabilidad de que se cometa un asesinato es la misma todos los días de la semana.

\ppart ¿Podría utilizarse el test de Kolmogorov-Smirnov para contrastar la misma hipótesis? Si tu
respuesta es afirmativa, explica cómo. Si es negativa, explica la razón.


\ppart Contrasta la hipótesis nula de que la probabilidad de que se cometa un asesinato es la misma desde el lunes hasta el viernes, y también es la misma los dos días del fin de semana (pero no es necesariamente igual en fin de semana que de lunes a viernes).

\solution

\spart $n = 329$, $E_i = \frac{329}{7}$ y $H_0 : p_i = \frac{1}{7}$

Calculamos el estadístico $T = ... = \sum_{i=1}^7 ... = ... = 13.32$

Por otro lado, $\chi^2_{6;0.05} = 12.59$, con lo que rechazamos la hipótesis.

\spart No podría utilizarse al tratarse de algo discreto y KS sólo sirve para continuas.

\spart

Tenemos la siguiente tabla:

\begin{center}
\begin{tabular}{c|ccccccc}
Día & Lunes & Martes & Miércoles & Jueves & Viernes & Sábado & Domingo \\\hline
Frecuencia & p & p & p & p & p & q & q
\end{tabular}
\end{center}

Con $5p + 2q = 1 \implies q = \frac{11-5p}{2}$

Entonces, tenemos \[ e.m.v.(p) =L(p)= p^{42+51+...+37} \left( \frac{11-5p}{2} \right)^{65+53} \]

Ahora, despejamos tomando $l(p) = ln(L(p)) = 211 ln(p) + 118ln\left(\frac{11-5p}{2}\right)$ y maximizamos:

\[
l'(p) = 0 \implies ... \left\{\begin{array}{c} \gor{p} = 0.128\\ \gor{q} = 0.179 \end{array}\right.
\]


Ahora construimos el estadístico:

\[
T = \sum_{i=1}^7 \frac{O_i^2}{\gor{E}_i^2} - n = ... = 5.4628
\]

Y comparamos con la $\chi^2$. ¿Cuántos grados de libertad? Si tenemos $7$ clases, siempre perdemos uno, con lo que serían 6. Sin embargo hemos estimado un parámetro, con lo que son $5$ grados de libertad. Entonces: $ c = \chi^2_{5;0.05} = 11.07$

Como $T < c$, no podemos rechazar la hipótesis.

\obs
Podríamos plantearnos contrastar que es uniforme de lunes a viernes ($H_1$) y otra uniforme distinta en fines de semana ($H_2$). Entonces tendríamos $H_0 : H_1 \cap H_2$, y construir la región $R = R_1 \cup R_2$. ¿Cuál es el problema de este camino?

El nivel de significación, ya que $P_{H_0}(R_1 \cup R_2) = P_{H_0}(R_1) + P_{H_0}(R_2) - P_{H_0}(R_1\cap R_2) = 2α - α^2 \sim 2α$. 

Podríamos tomar, chapucerillamente $α = \frac{α}{2}$ para que al final, $P_{H_0} ( R_1 \cup R_2) = α$. Aquí surge otro problema, que es que estamos despreciando la probabilidad de la intersección y tomándolo como independiente cuando no tiene porqué serlo. Es una aproximación ``buena'' que a veces se utiliza, pero pudiendo hacerlo bien... 

\end{problem}



\begin{problem}[5] Para estudiar el número de ejemplares de cierta especie en peligro de extinción que viven en un
bosque, se divide el mapa del bosque en nueve zonas y se cuenta el número de ejemplares de cada
zona. Se observa que 60 ejemplares viven en el bosque repartidos en las 9 zonas de la siguiente
forma:


\begin{center}
\begin{tabular}{|c|c|c|}
\hline
8&7&3 \\\hline
5&9&11 \\\hline
6&4&7 \\\hline
\end{tabular}
\end{center}

Mediante un contraste de hipótesis, analiza si estos datos aportan evidencia empírica de que los
animales tienen tendencia a ocupar unas zonas del bosque más que otras.

Tomamos $α = 0.01$
\solution

$T = 7.47$, $\chi^2_{8;0.001} = 20.09$

Aceptamos la hipótesis $H_0 : $ la especie se reparte uniformemente.

\end{problem}
\begin{problem}[6] Se ha desarrollado un modelo teórico para las diferentes clases de una variedad de moscas. Este
modelo nos dice que la mosca puede ser de tipo L con probabilidad p
2
, de tipo M con probabilidad
q
2 y de tipo N con probabilidad 2pq (p + q = 1). Para confirmar el modelo experimentalmente
tomamos una muestra de 100 moscas, obteniendo 10, 50 y 40, respectivamente.
\ppart
Hallar la estimación de máxima verosimilitud de p con los datos obtenidos.
\ppart
¿Se ajustan los datos al modelo teórico, al nivel de significación 0’05?
\solution

\doneby{Jorge}

\spart
Primero calculamos la función de verosimilitud para $p$:
\[L_n(p) = L_n(p) = \prod_{i=0}^n f(x_i;p) = (p^2)^{10} · (q^2)^{50} · (2pq)^{40}\]

El EMV lo obtendremos maximizando $\log L_n(p)$:
\[\log L_n(p) = 20 \log p + 100 \log q + 40 \log 2pq\]
\[\frac{\partial}{\partial p} \log L_n(p) = \frac{20}{p} - \frac{100}{1-p} + 40 \frac{2-4p}{2p(1-p)} = 0 \]

Maximizamos con $\hat{p}=\frac{3}{10} \implies \hat{q}=\frac{7}{10}$.

\spart
En este caso tomamos $H_0 \equiv P(X∈L)=p^2, P(X∈M)=q^2, P(X∈N)=2pq$

Usando el estado el contraste de bondad de ajuste de la $χ^2$, el estadístico de Pearson queda:
\[T = \sum_{i=1}^3 \frac{\left(O_i - \hat{E}_i\right)^2}{\hat{E}_i} = \sum_{i=1}^3 \frac{O_i^2}{\hat{E}_i} - n =\]
\[ = \frac{10^2}{p^2·100} + \frac{50^2}{(1-p)^2 · 100} + \frac{40^2}{2p(1-p)·100} - 100 ≈ 0.22\]

Puesto que en este caso $k=3$ y hemos estimado 1 parámetro ($p$), tenemos que $T$ se distribuye como una $χ^2_{3-1-1}$. En las tablas nos encontramos con que $χ^2_{1;0.05}=3.84 > T$ y no rechazamos $H_0$, es decir los datos se ajustan al modelo teórico.

\end{problem}
\begin{problem}[7]
\ppart
Aplica el test de Kolmogorov-Smirnov, al nivel 0.05, para contrastar si la muestra (3.5, 4, 5, 5.2, 6) procede de la $U(3,8)$.
\ppart
Aplica el test de Kolmogorov-Smirnov, al nivel 0.05, para contrastar la hipótesis de que la
muestra (0, 1.2, 3.6) procede de la distribución $N(µ~=~1;σ~=~5)$.
\solution
\doneby{Jorge}

\spart
La función de distribución de una $U(3,8)$ es:
\[
	F(x)=
	\begin{cases}
		0 & ,x<3 \\
		\frac{x-3}{5} & ,3≤x≤8 \\
		1 & ,x>8
	\end{cases}
\]

\begin{center}
	\begin{tabular}{ c | c | c | c | c }	
		$x_{(i)}$ & $\frac{i}{n}$ & $F_0(x_{(i)})$ & $D_n^{+}$ & $D_n^{-}$ \\ \hline
		3.5 & 0.2  & 0.1  & 0.1  & 0.1   \\ \hline
		4   & 0.4  & 0.2  & 0.2  & 0     \\ \hline
		5   & 0.6  & 0.4  & 0.2  & 0     \\ \hline
		5.2 & 0.8  & 0.44 & 0.36 & -0.16 \\ \hline
		6   & 1    & 0.6  & 0.4  & -0.2
	\end{tabular}
\end{center}

Tendremos por tanto que $D_n=\norm{F_n-F_0}_∞=0.4$. Si nos vamos a la tabla del contraste K-S vemos que $c=0.565$ para $α=0.05$.

Como $D_n<c$ \textbf{no rechazamos} la hipótesis nula de que las muestras vienen de la uniforme.

\spart
\begin{center}
	\begin{tabular}{ c | c | c | c | c }	
		$x_{(i)}$ & $\frac{i}{n}$ & $F_0(x_{(i)})$ & $D_n^{+}$ & $D_n^{-}$ \\ \hline
		0 & 0.3 & 0.42 & -0.12 & 0.42   \\ \hline
		1.2 & 0.6 & 0.52 & 0.08 & 0.22     \\ \hline
		3.6 & 1 & 0.7 & 0.3 & 0.1
	\end{tabular}
\end{center}

Tendremos por tanto que $D_n=\norm{F_n-F_0}_∞=0.42$. Si nos vamos a la tabla del contraste K-S vemos que $c=0.708$ para $α=0.05$.

Como $D_n<c$ \textbf{no rechazamos} la hipótesis nula de que las muestras vienen de la $N(1,5)$.

\end{problem}




\begin{problem}[8] Se ha clasificado una muestra aleatoria de 500 hogares de acuerdo con su situación en la ciudad
(Sur o Norte) y su nivel de renta (en miles de euros) con los siguientes resultados:
\begin{center}
	\begin{tabular}{c c c}
		\hline
		Renta & Sur & Norte\\ \hline
		0 a 10 & 42 & 53 \\
		10 a 20 & 55 & 90 \\
		20 a 30 & 47 & 88 \\
		más de 30 & 36 & 89\\ \hline
	\end{tabular}
\end{center}

\ppart
A partir de los datos anteriores, contrasta a nivel α = 0,05 la hipótesis nula de que en el sur los
hogares se distribuyen uniformemente en los cuatro intervalos de renta considerados.

\ppart
A partir de los datos anteriores, ¿podemos afirmar a nivel α = 0,05 que la renta de los hogares
es independiente de su situación en la ciudad?
\solution


\spart
Tenemos $H_0: p_i=\frac{1}{4}$ y usando el contraste de bondad de ajuste de la $χ^2$:
\[T = \sum_{i=1}^4 \frac{O_i^2}{E_i} - n_{\text{sur}} = \frac{42^2 + 55^2 + 47^2 + 36^2}{\frac{1}{4}·180} - 180 = 4.31\]

En las tablas encontramos que $χ^2_{k-1;α} = χ^2_{3;0.05} = 7.815$. Como $T<χ^2_{3;0.05}$, \textbf{no podemos rechazar} la hipótesis nula de que en el sur los hogares se distribuyen uniformemente en los cuatro intervalos de renta considerados.

\spart
Lo primero que haremos es estimar las probabilidades de que la v.a. caiga en cada una de las 6 clases que tenemos ($A_i$ serán los intervalos de renta y $B_i$ si el hogar es del norte o del sur):
\[p(x∈A_1) = \frac{42+53}{500} = 0.19\]
\[p(x∈A_2) = \frac{55+90}{500} = 0.29\]
\[p(x∈A_3) = \frac{47+88}{500} = 0.27\]
\[p(x∈A_4) = \frac{36+89}{500} = 0.25\]

\[p(x∈B_1) = \frac{42+55+47+36}{500} = 0.36\]
\[p(x∈B_2) = \frac{53+90+88+89}{500} = 0.64\]

Bajo la $H_0$ consideramos $A_i$ independiente de $B_i$, de modo que $p_{i,j} = p_i·p_j$ tal y como se muestra en la siguiente tabla:

\begin{center}
	\begin{tabular}{c | c}
		$p_{1,1} = 0.0684$ & $p_{1,2} = 0.1216$\\ \hline
		$p_{2,1} = 0.1044$ & $p_{2,2} = 0.1856$\\ \hline
		$p_{3,1} = 0.0972$ & $p_{3,2} = 0.1728$\\ \hline
		$p_{4,1} = 0.09$ & $p_{4,2} = 0.16$
	\end{tabular}
\end{center}

Sabiendo que $\hat{E}_{ij} = n·p_{i,j}$:

\begin{center}
	\begin{tabular}{c | c}
		$\hat{E}_{1,1} = 34.2$ & $\hat{E}_{1,2} = 60.8$\\ \hline
		$\hat{E}_{2,1} = 52.2$ & $\hat{E}_{2,2} = 92.8$\\ \hline
		$\hat{E}_{3,1} = 48.6$ & $\hat{E}_{3,2} = 86.4$\\ \hline
		$\hat{E}_{4,1} = 45$ & $\hat{E}_{4,2} = 80$
	\end{tabular}
\end{center}

\[T=\sum_{j=1}^2 \sum_{i=1}^4 \frac{O_{ij}^2}{\hat{E}_{ij}} - n = 8.39\]

Si nos vamos a las tablas vemos que $χ^2_{(k-1)(p-1); α} = χ^2_{3·1; 0.05} = 7.815 < T$ y por tanto \textbf{rechazamos la hipótesis nula} de que la renta de los hogares es independiente de su situación en la ciudad.


\end{problem}
\begin{problem}[9] A finales del siglo XIX el físico norteamericano Newbold descubrió que la proporción de datos
que empiezan por una cifra d, p(d), en listas de datos correspondientes a muchos fenómenos
naturales y demográficos es aproximadamente:
p(d) = log10
d + 1
d
!
, d = 1,2,...,9.
Por ejemplo, p(1) = log10 2 ≈ 0,301030 es la frecuencia relativa de datos que empiezan por 1. A raíz
de un artículo publicado en 1938 por Benford, la fórmula anterior se conoce como ley de Benford.
El fichero poblacion.RData incluye un fichero llamado poblaciones con la población total de los
municipios españoles, así como su población de hombres y de mujeres.
(a) Contrasta a nivel α = 0,05 la hipótesis nula de que la población total se ajusta a la ley de Benford.
(b) Repite el ejercicio pero considerando sólo los municipios de más de 1000 habitantes.
(c) Considera las poblaciones totales (de los municipios con 10 o más habitantes) y contrasta a nivel
α = 0,05 la hipótesis nula de que el primer dígito es independiente del segundo.
(Indicación: Puedes utilizar, si te sirven de ayuda, las funciones del fichero benford.R).
\solution

\end{problem}
\begin{problem}[10] Se ha llevado a cabo una encuesta a 100 hombres y 100 mujeres sobre su intención de voto. De
las 100 mujeres, 34 quieren votar al partido A y 66 al partido B. De los 100 hombres, 50 quieren
votar al partido A y 50 al partido B.
\ppart
Utiliza un contraste basado en la distribución $χ^2$ para determinar si con estos datos se puede
afirmar a nivel $α = 0,05$ que el sexo es independiente de la intención de voto.
\ppart
Determina el intervalo de valores de α para los que la hipótesis de independencia se puede
rechazar con el contraste del apartado anterior.
\solution

Este ejercicio ha caido en un examen.

\doneby{Jorge}

\spart
Procediendo como en el ejercicio anterior obtendremos que bajo la hipótesis nula de independencia:
\[p_{A,\text{mujer}} = p_{A, \text{hombre}} = 0.21\]
\[p_{B,\text{mujer}} = p_{B, \text{hombre}} = 0.29\]

Por tanto:
\[T=\sum_{j=1}^2 \sum_{i=1}^2 \frac{O_{ij}^2}{\hat{E}_{ij}} = 5.25\]

Si nos vamos a las tablas vemos que $χ^2_{(k-1)(p-1); α} = χ^2_{1; 0.05} = 3.841 < T$, y por tanto \textbf{rechazamos la hipótesis nula} de que el sexo es independiente de la intención de voto.

\paragraph*{En clase: } hemos contrastado homogeneidad (las intenciones de voto se distribuyen igual) en vez de independencia, pero viene a ser lo mismo.

\spart
El p-valor asociado a $T=5.25$ es $\left[1 - F_{χ^2_{1}}(5.25)\right] = 0.02$, por tanto para $α~∈~[0.02,1]$ rechazamos la hipótesis de independencia del apartado anterior.

Para calcular el p-valor, utilizamos que una $\chi^2_1$ es una normal al cuadrado, es decir:

\[p = P(X>5.25) = P ( Z^2 > 5.25) = P(|Z| > 2.29) = 0.022\]

siendo $Z\sim N(0,1)$



\end{problem}
\begin{problem}[11] Sea X1,...,Xn una muestra de una distribución Bin(1, p). Se desea contrastar H0 : p = p0. Para ello hay dos posibilidades: 

\ppart  Un contraste de proporciones basado en la región crítica
$R = \{|\gor{p} − p_0|\} > z\frac{α}{2} p p0(1 − p0)/n $
\ppart un contraste $χ2$ de bondad de ajuste con k = 2 clases. ¿Cuál es la relación entre ambos contrastes?
\solution

Consultar el ejercicio \ref{ej::2.3}.

\end{problem}
\begin{problem}[12] En un estudio de simulación se han generado 10000 muestras aleatorias de tamaño 10 de una
distribución $N(0,1)$. Para cada una de ellas se ha calculado con R el estadístico de Kolmogorov-Smirnov
para contrastar la hipótesis nula de que los datos proceden de una distribución normal
estándar, y el correspondiente p-valor.
\ppart
Determina un valor x tal que la proporción de estadísticos de Kolmogorov-Smirnov mayores
que x, entre los 10000 obtenidos, sea aproximadamente igual a 0.05. ¿Cuál es el valor teórico al que
se debe aproximar la proporción de p-valores menores que 0.1 entre los 10000 p-valores obtenidos?
\ppart
¿Cómo cambian los resultados del apartado anterior si en lugar de considerar la distribución
normal estándar se considera una distribución uniforme en el intervalo (0,1)?
\solution

\doneby{Jorge}

\spart
\begin{itemize}
	\item La $x$ que nos piden es $f_{D,α=0.05}$ ($f_D$ es la función de densidad del estadístico K-S). Si acudimos a la tabla vemos que para $n=10$ $x = f_{D,0.05} = 0.41$. Un poco más explicado el razonamiento:
	\[
		\underbrace{\frac{\#\{i : D_i > x\}}{10000}}_{P(D>x)} \simeq 0.05
	\]

	\item Precisamente el $10\%$ de los p-valores debería ser menor que $0.1$, ya que hacer un contraste nivel de significación $α=0.1$ significa que en el $10\%$ de los casos rechazamos la hipótesis nula, es decir, en le $10\%$ de los casos los p-valores son $<0.1$.

	Esto se debe al concepto de nivel de significación, ya que si el nivel de significación es $0.01$, entonces nos estamos equivocando en 1 de cada 100 contrastes que hagamos, es decir:

	\[
		\frac{\# \{ i : p^{(i)} < α\}}{10000} \simeq α
	\]
\end{itemize}

\spart
\begin{itemize}
	\item Al contrastar con una distribución $U(0,1)$ cabría esperar que las $1000$ $D_i$ tomaran valores más altos, pues la distancia entre $F_n$ (que se monta a partir de datos que vienen de una $N(0,1)$) y $F_0=F_{U(0,1)}$ sería más grande que al tomar como $F_0$ la de una $N(0,1)$. Por tanto \textbf{el valor $x$ debería ser mayor}.

	\item Por otra parte la proporción de p-valores menores que $0.1$ debería aumentar, ya que el test debería devolver p-valores más pequeños (pues debería de rechazar la hipótesis de que los datos vienen de una $U(0,1)$).
\end{itemize}

\paragraph{Solución de clase:}

Al tener muchas muchas muestras, las frecuencias deberían ser las probabilidades.

\end{problem}


%%%%%%%%%%%%
%% HOJA 3 %%
%%%%%%%%%%%%

\section{Hoja 3}
\begin{problem}[1] La Comunidad de Madrid evalúa anualmente a los alumnos de sexto de primaria de todos los colegios sobre varias materias. Con las notas obtenidas por los colegios en los años 2009 y 2010 (fuente: diario El País) se ha ajustado el modelo de regresión simple:
\[Nota2010 = β_0 + β_1Nota2009 + ε,\]
en el que se supone que la variable de error ε verifica las hipótesis habituales. Los resultados
obtenidos con R fueron los siguientes:\\[1em]

Coefficients:

\begin{tabular}{c | c | c | c | c}
	~ & Estimate & Std. Error & t value & Pr(>|t|) \\
	(Intercept) & 1.40698 & 0.18832 & 7.471 & 1.51e-13 \\
	nota09 & 0.61060  & 0.02817 & 21.676 & < 2e-16
\end{tabular}

 Residual standard error: 1.016 on 1220 degrees of freedom

 Multiple R-squared: 0.278,Adjusted R-squared: 0.2774

 F-statistic: 469.8 on 1 and 1220 DF,  p-value: < 2.2e-16 \\[1em]

También se sabe que en 2009 la nota media de todos los colegios fue 6,60 y la cuasidesviación típica fue 1,03 mientras que en 2010 la media y la cuasidesviación típica fueron 5,44 y 1,19, respectivamente.

\ppart ¿Se puede afirmar a nivel $α = 0,05$ que existe relación lineal entre la nota de 2009 y la de 2010? Calcula el coeficiente de correlación lineal entre las notas de ambos años.

\ppart Calcula un intervalo de confianza de nivel $95\%$ para el parámetro $β_1$ del modelo.

\ppart Calcula, a partir de los datos anteriores, un intervalo de confianza de nivel $95\%$ para la nota
media en 2010 de los colegios que obtuvieron un 7 en 2009.


\solution
\end{problem}




\chapter{Prácticas}
\includepdf[pages={1-5}]{pdf/_practica1E2.pdf}

\end{document}