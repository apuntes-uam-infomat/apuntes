\documentclass[nochap]{apuntes}

\usepackage{hyperref}

\usepackage{tikztools}
\usepackage{fastbuild}
\usepackage{tikz-3dplot}

\usepackage{tikz}
\usepackage{graphicx}
\usepackage{latexsym, amsfonts, amsmath, amssymb, amscd, epsfig,amsthm}
\input xy
\xyoption{all} %%!!
\usetikzlibrary{calc, intersections}
\author{Alberto Parramón}
\date{2014/2015 2º cuatrimestre}

\renewcommand*{\arraystretch}{1.5}
\title{Estadística II}
\precompileTikz

\begin{document}

\pagestyle{plain}
\maketitle

\tableofcontents
\newpage

\section{Parte que tiene que pasar Parra}

\subsection{Distribuciones condicionadas}

\begin{prop}

Sea $X=(X_1|X_2)$ con $X_1∈ℝ^p$ y $X_2∈ℝ^{p-q}$. Consideramos las particiones correspondientes de $µ$ y de $\Sigma$.

\end{prop}

\begin{proof}
Definimos $X_{2.1} = X_2 - Σ_{21}Σ_{11}^{-1}X_1$.

\[
\begin{pmatrix}
X_1\\
X_{2.1}
\end{pmatrix} = 
\begin{pmatrix}
I &| &0\\
\hline
- Σ_{21}Σ_{11}^{-1}  &| &I
 \end{pmatrix}
\]

Como es una combinación lineal de $(X_1,X_{2.1})'$, entonces $X_{2.1}$ es normal multivariante.

Vamos a calcular la media y la matriz de covarianzas de $X_{2.1}$

$X_{2-1} = N\left( µ_2-Σ_{21}Σ_{11}^{-1}µ_1 , \begin{pmatrix} Σ_{11} &|&0\\\hline 0&|&Σ_{2.1} \end{pmatrix} \right)$

Donde las covarianzas se calculan: $AΣA'$, siendo $A$ la matriz de la combinación lineal, es decir:

\[
A=\begin{pmatrix}
I &| &0\\
\hline
- Σ_{21}Σ_{11}^{-1}  &| &I
 \end{pmatrix}
\]



\paragraph{Conclusiones:}

\begin{itemize}
	\item $X_1$ es independiende de $X_{2.1}$
	\item $X_{2.1}$ es normal, con media y varianza calculadas anteriormente.
	\subitem $X_{2.1}|X_1$, al ser independientes, también se distribuye normalmente, con los mismos parámetros.
	\item Dado $X_1$, los vectores $X_{2.1}$ y $X_2$  difieren en el vector constante $Σ_{21}Σ_{11}^{-1}X_1 \implies X_2|X_1 = N\left( µ_{2.1}, Σ_{2.1} \right)$
\end{itemize}

\end{proof}

\begin{example}
Vamos a considerar $X_1, X_2$ como escalares, para entender la proposición. Este ejemplo le surgió a un investigador que quería predecir la estatura de los hijos en función de la de los padres (que no padres y madres, sólo padres).


\[
\begin{pmatrix}
X\\Y
\end{pmatrix} \equiv N_2\left( \begin{pmatrix} µ_x \\ µ_y \end{pmatrix}, \begin{pmatrix}
σ_x^2&σ_{xy}\\σ_{xy}&σ_y^2
\end{pmatrix} \right)
\]

Definimos $\gor{Y} = E(Y|X) = µ_y + \frac{σ_{xy}}{σ_x^2}(x-µ_x)$. La esperanza de la altura del hijo condicionada a la altura del padre será la media de las alturas de los hijos corregida por un factor en el que influye la diferencia de altura del padre con respecto a su media. Es de esperar que si Yao Ming tiene un hijo, sea más alto que la media.

El factor de corrección $\frac{σ_{xy}}{σ_x^2}$ es importante y no me he enerado bien de dónde sale.

Ahora vamos a calcular $V(Y|X) = σ_{y}^2 - \frac{σ_{xy}^2}{σ_x^2} = σ_y^2 \left( 1- \rho^2\right)$ donde $\rho = \frac{σ_{xy}^2}{σ_x^2σ_y^2}$, el coeficiente de correlación.

Ha dicho algo así como \textbf{La única relación que puede existir entre 2 variables normales es una relación lineal.}


Este coeficiente de correlación aparece también en la expresión de la esperanza. Vamos a verlo:

 \[\gor{Y} = µ_y + \frac{σ_{xy}}{σ_x^2}(x-µ_x) \dimplies \frac{\gor{Y}-µ_y}{σ_y} = \frac{σ_{xy}}{σ_xσ_y}\frac{x-µ_x}{σ_x}\]

 Es decir:

 \[
\frac{\gor{Y}-µ_y}{σ_y} = \rho \frac{x-µ_x}{σ_x}
 \]

Aplicado a la estatura de los hijos respecto de los padres, se interpreta como: ``Si un padre es muy alto, su hijo será alto pero no destacará tanto como el padre''. Este fenómeno lo definió como \concept{Regresión a la mediocridad}. 

\end{example}

\begin{defn}[Homocedástico]
$Σ_{2.1}$ no depende de $X_1$.

Esto se da cuando $\begin{pmatrix}X_1,X_2\end{pmatrix}$ es normal multivariante. Si no fueran normal multivariante, serían heterocedásticas. \footnote{Un ejemplo sería $X_1$ la renta de una familia y $X_2$ los ahorros de la misma. Los datos no se distribuyen conjuntamente normal, con lo que la $Σ_{2.1}$ si depende de $X_1$. Ya veremos más adelante este concepto con mayor detalle.}
\end{defn}


\begin{example}

Ahora vamos a ver un par de ejemplos numéricos:

Sea \[\begin{pmatrix}X,Y\end{pmatrix} \equiv N_2 \left( \begin{pmatrix}0,0\end{pmatrix}, \begin{pmatrix}10&3\\3&1\end{pmatrix} \right)\]
 
\paragraph{Distribución $Y|X$:}

\[E(Y|X) = \frac{3}{10}x\]
\[V(Y|X) = \frac{1}{10}\]

\paragraph{Distribución $X|Y$:}

\[E(X|Y) = 3y\]
\[V(X|Y) = 1\]

Ambas son normales unidimensionales ya que $(X Y)$ es normal multivariante.

Sea \[\begin{pmatrix}X,Y\end{pmatrix} \equiv N_2 ...\]

Sea $Z_1 = X+Y$ y $Z_2 = X-Y$.

\[
\begin{pmatrix}Z_1//Z_2\end{pmatrix} = \begin{pmatrix}1&1\\1&-1\end{pmatrix}\begin{pmatrix}X\\Y\end{pmatrix} \implies \begin{pmatrix}Z_1\\Z_2\end{pmatrix} = N_2\left(\begin{pmatrix}2\\0\end{pmatrix},\begin{pmatrix}7&1\\1&3\end{pmatrix}\right)
\]

Ahora vamos a calcular lo que nos piden: $E(Z_1|Z_2=1)$.

\[E(Z_1|Z_2=1) = 2 + \frac{1}{3}(1-0) = \frac{7}{3}\]

Es importante destacar que la distribución no depende del valor concreto por ser homocedásticas.

\end{example}



\end{document}