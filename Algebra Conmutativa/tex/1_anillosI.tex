% -*- root: ../AlgebraConmutativa.tex -*-
\chapter{Anillos I: nociones básicas}

Durante este curso vamos a trabajar mucho con anillos. Recordemos conceptos básicos vistos en Estructuras Algebraicas \cite{apuntesEA}.


\begin{defn}[Grupo]
Un grupo es un conjunto, G, con una operación binaria $'\cdot'$, que compone dos elementos $a \cdot b \in G$ para formar otro elemento notado como $a \cdot b$ o $ab$. Se escribe: $(G,\cdot)$. Para ser considerado grupo debe satisfacer:
\begin{enumerate}
	\item Cerrado: $\forall a,b \in G$, entonces $a \cdot b \in G$
	\item Asociatividad: $\forall a,b,c \in G$, entonces $(a\cdot b) \cdot c = a \cdot (b \cdot c)$
	\item Elemento neutro: $\exists e \in G$ tal que $\forall a \in G$ se cumple que $e\cdot a=a\cdot e = a$. A este elemento neutro se le denomina como elemento identidad o $1_G$.
	\item Elemento inverso: $\forall a \in G$ existe un elemento $b \in G$ tal que $a \cdot b = b \cdot a = e$. Sea $a \in G$, a su elemento inverso se le denomina $a^{-1}$.
\end{enumerate}
\end{defn}

\begin{defn}[Subgrupo]
Sea un grupo  $(G,\cdot)$, un subconjunto H de G es un subgrupo de G cuando H es un grupo con $\cdot$ restringida a los elementos de H. Por tanto, debe cumplir:
\begin{enumerate}
	\item H contiene al elemento identidad: $e \in H$.
	\item H es cerrado: $\forall a,b \in H$, entonces $a \cdot b \in H$.
	\item H contiene los elementos inversos: $\forall a \in H$ entonces $a^{-1} \in H$.
\end{enumerate}
\obs Las propiedades 2. y 3. son equivalentes a comprobar que: $\forall a,b \in H$, entonces $a \cdot b^{-1}  \in H$
\end{defn}

\begin{defn}[Subgrupo\IS normal]
	Un subgrupo normal N de un grupo G es un subgrupo invariante por conjugación; es decir, para cada elemento $n \in N$ y cada $g \in G$, el elemento $gng^{-1} \in N$ (Análogamente $Ng = gN$). N es un subgrupo normal de G se escribe $N \triangleleft G$
\end{defn}

\begin{defn}[Grupo\IS conmutativo]\index{Grupo!abeliano}
Un grupo $(G,\cdot)$ es conmutativo o abeliano si es un grupo y además la operación $\cdot$ es conmutativa. Es decir, $\forall a,b \in G, a\cdot b = b \cdot a$.
\end{defn}

\obs Si un subgrupo es conmutativo entonces también es un subgrupo normal.

\begin{defn}[Anillo]
Sea $R$ un conjunto no vacío, y sean $'+'$ y $'\cdot'$ dos operaciones binarias en $R$. Se dice que el conjunto $(R, +, \cdot )$ es un anillo si se cumplen las siguientes propiedades:

\begin{enumerate}
	\item $R$ es un grupo conmutativo con la primera operación $'+'$. Es decir, $(R,+)$ es un grupo conmutativo. Si utilizamos como primera operación la suma ($'+'$), al elemento neutro lo denotaremos por \zero.
	\item $R$ es cerrado con respecto a la segunda operación $'\cdot'$. Es decir $\forall a,b \in R$, entonces $a \cdot b \in R$
	\item $R$ es asociativo con respecto a la segunda operación $'\cdot'$. Es decir, $\forall a,b,c \in R$, entonces $(a\cdot b) \cdot c = a \cdot (b \cdot c)$
	\item La segunda operación $'\cdot'$ es distributiva con respecto a la primera $'+'$. Es decir, $a \cdot (b+c) = a\cdot b + a \cdot c$
	\[
	\left\{ \begin{array}{c}
	a \cdot (b+c) = a\cdot b + a \cdot c \\
	(b+c) \cdot a = b\cdot a + c\cdot a \\
	\end{array}
	\right.
	\]
\end{enumerate}
\end{defn}


\begin{defn}[Anillo\IS unitario]
	Se dice que un anillo $(R, +, \cdot)$ es unitario si tiene un elemento neutro respecto de la segunda operación $'\cdot'$. A ese elemento neutro lo denotaremos por \one):
	\[ x\cdot \one = \one \cdot x = x \]
\end{defn}

\begin{defn}[Anillo\IS conmutativo]
Se dice que un anillo $(R, +, \cdot)$ es conmutativo si la segunda operación $'\cdot'$ tiene la propiedad conmutativa:
\[ a\cdot b = b\cdot a \]
\end{defn}


\nota Durante el curso trabajaremos con anillos conmutativos con unidad (anillos conmutativos unitarios), aunque no se mencione de manera explícita.

\begin{example}
\begin{itemize}
	\item Anillo no conmutativo: matrices cuadradas.
	\item Anillo sin unidad: los pares en $\ent$.
\end{itemize}
\end{example}


\begin{example}
Anillos que utilizaremos:
\begin{itemize}
	\item $\ent$, $\ent_n$ $n \geq 1$ (casi siempre que pida un ejemplo, estará basado en ellos).
	\item $\rac$, $\real$, $\cplex$, $\field_{p^n}$.
	\item R anillo de polinomios: $R[x] = \set{ \sum_{i=0}^n a_i x^i \tq a_i \in R, n \in \nat}$. Dos expresiones en $R[x]$ son iguales $\iff$ los coeficientes son iguales uno a uno\ie $a_i = b_i \ \forall i$.
	\item R anillo de polinomios en n variables. Denotaremos como $R[x_1,\dots,x_n]$ a un anillo de polinomios con coeficientes en $R$ en $n$ variables:
	\[ R[x_1,\dots,x_n] = \set{ \sum_{i_1,\dots,i_n} a_{i_1,\dots,i_n} x^{i_1}\dots x^{i_n} \tq a_{i_1,\dots,i_n} \in R } \]
	\item $\ent_6$ es un anillo finito, mientras que $\ent_6[x]$ es infinito.
\end{itemize}
\end{example}


\nota Consideraremos que los $\nat$ contienen al 0.

\notacion Para abreviar en muchas ocasiones llamaremos $R$ a $(R,+,\cdot)$.

\begin{defn}[Unidades]\index{Elemento! invertible}
Sea $(R,+,\cdot)$. un anillo, se dice que $a\in R$ es invertible (también llamado unidad) ($a\in \U(R)$) si $\exists b\in R $ \st $a \cdot b = \one$. Es decir, tiene elemento inverso con respecto a la segunda operación $'\cdot'$.
\end{defn}

\begin{defn}[Unidades\IS de R: $\U(R)$]
Al conjunto de todas las unidades (o elementos invertibles) de un anillo R se le denomina $\U(R)$.
\end{defn}

\begin{defn}[Cuerpo]
Sea $(R,+,\cdot)$ un anillo, diremos que es un cuerpo si $R\setminus \set{0} = R^*$ es un grupo conmutativo con la segunda operación $'\cdot'$ \ie si todo elemento no nulo de $R$ es una unidad ($\U(R) = R^*$).
%, donde $R^* = R \setminus \set{0}$.
\end{defn}

\notacion Sea A un conjunto: $A\setminus \set{0} = A^*$

\begin{example} Ejemplos de cuerpos:
	\begin{itemize}
		\item $\rac, \real, \cplex, \field_{p^n}$ ($\ent$ no es un cuerpo) % si, los ejemplos se repiten
		\item $\rac[\sqrt2] = \set{ a + b\sqrt2 \tq a,b \in \rac }$
	\end{itemize}
\end{example}

\begin{prop}
	Si $\zero=\one$, entonces todos los elementos del anillo son 0.
\end{prop}

\begin{defn}[Divisor de cero]
Sea $(R,+,\cdot)$ un anillo. Se dice que $a \in R$, $a\neq\zero$, es un {\bf divisor de cero}, si $\exists b \in R$, $b \neq \zero$, \st $a\cdot b = \zero$.
\end{defn}

\begin{defn}[Nilpotente]
Se dice que $a\in R$ es nilpotente si $\exists n \in \nat$, $n\geq 1$, \st $a^n = \zero$.
\end{defn}

\begin{example}
	\begin{itemize}
		\item En $\ent_6$, $\cls{2} * \cls{3} = \zero$, y como tenemos que $\cls{2} \neq \zero, \cls{3}\neq \zero$, $\cls{2}\neq\cls{3}$, sabemos que $\cls{2}$ y $\cls{3}$ son divisores de cero.
		\item En $\ent_4$, $\cls{2}^2 = \zero \implies \cls{2}$ es nilpotente.
	\end{itemize}
\end{example}


\begin{lemma}
	Todo elemento nilpotente es divisor de cero.
\end{lemma}

\begin{proof}
	Sea A grupo y sea $a \in A$, $a\neq\zero$, elemento nilpotente y sea
		\[n = \min\set{n \tq a^n = \zero, n \in \nat}\]
	Luego $a^n = a * a^{n-1} = \zero$. Y tenemos que $\zero \neq b = a^{n-1} \in A$ es el elemento \st $a*b=\zero$ ya que como $a\neq \zero$, $a^n = \zero \implies n\geq 2 \implies a^{n-1} \neq \zero$.
\end{proof}

\begin{lemma}
	No todo divisor de cero es nilpotente.
\end{lemma}

\begin{proof} % los contraejemplos siempre son sencillos
	En $\ent_6$, $\cls{2}$ es divisor de $\zero$ pero no es nilpotente ya que:
	\begin{gather*}
		\cls{2}^0 = \cls{1}, \ \cls{2}^1 = \cls{2}, \ \cls{2}^2 = \cls{4}, \ \cls{2}^3 = \cls{2}, \ \cls{2}^4 = \cls{4}, \ \dots
	\end{gather*}
\end{proof}

\begin{defn}[Dominio\IS de integridad]
Diremos que R es un dominio si no tiene divisores de cero.
\end{defn}

\begin{example}
	$\ent$
\end{example}

\begin{defn}[Anillo\IS reducido]
Diremos que R es un anillo reducido si no tiene elementos nilpotentes no nulos.
\end{defn}

\begin{example}
	\begin{itemize}
		\item Anillos reducidos que son dominios: $\ent, \ent_2$
		\item Anillo reducido que no es dominio: $\ent_6$
		\item Anillos no reducidos: $\ent_4$, $\ent_8$, $\ent_9$, \dots
	\end{itemize}
\end{example}

%\obs Las variedades algebraicas usan anillos reducidos: si el anillo es dominio, la variedad es reducible; si el anillo no es dominio, se dice que la variedad es no reducible.
% Y os preguntaréis, ¿Qué es una variedad algebraica? Pues hasta el tema 3...

\obs Si $a \in R$ divisor de cero, entonces $a$ \underline{\bf NO} es invertible. El recíproco también es cierto. (Las unidades y los divisores de 0 se llevan muy mal)

\obs Un anillo con unidad tiene al menos una unidad (el $\cls{1}$), y dos si $\cls{1} \neq \cls{-1}$. % nada que demostrar, la verdad

\begin{defn}[Subanillo]
Sea $(R,+,\cdot)$ un anillo y sea $S \subseteq R$, $S \neq \emptyset$. Diremos que $S$ es subanillo de $R$ si $S$ es un anillo con las operaciones definidas en $R$. Basta ver que:
\begin{enumerate}
	\item \one $\in S$
	\item $\forall a,b \in S$, entonces $a-b \in S$
	\item $\forall a,b \in S$, entonces $a\cdot b \in S$
\end{enumerate}
\end{defn}



\begin{example} Ejemplos de subanillos.
	\begin{enumerate}
		\item $R \subset R[x]$
		\item $\ent \subset \rac \subset \real \subset \cplex$
		\item Ningún $\ent_n$ es subanillo de $\ent$, porque $\ent_n \not \subseteq \ent$. Esto es así porque ni siquiera $\ent_n$ es un subconjunto de $\ent$ ya que los elementos de $\ent$ no son los mismo que los de $\ent_n$, en el primer caso son números normales y corrientes, y en el segundo son $\{\cls{i},  \forall 0<i<n\}$, y representan el resto de dividir $i$ entre $n$.
		% porque en $\ent$, 4 y 6 son distintos, pero en $\ent_n$ pueden ser iguales
	\end{enumerate}
\end{example}


\begin{defn}[Ideal] % no, no se refiere a un modelito
Sea $(R,+,\cdot)$ un anillo y sea $I \subseteq R$. Se dice que $I$ es un ideal si:

	\begin{enumerate}
		\item $I \neq \emptyset$
		\item $(I, +)$ es un subgrupo de R.
		\item \concept{Propiedad de absorción}: $\forall i \in I, \forall r \in R \implies r\cdot i \in I$.
	\end{enumerate}
\end{defn}

\obs Como $R$ es abeliano con  $'+'$ por definición, entonces si $(I,+)$ es un subgrupo, entonces es un subgrupo normal.
% Revisado. Guille.

\begin{defn}[Ideal\IS propio]
	Decimos que un ideal $I \subseteq R$ es propio si $I \neq R$.
\end{defn}

\begin{example} Ejemplos de ideales:
	\begin{itemize}
		\item Fijado $k \in \ent$, el conjunto $\set{n\cdot k \tq n \in \nat }$ es un ideal de $\ent$.
		\item En $R$, hay siempre dos ideales, R (que es un ideal no propio) y $\set{\zero}$.
		\item En $\ent_{10}$, $[\cls{0},\cls{2},\cls{4},\cls{6},\cls{8} ]$ forman un ideal.
	\end{itemize}
\end{example}

\begin{example} Sea $\rac[x,y]$, sea $J=\{p(x,y): p(1,1)=0\}$ es un ideal. Lo comprobamos:
\begin{enumerate}
	\item $J \neq \emptyset$ porque $0 \in J$
	\item $(J,+)$ es un subgrupo.
	\begin{itemize}
		\item El 0 es el elemento identidad (o neutro) y ya hemos visto que pertenece a J.
		\item Hay que comprobar que J es cerrado con $'+'$ y que contiene los elementos inversos, pero como hemos visto esto es equivalente a probar que $\forall p(x,y), q(x,y) \in J$, entonces $p(x,y)-q(x,y) \in J$ (ya que con la suma, el inverso de $q(x,y)$ es $-q(x,y)$).

		Como $p(1,1)=0=q(1,1)$, evidentemente $p(1,1)-q(1,1)=0$ y por tanto $p(x,y)-q(x,y) \in J$.
	\end{itemize}
	\item Si escogemos $r(x,y) \in \rac[x,y]$ y si $p(x,y) \in J$, tenemos que ver que $r(x,y)\cdot p(x,y) \in J$. Obvio ya que $p(1,1)=0 \implies r(1,1)\cdot p(1,1) = 0$
\end{enumerate}
\end{example}

Algunas observaciones referentes a ideales. Sea $I \subseteq R$ un ideal de un anillo R:
\begin{itemize}
	\item Si $\one \in I \implies I=R$. Por la propiedad de absorción.
	\item Si $u\in I$ siendo u una unidad $\implies I=R$. Ya que si u es una unidad, entonces por la propiedad de absorción, cogiendo $r=u^{-1} \in R$ obtenemos \one (el elemento neutro respecto a $'\cdot'$).
\end{itemize}
\begin{prop}
	$R$ es un cuerpo $\iff$ los únicos ideales de $R$ son $\zerogen$ y $R$.
\end{prop}
\begin{proof}

	$\Leftarrow$) Sea $a \in R$, $a \neq 0$, queremos probar que $\exists a' \in R$ tal que $aa' = 1$ \footnote{Es decir, la definición de cuerpo, que todo elemento de $R$ es una unidad.}.

	Consideramos $I=\set{ra \tq r \in R}$ y vamos a ver que es un ideal. Está claro que $I \neq \zerogen$ ya que $a \in I$ \footnote{Cogiendo $r = \one$.}. Por tanto, como por hipótesis tenemos que $R$ solo tiene dos ideales, queda que $I=R$, y como $\one \in R$, entonces $\one \in I$, y por la propiedad de absorción, existe un $r$ tal que $ra=1$.

	$\Rightarrow$) Supongamos $\exists I$ ideal propio de $R$ distinto de $\zerogen$, y sea $a \in I$ tal que $a \neq \zero$. Por la propiedad de absorción, $\one \in I$ ya que $a*a^{-1} = \one$; luego, de nuevo, por la propiedad de absorción, tenemos que $I = R$. $\Rightarrow\Leftarrow$

	Luego el único ideal propio de $R$ es $\zerogen$, y $R$ siempre es un ideal de $R$.
\end{proof}

De la segunda observación (\textcolor{red}{Que segunda observacion?}) podemos deducir las siguientes 2 definiciones alternativas de ideal:

\begin{prop}\label{prop:def_ideal2}
	Sea $(R,+,\cdot)$ un anillo y sea $I \subseteq R$. Se dice que $I$ es un ideal si

	\begin{enumerate}
		\item $I \neq \emptyset$
		\item $\forall a,b \in I$, entonces $a-b \in I$.
		\item Propiedad de absorción
	\end{enumerate}
\end{prop}
\begin{prop}\label{prop:def_ideal3}
	Sea $(R,+,\cdot)$ un anillo y sea $I \subseteq R$. Se dice que $I$ es un ideal si

	\begin{enumerate}
		\item $I \neq \emptyset$
		\item $\forall a,b \in I$, entonces $a+b \in I$.
		\item Propiedad de absorción
	\end{enumerate}
\end{prop}


Más aún podemos ver que:
\begin{prop}
	La proposición \ref{prop:def_ideal2} es equivalente a la proposición \ref{prop:def_ideal3}.
\end{prop}
\begin{proof}
	$\ref{prop:def_ideal2} \implies \ref{prop:def_ideal3})$

	Sean $a,b \in I$. Como $b \in I$, entonces $-n \in I$ porque es un grupo aditivo.
	Como $a-b \in I \ \forall a,b \in I \implies a-(-b) \in I$. (\textcolor{red}{No me convence esta demostración y no me aclara nada, ¿donde tenemos un grupo aditivo?})

	$\ref{prop:def_ideal3} \implies \ref{prop:def_ideal2})$
	De manera similar, si $a+b \in I \ \forall a,b \in I \implies a+(-b) \in I$, ya que $-b \in I$.
\end{proof}

\textcolor{red}{Quedaría ver por qué es lo mismo que (I,+) sea subgrupo a que $\forall a,b \in I$, entonces $a-b \in I$}

\begin{lemma}
 Sea R un anillo, el conjunto de los elementos nilpotentes de R es un ideal que recibe el nombre de \concept{nilradical} de R. Que denotaremos por $\nil(R)$.
\end{lemma}
\begin{proof}
	\begin{itemize}
		\item $\nil(R) \neq \emptyset$ porque $\zero \in \nil(R)$
		\item Sean $a,b \in \nil(R)$ tenemos que ver que $a+b \in \nil(R)$:

		Como $a \in \nil(R)$ entonces $\exists n$ tal que $a^n=0$.\\
		Como $b \in \nil(R)$ entonces $\exists m$ tal que $b^m=0$.

		Por tanto:
		$$(a+b)^{n+m} = \sum_{j=0}^{n+m} \binom{n+m}{j} a^j \cdot b^{n+m-j} = 0$$

		Ya que si $j<n$ se anula b, y si $j \geq n$ se anula a.
		\item Sea $a \in \nil(R)$ y sea $r\in R$, queremos ver si $ra \in \nil(R)$.

		Como $a\in \nil(R) \implies \exists n \geq 1 \st a^n=0$, por tanto $(ra)^n = 0$ (recordemos que trabajamos con anillos conmutativos).
	\end{itemize}
\end{proof}

\begin{prop}
	R es reducido $\Leftrightarrow$ $\nil(R)=\set{\zero}$
\end{prop}


\begin{prop}
	Sea $T \subseteq R$, $T \neq \emptyset$, existe un ideal en $R$ tal que es el más pequeño que contiene a $T$.
\end{prop}
\begin{proof}
	Basta considerar la intersección de todos los ideales que contienen a $T$; que no puede ser vacía porque al menos, el ideal $R$ contiene a $T$.
\end{proof}

\begin{defn}[Ideal\IS generado por T]
Definimos el ideal generado por $T$ como el ideal más pequeño de $R$ que contiene a $T$ y se denota por $\gen{T}$.

Explícitamente, sea R un anillo y $T \subset R$ no vacío, entonces:
$$ <T> = \{r_1t_1+...+r_st_s: r_i \in R, t_i \in T, s \geq 1 \} $$
\end{defn}

La notación no es arbitraria, recordamos que $\gen{E}$ es el subespacio vectorial generado por $E$. Más adelante descubriremos que, filosóficamente, son la misma cosa.

\begin{example} Sea $\rac[x,y]$, y sea $T=\{x,y\}$, entonces el ideal generado por T es:
	$$<T> = <x,y>= \{p(x,y)\cdot x+q(x,y)\cdot y : p(x,y),q(x,y) \in \rac[x,y]\}$$
\end{example}

\section{Operaciones con Ideales}

\begin{prop}
	Sea $\{J_i\}_{i\in I}$ una familia de ideales en R. Entonces $\bigcap_{i \in I}J_i$ es un ideal.
\end{prop}


\begin{prop}
	Sean $I,J \subset R$ ideales, entonces $I \cup J$ en general no es un ideal.
\end{prop}

\begin{example} En $\ent$
	Sean $\gen{2}$ (los pares) y $\gen{3}$ (múltiplos del 3). Ambos son ideales, sin embargo $J=\gen{2} \cup \gen{3}$ no es un ideal.

	Esto es así porque $3\in \gen{3}$, $2 \in \gen{2}$ y sin embargo $3-2=1$, que no pertenece a $\gen{2} \cup \gen{3}$, por tanto, $(J,+)$ no es subgrupo de $\ent$, y por tanto $J$ no es ideal de $\ent$
\end{example}

\begin{defn}[Ideal más pequeño que contiene a dos ideales]
Sean $I$ y $J$ ideales, entonces llamamos $I+J$ al ideal más pequeño que contiene a I y a J, que explícitamente es:
$$ I+J = \{a+b: a\in I, b\in J\} $$

\textcolor{red}{No pillo esto, si cogemos el ejemplo no veo la diferencia entre $<2> \cup <3>$ y $<2> + <3>$, ambos contienen los mismos elementos no?}

\noteby{Guille}{No es exactamente lo mismo. $\gen{2} ∪ \gen{3}$ contiene los elementos que son múltiplos de 2 y/o de 3. Sin embargo, $\gen{2} + \gen{3}$ va a contener precisamente a elementos como el $1$ que nos destrozaban antes el ejemplo, ya que $1 = 3 + (-2)$ con $3 ∈ \gen{3},\, -2 ∈ \gen{2}$. También contendrá otros como 5, 8, que son sumas de múltiplos de 2 y 3. De hecho, como $1 ∈ \gen{2} + \gen{3}$ tenemos que $\gen{2} + \gen{3} = ℤ$, creo.}

\end{defn}

\begin{defn}[Producto\IS de ideales]
	Sean $I$ y $J$ dos ideales, entonces $J\cdot I = \gen{a\cdot b: a \in I, b \in J}$

	En general, sea $I=\gen{f_1,\dotsc ,f_s}$ y $J=\gen{g_1,\dotsc,g_t}$ entonces:
	$$I\cdot J=\gen{ f_i, g_j: i \in \{1,...,s\}, j \in \{1,...,t\}}$$
\end{defn}

\begin{example} Sea $\gen{x,y} \subset \real[x,y]$
	$$\gen{x,y}^2 = \gen{x^2,xy,y^2}$$
\end{example}

\obs Dados $I,J \subset R$ no es cierto que todo elemento $h \in I\cdot J$ sea igual al producto de un elemento de $I$ por otro de $J$.

\begin{example}
	Sea $I=J=\gen{x,y}$, entonces $I\cdot J = \gen{x^2,xy,y^2}$.

	Cogemos el elemento $x^2 + y^2$ perteneciente a $I\cdot J$. Podemos ver que $x^2 + y^2$ no se puede escribir como $p(x,y) \cdot q(x,y)$ con $p(x,y),q(x,y) \in <x,y>$. (Y con $p(x,y)$ o $q(x,y)$ distintos de 1)

	\obs En los complejos sí se podría ya que $(x+iy)(x-iy)=x^2 + y^2$
\end{example}

\obs Sean $I$ y $J$ dos ideales, se cumple que $I\cdot J \subseteq I$ y que $I \cdot J \subseteq J$, y por tanto $I \cdot J \subseteq I \cap J$. En general esta última inclusión es estricta.

\begin{example}
	\begin{itemize}
\item Inclusión estricta: en $\ent$, cogemos $I=\gen{2}$, $J=\gen{4}$. Tenemos que $I\cdot J=\gen{8}$ pero $I\cap J=\gen{4}$.
\item Inclusión no estricta: en $\ent$, cogemos $I=\gen{2}$, $J=\gen{3}$. Tenemos que $I\cdot J=\gen{6}$ pero $I\cap J=\gen{6}$.
	\end{itemize}
\end{example}

\begin{defn}[Radical\IS de un ideal]
Sea $I \subset R$ un ideal. Definimos el radical de I como:
$$Rad(I)=\sqrt{I}=\{a \in R: \exists n \in \nat \textbf{ con } a^n \in I \}$$
\end{defn}

\begin{example}
\begin{itemize}
	\item En $\ent$: $\sqrt{<2>}=<2>$, $\sqrt{<4>}=<2>$, $\sqrt{<8>}=<2>$, $\sqrt{<6>}=<6>$
	\item En $\rac[x,y]$: $\sqrt{<x^2,xy>}=<x>$
\end{itemize}
\end{example}

\textcolor{blue}{\obs $\rac[x,y]$ es dominio de factorización única. (ya definiremos y hablaremos de esto más adelante)}

Veamos algunas observaciones referentes al radical de un ideal. Sea $I \subset R$ un ideal. Entonces:
\begin{itemize}
	\item $I \subset Rad(I)$
	\item $Rad(I)$ es un ideal
	\item $\{\text{Elementos nilpotentes }\}=\nil(R)=\{a\in R: \exists n \text{ con }a^n=0 \}=Rad((0))$
\end{itemize}

\begin{defn}[Ideal primo]
	Diremos que un ideal $I\subset R$ es primo si $I\neq R$ y si cada vez que $a\cdot b \in I$ se tiene que $a\in I$ o $b\in I$.
\end{defn}

\begin{example} En $\ent$
	\begin{itemize}
		\item <p> con p primo es un ideal primo.
		\item (0) es un ideal primo.
	\end{itemize}
\end{example}

\begin{example} En $\ent_6$
	\begin{itemize}
		\item (\zero) no es ideal primo. Ya que $\cls{2} \cdot \cls{3}=\cls{0}=\zero$
		\item $<\cls{2}>=\{\cls{0},\cls{2},\cls{4} \}$ es un ideal primo.
		\item $<\cls{3}>=\{\cls{0},\cls{3} \}$ es un ideal primo.
	\end{itemize}
\end{example}

\begin{prop}
	Sea K un cuerpo, y sea $K[x]$, entonces $\gen{x}$ es primo.
\end{prop}
\begin{proof}
	Tenemos que $K[x]$ es un dominio de factorización única (\textcolor{red}{No se esto para que lo usa} \noteby{Guille}{Creo que para decir que $p(x)q(x) = x·r(x)$ y por lo tanto $x$ divide a $p(x)$ o a $q(x)$. Si no fuese DFU $xr(x)$ podría tener otra factorización alternativa y no sabríamos en cuál de las dos se cumple que $x$ divide a alguno de los factores.}). Consideramos el ideal generado por $x$, es decir: $<x>$.

	Supongamos que $p(x)\cdot q(x) \in \gen{x}$. Entonces, por ser un ideal, por la propiedad de absorción $\exists r(x) \in K[x]$ tal que $p(x) q(x)=r(x) x$. Por tanto, necesariamente $x|p(x)$, y en ese caso $p(x) \in \gen{x}$, o $x|q(x)$, y en ese caso $q(x) \in \gen{x}$.
\end{proof}

Generalizamos la proposición anterior:
\begin{prop}
	Sea K un cuerpo, y sea $K[x_1,...,x_n]$, entonces $\gen{x_i}$ es primo.
\end{prop}

\begin{prop}
	Sea K un cuerpo, y sea $K[x_1,...,x_n]$, se cumple que si $<p(x_1,...,x_n)>$ es primo,  entonces $p(x_1,...,x_n)$ es irreducible.
\end{prop}

\begin{example}
	En $\real[x]$: $<x^2+1>$ es primo y $p(x)=x^2+1$ es irreducible.
\end{example}

\begin{prop}
	Sea $K$ un cuerpo (\textcolor{red}{Cierto también si K es un anillo que no sea cuerpo no?} \noteby{Guille}{Yo diría que sí, en la demostración no veo que use que es un cuerpo.}), y sea el ideal $\gen{x,y}$ en $K[x,y]$, entonces:
	$$<x,y>=\{p(x,y)\in K[x,y]:p(0,0)=0 \}$$
\end{prop}

\begin{proof}
	\begin{itemize}
		\item $\subset)$ Hemos visto que $<x,y>=\{s(x,y) x+r(x,y) y: s(x,y),r(x,y) \in K[x,y] \}$. Es obvio que $s(x,y) x+r(x,y) y$ se anula en el $(0,0)$.
		\item $\supset)$ Sea $p(x,y) \in K[x,y]$ tal que $p(0,0)=0$. Tenemos que $p(x,y)=a_0+a_1x+a_2y+a_3x^2+...+a_nx^ny^m$. Entonces $p(0,0)=a_0=0$. Por tanto, si $a_0=0$, entonces $p(x,y)$ se puede expresar como $r(x,y)y+s(x,y)x$.
	\end{itemize}
\end{proof}

\obs De esta forma, podemos probar (siendo K un cuerpo) que $\gen{x,y} \in K[x,y]$ es un ideal primo de forma más sencilla. Supongamos que $p(x,y)q(x,y) \in <x,y>$, necesariamente $p(x,y) \in <x,y>$ o $q(x,y) \in <x,y>$. Ya que como $p(x,y)q(x,y) \in <x,y>$ entonces $p(0,0)q(0,0)=0$, pero como K es un cuerpo (y un dominio de integridad), no tiene divisores de 0, por tanto debe cumplirse que $p(0,0)=0$ y que $q(0,0)=0$.

\textcolor{red}{Y si no es cuerpo también es primo ya que no puede ser que los dos coeficientes independientes de p y q sean distintos de 0 para que $p(x,y)q(x,y)=0$, por tanto, al menos uno de los dos pertenece a $<x,y>$}

Se deja como ejercicio para el lector experimentado probar que en $K[x_1,...,x_n]$, el ideal $<x_1-a_1,...,x_n-a_n>$ con $a_1,...,a_n \in K$, es primo.

\begin{defn}
	Sea $I \subsetneq R$ un ideal (propio), diremos que I es un ideal maximal si cada vez que encontremos un ideal $J \in R$ con $I \subset J$, se tiene que o bien $I=J$ o bien $J=R$.

	En otras palabras $I$ es un ideal maximal si $I \subsetneq R$  (propio) y es maximal respecto a la inclusión de ideales.
\end{defn}

\begin{example}
	\begin{itemize}
		\item En $\ent$ son maximales los $<p>$ con p primo.
		\item En $\ent_8$ hay los siguientes ideales:
		\begin{itemize}
			\item $<\cls{2}>=\{\cls{0},\cls{2},\cls{4},\cls{6} \}$ que es el único maximal
			\item $\set{\cls{0}}$ que no es maximal por estar contenido en $<\cls{2}>$.
			\item $\ent_8$ que no es maximal porque no es propio.
			\item $<\cls{4}>=\{\cls{0},\cls{4} \}$ que no es maximal por estar contenido en $<\cls{2}>$.
		\end{itemize}
	\end{itemize}
\end{example}

\begin{defn}[Ideal principal]
	Un ideal es principal si está generado por un único elemento.
\end{defn}

\begin{example}
	Sea $R$ un anillo, el ideal principal generado por un elemento $a \in R$ es el conjunto:
	$$ <a>=(a)=\{ra: r \in R \} $$
\end{example}

\begin{prop}
	Sea K un cuerpo, y sea $K[x]$, todos los ideales de $K[x]$ son principales. $K[x]$ es un dominio d eideales principales, es decir, todos los ideales están generados por un polinomio.
\end{prop}

\begin{prop}
Sea K un cuerpo, y sea $p(x)\neq 0 \in K[x]$. $<p(x)>$ es maximal $\Leftrightarrow$ $p(x)$ es irreducible.
\end{prop}

\begin{example}
Sea $\rac[x]$, cogemos el ideal $I=<x-1,y-2>$, vamos a ver que es maximal.

\begin{proof}
	Supongamos que $I \subsetneq J$ para algún ideal $J \subset \rac[x,y]$. Osea que existe un $p(x,y) \in J$ pero $p(x,y) \notin I$

	Hacemos el desarrollo de Taylor de $p(x,y)$:

	$$ p(x,y)=p(1,2) + \underbrace{\left.\frac{d p(x,y)}{dx}\right|_{(1,2)} (x-1)+ \left.\frac{d p(x,y)}{dy}\right|_{(1,2)} (y-2)+\frac{1}{2}\left.\frac{d^2 p(x,y)}{dx^2}\right|_{(1,2)} (x-1)^2+...}_{\in <x-1,y-2>}$$

	Osea que $p(x,y)=p(1,2)+r(x,y)$ con $r(x,y) \in <x-1,y-2>$.

	Por tanto nos queda $\underbrace{p(1,2)}_{\in \rac^*}=\underbrace{\underbrace{p(x,y)}_{\in J}-\underbrace{r(x,y)}_{\in I \subset J}}_{\in J}$

	$p(1,2)$ es una constante no nula porque sino $p(x,y) \in I$. (Recordar que $\rac^*=\rac\setminus\{0\}$).

	Entonces, $p(1,2) \in \rac^*$ y $p(1,2) \in J = \textcolor{red}{\rac[x,y]}$ \textcolor{red}{No veo esto ultimo}.

	Y nos queda que $I=\{ q(x,y) \in \rac[x,y]: q(1,2)=0 \}$
\end{proof}
\end{example}

\textcolor{red}{Generalizamos una proposición anterior:}
\begin{prop}
	Sea $K$ un cuerpo (Cierto también si K es un anillo que no sea cuerpo no?), y sea el ideal $<x_1-a_1,...,x_n-a_n>$ contenido en $K[x_1,...,x_n]$, entonces:

	$$<x_1-a_1,...,x_n-a_n>=\{p(x_1,...,x_2)\in K[x,y]:p(a_1,...,a_n)=0 \}$$
\end{prop}
