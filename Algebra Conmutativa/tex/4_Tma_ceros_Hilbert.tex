\chapter{Teorema de los ceros de Hilbert}

Hemos jugado en $\Akn$ y en $\K[x_1,...,x_n]$. Y sabemos movernos de un sitio a otro:

\begin{align*}
	\Akn & \longmapsto  \K[x_1,...,x_N] \\
	X & \longmapsto  \I(X) \\
	\V(J) & \leftarrow  J \\
\end{align*}

\begin{itemize}
	\item Si cojo $X\in \Akn$ v.a.a., puedo ir a $\I(X) \in \K[x_1,...,x_n]$ y volver a $\V(\I(X)) \in \Akn$ que es igual a $X$ por ser una v.a.a.. 
	
	Tendría el siguiente diagramilla:
	
	\item Sin embargo, si cojo $J$ ideal radical en $\K[x_1,...,x_n]$, puedo ir a $\V(J)$ en $\Akn$, y volver a $\I(\V(J)) \supseteq J$. Pero puede pasar que la inclusión sea estricta:
	\begin{example}
		\textcolor{red}{Julian majo pon esto mas bonito}
		\begin{enumerate}
			\item Ejemplo de inclusión estricta: 
			\begin{align*}
				\A^1_{\real} & \longmapsto  \real[x] \\
				\V(\gen{x^2+1}) & \leftarrow  \gen{x^2+1} \\
				\shortparallel & \\
				\emptyset & \longmapsto  \I(\{ \emptyset \})=\real[x] \\
			\end{align*}
			\item Otro ejemplo de inclusión estricta:
			
			Cogemos en $\real[x,y]$ e ideal  radical $\gen{(x-1)^2+(y-2)^2}$, que es primo (y por tanto radical) porque $p(x,y)=(x-1)^2+(y-2)^2$ es irreducible en $\real[x,y]$.
			
			Esto es así porque $\real[x,y] \subset \cplex[x,y]$  que es un dominio de factorización única.
			
			Y en $\cplex[x,y]$ tenemos que $p(x,y)=(x-1)^2+(y-2)^2=((x-1)+i(y-2))((x-1)-i(y-2))$. Por tanto no se puede factorizar de otra forma.
			
			\begin{align*}
				\A^2_{\real} & \longmapsto  \real[x,y] \\
				\V(J) & \leftarrow  \gen{(x-1)^2+(y-2)^2}=J \\
				\shortparallel & \\
				X=\{1,2\} & \longmapsto  \I(X)=\gen{x-1,y-2}\\
			\end{align*}
			
			Y obviamente $\gen{(x-1)^2+(y-2)^2} \subsetneqq \gen{x-1,y-2}$
		\end{enumerate}
	\end{example}
\end{itemize}

\textbf{Conclusión:} No hay un diccionario entre v.a.a. en $\Akn$ e ideales radicales en $\K[x_1,...,x_n]$.

\begin{theorem}[Teorema de los ceros de Hilbert]
	Si $\K$ es algebraicamente cerrado y $J \subset K[x_1,...,x_n]$ es radical, entonces $J=\I(\V(J))$
\end{theorem}