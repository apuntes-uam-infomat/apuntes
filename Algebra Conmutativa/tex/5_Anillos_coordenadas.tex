% -*- root: ../AlgebraConmutativa.tex -*-
\chapter{Anillos de coordenadas y morfismos entre variedades}

Una vez que ya hemos presentado los objetos que querremos estudiar (las variedades algebraicas), vamos a describir los tipos de funciones que podremos definir sobre las variedades algebraicas afines.

\begin{defn}[Función\IS regular] Sea $X ⊂ \afesp$ una variedad algebraica afín. Diremos que $\appl{f}{X}{\K}$ es regular en $X$ si existe un polinomio $F(x_1, \dotsc, x_n) ∈ \K[x_1, \dotsc, x_n]$ tal que $∀(a_1, \dotsc, a_n) ∈ X$, se tiene que $f(a_1, \dotsc, a_n) = F(a_1, \dotsc, a_n)$.

Es decir, $f$ será regular si y sólo si hay un polinomio $F(x_1, \dotsc, x_n) ∈ \K[x_1, \dotsc, x_n]$ tal que $\restr{F}{X} = f$.
\end{defn}

\begin{example}
\begin{itemize}
\item En $\afesp[ℝ][2]$, podemos fijarnos en la recta $X = \set{y = 0}$ y en la función $f(a,b) = a + \sin b$. Sobre $X$, $f$ es regular ya que podemos tener el polinomio $F(x,y) = x$ que es igual a $f$ en todos los puntos de $X$. Ahora bien, $f$ no sería regular sobre todo $\afesp[ℝ][2]$ porque el seno no va a poder expresarse como un polinomio (finito).
\item Dado $\afesp[\K][2] ⊃ Y = \set{y^2 - x = 0}$, y $g(a,b) = a$, que es regular sobre $Y$. Si tomásemos $g(a,b) = a + (a^3 + 3 \sin b) (b^2 - a)$, sigue siendo regular porque la parte conflictiva desaparece.
\end{itemize}
\end{example}

Podemos ver que siempre tenemos al menos una función regular en variedades $X$: al menos tenemos la función $0$ que siempre es regular. Además, la suma de dos funciones regulares seguirá siendo regular, el producto también, y tenemos un neutro para la suma (la función $0$) y el producto (la función $1$). Así, podremos estudiar el anillo conmutativo y con unidad de funciones regulares sobre $X$, que describiremos como \( \label{eq:AnilloFuncRegulares} Γ(X, \K) ≝ \set{\appl{f}{X}{\K} \tq f\text{ es regular}} \)

Para poder hacer cuentas, querremos identificar y saber quién es ese anillo. Para ello, vemos que hay una relación entre los polinomios de $\K[x_1, \dotsc, x_n]$ y $Γ(x,\K)$:
\begin{align*}
\appl{ψ}{\K[x_1, \dotsc, x_n]&}{Γ(X, \K)} \\
p(x_1, \dotsc, x_n) &\longmapsto \appl{f_p}{X}{\K} \\
&\phantom{\longmapsto f_p:}\quad \va \longmapsto p(\va)
\end{align*} con $\va ∈ X$.

Esta función ψ es un homomorfismo de anillos trivialmente. Además, es sobreyectivo. Para ello, tomamos $f ∈ Γ(X, \K)$, está claro que existe un polinomio $p ∈ \K[x_1, \dotsc, x_n]$ tal que $ψ(p) = f$ , por la propia definición de función regular.

Una vez que sabemos que es sobreyectivo, podemos aplicar el \nref{thm:IsomorfiaAnillos1} y sabemos que \[ \quot{\K[x_1, \dotsc, x_n]}{\ker ψ} \simeq Γ(X,\K) \]

Querríamos saber entonces quién es $\ker ψ$, que será el conjunto siguiente: \[ \ker ψ = \set{ p(x_1, \dotsc, x_n) ∈ \K[x_1, \dotsc, x_n] \tq \restr{ψ(p(x_1, \dotsc, x_n))}{X} = 0} \]

En otras palabras, $\ker ψ = \I(X)$. Una vez hecho esto ya podremos definir bien $Γ$ y darle la notación definitiva.

\begin{defn}[Anillo\IS de coordenadas de $X$] Sea $X ⊂ \afesp$. Entonces, se define el anillo de funciones regular en $X$ o anillo de coordenadas de $X$ como \[ \K[X] ≝ \quot{\K[x_1, \dotsc, x_n]}{\I(X)} \simeq Γ(X, \K)\]
\end{defn}

A través de este anillo podremos estudiar la variedad. Por ejemplo, podremos saber su dimensión, si es irreducible ($\K[X]$ debería de ser dominio de integridad) o si tiene picos.

De momento, lo que empezaremos estudiando serán los morfismos entre las variedades algebraicas a través de este anillo de coordenadas.

\section{Morfismos entre variedades algebraicas afines}

\begin{defn}[Morfismo\IS entre variedades algebraicas afines] Sean $X ⊆ \afesp$ y $Y ⊆ \afesp[\K][m]$ dos variedades algebraicas afines definidas sobre el mismo cuerpo $\K$. Un morfismo $\appl{φ}{X}{Y}$ es una función cuyas coordenadas son funciones regulares en $X$, de tal forma que \[ φ(a_1, \dotsc, a_n) = \left(f_1(a_1, \dotsc, a_n), \dotsc, f_m(a_1, \dotsc, a_n)\right)\] con $f_i$, $i = 1, \dotsc, m$ regulares en $X$.
\end{defn}

\begin{example}
La aplicación \begin{align*}
\appl{φ}{\afesp[\K][1]&}{\afesp[\K][2]} \\
(a) &\longmapsto (a,0)
\end{align*} es un morfismo entre variedades. En general, podríamos también considerar $(a) \longmapsto (p_1(a), p_2(a))$ con $p_i ∈ \K[\afesp[\K][1]] = \K[x]$ (esta última igualdad sólo si \K es infinito), que es lo que esperábamos que pasase.

Igualmente, una proyección \begin{align*}
\afesp[\K][3] &\longmapsto \afesp[\K][2] \\
(a_1, a_2, a_3) &\longmapsto (a_1, a_2)
\end{align*} también será un morfismo.
\end{example}

Una cosa que esperaríamos es que los morfismos nos lleven funciones regulares en funciones regulares definidas sobre ambos anillos de coordenadas. Y como a veces el mundo matemático es justo, es precisamente lo que va a ocurrir.

\begin{prop} Sean $X ⊆ \afesp$ y $Y ⊆ \afesp[\K][m]$ dos variedades algebraicas afines definidas sobre el mismo cuerpo $\K$, y sea φ un morfismo. Entonces φ induce un homomorfismo de anillos $\appl{f_φ}{\K[Y]}{\K[X]}$.
\end{prop}

\begin{proof} Dada una aplicación $g ∈ \K[Y]$, entonces $g ○ φ$ es una aplicación $\appl{g ○ φ}{X}{\K}$. Hay que demostrar que es un homomorfismo de anillos.

Como tratar de construir un homomorfismo de cocientes a cocientes es arriesgado, lo que haremos será definir un homomorfismo de $\K[y_1, \dotsc, y_m]$ a $\K[X]$ y después ver que factoriza por el cociente. El homomorfismo lo definiremos de la siguiente forma: dejaremos fijos los $r ∈ \K$, y cada variable $y_i$ la consideraremos como función regular en $\K[Y]$. Decíamos antes que la aplicación que definiríamos $X \longmapsto \K$ sería la composición con φ, así que lo que hacemos es ver qué ocurre cuando $g = y_i$: \[
\begin{matrix}
X &\xrightarrow{φ}& Y &\xrightarrow{y_i}& \K \\
(a_1, \dotsc, a_n) &\longrightarrow & \left(f_1(a_1, \dotsc, a_n), \dotsc, f_m(a_1, \dotsc, a_n)\right) &\longrightarrow & f_i(a_1, \dotsc, a_n)
\end{matrix}\]

Así, el homomorfismo que definiremos será
\begin{align*}
\K[y_1, \dotsc, y_m] &\longmapsto \K[X] = \quot{\K[x_1, \dotsc, x_n]}{\I(X)} \\
r ∈ \K &\longmapsto r ∈ \K \\
y_i &\longmapsto \cls{f_i(x_1,\dotsc, x_n)}
\end{align*}
\end{proof}
