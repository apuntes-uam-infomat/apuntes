% -*- root: ../AlgebraConmutativa.tex -*-
\chapter{Anillos de coordenadas y morfismos entre variedades}

Una vez que ya hemos presentado los objetos que querremos estudiar (las variedades algebraicas), vamos a describir los tipos de funciones que podremos definir sobre las variedades algebraicas afines.

\begin{defn}[Función\IS regular] Sea $X ⊂ \afesp$ una variedad algebraica afín. Diremos que $\appl{f}{X}{\K}$ es regular en $X$ si existe un polinomio $F(x_1, \dotsc, x_n) ∈ \K[x_1, \dotsc, x_n]$ tal que $∀(a_1, \dotsc, a_n) ∈ X$, se tiene que $f(a_1, \dotsc, a_n) = F(a_1, \dotsc, a_n)$.

Es decir, $f$ será regular si y sólo si hay un polinomio $F(x_1, \dotsc, x_n) ∈ \K[x_1, \dotsc, x_n]$ tal que $\restr{F}{X} = f$.
\end{defn}

\begin{example}
\begin{itemize}
\item En $\afesp[ℝ][2]$, podemos fijarnos en la recta $X = \set{y = 0}$ y en la función $f(a,b) = a + \sin b$. Sobre $X$, $f$ es regular ya que podemos tener el polinomio $F(x,y) = x$ que es igual a $f$ en todos los puntos de $X$. Ahora bien, $f$ no sería regular sobre todo $\afesp[ℝ][2]$ porque el seno no va a poder expresarse como un polinomio (finito).
\item Dado $\afesp[\K][2] ⊃ Y = \set{y^2 - x = 0}$, y $g(a,b) = a$, que es regular sobre $Y$. Si tomásemos $g(a,b) = a + (a^3 + 3 \sin b) (b^2 - a)$, sigue siendo regular porque la parte conflictiva desaparece.
\end{itemize}
\end{example}

Podemos ver que siempre tenemos al menos una función regular en variedades $X$: al menos tenemos la función $0$ que siempre es regular. Además, la suma de dos funciones regulares seguirá siendo regular, el producto también, y tenemos un neutro para la suma (la función $0$) y el producto (la función $1$). Así, podremos estudiar el anillo conmutativo y con unidad de funciones regulares sobre $X$, que describiremos como \( \label{eq:AnilloFuncRegulares} Γ(X, \K) ≝ \set{\appl{f}{X}{\K} \tq f\text{ es regular}} \)

Para poder hacer cuentas, querremos identificar y saber quién es ese anillo. Para ello, vemos que hay una relación entre los polinomios de $\K[x_1, \dotsc, x_n]$ y $Γ(x,\K)$:
\begin{align*}
\appl{ψ}{\K[x_1, \dotsc, x_n]&}{Γ(X, \K)} \\
p(x_1, \dotsc, x_n) &\longmapsto \appl{f_p}{X}{\K} \\
&\phantom{\longmapsto f_p:}\quad \va \longmapsto p(\va)
\end{align*} con $\va ∈ X$.

Esta función ψ es un homomorfismo de anillos trivialmente. Además, es sobreyectivo. Para ello, tomamos $f ∈ Γ(X, \K)$, está claro que existe un polinomio $p ∈ \K[x_1, \dotsc, x_n]$ tal que $ψ(p) = f$ , por la propia definición de función regular.

Una vez que sabemos que es sobreyectivo, podemos aplicar el \nref{thm:IsomorfiaAnillos1} y sabemos que \[ \quot{\K[x_1, \dotsc, x_n]}{\ker ψ} \simeq Γ(X,\K) \]

Querríamos saber entonces quién es $\ker ψ$, que será el conjunto siguiente: \[ \ker ψ = \set{ p(x_1, \dotsc, x_n) ∈ \K[x_1, \dotsc, x_n] \tq \restr{ψ(p(x_1, \dotsc, x_n))}{X} = 0} \]

En otras palabras, $\ker ψ = \I(X)$. Una vez hecho esto ya podremos definir bien $Γ$ y darle la notación definitiva.

\begin{defn}[Anillo\IS de coordenadas de $X$] Sea $X ⊂ \afesp$. Entonces, se define el anillo de funciones regulares en $X$ o anillo de coordenadas de $X$ como \[ \K[X] ≝ \quot{\K[x_1, \dotsc, x_n]}{\I(X)} \simeq Γ(X, \K)\]
\end{defn}

Este anillo es siempre reducido porque $\I(X)$ es radical.

A través de este anillo podremos estudiar la variedad. Por ejemplo, podremos saber su dimensión, si es irreducible ($\K[X]$ debería de ser dominio de integridad) o si tiene picos.

De momento, lo que empezaremos estudiando serán los morfismos entre las variedades algebraicas a través de este anillo de coordenadas.

\section{Morfismos entre variedades algebraicas afines}

\begin{defn}[Morfismo\IS entre variedades algebraicas afines] Sean $X ⊆ \afesp$ y $Y ⊆ \afesp[\K][m]$ dos variedades algebraicas afines definidas sobre el mismo cuerpo $\K$. Un morfismo $\appl{φ}{X}{Y}$ es una función cuyas coordenadas son funciones regulares en $X$, de tal forma que \[ φ(a_1, \dotsc, a_n) = \left(f_1(a_1, \dotsc, a_n), \dotsc, f_m(a_1, \dotsc, a_n)\right)\] con $f_i$, $i = 1, \dotsc, m$ regulares en $X$.
\end{defn}

\begin{example}
La aplicación \begin{align*}
\appl{φ}{\afesp[\K][1]&}{\afesp[\K][2]} \\
(a) &\longmapsto (a,0)
\end{align*} es un morfismo entre variedades. En general, podríamos también considerar $(a) \longmapsto (p_1(a), p_2(a))$ con $p_i ∈ \K[\afesp[\K][1]] = \K[x]$ (esta última igualdad sólo si \K es infinito), que es lo que esperábamos que pasase.

Igualmente, una proyección \begin{align*}
\afesp[\K][3] &\longmapsto \afesp[\K][2] \\
(a_1, a_2, a_3) &\longmapsto (a_1, a_2)
\end{align*} también será un morfismo.
\end{example}

\notacion: Llamamos $\cls{x_i}$ a la i-ésima función coordenada, que es una función regular y que hace lo siguiente:

\begin{align*}
	\cls{x_i}: X & \rightarrow Y \\
	(a_1,...,a_n) & \rightarrow a_i
\end{align*}

Una cosa que esperaríamos es que los morfismos nos lleven funciones regulares en funciones regulares definidas sobre ambos anillos de coordenadas. Y como a veces el mundo matemático es justo, es precisamente lo que va a ocurrir.

\begin{prop} Sean $X ⊆ \afesp$ y $Y ⊆ \afesp[\K][m]$ dos variedades algebraicas afines definidas sobre el mismo cuerpo $\K$, y sea φ un morfismo $X \rightarrow Y$. Entonces φ induce un homomorfismo de anillos $\appl{f_φ}{\K[Y]}{\K[X]}$.
\end{prop}

\begin{proof} Dada una aplicación $g ∈ \K[Y]$, entonces $g ○ φ$ es una aplicación $\appl{g ○ φ}{X}{\K}$. Hay que demostrar que es un homomorfismo de anillos.

Como tratar de construir un homomorfismo de cocientes a cocientes es arriesgado, lo que haremos será definir un homomorfismo de $\K[y_1, \dotsc, y_m]$ a $\K[X]$ y después ver que factoriza por el cociente. El homomorfismo lo definiremos de la siguiente forma: dejaremos fijos los $r ∈ \K$, y cada variable $y_i$ la consideraremos como función regular en $\K[Y]$. Decíamos antes que la aplicación que definiríamos $X \longmapsto \K$ sería la composición con φ, así que lo que hacemos es ver qué ocurre cuando $g = \cls{y_i}$: \[
\begin{matrix}
X &\xrightarrow{φ}& Y &\xrightarrow{y_i}& \K \\
(a_1, \dotsc, a_n) &\longrightarrow & \left(f_1(a_1, \dotsc, a_n), \dotsc, f_m(a_1, \dotsc, a_n)\right) &\longrightarrow & f_i(a_1, \dotsc, a_n)
\end{matrix}\]

Así, el homomorfismo que definiremos será
\begin{align*}
f_{\varphi}: \K[y_1, \dotsc, y_m] &\longmapsto \K[X] = \quot{\K[x_1, \dotsc, x_n]}{\I(X)} \\
r ∈ \K &\longmapsto r ∈ \K \\
y_i &\longmapsto \cls{f_i(x_1,\dotsc, x_n)}
\end{align*}


%Clase 11/04/2016


Esta $\I(Y) \subset \ker(f_{\varphi})$?? Sabemos que $\I(Y) \subset \K[y_1,...,y_m]$. Sea $G \in \I(Y) \implies \forall (b_1,...,b_m) \in Y \implies G(b_1,...,b_m)=0$. Queremos ver que $f_{\varphi}(G)=0$.

\begin{example}
	$G(y_1,y_2)= y_1^2+2y_2 \rightarrow f_1^2+2f_2$
\end{example}

Entonces $f_{\varphi}(G)=G(f_1,...,f_m) \in \K[X]=\quot{\K[x_1,...,x_n]}{\I(X)}$ ¿Cuándo $G(f_1,...,f_m)=0$ en $\K[X]$?

$G(f_1,...,f_n)=0$ en $\K[X] \Leftrightarrow \forall (a_1,...,a_n) \in X$, $G(f_1,...,f_n)(a_1,...,a_n)=0 \Leftrightarrow G(f_1(a_1,...,a_n),...,f_m(a_1,...,a_n))=0 \; \forall(a_1,..,a_n) \in X$.

Sabiendo que $G \in \I(Y)$ podemos asegurar que $G(f_1(a_1,...,a_n),...,f_m(a_1,...,a_n))=0$ $\forall (a_1,...,a_n) \in X$??

Pues bien, tenemos
\begin{align*}
	X & \rightarrow Y \\
	(a_1,...,a_n) & \rightarrow (f_1(a_1,...,a_n),...,f_m(a_1,...,a_n))
\end{align*} 
Entonces $(f_1(a_1,...,a_n),...,f_m(a_1,...,a_n)) \in Y \; \forall (a_1,...,a_n) \in X \implies G\underbrace{(f_1(a_1,...,a_n),...,f_m(a_1,...,a_n))}_{\in Y}=0$ ya que $G \in \I(Y) \implies $
\begin{align*}
	\K[y_1,...,y_m] & \stackrel{f_{\varphi}}{\rightarrow} \K[X] \\
	y_i & \rightarrow f_i
\end{align*}

Factoriza por $\quot{\K[y_1,..,y_n]}{\I(Y)}$ y por tanto induce un homomorfismo de anillos $\cls{f_{\varphi}}: \K[Y] \rightarrow K[X]$
\end{proof}


\begin{example}
	Cogemos $X=\A_K^2$ e $Y=\A_K^3$:
	\begin{align*}
		\A_K^2 & \rightarrow \A_K^3 \\
		(a_1,a_2) & \rightarrow (a_1,a_2,0)
	\end{align*}
	
	Tendríamos que: \textcolor{red}{revisar} $$\K[X]=\quot{\K[x_1,x_2]}{\I(X)}=\quot{\K[x_1,x_2]}{\{0\}} = \K[x_1,x_2]$$
	
	$$ K[Y]=\quot{\K[y_1,y_2,y_3]}{\I(X)}=\quot{\K[y_1,y_2,y_3]}{\{0\}} = \K[y_1,y_2,y_3] $$
	
	Y según el teorema tenemos el siguiente homomorfismo en anillos de coordenadas:
	\begin{align*}
		K[y_1,y_2,y_3] & \rightarrow \K[x_1,x_2] \\
		y_1 & \rightarrow x_1 \\
		y_2 & \rightarrow x_2 \\
		y_3 & \rightarrow 0 \\
	\end{align*}
\end{example}
	
	
	\begin{example}
		\begin{align*}
			\A_K^3 & \rightarrow \A_K^2 \\
			(a_1,a_2,a_3) & \rightarrow (a_1,a_2)
		\end{align*}
		
		Y según el teorema tenemos el siguiente homomorfismo en anillos de coordenadas:
		\begin{align*}
			K[y_1,y_2] & \rightarrow \K[x_1,x_2,x_3] \\
			y_1 & \rightarrow x_1 \\
			y_2 & \rightarrow x_2 \\
		\end{align*}
	\end{example}
	
	\begin{example}
		Cogemos la hipérbola $H=\{x_1x_2-1=0\}$, cogemos un punto y lo dejamos caer.
		\begin{align*}
			H=\{x_1x_2-1=0\} & \rightarrow \A_K^1 \\
			(a_1,a_2) & \rightarrow a_1
		\end{align*}
		
		Y según el teorema tenemos el siguiente homomorfismo en anillos de coordenadas:
		\begin{align*}
			K[y_1] & \rightarrow \quot{\K[x_1,x_2]}{\gen{x_1x_2-1}} \simeq \K[x_1]_{\{x_1\}} \\
			y_1 & \rightarrow x_1 \\
		\end{align*}
		
		Lo de que $\quot{\K[x_1,x_2]}{\gen{x_1x_2-1}} \simeq \K[x_1]_{\{x_1\}}$ parece un poco by the face, pero vamos, es básicamente que al poner $x_1x_2-1$ lo que estamos haciendo realmente es añadir a $\K[x_1]$ los elementos inversos de $x_1$, que es precisamente lo que significa localizar en $x_1$.
		
		El homomorfismo que hemos definido además es una inclusión, es decir $K[y_1] \subset \quot{\K[x_1,x_2]}{\gen{x_1x_2-1}}$. Si no fuera por el cociente lo veríamos clarísisisisisimo, el cociente jode un poco, pero podemos decir que es una inclusión porque si no lo fuese habría algun polinomio no nulo en $Y$ cuya imagen caería dentro del ideal $\gen{x_1x_2-1}$. Pero un polinonmio no nulo en $Y$ es un polinomio no nulo en $x_1$ y estaríamos diciendo que ese polinomio pertenecería a $\gen{x_1x_2-1}$, lo cual es imposible porque no tendríamos $x_2$.  Otra forma de verlo más es sencillo es con lo de que $\quot{\K[x_1,x_2]}{\gen{x_1x_2-1}} \simeq \K[x_1]_{\{x_1\}}$, aquí se ve mas claro, pues estoy añadiendo a $\K[x_1]$ más elementos.
		
		Es inyectiva pero no es sobreyectiva %(PERO POR POCOO! ya que el único punto al que no llego es al 0).
	\end{example}
	
	\begin{example}
		Cogemos la parábola $X=\{x_2^2-x_1=0\}$, cogemos un punto y lo dejamos caer.
		\begin{align*}
			X=\{x_2^2-x_1=0\} & \rightarrow \A_K^1 \\
			(a_1,a_2) & \rightarrow a_1
		\end{align*}
		
		Y según el teorema tenemos el siguiente homomorfismo en anillos de coordenadas:
		\begin{align*}
			K[y_1] & \rightarrow \quot{\K[x_1,x_2]}{\gen{x_2^2-x_1}} \\
			y_1 & \rightarrow x_1
		\end{align*}
		
		El homomorfismo es una inclusión como antes. $x_2$ es entero sobre $\K[x_1]$. Es decir $\cls{x_2}$ es raíz del $p(t)=t^2-x_1$, $p(t)\in \K[x_1][t] \implies \quot{\K[x_1,x_2]}{\gen{x_2^2-x_1}}$ es una extensión finita (y por tanto entera) de $\K[y_1]$ vía f.
	\end{example}
	
	\begin{prop}
		Sean $X \subseteq \Akn$ e $Y \subseteq \Akn$ v.a.a.. Todo homomorfismo de anillos $f: \K[Y] \rightarrow \K[X]$ induce un morfismo $\varphi_g: X \rightarrow Y$ de K-álgebras (los elementos de $\K$ se quedan fijos).
	\end{prop}
	
	\begin{proof}		
		Tengo lo siguiente:
		
		\begin{tikzcd}
			\quot{\K[y_1,...,y_m]}{\I(Y)} \arrow[rightarrow]{rr}{f}
			\arrow[leftarrow]{dr}{}
			& & \begin{matrix}
				\quot{\K[x_1,...,x_n]}{\I(X)} \\ f_1
			\end{matrix}  \arrow[leftarrow]{dl}\\
			& \begin{matrix}
				\K[y_1,...,y_m] \\y_1
			\end{matrix}&
		\end{tikzcd}
		
		Queremos inducir un homomorfismo $\varphi_g: X \rightarrow Y$.
		
		Si escogemos m funciones regulares tendremos:
		
		
		\begin{align*}
			X & \rightarrow \A_K^m \\
			(a_1,...,a_n) & \rightarrow (f_1(a_1,...,a_n),...,f_m(a_1,...,a_n))
		\end{align*}
		
		Entonces queremos probar que $(f_1(a_1,...,a_n),...,f_m(a_1,...,a_n)) \in Y$. Es decir, si $(a_1,...,a_n) \in X \implies (f_1(a_1,...,a_n),...,f_m(a_1,...,a_n)) \in Y$???
		
		Entonces $(f_1(a_1,...,a_n),...,f_m(a_1,...,a_n)) \in Y \Leftrightarrow \forall G \in \I(Y)$, $G(f_1(a_1,...,a_n),...,f_m(a_1,...,a_n))=0$.
		
		Sea $G\in \I(Y)$ queremos ver si $G(f_1(a_1,...,a_n),...,f_m(a_1,...,a_n))=0$.
		
		$$G(f_1(a_1,...,a_n),...,f_m(a_1,...,a_n))=G(f_1,...,fm)(a_1,...,a_m)= \underbrace{\hat{g}(G)}_{=0 \text{ porque } \I(Y) \subset \ker(g)}(a_1,...,a_m)$$
		
		$\hat{g}(G) \in \ker \hat{g} \implies \hat{g}(G) \equiv 0$ en $X$ $\implies \forall(a_1,...,a_n) \in X$, $\hat{g}(G)(a_1,...,a_n)=0$
	\end{proof}
	








