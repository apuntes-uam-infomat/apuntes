% -*- root: ../AlgebraConmutativa.tex -*-

\chapter{La dimensión de las variedades algebraicas afines}

Esquema de contenidos del tema:

\begin{itemize}
	\item Definición intuitiva de lo que debería ser la dimensión de una v.a.a.
	\begin{itemize}
		\item Traducción de la noción anterior en términos de lo que vamos a llamar dimensión de Krull de un anillo.
		\item Nos vamos a creer que $\dk\K[x_1,...,x_n]=n$ (dimensión de Krull de un anillo de polinomios con coeficientes en un cuerpo).
	\end{itemize}
	\item Veremos que si tenemos $A \subset B$, una extensión entera de anillos, entonces $ \dk A = \dk B$
	\item Vamos a ver el \textbf{Lema de Normalización de Noether:} Toda v.a.a. X se puede proyectar de manera finita sobre un espacio afín $\Akn$, donde $n=\dk\K[X]$. (siendo $\K[X]$ un anillo de coordenadas) %(Enunciado geométrico)
	\item Y si nos diese tiempo veríamos el teorema que nos dice que $\dk\K[x_1,...,x_n]=n$.
\end{itemize}

\section{Definición intuitiva de lo que debería ser la dimensión de una v.a.a.}

Por ejemplo, si tenemos un espacio vectorial V sobre un cuerpo $\K$, para saber la dimensión de V buscamos una base $B=\{ v_1,...,v_n\}$ que es una serie de vectores linealmente independientes que generan el espacio.

Cada vez que me dan una base puedo construir una cadena de subespacios vectoriales:
$\{0\} \subsetneqq \gen{v_1} \subsetneqq \gen{v_1,v_2} \subsetneqq ... \subsetneqq \gen{v_1,...,v_n}$. Entonces esta cadena es la más larga dentro de $V$, y la longitud de esta cadena se dice que es n (el primero se considera 0). Por tanto $n= \dim V$

Vamos a usar el mismo razonamiento para calcular la dimensión de $X$ v.a.a.. Vamos a suponer $X$ es irreducible y a considerar dentro de $X$ subvariedades y construir la cadena más larga dentro de $X$. Consideramos dentro de $X$ cadenas estrictamente crecientes de subvariedades irreducibles. Tendríamos:

$$Y_0 \subsetneqq Y_1 \subsetneqq ... \subsetneqq Y_n=X$$

Entonces podríamos definir $\dim X$ como la longitud de la cadena más larga.

Si $X$ no fuese irreducible se separa en variedades irreducibles y se le asigna como dimensión el máximo de las longitudes.

\subsection{Traducción de la noción de dimension en términos del álgebra conmutativa}

Tenemos $X \subseteq  \Akn$, y tenemos también $\I(X) \subseteq \K[x_1,...,x_n]$

La cadena anterior de ideales irreducibles se traduce en una cadena de ideales  primos (por la proposición \ref{prop:VaaIrreducibleIdPrimo}):

$$\I(Y_0) \supsetneqq \I(Y_1) \supsetneqq ... \supsetneqq \I(Y_n)=\I(X)$$

Pero estas cadenas (que contienen a $\I(X)$) se corresponden biyectivamente con cadenas de ideales primos en $\K[X]=\quot{\K[x_1,...,x_n]}{\I(X)}$ . Así, para definir la dimensión de $X$ miraremos al máximo de las longitudes de las cadenas  de ideales primos en $\K[X]$.

Esto es bastante fuerte ya que vimos que dada una v.a.a. $X$ el anillo de coordenadas es el mismo sea donde sea que viva $X$ (dando igual que la $X$ viva en un espacio afín o en otro). Imaginaos que $X$ es una curva en $A^3$, la misma curva la puedo ver en $A^4$, su anillo de coordenadas es el mismo. \textcolor{red}{Donde hemos visto esto?}

\textcolor{red}{Poner reflexion de anillos de coordenadas unicos para variedades dando igual donde viven}

\begin{defn}
	Sea $A$ un anillo y sea $p_0 \subsetneqq p_1 \subsetneqq ... \subsetneqq p_r$ una cadena estrictamente creciente de ideales primos en $A$. En tal caso, diremos que la longitud de la cadena es $r$, y definimos $\dk A$ como el máximo (si existe) de las longitudes tras considerar todas las posibles cadenas de ideales primos en $A$:

	$$ \dk A = \max \left\{ r: \exists \text{ cadena } p_0 \subsetneqq p_1 \subsetneqq ... \subsetneqq p_r \text{ con } p_i \text{ primo } \subset A \right\} $$
\end{defn}

\begin{example}
	Sea $\K$ un cuerpo (su único ideal primo es el $\{0\}$), la cadena más larga que puedo montar es $\{0\}$, por tanto: $\dk \K = 0$
\end{example}

\begin{example}
	Sea $\ent$, los ideales primos en $\ent$ son el $\{0\}$ y los generados por números primos, por tanto las cadenas serán:

	$$ \{0\} \subsetneqq \gen{p} $$

	Pero $\gen{p}$ es maximal así que ya no puedo hacer la cadena más larga, luego $\dk \ent = 1$
\end{example}

\begin{example}
	Sea $\K[x]$ con $\K$ cuerpo. Construyo la cadena:

	$$\{0\} \subsetneqq  \gen{p(x)} $$

	Con $p(x)$ irreducible, pero $\gen{p(x)}$ es maximal así que ya no puedo seguir. Luego  $\dk \K[x] = 1$

	Además $\K[x]$ es el anillo de coordenadas de $\A^1_K$, que esperamos que tenga dimensión 1.
\end{example}

\begin{example}
	Sea $\K[x,y]$:

	$$\{0\} \subsetneqq  \gen{x} \subsetneqq \gen{x,y} $$

	Por tanto  $\dk \K[x,y] \geq 2$
\end{example}

	Nos vamos a creer lo siguiente: \textbf{Teorema: Sea $\K$ un cuerpo, entonces $\dk \K[x_1,...,x_n]=n$}

	\begin{defn}[Dimension de una v.a.a.]
		Sea $X$ una v.a.a. definimos $\dim X = \dk \K[X]$
	\end{defn}

\section{Dimensión de Krull en el caso de extensiones enteras de anillos}

Recordatorio extensiones enteras:

Tenemos $A \hookrightarrow B$ ($A \subset B$) extensión de anillos. Cada elemento de b es raíz de un polinomio mónico de elementos en A.
\begin{itemize}
	\item Si $A \subset B$ es una extensión entera y si $J \subset B$ es un ideal, entonces $\quot{A}{J \cap A} \hookrightarrow \quot{B}{J}$ es entera.
	\item Si $S \subset A$ es una parte multiplicativa (conjunto multiplicativamente cerrado) entonces $S^{-1}A \hookrightarrow S^{-1}B$ es una extensión entera.
	\item Si fijamos un ideal primo $p \subset A$ y tomamos $S=A \setminus p$, entonces $A_p=S^{-1}A$ es un anillo local con maximal $pA_p$
	\item \textbf{Teorema:} Sea $A \subset B$ una extensión entera de dominios de integridad. Entonces $A$ es un cuerpo  $\Leftrightarrow B$ es un cuerpo.
\end{itemize}

\begin{prop}
	Sea $A \subset B$ una extensión entera y sea $q \subset B$ un ideal primo (si tengo un homomorfismo de A a B, $f^{-1}(ideal)=ideal$) entonces $q \cap A$ es maximal $\Leftrightarrow$ q es maximal.

\end{prop}

\begin{proof}
	Sea $A \subset B$ entera $\implies \underbrace{\quot{A}{q \cap A}}{\text{D.I}} \subset \underbrace{\quot{B}{q}}{\text{D.I}}$ entera.

	Por el teorema anterior $\quot{A}{q \cap A}$ es un cuerpo $\Leftrightarrow$ $\quot{B}{q}$ es un cuerpo. Por lo tanto q es maximal si y solo si $q \cap A$ es maximal.
\end{proof}


Vamos a ver ejemplos para ver que la hipótesis sobre la extensión es necesaria:
\begin{example}
	$\ent \subset \rac$ es algebraica pero no entera. En $\rac$ cogemos el único primo que hay que es el $\{0\}$ que además es maximal. Ahora miramos $\ent \cap \{0\}=\{0\}$ que es un primo no maximal.
\end{example}

\begin{example}
	$\ent \subset \ent[x]$ y cogemos $p=\gen{3}=3 \ent[x] \in \ent[x]$. Este ideal es primo ya que al pasar al cociente me sale $\quot{\ent[x]}{\gen{3}}=F_3[x]$ un dominio de integridad, pero no un cuerpo. Por tanto tengo un primo no maximal.

	Ahora hago $\ent \cap 3 \ent[x]=3 \ent$ que es un primo maximal.
\end{example}

\begin{prop}
	Sea $A \subset B$ una extensión entera, y sea $p \subset A$ un ideal primo. Entonces existe un ideal primo $q \subset B$ tal que $q \cap A=p$.
\end{prop}

\begin{proof}
	$A \subset B$. Sea un primo $p \subset A$. Buscamos un primo $q \subset B$.

	Definimos la siguiente parte multiplicativa. Sea $S=A \setminus p$ una parte multiplicativa en A y en B.

	\begin{tikzcd}
		A \arrow[hookrightarrow]{r}{entera}
		\arrow[rightarrow]{d}{\varphi}
		\arrow[rightarrow]{dr}{f (h)}
		&  B
		\arrow[rightarrow]{d}{g}\\
		S^{-1}A=Ap \arrow[hookrightarrow]{r}{entera} & S^{-1}B
	\end{tikzcd}

	$Ap$ es un anillo local. ideal maximal $pA_p$.

	De modo que:
	\begin{enumerate}
	 	\item $S^{-1}B$ tiene algún ideal maximal $m \subset S^{-1}B$
	 	\item $m \cap A_p$ es un ideal maximal luego necesariamente $m \cap A_p=pA_p$.
		\item $\varphi^{-1}(pA_p)=p$
		\item $f^{-1}(m)=p$ (usando 2 y 3).
		\item $g^{-1}(m)=q$ es un ideal primo en $B$.
		\item $q \cap A = p$ porque el diagrama es conmutativo. $h^{-1}(m)=f^{-1}(m)$.
	\end{enumerate}
\end{proof}

Vamos a ver ejemplos sobre la necesidad de la hipótesis de que la extensión sea entera.
\begin{example}
	$\ent \subset \rac $. Cojo $\gen{3} \in \ent$ primo, no existe ningún ideal primo $q \subset \rac$ tal que $q \cap \ent = \gen{3}$.
\end{example}

\begin{example}
	Sea $\real[x] \subset \quot{\real[x,y]}{\gen{xy-1}}$. El ideal $\gen{x}$ es un ideal primo en $\real[x,y]$, y no existe ningún ideal primo $q \subset  \quot{\real[x,y]}{\gen{xy-1}}$ tal que $q \cap \real[x]=\gen{x}$.

	Para ello nos basta probar que $\nexists$ ningún ideal primo en $ \quot{\real[x,y]}{\gen{xy-1}}$ que contenga a $\gen{\cls{x}}^e$ (extendido de x).

	$$ \gen{\cls{x}}^e=\gen{1}$$

	\begin{align*}
		H=\{xy-1=0\} & \rightarrow \A^1_K \\
		(a,b) & \rightarrow (a)
	\end{align*}

	La fibra sobre $\{0\}$ es vacía.

	Luego $\real[x] \subset \quot{\real[x,y]}{\gen{xy-1}}$ no puede ser una extensión entera.
\end{example}

\begin{prop}
	Sea $A \subset B$ una extensión entera. Sea $p \subset A$ un ideal primo. Sean $q_1 \subseteq q_2 \subset B$ dos ideales primos tales que $q_1 \cap A=q_2 \cap A = p$. Entonces $q_1=q_2$.
\end{prop}

\begin{proof}
	Sea $S = A \setminus p$

	\begin{tikzcd}
		A \arrow[hookrightarrow]{r}{entera}
		\arrow[rightarrow]{d}{\varphi}
		\arrow[rightarrow]{dr}{f (h)}
		&  B
		\arrow[rightarrow]{d}{g}\\
		S^{-1}A=Ap \arrow[hookrightarrow]{r}{entera} & S^{-1}B
	\end{tikzcd}

	\begin{enumerate}
		\item 	Como $S= A \setminus p$ entonces $q_1 \cap S = q_2 \cap S = \emptyset$
		\item Por 1, $q_1S^{-1}B$ es primo y $q_2S^{-1}B$ es primo y $q_1S^{-1}B  \subseteq q_2S^{-1}B$. Ambos pertenece a $S^{-1}B$.

		\begin{tikzcd}
			A \arrow[hookrightarrow]{r}{entera}
			\arrow[rightarrow]{d}{\varphi}
			\arrow[rightarrow]{dr}{f (h)}
			&  B
			\arrow[rightarrow]{d}{g}\\
			\underbrace{S^{-1}A=Ap}{pA_p} \arrow[hookrightarrow]{r}{entera} & \underbrace{S^{-1}B}{q_1S^{-1}B  \subseteq q_2S^{-1}B}
		\end{tikzcd}
		\item $q_1S^{-1}B \cap A_p=pA_p=q_2S^{-1}B \cap A_p$
		\item $pA_p$ es maximal $\implies q_1S^{-1}B$ y $q_2S^{-1}B$ son maximales, luego $q_1S^{-1}B = q_2S^{-1}B$.

		Tenemos que $q_1=g^{-1}(q_1S^{-1}B)$ y $q_2=g^{-1}(q_2S^{-1}B) \implies q_1 = q_2$
	\end{enumerate}

	Ejemplo para ver la necesidad de las hipótesis:

	\begin{example}
		$\ent \subset \ent[x]$. Cojo $\gen{3} \in \ent$ y $\gen{3} \subsetneqq \gen{3,x} \in \ent[x]$ maximal.
	\end{example}

	\begin{example}
		\begin{align*}
			\real[x] & \hookrightarrow \real[x,y] \\
			\gen{x} & \gen{x} \subseteq \gen{x,y}
		\end{align*}

		\begin{align*}
			\A^2 & \rightarrow \A^1 \\
			(a,b)& \rightarrow a
		\end{align*}

	\end{example}

\end{proof}

\begin{corol}
	Sea $A \subset B$ entera. Entonces $\dk A \geq \dk B$
\end{corol}

\begin{proof}
	Sea $q_0 \subsetneqq q_1 \subsetneqq ... \subsetneqq q_r \subset B$ una cadena de ideales prmos en B. Entonces $q_0\cap A \subsetneqq q_1\cap A \subsetneqq ... \subsetneqq q_r\cap A \subset A$
\end{proof}

%Clase 25/04/2016

\begin{theorem}[Going-up]
	Sea $q_0 \subsetneqq q_1 \subsetneqq ... \subsetneqq q_q \subsetneqq q_{s+1} \subsetneqq ... \subsetneqq q_n$ una cadena de primos en A, y sea $p_0 \subsetneqq p_2 \subsetneqq ... \subsetneqq p_s$ una cadena de primos en B con $p_i \cap A=q_i $ para $i=0,...,s$. Entonces existen primos $p_{s+1} \subsetneqq ... \subsetneqq p_n$ en $B$ tal que $p_s \subsetneqq p_{s+1}$ y $pj \cap A = q_i$ para $j=s+1,...,n$.
\end{theorem}

\begin{proof}
	Podemos suponer que entonces estamos en el caso siguiente:
	\begin{itemize}
		\item $q_0 \subsetneqq q_1$ primos en A.
		\item $p_0$ primo en B. Y se que $p_0 \cap A=q_0$
	\end{itemize}

	¿Cómo encontramos $p_1 \subset B$ primo tal que $p_1 \cap A=q_1$ y tal que $p_1 \supsetneqq p_0$?

	\begin{tikzcd}
		A \arrow[hookrightarrow]{}{entera}
		\arrow[rightarrow]{d}{\pi}
		\arrow[rightarrow]{dr}{f}
		&  B
		\arrow[rightarrow]{d}{g}\\
		\quot{A}{p_0 \cap B} \arrow[hookrightarrow]{r}{entera} & \quot{B}{p_0}
	\end{tikzcd}

	$P_0 \cap B = q_0$. Primo $\cls{q_1}=\quot{q_1}{q_0}$.

	Observamos que:
	\begin{enumerate}
		\item $\cls{q_1}=\quot{q_1}{q_0}$ es primo en $\quot{A}{q_0}$
		\item Como $\quot{A}{q_0} \hookrightarrow \quot{A}{p_0}$ es entera, entocnes existe $\cls{p_1} \subset \quot{B}{p_0}$ tal que $\cls{p_1} \cap \quot{A}{q_0}=\cls{q_1}$.
		\item $f^{-1}(\cls{p_1})=q_1$
		\item $\cls{p_1}$ esta en correspondencia biyectiva con un único primo $p_1 \subset B$ que contiene a $p_0$. $\pi{-1}(\cls{p_1})=p_1$, $p_1 \supseteqq p_0$.
		\item $p_1 \cap A=q_1$ porque el diagrama conmuta.
	\end{enumerate}
\end{proof}

\obs
$A \hookrightarrow B$ extensión entera y sea $J \subsetneqq B$ un ideal. Afirmación:

\begin{proof}
$$\quot{A}{J \cap A} \hookrightarrow \quot{B}{J}$$

Es una extension entera porque:

Es inyectiva:

$$ A \hookrightarrow B \rightarrow \quot{B}{J}$$ y la composición g es un homomorfismo. Ahora tenemos:

$$\quot{A}{\ker g} \hookrightarrow \quot{B}{J}$$

Luego $\ker A=A \cap J$. Ahora vamos a ver que:

$$ \quot{A}{J \cap A} \hookrightarrow \quot{B}{J}$$ es una extensión entera.

Sea $\cls{b} \in \quot{B}{J}$, existe $q(x) \in \quot{A}{J \cap A}[x]$ mónico tal que $q(\cls{b})=0$.

\begin{tikzcd}
	A \arrow[hookrightarrow]{}{entera}
	\arrow[rightarrow]{d}{\pi}
	\arrow[rightarrow]{dr}{f}
	&  B
	\arrow[rightarrow]{d}{g}\\
	\quot{A}{J \cap A} \arrow[hookrightarrow]{r}{entera} & \quot{B}{J }
\end{tikzcd}

Sea $b \in B$ un elemnto tal que $b \equiv \cls{b} mod J$. Como b es entero sobre A entonces existe $p(x) \in A[x]$ mónico tal que $p(b)=0$.

$p(x) = a_0+a_1x+...+a_{n-1}x^{n-1}+x^n$ con $a_1 \in A$. Luego $p(b) = a_0+a_1b+...+a_{n-1}b^{n-1}+b^n=0$ en B. Entonces: $a_0+a_1b+...+a_{n-1}b^{n-1}+b^n=\cls{0}$ en $\quot{B}{J}$ y por tanto $\cls{a_0}+\cls{a_1}\cls{b}+...+\cls{a_{n-1}}\cls{b^{n-1}}+\cls{b}^n=\cls{0}$  en $\quot{B}{J}$. $\cls{a_i} \in \quot{A}{J \cap A}$ basta tomar $q(x)=\cls{a_0}+\cls{a_1}x+...+\cls{a_{n-1}}x^{n-1}+x^n=\cls{0}$
\end{proof}

\begin{corol}
	Si $A \subset B$ es entera entonces $\dk B \geq \dk A$
\end{corol}

\textbf{conclusion} Si $A \subset B$ es entera entonces:
$$ \dk A = \dk B $$

\begin{example}
	Sea $\real \subset \quot{\real[x]}{\gen{x(x^2-1)}}$ entonces es como añadirle $\cls{x}$.

	$\cls{x}$ es entero sbre $\real$ porque es raíz de $p(T)=T^3-T \in \real[T]$ Como $\cls{x}$ es entero sobre $\real$ entonces $\real \hookrightarrow \quot{\real[x]}{\gen{x^3-x}}$ es finita, por tanto es entera.

	\textcolor{red}{incluir duda del tio del primera fila}

	$$ \dk \quot{\real[x]}{\gen{x^3-x}} = 0$$

	Toda cadena de primos en $\quot{\real[x]}{\gen{x^3-x}}$ tiene longitud 0. Luego en $\quot{\real[x]}{\gen{x^3-x}}$ todos los primos son maximales y minimales. ¿Cuántos puede haber?. Sólo puede haber un número finito de primos porque es un anillo noetheriano (cociente de noetheriano es noetheriano, y en su día dijimos que en un anillo noetheriano sólo hay un número finito de primos minimales)

	Cuáles son los primos de $\quot{\real[x]}{\gen{x^3-x}}$, pues estan en correspondencia biyectiva con los de $\real[x]$ que contienen a $\gen{x^3-x}$,, osea que sólo son $\gen{\cls{x}}$, $\gen{\cls{x-1}}$, $\gen{\cls{x+1}}$.
\end{example}


\subsection{Interpretación geométrica}

\begin{defn}
	Sean $X,Y$ v.a.a. y sea $\varphi: X \rightarrow Y$ un morfismo. Diremos que $\varphi$ es un morfismo finito si $\varphi^*: \K[Y] \hookrightarrow \K[X]$ es una extensión finita (y por tanto entera).
\end{defn}

\begin{example}
	Sea $|\K|=\infty$, y sea;
	\begin{align*}
		\varphi: P=\{x_2 2-x_1=0 \}& \rightarrow \A^1 \\
		(a,b) & \rightarrow a
	\end{align*}

	 entonces es finita.

	 Porque:
	 \begin{align*}
	 	\varphi^*: \K[y_1]& \rightarrow \quot{K[x_1,x_2]}{\gen{x_2^2-x_1}} \\
	 	y_1 & \rightarrow \cls{x_1}
	 \end{align*}
\end{example}

\subsection{Morfismos finitos}

$\varphi: X \rightarrow Y$ es finito si $\varphi^*: \K[Y] \hookrightarrow \K[X]$ es finito. Es particular $\K[Y] \subset \K[X]$ es una extensión entera.

\begin{theorem}
	Sea $\varphi: X \rightarrow Y$ un morfismo finito. Entonces:
	\begin{enumerate}
		\item $\varphi$ es un morfismo sobreyectivo.
		\item Para cada punto $y \in Y$, el número de preimágenes $\{ \varphi^{-1}(y) \}$ es un número finito de puntos.
	\end{enumerate}
\end{theorem}
%#

\begin{proof}
	Suponemos que $\K$ es algebraicamente cerrado (para simplifcar la demostración, pero vamos que no es una condición necesaria).

	1) Sea $(b_1,...,b_m) \in Y$, queremos probar que existe un punto $(a_1,...,a_n) \in X$ tal que $\varphi(a_1,...,a_n)=(b_1,...,b_m)$. Vamos a calcular la fibra de $\varphi$ sobre $(b_1,...,b_m)$. Para ello consideramos el ideal de definición del punto $(b_1,...,b_m)$ que es $\gen{y_1-b_1,....,y_m-b_m}$

	Miramos a:

	\begin{align*}
		\varphi^*: \K[Y]  \quot{\K[y_1,...,y_m]}{\I(Y)}& \rightarrow K[X] = \quot{\K[x_1,...,x_n]}{\I(X)} \\
		\gen{y_1-b_1,....,y_m-b_m} & \rightarrow \varphi^*(\gen{y_1-b_1,....,y_m-b_m})^e=J
	\end{align*}

	Lo que probamos es que $\varphi^{-1}(b_1,...,b_m)= \V(\varphi^*(\gen{y_1-b_1,....,y_m-b_m})^e)$ queremos ver que es no vacío.

	Como $\K[Y] \hookrightarrow \K[X]$ es entera, entonces $\quot{\K[Y]}{\underbrace{J \cap \K[Y]}_{(\varphi^*)^{-1}(J)}}$.

	Por tanto ambas tienen la misma $\dk$

	En tesumen tenemos:
	\begin{align*}
		\varphi^*: \K[Y]  \quot{\K[y_1,...,y_m]}{\I(Y)}& \rightarrow K[X] = \quot{\K[x_1,...,x_n]}{\I(X)} \\
		\gen{\cls{y_1-b_1},....,\cls{_m-b_m}} & \rightarrow \varphi^*(\gen{y_1-b_1,....,y_m-b_m})^e=J
	\end{align*}

	Entonces $\gen{\cls{y_1-b_1},....,\cls{_m-b_m}} \subset (\varphi^*)^{-1}(J)$, pero es más afirmamos que son iguales.

	Supongamos que no, entonces $(\varphi^*)^{-1}(J)=\gen{\cls{1}}$.

	Ahora:

	\begin{enumerate}
		\item Como $\K[Y] \hookrightarrow \K[X]$ es entera, existe un primo $p \subset \K[X]$ tal que $\varphi^{-1}(p)=	\gen{\cls{y_1-b_1},....,\cls{_m-b_m}}$

		Pero si yo tengo:
		\begin{align*}
			\varphi^*: \K[Y] =  \quot{\K[y_1,...,y_m]}{\I(Y)}& \rightarrow K[X] = \quot{\K[x_1,...,x_n]}{\I(X)} \\
			\gen{\cls{y_1-b_1},....,\cls{_m-b_m}} & \rightarrow p
		\end{align*}

		Entonces $\underbrace{\varphi^*(\gen{y_1-b_1,....,y_m-b_m})^e}_{J} \subset p$.

		Entonces tendríamos que $(\varphi^*)^{-1}(J) \subset (\varphi^*)^{-1}(p)=\gen{\cls{y_1-b_1},....,\cls{_m-b_m}}$. Entonces $(\varphi^*)^{-1}(J) \neq \gen{1}$ y es más: $(\varphi^*)^{-1}(J) = \gen{\cls{y_1-b_1},....,\cls{_m-b_m}}$.

		Esto significa que $J \neq \gen{1} \underbrace{\implies}_{\K = \cls{\K}} \V(J) \neq \emptyset \implies \varphi^{-1}((b_1,...,b_m)) \neq \emptyset \implies \varphi$ es sobreyectiva.

		\item Veamos que $\varphi^{-1}((b_1,...,b_m))$ consta de un número finito de puntos.

		\begin{align*}
			\varphi^*: \K[Y]  \quot{\K[y_1,...,y_m]}{\I(Y)}& \rightarrow K[X] = \quot{\K[x_1,...,x_n]}{\I(X)} \\
			\gen{\cls{y_1-b_1},....,\cls{_m-b_m}} & \rightarrow p
		\end{align*}

		\begin{tikzcd}
			\K[Y] \arrow[hookrightarrow]{}{entera}
			\arrow[rightarrow]{d}{\varphi^*}
			&  \K[X]
			\arrow[rightarrow]{d}{g}\\
			\quot{\K[Y]}{J \cap \K[Y]} \arrow[hookrightarrow]{r}{entera} & \quot{\K[X]}{J }
		\end{tikzcd}

		De aquí concluimos que $\dk \quot{\K[X]}{J}=0 \implies$ todos los ideales de $\quot{\K[X]}{J}$ son minimales y maximales, y como $\quot{\K[X]}{J}$ es noetheriano $\implies \quot{\K[X]}{J}$ solo tiene un número finito de primos (y son todos maximales) de la punta:

		$$ \gen{x_1-a_{11},...,x_n-a_{1n}},...,\gen{x_1-a_{s1},...,x_n-a_{sn}} $$

		Entonces $\varphi^{-1}((b_1,...,b_m))= \V(J)=\{ (a_{11},...,a_{1n}),...,(a_{s1},...,a_{sn}) \}$
	\end{enumerate}
\end{proof}

\begin{example}

	$|K| = \infty$
	\begin{align*}
		\varphi^*: \K[x,y]& \rightarrow \quot{K[x,y,z]}{\gen{z^4+xyz+3}} \\
		x & \rightarrow \cls{x} \\
		y & \rightarrow \cls{y} \\
	\end{align*}
	Entonces $\cls{z}$ es raíz de $T^4+xyT+3 \in \K[x,y][T]$

	\begin{align*}
		\varphi: X=\{ z^4+xyz+3=0  \} & \rightarrow \A^2 \\
		(a,b,c) & \rightarrow (a,b)
	\end{align*}

	Suponemos $\K= \cls{\K}$ ($\K$ algebraicamente cerrado).

	Caculamos la fibra sobre (0,0):

	\begin{align*}
		\gen{x,y} & \rightarrow \varphi^*(\gen{x,y})^e \\
	\end{align*}

	$z^4+xyz+3=0$ con $x=0$ e $y=0$, entonces $z^4+3=0$ hay 4 maximales:

	$$\gen{x,y,z- \sqrt[4]{3}} \Leftrightarrow (0,0,\sqrt[4]{3})$$

	$$\gen{x,y,z- \sqrt[4]{3}e^{\frac{2 \pi i}{4}}} \Leftrightarrow (0,0,\sqrt[4]{3}e^{\frac{2 \pi i}{4}}$$

	$$\gen{x,y,z- \sqrt[4]{3}e^{\frac{4 \pi i}{4}}}$$

	$$\gen{x,y,z- \sqrt[4]{3}e^{\frac{6 \pi i}{4}}}$$


	Podríamos haber calculado la fibra sobre (2,1):
	\begin{align*}
		\gen{x-2,y-1} & \rightarrow \varphi^*(\gen{x-2,y-1})^e \\
	\end{align*}

	Y tendríamos que resolver $z^4+xyz+3=0$ con $x=2$ e $y=1$. Y resolveríamos $z^4+2z+3=0$ y tendríamos la siguiente solución:
	$$\varphi^{-1}((2,1))=\{ (2,1,\alpha_1), (2,1,\alpha_2),(2,1,\alpha_3), (2,1,\alpha_4) \}$$
\end{example}

\begin{example}
	Consideramos $|K| = \infty$

	\begin{align*}
		\varphi^*: \K[x]& \rightarrow \quot{K[x,y]}{\gen{xy}} \\
		x & \rightarrow \cls{x}
	\end{align*}

	\begin{align*}
		\varphi: X=\{ xy=0  \} & \rightarrow \A^1_K \\
		(a,b) & \rightarrow a
	\end{align*}

	Vemos que la fibra sobre el $0$ es toda una recta  y sobre cualquier otro punto es sólo un punto.

	Fibra sobre (0):
	\begin{align*}
		\gen{x} & \rightarrow \varphi^*(\gen{x})^e \\
	\end{align*}

	Y tengo que resolver el sistema $xy=0$ y $x=0$.
\end{example}

\begin{example}
	Consideramos $|K| = \infty$

	\begin{align*}
		\varphi^*: \real[x]& \rightarrow \quot{\real[x,y]}{\gen{y^2-x}} \\
	\end{align*}

	\begin{align*}
		\varphi: X=\{ \} & \rightarrow Y \\
	\end{align*}

	Fibra sobre el punto $\{-1\}$

	Fibra sobre (0):
	\begin{align*}
		\real[x]& \rightarrow \quot{\real[x,y]}{\gen{y^2-x}} \\
		\gen{x+1} & \rightarrow \gen{x+1}
	\end{align*}

	Y tengo que resolver el sistema $y^2-x=0$ y $x=-1$. Y sale $y^2=-1$.

	Pero el ideal generado por $\gen{x+1} \in \quot{\real[x,y]}{\gen{y^2-x}}$ es $\gen{x+1, y^2-x}$, de hecho esta contenido en $\gen{x+1, y^2+1}\neq \gen{1}$
\end{example}

\begin{example}
	Proyecto la hipérbola sobre la recta:

	\begin{align*}
		\varphi^*: \real[x]& \rightarrow \quot{\real[x,y]}{\gen{xy-y}} \\
	\end{align*}

		\begin{align*}
			\varphi: H & \rightarrow \A^1_{\real} \\
			(a,b) & \rightarrow a
		\end{align*}

		Fibra sobre el punto $\{0\}$

		Fibra sobre (0):
		\begin{align*}
			\real[x]& \rightarrow \quot{\real[x,y]}{\gen{xy-1}} \\
			\gen{x} & \rightarrow \gen{\cls{x}}
		\end{align*}

		Pero $\gen{\cls{x}} = \gen{1}$ (como $\cls{x}$ es invertible genera el total)
\end{example}

\subsection{Lema de normalización de Noether}

\textbf{Previo al lema de normalización de Noether}

Vamos a trabajar con $\K[x_1,...,x_n]$ con $\K$ cuerpo. Y con los siguientes conceptos:
\begin{enumerate}
	\item Monomio: $x_1^{a_1}\cdot...\cdot x_n^{a_n} \in \nat$
	\item Grado de un monomio es la suma de los exponentes
	\item Polinomio homogéneo: es un polinomio que es combinación lineal sobre $\K$ de monomios del mismo grado.
	Por ejemplo $x^2+2xy+3yz$, Un ejemplo de no homogéneo es $y^2-x^3$.
\end{enumerate}

\begin{defn}[Polinomio mónico en la variable $x_i$]
	Diremos que un polinomio $p(x_1,...,x_n)$ no nulo es mónico en la variable $x_i$ si se puede escribir de la siguiente forma:

	$$ p(x_1,...,x_n) = q_0(x_1,...,x_{i-1},x_{i+1},...,x_n)+q_1(x_1,...,x_{i-1},x_{i+1},...,x_n)x_i+...+q_{s-1}(x_1,...,x_{i-1},x_{i+1},...,x_n)x_i^s+ux_i^s $$

	Con $u$ unidad.
\end{defn}

\begin{example}
	\begin{itemize}
		\item $p(x,y,z)=z^2+xyz$ es mónico en z, pero no en x ni en y.
		puedo escribir:
		$$ \K[x,y] \stackrel{finita}{\rightarrow} \quot{\K[x,y,z]}{\gen{z^2+xyz}} $$
		\item $p(x,y,z)=x^2+y^3+z^4+xyz$ es mónico en x,y y z.
		puedo escribir: $\K[x,z]$, $\K[y,z]$
		$$ \K[x,y] \stackrel{finita}{\rightarrow} \quot{\K[x,y,z]}{\gen{x^2+y^3+z^4+xyz}} $$
		\item $p(x,y,z)=xy+xz+yz2$ no es mónico en x, ni en y, ni en z.
		\item $p(x,y,z)=z^2+xyz^3+3$ no es mónico en x, ni en y, ni en z.
	\end{itemize}
\end{example}

\obs: Si un polinomio $p(x_1,...,x_n)$ es mónico en la variable $x_i$ entonces la extensión $\K[x_1,...,x_{i-1},x_{i+1},...,x_n] \rightarrow \quot{\K[x_1,...,x_n]}{\gen{p(x_1,...,x_n)}}$ es finita y por tanto entera.

Pregunta ¿Qué se puede hacer si un polinomio no es mónico en ninguna variable?

\begin{example}
	$F(x,y)=xy \in \K[x,y]$. Hago un cambio de coordenadas $x_1=x-ay$ $y_1=y$. Y me queda:

	$xy=(x_1-ay_1)y_1=x_1y_1+ay_1^2$. (Vemos que $a=F(a,1)$). Despues de este cambio de variable si $a \neq 0$ entonces este polinomio sí es mónico en y.
\end{example} \\
\begin{example}
	$F(x,y,z)=xyz \in \K[x,y,z]$. Hago un cambio de coordenadas $x_1=x-az$ $y_1=y-bz$ y $z_1=z$. Y me queda:

	$xyz=(x_1+az_1)(y_1+bz_1)z_1= cosas + abz_1^3$. Vemos que $ab=F(a,b,1)$

	Si $ab \neq 0$ entonces este polinomio es mónico en $z_1$.
\end{example}

Tengo un monomio cualquiera $F(x_1,...,x_n)=x_1^{\alpha_1}...x_n^{\alpha_n}$.

Y hago el cambio de variable $y_1=x_1-a_1x_n$, $y_2=x_2-a_2x_n$,...,$y_{n-1}=x_{n-1}-a_{n-1}x_n$ y $y_n=x_n$.

Y nos queda $F(x_1,...,x_n)=x_1^{\alpha_1}...x_n^{\alpha_n}= (y_1+a_1x_n)^{\alpha_1}(y_2+a_2x_n)^{\alpha_2}...(y_{n-1}+a_{n-1}x_{n-1})^{\alpha_{n-1}} y_n^{\alpha_n} = cosa y_n+cosa y_n^2+....+cosa y_n^{\alpha_1+...+\alpha_{n-1}+\alpha_n -1} + (a_1^{\alpha_1}a_2^{\alpha_2}...a_{n-1}^{\alpha_{n-1}})y^{\alpha_1+ \alpha_2+...+\alpha_n}=F(a_1,...,a_{n-1},1)\neq 0$.

\begin{example}
 $G(x,y,z)=\stackrel{H}{x^2y}+\stackrel{L}{z^2x}$. Hago el sistema de ecuaciones $x_1=x-az$, $y_1=y-bz$, $z_1=z$. Y me queda $cosas + a^2b x_1^3+cosas+ az_1^3$ Y $H(a,b,1)=a^2b$, y $L(a,b,1)=a$.

 $= cosas z_1+ cosas z_1^2+(a^2b+a)z_1^3$ Y $G(a,b,1)=a^2b+a$. Por ser el cuerpo infinito siempre voy a encontrar valores a,b para que $G(a,b,1) \neq 0$. \textcolor{red}{se queda a medias por Ana}
\end{example}


¿Qué sucede si el polinomio que tengo no es homogéneo? Pues sucede que $p(x_1,...,x_n)=F_0+F_1+...+F_s$ con $F_i$ homogéneo de grado i.

Si hago el cambio de variable $y_1=x_1-a_ix_n$ desde $i=1,...,n-1$ y $y_n=x_n$

Entonces $p(x_1,...,x_n)=F_0(y_1,...,y_n)+F_1(y_1,...,y_n)+...+F_s(y_1,...,y_n)$. Es el último el que me da el mayor exponente. Los otros $s-1$ polinmios me dan exprsiones de grado menor que s. pero en $F_s(y_1,...,y_n)$ me va a aparecer $F_s(a_{1},...,a_{n-1}, 1) = y_n^s$

Si $|K|= \infty$ entones existe $(a_1,...,a_n)$ tal que  $f_s(a_1,...,a_{n-1}, 1)$


