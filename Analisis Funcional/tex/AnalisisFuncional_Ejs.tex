% -*- root: ../AnalisisFuncional.tex -*-
\newcommand{\hard}{\hspace{-3pt}(\dag)\hspace{5pt}}

Los ejercicios marcados con (\dag) están marcados como de ``dificultad especial'' en las hojas.

\section{Hoja 1}


\begin{problem}
\ppart Probar, usando el \nref{thm:CategoriaBaire}, que $I = \set{x ∈ ℝ \tq x ∉ ℚ} ≠ ∅$ y que, de hecho, $I$ es un \nlref{def:ConjuntoGDelta}.

\ppart Probar que $ℝ$ no es numerable.

\ppart Sea $X = ℤ$ con $\dst(x,y) = \abs{x-y}$. Probar que \sdst es un espacio métrico completo y que, sin embargo, es numerable. ¿Por qué no contradice esto al Teorema de Baire?

\solution

\spart

Sabemos que $ℚ$ es numerable, así que podemos enumerar todos los racionales en una serie $q_1, q_2, \dotsc, q_n, \dotsc ∈ ℚ$. Definimos entonces $X_n = ℝ \setminus \set{q_n}$ como una serie de conjuntos abiertos y densos. La intersección de todos ellos son todos los $x ∈ ℝ$ no racionales, que es el conjunto $I$ que buscábamos. Además, por el \nref{thm:CategoriaBaire}, esa intersección es un $G_δ$ denso en $ℝ$, y por eso mismo es no vacío.

\spart

Si $ℝ$ fuese numerable, entonces podríamos enumerarlo: $ℝ \equiv\set{ x_1, x_2, \dotsc, x_n, \dotsc}$. Por otra parte, los conjuntos formados por un único punto son diseminados, por lo que podríamos definir que $ℝ = \bigcup_{n≥1} \set{x_n}$. Sin embargo, esto entraría en contradicción con el Teorema de Baire, que nos dice que no podemos escribir $X$ como una unión numerable de conjuntos diseminados.

\spart

Para que \sdst sea un espacio métrico completo, toda \nlref{def:SucesionCauchy} ha de converger en el espacio. Que una sucesión sea de Cauchy implica que $∀ε> 0$ existe un $N ∈ ℕ$ tal que si $m,n ≥ N$, entonces $\dst(x_m, x_n)$. La cuestión es que, como estamos en $ℤ$, si tomamos\footnote{Cosa que no sé si podemos hacer.} un $ε < 1$, entonces $\dst(x_m, x_n) = 0$ (no podemos tener distancias fraccionarias entre elementos de $ℤ$). Por lo tanto, por ser de Cauchy llega un momento en el que la sucesión se repite constantemente. El límite será entonces es elemento que se repite, que por ser parte de la misma sucesión está en $ℤ$.

% Triplazo.
Esto no contradice el Teorema de Baire porque en $ℤ$ no hay conjuntos densos, y lo vamos a demostrar. Sea $Y \subsetneq ℤ$ un conjunto cualquiera de $ℤ$, y sea $z ∈ ℤ \setminus Y$. La bola de radio $\sfrac{1}{2}$ centrada en $z$ tiene intersección vacía con $Y$ (no hay ningún entero a distancia $\sfrac{1}{2}$ de $z$), por lo que $Y$ no puede ser denso.

Como no hay conjuntos densos, no puede haber tampoco conjuntos diseminados y por lo tanto sigue cumpliéndose el Teorema de Baire: no podemos escribir $ℤ$ como unión numerable de conjuntos diseminados.
\end{problem}

\begin{problem} \hard Sean $\set{a_n}_n≥1$, $\set{b_n}_{n≥1}$ dos sucesiones de números reales y $a_n$ absolutamente convergente. Probar que:

\ppart La función $f$ dada por \[ f(x) = \sum_{n≥1} a_n φ(b_n) \] con $φ(x) ≝ \set{x}$ es continua y $f ∈ C[0,1]$. Además, la serie que define a $f(x)$ es absolutamente convergente.

\ppart Sea $a_n = 2^{-n}$, $b_n = 2^n$, y $h_m = ε_n 2^{-m}$ con $ε_m = \pm 1$ para todo $m$. Probar que \[ \frac{f(x + h_m) - f(x_m)}{h_m} = ε_m \sum_{n=1}^{m-1} 2^{m-n} \left(φ(2^n(x+h_m)) - φ(2^nx)\right)\]

\ppart Si escribimos $x = [x] + \sum_{k>0} α_k 2^{-k}$ con $α_k ∈ \set{0,1}$, entonces \[ φ(2^nx) = φ\left(\sum_{l≥1} α_{n +l} · 2^{-l}\right)\] y además $\sum_{l≥1}α_{n +l} · 2^{-l} ∈ [0,1]$. Del mismo modo, \[ φ(2^n(x+h_m)) = φ(ε_m2^{2-m} + \sum_{l≥1} α_{n +l} · 2^{-l})\] y \[ ε_m2^{n-m} + \sum_{k≥1} α_{n +l} · 2^{-l} = \sum_{l≥1} α_{n +l}' · 2^{-l}\], siendo $α_{n+l}' = α_{n+l} + δ_{m-n, k} ε_m$, con $δ_{i,j}$ la delta de Kronecker\footnote{Esto es, $δ_{i,j} = 1$ si $i = j$, y $0$ si $i ≠ j$.}.

\ppart Tomamos $ε_m = (-1)^{α_m}$. Entonces $α_{n+l}' ∈ {0,1}\;∀l≥1$. Fijemos $m > 1$. Entonces $\sum_{l≥1} α_{n+l} 2^{-l}$ y $\sum_{l≥1}α_{n+l} 2^{-l}$ están ambos en la misma mitad del intervalo $[0,1]$. Usar esto para probar que \[ \frac{f(x+h_m) - f(x_m)}{h_m} = m-1\]

\ppart Del apartado anterior se sigue que $f(x)$ no es diferenciable en ningún $x ∈ ℝ$. Sin embargo, $f$ no está muy lejos de serlo en el sentido siguiente: si $\abs{h} ≤ 1$, entonces $∃ C ∈ (0,∞)$ independiente de $x$ y $h$ tal que \[ \abs{f(x+h) - f(x)} ≤ C\abs{h} \left(1 + \log \frac{1}{\abs{h}}\right) \]

\textbf{Indicación}: Dado $0 < \abs{h} ≤ 1$, existe un único $k ∈ ℕ$ con $2^{-k-1} < \abs{h} <≤ 2^{-k}$. Entonces estimar $f(x+h) - f(x)$ dividiendo la suma en los términos $n < k$ y $n ≥ k$ y estimando cada suma por separado.

\solution


\end{problem}

\begin{problem} Sea \sdst un espacio métrico compacto. Probar que es completo. ¿Es cierto el recíproco?
\solution

Sea $\set{x_n}$ una sucesión de Cauchy en $X$, y sea $\set{ε_n}$ otra sucesión que tiende a $0$. Por ser $\set{x_n}$ de Cauchy, para cada $ε_n$ existe un $M_n ∈ ℕ$ tal que si $m,n ≥ M_n$ entonces $\dst(x_m, x_n) < ε_n$. Equivalentemente, para todo $n ∈ N$ tendremos que $x_n ∈ \bola_{ε_n} (x_{M_n}) = B_n$. Entonces, por ser $X$ compacto existe un subrecubrimiento finito $B_{n_i}$ de $\bigcup B_n$.

Por ser un subrecubrimiento, tendremos que a partir de un cierto $n$ suficientemente grande, las bolas $\bola_{ε_n} (x_{M_n})$ que definíamos antes están estrictamente contenidas en él, por lo que el límite debe de estar ahí también. % Me convence esto más bien poco.

\end{problem}
