\section{Exámenes}

\subsection{Examen Enero 2013}

\begin{problem} Decide de manera razonada si las siguientes afirmaciones son verdaderas o falsas:

\ppart Sea $G = ℤ/15ℤ×ℤ/21ℤ$, abeliano. Entonces $G$ contiene un elemento de orden 9.
\ppart Todo 3-ciclo de $S_4$ está en $A_4$.
\ppart Hay exactamente 6 subgrupos de orden $3$ en $A_4$.
\ppart El polinomio $g(x) = x^7 + 6x^4 + 9x -3$ es irreducible sobre $ℚ$

\solution

\spart Imposible. Si $9\cdot(a,b) = (0,0)$, el orden de $a$ y el de $b$ tienen que dividir a 9, y además al menos uno de ellos tiene que tener orden exactamente 9. Sin embargo, eso es imposible porque ni $ℤ/15ℤ$ ni $ℤ/21ℤ$ pueden tener elementos de orden 9. 

\spart Cierto: un 3-ciclo se puede expresar como la composición de dos trasposiciones , por lo que tiene un índice par.

\spart Falso. $\card{A_4} = 12$. Si hubiese 6 subgrupos de orden 3, la unión de ellos nos daría un grupo de orden 18, lo que es imposible.

\spart Cierto: vemos que 3 divide a todos los coeficientes menos al de mayor grado, pero $3^2$ no divide a 3. El criterio de Eisenstein (\ref{thmEisenstein}) nos dice que entonces es irreducible en $ℚ$.
\end{problem}

\begin{problem} Justifica todas tus respuestas

\ppart Define un homomorfismo $\appl{φ}{ℤ/3ℤ}{S_6}$ de modo que $φ$ sea inyectivo y $φ(1)$ no sea un 3-ciclo.
\ppart Define un homomorfismo inyectivo del grupo aditivo $ℤ/5ℤ$ en el grupo de permutaciones $S_5$.

\solution

\spart Sea $σ=(123456)$ un 6-ciclo, definimos $φ(a) = σ^{2a}$. Es obviamente homomorfismo, inyectivo, y $φ(1)$ no es un 3-ciclo.

\spart Sea $τ=(12345)$, $φ(a) = τ^a$. Inyectivo y toda la pesca.

\end{problem}

\begin{problem} Sea $G=D_6$ el grupo de isometrías que fijan el hexágono regular.

\ppart Encuentra todos los subgrupos de orden 3 de $G$.

\ppart ¿Cuántos subgrupos de orden 4 podría haber en un grupo de orden 12?

\ppart ¿Cuántos subgrupos de orden 4 hay en $G$? Encuéntralos todos.

\ppart Decide de manera razonada si $G$ tiene algún subgrupo normal no trivial.

\solution

\[ D_6 = \{ 1, g, g^2, g^3, g^4, g^5, s, sg, sg^2, sg^3, sg^4, sg^5 \} \]

\spart El subgrupo de orden $3$ es $\{ 1, g^2, g^4 \}$.

\spart Sólo puede haber uno o tres subgrupos de orden 4, según el tercer teorema de Sylow ($n_2|3, n_2\equiv 1 \mod 2$). 

\spart En un subgrupo de orden $4$ sólo puede haber elementos de orden 1, 2 y 4. Es decir

\[ H_4 = \{ 1, g^3, s, sg^3 \}  \]

\spart El subgrupo $\{1, g^3\}$ es normal.

\end{problem}

\begin{problem} 

\ppart Demuestra que el ideal $\gen{2,6X^5+X+1}$ es maximal en $ℤ[X]$.

\ppart Indica cuántos maximales tiene el anillo $ℤ[X]/J$ siendo \[ J = \gen{2, 4X^5 +X^3 + X^2 + X} \]

\solution

\spart Consideramos dos aplicaciones $f,g$ de la siguiente forma

\[ \appl{f}{ℤ[X]}{ℤ[x]/\gen{2}};\quad \appl{g}{ℤ[x]/\gen{2}}{(ℤ/2ℤ[X])/\gen{6x^5+x+1}} \]

teniendo en cuenta que $ℤ[X]/\gen{2} \simeq ℤ/2ℤ[X]$. Por los teoremas de isomorfía \[ ℤ[X]/\ker g\circ f \simeq (ℤ/2ℤ[X])/\gen{6x^5+x+1} \]. En $ℤ/2ℤ$, $6x^5 + x + 1= x+1$, que es irreducible. Como $ℤ/2ℤ$ es un cuerpo, $\gen{x+1}$ es un maximal y entonces $ℤ/2ℤ\big/\gen{x+1}$ es un cuerpo.

Exploremos por otra parte $\ker g\circ f$. $\ker f = \gen{2},\, \ker g = \gen{6x^5+x+1},\;\ker g\circ f = \gen{2,6x^5+x+1}$. Como el conjunto de llegada es un cuerpo, $ℤ[X]/\gen{2,6x^5+x+1}$ también lo es y por lo tanto $\gen{2,6x^5+x+1}$ es maximal.

\spart 

\end{problem}