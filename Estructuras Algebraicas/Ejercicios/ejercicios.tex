\documentclass[nochap]{apuntes}

\usepackage{tikz}

\title{Ejercicios de Estructuras Algebraicas}
\date{2013 / 2014}
\author{Pedro Valero, Victor de Juan, Guillermo Julián, Guillermo Ruiz}

\begin{document}
\maketitle
\newpage
\tableofcontents
\newpage
\section{Hoja 1}
\begin{problem}[1]
 Decide de manera razonada si los siguientes conjuntos son grupos con la operación definida.
 
 \begin{itemize}
  \item a) $(\real,+)$
  \item b) Fijado $n \in \mathbb{Z}_{n>0}$, el conjunto de los enteros módulo $n$ con la suma.
  \item c) $(C^*,\cdot)$
  \item d) $(U (n), ·)$, donde $U (n)$ denota los restos módulo $n$ de enteros coprimos con $n$.
  \item e) Dado un conjunto no vacío $X$, el conjunto $G$ de las biyecciones de $X$ con la composición, $(G, \circ)$.Calcula el cardinal de G si X es un conjunto finito.
 \end{itemize}

 \solution

\spart
$$\left. \begin{matrix}
\text{¿Es asociativo? Sí}\\
\text{¿Tiene elmento neutro? Sí: } \\a + 0 = a, \forall a \in \real\\
\text{¿Existe el inverso de todo elemento? Sí: } \\a^{-1} = -a, \forall a \in \real
\end{matrix}\right\} \implies \text{Sí es un grupo.}$$

\spart
$$\left. \begin{matrix}
\text{¿Es asociativo? Sí}\\
\text{¿Tiene elmento neutro? Sí: } \\\gor{a} + \gor{n} = \gor{a}, \forall \gor{a} \in \mathbb{Z}_n\\
\text{¿Existe el inverso de todo elemento? Sí: } \\ \gor{a} + (\gor{n} - \gor{a}) = \gor{n} = e \implies a^{-1} = \gor{n} - \gor{a}, \forall a \in \real
\end{matrix}\right\} \implies \text{Sí es un grupo.}$$

\end{problem}

\begin{problem}[14] Sea $G$ un grupo finito y $x∈G$. Demuestra que existe $n∈ℕ$ tal que $x^n = e$.
\solution

Consideremos la sucesión $\{x,x^2,\dotsc,x^k+1\}$ con $k=\card{G}$. Como $G$ es finito, por el principio del palomar hay un elemento repetido en esa sucesión. Es decir, existe $α,β$ tales que $x^α=x^β$. Operando, $x = x^{α+β}$ y por lo tanto $x^{α+β-1}$ sería el elemento neutro.
\end{problem}

\section{Hoja 2}
\begin{problem}[4]
Sea G un grupo. Demostrar que \[ Z(G) = \{x \in G \tq xy = yx\; \forall Y \in G\} \] es un grupo. 

\ppart Calcula el centro de $D_3$.

\ppart Calcula el centro de $D_4$.

\solution

Demostramos que es un grupo. Primero vemos que $e \in Z(G)$. Buscamos ahora la existencia del inverso.

Dado $x∈Z(G)$, tenemos que $xy = yx$. Operando 

\begin{align*}
xy &= yx \\
\inv{x}xy\inv{x} &= \inv{x}yx\inv{x} \\
y \inv{x} &= \inv{x} y
\end{align*}

Es decir, que $\inv{x}$ también está en $Z(G)$. Demostramos por último que es una operación cerrada. Siendo $x_1,x_2∈Z(G)$:

\[ x_1 x_2 y = x_1 y x_2 = y x_1 x_2 \]

por lo tanto $x_1x_2 ∈ Z(G)$. Hemos demostrado que $Z(G)$ es un grupo.

\spart
$$Z(D_3), D_3 = \left\{\begin{matrix} 1,a,a^2\\b,ab,a^2b\end{matrix}\right\},\left\{\begin{matrix}a^3 = 1 = b^2\\ ba^j = a^{-j}b\end{matrix}\right\}$$

\spart 
$$Z(D_4) =\left\{\begin{matrix} 1,a,a^2,a^3\\b,ab,a^2b,a^3b\\a^4=1=b^2\\ba=a^{-1}b\end{matrix}\right\}$$

$$Z(D_4) = \left\{x \tq x\in D_4 axa^{-1} = x = bxb^{-1}\right\}$$

Sin sentido...
$a^ib \notin Z(D_4). (a^2 \neq 1)$

$a^i \in Z(D_4) \dimplies (a^{2i} = 1)$

$Z(D_4) = \{1,a^2\} = <a^2>$
\end{problem}

\begin{problem}[5]
Sea $a∈G$. Demuestra que \[ C_G(a) = \{ x∈G\tq xa = ax \} \], el centralizador de $a$ en $G$, es un subgrupo de $G$.

\ppart Demuestra que $Z(G) ≤ C_G(a)$.

\ppart Calcula $C_{D_4}(b)$

\ppart Concluye que la inclusión $Z(G) ≤ C_G(a)$ puede ser estricta.
\solution

Para demostrar que es un subgrupo, vemos que es un subconjunto y que además, dados $x,y∈C_G(a)$:

\begin{align*}
 x\inv{y}a &\qeq ax\inv{y} \\
 x\inv{y}ay &\qeq ax\inv{y}y \\
 x\inv{y}ya &\qeq ax \\
 xa &= ax
\end{align*}

Vemos que $x\inv{y}∈C_G(a)$, por lo tanto es un subgrupo.

\spart Ya sabemos que $Z(G)$ es un grupo: sólo falta demostrar que es un subconjunto. Es trivial dada la definición de centro, que son los elementos que conmutan con todos los elementos de $G$.

\spart

$C_{D_4}(b)$. Basta con comprobar la conmutación con $a^j$ y con $a^jb$ siendo $j = 0,1,2,3$, ya que con eso podemos ver la conmutación con todos los elementos. Se puede demostrar la conmutatividad multiplicando a derecha e izquierda por $b$ y $b^{-1}$ y si nos queda $=1$, es conmutativo.

$$\left\{\begin{matrix}b(a^j)b^{-1} = a^{-j}, a^j \in C_{D_4}(b) \dimplies a^2j = 1\\
b(a^jb)b^{-1} = a^{-j} = a^{-j}b, a^jb \in C_{D_4}(b) \dimplies a^2j = 1\end{matrix}\right.$$
\end{problem}

\section{Hoja 3}


\begin{problem}[1] Subgrupos de $(ℤ,+)$.

\ppart Demuestra que todo subgrupo de $ℤ$ es cíclico, de la forma $kℤ$ para algún entero $k$.

\ppart Demuestra que hay tantos subgrupos en $ℤ$ como enteros no negativos.

\ppart Dados dos enteros positivos $r,s$, demuestra que $rℤ\subset sℤ$ si y sólo si $r$ es un múltiplo de $s$

\ppart Halla todos los subgrupos de $ℤ$ que contienen a $6ℤ$.

\solution

\spart Queremos demostrar que todo subgrupo finito de $\mathbb{Z}$ es cíclico. Como $H$ es un subgrupo, tenemos que $0∈H$ y que $-x∈H\;∀x∈H$. Además, como es operación cerrada, para todo $x∈H$ tenemos que $kx∈H\,∀k∈ℤ$. Buscamos el elemento $b$ más pequeño que no se puede expresar como $b = kx$ con $0 < x < b$ y $k>1$. 

Este elemento $b$ es único. Demostrémoslo: supongamos $H≠Z$ y que otro $b'$ cumple esas condiciones. Podemos plantear la ecuación $kb + kb' = 1$, que tiene solución en todo caso. Por ser operación cerrada, $1∈H$ e igualmente $k∈H\,∀k∈ℤ$, y tendríamos que $H=Z$, contradicción.

Como cualquier otro elemento de $H$ se puede expresar como $kb$ con $k∈ℤ$, tenemos que $H=\gen{b}=bℤ$ y por lo tanto es cíclico.

\spart Trivial.

\spart Queremos demostrar $r\mathbb{Z} \subset s\mathbb{Z} \iff s|r$. Empezamos demostrando la implicación a la derecha, que es trivial si vemos que $r∈rℤ$ se puede expresar como $r=ks∈sℤ$ con $k$ entero.

Por otra parte, si $s|r$ (que podemos expresar como $r=αs$ tenemos que 

\[ x = kr ∈ rℤ \implies x = kαs \implies x ∈ sℤ \]

\spart Conociendo la demostración del apartado anterior, tenemos que los posibles subgrupos son los asociados a sus divisores: $\mathbb{Z}, 2\mathbb{Z}, 3\mathbb{Z}, 6\mathbb{Z}$.
\end{problem}

\begin{problem}[3] Propiedades de los subgrupos normales.

\ppart Demuestra que si $H$ y $K$ son normales en $G$, entonces $H\cap K$ también lo es.

\ppart Si $N \lhd G$ y $N⊆H≤G$, demuestra que 
\begin{enumerate}
\item $N\lhd H$
\item $H/N ≤ G/N$
\item $H/N\lhd G/N \iff H\lhd G$
\end{enumerate}

\ppart Supongamos que $N\lhd G$ y $N⊆H≤G$. Demuestra que

\[ \card{G:H} = \card{G/N : H/N} \]

\solution

\spart Queremos ver que \[ g(H\cap K) = (H\cap K) g\; ∀g∈G \]. Operando:

\[ g(H\cap K) = \{ gα \tq α∈H \y α∈K \} \]

Como $H$ y $K$ son normales, podemos conmutar $gα$ y nos queda que $g(H\cap K)=(H\cap K)g$.

\spart 

\paragraph{$N\lhd H$} Sabemos que $gN=Ng$ para todo $g∈G$. Como $H≤G$, cualquier $h∈H$ cumple esa misma condición $(hN=Nh)$ y por lo tanto $N\lhd H$.

\paragraph{$H/N ≤ G/N$} Como $H$ es subgrupo de $G$, la clase lateral de $N$ con un elemento de $H$ corresponde a la misma clase lateral en $G/N$ ya que todos los elementos de $H$ están en $G$.

\paragraph{$H/N \lhd G/N \iff H\lhd G$} Queremos ver que $HNGN \qeq GNHN$, que es equivalente a $NHGN \qeq NGHN$, lo que sólo ocurre si $HG = GH$, es decir, si $H\lhd G$.\footnote{Con un par.}

\spart Ni idea.

\end{problem}

\begin{problem}[6] Sea $\appl{f}{G}{H}$ un homomorfismo de grupos. Usando la notación multiplicativa demuestra que 

\ppart $f(e) = e'$.

\ppart $f(\inv{a}) = \inv{f(a)}$

\ppart Si $a^k = e$ entonces $f(a)^k = e'$ y concluye que $\card{f(a)}$ divide a $\card{a}$.

\ppart Sean $\appl{f,g}{G}{H}$ dos homomorfismos de grupos. Suponemos que $G=\gen{S}$. Demostrar que si $f(s) = g(s)\,∀s∈S$, entonces $f=g$.

\solution

\spart Consideramos la multiplicación de un elemento cualquiera de $G$ con el neutro: $f(ae) = f(a)f(e)$. Por otra parte, $ae = e$ y entonces $f(ae)=f(a)$. Si $f(a)f(e)=f(a)$, entonces $f(e) = e'$ ya que el neutro es único.

\spart Usando el mismo argumento de antes, \[ e' = f(e) = f(a\inv{a}) = f(a) f(\inv{a}) \implies \inv{f(a)} = f(\inv{a}) \]

\spart La primera parte es trivial. Sobre la segunda, tenemos que $\card{a} = k$. Si $f(a)^k=e'$, o bien $\card{f(a)} = k$ o bien $\card{f(a)}$ divide a $k$. \footnote{Es que es \textit{muy} trivial}.

\spart Ya paso, esto es coña.
 
\end{problem}

\begin{problem}[10] Subgrupos de $C_n$
\ppart Halla todos los subgrupos de $C_6$. 

\ppart Demuestra que todo subgrupo de $C_n$ es cíclico.

\ppart Indica cuántos subgrupos tiene $C_{100}$.

\solution


\end{problem}

\subsection{Hoja 4}

\begin{problem}[4] Determina $\mop{In}(D_4)$.

\solution

Partimos de que $D_4 = \{ e, g, g^2, g^3, s, sg, sg^2, sg^3 \}$, donde $s$ es la simetría con respecto a un eje y $g$ es un giro. Sabemos también que $\mop{In}(D_4)$ es isomorfo a $G/Z(D_4)$, así que vamos a buscar los elementos del centro de $D_4$.

 Es obvio que $C(g)$ debe contener a $\gen{g}$, ya que $g^kg = g g^k$. Dado que $\ord \gen{g} = 4$ y $\ord D_4 = 8$, si encontramos otro elemento más en $C(r)$, tendríamos que $C(g) = D_4$ (el orden del subgrupo tiene que dividir al del grupo. Sin embargo, como $sg≠gs$, $s∉C(g)$ y por lo tanto $C(g)≠D_4$, sino que $C(g) = \gen{g}$.
 
 Consideremos $C(s)$. Sabemos que $g∉C(s)$ así que $C(s)$ no puede tener orden 8. Vemos que los elementos que conmutan son $g^2$, así que de momento $C(s) = \{e, g^2, s\}$. Tiene al menos tres elementos pero no puede tener orden 8, así que simplemente completamos lo que nos falta: $C(s) = \{ e, g^2, s, g^2s \}$.

 Sabemos que $Z(G)$ debe estar contenido en $C(s)$ y en $C(g)$. La intersección de los dos grupos es $\{e,g^2\}$, y ese es el centro\footnote{Lo decía Yahoo! respuestas}.
 
 Esto nos indica que el orden de $\mop{In}(D_4)$ es 4.

\end{problem}

\end{document}