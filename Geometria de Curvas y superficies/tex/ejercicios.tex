\section{Ejercicios}

\subsection{Hoja 1}

\begin{problem}[1] Sean $\appl{α,β}{ℝ}{ℝ^3}$ suaves. Hallar la derivada de cada una de estas funciones

\ppart $\appl{f}{ℝ}{ℝ}$ con $f(t) = \pesc{α(t), β(t)}$.
\ppart $\appl{g}{ℝ}{ℝ^3}$ con $g(t)=  α(t)×β(t)$.
\ppart $\appl{h}{ℝ}{ℝ^3}$ fijada $A$ matriz $3×3$, con $h(t) = Aα(t)$

\solution

\spart Primero desarrollamos $f$ y vemos qué nos sale. Tenemos que

\begin{gather*}
α(t) = \left(α_x(t), α_y(t), α_z(t) \right) \\
β(t) = \left(β_x(t), β_y(t), β_z(t) \right)
\end{gather*}

y por lo tanto tenemos que

\[ f(t) = α_x(t)β_x(t) + α_y(t)β_y(t) +α_z(t)β_z(t) \]

. Derivamos:

\[ f'(t) = α_x'(t)β_x(t) + α_x(t)β_x'(t) + α_y'(t)β_y(t) + α_y(t)β_y'(t) + α_z'(t)β_z(t) + α_z(t)β_z'(t) \]

Sin embargo, queremos reescribir esto en función de las aplicaciones originales α y β. Simplemente nos fijamos que esto es una suma de productos escalares:

\[ f'(t) = \pesc{α'(t),β(t)} + \pesc{α(t),β'(t)} \]

\todo{Apartados b y c}.

\spart

\spart


\end{problem}

\begin{problem}[2] Sea $α(t)$ una curva que no pasa por el origen. Si $α(t_0)$ es el punto de la traza de α más cercano al origen y $α'(t)≠0$, demuestre que el vector de posición $α(t_0)$ es ortogonal a $α'(t_0)$.

\solution

Tenemos que ver que $\pesc{α(t_0),α'(t_0)} = 0$. En $t_0$ tenemos un mínimo global de la distancia al origen, que es $d(t) = \md{α(t)}$. Usando el resultado del primer apartado del ejercicio anterior, $d(t)=\pesc{α(t), α(t)}^{\frac{1}{2}}$. Para ahorrarnos la raíz cuadrada, minimizamos la distancia al cuadrado $c(t)=\pesc{α(t), α(t)}$:

\[ c'(t)= 2\pesc{α'(t), α(t)} \]

Igualamos a cero:

\[ 0 = 2\pesc{α'(t), α(t)} = \pesc{α'(t), α(t)} \]

que es lo que queríamos demostrar. Además, como α no pasa por el origen y su derivada no es nula, sabemos que $d$ es diferenciable en todo punto (la raíz cuadrada no es diferenciable en el 0).
\end{problem}

\begin{problem}[5]
La curva engendrada por un punto P de una circunferencia de radio $r$ que rueda sin deslizar por una recta fija se llama cicloide. Tomando dicha recta como eje de las $X$, y como parámetro $t$ el ángulo orientado $MCP$ ($C$ es el centro de la circunferencia, y $M$ el punto de contacto con el eje), pruebe que la posición de $P$ para cada $t$ es
\[ α(t) = (rt − r \sin t, r − r \cos t) \]

Se ha supuesto que en $t = 0$, $P$ coincide con $M$, y con el origen de coordenadas. Determine los puntos t donde $α′(t) = 0$ (llamados de retroceso). (Nota: "sin deslizar" significa a efectos prácticos que la longitud del arco $MP$ coincide con la longitud del segmento $OM$).

\solution

\easyimg{img/Hoja1_5_Cicloide.png}{Cicloide.}{figEj5_Cicloide}

Viendo el dibujo \ref{figEj5_Cicloide} podemos ver claramente la definición de la parametrización, que es la que nos dan. Calculamos ahora $α'$:

\[ α'(t) = (r-r\cos t, r\sin t) \]

Igualando a 0, tenemos que $\cos t = 1$ y que $\sin t = 0$. Por lo tanto, los puntos en los que $α'(t)=0$ son de la forma $t=2kπ$, es decir, $(2r,0)$.


\end{problem}

\begin{problem}[8] Parametrizamos la espiral logarítima como

\begin{align*}
\appl{α&}{ℝ}{ℝ^2} \\
α(t) &= (e^{bt} \cos t, e^{bt} \sin t)
\end{align*}

con $b < 0$. Demostrar que cuando $t\to ∞$, $α(t)$ se acerca al origen. Calcule, para cada intervalo $[t_0, t_1] ⊆ ℝ$, la longitud de arco de $α$. Halle las ecuaciones de la parametrización por longitud de arco y demuestre que ningún arco suyo es circular.

\solution

\easyimgw{img/Hoja1_8_EspLog.png}{Espiral logarítimica.}{figEspLog}{0.7}

Calculamos $\lim_{t\to ∞} α(t)$. Dado que $\abs{\cos t} ≤ 1\;∀t∈ℝ$,

\[ \abs{e^{bt}\cos t } ≤ \abs{e^{bt}} = e^{bt} \]

y tomando límites, $\lim_{t\to ∞} e^{bt} = 0$ ya que $b$ es negativa. La demostración es análoga para la otra coordenada.

Pasamos ahora a calcular la longitud de cada intervalo. Para $[t_0, t_1] ⊆ ℝ$, tenemos que

\[ L_{t_0}^{t_1}(α)= \int_{t_0}^{t_1}\md{α'(t)} \dif t \]

Para hallar la parametrización por longitud de arco, tomamos un $t_0∈I$ en el dominio de α y calculamos

\begin{equation}\label{eq1_8} s(t) = \int_{t_0}^t \md{α'(τ)} \dif τ \end{equation}

Esta aplicación $s$ va de $I$ a $ℝ$, así que hallamos el intervalo $J=s(I)$. Despejaremos para $t=t(s)$ de (\ref{eq1_8}) y escribo $\appl{β}{J}{ℝ^2}$ con $β(s)=α(t(s))$. Operamos

\begin{align*}
α'(t) &= (be^{bt}\cos t - e^{bt} \sin t, be^{bt}\sin t + e^{bt}\cos t) \\
\md{α'(t)}^2 &= b^2e^{2bt}\cos^2 t + e^{2bt}\sin^2 t - 2be^{bt}\cos t e^{bt}\sin t + \\
& \quad + b^2e^{2bt}\sin^2 t + e^{2bt}\cos^2 t + 2be^{bt}\cos t e^{bt}\sin t = \\
&= b^2e^{2bt} + e^{2bt} = e^{2bt}(b^2+1) \\
\md{α'(t)} &= e^{bt}\sqrt{b^2 +1}
\end{align*}

Una vez que tenemos la expresión del módulo de la derivada, integramos. Elegimos $t_0 = 0$ para facilitar la integración:

\begin{align*}
s(t) &= \int_0^t e^{bτ}\sqrt{b^2 +1} \dif τ = \\
&= \frac{\sqrt{b^2+1}}{b} \left(e^{bt}\right|_0^t = \frac{\sqrt{b^2+1}}{b}(e^{bt}-1)
\end{align*}

Hallamos la imagen:

\[ s(ℝ) = \left(-\frac{\sqrt{b^2+1}}{b}, ∞\right) = J \]

Ahora despejamos $t(s)$:

\begin{align*}
s &= \frac{\sqrt{b^2+1}}{b}(e^{bt}-1) \\
\frac{b}{\sqrt{b^2+1}}s + 1&= e^{bt} \\
\log \left(\frac{b}{\sqrt{b^2+1}}s + 1\right)&= bt \\
t &= \frac{1}{b} \log \left(\frac{b}{\sqrt{b^2+1}}s + 1\right)
\end{align*}

y sustituimos en $α$ para hallar la parametrización por longitud de arco $β$:

\begin{multline*} β(s) = α(t(s))  = \left(\left(\frac{b}{\sqrt{b^2+1}}s + 1\right)\cos \left(\frac{1}{b} \log \left(\frac{b}{\sqrt{b^2+1}}s + 1\right)\right),\right. \\ \left. \left(\frac{b}{\sqrt{b^2+1}}s + 1\right)\cos \left(\frac{1}{b} \log \left(\frac{b}{\sqrt{b^2+1}}s + 1\right)\right)\right) \end{multline*}

\easyimgw{img/Hoja1_8_EspLogArco.png}{Buscamos un arco circular en la espiral logarítmica.}{figEspLogArco}{0.7}

Vamos ahora a comprobar si existe algún arco circular. Buscaremos si existe algún intervalo $I⊆ℝ$ tal que $α(A)$ sea el arco de una circunferencia (no necesariamente centrada en el origen), como se ve en la figura \ref{figEspLogArco}.

Si existiese tal circunferencia, habría algún centro $(a_0,b_0)$ y un radio $r>0$ fijo de tal forma que $∀t∈I\;\md{α(t) - (a_0,b_0)} = r$. La condición se puede expresar mejor como $\md{α(t) - (a_0,b_0)}^2 = r^2$. Operamos:

\[ \md{α(t) - (a_0,b_0)}^2 = (e^{bt}\cos t -a)^2 + (e^{bt}\sin t -b_0)^2 = r^2 \]

Si buscamos que la parte de la izquierda sea constante, derivamos y buscamos cuándo vale 0:

\begin{gather*}
2(e^{bt}\cos t -a_0)(be^{bt}-e^{bt}\sin t) + 2(e^{bt}\sin t - b_0)(be^{bt}\sin t + e^{bt} \cos t) =
\end{gather*}

Y se halla. Normalmente lo haríamos con una función de curvatura, viendo si es constante positiva o negativa.
\end{problem}

\begin{problem}[9] Cisoide.

Copias tú el enunciado.

\solution

Por la forma del problema vamos a escribir todo en polares.

Empezamos con $M_2$. La recta $x=2a$ se reescribe en polares como $r_{M_2}(θ) = \frac{2a}{\cos θ} = 2a \sec θ$.

Por otra parte, para reescribir $M_1$ reescribimos la circunferencia en polares y nos queda que $r_{M_1}(θ) = 2a\cos θ$. 

Para $P$, la distancia es la resta de las anteriores, entonces

\[ r_P(θ) = 2a \sec θ - \cos θ = 2a\left(\frac{1}{\cos θ} - \cos θ\right) \]

con $θ∈(-π/2, π/2)$.  Para hallar la ecuación implícita, 

\end{problem}

\begin{problem}[16] Sea $b$ una constante no nula. Halle una parametrización por longitud de arco de la hélice

\[ α(t) = (a\cos t, a \sin t, bt) \]

\solution

Vemos que las dos primeras coordenadas van a ir describiendo una circunferencia de radio $a$ en el plano $XY$, donde $t$ será el ángulo formado con el eje de las $X$. La última coordenada nos da la altura, que es siempre creciente.

Para reparametrizar, buscamos cambiar cómo nos movemos por el parámetro sin variar el conjunto imagen en $ℝ^3$, según el siguiente esquema:

\begin{figure}[hbtp]
\centering
\begin{tikzpicture}
\node (I) at (0,0) {$t∈I⊆ℝ$};
\node (J) at (0,-3) {$s∈J⊆ℝ$};
\node (R) at (3,0) {$ℝ^3$};

\draw[->] (J) -- node[left] {$f$} (I);
\draw[->] (I) -- node[above] {$α$} (R);
\draw[->] (J) -- node[below right] {$β=α○f$} (R);
\end{tikzpicture}
\caption{Reparametrización.}
\label{figEj1}
\end{figure}

donde buscamos que $f$ sea un difeomorfismo (esto es, que exista su inversa y que sea diferenciable).

La reparametrización por arco de $α$ es la aplicación $β(s)$ construida como en (\ref{figEj1}) tal que $\md{β'(s)}=1$.

Sabemos que

\[ β'(s) = \deriv{β}{s}(s) = \deriv{α}{t}(f(s))\cdot \deriv{f}{s}(s) \]

y por lo tanto
\[ \abs{f'(s)} = \frac{1}{\md{α'(f(s))}} \]

Entonces necesitamos que $α'(t)≠0\; ∀t∈I$ (que sea regular). Además, si $f(s)$ es creciente

\[ f'(s) = t'(s) = \frac{1}{\md{α'(t(s)}} \implies s'(t) = \md{α'(t)} \]

y nos quedaría

\[ s(t) = \int_{t_0}^t \md{α'(t)}\dif t \]

Aplicamos todo esto al ejercicio en el que estamos.

\begin{gather*}
 α'(t) = (-a\sin t, a \cos t, b) \\
 \md{α'(t)} = \sqrt{a^2+b^2}
\end{gather*}

Integramos:

\[ s(t) = \int_0^t\md{α'(t)} \dif t = t\sqrt{a^2+b^2} \]

de tal forma que la inversa es \[ t(s) = \frac{s}{\sqrt{a^2+b^2}} \]

Nuestra reparametrización es

\[ β(s) =α(t(s)) = \left(a\cos \frac{s}{\sqrt{a^2+b^2}},a\sin \frac{s}{\sqrt{a^2+b^2}}, b \frac{s}{\sqrt{a^2+b^2}}\right) \]

\end{problem}

\begin{problem}[19] Si $\appl{α}{I}{ℝ^3}$ es una curva y $\appl{M}{ℝ^3}{ℝ^3}$ es un movimiento rígido de $ℝ^3$, demuestre que las longitudes de $α$ y $M○α$ entre $a$ y $b$ coinciden.

\solution

Un movimiento rígido conserva el movimiento escalar. Pasa algo. Y sale que sí. Hallas productos escalares, longitudes de arco y toda la pesca.

\end{problem}

\subsection{Hoja 2}

\begin{problem}[2] \textbf{Tacnodo vertical.} Sea $C$ la curva definida implícitamente por $y^4=x^2(x+1)$ en todo el plano $xy$. Demuestre que la parte de $C$ cercana al punto $(0,0)$ es la unión 

\[ \{ x= h_1(y)\} \cup \{ x=h_2(y) \} \]

de los grafos de dos funciones suaves $h_1,h_2$, definidas en $-ε<y≤ε$ y tales que $y=0$ es mínimo local de $h_1$ y máximo local de $h_2$.

Indicación: estudie si $φ(x) \equiv x\sqrt{x+1} $ tiene inversa cerca de $x=0$. Dibuje $C$ y demuestre que $C\cap\left( ℝ^2\setminus\{(0,0)\}\right)$ es suave y sin autointersecciones.

\solution

\easyimg{img/Hoja2_2_Tacnodo.png}{Tacnodo vertical}{imgTacnodo}

Consideramos $\appl{F}{ℝ^2}{ℝ}$ dada por 

\[ F(x,y) = y^4 - x^2(x+1) \]

Está claro que $C\cap ℝ^2\setminus (0,0) = \inv{F}(0)$. Calculemos el gradiente de $F$:

\[ \grad F (x,y) = \left(-3x^2-2x, 4y^3\right) =\left(x(-3x-2), 4y^3 \right) \]

Queremos ver si el gradiente se anula en algún punto de nuestra curva. Vemos que $\grad F = 0$ en dos casos:

\[ (x,y) = (0,0);\quad (x,y) = \left(-\frac{2}{3}, 0\right) \]

Sin embargo, $(0,0)$ no está en la curva, y dado que $F \left(-\frac{2}{3}, 0\right) = -\frac{4}{27} ≠ 0$  $\left(-\frac{2}{3}, 0\right)$ tampoco está. Por lo tanto, la función es \textbf{suave y sin autointersecciones}.

Vamos ahora a demostrar la segunda parte del problema. Buscamos dos funciones $h_1$ y $h_2$. Nos piden que las funciones dependan de $y$ y no de $x$, así que tenemos que comprobar que $x\sqrt{x+1}$ tiene una inversa cerca de cero para saber que podemos despejar $x$ en $φ(x) = y$ Calculamos la derivada de $φ$:

\[ φ'(x) = \sqrt{x+1} + x\frac{1}{2\sqrt{x+1}} \]

que no se anula en $φ'(0)$. Por el Teorema de la Función Inversa, existen $δ>0, ε>0$ tales que $\appl{\inv{φ}}{(-ε,ε)}{(-δ,δ)}$ existe y es diferenciable.  

Entonces, si $x≥0$, $(x,y)∈C \dimplies y^2 = φ(x)$, y de la misma forma si $x≤0$, $(x,y) ∈ C \dimplies y^2 = -φ(x)$. Por lo tanto, podemos construir las dos funciones $h_i$ de la siguiente forma:

\[ h_1(y) = \inv{φ}(y^2);\quad h_2(y) = \inv{φ}(-y^2) \]

\end{problem}

\begin{problem}[4] $\appl{α}{(-a,a)}{ℝ^2}$ PPA. Defino $β(s)=α(-s)$, de tal forma que cambiamos la orientación. Verifica que β PPA y halle la curvatura.

\solution

Obvio que β también está PPA.
\end{problem}

\begin{problem}[5] Halle una curva plana parametrizada por longitud de arco tal que 

\[ k_α(s) = \frac{1}{1+s^2} \]

\solution

Hallamos el ángulo:

\[ θ(s) = \int \frac{1}{1+s^2}\dif s = \arctan s \]

Despejamos la arcotangente:

\begin{wrapfigure}{r}{0.4\textwidth}
\begin{tikzpicture}
\coordinate (A) at (0,0);
\coordinate (B) at (4,0);
\coordinate (C) at (4,2);

\draw[-] (A) -- node[above,sloped] {$\sqrt{1+s^2}$} (C);
\draw[-] (A) -- node[below,sloped] {$1$} (B);
\draw[-] (C) -- node[right] {$s$} (B);

\begin{scope}
\path[clip] (A) -- node[below,sloped] {$1$} (B) -- node[right] {$s$} (C) -- node[above,sloped] {$\sqrt{1+s^2}$} (A);
\fill[green, opacity=0.3, draw=black] (A) circle (5mm);
\node at ($(A)+(10:12mm)$) {$\theta(s)$};
\end{scope}
\end{tikzpicture}
\end{wrapfigure}

\begin{gather*}
\cos θ(s) = \frac{1}{\sqrt{1+s^2}} \\
\sin θ(s) = \frac{s}{\sqrt{1+s^2}}
\end{gather*}

Integrando, tenemos que 

\begin{align*}
x(s) &= \log \left(s + \sqrt{s^2+1}\right) \\
y(s) &= \sqrt{1+s^2}
\end{align*}

Añadiendo el movimiento rígido tendríamos todas las curvas que cumplen esa ecuación:

\[ β(s) = \begin{pmatrix}
\cos θ_0 & - \sin θ_0 \\
\sin θ_0 & \cos θ_0
\end{pmatrix}\begin{pmatrix}
 \log \left(s + \sqrt{s^2+1}\right) \\
 \sqrt{1+s^2}
\end{pmatrix} + \begin{pmatrix}
x_0 \\ y_0 
\end{pmatrix} \]
\end{problem}

\begin{problem}[8] Sea $\appl{α}{I}{ℝ^2}$ PPA. Demuestra

\ppart α es segmento de recta si y sólo si $∃p_0∈ℝ^2$ por el cual pasan todas las rectas tangentes.

\ppart α es un arco de circunferencia si y sólo si $∃p_0∈ℝ^2$ que esté en todas las rectas normales.

\solution

\spart Empezamos demostrando la implicación a la derecha. Si $α$ es un segmento de recta, lo podemos escribir como 

\[ α(s) = p_0 + s\vv \]

con $\vv$ unitario. La recta tangente a $α$ en $s=s_0$ será 

 \[ α(s_0) + λα'(s_0) = p_0 + s_0\vv + λ\vv = p_0 + (s_0 + λ)\vv \]
 
y por lo tanto $p_0$ está en todas las rectas tangentes a $α$.

Ahora la implicación a la izquierda. Para $s∈I$, sabemos que la recta tangente a $α$ por $s$ es

\begin{equation}\label{eqH2E8} α(s) + λα'(s)\quad λ∈ℝ \end{equation}

Como $p_0$ está en esa recta, $∃λ(s)\tq p_0 = α(s) + λ(s)α'(s)$. Ahora la idea que se nos viene a la cabeza es derivar, pero no sabemos si λ es derivable. Despejamos y 

\[ p_0 - α(s) = λ(s)α'(s) \]

Ahora no podemos dividir por $α'$ (es un vector) así que mltiplicamos a ambos lados por $α'$:

\[ \pesc{p_α - α(s), α'(s)} =λ(s) \pesc{α'(s), α'(s)} = λ(s) \]

por lo tanto $λ(s)$ es suave en $s$. Podemos derivar, así que derivamos en (\ref{eqH2E8})

\begin{gather*}
 0 = α'(s) + λ'(s) α'(s) + λ(s) k_α(s) \mv{n}_α(s) \\
 0 = \mv{t}_α(s) + λ'(s) \mv{t}_α(s) + λ(s)k_α(s)\mv{n}_α(s) \\
 0 = (1+λ'(s))\mv{t}_α(s) + λ(s)k_α(s)\mv{n}_α(s)
 \end{gather*}
 
Dado que $\{\mv{t}_α,\mv{n}_α\}$ es una base ortonormal del plano, una combinación lineal de ambos dos sólo puede ser cero si los dos coeficientes son cero. Esto nos lleva a las dos siguientes ecuaciones

\[ \left.\begin{matrix}
1+λ'(s) = 0 \\
λ(s)k_α(s) = 0 \\
\end{matrix}\right\} ∀s∈I \]

Como $λ'(s) = -1$, $λ(s) = s_0 - s$ para algún $s_0$. Sustituimos en la segunda ecuación y nos queda que $(s_0-s)k_α(s) = 0$, así que si $s≠s_0$ $k_α(s)=0$. Y como además, $k_α$ es continua, en ese punto $s_0$ también es cero. La curvatura es cero en todo punto $s$, y por lo tanto es un segmento de recta.

\spart Visualmente, vemos que $p_0$ deberá ser el centro. La implicación a la derecha se demuestra fácilmente. Sabemos la parametrización de $α$

\[ α(s) = p_0 + r\left(\cos \frac{s}{r}, \sin \frac{s}{r}\right) \]

Calculamos la recta normal por $α(s)$ y veo que $p_0$ está en ella.

Calculamos ahora la \textbf{implicación a la izquierda}. Queremos demostrar que la curvatura es constante y distinto de 0. Sabemos que 

\[ p_0 = α(s) + λ(s)\mv{n}_α(s) \]

para algún $λ(s)$. Como podemos expresar $λ(s) = \pesc{p_0-α(s),\mv{n}_α(s)}$, es derivable. Así que derivamos.

\begin{gather*}
0 = \mv{t}_α(s) + λ'(s)\mv{n}_α(s) + λ(s)\left(-k_α(s) \mv{t}_α(s)\right) \\
0 = (1-k(s)λ(s))\mv{t}_α(s) + λ'(s)\mv{n}_α(s) 
\end{gather*}

Al igual que veíamos en el anterior apartado, sabiendo que tenemos una base normal sacamos dos ecuaciones:

\begin{gather*}
1-k(s)λ(s) = 0 \\
λ'(s)=0 
\end{gather*}

λ es constante, así que $λ(s) = c_0$ para algún $c_0∈ℝ$. Entonces

\[ 1-k_α(s)c_0 = 0 \implies k_α(s) = \frac{1}{c_0} \]

La curvatura escalar es constante, así que efectivamente trabajamos en un arco de circunferencia.

\end{problem}

\begin{problem}[9] Sea $\appl{α}{I}{ℝ^2}$ PPA. Demuestra que todas las rectas normales equidistan de un $p_0$ dado si y sólo si existen $a,b∈ℝ$ tal que \[ k_α(s) = \pm \frac{1}{\sqrt{as + b}} \]

\solution

Hagamos un dibujito

\begin{figure}[hbtp]
\centering
\begin{tikzpicture}[pnt/.style={draw,shape=circle,fill=white, inner sep=2pt}]
\draw[-] (0,0) -- (6,0);
\node[pnt,label=below:{$α(s)$}] (S) at (2,0) {};
\node[label=below:{$\cn(s)$}] (N) at (3.5,0) {};
\node[pnt] (PA) at (4,0) {};
\draw[thick,->] (S) -- (N);
\draw[->] (S) -- node[left] {$-\mv{t}_α(s)$} (2,2);
\node[pnt,label=right:{$p_0$}] (P) at (4,3) {};
\draw[->] (S) -- node[midway, above, sloped] {$p_0-α(s)$} (P);
\draw[dashed] (P) -- node[right] {$\pesc{p_0-α(s), -\mv{t}_α(s)}$} (PA);
\end{tikzpicture}
\end{figure}

Derivamos la distancia $\pesc{p_0-α(s), -\mv{t}_α(s)}$

\[ 0 = \pesc{-\mv{t}_α(s),-\mv{t}_α(s)} + \pesc{p_0-α(s),-k_α(s) \mv{n}_α(s)} \]

y por lo tanto $k_α(s)\pesc{p_60-α(s),\mv{n}_α(s)} = 1$. Volvemos a derivar (total, es gratis)

\[ \cv(s)\pesc{p_0-α(s)\cn(s)} + \cv(s)\left(\pesc{\ct(s), \cn(s)} + \pesc{p_0-α(s), -\cv(s)\ct(s)}\right) = 0 \]

y dividimos:

\[ \frac{\cv'(s)}{\cv(s)} - c\cv^2(s) = 0 \]

Resolvemos la ecuación diferencial y nos queda que

\[ \frac{-1}{2k^2}=cs + d \]

Por lo tanto

\[ k^2 = \frac{1}{(-2c)s + (-2d)} \]

de tal forma que hemos llegado a la fórmula que nos daban al principio.

Ahora vamos a por la \textbf{otra implicación}. Pero no, que es muy larga.
\end{problem}

\begin{problem}[11] Sea $Γ$ una curva cerrada, simple, en el plano, contenida en el interior de la circunferencia $\{ x^2+y^2=r^2 \}$. Demuestra que existe un punto $p∈Γ$ tal que $\md{k(p)} ≥ \frac{1}{r}$.

\solution

Partimos de que $\frac{1}{r}$ es la curvatura del círculo. Trasladamos la circunferencia en cualquier dirección hasta tocar la curva por primera vez. En ese punto de contacto, la tangente es la misma y por lo tanto la normal también.

Si trasladásemos y rotamos las curvas para que el plano sea el eje $OX$ y el punto de contacto fuese el origen, tendríamos algo parecido a la imagen \ref{imgCirc}.

\easyimg{img/Hoja2_11_CircCurva.png}{Curva y circunferencia}{imgCirc}

Si podemos escribir la curva $α=α(s)$ como grafo de una función $h_1(x)$, entonces

\[ k(α(0)) = \frac{h_1''(x)}{\left(1+h_1'(x)^2\right)^{\frac{3}{2}}} \]

Dado que tenemos un grafo sobre la recta tangente, $h_1'(0) = 0$ y entonces $k_Γ(0)=h_1''(0)$. Por otra parte, sabemos que $k_C(0) = h_2''(0) = \frac{1}{4}$. Además, $h_1(x) ≥ h_2(x)$ en un intervalo $(-δ,δ)$ y las funciones y sus derivadas valen lo mismo en $0$. Por lo tanto, tiene que cumplirse

\[ h_1''(0) ≥ h_2''(0) \implies k_Γ(0) ≥ \frac{1}{r} \]

\end{problem}

\subsection{Hoja 3}

\begin{problem}[2] Tenemos una curva $α=α(s)$ PPA biregular. Demuestra que 

\[ \ctr (s) = \frac{α'(s) × α''(s) \cdot α'''(s)}{\abs{\cv(s)}^2} \]

\solution

Sabemos que 

\begin{align*}
α'(s) &= \ct (s) \\
α''(s) &= \ct'(s) = \cv (s) \cn (s) \\
α'''(s) &= \cv'(s) \cn(s) + \cv(s) \cn'(s) = \\
&= \cv'(s) \cn(s) + \cv(s) \left(-\cv(s)\ct(s) + \ctr(s) \cb(s) \right)
\end{align*}

Mientras arreglaba una cosa de \LaTeX me ha escrito una pizarra y me he perdido. Así que nada.

\end{problem}

\begin{problem}[3] A saber.

\solution

\spart Defino $β(s) = \ct(s)$. Comprobamos si es regular

\[ β'(s) = \ct'(s) = \cv(s) \cn (s) ≠ 0 \]

ya que $k_α(s)≠0$ para todo $s$ ya que α es birregular.

\spart Queremos demostrar que

\[ \cvv[β] = \frac{\ctr}{\cv}\cb(s) - \ct(s) \]

No sabemos si β está parametrizada por arco. Viendo la ecuación anterior, $\md{β'} = \md{\cv}$, distinto de uno si la curvatura no es constante igual a 1. 

En esta sitación tenmos varias posibilidades. O bien primero reparametrizamos β por arco $r$ tal que $β(s(r)) = γ(r)$ y hallo $γ''(r)$ o bien hallo $\cvv[β](s)$usando la fórmula para curvas no PPA. 

\end{problem}