\section{Ejercicios}

\subsection{Hoja 1}

\begin{problem}[1] Sean $\appl{α,β}{ℝ}{ℝ^3}$ suaves. Hallar la derivada de cada una de estas funciones

\ppart $\appl{f}{ℝ}{ℝ}$ con $f(t) = \pesc{α(t), β(t)}$. 
\ppart $\appl{g}{ℝ}{ℝ^3}$ con $g(t)=  α(t)×β(t)$.
\ppart $\appl{h}{ℝ}{ℝ^3}$ fijada $A$ matriz $3×3$, con $h(t) = Aα(t)$

\solution

\spart Primero desarrollamos $f$ y vemos qué nos sale. Tenemos que

\begin{gather*}
α(t) = \left(α_x(t), α_y(t), α_z(t) \right) \\
β(t) = \left(β_x(t), β_y(t), β_z(t) \right) 
\end{gather*} 

y por lo tanto tenemos que 

\[ f(t) = α_x(t)β_x(t) + α_y(t)β_y(t) +α_z(t)β_z(t) \]

. Derivamos:

\[ f'(t) = α_x'(t)β_x(t) + α_x(t)β_x'(t) + α_y'(t)β_y(t) + α_y(t)β_y'(t) + α_z'(t)β_z(t) + α_z(t)β_z'(t) \]

Sin embargo, queremos reescribir esto en función de las aplicaciones originales α y β. Simplemente nos fijamos que esto es una suma de productos escalares:

\[ f'(t) = \pesc{α'(t),β(t)} + \pesc{α(t),β'(t)} \]

\todo{Apartados b y c}.

\spart

\spart


\end{problem}

\begin{problem}[2] Sea $α(t)$ una curva que no pasa por el origen. Si $α(t_0)$ es el punto de la traza de α más cercano al origen y $α'(t)≠0$, demuestre que el vector de posición $α(t_0)$ es ortogonal a $α'(t_0)$.

\solution

Tenemos que ver que $\pesc{α(t_0),α'(t_0)} = 0$. En $t_0$ tenemos un mínimo global de la distancia al origen, que es $d(t) = \md{α(t)}$. Usando el resultado del primer apartado del ejercicio anterior, $d(t)=\pesc{α(t), α(t)}^{\frac{1}{2}}$. Para ahorrarnos la raíz cuadrada, minimizamos la distancia al cuadrado $c(t)=\pesc{α(t), α(t)}$:

\[ c'(t)= 2\pesc{α'(t), α(t)} \]

Igualamos a cero:

\[ 0 = 2\pesc{α'(t), α(t)} = \pesc{α'(t), α(t)} \]

que es lo que queríamos demostrar. Además, como α no pasa por el origen y su derivada no es nula, sabemos que $d$ es diferenciable en todo punto (la raíz cuadrada no es diferenciable en el 0).
\end{problem}

\begin{problem}[5]
La curva engendrada por un punto P de una circunferencia de radio $r$ que rueda sin deslizar por una recta fija se llama cicloide. Tomando dicha recta como eje de las $X$, y como parámetro $t$ el ángulo orientado $MCP$ ($C$ es el centro de la circunferencia, y $M$ el punto de contacto con el eje), pruebe que la posición de $P$ para cada $t$ es
\[ α(t) = (rt − r \sin t, r − r \cos t) \]

Se ha supuesto que en $t = 0$, $P$ coincide con $M$, y con el origen de coordenadas. Determine los puntos t donde $α′(t) = 0$ (llamados de retroceso). (Nota: "sin deslizar" significa a efectos prácticos que la longitud del arco $MP$ coincide con la longitud del segmento $OM$).

\solution

\easyimg{img/Hoja1_5_Cicloide.png}{Cicloide.}{figEj5_Cicloide}

Viendo el dibujo \ref{figEj5_Cicloide} podemos ver claramente la definición de la parametrización, que es la que nos dan. Calculamos ahora $α'$:

\[ α'(t) = (r-r\cos t, r\sin t) \]

Igualando a 0, tenemos que $\cos t = 1$ y que $\sin t = 0$. Por lo tanto, los puntos en los que $α'(t)=0$ son de la forma $t=2kπ$, es decir, $(2r,0)$.


\end{problem}

\begin{problem}[16] Sea $b$ una constante no nula. Halle una parametrización por longitud de arco de la hélice

\[ α(t) = (a\cos t, a \sin t, bt) \]

\solution

Vemos que las dos primeras coordenadas van a ir describiendo una circunferencia de radio $a$ en el plano $XY$, donde $t$ será el ángulo formado con el eje de las $X$. La última coordenada nos da la altura, que es siempre creciente.

Para reparametrizar, buscamos cambiar cómo nos movemos por el parámetro sin variar el conjunto imagen en $ℝ^3$, según el siguiente esquema:

\begin{figure}[hbtp]
\centering
\begin{tikzpicture}
\node (I) at (0,0) {$t∈I⊆ℝ$};
\node (J) at (0,-3) {$s∈J⊆ℝ$};
\node (R) at (3,0) {$ℝ^3$};

\draw[->] (J) -- node[left] {$f$} (I);
\draw[->] (I) -- node[above] {$α$} (R);
\draw[->] (J) -- node[below right] {$β=α○f$} (R);
\end{tikzpicture}
\caption{Reparametrización.}
\label{figEj1}
\end{figure}

donde buscamos que $f$ sea un difeomorfismo (esto es, que exista su inversa y que sea diferenciable). 

La reparametrización por arco de $α$ es la aplicación $β(s)$ construida como en (\ref{figEj1}) tal que $\md{β'(s)}=1$.

Sabemos que 

\[ β'(s) = \deriv{β}{s}(s) = \deriv{α}{t}(f(s))\cdot \deriv{f}{s}(s) \]

y por lo tanto
\[ \abs{f'(s)} = \frac{1}{\md{α'(f(s))}} \]

Entonces necesitamos que $α'(t)≠0\; ∀t∈I$ (que sea regular). Además, si $f(s)$ es creciente 

\[ f'(s) = t'(s) = \frac{1}{\md{α'(t(s)}} \implies s'(t) = \md{α'(t)} \]

y nos quedaría

\[ s(t) = \int_{t_0}^t \md{α'(t)}\dif t \]

Aplicamos todo esto al ejercicio en el que estamos.

\begin{gather*}
 α'(t) = (-a\sin t, a \cos t, b) \\
 \md{α'(t)} = \sqrt{a^2+b^2}
\end{gather*}

Integramos:

\[ s(t) = \int_0^t\md{α'(t)} \dif t = t\sqrt{a^2+b^2} \]

de tal forma que la inversa es \[ t(s) = \frac{s}{\sqrt{a^2+b^2}} \]

Nuestra reparametrización es

\[ β(s) =α(t(s)) = \left(a\cos \frac{s}{\sqrt{a^2+b^2}},a\sin \frac{s}{\sqrt{a^2+b^2}}, b \frac{s}{\sqrt{a^2+b^2}}\right) \]

\end{problem}