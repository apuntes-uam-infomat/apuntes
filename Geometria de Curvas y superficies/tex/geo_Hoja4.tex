\subsection{Hoja 4}

\begin{problem}[1] Sea $f(x) = (x+y+z-1)^2$. Halle los valores y puntos críticos de $f$. Decida para qué valores $c$, $\inv{f}(c)$ es una superficie regular.

\solution

Calculamos las derivadas parciales:

\[ f_x = f_y = f_z = 2(x + y + z -1) \]

Igualando a cero, tenemos que los puntos críticos están en el plano

\[ x+ y+z = 1 \]

y el único valor crítico es el 0.

Ahora nos piden buscar para qué valores $c∈ℝ$ $\inv{f}(c)$ es una superficie regular. Consideramos varias opciones.

\begin{itemize}
\item Si $c<0$, $\inv{f}(c)= \emptyset$ y no es superficie regular.
\item Si $c>0$, $\inv{f}(c) ≠\emptyset$. Por ejemplo $P=(\sqrt{c} + 1, 0,0∈\inv{f}(c)$ y como $c$ no es valor crítico, $f^{-1}(c)$ es superficie regular.
\item Si $c=0$, antes habíamos calculado que ese era un valor crítico para los puntos críticos del plano $x+ y+z = 1$, así que efectivamente es una superficie regular.
\end{itemize}

\end{problem}

\begin{problem}[6] Sean $r,R>0$ y $r<R$. Llamamos toro $T$ al conjunto de puntos $(x,y,z)$ obtenidos al girar alrededor del eje $OZ$ la circunferencia $C$ de centro $(0,R,0)$, radio $r$ y situada en el plano $x=0$.

Demuestra que $T$ se puede expresar como un conjunto de nivel $T=\inv{F}(0)$ de una función $\appl{F}{ℝ^3}{ℝ}$, con 

\[ F(x,y,z) = (x^2+y^2+z^2-R^2-r^2)^2-4R^2(r^2-z^2) \]

\solution

Partimos del siguiente dibujo.

\easyimg{img/Hoja4_6_Toro.png}{Planteamiento del problema.}{imgToro}

Proponemos que todo punto de $T$ lo obtengo a partir de un $p_0$ en $C$ tras rotarlo un cierto ángulo θ con respecto al eje $OX$. Entonces, la parametrización es

\(\label{eqH4_6_Param} \begin{matrix}
x &=& (R + r\cos ψ)\cos θ \\
y &=& (R + r\cos ψ)\sin θ \\
z &=& r\sin ψ
\end{matrix} \)

¿Cómo demostramos que ambos conjuntos $T$ y $\inv{F}(0)$ son iguales? Pues lo de siempre: doble contención. Primero vemos si $T⊆\inv{F}(0)$.

Calculamos $F(T)$. Para $(x,y,z)∈T$, 

\[ F(x,y,z) = \dotsb =\footnote{Magia!} 0 \]

Vamos ahora al revés: queremos mirar si $\inv{F}(0)⊆T$. Si $(x,y,z)∈\inv{F}(0)$, sabemos que

\[ (x^2+y^2+z^2-R^2-r^2)^2 = 4R^2(r^2-z^2) \]

Hay que buscar $θ,ψ$ para que se cumpla \eqref{eqH4_6_Param}.

\end{problem}

