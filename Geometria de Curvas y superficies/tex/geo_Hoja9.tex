\subsection{Hoja 9}

\begin{problem}[1] Dadas las siguientes superficies $S$ y $Q$ sus métricas Riemannianas, halla la curvatura gaussiana $K$.

\ppart $Q = a^2 \dif u ^2 + b^2 \dif v ^2$.

\solution

Si $S$ estuviese dada como parametrización y $Q$ fuese la primera forma fundamental, ya sabríamos cómo obtener la curvatura. Pero no lo sabemos.

Escribimos la métrica como la suma de dos formas\footnote{Formas de adsafdsf} \[ Q = ω_1^2 + ω_2^2 \], y hallamos un $ω_3$ tal que\[ \dif ω_1 = ω_2 \y ω_3\quad \dif ω_2 = ω_3 \y ω_1 \]. De esta forma, obtendremos 

\[ \dif ω_3 = K (ω_1 \y ω_2 \]

\spart Sea $ω_1 = a\dif u$, $ω_2 = b \dif v$. El paso dos nos exige hallar $\dif ω_1$ y $\dif ω_2$ primero.

\[ \dif ω_1 = \dif (a\dif u) = \dif a \y \dif u = (a_u \dif u + a_v \dif v) \y \dif u = - a_v \dif u \y \dif v \]

Análogamente, nos queda que $\dif ω_2 = b_u \dif u \y \dif v $. Nos falta hallar ahora $ω_3$. Si no la vemos al principio, tenemos que darnos cuenta que $ω_3$ será de la forma $A \dif u + B \dif v$, y sustituimos en las ecuaciones.

\begin{align*}
 \dif ω_1 &= ω_2 \y ω_3 \\
 - a_v \dif u \y \dif v &= b \dif v \y (A\dif u + B \dif v) \\
 - a_v \dif u \y \dif v &= -Ab \dif u \y \dif v \\
 A &= \frac{a_v}{b}
\end{align*}

De la misma forma obtenemos $B = - \frac{b_u}{a}$, y entonces

\[ ω_3 = \frac{a_v}{b} \dif u - \frac{b_u}{a} \dif v \]

Hallamos la diferencial:

\begin{align*}
 \dif ω_3 &= \dif\left(\frac{a_v}{b} \dif u\right) - \dif = \left(\frac{b_u}{a} \dif v \right) = \\
 	&= \left(\frac{a_v}{b}\right)_u \dif u \y \dif u + \left(\frac{a_v}{b}\right)_v \dif v \y \dif u - \left(\frac{b_u}{a}\right)_u \dif u \y \dif v + \left(\frac{b_u}{b}\right)_v \dif v \y \dif v = \\
 	&= \left(-\left(\frac{b_u}{a}\right)_u - \left(\frac{a_v}{b}\right)_v \right)\dif u \y \dif v
\end{align*}

Comparamos ahora ese resultado con $Kω_1 \y ω_2$:

\[ \left(-\left(\frac{b_u}{a}\right)_u - \left(\frac{a_v}{b}\right)_v \right)\dif u \y \dif v = K \left(a\dif u \y b \dif v\right) = Kab \dif u \y \dif v 
\]
y por lo tanto
\[ K = \frac{-1}{ab} \left(\left(\frac{b_u}{a}\right)_u + \left(\frac{a_v}{b}\right)_v\right) \]
\end{problem}

\begin{problem}[2] Dado \[ \frac{\dif x^2 + \dif y^2}{(x^2+y^2+C)^2} \] di en qué lugar del plano $XY$ esto define una métrica, y calcula su curvatura gaussiana.

\solution

Podemos reexpresar la fórmula así:

\[ \frac{1}{(x^2+y^2+C)^2}  \dif x^2 +  \frac{1}{(x^2+y^2+C)^2} \dif y^2\]

Que se parece bastante a la primera forma fundamental $E \dif x^2 + 2F \dif x \dif y + G \dif y^2$. Podemos entonces escribir la matriz de esa forma fundamental: cuando esté definida positiva, será una métrica. Estudiamos

\[Q =  \begin{pmatrix}
 \dfrac{1}{(x^2+y^2+C)^2}  & 0 \\
0 & \dfrac{1}{(x^2+y^2+C)^2} 
\end{pmatrix} \]

Vemos que hay un problema cuando $x^2+y^2+C = 0$. Tenemos varios casos:

\begin{itemize}
\item Si $C > 0$, $x^2+y^2 + C > 0$ y $Q$ es definida positiva.
\item Si $C ≤ 0$, hay puntos $(x,y)$ tales que $x^2 + y^2 + C = 0$ (los de la circunferencia de radio $\sqrt{C}$). Salvo en esa circunferencia, $Q$ es definida positiva.
\end{itemize}

La curvatura gaussiana sale $4C$, por si os apetece hacerlo.

\end{problem}

\begin{problem}[3] Sea $c∈ℝ$ y $S_c$ la superficie parametrizada por

\[ Φ(u,v) = \left(\frac{u^3}{3} - uv^2, \frac{v^3}{3} - u^2v, cv\right) \]

\ppart Halla la primera forma fundamental $I_Φ$.
\ppart Ver que $X = (2cuv, cu^2-cv^2, (u^2+v^2)^2)$ es normal a $S_c$.
\ppart Halla $\md{X}$.
\ppart Calcula la segunda forma fundamental ($e,f,g$).
\ppart Calcula la curvatura gaussiana $K$ de dos formas diferentes.
\ppart Di cuándo es la métrica localmente isométrica al plano.

\solution

\spart Hallamos las derivadas:

\begin{align*}
Φ_u &= (u^2-v^2, -2uv, 0) \\
Φ_v &= (-2uv, v^2-u^2, c) \\
E   &= (u^2-v^2)^2 + 4u^2v^2 = (u^2 + v^2)^2 \\
F	&= 0 \\
G 	&= (u^2 + v^2)^2 + C^2
\end{align*}

Luego \[ I_Φ = (u^2 + v^2)^2 \dif u^2 + \left((u^2+v^2)^2 + C)\right) \dif v^2 \].

Salvo en el $(0,0)$, la 1FF es una métrica riemanniana y podemos calcular la curvatura gaussiana con la formulita del ejercicio anterior o con la otra \footnote{Ver cuál es la otra.}

\spart 

Podemos verlo fácilmente comprobando que $X$ es ortogonal a una base del plano tangente, es decir, viendo que $\pesc{X,Φ_u} = \pesc{X,Φ_v} = 0$.

\spart

Calculamos 

\begin{gather*}
 \md{X}^2 = \cdots = (u^2+v^2)^2\left((u^2+v^2)^2 + C^2\right) \\
 \md{X} = (u^2+v^2) \sqrt{(u^2 + v^2)^2 + C^2} = (u^2+v^2) \sqrt{G} = \sqrt{EG}
 \end{gather*}

\spart 

\begin{align*}
Φ_{uu} &= (2u,-2v,0) \\
Φ_{uv} &= (-2v, -2u, 0) \\
Φ_{vv} &= (-2u, 2v, 0) \\
e &= \frac{2cv}{\sqrt{G}} \\
f &= \frac{-2cu}{\sqrt{G}} \\
g &= \frac{-2cv}{\sqrt{G}}
\end{align*}

\spart La primera forma de calcular $K$ es usar la 1FF $I_Φ$ y lo que habíamos visto en el problema 1. Tenemos

\begin{gather*}
a = (u^2+v^2) \\
b = \sqrt{(u^2+v^2)^2 + C^2} \\
K  = \frac{-1}{ab} \left(\left(\frac{b_u}{a}\right)_u + \left(\frac{a_v}{b}\right)_v\right) = \frac{-4C^2}{G^2} 
\end{gather*}

La otra forma de calcularla es la formulita

\[ K = \frac{eg-f^2}{EG-F^2} = \frac{-4C^2}{G^2} \]

Siempre coinciden ambas formas cuando la métrica riemanniana es la 1FF.

\spart Hay que acordarse de varios detalles. Lo primero es que hay que recordar que $K$ se preserva mediante isometrías locales, ya que estas preservan la 1FF y por lo tanto preservan $K$, que sólo depende de la 1FF. 

Entonces, si $S_c$ es localmente isométrico mediante $\appl{ψ}{A⊆S_c}{B⊆ℝ^2}$ al plano, tenemos que tener que $K_{S_c} (p) = K_{ℝ^2} (ψ(p)) = 0$. Entonces, está claro que cuando $c$ sea $0$ $S_C$ será localmente isométrica al plano.

Ahora bien, esto es sólo una condición necesaria (el teorema egregio de Gauss nos da condición necesaria, no suficiente). Si $c=0$, ¿es $S_c$ localmente isométrica al plano? 

Usamos el teorema de Minding \eqref{thmMinding}, que contesta a la pregunta directamente: si tienen curvaturas constantes, iguales, las dos superficies son localmente isométricas.
\end{problem}

\begin{problem}[4] Sea $Q$ métrica Riemanniana en $S$, $c> 0 ∈ ℝ$. ¿Cuál es la relación entre la $K$ de $Q$ y la de $cQ$?

\solution

A grandes rasgos, multiplicar por la constante $c$ significa \textit{agrandar} la superficie (por ejemplo, de una esfera a una esfera con más radio). En este primer análisis vemos que debería disminuir la curvatura: vamos a probarlo.

Trabajamos de forma paralela en ambas superficies:

\[
\begin{matrix}
K_Q 				& K_{cQ} \\
Q = ω_1^2 + ω_2^2	& cQ = c(ω_1^2+ω_2^2) = (\sqrt{c}ω_1)^2 + (\sqrt{c}ω_2)^2 \\
\dif ω_1 = ω_2 \y ω 3 & \dif \bar{ω}_1 = \sqrt{c} \dif ω_1 = \sqrt{c} ω_2 \y ω_3  \implies \dif \bar{ω}_1 = \bar{ω_2} \y ω_3 \\ 
\dif ω_2 = ω_3 \y ω 2 & \dif \bar{ω}_2 = \sqrt{c} \dif ω_2 = \sqrt{c} ω_3 \y ω_1  \implies \dif \bar{ω}_2 = ω_3 \y \bar{ω}_1 \\
					& ω_3 = \bar{ω}_3 \\
\end{matrix} \]

Seguimos operando:

\[ \dif \bar{ω}_3 = K_{cQ} \bar{ω}_1 \y \bar{ω}_2 = c K_{cQ} ω_1 \y ω_2 \]

Por otra parte, como \[ \dif \bar{ω}_3 = \dif ω_3 = K_Q ω_1 \y ω_2 \] entonces

\[ K_Q = cK_{cQ} \]

\end{problem}

\begin{problem}[5] Dada la siguiente forma fundamental \[ I = \dif u^2 + 2u\dif u \dif v + \dif v^2 \] con $\abs{u} < 1$, demuestra que $S$ es localmente isométrica al plano.

\solution

Buscamos aplicar el teorema de Minding \eqref{thmMinding}: curvatura gaussiana y a ver si es $0$. Buscamos aplicar lo del problema 1 de forma pedestre.

\[ ω_1 = A \dif u + B \dif v\quad ω_2 = C\dif u + D \dif v \]

Calculamos los cuadrados y a ver qué sale.

\begin{align*}
ω_1^2 &= A^2\dif u^2 + 2AB \dif u \dif v  + B^2\dif v^2 \\
ω_2^2 &= C^2\dif u^2 + 2CD \dif u \dif v + D^2 \dif v^2 \\
ω_1^2 +ω_2^2 &= (A^2+C^2) \dif u^2 + 2(AB+CD)\dif u \dif v + (B^2+D^2)\dif v^2 
\end{align*}

lo que nos da un sistema

\[\begin{cases} 1 = A^2+C^2 \\ u = AB +CD \\ 1 = B^2 +D^2 \end{cases} \], que podemos resolver diciendo que $D=0$, y nos queda $A=u$, $B=1$, $C=\sqrt{1-u^2}$. Finalmente

\begin{align*}
ω_1 &= u\dif u + \dif v \\
ω_2 &= \sqrt{1-u^2} \dif u
\end{align*}

Buscamos ahora $\dif ω_1$ y $\dif ω_2$:

\[ \begin{cases}
\dif ω_1 &= 0 \\
\dif ω_2 &= 0 
\end{cases} \implies ω_3 = 0 \implies \dif ω_3 = 0 \implies K = 0 
\]
y ya está.

\end{problem}

\begin{problem}[6]\footnote{DIce que te puedes pasar una vida calculándolo. Miedito.} Sea $S_1$ una superficie dada por la parametrización

\[ Φ(u,v) = (\cos v -u\sin v, \sin v + u \cos v, v) \] y $S_2$ dada por 

\[ ψ(z,θ) = \left(\sqrt{1+z^2}\cos θ, \sqrt{1+z^2}\sin θ, z\right) \] que como se puede ver, es de revolución.

Halla una isometría local $\appl{h}{S_1}{S_2}$ expresada como \[ Φ(u,v) \longmapsto ψ(z(u,v), θ(u,v)) \]

Utiliza además que si $\appl{h}{S_1}{S_2}$ es isometría local, entonces $∀c∈ℝ$ la imagen por $h$ de la curva de nivel $\{K = c\}$ en $S_1$ van a puntos en la misma curva de nivel $\{ K = c \}$ en $S_2$.

\solution

Calculo la $K$ de $S_1$ y $S_2$ y veo qué ocurre con ellas. Como tenemos las parametrizaciones, podemos hallar la primera y segunda forma fundamental y obtener la curvatura. Al hacerlo, tenemos que

\begin{gather*}
K_{S_1}(u,v) = -\frac{1}{1+u^2)^2} \\
K_{S_2}(z,θ) = -\frac{1}{(1+2z^2)^2}
\end{gather*}

Entonces en un punto $(z(u,v),θ(u,v))$ la curvatura es

\[ K_{S_2} = - \frac{1}{(1+2z(u,v)^2)^2} \]

Si $h$ es isometría local, tiene que darse 

\begin{align*}
 K_{S_1}(u,v) &= K_{S_2}(z(u,v),θ(u,v)) \\
 - \frac{1}{(1+u^2)^2} &= -\frac{1}{(1+2z(u,v)^2)^2} \\
 z(u,v) &= \frac{u}{\sqrt{2}}
\end{align*}

Hemos sacado entonces que la pinta de $h$ tiene que ser algo como 
\[ Φ(u,v) \longmapsto ψ\left(\frac{2}{\sqrt{2}}, θ(u,v)\right)\]

Nos falta encontrar el ángulo, y ahora vamos a usar que $h$ es isometría local. En este caso, $\md{Φ_u}^2 = \md{\dif h (Φ_u)}^2$ ya que la norma preserva los vectores. Además, $\pesc{Φ_u, Φ_v} = \pesc{\dif h(Φ_u), \dif h (Φ_v)}$ y análogamente con ψ. Estas ecuaciones nos darán información sobre el ángulo θ.

\[ \md{Φ_u}^2 = E(u,v) = 1 \] y 

\[ \dif h(Φ_u) = ψ_z\left(\frac{u}{\sqrt{2}}, θ\right) · \frac{1}{\sqrt{2}} + ψ_θ\left(\frac{u}{\sqrt{2}}, θ\right) θ_u 
\], luego podemos calcular la norma al cuadrado:

\[ \md{\dif h(Φ_u)}^2 = \md{ ψ_z\left(\frac{u}{\sqrt{2}}, θ\right)}^2·\frac{1}{2} + 2 \frac{1}{\sqrt{2}}\pesc{ ψ_z\left(\frac{u}{\sqrt{2}}, θ\right),  ψ_θ\left(\frac{u}{\sqrt{2}}, θ\right)} θ_u = \bar{E}\left(\frac{u}{\sqrt{2}}, θ\right) \frac{1}{2} + \sqrt{2} θ_u \bar{F}\left(\frac{u}{\sqrt{2}}, θ\right) + \bar{G}\left(\frac{u}{\sqrt{2}}, θ\right) \]

Obtenemos $\bar{E}, \bar{F}, \bar{G}$ y sustituyendo cosas ahí, tenemos que \[ θ_u = \pm \frac{\sqrt{2}}{2+u^2} \], y cogemos el valor positivo por no complicarnos la vida.

Necesitamos también la derivada con respecto a $v$, información que vamos a sacar de la tercera igualdad o de la primera. Usamos que \[ \md{Φ_v}^2 = \md{\dif h(Φ_v)}^2 \]. Entonces

\[ \dif h(Φ_v) = \pd{}{v} h(Φ(u,v)) = \pd{}{v} ψ\left(\frac{u}{\sqrt{2}}, θ\right) = ψ_z\left(\frac{u}{\sqrt{2}}, θ\right) \cdot 0 + ψ_θ\left(\frac{u}{\sqrt{2}}, θ\right)θ_v \]

Aplicamos ahora la condición:

\begin{align*}
\md{Φ_v}^2 &= \md{\dif h(Φ_v)}^2  \\
2+u^2 &= θ_v^2 \md{ψ_θ\left(\frac{u}{\sqrt{2}}, θ\right)} \\
G(u,v) &= θ_v^2 \md{\bar{G}\left(\frac{u}{\sqrt{2}}, θ\right)}^2 \\
θ_v &= \sqrt{2}
\end{align*}

Ahora sólo falta hallar $z$ integrando. Integramos primero $θ_u$

\[ θ = \int \frac{\sqrt{2}}{2+u^2}\dif u = \arctan \frac{u}{\sqrt{2}} + C(v) \], derivamos con respecto a $v$:

\[ θ_v = c'(v) \implies c(v) = \sqrt{2}v + C_0 \] luego
\[ θ(u,v) = \arctan\frac{u}{\sqrt{2}} + \sqrt{2} v + C_0 \]
\end{problem}