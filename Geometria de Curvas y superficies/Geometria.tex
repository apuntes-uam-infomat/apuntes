\documentclass[nochap]{apuntes}

\usepackage{changepage}
\usetikzlibrary{calc}
\title{Geometría de curvas y superficies}
\author{Por definir}
\date{13 / 14 C2}

\usepackage{tikz}

\begin{document}

\maketitle
\newpage

\section{Introducción}

\subsection{Parametrización de curvas}
Existen varias formas de representar curvas en $\real^2$.

\begin{defn}[Curva\IS parametrizada diferenciabble]
Es una aplicación diferenciable $\appl{sigma}{I=(a,b)}{\real^n}$
\end{defn}
Vamos a ver varios ejemplos de curvas parametrizadas:

\begin{example}
Definimos la curva $x^3+y^3 = xy$ y la parametrizamos utilizando la indicación $t = \frac{x}{y}$.

Tras unos cuantos cálculos llegamos a \[\sigma(t)=\left(\frac{t}{1+t^3},\frac{t^2}{1+t^3}\right) \]

Nos damos cuenta que en $t=-1$ no está definida. Vamos a estudiar a ver que sale. Nos fijamos en el intervalo $(-1,1]$.Si $t\to-1^{-1}$ Estudiamos la asíntota


Además es interesante estudiar la simetría. Suponemos que puede ser simétrica respecto de la recta $y=x$. Para ver si una curva parametrizada es simétrica respecto de la recta estudiamos qué pasa con $\sigma\left(\frac{1}{t}\right)$. 

Sustituímos y vemos que $(x(t),y(t)) = (y(1/t),x(1/t))$ con lo que concluimos que es simétrica.

\end{example}
Folium de Descartes (análisis completo)

\[ x^3+y^3 = 3axy \]

con $a$ número real, que corresponde a la asíntota que tiene la función. Podemos obtener una parametrización del folium a través de una aplicación $\appl{F}{I⊆ℝ}{ℝ^2}$. El conjunto de partida $I$ se divide en dos subintervalos que tienen su correspondiente imagen: $I = I_1 \cup I_2 = (-∞, -1) \cup (-1,∞)$. La parametrización sería

\[ φ(t) = \left(\frac{t}{1+t^3} ,\frac{t^2}{1+t^3}\right) \]

El folium tiene una asíntota, de ecuación \[0=x+y+\frac{1}{3} = \frac{1}{9} \frac{(t+1)^2}{t^2-t+1}\]. También es simétrica por la recta $y=x$. Por otra parte, cuando $t\to ∞$, $σ(t)\to (0,0)$.

\todo{Dibujitos de lo que vale para cada $t$ y los intervalos.}

Es una parametrización regular de la curva, ya que $σ'$ es distinta de cero siempre.

Otro ejemplo:

\begin{example}
Vamos a comparar las dos curvas siguientes: 

\begin{gather*}
C_1=\{ (x,y)\tq e^x-y^3=0 \}\\
C_2=\{(x,y)\tq e^x-y^2=0\}
\end{gather*}

Podemos parametrizar $C_1$ con una aplicación $\appl{σ}{ℝ}{ℝ^2}$ tal que $σ(x)=(x,e^{x/3})$.

\todo{Dibujito de $C_1$}

En el caso de $C_2$, tenemos dos ramas debido a la raíz cuadrada. Aquí tendríamos que $y=\pm e^{x/2}$. Es decir, que a partir de un conjunto podemos sacar una curva conexa o no.
\end{example}

\begin{example}[Parábola semicúbica]
Tomamos la parábola semicúbica con la aplicación 
\begin{align*}
\appl{σ}{ℝ&}{ℝ^2} \\
σ(t) &= (t^3, t^2)
\end{align*}

Esta curva tiene un pincho (\textit{sic}) en el origen, ya que $σ'$ se anula cuando $t=0$.

\end{example}

\begin{defn}[Velocidad\IS de una trayectoria] Dada una curva parametrizada de $ℝ^n$  $\appl{σ}{I}{ℝ^n}$, el vector velocidad de la trayectoria en $σ(t)$ es $σ'(t)$, su derivada.
\end{defn}

\begin{defn}[Rapidez\IS de una trayectoria] Dada una curva parametrizada de $ℝ^n$  $\appl{σ}{I}{ℝ^n}$, la rapidez de la trayectoria es $\md{σ'}$, el módulo su derivada.
\end{defn}

\begin{example}[Cúbica nodal]La cúbica nodal es una curva plana que viene dada por el siguiente conjunto:

\[ C = \{ (x,y)\tq y^2=x^2(x+1) \} \]

\todo{Otro dibujito}

Para hallar la parametrización, consideramos $t=\frac{x}{y}$ como parámetro, y nos queda que

\[ α(t) = (t^2-1,t(t^2-1)) \]

, que es regular ya que $α'$ no se anula. Sin embargo, esta parametrización no es inyectiva (por ejemplo, $α(t) = (0,0) \implies t= \pm 1$). Dividimos la curva en dos partes que se solapan: 

\[ α_1(t) = \eval[2]{α(t)}_{t> -\frac{1}{2}},\qquad α_2(t) = \eval[2]{α(t)}_{t<\frac{1}{2}} \]

\todo{Dibujito de cada una de las ramas. De Juan, esto te pasa por escaquearte y querer hacer los dibujitos xD}

¿Qué ganamos con esto? No lo sé, salvo varias preguntas estúpidas (Parra y Pablo), mucho rato tratando de definir qué es solapamiento y nada útil.
\end{example}

Podemos tener varios \textit{accidentes} en una parametrización:

\begin{enumerate}
\item Que la parametrización no sea inyectiva.
\item Que sea inyectiva pero su inversa no sea continua.
\end{enumerate}

Por lo tanto, buscaremos un tipo de parametrizaciones \textit{buenas}:

\begin{defn}[Parametrización\IS bicontinua] Se dice que una parametrización es bicontinua si ella y su inversa son continuas.
\end{defn}

Por la propia definición, una parametrización regular es localmente bicontinua. 

\begin{defn}[Longitud\IS de arco] Sea $\appl{σ}{(a,b)}{ℝ^N}$ una trayectoria de clase $C^1$. Entonces la longitud de arco de σ está definida por 

\[ L(σ) = \int_a^b \md{σ'(t)} \dif t \]
\end{defn}

La longitud de arco nos permite definir una parametrización para cualquier curva:

\begin{defn}[Parametrización\IS por longitud de arco] Se dice que una curva $\appl{σ}{I}{ℝ^n}$ está parametrizada por longitud de arco si $\md{σ'(s)} = 1\;\forall s∈I$.
\end{defn}

Igualmente, también querremos crear una reparametrización que nos lleve una parametrización regular cualquiera a una parametrización por longitud de arco.

\begin{prop} Para toda curva parametrizada regular $\appl{σ}{I}{ℝ^N}$ existe una transformación de parámetros que conserva la orientación , $h$, tal que $σ○h$ está parametrizada por longitud de arco.
\end{prop}

\begin{proof} Para reparametrizar, buscamos cambiar cómo nos movemos por el parámetro sin variar el conjunto imagen en $ℝ^3$, según el esquema de \ref{figParamLongArco}.

\begin{wrapfigure}{r}{0.4\textwidth}
\centering
\begin{tikzpicture}
\node (I) at (0,0) {$t∈I⊆ℝ$};
\node (J) at (0,-3) {$s∈J⊆ℝ$};
\node (R) at (3,0) {$ℝ^N$};

\draw[->] (J) -- node[left] {$f$} (I);
\draw[->] (I) -- node[above] {$σ$} (R);
\draw[->] (J) -- node[below right] {$β=σ○f$} (R);
\end{tikzpicture}
\caption{Reparametrización por longitud de arco.}
\label{figParamLongArco}
\end{wrapfigure}

Buscamos que $f$ sea un difeomorfismo (esto es, que exista su inversa y que sea diferenciable). 

La reparametrización por arco de $σ$ es la aplicación $β(s)$ construida como en (\ref{figParamLongArco}) tal que $\md{β'(s)}=1$.

Sabemos que 

\[ β'(s) = \deriv{β}{s}(s) = \deriv{σ}{t}(f(s))\cdot \deriv{f}{s}(s) \]

y por lo tanto
\[ \abs{f'(s)} = \frac{1}{\md{α'(f(s))}} \]

Entonces necesitamos que $σ'(t)≠0\; ∀t∈I$ (que sea regular). Además, si $f(s)$ es creciente 

\[ f'(s) = t'(s) = \frac{1}{\md{σ'(t(s)}} \implies s'(t) = \md{σ'(t)} \]

y nos quedaría

\[ s(t) = \int_{t_0}^t \md{σ'(t)}\dif t \]

que es la reparametrización que buscamos.

\end{proof}

\begin{lemma} Sean $\appl{σ_1}{I_1}{ℝ^N}$, $\appl{σ_2}{I_2}{ℝ^N}$ dos parametrizaciones por arco de la misma curva. Entonces la transformación de parámetros correspondiente $\appl{h}{I_1}{I_2}$ tal que $σ_1=σ_2○h$ es de la forma $h(s) = \pm s + s_0$ con $s_0∈ℝ$ constante. $s$ será positivo si la orientación es compatible y negativa si la orientación de ambas parametrizaciones es opuesta.
\end{lemma}

\begin{proof} Sabemos que 

\[ 1 = \md{σ_1'(s)} = \md{(σ_2○h)'(s)} = \md{σ_2'(h(s))\cdot h'(s)} = \md{σ_2'(h(s))} ·  \abs{h'(s)} = \abs{h'(s)} \]

Integrando $\abs{h'(s)}$ nos queda que $h(s) = \pm s + s_0$.
\end{proof}

\section{Curvas en el plano}

\begin{defn}[Curvatura] Dada una curva regular en $ℝ^N$, el campo \mv{k} de los vectores curvatura de la misma es el formado por la derivada segunda $β''(s)$, siendo β una parametrización por longitud de arco.
\end{defn}

De forma trivial, se verifica que la derivada segunda \textbf{no depende de la parametrización} por arco elegida, ya que dos parametrizaciones distintas sólo diferían en una constante y en la orientación: $β(s) = α(c\pm s)$ con $c$ constante.

Si además consideramos el campo \mv{t} de las tangentes unitarias de la curva\footnote{$\displaystyle\mv t(s) = \frac{α'(s)}{\md{α'(s)}}$}, tenemos que, independientemente de la parametrización:

\[ \mv k(s) \equiv α''(s) \equiv \mv t'(s) \]

Esto nos lleva a una primera propiedad interesante de la curvatura:

\begin{lemma} Dada una curva regular en $ℝ^N$, su vector curvatura es normal a la curva en cada punto
\end{lemma}

Por otra parte, la curvatura cumple una idea muy sencilla: si su valor es 0, estamos ante una recta. Dicho de otra forma:

\begin{lemma} Una curva regular en $ℝ^N$ es un trozo de recta afín si y sólo si $\mv k = 0$. \end{lemma}

\par 

Hasta ahora hemos tratado la curvatura sobre parametrizaciones por longitud de arco. ¿Cómo podemos obtenerla si estamos tratando con cualquier otro parámetro α?

Sabemos que podemos escribir una reparametrización de α, así que podemos efectivamente "normalizar" la curvatura y el campo de tangentes:

\begin{gather*}
\mv t(s) = \frac{α'(s)}{\md{α'(s)}} \\
\mv k(s) = \frac{α''(s)}{\md{α'(s)}^2} - \left(\mv t \frac{α''(s)}{\md{α'(s)}^2}\right) \mv t
\end{gather*}

En el caso de \mv{k}, lo que hacemos es quitarle a la normalización la componente tangencial, la que tiene la misma dirección que \mv{t}.\footnote{A grandes rasgos, esto equivale a aplicar la ortogonalización de Gram-Schmidt al par de vectores $\displaystyle\left\{\mv t, \frac{α''(s)}{\md{α'(s)}^2}\right\}$.}

\subsection{Resumen de curvas planas}

Partimos de una curva $\appl{α}{I}{ℝ^2}$, parametrizable por arco con $α=α(s)$. De ella podemos obtener el \textbf{diedro de Frenet}, una base ortonomal del plano dada por

\[ \mv{t}_{α} = α'(s);\quad \mv{n}_α(s) = R_{\frac{π}{2}} \mv{t} (s) \]

donde $R_{β}$ es una rotación por un ángulo β.

Podemos considerar el producto escalar $\pesc{\mv{n}_α(s),\mv{n}_α(s)}$ que es igual a 1  (el vector normal es unitario). Podemos derivar, y nos queda 

\[ 0 = 2\pesc{\mv{n}'_α(s), \mv{n}_α(s)} \]

Por lo tanto, la tangente al normal es perpendicular al normal. La cuestión es encontrar ahora la relación entre sus módulos. Para ello derivamos el producto escalar del vector normal y del tangente:

\[0 = \pesc{\mv{n}_α,\mv{t}_α}' = \pesc{\mv{n}_α', \mv{t}_α} + \pesc{\mv{n}_α, \mv{t}_α'} = \stackrel{\text{Ec. Frenet}}{=} \pesc{\mv{n}_α', \mv{t}_α} + \pesc{\mv{n}_α,k_α\mv{n}_α} \]

y entonces

\[ \pesc{\mv{n}_α', \mv{t}_α} = -k_α(s) \]

Con la base ortonormal del diedro de Frenet podemos escribir un vector $\vu$ como 

\[ \vu = \pesc{\vu, \mv{t}_α(s)}\mv{t}_α(s) + \pesc{\vu, \mv{n}_α(s)}\mv{n}_α(s)\]

y por lo tanto podemos reescribir 

\[ \mv{n}' = \underbrace{\pesc{\mv{n}', \mv{t}_α(s)}}_{-k_α(s)} \mv{t}_α(s) + \underbrace{\pesc{\mv{n}', \mv{n}_α(s)}}_{0}\mv{n}_α(s)\]

\paragraph{Reconstrucción a partir de una función de curvatura} Dada una función suave $\appl{k}{I}{ℝ}$, las curvas que tienen $k$ como función de curvatura se construyen como

\[ θ(s) = \int k(s) \dif s + θ_0 \]

de tal forma que 

\[ \left.\begin{matrix}
x'(s) = \cos θ(s) \\
y'(s) = \sin θ(s)
\end{matrix}\right\} \implies
\begin{matrix}
x(s) = \int\cos θ(s)\dif s + x_0 \\
y(s) = \int \sin θ(s) \dif s + y_0
\end{matrix} \]

Así reconstruimos la curva salvo un movimiento rígido (giro o traslación).

Además, dadas dos curvas PPA, $\appl{α,b}{I}{ℝ^2}$ con la misma curvatura escalar $k_α=k_β$, entonces existe un único movimiento rígido $M$ de $ℝ^2$ que preserva orientación, tal que 

\[ β(s) = M ○ α(s)\; ∀s∈I \]

\section{Curvas en el espacio} 

Tal y como hacíamos en curvas en el plano, en el espacio partimos del triedro de Frenet\index{Triedro!de Frenet}. Además de los vectores tangente $\ct$ y normal $\cn$, tendremos el vector binormal:

\[ \cb = \ct × \cn \]

de tal forma que $\{\cb,\ct,\cn\}$ es una base del espacio $ℝ^3$. Además, $\{\ct,\cn\}$ es una base del plano osculador.

\appendix
\chapter{Apéndices}
\documentclass{apuntes}

\begin{document}
\section{Hoja 1}
\begin{problem}[1]
 Decide de manera razonada si los siguientes conjuntos son grupos con la operación definida.
 
 \begin{itemize}
  \item a) $(\real,+)$
  \item b) Fijado $n \in \mathbb{Z}_{n>0}$, el conjunto de los enteros módulo $n$ con la suma.
  \item c) $(C^*,\cdot)$
  \item d) $(U (n), ·)$, donde $U (n)$ denota los restos módulo $n$ de enteros coprimos con $n$.
  \item e) Dado un conjunto no vacío $X$, el conjunto $G$ de las biyecciones de $X$ con la composición, $(G, \circ)$.Calcula el cardinal de G si X es un conjunto finito.
 \end{itemize}

 \solution

\paragraph{a)}
$$\left. \begin{matrix}
\text{¿Es asociativo? Sí}\\
\text{¿Tiene elmento neutro? Sí: } \\a + 0 = a, \forall a \in \real\\
\text{¿Existe el inverso de todo elemento? Sí: } \\a^{-1} = -a, \forall a \in \real
\end{matrix}\right\} \implies \text{Sí es un grupo.}$$

\paragraph{b)}
$$\left. \begin{matrix}
\text{¿Es asociativo? Sí}\\
\text{¿Tiene elmento neutro? Sí: } \\\gor{a} + \gor{n} = \gor{a}, \forall \gor{a} \in \mathbb{Z}_n\\
\text{¿Existe el inverso de todo elemento? Sí: } \\ \gor{a} + (\gor{n} - \gor{a}) = \gor{n} = e \implies a^{-1} = \gor{n} - \gor{a}, \forall a \in \real
\end{matrix}\right\} \implies \text{Sí es un grupo.}$$

\end{problem}

\section{Hoja 2}
\subsection{Problema 4}
Sea G un grupo. Demostrar que $Z(G) = \{X \tq X\in G (\forall Y \in G) XY = YX\}$

\begin{itemize}
\item $1 \in Z(G)$
\item {$X_1,X_2 \in G X_1Y = YX_1 \rightarrow X^{-1}XYX^{-1} = X^{-1}YXX^{-1} \implies X^{-1}Y=YX^{-1}$, es decir, el inverso también conmuta, por lo que los inversos de $X_1,X_2 \in Z(G)$}
\item {$X_1 X_2 Y = X_1 Y X_2 = YX_1 X_2, X_1 \cdot X_2 \in Z(G)$ el producto de 2 elementos del grupo está en el centro, por lo que es cerrado por la operación.}
\end{itemize}
\paragraph{a)}
$$Z(D_3), D_3 = \left\{\begin{matrix} 1,a,a^2\\b,ab,a^2b\end{matrix}\right\},\left\{\begin{matrix}a^3 = 1 = b^2\\ ba^j = a^{-j}b\end{matrix}\right\}$$

\paragraph{b)}
$$Z(D_4) =\left\{\begin{matrix} 1,a,a^2,a^3\\b,ab,a^2b,a^3b\\a^4=1=b^2\\ba=a^{-1}b\end{matrix}\right\}$$

$$Z(D_4) = \left\{x \tq x\in D_4 axa^{-1} = x = bxb^{-1}\right\}$$

Sin sentido...
$a^ib \notin Z(D_4). (a^2 \neq 1)$

$a^i \in Z(D_4) \dimplies (a^{2i} = 1)$

$Z(D_4) = \{1,a^2\} = <a^2>$
\subsection{5}

\paragraph{b)}
$C_{D_4}(b)$. Basta con comprobar la conmutación con $a^j$ y con $a^jb$ siendo $j = 0,1,2,3$, ya que con eso podemos ver la conmutación con todos los elementos. Se puede demostrar la conmutatividad multiplicando a derecha e iezquierda por $b$ y $b^{-1}$ y si nos queda $=1$, es conmutativo.

$$\left\{\begin{matrix}b(a^j)b^{-1} = a^{-j}, a^j \in C_{D_4}(b) \dimplies a^2j = 1\\
b(a^jb)b^{-1} = a^{-j} = a^{-j}b, a^jb \in C_{D_4}(b) \dimplies a^2j = 1\end{matrix}\right.$$

\subsection{9}
\paragraph{a)}
\paragraph{b)}

$$f(x,y) = \int_a^xy g(s)ds$$
Aplicando el teorema fundamental del cálculo $\left(f \text{ continua } \implies\displaystyle\int_a^b f(x)dx = F(b)-F(a)\right)$
$$\dpa{f}{x} = g(xy)\underbrace{\dpa{xy}{x}}_{=y} - \underbrace{g(a)\dpa{a}{x}}_{=0}  = g(xy)y$$
$$\dpa{f}{y} = \cdots  = g(xy)x$$

\subsection{Ejercicio de examen:}
$\appl{g}{\real}{\real}$ continua, con $g(1) = 4$.

Sea $f(x,y,z)=\displaystyle \int_0^{x^2ye^z} g(t)dt$.

Demostrar que $f$ es diferenciable y calcular $\nabla f(1,1,0)$.

\section{Hoja 3}
\subsection{Problema 3:}
\paragraph{a)}

$\appl{f}{\real}{\real} \exists f'(x) \neq 0 \implies f$ inyectiva.

Solución: aplicando el teorema del valor medio.

\paragraph{b)}
$\appl{f}{\real^2}{\real^2}$

$f(x,y) =( e^xcos(y) + 2e^xsen(y),-e^xcos(y))$

$$J = \begin{pmatrix}
       e^xcos(y)+2e^xsen(y) & e^xcos(y)+2e^xsen(y) \\
       -e^xcos(y) & e^xsen(y)
      \end{pmatrix}
$$

Calculamos $$det(J) = (e^xcos(y)+2e^xsen(y))+e^xsen(y) + e^xcos(y)sen(y)(-e^xsen(y)+2e^xcos(y)) = $$
$$ = ... = 2e^x > 0 \forall x \in \real$$

Aunque el jacobiano sea siempre positivo, $f$ no es inyectiva porque si tomamos $f(0,0) = (1,-1) = f(0,2\pi)$.

\subsection{inventado:}
\label{inventado}
Sea $F(x,y) = (x^2-y^2,2xy)$. Encontrar los puntos en los que la siguiente aplicación es localmente inversible de clase $C^1$.

\begin{itemize}
 \item 1) $F \in C^1$ por ser $F_1,F_2$ polinomios.
 \item 2)$det(J)>0 \forall (x,y)\in \real^2$. 
 
 En este caso: $$det\begin{pmatrix}
                  2x&-2y\\
                  2x&2y
                 \end{pmatrix} = 4x^2 + 4y^2 = 0 \dimplies (x,y) = (0,0)$$           
 \item 3) Por el teorema de la funcion inversa, existe una inversa local de $F,C^1$ en todo entorno de $(x,y) \in \real^2$ con $(x,y)\neq (0,0)$. 
 
 Está la posibilidad de que exista la función inversa, pero no podemos deducir nada del teorema. Para verlo, recurrimos a la definición de inyectividad, y en este caso, no es inyectiva porque es una función par.
 \end{itemize}
 \subsection{5}
 \paragraph{a)}
 $f\in C^1(\real), f'\neq0$.
 No tiene sentido...
 $$\left\{\begin{matrix} u(x,y) =f(x)\\v(x,y) = -y + f(x)\end{matrix}\right.$$
 Probar que tiene inversa global.
 
 Mismos pasos que en el ejercicio anterior:
 \begin{itemize}
 \item 1) $F \in C^1$ por ser $F_1,F_2$ , porque $f\in C^1$.
 \item 2)$det(J)>0 \forall (x,y)\in \real^2$.
 
 $$det(J) = det\begin{pmatrix} f'(x)&0\\f(x)+xf'(x)&-1\end{pmatrix} = -f'(x) \neq 0\text{ por hipótesis}$$
 
 Como nos piden calcular las derivadas parciales de la función inversa. (La inversa de la matriz jacobiana, es la jacobiana de la matriz inversa)
 $$J(0,0) = \begin{pmatrix}f'(0) & 0 \\f(0) & -1\end{pmatrix}$$
 Lo que buscamos en la matriz inversa, que en este caso es ella misma.
 
 El teorema solo nos demuestra la existencia de la inversa local (contraejemplo:\ref{inventado}. Hay que ver la inyectividad para hablar de inversa global.
 
\begin{gather*}
F(x,y) = (u(x,y),v(x,y))\\
\text{Condición: }F(x_1,y_1) = F(x_2,y_2) \implies x_1=x_2, y_1=y_2\\
u(x_1,y_1) = u(x_2,y_2) \implies f(x_1) = f(x_2)\\
f' \text{ no se anula } \implies \text{f es inyectiva} \implies x_1=x_2\\
v(x_1,y_1) = v(x_2,y_2) -y_1 + x_1f(x_1) = -y_2 + x_2f(x_2) \underbrace{\implies}_{x_1=x_2}\\
y_1=y_2
\end{gather*}
Hemos demostrado que $F$ es inyectiva y por lo tanto admite inversa global.
 
\end{itemize}


\subsection{Problema 6:}

$$F(x,y,z) = \left\{\begin{matrix}u = 2x+2x^2y+2x^2z+2xy^2+2xyz\\v=x+y+2xy+2x^2\\w=4x+y+z+3y^2+3z^2+6yz\end{matrix}\right.$$
\begin{itemize}
 \item $u,v,w \in C^1$ por se surma de polinomios. 
 \item \begin{gather*}
\dpa{u}{x} = ... \implies \dpa{u}{x}(0,0) = 2\\
\dpa{u}{y} = ... \implies \dpa{u}{y}(0,0) = 0\\
\dpa{u}{z} = ... \implies \dpa{u}{z}(0,0) = 0\\
\dpa{v}{x} = ... \implies \dpa{v}{x}(0,0) = 1\\
\dpa{v}{y} = ... \implies \dpa{v}{y}(0,0) = 0\\
\dpa{v}{z} = ... \implies \dpa{v}{z}(0,0) = 0\\
\dpa{w}{x} = ... \implies \dpa{w}{x}(0,0) = 4\\
\dpa{w}{y} = ... \implies \dpa{x}{y}(0,0) = 1\\
\dpa{w}{z} = ... \implies \dpa{w}{z}(0,0) = 1
       \end{gather*}
       
   $det(J) =\begin{pmatrix}
             2&0&0\\
             1&1&0\\
             4&1&1
            \end{pmatrix}
 = 2 \neq 0 \implies \exists $ inversa local de clase $C^1$ en un entonrno del origen.
\end{itemize}

\subsection{8:}
\paragraph{a)}

$$det(J) = det\begin{pmatrix}
       cos(\varphi)&-rsen(\varphi)&0\\
       sen(\varphi)&rcos(\varphi)&0\\
       0&0&1
      \end{pmatrix} = rcos^2(\varphi) + rsen^2(\varphi) = r$$
      
      Por tanto, por el teorema de la función inversa, existe una inversa de clase $C^1, \forall (r,h,\varphi) \dimplies r\neq 0$.

 \subsection{9:}
 
 \paragraph{b: Calcular la inversa en (2,-2$\sqrt{3}$)}
 
 Resolver: $$\left\{\begin{matrix} 2 = rcos(\varphi)\\-2\sqrt{3} = rsen(\varphi)\end{matrix}\right.$$
 
 Hay que hallar la inversa de: $$\begin{pmatrix}
                                  \frac{1}{2}&2\sqrt{3}\\
                                  \frac{-\sqrt{3}}{2}&2
                                 \end{pmatrix}$$

                                 
  \subsection{13}
  
  $$det(J) = \begin{pmatrix}
              \dpa{f_1}{x}&\dpa{f_1}{y}\\
              \dpa{f_2}{x}&\dpa{f_2}{y}
             \end{pmatrix} = 
             \begin{pmatrix}
              \dpa{f_1}{x}&-\dpa{f_2}{x}\\
              \dpa{f_2}{x}& \dpa{f_1}{x}
             \end{pmatrix}
	    = \left(\dpa{f_1}{x}\right)^2 + \left(\dpa{f_2}{x}\right)^2 \implies \left(\dpa{f_1}{x},\dpa{f_2}{x}\right)$$
Esto es aplicando la primera ecuación de Cauchy-Riemman. Obteniendo una condición

Aplicando la otra condición en el jacobiano llegamos a $\displaystyle\left(\dpa{f_1}{y},\dpa{f_2}{y}\right)\neq (0,0)$
\paragraph{c)}

Queremos ver que $g(x,y) = (f_1(x,y)^2-f_2(x,y)^2,2f_1(x,y)f_2(x,y))$ cumple las ecuaciones de Cauchy-Riemman. Facilito.
\end{document}

\end{document}