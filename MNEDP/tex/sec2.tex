\newpage
\section{Método de elementos finitos}

\subsection{Espacio de funciones}
Vamos a trabajar sobre un abierto $\Omega\subset\mathbb{R}^d$, $d=1,2,3$.
Sea $\alpha = (\alpha_1,\hdots,\alpha_d)\in\mathbb{R}^d$ y denotaremos como $D^\alpha u=\frac{\partial}{\partial x_1}\alpha_1\frac{\partial}{\partial x_2}\alpha_2\hdots\frac{\partial}{\partial x_d}\alpha_d$.

Denotaremos como 
$$e^k(\Omega) = \{\textbf{Funciones continuas y con } D^\alpha u \textbf{ continua } \forall\alpha\textbf{ con }|\alpha|\le k\}$$

En cuanto a la norma en $e^k(\Omega)$ es
$$||u||_{e^k(\Omega)} = \sum_{|\alpha|\le k}\sup_{x\in\Omega}|D^\alpha u(x)|$$

El soporte de una función continua en $\Omega$ es la adherencia del conjunto $\{x\in\Omega \tq u(x) \neq 0\}$.

Definimos
$$e_0^k (\Omega)= \{u\in e^k(\Omega) \textbf{ con soporte compacto}\}$$

Si $u\in e_0^k(\Omega)$, $u=0 en \delta\Omega$.

\subsection{Espacios de funciones integrables}
$$L^p(\Omega) = \{u\textbf{ definida en }\Omega \textbf{ con }\int_\Omega |u|^p dx < \infty \}$$
$$L^2(\Omega)=\{\textbf{funciones de cuadrado integrable}\}$$
$$L^\infty (\Omega) =\{\textbf{funciones acotadas}\}$$

$$||u||_{L^p(\Omega)} = (\int_{\Omega}|u|^p dx)^\frac{1}{p}$$
$$||u||_{L^2(\Omega)} = (\int_{\Omega}|u|^2 dx)^\frac{1}{2}$$
$$||u||_{L^\infty(\Omega)} = \sup_{x\in\Omega}|u(x)|$$

$L^2(\Omega)$ es un espacio de Hilbert: un espacio normado completo donde la norma deriva de un producto interno.

\paragraph{Desigualdad de Cauchy-Schwarz}
Si $u,v\in L^2(\Omega)\implies u,v\in L^1(\Omega)$ y $(\int_\Omega |u|^2dx)^\frac{1}{2} = ||u||_{L^2(\Omega)}$ y $(\int_\Omega|v|^2)^\frac{1}{2} = ||v||_{L^2(\Omega)}$

\paragraph{Desigualdad triangular}
$$||u+v||\le ||u||_{L^p(\Omega)}+||v||_{L^p(\Omega)}$$

\paragraph{Desigualdad de Holder}
Sean $p,q$ con $\frac{1}{p}$+$\frac{1}{q} = 1$

Entonces
$$|\int_{\Omega}u(x)v(x)dx|\le||u||_{L^p(\Omega)}||v||_{L^q(\Omega)}$$

Si tomamos $p=q=2$, entonces tenemos la desigualdad de Cauchy-Schwarz.

\subsection{Espacios de Sobolev}
\paragraph{Concepto de derivada débil}
Se dice que $w_\alpha(x)$ es la derivada $D^\alpha u$ en sentido débil si
$$\int_{\Omega}w_\alpha u(x) v(x)dx = (-1)^{|\alpha|}\int_{\Omega}u(x)D_v^\alpha (x) dx 
\forall v\in e_0^\infty(\Omega)$$

Si existe $D^\alpha u$ entonces $w_\alpha = D^\alpha u$.

$$\int_\Omega w_\alpha (x)v(x) dx = \int_{\Omega}D^\alpha u(x) v(x)dx = (-1)^|\alpha|\int_{\Omega}u(x)D^\alpha v$$

\begin{example}
	Sea la función 
	\begin{equation*}
		u=\left\{
		\begin{array}{l l l}
			2x & x\in[0,\frac{1}{2}]\\
			2-2x & x\in [\frac{1}{2},1]
		\end{array}
		\right.
	\end{equation*}
	
	La función $u$ es continua pero no derivable. Existe la derivada débil de $u$ que es
	\begin{equation*}
		\omega_1(x)=\left\{
		\begin{array}{l l l}
			2 & x\in[0,\frac{1}{2}]\\
			-2 & x\in [\frac{1}{2},1]
		\end{array}
		\right.
	\end{equation*}
	
	Si $\omega_1$ es la derivada débil se debe cumplir que
	$$\int_0^1 \omega_1(x)v(x) = -\int_0^1u(x)v'(x) \forall v\in e_0^\infty (0,1)$$
	
	\begin{itemize}
		\item \textbf{Paso 1:}
		$$\int_0^1\omega_1(x)v(x) = 2\int_0^\frac{1}{2} v(x)dx-2\int_\frac{1}{2}^1v(x)dx$$
		\item \textbf{Paso 2:}
		$$-\int_0^1u(x)v'(x)dx = -\int_0^\frac{1}{2}u(x)v'(x)dx -\int_\frac{1}{2}^1 u(x)v'(x)dx$$
		
		Si integramos por partes, obtenemos $$2\int_0^\frac{1}{2} v(x)dx-2\int_\frac{1}{2}^1v(x)dx$$
	\end{itemize}
\end{example}

\begin{defn}[Espacio de Sobolev $H^k(\Omega)$]
	$$H^k(\Omega) = \{u \textbf{ definida en }\Omega \textbf{ con } D^\alpha u\in L^2(\Omega)\forall \alpha \textbf{ con }|\alpha|\le k\}$$
	
	Nos referimos a funciones en $L^2$ con derivadas hasta orden $k$ en $L^2$. $D^\alpha u$ es la derivada en sentido débil.
\end{defn}

	Sea $u\in H^1(\Omega)$, si $H^k(\Omega)$ es un espacio de Hilbert.
	$$||u||_{H^k(\Omega)} = (\sum_{|\alpha|\le k}||D^\alpha u||_{L^2(\Omega)}^2)^\frac{1}{2}$$
	
	El producto interno es:
	$$(u,v)_{H^k(\Omega)} = \sum_{|\alpha|\le k}(D^\alpha u, D^\alpha v)_{L^2(\Omega)}$$
	
	con 
	$$||u||_(H^k(\Omega)) = (u,u)^\frac{1}{2}_{H^k(\Omega)}$$
	
	\begin{example}
		Sea $\Omega\subset\mathbb{R^2}$
		
		Definimos
		$$H^1(\Omega) = \{u\in L^2, u_x\in L^2, u_y\in L^2\}$$
		
		$$||u||^2_{H^1(\Omega)} = ||u||^2_{L^2} + ||u_x||^2_{L^2} + ||u_y||^2_{L^2} = 
		\int_\Omega u^2 + \int_\Omega u_x^2 + \int_\Omega u_y^2$$
	\end{example}
	
	En $H^k(\Omega)$ definimos también una seminorma:
	$$|u|_{H^k(\Omega)}  = (\sum_{|\alpha|=k}||D^\alpha u||^2_{L^2(\Omega)})^\frac{1}{2}$$
	
	\begin{example}
		$$|u|^2_{H^1(\Omega)} = \int_{\Omega}u_x^2+\int_{\Omega}u_y^2$$
		
		Si $|u|_{H^k(\Omega)} = 0 \nimplies u = 0$ en $\Omega$.
	\end{example}
	
	$$H_0^1(\Omega) = \{\textbf{Funciones de } H^1(\Omega)\textbf{ con } u = 0 \textbf{ en } \delta \Omega \}$$
	
	\paragraph{Desigualdad de Poincaré}
	Sea $\Omega\in\mathbb{R}^d$ abierto acotado con frontera suficientemente regular. Sea $u\in H_0^1(\Omega)$, entonces existe una constante $C_*(\Omega)$ independiente de $u$ tal que
	$$\int_{\Omega}|u|^2 dx \le c_* \sum_{i=1}^n\int_{\Omega}|\frac{\partial u}{\partial x_i}|^2dx$$
	
	$$||u||^2_{L^2(\Omega)}\le C_*||\nabla u||^2_{L^2(\Omega)}$$
	
	Siendo $\nabla u = (\frac{\partial u}{\partial x_1}\hdots\frac{\partial u}{\partial x_d})$
	
	\begin{example}
		Si $u\in H_0^1(\Omega)\implies$
		$$\int_{\Omega}u^2 \le C^*\left[\int_{\Omega}u_x^2+\int_{\Omega}u_y^2\right]$$
	\end{example}
	
	En $H_0^1(\Omega)$ son equivalentes la norma y la seminorma.

