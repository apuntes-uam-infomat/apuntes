\subsection{Máximos y mínimos}

\begin{defn}[Máximo/mínimo\IS local] Sea $\appl{f}{\real^N}{\real}$. Diremos que $\vx_0 \in \real^N$ es un punto de máximo local si $\exists \epsilon > 0 \tq F(\vx_0) \geq F(\vx)\;\; \forall \vx \in B_{\epsilon} (\vx_0) $

La definición es análoga para el mínimo\end{defn}

\begin{remark} Por las propiedades del gradiente, si $F$ es diferenciable y $\vx_0$ es un máximo o mínimo local, entonces debe ser $\nabla F(\vx_0) = \vec{0}$.\end{remark}

\begin{defn}[Punto\IS crítico] $\vx \in \real^N$ es un punto crítico de $F$ si y sólo si $\nabla F(\vy) = \vec{0}$\end{defn}

No todos los puntos críticos son máximos o mínimos, así que tenemos que clasificarlos de alguna forma. Para ello, usamos el polinomio de Taylor de orden 2, de forma que 

\[F(x,y) = F(x_0, y_0) + \pesc{\nabla F(x_0, y_0), (x-x_0, y-y_0)} +\]\[\frac{1}{2}(x-x_0, y-y_0)\left(\begin{matrix} \frac{\partial^2 f}{∂ x^2} (x_0,y_0) & \frac{\partial^2 f}{\partial x \partial y} (x_0,y_0) 
\\ \frac{\partial^2 f}{\partial y \partial x} (x_0,y_0) & \frac{\partial^2 f}{\partial y^2} (x_0,y_0) \end{matrix}\right) \left(\begin{matrix} x - x_0 \\ y - y_0 \end{matrix}\right) + \epsilon \]

Simplificando nos queda que:

\[F(\vx) = F(\vx_0) + \pesc{\nabla F(\vx_0),\vx - \vx_0} + \frac{1}{2}(\vx - \vx_0) D^2F(\vx_0) (\vx - \vx_o) ^T + \epsilon\]

Dado que el gradiente es 0, el punto clave es el signo de $\frac{1}{2}(\vx - \vx_0) D^2F(\vx_0) (\vx - \vx_o) ^T$. Para ello, usamos las siguientes definiciones del álgebra lineal.

\subsubsection{Resultados de álgebra lineal}
\index{Matriz!semidefinida positiva/negativa}
\index{Matriz!definida positiva/negativa}
\begin{defn}[Matriz semidefinida y definida positiva y negativa][]\noindent\\ \indent
La matriz $A$ de dimensión $N\x N$ es semidefinida positiva si y sólo si $\vv A \vv^T \geq 0 \;\; \forall \vv \in \real^N$.

La matriz $A$ de dimensión $N\x N$ es definida positiva si y sólo si $\vv A \vv^T > 0 \;\; \forall \vv ≠0 \in \real^N$.

La matriz $A$ de dimensión $N\x N$ es semidefinida negativa si y sólo si $\vv A \vv^T \leq 0 \;\; \forall \vv \in \real^N$.

La matriz $A$ de dimensión $N\x N$ es definida negativa si y sólo si $\vv A \vv^T < 0 \;\; \forall \vv ≠0 \in \real^N$.
\end{defn}

\begin{theorem}
Si una matriz es simétrica, existe una base en la cual la matriz es diagonal.
\end{theorem}

\index{Autovalor}
\index{Autovector}
Sea $A$ una matriz $N\x N$. Entonces diremos que un vector $\vv ≠ \vec{0}$ es un autovector asociado al autovalor $\lambda\in \real$ si y sólo si $A\vv = \lambda\vv$. Dado que podemos escribir

\[ A = \left(\begin{matrix}
a & b \\ c &  d
\end{matrix}\right)\;\;\;\; \vv = \left(\begin{matrix}
x\\ y
\end{matrix}\right) \], entonces tenemos que $A\vv = \lambda \vv$ si y sólo si

\[ \left\lbrace\begin{matrix}ax+by=\lambda x \\ cx+dy = \lambda y \end{matrix}\right. \]

Es decir, la autorrecta $\begin{pmatrix} x \\ y \end{pmatrix}$ es una solución no trivial del sistema anterior. Sin embargo, para que haya soluciones no triviales el determinante de la matriz $\begin{pmatrix}a-\lambda & b \\ c & d - \lambda\end{pmatrix}$ debe ser 0.

Por lo tanto, los autovalores son las soluciones de la ecuación $det(A-\lambda I) = 0$, siendo $I$ la matriz identidad.

\begin{theorem}
Si un conjunto de autovectores es una base, entonces la matriz $A$ expresada respecto a esa base pasa a ser diagonal, y los elementos de la diagonal son los autovalores. 

Si dos autovalores son distintos, los autovectores asociados son distintos.

Si A es simétrica, entonces el conjunto de autovectores es una base.
\end{theorem}

Volvemos ahora al cálculo.

\begin{theorem}[Clasificación de puntos críticos][]
Sea $\appl{F}{\real^N}{\real}$, $F\in C^2$ (con dos derivadas continuas), y sea $\vx_0$ un punto crítico. Entonces

\begin{enumerate}
\index{Máximo/mínimo!local}
\index{Punto!de silla}
\index{Punto!crítico degenerado}

\item Si \textbf{todos} los autovalores de $D^2F(\vx_0)$ son \textbf{mayores que cero}, entonces $D^2F(\vx_0)$ es definida positiva y $\vx_0$ es un \textbf{mínimo local}.
\item Si \textbf{todos} los autovalores de $D^2F(\vx_0)$ son \textbf{menores que cero}, entonces $D^2F(\vx_0)$ es definida negativa y $\vx_0$ es un \textbf{máximo local}.
\item Si \textbf{algunos} autovalores son \textbf{mayores que cero} y otros son \textbf{menores que cero}, entonces $\vx_0$ es un \textbf{punto de silla.}
\item Si algún autovalor \textbf{es 0}, y el resto son mayores o menores que cero, entonces $\vx_0$ es un \textbf{punto crítico degenerado}.
\end{enumerate}
\end{theorem}

\subsubsection{Ejemplos}

Tomamos $F(x,y) = x^2 + y^2 +xy$. Obtenemos los puntos críticos, es decir, los puntos en los que $\nabla F(x,y) = (0,0)$. El punto resultante es $(0,0)$. Estudiamos el tipo de punto crítico. Para ello, calculamos la matriz hessiana en ese punto:

\[ D^2F(0,0) = \begin{pmatrix}2&1\\1&2\end{pmatrix}\]. 

Los autovalores son las soluciones de

\[ 0 = det\left(\begin{pmatrix}2&1\\1&2\end{pmatrix} - \lambda \begin{pmatrix}1&0\\0&1\end{pmatrix}\right) = det\begin{pmatrix}2-\lambda & 1 \\ 1 & 2-\lambda\end{pmatrix} = (2-\lambda)^2  - 1\]

Por lo tanto, $\lambda$ es $3$ o $1$. Dado que ambos autovalores son mayores que 0, entonces $D^2F$ es definida positiva y $(0,0)$ es un mínimo local.

\subsection{Máximos y mínimos absolutos}

\begin{defn}[Máximo/mínimo\IS absoluto] Sea $\appl{F}{\real^N}{\real}$ y $A\subset \real^N$. $\vx_m$ es un máximo absoluto de $F$ en $A$ si y sólo si $F(\vx_m) ≥ F(\vx)\;\; \forall \vx \in A$. La definición es análoga para el mínimo.
\end{defn}

\begin{theorem}[Teorema\IS de compacidad]
Tenemos un conjunto $K\subset \real^N$ compacto (cerrado y acotado). Supongamos la sucesión $\{\vx_n\}_{n\in\nat}\subset K$. Entonces podemos encontrar al menos una subsucesión $\{\vx_{n_j}\}_{j\in\nat} \subset \{\vx_n\}_{n\in\nat}$ tal que $\{\vx_{n_j}\}$ es convergente.
\end{theorem}

\begin{proof}
Trabajamos en dimensión 2, pero la demostración es análoga.
Como $K$ es compacto, podemos encontrar un cuadrado $Q_0$ de lado $L$ que encierre completamente a $K$. Divido $Q_0$ en $2^2$ cuadrados de lado $L/2$.  En alguno de ellos hay infinitos términos de la sucesión: lo llamamos $Q_1$ y me quedo con uno de los términos de la sucesión, al que llamamos $x_1$. Volvemos a dividir este cuadrado en cuatro cuadrados, elegimos uno que tenga infinitos términos de la sucesión y seleccionamos un elemento de la sucesión dentro al que llamamos $x_2$. Repetimos esto muchas veces, de forma que cada término $x_n$ está encerrado en el cuadrado $Q_n$ de lado $\frac{L}{2^n}$. 

Si $k,l > n$, entonces es claro que $\md{\vx_k-\vx_l}$ es menor o igual que la diagonal de $Q_n$, que es $\frac{L}{2^n}\sqrt{2}$, que tiende a cero cuando $n\to\infty$. Por el criterio de Cauchy, entonces esta sucesión es convergente, y como $K$ es cerrado el límite pertence a $K$.
\end{proof}

\begin{theorem}
Sea $K\subset \real^N$ compacto y $\appl{F}{\real^N}{\real}$, continua en $K$. Entonces, $F$ alcanza su máximo y mínimo absolutos en $K$.
\end{theorem}

\begin{proof}

Como $F$ es acotada, existe $\alpha = \sup \{ F(x) \tq x \in K\}$. Existe entonces una sucesión $\{x_n\}$ tal que si $n\to \infty$ entonces $F(x_n)\to \alpha$. 
Sabemos que existe $\{x_n\}\subset K$, por lo que existe una subsucesión$\{x_{n_j}\}$ convergente tal que $x_{n_j} \to x_0 \in K$

Como $F$ es continua, $F(x_{n_k})\to F(x_0)$, es decir $F(x_{n_j}) \to \alpha$, por lo tanto el supremo es el máximo.
\end{proof}

\subsubsection{Ejemplos}

La función a estudiar es $F(x,y) = x^2-y^2$ en la bola $\omega = \{ x^2 + y ^2 ≥ 1\}$. Es diferenciable en todo $\real$ porque es un poliniomio. 

Calculamos el diferencial y vemos qué ocurre cuando es 0 \[\nabla F = (2x, -2y) = (0,0) \implies (x,y) = (0,0)\]

Operando, vemos que el punto $(0,0)$ es un punto de silla. Ahora sólo queda ver el comportamiento en la frontera $C$, cuando $x^2+y^2 = 1$. $F$ restringida a $C$ quedaría de la siguiente forma:

\[ F(\cos t, \sin t) = \cos^2 t - \sin^2 t = \cos 2t\]

El coseno tiene máximos cuando $t=0$ y $t=\pi$, y mínimos cuando $t=\pi /2$ y $t=3\pi /2$. Es decir, tiene máximos absolutos en los puntos $(1,0),\;(-1,0)$ y mínimos absolutos en $(0, -1)$, $(0, 1)$. 

\begin{theorem}[Teorema\IS de los multiplicadores de Lagrange]
Tenemos una función $\appl{F}{\real^2}{\real}$ y una restricción $G(x_1,\cdots,x_n) = k$, resolvemos el siguiente sistema:

\begin{align*}
\nabla F &= \lambda \nabla G \\
G &= k
\end{align*} 

\end{theorem}
