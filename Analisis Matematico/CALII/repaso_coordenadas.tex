\paragraph{Ejemplos}

\index{Helicoide}\label{Helicoide}
El \textbf{helicoide} es una función $\Phi$ de dos variables, que crea una superficie helicoidal (escalera de caracol) en el espacio, tal que 

\begin{align*} \appl{\Phi}{\real^2 &}{\real^3} \\
 (s,t) &\longmapsto \Phi(s,t) = (s\cos t, s\sin t, t) \end{align*}
 
 \easyimg{CALII/Helicoide.png}{El helicoide}{lblHelic}

En general, una superficie parametrizada es una aplicación $\Phi$ de la siguiente forma:

\begin{align*} \appl{\Phi}{\Omega \subset \real^2 &}{\real^3} \\
 (s,t) &\longmapsto \Phi(s,t) = (x(s,t), y(s,t), z(y,t)) \end{align*}

\index{Paraboloide}
Por ejemplo, el \textbf{paraboloide} se puede definir como una superficie parametrizada \[\Phi(x,y) = (x,y,x^2+y^2)\].

\index{Cilindro}
Otro ejemplo es el \textbf{cilindro} de radio 2. Para parametrizarlo, usamos coordenadas polares de la siguiente forma:

\[ \Phi(t,z) = (2\cos t, 2\sin t, z)\;\;t\in [0, 2\pi],\;\; z\in \real \]

\index{Esfera}
Si queremos hacer la \textbf{esfera} de radio $R$, en la parametrización usamos dos parámetros (longitud y latitud)

\[ \Phi(\theta, \phi) = (R\sin\phi\cos\theta, R\sin\phi\sin\theta, R\cos\phi)\;\; \theta\in[0,2\pi],\;\;\phi\in[0, \pi] \]

\index{Toro}
El \textbf{toro} (una especie de flotador) se produce al girar una circunferencia de radio $R_1$ en el plano $XZ$ con el centro sobre el eje $X$ a $R_2$ del origen alrededor del eje $Z$.

\[ \Phi(\theta, \phi) = ((R_2+R_1\cos\phi)\cos\theta,(R_2+R_1\cos\phi)\sin \theta,R1\sin \phi)\;\; \theta, \phi \in [0, 2\pi] \]

 \easyimg{CALII/Toro.png}{Toro}{lblToro}
\paragraph{Parametrización de superficies de revolución}
\index{Superficie!de revolución}

Dada una función $\appl{f}{\real}{\real}$, la superficie de revolución que surge al rotar esta función sobre el eje z es la siguiente

\begin{align*} z(r,\theta) &= f(r) \\
x(r,\theta) &= r \cos \theta \\
y(r,\theta) &= r \sin \theta \end{align*} 

\paragraph{Coordenadas cilíndricas}
\index{Coordenadas!cilíndricas}
Dado un punto $P$, sus coordenadas cartesianas son $(x,y,z)$. Entonces, sus coordenadas cartesianas son $(r,\theta, z)$, donde $r\in [0,\infty)$, $\theta \in [0,2\pi]$ y $z\in \real$. La correspondencia es la siguiente:

\begin{align*}
x &=r \cos \theta \\
y &= r \sin \theta \\
z &= z
\end{align*}

Geométricamente, $z$ es la altura de $P$. Haciendo la proyección del punto sobre el plano $xy$, $r$ es la distancia de la proyección al origen y $\theta$ el ángulo del eje $X$ con la recta que une el origen y la proyección del punto.

\paragraph{Coordenadas esféricas}
\index{Coordenadas!esféricas}
Las coordenadas esféricas de un punto $P$ son $(\rho, \theta, \phi)$, donde $\rho \in [0, \infty)$, $\theta\in [0, 2\pi]$, $\phi \in [0, \pi]$. La correspondencia con coordenadas cartesianas es

\begin{align*}
x&=\rho \cos \theta \sin \phi \\
y&= \rho \sin \theta \sin \phi \\
z &= \rho \cos \theta
\end{align*}

Geométricamente, $\rho$ es la distancia de $P$ al origen, $\theta$ es el ángulo que forman el eje $X$ y la recta que une $P$ y el origen, y $\phi$ es el ángulo que forman el eje $Z$ y esa misma recta.

\paragraph{Usos de coordenadas esféricas y cilíndricas}

En las coordenadas cilíndricas, si mantenemos constante un parámetro y variamos los otros dos obtenemos varias superficies:

\begin{itemize}
\item Si $r$ constante, tenemos un cilindro. \index{Cilindro}
\item Si $\theta$ constante, tenemos un semiplano. \index{Semiplano}
\item Si $z$ constante, tenemos un plano horizontal. \index{Plano}
\end{itemize}

Lo mismo ocurre con las coordenadas esféricas:

\begin{itemize}
\item Si $\rho$ constante, tenemos una esfera. \index{Esfera}
\item Si $\theta$ constante, tenemos un semiplano.\index{Semiplano}
\item Si $\phi$ constante, tenemos un cono. \index{Cono}
\end{itemize}