\section{Teoría de la integración}

\subsection{Repaso rápido}

\paragraph{Dimensión 1} Integral de Riemann.

Tomábamos una partición de $\mathcal{P}([a,b]) = a = t_0 < t_1<...<t_k = b$ y a cada subintervalo de esa partición $[t_i,t_{i+1}] \rightarrow \begin{array}{cc} 
M_i &= sup\{f(x)\tq x\in []\\
m_i &= inf \{f(x)\tq x\in []\end{array}$ 

Definíamos \[(1) = \mathbb{U}(f,\mathcal{P}) = \sum_{i=0}^{k-1} M_i(t_{i+1}-t_i)\]
\[(2) = \mathbb{L} = \sum_{i=0}^{k-1} \sum_{i=0}^{k-1} m_i(t_{i+1}-t_i)\]

Sea $\alpha = inf(1), \beta = sup(2)$

\textbf{Definición:} $f$ integrable en $[a,b] \dimplies \alpha = \beta \left(\text{ notación } \int_a^b f(x)dx = \alpha = \beta\right)$.

\begin{theorem}[Continua $\implies$ integrable]
$f$ continua en $[a,b] \implies f$ integrable en $[a,b]$.
\end{theorem}

Rl recíproco es \textbf{falso}. Una función escalonada.

\begin{theorem}[Integración fácil]
$f$ integrable en $[a,b]$.

Tomamos $\mathcal{P}_h$ una partición de $[a,b] \tlq max(t_{i+1}-t_i) = h$ , es decir, $h$ mide el trozo más grande de la partición.

$\lim_{h\rightarrow 0} \sum_{i=1}^{k-1} f(s_i)(t_{i+1}-t_i) = \int_a^bf(x)dx$ para \textbf{cualquier} elección $s_i\in[t_i,t_{i+1}]$	

\end{theorem} 
