\documentclass[a4paper,10pt]{apuntes}
%\documentclass[a4paper,10pt]{scrartcl}


\newcommand{\ejer}[1]{\paragraph{Ejercicio #1:}}
\newcommand{\acum}[1]{\#1 \subset(#1) }
\newcommand{\dem}{\paragraph{Demostración:}}
\title{}
\author{}
\date{}

\pdfinfo{%
  /Title    ()
  /Author   ()
  /Creator  ()
  /Producer ()
  /Subject  ()
  /Keywords ()
}

\begin{document}
\section{Hoja 1}

$$\overline{A} = { x \in \real^N; \forall V_x \tq V_x \cup A \neq \text{\O}}$$, siendo $V_x$ un entorno abierto de x.
	$\overline{A} = A \cap $ 
	
\begin{theorem}
$A \subset \real^N$ es cerrado $\dimplies \acum{A}\subset A$ 
\end{theorem}

 

\dem{}
$$A \text{es cerrado} \implies A^c \text{ es abierto} \implies \forall x \in A^c, \exists \varepsilon > 0 \tq B(x,\varepsilon) \subset A^c \implies A \cap B(x,\varepsilon) = \O \implies x \nexists \acum(A)$$
Falta la recíproca.
\ejer{3}

a)

$\displaystyle\bigcup_{k=1}^{\infty} \left[-1,\frac{1}{k}\right)$

Es cerrado, porque $=[-1,0]$
Demostración: (hay que demostrar las inclusiones $\subseteq$ y $\supseteq$)

b)
No es ni cerrado ni abierto.
\obs $\real$ es el cierre de $\mathbb{Q}$.

c)
\ejer{4}

\end{document}
