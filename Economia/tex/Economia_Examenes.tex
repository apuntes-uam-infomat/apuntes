% -*- root: ../Economia.tex -*-
\section{Final Junio 2011}
\begin{problem}[1]
Consideramos una opción que pagará a vencimiento (dentro de un año) $(S_4-A_4)^+$ siendo $A_4=\frac{1}{4}\sum_{i=1}^4S_i$, la media de los valores del subyacente dentro de tres, seis, nueve y doce meses.

\ppart Calcular el valor de esta opción. Para ello supondremos que la dinámica para los doce próximos meses viene descrita por un modelo binomial de cuatro periodos. En cada periodo el activo puede subir o bajar un 3\%. El tipo libre de riesgo es el 5\% anual para composición continua. El activo no paga dividendos y su valor actual es 10.

\ppart ¿Cuál sería el impacto en este valor si el subyacente pagase dividendos en el noveno mes?

\solution
\doneby{Pedro}

\spart

Puesto que el activo puede subir o bajar un 3\% en cada período tenemos $a=1.03$ y $b=0.97$. El enunciado nos dice que el tipo libre de riesgo es $r=0.05$ en composición continua y tenemos $Δt=1/4=0.25$. Para calcular la probabilidad riesgo neutro nos apoyamos en el hecho de que no pueden existir arbitajes en nuestro modelo por lo que el valor actual del subyacente debe coincidir con el proceso de precios descontados, es decir:
\[S_0 = \frac{aS_0p + bS_0(1-p)}{e^{rΔt}} \implies p = \frac{e^{rΔt}-b}{a-b} = 0.71\]

La siguiente tabla contiene los posibles caminos que debemos estudiar:

\begin{center}
\begin{tabular}{|c|c|c|c|}
\hline
\textbf{Camino} & \textbf{Probabilidad} & \textbf{Valor} & \textbf{Valor ponderado}\\
\hline
aaaa & 0.25  & 0.482 & 0.1205 \\
aaab & 0.10  & 0.316 & 0.0316 \\
aaba & 0.10  & 0.473 & 0.0473 \\
aabb & 0.04  & 0     & 0 \\
abaa & 0.10  & 0.307 & 0.0307 \\
abab & 0.04  & 0     & 0 \\
abba & 0.04  & 0     & 0 \\
abbb & 0.02  & 0     & 0 \\
baaa & 0.10  & 0.457 & 0.0457 \\
baab & 0.04  & 0     & 0\\
baba & 0.04  & 0.15  & 0.006 \\
babb & 0.02  & 0     & 0 \\
bbaa & 0.04  & 0.3   & 0.012 \\
bbab & 0.02  & 0     & 0 \\
bbba & 0.02  & 0     & 0 \\
bbbb & 0.007 & 0     & 0 \\
\hline
\end{tabular}
\end{center}

Para calcular el valor de la opción debemos sumar los valores ponderados y descontar dicho valor patra ``traerlo al presente''.

\[V_0(X) = \frac{0.1205 + 0.0316+0.0473 + 0.0307 + 0.0457 + 0.006 + 0.012}{e^{0.05\cdot 0.25\cdot 4}}=\frac{0.294}{1.05}=0.27\]

\spart

Si se reparten dividendos en un cierto momento, el valor del subyacente en ese momento caerá. Esto hará que la media de los valores por los que pasa el subyacente se reduzca pero también lo haga el valor final. En concreto, pagando los dividendos en el noveno mes, los valores calculados de $S_1$ y $S_2$ no se ven afectados pero $S_3$ y $S_4$ se ven reducidos un cierto porcentaje $D$, igual al dividendo pagado por acción. Así la reducción en $S_4$ es mayor que la reducción que experimenta la media $A_4$ con lo que el valor de la opción bajaría.
\end{problem}

\begin{problem}[2]
Consideremos el subyacente $S_0=50$ cuya dinámica para los próximos seis meses viene descrita por un modelo binomial de dos periodos. En cada periodo el activo puede subir o bajar un 20\%. El tipo libre de riesgo es el 12\% anual para composición continua. El activo no paga dividendos.

\ppart Calcular la probabilidad riesgo neutro de pasar al estado alto

\ppart Calcular el valor de la put americana de precio de ejercicio 52

\ppart Si el activo pagase dividendos, ¿el precio de la put sería mayor o menor?

\ppart Consideremos ahora un derivado sobre dicho subyacente que paga, al final de los seis meses, $M_2(ω)-m_2(ω)$ donde
\[M_2(ω)=\max(S_0(ω),S_1(ω),S_2(ω)), \;\;\; m_2(ω)=\min(S_0(ω),S_1(ω),S_2(ω))\]
Calcular la caretara de cobertura en cada nodo.

\solution
\doneby{Pedro}

\spart

Si en cada período el activo puede subir o bajar un 20\% tenemos $a=1.2$ y $b=0.8$. Como hicimos en el ejercicio anterior, calculamos la probabilidad riesgo neutro forzando que el valor actual del activo coincida con su valor de precios descontados, de modo que no hay arbitrajes:
\[S_0 = \frac{aS_0p + bS_0(1-p)}{e^{rΔt}} \implies p = \frac{e^{rΔt}-b}{a-b} = 0.58\]
recordemos que en este caso $r=0.12$.

\spart

La put es un contrato que nos da derecho a vender por un determinado precio. Esta opción tendrá valor no nulo cuando el subyacente al que está asociada haya disminuido de precio. En caso contrario nos interesará vender al valor actual y no al acordado en la put, por lo que no estaremos ejerciendo dicho contrato.

La put americana presenta la particularidad de que nos permite ejercer en cualquier momento.

La siguiente tabla muestra el precio de opción en $T=2$.

\begin{center}
\begin{tabular}{|c|c|}
\hline
\textbf{Camino} & \textbf{Valor}\\
\hline
aa & 0 \\
ab & 4 \\
ba & 4 \\
bb & 20 \\
\hline
\end{tabular}
\end{center}

Sin embargo, podía ocurrir que hubiésemos ejercicio antes de llegar a esta situación. Vamos a estudiar el valor de esta opción en los estados $aS_0$ y $bS_0$.

\begin{itemize}
\item \textbf{Estado $aS_0$}
En este momento no tiene sentido ejercer puesto que el subyacente vale más que el precio de ejercicio. Por tanto el valor en este estado es el resultado de los valores en $abS_0$ y $aaS_0$ descontados:

\[V_{a} = \frac{p\cdot (abS_0-K)^+ + (1-p)\cdot (aaS_0-K)^+}{e^{0.12\cdot 0.25}}=1.65\]

\item \textbf{Estado $bS_0$}

El valor de la opción si ejercemos en este punto es $V_{b_1} = K-bS_0=12$. Si no ejercemos tendremos un valor:
\[V_{b_2} = \frac{p\cdot (baS_0-K)^+ + (1-p)\cdot (bbS_0-K)^+}{e^{0.12\cdot 0.25}} = 10.46\]

Por tanto en esta situación nos interesa ejercer con lo que el valor de la opción en este punto será
\[V_b = 12\]
\end{itemize}

Ahora en el estado inicial tenemos que el precio de la opción es:
\[V(X) = \frac{12\cdot (1-p) + 1.65\cdot p}{e^{0.12\cdot 0.25}} = 5.86\]

\spart Si pagase dividendos, en el momento del pago el valor decaería con lo que la put ganaría valor haciendo que el precio de la opción aumentase.

\spart

\begin{itemize}
\item \textbf{Nodo $aS_0$}

En este momento tenemos una opción que paga $a^2S_0-S_0$ con probabilidad $p$ y $aS_0-S_0$ con probabilidad $(1-p)$.

Para replicar esta cartera podemos jugar con el subyacente $S_0$ y la cuenta bancaria con rendimiento fijo conocido. Así necesitamos $\varphi_1$ acciones del activo $S_0$ e $\varphi_0$ de la cuenta bancaria de modo que:

\[\left\{\begin{array}{l}
\varphi_0 e^{0.12\cdot 0.25}+\varphi_1 \cdot 72 = 22 \\
\varphi_0 e^{0.12\cdot 0.25}+\varphi_1 \cdot 50 = 10 \\
\end{array}\right. \implies \varphi_1 = \frac{12}{22} = 0.5, \;\; \varphi_0 = -14.56\]

\item \textbf{Nodo $bS_0$}

En este momento tenemos una opción que paga $S_0-bS_0$ con probabilidad $p$ y $S_0-b^2S_0$ con probabilidad $(1-p)$. Calculamos la cartera de cobertura:
\[\left\{\begin{array}{l}
\varphi_0 e^{0.12\cdot 0.25}+\varphi_1 \cdot 50 = 10 \\
\varphi_0 e^{0.12\cdot 0.25}+\varphi_1 \cdot 32 = 18 \\
\end{array}\right. \implies \varphi_1 = \frac{-8}{18} = -0.44, \;\; \varphi_0 = 31.13\]

\item \textbf{Nodo $S_0$}
No podemos calcular una cartera de cobertura en este nodo pues tenemos demasiados estados finales.
\end{itemize}

\end{problem}

\begin{problem}[3]
Una institución financiera entró en su día en un swap de tipos en el que acordó pagar un 6\% anual y recibir LIBOR a tres meses sobre un principal de 100 millones de euros con pagos intercambiados cada tres meses. Al swap le quedan 14 meses de vida. El tipo swap para intercambio por el LIBOR a tres meses en estos momentos es del 9\% anual para todos los vencimientos. El tipo LIBOR a tres meses publicado hace dos meses era del 12\%. Todos los tipos son compuestos trimestralmente. ¿Cuál es el valor actual del swap?
\solution
\doneby{Pedro}

A la hora de contar los pagos que quedan trabajamos igual que con los bonos: Hacemos coincidir el último pago con la última fecha y vamos contando qué
pagos previos, con los plazos dados, pueden realizarse. Así ten esta ocasión los pagos se harán a los 2, 5, 8, 11 y 14 meses.

Puesto que la parte que paga fijo acordó pagar un 6\% anual dividido en pagos cada 3 meses, cada pago consta de un 1.5\% del principal.

El valor del swap vendrá dado por la diferencia entre los dos bonos implícitos en el swap: el que paga fijo y el que paga variable.

\[B_{\text{fix}} = 1.5\cdot e^{-0.09\cdot \frac{1}{6}} + 1.5\cdot e^{-0.09\cdot \frac{5}{12}} + 1.5\cdot e^{-0.09\cdot \frac{2}{3}} + 1.5\cdot e^{-0.09\cdot \frac{11}{12}} + 101.5\cdot e^{-0.09\cdot \frac{7}{6}} = 97.09926254\]

El tipo LIBOR publicado a tres meses es del 12\% con composición trimestral lo que significa que cada pago es de un 3\%.

\[B_{\text{fl}} = 103\cdot e^{-0.09\cdot \frac{1}{6}} = 101.4665298\]

Así, para la parte que paga variable, el valor del swap el valor del mismo es:
\[V = B_{\text{fix}} - B_{\text{fl}} = -4367267.24 \text{ euros }\]

\end{problem}

\begin{problem}[4]
Una institución financiera acaba de vender 1000 opciones europeas de compra a seis meses sobre
un activo S. El valor actual del mismo es de 10 euros y el precio de ejercicio es de 11 euros. El tipo
de interés libre de riesgo para el periodo es del 4\% anual para composición continua.
\ppart ¿Cómo debería cubrirse esta institución financiera?
\ppart ¿Cuál sería la variación del precio del derivado si el precio del subyacente sube un 5\% a lo largo de la sesión, quedando invariantes los demás parámetros que inciden en su precio?
\ppart ¿Cuál sería el impacto en dicho precio si, con el subyacente en 10 euros, su volatilidad crece un 10\%?

\solution
\doneby{Pedro}

\spart

La empresa buscará cubrise respecto a los cambios en el precio de la acción para lo que recurrimos a las sensibilidades de la fórmula de la call Cox-Rubinstein. Recordemos que
\[δ = \partial_S = N(d_+)\]

A la hora de cubrirnos de posibles variaciones en el precio lo que queremos es encontrar una cartera que tenga $δ=0$. Para ello consideramos como elementos de nuestra cartera la call que hemos vendido y el activo $S$. Si por cada call vendida compramos $x$ acciones de $S$, la suma de sus deltas será:
\[δ - x\cdot 1 = 0 \implies x = δ\]
por lo que deberemos comprar $1000\cdot δ$ acciones del activo $S$ para cubrirnos.

\obs En la ecuación $δ-x=0$ nos hemos apoyado en el hecho de que la variación del precio del activo $S$ respecto a su precio es 1.

\spart

La delta nos da precisamente la variación del precio de la opción cuando varía el precio del subyacente. Por tanto:
\[ΔC = δΔS = N(d_+)ΔS \]
siendo $C$ el valor de la opción.

\spart

Atendiendo a la derivada respecto de la volatilidad tenemos:
\[ΔC = \upsilon Δσ = S_0\sqrt{T}N'(d_+)\cdot 0.1\]

con lo que queda claro que el precio subirá.

\end{problem}

\section{Recuperación Septiembre 2011}
\begin{problem}[1]
Explicar el principio de valoración por ausencia de oportunidad de arbitraje
\solution

\doneby{Pedro}

Un arbitraje es una acción que me permite obtener un beneficio sin arriesgar nada. Si suponemos que en nuestro modelo no hay arbitrajes, al descontar los posibles precios futuros (ponderados con la probabilidad de ocurrencia) de una opción debemos obtener el precio actual de dicha opción.

Del mismo modo, si tenemos una cartera que replica una opción, la cartera debe tener el mismo valor que la opción.
\end{problem}

\begin{problem}[2]
Calcula la curva cupón-cero (tiempo continuo) determinada por los siguientes bonos:

\begin{center}
\begin{tabular}{cccc}
\hline
\textbf{Principal} & \textbf{T} & \textbf{Cupón Anual} & \textbf{P} \\
\hline
100 & 0.25 & 0 & 97.50\\
100 & 0.50 & 0 & 94.90\\
100 & 1.00 & 0 & 90.00\\
100 & 1.50 & 8$^*$ & 96.00\\
100 & 2.00 & 12 & 101.6\\
\hline
\end{tabular}
\end{center}
(*) Este tipo es anual para composición semestral.
\solution

\doneby{Pedro}

Se denomina \concept{curva cupón cero} a la construida con los tipos de interés para diferentes plazos que cumplen la ecuación $FC_t=(1+i)^{t}$\footnote{O su equivalente en composición continua} que se corresponde con la cantidad a pagar (FC) por una unidad monetaria prestada hoy y devuelta en el momento t. Tanto el tipo de interés $i$ como el tiempo $t$ se expresan en años.

Así para cada bono de la tabla tenemos los siguientes rendimientos:
\begin{center}
\begin{tabular}{|c|c|}
\hline
\textbf{Tiempo discreto} & \textbf{Tiempo continuo} \\
\hline
$r^d_{0.25}=2.5/97.5=0.025 \text{ trimestral.}$ & $r_{0.25}=\ln(1+r^d_{0.25}) \text{ trimestral.}$\\
$r^d_{0.5}=5.1/94.9=0.054 \text{ semestral.}$ & $r_{0.5}=\ln(1+r^d_{0.5})=0.052 \text{ semestral.}$\\
$r^d_{1}=10/90.0=0.111 \text{ anual.}$ & $r_{1}=\ln(1+r^d_{0.1})=0.105 \text{ anual.}$\\
\hline
\end{tabular}
\end{center}

Ahora debemos manipular estos rendimientos continuos para expresarlos en forma anual. La relación entre dos rendimientos \textbf{equivalentes} $r_1$ y $r_2$ definidos en períodos $t_1$ y $t_2$ viene dada por la ecuación:
\[(1+r_1)^{1/t_1} = (1+r_2)^{1/t_2} \implies r_2=(1+r_1)^{t_2/t_1}-1\]

\obs La idea intuitiva de esta relación es que para comparar los dos rendimientos calculo cuánto me dan en tiempo $t_1\cdot t_2$ y espero que den los mismo si son \textbf{equivalentes}. Así el primero me dará $(1+r_1)^{t_2}$ mientras que el segundo dará $(1+r_2)^{t_1}$ puesto que elevamos a la cantidad de veces que se repite el periodo sobre el que están definidos.

En composición continua la relación es
\[e^{r_1t_2}=e^{r_2t_1} \implies r_2 = r_1\frac{t_2}{t_1}\]

Para este ejercicio tenemos
\[r_{0.25}=0.1013\text{ anual.}\]
\[r_{0.5}=0.1047\text{ anual.}\]
\[r_{1}=0.1054\text{ anual.}\]

Los dos bonos que quedan por estudiar son algo diferentes puesto que incluyen el pago de cupones. El cuarto bono tiene un cupón de valor 8 semestral, es decir que paga 8 al año pagando cada 6 meses por lo que paga 4 cada 6 meses. Conociendo los rendimientos anteriores tenemos:
\[96 = 4e^{-r_{0.5}\cdot 0.5}+4e^{-r_{1}\cdot 1} + 104e^{-r_{1.5}\cdot 1.5} \implies r_{1.5} = 0.1068\]

Para el último caso tenemos
\[101.6 = 12e^{-r_1}+112 e^{-r_2\cdot 2} \implies r_{2} = 0.1049\]
\end{problem}

\begin{problem}[3]
Consideramos un activo, de valor inicial $S_0 = 10$ cuya dinámica viene descrita por un árbol binomial
de cuatro periodos correspondientes a un horizonte temporal de seis meses. La volatilidad del
subyacente es del 25\% y el tipo libre de riesgo para cada periodo es del 0,6%

\ppart Calcular el coeficiente $a$ del árbol binomial
\ppart Calcular la probabilidad riesgo neutro asociada al modelo
\ppart Dar el valor de la opción lookback cuyo pago final es $T_{0.5}=S_4-\min\{S_i \mid 1 \leq i \leq 4\}$

\ppart ¿Qué pasaría si el activo pagase dividendos en el inicio del cuarto periodo?
\solution

\doneby{Pedro}

\spart

Si los cuatro períodos cubren un intervalo temporal de 6 meses, es claro que cada período tiene una duración de mes y medio, es decir $Δt=\frac{1.5}{12}=0.125$. El enunciado nos da el tipo libre de riesgo $r=0.6$ y gracias a la volatilidad podemos calcular $a=e^{σ\sqrt{Δt}}=1.09$, $b=\frac{1}{a}=0.92$.

\spart

La probabilidad riesgo neutro es aquella que nos garantiza que el proceso de precios descontados aplicado a un subyacente nos da el valor actual del mismo, es decir:
\[S_0 = \frac{q\cdot aS_0 + (1-q)bS_0}{e^{rΔt}}\implies e^{rΔt}=q\cdot a +b - b\cdot q \implies q = \frac{e^{rΔt}-b}{a-b} = 0.92\]

\spart

La opción descrita paga el valor final del subyacente $S$ menos el mínimo valor alcanzado. La tabla \ref{figure:trayectoriasEj3} recoge el análisis de las posibles trayectorias. Con estos datos obtenemos que a vencimiento la opción vale $V_T(X)=2.8$. Trayendo al presente este valor tenemos: $V_0(X) = V_T(X)e^{-rΔt}=2.07$.

\begin{figure}[hbpt]
\centering
\begin{tabular}{|c|c|c|c|}
\hline
\textbf{Trayectoria} & \textbf{Valor} & \textbf{Probabilidad} & \textbf{Valor ponderado}\\
\hline
aaaa & $a^4S_0-aS_0 = 3.32$ & $0.71$ & $2.36$ \\
aaab & $a^2S_0-aS_0 = 1.01$ & $0.06$ &  $0.06$ \\
aaba & $a^2S_0-aS_0= 1.01$ & $0.06$ & $0.06$  \\
aabb & $S_0 - S_0 = 0$ & $0.006$ & $0$ \\
abaa & $a^2S_0-S_0 = 1.93$ & $0.06$ & $0.12$ \\
abab & $S_0-S_0 = 0$ & $0.006$ & $0$ \\
abba & $S_0-b_S0 = 0.85$ & $0.006$ & $0.005$ \\
abbb & $b^2S_0-b^2S_0 = 0$ & $0.005$ & $0$ \\
baaa & $a^2S_0-bS_0= 2.77$ & $0.06$ & $0.17$ \\
baab & $S_0-bS_0 = 0.85$ & $0.006$ & $0.005$ \\
baba & $S_0-bS_0 = 0.85$ & $0.006$ & $0.005$ \\
babb & $b^2S_0-b^2S_0 = 0$ & $0.005$ & $0$ \\
bbaa & $S_0-b^2S_0 = 1.62$ & $0.006$ & $0.01$ \\
bbab & $b^2S_0-b^2S_0 = 0$ & $0.005$ & $0$ \\
bbba & $b^2S_0-b^3S_0 = 0.71$ & $0.005$ & $0.003$ \\
bbbb & $b^4S_0-b^4S_0 = 0$ & $0.00005$ & $0$ \\
\hline
\end{tabular}
\caption{Análisis de las trayectorias posibles para la opción del ejercicio 3}
\label{figure:trayectoriasEj3}
\end{figure}

\spart

Si el activo pagase dividendos al inicio del cuarto período el valor del subyacente bajaría. Puesto que el valor de la opción viene dado por $S_4-\min_{1\leq i\leq 4}\{S_i\}$, al reducirse el valor del subyacente bajaría el valor de la opción. Al bajar el valor a vencimiento de la opción también bajaría su precio actual.
\end{problem}

\begin{problem}[4]
Una compañía de aviación tendrá que comprar un millón de galones de keroseno dentro de tres
meses. La desviación típica de las variaciones de los precios del galón de keroseno para un periodo
de tres meses se ha estimado en 0,032. La compañía decide cubrirse con contratos de futuros sobre
gasóleo de calefacción. La desviación típica de los cambios en los precios de los futuros, para un
periodo de tres meses, se estima en 0,040 y la correlación lineal entre los cambios de ambos precios
es de 0,8.

\ppart Explicar el riesgo que supone elegir una tal cobertura
\ppart Calcular el ratio de cobertura
\ppart Cada contrato de futuros sobre keroseno es por 42000 galones, ¿cuál es el número de contratos que deberá usar la compañía? Exlicar si usará una posición larga o corta.
\solution

\doneby{Pedro}

\spart

La estrategia de cobertura nos hace no tener riesgos por variación del precio del subyacente.

\spart

El ratio de cobertura viene dado por
\[h^* = ρ \frac{σ_S}{σ_F} = 0.8 \cdot \frac{0.032}{0.04} = 0.64\]

\spart

El número de contratos que se deberán usar es
\[N^* = h^* \frac{N_A}{Q_F} = 0.64 \cdot \frac{1000000}{42000} = 15.23 \approx 15\]

El objetivo de la empresa es protegerse frente a posibles cambios del precio del queroseno, que deberá comprar dentro de tres meses. Para ello la empresa deberá
ponerse en largo (comprar) de los 15 contratos futuros a fin de asegurarse un precio determinado. En concreto, el precio al que se asegura realizar la compra es el precio del futuro por unidad.

\end{problem}

\begin{problem}[5]
Los swaps de divisas son instrumentos muy usados en los mercados internacionales

\ppart A las empresas $A$ y $B$ les han ofrecido los siguientes tipos anuales en euros y en dólares

\begin{center}
\begin{tabular}{ccc}
\hline
& Euro & Dólar \\
Empresa A & 3.5 \% & 6.6\% \\
Empresa B & 5 \% & 7\% \\
\hline
\end{tabular}
\end{center}

A la empresa A le interesa un préstamos en dólares, mientras que a la empresa B le interesa que
sea en euros. Diseñar un swap igualmente atractivo para ambas empresas con un intermediario
financiero que se queda con 10 puntos básicos a cambio de asumir todo el riesgo de tipo de
cambio.(1 punto)

\ppart Una institución financiera A ha entrado en un swap de tipos, con pagos semestrales y a cinco,
años con la empresa B. Bajo las condiciones del swap, A recibe un 5\% anual a cambio del
LIBOR a 6 meses sobre un principal de 100 millones de euros. La empresa deja de pagar
al final del tercer año cuando le tocaba pagar la sexta liquidación. El LIBOR a 6 meses
publicado medio año antes era del 6,6\% anual. La media del tipo fijo ofertado y demandado
intercambiada actualmente por un LIBOR a 6 meses es del 7\% ¿Cuál será el coste para la
institución financiera? (Todos los tipos se componen semestralmente). (1 punto)

\solution

\doneby{Pedro}

\spart

La figura \ref{figure:swapMayo11.5.a} representa el swap pedido.

\begin{figure}[hbpt]
\centering
\begin{tikzpicture}
  \node[rectangle, draw=black, thick, minimum width=1cm, minimum height=1cm] (C) at (0,3) {$C$};
  \node[rectangle, draw=black, thick, minimum width=1cm, minimum height=1cm] (A) at (-3,3) {$A$};
  \node[rectangle, draw=black, thick, minimum width=1cm, minimum height=1cm] (B) at (3,3) {$B$};
  \node (A1) at (-6,3) {};
  \node (B1) at (6,3) {};

  \draw[->] (A) -- (A1) node[midway, above]{\texteuro $4.5\%$};
  \draw[->] (B) -- (B1) node[midway, above]{$\$ 7\%$};

  \draw[->, thick] ($(A.north east)!0.25!(A.south east)$) -- ($(C.north west)!0.25!(C.south west)$) node[black, midway, above]{$\$ x$};
  \draw[->, thick] ($(C.north west)!0.75!(C.south west)$) -- ($(A.north east)!0.75!(A.south east)$) node[black, midway, below]{\texteuro $4.5 \%$};

  \draw[->, thick] ($(B.south west)!0.25!(B.north west)$) -- ($(C.south east)!0.25!(C.north east)$) node[black, midway, below]{\texteuro $y$};
  \draw[->, thick] ($(C.north east)!0.25!(C.south east)$) -- ($(B.north west)!0.25!(B.south west)$) node[black, midway, above]{$\$ 6 \%$};

\end{tikzpicture}
\caption{Swap ventajoso para ambas empresas}
\label{figure:swapMayo11.5.a}
\end{figure}

Para que ambas empresas obtengan el mismo beneficio porcentual y la entidad financiera obtenga sus 10 puntos básicos de beneficio debemos calcular $x$ e $y$ que resuelven el sistema:

\[\left\{ \begin{array}{l} 6.6 - x = 5-y \\ x + y -4.5 - 6 = 0.1\end{array}\right. \implies 6.6 +y -4.5 -6 =0.1+5 -y  \implies y = 4.5 \implies x = 6.1\]

\spart

La empresa $B$ deja de pagar cuando quedan aún 2 años de swap. La pérdida/beneficio que supone este cese del pago dependerá del valor del swap en ese momento. Por tanto lo único que debemos hacer es evaluar el swap restante en ese punto y comprobar el valor del mismo para la institución financiera. Puesto que $A$ pagaba LIBOR, tendremos:
\[V_{\text{swap}} = V_{\text{fix}} - V_{\text{fl}}\]

Como quedan sólo 5 pagos por realizar, siendo el primero de ellos justo ahora, tenemos:
\[V_{\text{fix}} = 2.5\cdot e^{-0.07\cdot 0} + 2.5\cdot e^{-0.07\cdot 0.5} + 2.5\cdot e^{-0.07} + 2.5\cdot e^{-0.07\cdot 1.5} + 102.5\cdot e^{-0.07\cdot 2} = 98.60\]
\[V_{\text{fl}} = 103.3 \cdot e^{-0.07 \cdot 0} = 103.3\]

Por tanto
\[V_{\text{swap}} = -4694971.47\text{\texteuro}\]
\end{problem}

\section{Final Mayo 2015}
\begin{problem}[1]
A las empresas $A$ y $B$ les han ofrecido los siguientes tipos anuales sobre un préstamo de 10 millones de euros a cinco años

\begin{center}
\begin{tabular}{ccc}
\hline
& \textbf{Tipo fijo} & \textbf{Tipo variable} \\
Empresa A & 2.5 \% & LIBOR\% \\
Empresa B & 3.5 \% & LIBOR\% \\
\hline
\end{tabular}
\end{center}

La empresa $A$ necesita un préstamo a tipo variable y la empresa $B$ lo necesita a tipo fijo.

\ppart Diseñar un swap, a cinco años y con pagos anuales, en el cual el banco que actúe como intermediario obtenga un beneficio del 10pb anual y que sea igualmente atractivo para ambas partes.
\ppart Han pasado tres años y ocho meses desde que se inició el swap anterior y B suspende pagos, siendo el tipo swap para los tres próximos años del 3\%. El tipo LIBOR a un año, publicado hace ocho meses era
del 3,2\% anual. Todos los tipos son compuestos anualmente. ¿Cuál es el valor actual del swap? ¿Cuál es el impacto de dicho default para A?. (1,5 punto).
\solution
\doneby{Pedro}

\spart

Puesto que la $A$ necesita variable y la empresa $B$ fijo pero a $A$ le tratan mejor al pedir fijo, lo que harán será pedir cada empresa el contrario del que quieren e intercambiar los préstamos. Así el swap será el mostrado en la figura \ref{figure:swapMayo151a}

\begin{figure}[hbpt]
\centering
\begin{tikzpicture}
  \node[rectangle, draw=black, thick, minimum width=1cm, minimum height=1cm] (A) at (-3,3) {$A$};
  \node[rectangle, draw=black, thick, minimum width=1cm, minimum height=1cm] (B) at (3,3) {$B$};
  \node (A1) at (-5,3) {};
  \node (B1) at (5,3) {};

  \draw[->, thick] ($(A.north east)!0.25!(A.south east)$) -- ($(B.north west)!0.25!(B.south west)$) node[black, midway, above]{$L$};
  \draw[->, thick] ($(B.north west)!0.75!(B.south west)$) -- ($(A.north east)!0.75!(A.south east)$) node[black, midway, below]{$3\%$};
  \draw[->] (A) -- (A1) node[midway, above]{$2.5\%$};
  \draw[->] (B) -- (B1) node[midway, above]{$L$};

\end{tikzpicture}
\caption{Swap ventajoso para ambas empresas}
\label{figure:swapMayo151a}
\end{figure}

Así, $A$ está pagando LIBOR - $5\%$ y $B$ está pagando $3\%$, con lo que ambos mejoran sus ofertas iniciales en un $5\%$.

\spart

Si han pasado 3 años y 8 meses, sólo quedan 2 pagos por realizar: el correspondiente al cuarto año y el último. En este instante el valor del swap para $A$, que paga variable, viene dado por:
\[V_{\text{swap}} = V_{\text{fix}} - V_{\text{fl}} \]
siendo
\[V_{\text{fix}} = 3\cdot e^{-0.03/3} + 103 \cdot e^{-0.03\cdot 4/3} = 101.93\]
\[V_{\text{fl}} = 103.2 \cdot e^{-0.03/3} = 102.17\]
Por tanto,
\[V_{\text{swap}} = -0.241681109 \text{ millones de \texteuro}\]

La empresa $A$ sale ganando ya que el swap tenía valor negativo en ese punto.

\end{problem}

\begin{problem}[2]
Consideramos una opción que pagará a vencimiento (dentro de cuatro meses) $(S_4-A_4)^+$ siendo $A_4=\frac{1}{4}\sum_{i=1}^4S_i$, la media de los valores del subyacente a finales de cada mes, hasta vencimiento.

\ppart Calcular el valor de esta opción. Para ello supondremos que la dinámica para los doce próximos meses viene descrita por un modelo binomial de cuatro periodos. En cada periodo el activo puede subir o bajar un 3\%. El tipo libre de riesgo es el 2.5\% anual para composición continua. El activo no paga dividendos y su valor actual es 10.

\ppart ¿Cuál sería el impacto en este valor si el subyacente pagase dividendos en el tercer mes?

\solution

\doneby{Pedro}

\spart

Lo primero que debemos hacer es calcular $a$, $b$ y $q$ con los datos del enunciado. Así tenemos
\[a=1.03, \;\; b = 0.97, \;\; q = \frac{e^{rΔt}-b}{a-b} = 0.60 \]

La tabla \ref{tabla:Mayo2015_2} contiene los posibles caminos que puede llevar el activo, cada uno con su valor y su probabilidad asociados.

\begin{figure}[hbpt]
\centering
\begin{tabular}{|c|c|c|c|}
\hline
\textbf{Path} & \textbf{Valor} & \textbf{Probabilidad} & \textbf{Ponderado}\\
\hline
aaaa & $(a^4S_0-\frac{S_0}{4}(a+a^2+a^3+a^4))^+ = 0.48$ & $q^4 = 0.13$ & 0.06 \\
aaab & $(a^2S_0-\frac{S_0}{4}(a+a^2+a^3+a^2))^+ = 0$ & $q^3(1-q)=0.09$  & 0 \\
aaba & $(a^2S_0-\frac{S_0}{4}(a+a^2+a+a^2))^+ = 0.15$ & $q^3(1-q)=0.09$  & 0.01\\
aabb & $(S_0-\frac{S_0}{4}(a+a^2+a+1))^+ = 0$ & $q^2(1-q)^2=0.06$  & 0 \\
abaa & $(a^2S_0-\frac{S_0}{4}(a+1+a+a^2))^+ = 0.31$ & $q^3(1-q)=0.09$  & 0.03\\
abab & $(S_0-\frac{S_0}{4}(a+1+a+1))^+ = 0$ & $q^2(1-q)^2=0.06$  & 0 \\
abba & $(S_0-\frac{S_0}{4}(a+1+b+1))^+ = 0$ & $q^2(1-q)^2=0.06$  & 0 \\
abbb & $(b^2S_0-\frac{S_0}{4}(a+1+b+b^2))^+ = 0$ & $q(1-q)^3=0.04$  & 0 \\
baaa & $(a^2S_0-\frac{S_0}{4}(b+1+a+a^2))^+ = 0.45$ & $q^3(1-q)=0.09$  & 0.04\\
baab & $(S_0-\frac{S_0}{4}(b+1+a+1))^+ = 0$ & $q^2(1-q)^2=0.06$  & 0 \\
baba & $(S_0-\frac{S_0}{4}(b+1+b+1))^+ = 0.15$ & $q^2(1-q)^2=0.06$  & 0.009\\
babb & $(b^2S_0-\frac{S_0}{4}(b+1+b+b^2))^+ = 0$ & $q(1-q)^3=0.04$  & 0 \\
bbaa & $(S_0-\frac{S_0}{4}(b+b^2+b+1))^+ = 0.30$ & $q^2(1-q)^2=0.06$  & 0 \\
bbab & $(b^2S_0-\frac{S_0}{4}(b+b^2+b+b^2))^+ = 0$ & $q(1-q)^3=0.04$  & 0 \\
bbba & $(b^2S_0-\frac{S_0}{4}(b+b^2+b^3+b^2))^+ = 0$ & $q(1-q)^3=0.04$  & 0 \\
bbbb & $(b^4S_0-\frac{S_0}{4}(b+b^2+b^3+b^4))^+ = 0$ & $(1-q)^4=0.02$  & 0 \\
\hline
\end{tabular}
\caption{Análisis de las trayectorias posibles para la opción del ejercicio 3}
\label{tabla:Mayo2015_2}
\end{figure}

Para calcular el valor de la opción debemos descontar los valores ponderados en $T=4$ obteniendo:
\[V(X) = \frac{0.06+0.01+0.03+0.04+0.09}{e^{0.025}\cdot 1} = 0.14\]

\spart

Si pagase dividendos en el tercer mes, el valor de la acción en ese punto decaería lo que implicaría que la media fuese más baja y pero también lo fuese el valor final de la acción. Sin embargo, si la bajada es de un $x\%$, el valor final de la acción caería un $x\%$ y lo mismo ocurriría con la media ya que todos los sumandos sufren el mismo descuento. Por tanto la opción no cambia de valor.
\end{problem}


\begin{problem}[3]
El activo $A$ tiene un precio actual de 11.68 euros y una volatilidad del 25\%. El tipo de interés libre de riesgo (composición continua) es del 2.5\%.

\ppart ¿Cuál es el precio (Black-Scholes) de una opción que da derecho, dentro de tres meses, a comprar una unidad del subyacente a su precio actual? (Se calculará $N(x)$ como un desarrollo de Taylor)

\ppart ¿Qué impacto tendrá en el precio de este derivado una variación positiva del 1\% en el precio del subyacente, permaneciendo todos los demás parámetros inalterados?

\ppart Calcular las sensibilidades del precio de la opción a la volatilidad y al tipo libre de riesgo.

\ppart ¿Cómo variaría el precio de la opción si la volatilidad sube al 30\%, permaneciendo el resto de parámetros inalterados?

\ppart ¿Y si el tipo de interés baja al 150 puntos básicos, permaneciendo todos los demás parámetros inalterados?

\solution

\doneby{Pedro}

\spart

Como ya se ha explicado en el ejercicio \ref{ej:NormalTaylor} tenemos:
\[N(x) = \frac{1}{\sqrt{2π}}\left(x-\frac{x^3}{3 \cdot 2 \cdot 1!} +\frac{x^5}{5 \cdot 4 \cdot 2!} - \frac{x^7}{7\cdot 8 \cdot 3!}+...\right)\]

Con los datos del enunciado tenemos
\[d_{\pm} = \frac{1}{0.25\sqrt{0.25}}\left(\ln \frac{11.68}{11.68e^{-0.025\cdot 0.25}}\pm\frac{1}{2}0.25^2 \cdot 0.25\right)=8 \left(0.0063 \pm 0.0078\right) \implies\]
\[\implies d_+ = 0.11, \;\; d_-=-0.012 \]

Ahora evaluamos la normal:
\[N(d_+)=0.54, \;\; N(d_-)=0.495\]
Así, empleando la fórmula de Bacl-Scholes tenemos:
\[C_0(11.68,11.68,0.25,0.025,0.25)=0.56\]

\spart

Para conocer la respuesta nos apoyamos en las sensibilidades de la fórmula de Black-Scholes. Para este caso concreto nos interesa la delta.
\[ΔC_{BS} \approx ΔS\cdot δ = ΔS \cdot N(d_+) = 0.1168 \cdot 0.54 = 0.063\]
Es decir, al aumentar el precio del activo un $1\%$ la opción aumenta un $6.3\%$.
\end{problem}

\begin{problem}[4]
Sea una put americana, At-The-Money, a 2 meses sobre un subyacente que no paga dividendos. El valor del subyacente es 50. Para valorar dicha opción se usa un árbol binomial con periodicidad mensual. En cada periodo, el subyacente puede subir o bajar un 4\%. El tipo libre de riesgo (composición continua) es del 1\%. Calcular el valor de la put.
\solution

\doneby{Pedro}

El enunciado nos dice que $a=1.04$ y que $b=0.96$. Sabiendo que $r=0.01$ y que $Δt=0.083$ podemos calcular $q=\frac{e^{rΔt}-b}{a-b}=0.51$

Para conocer el valor de la put americana en $t=0$ debemos estudiar cada nodo para ver si nos interesaría ejercer o no. Así, empezando por el penúltimo período, obtenemos el valor de la put como el máximo entre el beneficio obtenido ejerciendo en este instante y el beneficio potencial de no ejercer.

\begin{itemize}
\item \textbf{Nodo $aS_0$}
En este momento la acción vale 52. Evidentemente en este instante no nos interesa ejercer, puesto que ejercer la put nos permitiría venderla por 50, que es menos que su valor actual.

Por tanto, el valor de la put en este nodo viene dado por
\[V = \left(0 + 0.08 \cdot (1-q)\right)\cdot e^{-rΔt} = 0.039\]

\item \textbf{Nodo $bS_0$}
En este momento podemos ejercer obteniendo un valor en este instante de la put igual a $V=2$. Si no ejercemos, el valor de esta será
\[V = \left(0.08 \cdot q + 3.92 \cdot (1-q) \right)\cdot e^{-rΔt} = 1.96\]

Por tanto en esta situación nos interesa ejercer.

\item \textbf{Nodo $S_0$} (Instante inicial)
Es obvio que en este instante no nos interesa ejercer dado que venderíamos al mismo valor que tiene el subyacente en el mercado con lo que no obtendríamos beneficio alguno.

El valor en este instante es:
\[V = \left( 0.039 \cdot q + 2 \cdot (1-q)\right)\cdot e^{-rΔt} = 0.998\]
\end{itemize}

\end{problem}

\section{Recuperación Junio 2015}

\begin{problem}[1]
Consideramos una opción que pagará a vencimiento (dentro de tres meses) $(S_4-A_3)^+$ siendo $A_3=\frac{1}{3}\sum_{i=1}^3S_i$, la media de los valores del subyacente a finales de cada mes, hasta vencimiento.

\ppart Calcular el valor de esta opción. Para ello supondremos que la dinámica para los tres próximos meses viene descrita por un modelo binomial de tres periodos. En cada periodo el activo puede subir o bajar un 3\%. El tipo libre de riesgo es el 5\% anual para composición continua. El activo no paga dividendos y su valor actual es 10.

\ppart ¿Cuál sería el impacto en este valor si el subyacente pagase dividendos en el segundo mes?

\solution




\end{problem}

\begin{problem}[2]
Consideremos el subyacente $S_0=50$ cuya dinámica para los próximos seis meses viene descrita por un modelo binomial de dos periodos. En cada periodo el activo puede subir o bajar un 20\%. El tipo libre de riesgo es el 12\% anual para composición continua. El activo no paga dividendos.

\ppart Calcular la probabilidad riesgo neutro de pasar al estado alto

\ppart Calcular el valor de la put americana de precio de ejercicio 52

\ppart Si el activo pagase dividendos, ¿el precio de la put sería mayor o menos?

\ppart Consideremos ahora un derivado sobre dicho subyacente que paga, al final de los seis meses, $M_2(ω)-m_2(ω)$ donde
\[M_2(ω)=\max(S_0(ω),S_1(ω),S_2(ω)), \;\;\; m_2(ω)=\min(S_0(ω),S_1(ω),S_2(ω))\]
Calcular la caretara de cobertura en cada nodo.

\solution

\end{problem}

\begin{problem}[3]
Una institución financiera entró en su día en un swap de tipos en el que cordó pagar un 5\% anual y recibir LIBOR a tres meses sobre un principal de 100 millones de euros con pagos intercambiados cada tres meses. Al swap le quedan 14 meses de vida. El tipo swap para intercambio por el LIBOR a tres meses en estos momentos es del 7\% anual para todos los vencimientos. El tipo LIBOR a tres meses publicado hace dos meses era del 8\%. Todos los tipos son compuestos trimestralmente. ¿Cuál es el valor actual del swap?

\solution

\end{problem}

\begin{problem}[4]
Una institución financiera acaba de vender 1000 opciones europeas de compra a seis meses sobre
un activo S. El valor actual del mismo es de 10 euros y el precio de ejercicio es de 11 euros. El tipo
de interés libre de riesgo para el periodo es del 4\% anual para composición continua.
\ppart ¿Cómo debería cubrirse esta institución financiera?
\ppart ¿Cuál sería la variación del precio del derivado si el precio del subyacente sube un 5\% a lo largo de la sesión, quedando invariantes los demás parámetros que inciden en su precio?
\ppart ¿Cuál sería el impacto en dicho precio si, con el subyacente en 10 euros, su volatilidad crece un 10\%?
\ppart Un incremento de la volatilidad, ¿resultaría positivo o negativo para el titular de la cartera?
\solution

\end{problem}

\section{Primer parcial 2016}
\begin{problem}[1]
El subyacente vale hoy 10 y su dinámica está descrita por un árbol binomial recombinante de tres períodos a un horizonte temporal de seis meses. La volatilidad del subyacente es del 30\% y el tipo libre de riesgo para el periodo es del 2.5\% (composición continua)
\ppart Calibrar el árbol binomial para la valoración (calcular a, b y q).
\ppart Calcular el precio de la \textbf{call lookback} cuyo flujo final es $S_3-\min\{S_i \mid i \leq 3\}$.
\ppart ¿Cuál es la cartera que replica el derivado cuando estamos en el segundo período y el subyacente vale 10?
\solution
\doneby{Pedro}

\spart
Aplicando las fórmulas tenemos:
\[Δt = \frac{0.5}{3} = \frac{1}{6},\;\;\; a=e^{σ\sqrt{Δt}}=1.13, \;\;\; b=\frac{1}{a}=0.88, \;\;\; q = \frac{e^{rΔt}-b}{a-b}=0.49\]

\spart

La siguiente tabla recoge los caminos estudiados:
\begin{center}
\begin{tabular}{|c|c|c|c|}
\hline
\textbf{Path} & \textbf{Probabilidad} & \textbf{Valor} & \textbf{Valor ponderado}\\
\hline
aaa & 0.11 & $S_0 \cdot \left( a^3-1\right) = 4.44 $ & 0.49\\
aab & 0.12 & $S_0 \cdot \left( a-1  \right) = 1.3 $ & 0.16\\
aba & 0.12 & $S_0 \cdot \left( a-1 \right) = 1.3 $ & 0.16\\
abb & 0.13 & 0 & 0 \\
baa & 0.12 & $S_0 \cdot \left( a-b \right) = 2.46 $ & 0.3\\
bab & 0.13 & 0 & 0 \\
bba & 0.13 & $S_0 \cdot \left( b-b^2 \right) = 1.02 $ & 0.13\\
bbb & 0.14 & 0 & 0 \\
\hline
\end{tabular}
\end{center}

Así, el valor en $T=0.5$ de la opción es
\[V_{0.5} = 0.49+0.16+0.16+0.3+0.13 = 1.24\]
Trayendo este valor al presente obtenemos el valor actual de la opción:
\[V_0 = V_{0.5}\cdot e^{-rΔt} = 1.24 \cdot e^{-0.025\cdot 0.5} = 1.22\]

\spart

Cuando estamos en el segundo período y el subyacente vale $10$ es que hemos llevado el camino ``ab'' o el camino ``ba''. Puesto que el valor de la opción depende
del mínimo alcanzado por el activo durante su desarrollo, la cartera de cobertura en este punto será distinta según el camino seguido.

\begin{itemize}
\item \textbf{Para ``ab''}
Siendo $\varphi_0$ el número de acciones del activo sin riesgo y $\varphi_1$ el número de acciones del activo $S$ tendremos:
\[\left\{ \begin{array}{l}
\varphi_0 \cdot e^{rΔt} + \varphi_1 \cdot 11.3 = 1.3\\
\varphi_0 \cdot e^{rΔt} + \varphi_1 \cdot 8.85 = 0
\end{array}\right. \implies \varphi_1 = 0.53 \implies \varphi_0 = -4.66\]

\item \textbf{Para ``ba''}
\[\left\{ \begin{array}{l}
\varphi_0 \cdot e^{rΔt} + \varphi_1 \cdot 11.3 = 2.46\\
\varphi_0 \cdot e^{rΔt} + \varphi_1 \cdot 8.85 = 0
\end{array}\right. \implies \varphi_1 = 1 \implies \varphi_0 = -8.83\]
\end{itemize}


\end{problem}

\begin{problem}[2]
Consideramos un subyacente $S_0=50$ cuya dinámica para los próximos seis meses viene descrita por un modelo binomial de dos períodos. En cada periodo el activo puede subir o bajar un 20\%. El tipo libre de riesgo es del 10\% anual para composición continua. El activo no paga dividendos.
\ppart Calcular la probabilidad riesgo neutro de pasar al estado alto
\ppart Calcular el valor de la put americana de precio de ejercicio 52.

\solution
\doneby{Pedro}

\spart

Con los datos del enunciado vemos que $a=1.2$, $b=0.8$, $Δt=0.25$ y $r=0.1$. Para calcular la probabilidad riesgo neutro nos apoyamos en el hecho de que no existen arbitrajes en nuestro modelo y, por tanto, el valor descontado del activo debe coincidir con su valor actual, es decir:
\[50 = \frac{q\cdot 50 \cdot 1.2 + (1-q)\cdot 50 \cdot 0.8}{e^{0.1\cdot 0.25}} \implies q=0.56\]

\spart

La put americana nos permite ejercer en cualquier instante. Por tanto, en cada nodo debemos comprobar, para poder conocer el valor de la opción en ese instante, si nos interesa o no ejercer.

\begin{itemize}
\item \textbf{Nodo $aS_0$}
La acción en este instante vale $60$. Dado que el precio de ejercicio es 52, está claro que no nos interesa ejercer en este instante. Por tanto, el valor de la opción es:
\[V = \left((K-a^2S_0)^+\cdot q + (K-abS_0)^+\cdot (1-q)\right)\cdot e^{-rΔt} = 1.7\]

\item \textbf{Nodo $bS_0$}
En este instante la acción vale $40$ por lo que podría intersarnos vender a precio $52$ obteniendo un beneficio de $12$. En caso de no vender, el valor de la opción sería:
\[V = \left((K-abS_0)^+\cdot q + (K-bbS_0)^+\cdot (1-q)\right)\cdot e^{-rΔt} = 10.72\]

En este caso nos interesa ejercer.

\item \textbf{Nodo $S_0$}
Nada mas empezar podríamos ejercer la opción obteniendo un beneficio de $2$. En cambio, en caso de no vender tendríamos un valor:
\[V = \left(1.7\cdot q + 12 \cdot (1-q)\right)e^{-rΔt} = 6.047\]
\end{itemize}

\end{problem}