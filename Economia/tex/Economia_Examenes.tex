% -*- root: ../Economia.tex -*-
\section{Final Junio 2011}
\begin{problem}[1]
Consideramos una opción que pagará a vencimiento (dentro de un año) $(S_4-A_4)^+$ siendo $A_4=\frac{1}{4}\sum_{i=1}^4S_i$, la media de los valores del subyacente dentro de tres, seis, nueve y doce meses.

\ppart Calcular el valor de esta opción. Para ello supondremos que la dinámica para los doce próximos meses viene descrita por un modelo binomial de cuatro periodos. En cada periodo el activo puede subir o bajar un 3\%. El tipo libre de riesgo es el 5\% anual para composición continua. El activo no paga dividendos y su valor acutal es 10.

\ppart ¿Cuál sería el impacto en este valor si el subyacente pagase dividendos en el noveno mes?

\solution

\end{problem}

\begin{problem}[2]
Consideremos el subyacente $S_0=50$ cuya dinámica para los próximos seis meses viene descrita por un modelo binomial de dos periodos. En cada periodo el activo puede subir o bajar un 20\%. El tipo libre de riesgo es el 12\% anual para composición continua. El activo no paga dividendos.

\ppart Calcular la probabilidad riesgo neutro de pasar al estdo alto

\ppart Calcular el valor de la put americana de precio de ejeercicio 52

\ppart Si el activo pagase dividendos, ¿el precio de la put sería mayor o menos?

\ppart Consideremos ahora un derivad sobre dicho subyacente que paga, al final de los seis meses, $M_2(ω)-m_2(ω)$ donde
\[M_2(ω)=\max(S_0(ω),S_1(ω),S_2(ω)), \;\;\; m_2(ω)=\min(S_0(ω),S_1(ω),S_2(ω))\]
Calcular la caretara de cobertura en cada nodo.

\solution

\end{problem}

\begin{problem}[3]
Una instituçión financiera entró en su día en un swap de tipos en el que cordó pagar un 6\% anual y recibir LIBOR a tres meses sobre un principal de 100 millones de euros con pagos intercambiados cada tres meses. Al swap le quedan 14 meses de vida. El tipo swap para intercambio por el LIBOR a tres meses en estos momentos es del 9\% anual par atodos los vencimientos. El tipo LIBOR a tres meses publicado hace dos meses era del 12\%. Todos los tipos son compuestos trimestralmente. ¿Cuál es el valor actual del swap?
\solution

\end{problem}

\begin{problem}[4]
Una institución financiera acaba de vender 1000 opciones europeas de compra a seis meses sobre
un activo S. El valor actual del mismo es de 10 euros y el precio de ejercicio es de 11 euros. El tipo
de interés libre de riesgo para el periodo es del 4\% anual para composición continua.
\ppart ¿cómo debería cubrirse esta institución financiera?
\ppart ¿Cuál sería la variación del precio del derivado si el precio del subyacente sube un 5\% a lo largo de la sesión, quedando invariantes los demás parámetros que inciden en su precio?
\ppart ¿Cuál sería el impacto en dicho precio si, con el subyacente en 10 euros, su volatilidad crece un 10\%?

\solution
\end{problem}