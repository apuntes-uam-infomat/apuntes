\documentclass{apuntes}

\title{Teoría de Galois}
\author{Víctor de Juan}
\date{14/15 C1}

% Paquetes adicionales

% --------------------

\begin{document}
\pagestyle{plain}
\maketitle

\tableofcontents
\newpage

Notación: si $F$ y $E$ son cuerpos tales que $F⊆E$ diremos que $E$ es una extensión de $F$. Se escribe $F⊆E$ (obviamente) o también $E/F$.

\begin{defn}[Elemento\IS algebraico]
Sea $E/F$ una extensión de cuerpos (conmutativos). Se dice que un elemento $α∈E$ es algebraico sobre $F$ si existe un polinomio $h(t) ∈ F[t]$ tal que $h(α) = 0$.
\end{defn}

Por ejemplo, $ℚ⊆ℝ$ o $ℚ⊆ℂ$ son extensiones de $ℚ$. $\sqrt{2}$ es algebraico sobre $ℚ$ porque si tomo $h(t) = t^2 - 2$, entonces $h(\sqrt{2}) = 0$.

Es muy difícil encontrar números (reales o complejos) que no sean algebraicos (o trascendentes). Sin embargo hay muchísimos (es un conjunto no numerable).

\begin{prop} SI $α∈E$ es algebraico sobre $F$ existe un único polinomio mónico de grado mínimo $p(t) ∈ F[t]$ tal que $p(α) = 0$. 

A $p(t)$ se le llama el polinomio mínimo o irreducible de $α$.

Además, $p(t)$ es un polinomio irreducible.
\end{prop}

\begin{proof}
Sea $h(t) ∈ F[t]$ mónico tal que $h(α) = 0$. Dividiendo $h(t)$ entre $p(t)$ podemos escribir \[ h(t) = q(t) · p(t) + r(t) \], donde $\deg(r(t)) < \deg(p(t))$. Si evalúo en $α$, tenemos que \[ h(α) = q(α)p(α) + r(α) \], pero sabemos que $h(α) = p(α) = 0$, luego obligatoriamente tiene que ser $r(α) = 0$.

De ahí se deduce que $r(t)$ es cero, porque habíamos dicho que $p(t)$ era el polinomio de grado mínimo que anulaba a α. La única forma de que $r(α)$ sea 0 y que además tenga menor grado que $p(t)$ es que $r(t) = 0$.

Por otra parte, tenemos que $h(t) = q(t)p(t)$, y si suponemos que $h(t)$ es de grado mínimo entre los que anulan a α, entonces $h(t) = p(t)$ luego $p$ es único.

Además, es irreducible porque si existiese $p(t) = p_1(t) p_2(t)$, tendríamos que $0 = p(α) = p_1(α) p_2(α)$. Si suponemos $p_1(α) = 0$, entonces $\deg p(t) = \deg p_1(t)$.
\end{proof}

\begin{corol} El ideal \[ I = \{ h(t) ∈ F[t] \tq h(α) = 0 \} ⊆ F[t] \] coincide con $(p(t))$, es decir, con el ideal generado por los múltiplos de $p(t)$.
\end{corol}

Considero el homomorfismo $\appl{φ}{F[t]}{E}$, con \[ \img φ = K[α] = \left\lbrace\sum a_i α^i \tq a_i ∈ F \right\} ⊆ E \]

$k(α)$ es el subcuerpo de $E$ más pequeño que contiene a $F$ y a $α∈E$. De hecho \[ k(α) = \left\{ \frac{\sum a_i α^i}{\sum b_i α^i} \tq a_i, b_i ∈ F \right\} ⊆ E \]

\begin{theorem} Si $α$ es algebraico, entonces $k(α) = k[α]$, es decir, $k[α]$ es un cuerpo.
\end{theorem}

\begin{proof} Consideramos el homomorfismo $F[t] \mapsto^φ E$. Según el primer teorema de isomorfía, $F[t] / \ncl φ $ es isomorfo a $\img φ = k[α]$. 

Pero $\ncl φ = I = (p(t))$. 
\end{proof}

Por ejemplo, $ℚ[\sqrt{2}]$ es un cuerpo, de hecho $ℚ[\sqrt{2}] = \frac{ℚ[t]}{(t^2 -2)}$. Pero, ¿cuál es el inverso de $1 + \sqrt{2}$? Es $\frac{1}{1+\sqrt{2}}$ que está en $ℚ[\sqrt{2}]$ porque lo podemos expresar como $\sqrt{2} - 1$.

En la extensión $\fd_3(t) / \fd_3 (t^3)$, con $\fd_3 = ℤ/3ℤ$, tenemos que $\fd_3(t^3) ⊆ \fd_3(t)$, y que $t$ es algebraico sobre $\fd_3(t^3)$, el polinomio irreducible es $p(x) = x^3 - t^3$.

\end{document}
