\documentclass[nochap]{apuntes}
%
\author{Guillermo Julián}
\date{C1 - 2011/2012}
\title{Conjuntos y Números}
%


\begin{document}
\pagestyle{plain}
\maketitle
\newpage

\section{Lógica}
Lógica básica.
\begin{defn}
Dos FBF A y B son lógicamente equivalentes si para cualquier atribución veritativa para las variables que contienen toman el mismo valor lógico. Entonces $A\equiv B$ sea cual sea la interpretación.

\textbf{Propiedad transitiva:} $A\equiv B \y B\equiv C \implies A\equiv C$.
\end{defn}

\subsection{Operadores lógicos}
Creo que no hace falta que diga los básicos, ¿no? Negación, y, o, implica, doble implicación. 

\subsubsection*{Conector Sheffer} $A|B\equiv \neg (A \y B)$.

\subsubsection{Interdefinibilidad} Podemos definir con todos los operadores usando dos como base (negación y otro más).

\subsubsection{Implicaciones}
\textbf{Directa} $p\implies q$

\textbf{Recíproca}  $q \implies p$

\textbf{Contraria} $\neg p \implies \neg q$

\textbf{Contrarrecíproca}  $\neg q \implies \neg p$

\indent

Directa $\equiv$ contrarrecíproca, contraria $\equiv$ recíproca.

\subsubsection{Cuantificadores}

Tomamos $U$ como el universo. Definimos tres cuantificadores

\textbf{Universal}: Para todo elemento de $U$: $\forall x \in U$.

\textbf{Existencial}: Existe un elemento en $U$: $\exists x \in U$.

\textbf{Existencia y unicidad}: Existe un único elemento en $U$: $\uexists x \in U$.

\begin{remark}
El orden de los cuantificadores es importante. No es lo mismo $\exists a\: \forall b \tq C$ (Existe un $a$ para el que todos los $b$ cumplen $C$) que $\forall b\: \exists a \tq C$ (Para cada $b$ existe un $a$ que cumple $C$).
\end{remark}
\subsection{Métodos de demostración}

Hay dos tipos: Directo e indirecto. La reducción al absurdo es un método indirecto, por ejemplo.

\subsubsection{Inducción}

Sea $P_n$ una propiedad asociada a $n \in \nat$. Para demostrar $P_n$ seguimos el siguiente método:
\begin{enumerate}
\item Compramos que se cumple para $n=1$ (o el número de inicio que sea).
\item Suponemos que es cierto para $n=k$. Demostramos que también lo es para $n=k+1$.
\end{enumerate}

Si ambos pasos son ciertos, entonces $P_n$ es cierto $\forall n \in \nat$.


\section{Conjuntos}

Un conjunto es una colección de elementos. Hay tres formas de definirlo:

\begin{enumerate}
\item Enumerar los elementos: $A=\{a,b,c,...\}$.
\item Operaciones con conjuntos: $A=B\cap C$.
\item A través de una fórmula: $A=\{x\in B \tq P(x)\}$
\end{enumerate}

\begin{defn}[Par ordenado]
$(a, b)=\{\{a\},\{a,b\}\}$. 
\end{defn}
\begin{lemma}
$(a,b)=(c,d) \dimplies (a=c) \y (b=d)$.
\end{lemma}

\begin{defn}[Producto cartesiano o directo]
Sean $X$ e $Y$ dos conjuntos. Entonces $X\x Y=\{(a,b\}\tq a\in X\; b\in B\}$.

Si $X$ e $Y$ son finitos, $\#(X\x Y) =\#X \cdot \#Y$
\end{defn}

\section{Relaciones}

\begin{defn}[Relación]
Sean $A$ y $B$ dos conjuntos. Una relación entre A y B es un subconjunto de $A\x B$.

\noindent Una relación binaria entre A y B es una terna $(A, B, \rel)$ donde $\rel \subset A\x B$. \\

\noindent \textbf{Notación}: Si dos elementos a y b están relacionados por $\rel$, se escribe $a\rel b$.
\end{defn}

\begin{defn}[Relación inversa]
Sea $\rel$ una relación entre $X$ e $Y$. La relación inversa es $\rel^{-1}=\{(y,x) \tq (x,y) \in \rel\}$.
\end{defn}

\subsection{Funciones}

\begin{defn}[Función]
Una relación $f\subset A\x B$ es función si $\forall x \in A \; \uexists y \in B $. Se denota como $\appl{f}{A}{B}$.
\end{defn}

\begin{defn} Una función \stdf es inyectiva si elementos distintos tienen imágenes distintas: $f(x)=f(y) \implies x=y$.
\end{defn}

\begin{defn}
Una función \stdf es sobreyectiva si $f(X)=Y$, es decir: $\forall y \in Y \exists x \in X \tq f(x)=y$.\end{defn}

\begin{defn}
Una funcion es biyectiva si es inyectiva y sobreyectiva.\end{defn}

\begin{defn}[Función inversa] Sea \stdf una función. La relación inversa es $\appl{f^{-1}}{Y}{X}$, y si es función, se dice que $f^{-1}$ es la inversa de $f$.\end{defn}

\begin{prop} $f$ es invertible si y sólo si es inyectiva.

\begin{proof}
Para que \stdf sea invertible, $\appl{f^{-1}}{Y}{X}$ tiene que ser función. Es decir, que $\forall y\in Y \uexists x \in X$. Comprobamos primero la existencia de imagen para cualquier elemento de $Y$. Si $f$ es sobreyectiva, entonces tenemos que $Y=f(X)$. Por lo tanto, cualquier elemento de $Y$ tiene una imagen en $X$. Si no fuese sobreyectiva existiría algún elemento en $Y$ que no fuese imagen de un elemento de $X$, así que $f^{-1}$ no sería función.
 
Pasamos ahora a demostrar la unicidad de la imagen para cualquier elemento de $Y$. Si $f$ es inyectiva, tenemos que $\forall x,x' \in X \: f(x)=f(x') \dimplies x=x'$. Cada elemento de $X$ esta relacionado con un sólo un elemento de $Y$, por lo que cada elemento de $Y$ tiene una sola imagen. Si no fuera inyectiva, algún elemento de $Y$ tendría dos imágenes en $X$ y la relación inversa no sería función.
\end{proof}
\end{prop}

\begin{defn}[Composición] Sean $\appl{f}{A}{B}$ y $\appl{g}{C}{D}$, y $f(A)\subset C$. Entonces se define la composición $f$ compuesto con $g$ como $\appl{g\circ f}{A}{D}$, tal que $(g \circ f)(x)=g(f(x)), x \in A$.

La composición de funciones cumple la propiedad asociativa ($(f\circ g)\circ h=f\circ (g\circ h)$).

Si $f$ y $g$ son sobreyectivas, entonces $g\circ f$ también lo es.
\end{defn}

\subsection{Relaciones de orden}

Una relación de orden $\rel \subset X^2$ es una relación que cumple las siguientes propiedades para dos elementos cualesquiera $a,b \in X$.

\begin{description}
\item[Reflexiva] $a\rel a$.
\item[Antisimétrica] $a\rel b \y b\rel a \implies b=a$.
\item[Transitiva] $a\rel b \y b\rel c \implies a\rel c$.
\end{description}

\begin{defn}[Relación lineal] Una relación de orden es lineal o total si $\forall \; a ,b \in X (a\rel b) \Or (b\rel a)$.\end{defn}

\begin{defn}Se dice que $X$ está bien ordenado si cualquier subconjunto no vacío de $X$ tiene un mínimo. Cualquier buen orden es lineal.\end{defn}

\begin{prop}
Sea $\rel$ un orden total en $X$, $A$ un subconjunto finito y no vacío de $X$. Entonces, $A$ tiene máximo y mínimo.\end{prop}

\begin{prop} Cualquier conjunto finito con orden total está bien ordenado. \end{prop}

\subsection{Relaciones de equivalencia}

Una relación de equivalencia $\rel \subset X^2$ es una relación que cumple las siguientes propiedades para dos elementos cualesquiera $a,b \in X$.

\begin{description}
\item[Reflexiva] $a\rel a$.
\item[Simétrica] $a\rel b \dimplies b\rel a$.
\item[Transitiva] $a\rel b \y b\rel c \implies a\rel c$.
\end{description}

\begin{defn}[Clases de equivalencia] Una clase de equivalencia es un subconjunto de $\rel$ tal que todos los elementos de ese subconjunto están relacionados entre sí. Es decir, para una relación de equivalencia $\rel\subset X^2$, la clase de equivalencia del elemento $a\in X$ (que se denota como $[a]$ o $\overline{a}$) es:

\[ [a]=\{x\in X \tq x\rel a\} \]
\end{defn}

Las clases de equivalencia cumplen cuatro propiedades:

\begin{itemize}
\item $\forall a \in A \: a \in [a]$
\item $ c \in [a] \implies [c]=[a] $
\item $ a\rel b \implies [a]=[b] $
\item $ [a]\neq [b] \implies [a] \cap [b] = \emptyset $
\end{itemize}

\begin{defn}[Particiones]
Un conjunto $B$ es partición de $A$ si $(\forall x \in B \: x \subset A \y x \neq \emptyset) \y (\bigcup_{X\in B} X = A) \y (\forall x,y \in B \: x\neq y \: x \cap y = \emptyset)$.

Es decir, $B$ es partición de $A$ si cumple que todos los elementos de $B$ son subconjuntos de $A$ y ninguno de ellos es el vacío, que todos los elementos de $A$ están en los conjuntos de $B$ y que dos elementos cualquiera de $B$ no comparten ningún elemento común.
\end{defn}

\begin{prop}
A una partición $B$ de un conjunto $A$ le corresponde una relación de equivalencia $\rel_B$ en $A$, de tal forma que $\forall x,y \in A \: x\rel y \dimplies \exists X \in B \tq x,y \in X$.
\end{prop}

\begin{proof}
Demostramos que la relación es de equivalencia.

1. Reflexiva: al ser una partición, cualquier elemento de $A$ está en un conjunto $X$ de la partición, por lo tanto $\exists X \in B \tq x \in X$.

2. Simétrica. Evidente.

3. Transitiva. $a\rel b \y b\rel c \implies (\exists X \in B \tq a,b \in X) \y (\exists X' \in B \tq c,b \in X')$. Por ser $B$ partición, si $X\neq X' \implies X \cap X'$. Como $X$ y $X'$ comparten un elemento, $b$, entonces $X=X' \implies a,c \in X \implies a\rel c$.
\end{proof}

\begin{prop}
De forma análoga, una relación $\rel$ en $A$ define una partición $B$, que es el conjunto de todas las clases de equivalencia de $\rel$ en $A$.
\end{prop}

\begin{proof}
Demostramos que el conjunto de las clases de equivalencia, llamado $A_\rel$ es partición de $A$.

1ª condición: Según la definición de clase de equivalencia, $[x] =\{ a \in A \tq x\rel a\}$. De aquí deducimos que $[x] \subset A$ y que $[x]\neq \emptyset$ porque la clase de equivalencia de $x$ tiene al menos un elemento, que es $x$

2ª condición: Demostramos por reducción al absurdo. $\exists a \in A \tq \neg(a \in \bigcup_{X\in A_\rel} X)$. Por ser $a\in A$, $a\rel a$ por lo tanto $\exists [a] \in A_\rel$. Contradicción.

3ª condición. Demostramos por reducción al absurdo. Sean $[x]$ y $[y]$ dos elementos de $A_\rel$ tal que $[x] \cap [y] \neq \emptyset$. Por tanto, $\exists \alpha \in [x] \cap [y] \implies (\alpha \rel x \y \alpha \rel y)$. Por la propiedad transitiva, $x\rel y$ de lo que se deduce que $[x]=[y]$. Contradicción.
\end{proof}

\begin{defn}[Conjunto cociente] El conjunto cociente $A/\sim$ es el conjunto de las clases de equivalencia de $A$ con respecto a $\sim$.\end{defn}

\begin{defn}[Proyección canónica] La proyección canónica se define como $\appl{\Pi}{A}{A/\sim}$ tal que $a\longrightarrow^{\Pi}[a]$ y asigna a cada elemento de $A$ su clase de equivalencia correspondiente.\end{defn}

\subsubsection{Congruencias módulo $n$}
 Sea $\rel\subset \ent^2$, $a\rel b\dimplies a\equiv b$ (mod n). Es decir, dos elementos están relacionados si $n$ divide a su diferencia. Las clases de equivalencia de $\rel$ son las congruencias módulo $n$.

\section{Cotas, supremos e ínfimos, máximos y mínimos}

\begin{defn}[Cota superior e inferior] Sea $A$ un conjunto ordenado con $\rel$. $x$ es cota superior si y sólo si $\forall a\in A \: a\rel x$.

Si un conjunto tiene cota superior, se dice que está acotado superiormente.

La definición es análoga para la cota inferior.
\end{defn}

\begin{defn}[Supremo e ínfimo]
Una cota superior $x$ del conjunto ordenado con $\rel$ $A$ es el supremo del conjunto si es la menor cota superior de ese conjunto. Es decir

\[sup(A)=x \dimplies (\forall a \in A \; a\rel x) \y (\forall  \alpha \rel x\: \exists a \in A \tq \alpha \rel a) \]
ó
La definición es análoga para el ínfimo.
\end{defn}

\begin{prop}
Si el orden es total no hay más de un supremo ni más de un ínfimo.
\end{prop}

\begin{prop}
Si $a = max(A)$, entonces $a$ es el supremo.\end{prop}

\begin{axiom}[Axioma del supremo] Todo conjunto acotado superiormente tiene supremo.\end{axiom}
\begin{theorem}[Principio del ínfimo] Todo conjunto acotado inferiormente tiene ínfimo.\end{theorem}

\begin{defn}[Máximo y mínimo] Sea $A$ un conjunto con orden $\rel$ y $x$ un elemento de $A$. $x$ es máximo si y sólo si $\forall a \in A \: a\rel x$.

La definición es análoga para el mínimo.
\end{defn}

\begin{prop}No hay más que un máximo y un mínimo en el conjunto. Si existen, son únicos.

\begin{proof}
Sea $A$ un conjunto, y $x, y \in A$ dos máximos. Según la definición de máximo tenemos por una parte que \[ \forall a \in A \: a\rel x \implies y\rel x \] y que \[ \forall b \in A \: a\rel y \implies x\rel y \]. Es decir, que tenemos $y\rel x$ y $x\rel y$. Como $\rel$ es una relación de orden, $x=y$.
\end{proof}

\end{prop}


\begin{defn}[Maximal y minimal] Sea $A$ un conjunto con orden $\rel$ y $x\in A$. $x$ es maximal si y sólo si $\nexists y \in A \tq x\rel y$. La definición es análoga para el minimal.\end{defn}

\begin{remark} Si $A$ no tiene un orden total (esto es, que no todos los elementos de $A$ se puedan comparar entre sí), puede haber varios maximales y minimales. Por ejemplo, sea $A=\{ (x,y) \in \rel^2 \tq x^2+y^2=1\}$ (una circunferencia) y la relación de orden $\rel \subset A^2 : \; (x,y)\rel(x',y') \dimplies (x\leq x') \y (y \leq y')$. Los maximales son los elementos del primer cuadrante: son mayores que cualquier elemento con el que los podamos comparar. De la misma forma, los minimales son todos los elementos del tercer cuadrante.\end{remark}



\section{Teoría de conjuntos}

\subsection{Teoría de conjuntos de Zermelo-Fraenkel (ZFC)}
La teoría de conjuntos de Zermelo-Fraenkel usa el lenguaje de cálculo de predicados y dos predicados: \[=(x_1,x_2) \dimplies x_1=x_2\] \[\in (x_1,x_2) \dimplies x_1 \in x_2 \]

Se basa en los siguientes axiomas

\begin{description}
\item[ZF1: Extensión] (Igualdad de conjuntos). $A=B \dimplies (\forall x\: x\in A \dimplies x \in B)$.
\item[ZF2: Conjunto vacío] $\exists \emptyset \: \forall x \: \neg(x\in \emptyset)$.
\item[ZF3: Axioma del par] $\forall x,y \exists z \tq \forall a a \in z \dimplies (a=x) \Or (a=y)$.
\item[ZF4: Unión] $\forall x \exists y \tq \forall z \: z\in y \dimplies \exists u \tq u\subset x \y x\in u$.
\item[ZF5: Axioma conjunto potencia - conjunto de partes] $ \forall x \:\exists \parts{x} \tq \forall  z \; z\in \parts{x} \dimplies z\subset x$.
\item[ZF6: Axioma de separación] $\forall A(y) \:\forall x \: \exists c=\{y\in x \tq A(y)\}$.
\item[ZF7: Axioma de recíproco] $\forall u \: \exists v \tq v=\{y \: \exists x\in u \tq F(x,y)\}$.
\item[ZF8: Existencia conjunto infinito.]
\end{description}

A la teoría ZFC se le añade el axioma de elección. Su inclusión o no no provoca ninguna contradicción con ZFC.

\begin{defn}[Axioma de elección]
Dado un conjunto $M$ cuyos elementos son conjuntos no vacíos existe una función de elección $\appl{f}{M}{\bigcup_{X\in M}X} \tq f(X)\in X \;\forall X\in M$.
\end{defn}

Hay varias afirmaciones equivalentes al axioma de elección.

\begin{theorem}[Principio de buena ordenación]
Todo conjunto puede ser bien ordenado.
\end{theorem}

\begin{lemma}[Lema de Zorn]
Todo conjunto ordenado inductivo no vacío tiene al menos un maximal (ver definiciones \ref{dfCadena}, \ref{dfCInduc}).
\end{lemma}

\begin{defn}[Cadena]\label{dfCadena}
Sea $\rel$ un orden en $X$. $Y\subset X$ es una cadena si $\rel$ define un orden total en $Y$.
\end{defn}

\begin{defn}[Conjunto inductivo]\label{dfCInduc}
Sea $\rel$ un orden en $X$. $X$ es inductivo si toda cadena en $X$ tiene cota superior.
\end{defn}

\subsection{Cardinalidad y numerabilidad}

\begin{defn}[Cardinal] El cardinal de un conjunto es el número de elementos de ese conjunto.\end{defn}

\begin{defn}[Precedencia] $A$ precede a $B$ (notación: $A \prec B$) si $\card{A} \leq \card{B}$. Esto ocurre si $\exists \appl{\varphi }{A}{B}$ inyectiva.\end{defn}

\begin{defn}[Equipotencia] $A$ es equipotente a $B$ (notación: $A \sim B$) si $\card{A} = \card{B}$. Esto ocurre si $\exists  \appl{\varphi }{A}{B}$ biyectiva.
\end{defn}

\begin{prop}[Propiedades de los cardinales]

\begin{enumerate}
\item $\card{A}=\card{B} \y \card{B}=\card{C} \implies \card{A}=\card{C}$.
\item $\card{A}<\card{B}=\card{C} \implies \card{A}=\card{C}$.
\item $\card{A}\leq \card{B} \y \card{B} \leq \card{B} \implies \card{A} \leq \card{C}$.
\end{enumerate}
\end{prop}

\begin{theorem}[Teorema de Cantor]
\[ [0,1) \nsim \nat \]
\end{theorem}

\begin{proof}
Demostramos por reducción al absurdo. Suponemos que $\exists \appl{f}{[0,1)}{\nat}$ biyectiva, tal que $x_n=f(n)\in [0,1)$ y que $\{x_n\tq n\in \nat\} = \nat$.

Cada $x_n$ se puede escribir como $x_n=0,j_{n0}j_{n1}\cdots j_{nm}$. Todos los $x_n$ son distintos y además cubren todo el intervalo $[0,1)$. 

Sea $\varphi=0,q_0q_1q_2\cdots q_m$, seleccionando los $q_n$ de tal forma que $q_n \neq j_{nn}$ y que $q_n \neq 9$. Tal y como hemos construido $\varphi$ no va a ser igual a ningún $x_n$, pero sin embargo está en el conjunto $[0,1)$. Contradicción
\end{proof}

\begin{prop}\[\nat \sim \ent\]\end{prop}

\begin{proof}
Sea $\appl{f}{\nat}{\ent}$ tal que $f(2n+1)=-n$ y $f(2n)=n$. Es obvio que $f$ es biyectiva. 
\end{proof}

\begin{prop}$\rac$ es infinito y numerable\end{prop}

\begin{proof}
Sea $\appl{f}{\ent\x\ent^{+}}{\rac}$ tal que $f(n,m)=\frac{n}{m}$, es trivial que $f$ es biyectiva. Sabemos que $\ent\x\ent$ es numerable, así $\ent\x\ent^+\sim \rac$.
\end{proof}

\begin{defn}[Conjuntos numerables]
Un conjunto $A$ es numerable si y sólo $\exists \appl{f}{A}{\nat}$ inyectiva, equivalente a que $A\neq \emptyset ; \exists \appl{g}{\nat}{A}$ sobreyectiva.

Es decir, un conjunto es numerable si $A$ es finito o $A\sim \nat$.
\end{defn}

\begin{prop} Si $A$ y $B$ son numerables, $A\x B$ y $A \cup B$ también lo son.\end{prop}

\begin{prop}Un subconjunto de un conjunto numerable es numerable.\end{prop}

\begin{theorem}
Una unión numerable de conjuntos numerables es numerable.
\end{theorem}

\begin{defn}[Conjuntos infinitos] Un conjunto $A$ es infinito en el sentido de Delankel si existe un subconjunto de $A$ equipotente a $A$.

Un conjunto es infinito en el sentido de Delankel si y sólo si es infinito.\end{defn}

\begin{theorem}[Teorema de Cantor-Schröder-Bernstein]
Sean $A$ y $B$ dos conjuntos. Entonces \[ (A<B) \y (B<A) \implies A\sim B \]
\end{theorem}

\begin{defn}
\[\aleph_0 =\card{\nat} \]
\[ c = \card{\real} \]
\end{defn}

\begin{theorem}
\[c+c = c\]
\[ c\cdot c = c \]
\end{theorem}

\begin{prop} Sean $\alpha=\card{A}$ y $\beta = \card{B}$, y $A\cap B =\emptyset$. Entonces 

\begin{align*}
\alpha\beta &= \card{A\x B} \\
\alpha + \beta &= \card{A\cup B}\\
\end{align*}

Además, la suma de cardinales cumple las propiedades conmutativa, distributiva y asociativa.

Por otra parte, si $A^B=\{\appl{f}{B}{A}\}$, entonces $\card{A^B}=\card{A}^{\card{B}}=\alpha^\beta$.
\end{prop}

\begin{lemma} Sean $A$, $B$, $C$ conjuntos tales que $B\cap C = \emptyset$. Entonces, $\appl{f}{A^B\x A^C}{A^{B\cup C}}$ es biyectiva.
Además, $\left(A^B\right)^C \sim A^{B\x C}$.
\end{lemma}

\begin{corol}
Sean $\alpha$, $\beta$, $\gamma$ cardinales de los conjuntos $A$, $B$ y $C$. Entonces \[ \alpha^{\beta+\gamma}=\alpha^\beta\alpha^\gamma\] \[ {\alpha^\beta}^\gamma=\alpha^{\beta\gamma}\]
\end{corol}

\begin{theorem}
\[ \real \sim \{0,1\}^\nat=\parts{\nat} \]
\[ c = 2^{\aleph_0} \]
\end{theorem}

\section{Grupos}
\begin{defn}[Operación binaria]
Sea $G$ un conjunto. $\appl{\varphi}{G\x G}{G}$ es una operación binaria. Se denota como $\varphi (g_1, g_2)=g_1\ast g_2$.
\end{defn}

\begin{defn}[Grupos]
Sea $(G, \varphi)$ un conjunto con una operación binaria. Se dice que es un grupo si se cumple lo siguiente:
\begin{enumerate}
\item Asociativa: $\forall a,b,c \in G \; a\ast (b\ast c) = (a \ast b) \ast c$.
\item Existencia del neutro: $\exists e \in G \tq \forall a \in G \; a \ast e = e \ast a = a$. El neutro es único (si $e_1$ y $e_2$ son elementos neutros, tenemos que $e_1=e_1 \ast e_2 =e_2$.
\item Existencia del simétrico: $\forall a \in G \; \exists \inv{a} \tq a \ast \inv{a} =\inv{a}\ast a = e$.
\end{enumerate}
\end{defn}

\begin{remark} No incluye la propiedad de operación cerrada porque está incluida en la definición de operación binaria.\end{remark}

\subsection{Propiedades de los grupos}

\begin{description}
\item[Cancelación]. $a\ast b = a \ast c \implies b=c$.
\item[Unicidad del inverso]. Cada elemento tiene un único inverso.
\item[Doble inversión]. $\inv{\inv{g}}=g$.
\end{description}

\indent \\
Demostraciones:

\begin{description}
\item[Cancelación]: 
\begin{align*}
a\ast b &= a \ast c \\
\inv{a} \ast a \ast b &= \inv{a} \ast a \ast c \\
e \ast b &= e \ast c \\
b&=c
\end{align*}
\item[Unicidad del inverso]. Sean $\inv{g}, g'$ dos inversos de $g$. Entonces $\inv{g}\ast g = g' \ast g$. Por la propiedad de cancelación, $\inv{g}=g'$.
\\
\item[Doble inversión]. $\inv{\inv{g}} \ast \inv{g} = g \ast \inv{g}$. Por la propiedad de cancelación, $\inv{\inv{g}} = g$.
\end{description}

\begin{defn}[Grupo conmutativo o abeliano] Sea $G$ un grupo. $G$ es conmutativo si $\forall a,b \in G\; a\ast b =b \ast a$.\end{defn}

\begin{remark} En general, \[ \inv{\left(a\ast b\right)} = \inv{b}\ast \inv{a}\].\end{remark}

\begin{prop} $G$ es abeliano si y sólo si $\inv{(a\ast b)}=\inv{a} \ast \inv{b}$.\end{prop}

\begin{proof}
Un grupo conmutativo cumple que $\forall a,b$ elementos del grupo $a \ast b =b \ast a$. Por lo tanto, si el grupo no fuera conmutativo $\inv{(a\ast b)}=\inv{b}\ast \inv{a}$ no sería igual a $\inv{a} \ast \inv{b}$
\end{proof}

\section{Anillos}
\begin{defn}[Anillo] Se dice que un conjunto $A$ dotado de dos operaciones binarias llamadas suma y producto ($+$ y $\cdot$) es un anillo si se cumple que: 
\begin{enumerate}
\item $(A, +)$ es un grupo abeliano.
\item El producto es asociativo.
\item Se cumplen las leyes distributivas: $\forall a,b,c \in A \; a\cdot(b+c)=a\cdot b + a\cdot c$ y $(a+b)\cdot c= a \cdot c + b \cdot c$.
\end{enumerate}
\end{defn}

\begin{defn}[Anillo conmutativo] Si el producto es conmutativo, el anillo es conmutativo.\end{defn}

\begin{defn}[Anillo con unidad] Si hay un elemento neutro para el producto, el anillo es un anillo con unidad.\end{defn}

\begin{defn}[Divisores de 0] Sea $A$ un anillo. $a\in A$ es divisor de 0 en $A$ si $a\neq 0$ y $\exists b\neq 0 \in A \tq a\cdot b=b\cdot a =0$. 

Ejemplo: $[3], [4]$ son divisores de 0 en las clases de congruencia de los enteros módulo 12.\end{defn}

\begin{defn}[Dominio de integridad] Un anillo es un dominio de integridad si no tiene divisores de 0.\end{defn}

\begin{defn}[Asociados] $a, b\in A \backslash \{0\}$ son asociados $(a\sim b)$ si $a|b$ y $b|a$. \end{defn}

\begin{defn}[Elementos invertibles] Sea $A$ un anillo con unidad $e$. $a\in A$ es invertible si $\exists b \in A \tq b\cdot a = a\cdot b = e$.\end{defn}

\begin{lemma}Sea $A$ un anillo con unidad. El conjunto de todos sus elementos invertibles es un grupo respecto del producto.\end{lemma}

\begin{defn}[Anillo euclídeo] Un anillo $A$ es euclídeo si es conmutativo, con unidad, sin divisores de 0 y además $\exists  \appl{f}{A \backslash \{0\}}{\nat}$ que cumple que \begin{enumerate}
\item $f(r)\leq f(rs) \; \forall r,s\neq 0 \in A$.
\item $\forall t,s\in A \; s\neq 0 \; \exists c,r \tq t=c\cdot s +r$.
\item $r=0 \Or f(r)\leq g(s)$.
\end{enumerate}
\end{defn}

\begin{defn}[Máximo común divisor]. Sea $A$ un anillo conmutativo, y $a,b$ dos elementos del anillo. $a|b$ si $\exists c\in A \tq b=a\cdot c$. $d$ es MCD de $a$ y $b$ si se cumple que $d|a \y d|b \y \left((d'|a \y d'|b)\implies d'|d\right)$.\end{defn}

\begin{defn}[Algoritmo euclídeo] Sirve para hallar el MCD de dos números. Dividimos el mayor entre el menor: el cociente es $c$ y el resto $r$. La siguiente división es entre los dos números mayores del cociente, resto y divisor. Las divisiones se hacen sucesivamente, hasta que el resto es cero. El último resto no nulo es el MCD.\end{defn}

Damos un ejemplo de aplicación del algoritmo de Euclides para hallar el MCD de 329 y 273.

\begin{align*}
329&=1\cdot 273 +56 \\
273&=4\cdot 56 + 49 \\
56&= 1 \cdot 49 + 7 \\
49&= 7\cdot 7 +0
\end{align*}

Hemos llegado al resto 0, así que el MCD es 7 (el último resto no nulo).

\begin{corol}Para dos elementos cualquiera de un anillo euclídeo, tienen máximo común divisor.

El máximo común divisor $d$ de dos elementos $a, b$ de un anillo euclídeo siempre se puede representar de forma lineal: $d=sa +tb$.\end{corol}

\begin{lemma}
Si $A$ no tiene divisores de 0 y $\alpha \neq 0$, entonces $\alpha a = \alpha b \implies a=b$.\end{lemma}
\begin{proof}
Por la propiedad distributiva, $\alpha (a-b)=0$, y como $\alpha \neq 0$ $a-b=0$.
\end{proof}

\begin{lemma}Sea $A$ un dominio euclídeo y $a,b,\alpha$ elementos no nulos, entonces $mcd(\alpha a, \alpha b)=\alpha \cdot mcd(a, b)$.\end{lemma}


\section{Cuerpos}

\begin{defn}[Cuerpo] $(A, +, \cdot)$ es un cuerpo si \begin{enumerate}
\item Es un anillo conmutativo con unidad.
\item $A^*=A \backslash \{0\}$ tiene que ser un grupo respecto del producto.
\end{enumerate}
\end{defn}

\section{Números complejos}

\begin{defn}[Conjunto de complejos]\[ \cplex=\real \x \real = \real ^2 \] \end{defn}

$\real^2$ es un espacio vectorial, por lo que podemos represenar un número complejo en el espacio vectorial (el plano complejo).

\begin{defn}[Producto en los complejos]
\[ (a,b) \cdot (c,d) = (ac-bd, ad+bc) \]
\end{defn}

\begin{remark} Hay una función que lleva los reales a los complejos:

\begin{align*}
\appl{\varphi}{\real}{\cplex} \\
a\overset{\varphi}{\rightarrow}(a,0)
\end{align*}

$\varphi$ conserva las operaciones de suma y producto. Identificamos cualquier número real con su imagen $\varphi(a)$. En particular, $\varphi(0)=(0,0)$ es el elemento neutro respecto de la suma en los complejos y $\varphi(1)=(1,0)$ es el elemento neutro respecto del producto.
\end{remark}

\begin{defn}[Módulo] \[|z|=\sqrt{x^2+y^2} \] \end{defn}

\begin{defn}[Conjugación] Sea $z=x+yi \in \cplex$, entonces $\conj{z}=x-yi$. Es una biyección de $\cplex$ en $\cplex$ y conserva las operaciones, es un automorfismo.\end{defn}


Propiedades de los conjugados:
\begin{enumerate}
\item $\conj{z+w}=\conj{z}+\conj{w}$.
\item $\conj{z\cdot w}=\conj{z}\cdot \conj{w}$.
\item $z\cdot\conj{z}=|z|^2$.
\item $\conj{\left(\frac{1}{z}\right)}=\frac{1}{\conj{z}}$.
\end{enumerate}

\begin{theorem}$(\cplex, +, \cdot)$ es un cuerpo.\end{theorem}

\begin{proof}
Partimos de que es un anillo: demostración trivial.

Elemento unidad: $1=(1,0);\; 1\cdot(x+yi)=x+yi$.

Inverso: Todo $z \in \cplex \neq 0$ tiene inverso:

\begin{align*}
z\cdot \conj{z}&=|z|^2 \\
z \frac{\conj{z}}{|z|^2}&=1 \\
\exists \inv{z} &= \frac{\conj{z}}{|z|^2}
\end{align*}
\end{proof}

\subsection{Representación trigonométrica}

\begin{lemma} Sea $(x,y) \in \mathbb{T}$ (un punto de la circunferencia unidad). Entonces 
\[ \exists \varphi  \tq x=\cos \varphi \;\y\; y=\sin \varphi \].

$\varphi$ es única salvo el cambio $\varphi \to \varphi + 2k\pi$, $k\in \ent$.

En otras palabras, sea $z=x+iy \tq |z|=1$, entonces $z=\cos\varphi +i \sin \varphi$ para algún $\varphi \in \real$.
\end{lemma}

De esta forma podemos representar cualquier $z\in \cplex$ como 
\[ z=|z|\left(\cos \varphi + i \sin \varphi\right) \]

$\varphi$ es el argumento de $z$.

\begin{theorem}[Fórmula de De Moivre]
\[ (\cos \alpha + i \sin \alpha)^n=\cos n\alpha+i\sin n\alpha \]
\end{theorem}

El producto de complejos es intuitivo:
\[\ceul{a}{\alpha}\cdot\ceul{b}{\beta}=\ceul{(ab)}{(\alpha+\beta)} \]

Propiedades de los complejos:

\begin{enumerate}
\item $|z+w|\leq |z|+|w|$
\item $Re z = \frac{z+\conj{z}}{2}$, $Im z = \frac{z-\conj{z}}{2i}$
\item $|zw| = |z| \cdot |w|$
\item $arg \inv{z} = -arg z$.
\end{enumerate}

\subsection{Raíces de números complejos}

La ecuación $w^n=z$ tiene $n$ soluciones. Igualando con la fórmula de Euler
\[\ceul{\varphi^n}{n\alpha}=\ceul{r}{\theta} \] por lo tanto $\varphi=\sqrt[n]{r}$ y $\alpha=\frac{\theta +2k\pi}{n}$. Es decir:

\[ \sqrt[n]{z}=\ceul{\sqrt[n]{r}}{\frac{\theta+2k\pi}{n}} \]

\subsection{Series de números complejos}

\begin{defn} La suma infinita de $a_n$ real converge si existe el límite de la suma.\end{defn}

\begin{defn} Si $z_n=x_n+iy_n$ es una serie compleja, se dice que la suma infinita converge si las sumas infinitas de $x_n$ y $y_n$ convergen.\end{defn}

\begin{lemma}\[ \sum^\infty |z_n|\; \textit{converge}\implies \sum^\infty z_n \;\textit{converge} \]\end{lemma}

\begin{defn}Se dice que $\sum^\infty z_n$ converge absolutamente si converge $\sum^\infty |z_n|$.\end{defn}

\begin{lemma}
\[ \sum^\infty z_n \; \textit{conv.} \dimplies \sum^\infty\conj{z_n} \; \textit{conv.} \]
\end{lemma}

\begin{theorem}
\[\forall z,w \in \cplex \;\; e^ze^w=e^{z+w} \]
\end{theorem}

\begin{lemma}
\[\conj{e^z}=e^{\conj{z}}\]
\end{lemma}

\begin{lemma}
\[ e^{iy} \in \mathbf{T} \]
\end{lemma}

\begin{lemma}Sea $f(t)=\sum^\infty a_n t^n$. Supongamos que esta serie converge $\forall t \in [-\epsilon, \epsilon]$. Entonces $\exists f'(0)=a_1$.\end{lemma}

\begin{theorem}
Sea $\alpha \in \cplex$, y $f(t)=f_\alpha (t)=e^{\alpha t}$. Entonces, $\forall t \in \real\;\exists f'(t)=\alpha f(t)$.
\end{theorem}

\begin{theorem}
\[e^{it}=\cos t + i \sin t \; \forall t \in \real \]
\[ \cos t = \frac{e^{it}+e^{-it}}{2} \]
\[ \sin t = \frac{e^{it}-e^{-it}}{2} \]
\end{theorem}

\begin{theorem}
Sea $n\in \nat, n\geq 2$. Entonces las siguientes proposiciones son equivalentes: $\ent_n$ no tiene divisores de 0, $\ent_n es un cuerpo$, $n$ es primo.
\end{theorem}

\section{La función $\varphi$ de Euler}

\begin{theorem}
Sean $a,c,m\;\in\ent$, y $m>0$. $d=mcd(a,m)$. Consideramos la ecuación $ax\equiv c \mod n$. Entonces, si $d$ no divide a $c$ no hay soluciones. Si $d|c$, hay $d$ soluciones no congruentes módulo n.
\end{theorem}

\begin{theorem}
Sen $a,b,m,n$ números enteros, $m$ y $n$ positivos y con mcd = 1. Entonces, el sistema de congruencias
\[\begin{matrix}
x\equiv a \mod m \\
x\equiv b \mod n 
\end{matrix}\] tiene una única solución módulo $mn$.
\end{theorem}

\begin{lemma}
Sean $m,n>0$, con mcd = 1. Sea $\appl{f}{\ent_m\x\ent_n}{\ent_{nm}}$, tal que si $[a] \in \ent_m$, $[b] \in \ent_n$, $f([a],[b])=[x]$. Entonces \begin{enumerate}
\item $f$ está bien definida y es biyectiva.
\item $f(U(\ent_m)\x U (\ent_n))=U(\ent_{nm})$.
\item $\begin{matrix}
x\equiv a \mod m \\
x\equiv b \mod n 
\end{matrix}$
\end{enumerate}
\end{lemma}

\begin{defn}[Función de Euler]
\[ mcd(m,n)=1\implies \varphi(mn)=\varphi(m)\varphi(n) \]
\end{defn}

\begin{theorem}
\[\forall n\in \ent^+ \; \varphi(n)=n\prod_{p|n} \left(1-\frac{1}{p}\right)\; ; p \textit{ los primos que dividen a }n\]
\end{theorem}

\begin{lemma}
\[\#(U(\ent_n)) = \varphi (n) \]
\end{lemma}

\begin{theorem}[Teorema de Euler]
Para $n>1$ y un $a$ tal que $\mathrm{mcd}(a,n) =1$, $a^{\varphi(n)}=1 \mod n$.
\end{theorem}

\begin{theorem}[Teorema de Fermat]
Si $p$ es primo y $p$ no divide a $a$, entonces $a^{p-1}\equiv 1 \mod p$.
\end{theorem}

\section{Anillos de polinomios}
\begin{defn}Sea $A$ un anillo conmutativo. Un polinomio en $X$ con coeficientes en $A$ es una expresión formal $P(X)$. Si $a_n \neq 0$, $a_n$ es el coeficiente principal. El grado de $P$ es $n$.\end{defn}

Operaciones con polinomios:
\begin{enumerate}
\item $(P+Q)(X) = \sum_{k=0}^{max(n,m)} (a_k+b_k)X^k$.
\item $P(X)Q(X) = \sum_{k=0}^{n+m} c_k X^k$, donde $c_k=\sum a_l b_{k-l}$ y $0\leq l \leq n$, $0\leq k-l\leq m$.
\end{enumerate}

\begin{defn} Denotamos como $A[X]$ el conjunto de todos los polinomios con coeficientes en $A$.\end{defn}

\begin{lemma}Si $A$ es un anillo conmutativo, $A[X]$ también lo es.

Si $A$ no tiene divisores de cero, entonces $A[X]$ tampoco y el grado de $P\cdot Q \in A[X] = gr(P) + gr(Q)$.

Sean $P, Q \in A[X]$, $A$ sin divisores de cero. Entonce $\uexists C(X), R(X) \tq P(X) = C(X)Q(X) + R(X) \y gr(R) < gr(Q)$.
\end{lemma}

\begin{theorem}[Teorema principal del álgebra]
Sea $P\in \cplex [X] \tq gr(P) \geq 1$. Entonces, $P$ tiene al menos una raíz compleja.
\end{theorem}

\begin{corol} Si $P \in \cplex [X] \tq gr(P) = n \geq 1$, entonces $P$ tiene exactamente $n$ raíces complejas (contando con los conjugados de esas raíces. 

$P$ es irreducible en $\cplex [X]$ si y sólo si $gr(P) = 1$.
\end{corol}

\begin{lemma} Si $\mathbb{K}$ es un cuerpo, las unidades en $\mathbb{K}[X]$ son los polinomios constantes no nulos.\end{lemma}

\begin{theorem}
Sea $\mathbb{K}$ un cuerpo, y $P\in \mathbb{K} [X]$. Entonces $P$ es reducible en $\mathbb{K}[X]$ si y sólo si $P$ tiene una raíz en $\mathbb{K} [X]$.
\end{theorem}

\begin{theorem}
$P \in \real [X]$. $P$ es irreducible si y sólo si $gr(P) = 1$ o $gr(P) = 2 \y P(X) = aX^2+bX+c \tq a,b,c\in \real \y a \neq 0 \y b^2 -4ac < 0$.
\end{theorem}

\subsection{Polinomios en $\ent [X]$ y $\rac [X]$}

\begin{prop}$P(X) = a_n X^n +\cdots a_0$, $P \in \ent [X]$. Supongamos que $P(c/d) = 0$, donde $\frac{c}{d} \in \rac$ es una fracción irreducible. Entonces $d|a_n \y c|a_0$.\end{prop}

\begin{defn}$A$ un dominio euclídeo, $a_1\cdots a_l \in A$. Decimos que $d=mcd(a_1,\cdots,a_l)$ si $d|a_j\; j=1,\cdots, l$ y $ \forall d'\in A \tq d'|a_j\; j=1,\cdots, l \; ; d'|d$.

$d$ existe y es único.\end{defn}

\begin{defn}[Contenido] Sea $P \in \ent [X] = a_nX^n +\cdots + a_0$, $a_i \in \ent$. Definimos el contenido $C(P) = mcd (a_0,\cdots,a_n)$.\end{defn}

\begin{defn}[Primitivo] $P$ es primivito si $C(P) = 1$.\end{defn}

\begin{lemma}\[\forall P,Q\in \ent [X] ; C(P)=C(Q)=1 \implies C(PQ) = 1 \]\end{lemma} 

\begin{lemma}[Lema de Gauss] \[P,Q\in \ent [X] \implies C(PQ) = C(P) \cdot C(Q)\]\end{lemma}

\begin{remark} Si $R\in \ent [X]$ es reducible en $\rac [X]$ también lo es en $\ent [X]$.\end{remark}

\begin{theorem}
Sea $A$ un dominio euclídeo y $\mathrm{K}$ un cuerpo. $P(X) \in A[X] \tq gr(P) =n$. $P$ tiene $n$ raíces $b_j$ contando con sus multiplicidades. Entonces \[P=a_n \prod_{i=0}^n (X-b_i)\].
\end{theorem}

\subsection{Factorización de polinomios en $\ent [X]$ y en $\rac [X]$}

\begin{theorem}
Sea $P \in \ent [X]$ primitivo. Entonces $P$ es reducible en $\rac [X]$ si y sólo si es reducible en $\ent [X]$.
\end{theorem}

\subsection{Criterios de irreducibilidad en $\ent [X]$ y en $\rac[X]$}

\begin{remark} Un polinomio $P$ es reducible en $\ent [X]$ si $C(P) > 1$.\end{remark}

\begin{defn}[Reducción módulo $p$] Sea $p\in \nat$ primo. $\appl{\varphi_p}{\ent[X]}{\ent_p [X]}$. $\varphi(a_nX^n+\cdots)=\conj{a_n}X^n+\cdots$.\end{defn}

\begin{theorem}
Sea $R(X)$ primitivo, y $p$ no divide al coeficiente principal de $R$. Si $\varphi_p(R)\in \ent_p [X]$ es irreducible entonces $R$ es irreducible en $\rac[X]$ y en $\ent[X]$.
\end{theorem}

\section{Homomorfismos de anillos}

\begin{defn}[Homomorfismo] Sean $A,B$ dos anillos. Sea $\appl{\varphi}{A}{B}$ una función. $\varphi$ es un homomorfismo si $\forall a_1, a_2 \in A \;\; \varphi(a_1+a_2) =\varphi(a_1) + \varphi(a_2) \y \varphi(a_1a_2) = \varphi(a_1)\varphi(a_2)$.
\end{defn}

\label{LastPage}
\end{document}
