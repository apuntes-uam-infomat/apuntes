\documentclass[palatino]{apuntes}

\usepackage{wasysym}

\title{Topologia2017}
\author{}
\date{17/18 C1}

% Paquetes adicionales

% --------------------

\begin{document}
\pagestyle{plain}
\maketitle

\tableofcontents
\newpage
% Contenido.

\chapter{Espacios Métricos}

\begin{defn}[Métrica]
	Sea $ X \neq $ un conjunto. Una \textbf{función} $ \appl{f}{X × X}{ℝ} $ es una métrica si cumple las siguientes propiedades:

	\begin{itemize}
		\item $ d(x,y) \geq 0, \forall x, y \in X $
		\item $ d(x,y) = 0, \dimplies x = y $
		\item $ d(x,y) = d(y,x), \forall x, y \in X $
		\item $ d(x,z) \leq d(x,y) + d(y,z), \forall x, y, z \in X $
	\end{itemize}
	
	Llamaremos distancia entre $ x $ e $ y $ a $ d(x,y) $
\end{defn}

\begin{defn}[Espacio métrico]
	Un espacio métrico es un par formado por un conjunto y una métrica en él y se denota por (X,d)
\end{defn}


% TODO buscar como se ponen los ejemplos

\textbf{Ejemplos}
\begin{enumerate}
	\item $ X = ℝ $, $d(x,y) = \abs{x-y} $ Podemos definir otras métricas en $ ℝ $
	\item $ d(x,y)= \frac{\abs{x-y}}{1+\abs{x-y}} \leq 1 $
	\item $ d(x,y)= \arctan{\abs{x-y}} \leq \frac{π}{2} $
\end{enumerate}



% TODO meter algo aquí para serpararlo y que quede mejor

Si $ \appl{φ}{[0,+∞)}{[0,+∞)} $ tiene las siguientes propiedades:

\begin{itemize}
	\item $ φ(0) = 0 $
	\item $ φ \nearrow $ debe ser estrictamente creciente
	\item $ φ(a+b)≤ φ(a) + φ(b) $ (subaditiva)
\end{itemize}

% TODO MIRAR COMO ES LA COMPOSICIÓN

y sea $ d $ una métrica en X, entonces $ φ \circ d $ es una métrica en X

\begin{itemize}
	\item $ ρ(x,y)= φ (d(x,y)) $
	\item $ \appl{ρ}{X \x X}{ℝ} $
\end{itemize}

φ es una metrica

\begin{enumerate}
	\item \checked
	\item $ρ(x,y) = 0 \Leftrightarrow d(x,y) = 0 \Leftrightarrow x = y $
	\item \checked
	\item $ρ(x,z) = φ(d(x,z)) ≤ φ(d(x,y) + d(x,z)  ≤ φ(d(x,y)) + φ(d(x,z)) = ρ (x,y) + ρ(y,z) $	
\end{enumerate}

\begin{example}
	$ X ≠ ∅ $ arbitrario \\
	$d(x,y) = \left\{
	\begin{matrix}
	1,   x ≠ y \\
	0,   x = y\\
	\end{matrix}
	\right.$
	
	\begin{enumerate}
		\item \checked
		\item \checked
		\item \checked
		\item 
		    \begin{itemize}
		    	\item  $ d(x,z) ≤ 1 ≤ d(x,y) + d(y,z) $
			    	\item Pero si x = y e y = z $ \implies $ x = z y entonces $ d(x,z) ≤ 0 ≤ d(x,y) + d(y,z) $
		    \end{itemize}
	\end{enumerate}
\end{example}

% TODO en la pagina 5 del escaneo hay otro ejemplo

\section{Conjuntos abiertos y cerrados}

\begin{defn}[Bola abierta]
	Sea $ (X,d) $ un espacio metrico. Si $ a ∈ X $ y $ r > 0 $, la \textbf{bola abierta} de centro a y radio r es el conjunto
	$$ B(a;r) = \{x ∈ X : d(x,a)< r\}$$
\end{defn}

\begin{defn}[Bola cerrada]
	$$ \overline{B}(a;r) = \{x ∈ X : d(x,a)≤ r\}$$
\end{defn}

\begin{defn}[Esfera]
	$$ S(a;r) = \{x ∈ X : d(x,a) = r\}$$
\end{defn}

\begin{example}
	$ X = ℝ $, $ d(x,y)= \abs{x-y}$ \\
	$ B(a;r) = \{x ∈ X : \abs{x-a}< r\}=(a-r, a+r)$ \\
	$ \overline{B}(a;r) = \{x ∈ X : \abs{x-a}≤ r\} = [a-r, a+r]$ \\
	$ S(a;r) = \{x ∈ X : \abs{x-a} = r\}=\{a-r,a+r\}$
\end{example}

% TODO ejemplos pagina 6 a partir del 2




\begin{defn}[Conjunto abierto]
	Sea $ (X,d) $ un espacio metrico. Sea $ G ⊆ X $ es un cojunto abierto en X si $ ∀ x ∈ G,  ∃ε>0 $ tal que $ B(x;ε)⊆G $
\end{defn}

% TODO jemplos de la pagina 8

\textbf{Propiedades de los conjuntos abiertos:}
\begin{itemize}
	\item $ ∅ $ y X son abiertos en X
	\item Si $ G_α, α∈I $ es una coleccion arbitraria (finita o no) de conjuntos abiertos en X, entonces $ \cup_{α∈I}G_α $ es abierto en X
	\item Si $ G_1, \dots, G_n $ son abiertos, entonces $ \cap_{k=1}^n G_k$ es un abierto
\end{itemize}

% TODO demostracion pagina 8

\begin{obs}
	La coleccion $ \{G⊆X: G $ abierto $\} $ se suele llamar \textbf{topologia} (inducida por la metrica d de X)
\end{obs}



\begin{obs}
Si en un conjunto en $ ℝ^2 $ es abierto en cualquier de las metricas $ d_1, d_2, d_∞ $ tambien lo es en las restantes. Por tanto las colecciones de todos los conjuntos abiertos en $  ℝ^2 $ (en las metricas) coinciden. Lo mismo se cumple con las otras metricas $ d_p, 1 < p < ∞ $, y tambien en $ ℝ^n $.
\end{obs}













































%% Apendices (ejercicios, examenes)
\appendix

\chapter{---}
% -*- root: ../Topo2016.tex -*-


\printindex
\end{document}
