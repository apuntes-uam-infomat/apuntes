\documentclass[palatino]{apuntes}

\title{Topologia2017}
\author{}
\date{17/18 C1}

% Paquetes adicionales

% --------------------

\begin{document}
\pagestyle{plain}
\maketitle

\tableofcontents
\newpage
% Contenido.

\chapter{Espacios Métricos}

\begin{defn}[Métrica]
	Sea $ X \neq \varnothing $ un conjunto. Una \textbf{función} $ \appl{f}{X \x X}{\real} $ es una métrica si cumple las siguientes propiedades:

	\begin{itemize}
		\item $ d(x,y) \geq 0, \forall x, y \in X $
		\item $ d(x,y) = 0, \dimplies x = y $
		\item $ d(x,y) = d(y,x), \forall x, y \in X $
		\item $ d(x,z) \leq d(x,y) + d(y,z), \forall x, y, z \in X $
	\end{itemize}
	
	Llamaremos distancia entre $ x $ e $ y $ a $ d(x,y) $
\end{defn}

\begin{defn}[Espacio métrico]
	Un espacio métrico es un par formado por un conjunto y una métrica en él y se denota por (X,d)
\end{defn}


TODO buscar como se ponen los ejemplos

\textbf{Ejemplos}
\begin{enumerate}
	\item $ X = \real $, $d(x,y) = \abs{x-y} $ Podemos definir otras métricas en $ \real $
	\item $ d(x,y)= \frac{\abs{x-y}}{1+\abs{x-y}} \leq 1 $
	\item $ d(x,y)= \arctan{\abs{x-y}} \leq \frac{\pi}{2} $
\end{enumerate}



// TODO meter algo aquí para serpararlo y que quede mejor

Si $ \appl{\varphi}{ [ 0 , + \infty ) }{ [ 0 , + \infty ) } $ tiene las siguientes propiedades:

\begin{itemize}
	\item $ \varphi(0) = 0 $
	\item $ \varphi \nearrow$ debe ser estrictamente creciente
	\item $ \varphi(a+b) \leq \varphi(a) + \varphi(b) $ (subaditiva)
\end{itemize}

// TODO MIRAR COMO ES LA COMPOSICIÓN

y sea $ d $ una métrica en X, entonces $ \varphi \circ d $ es una métrica en X

\begin{itemize}
	\item $ \varphi(x,y)= \varphi (d(x,y)) $
	\item $ \appl{\varphi}{X \x X}{\real} $
\end{itemize}

%% Apendices (ejercicios, examenes)
\appendix

\chapter{---}
% -*- root: ../Topo2016.tex -*-


\printindex
\end{document}
