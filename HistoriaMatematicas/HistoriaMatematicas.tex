\documentclass{apuntes}

\title{HistoriaMatematicas}
\author{Pedro Valero}
\date{15/16 C1}

% Paquetes adicionales

% --------------------

\begin{document}
\pagestyle{plain}
\maketitle

\begin{abstract}
Este documento resume de modo esquemático las ideas vistas en la asignatura de Historia de las Matemáticas. Desde el punto de vista matemático, este resumen pretende ser completo aunque no desde el punto de vista histórico.

Aunque se realizarán algunas menciones a determinados detalles contextuales, se recomienda encarecidamente la lectura de los apuntes proporcionados por el profesor para una mejor comprensión del contenido histórico de la asignatura.
\end{abstract}

\tableofcontents
\newpage
% Contenido.

\chapter{La antigua Grecia}

\section{Pitágoras}
El teorema de Pitágoras, bien conocido por todos, llamó enormemente la atención de los griegos desde el momento de su descubrimiento.

Una vez planteada la identidad que reza el teorema hay dos cosas fundamentales que hacer a continuación:
\begin{enumerate}
\item Demostrar el enunciado del Teorema
\item Generar triángulos rectángulos que satisfagan el Teorema \textbf{con longitudes enteras}
\end{enumerate}


\begin{defn}[Ternas Pitagóricas]
Dada una tupla de tres números enteros $(a,b,c)$, diremos que es una \textbf{terna pitagórica}, es decir, $(a,b,c) \in \algb{T}$ si satisface:
\[a^2+b^2=c^2\]
\end{defn}

\subsection{Generación de ternas pitagóricas}
\subsubsection{Teorema de Euclides y Diofanto}
Para calcularlas podemos emplear el Teorema de Euclides y Diofanto que veremos a continuación:
\begin{theorem}[Teorema de Euclides y Diofanto (300-250 a.c.)]
Dada una terna $(a,b,c)$,
\[(a,b,c) \in \algb{T} \iff \left\{ \begin{array}{l} a = (p^2-q^2)\cdot r \\ b = 2\cdot p\cdot q\cdot r \\ c = (p^2+q^2)\cdot r\end{array}\right.\]

para algunos valores de $p,q,r \in \nat$.
\end{theorem}

\begin{proof}
Demostrar la implicación de derecha a izquierda es trivial, basta con operar.

Veamos cómo hicieron los griegos la demostración de la existencia de los $p,q,r$ que aparecen en el teorema. La demostración se basa en ir dando pequeños pasos que nos permiten acercarnos poco a poco al resultado deseado:

\begin{enumerate}
\item \textbf{Ternas primitivas}

Sea $k=m.c.d.(a,b,c)$, podemos escribir:
\[\begin{array}{l} a=ka' \\ b=kb' \\ c=kc' \end{array}  \implies K^2(a'^2+b'^2)=a^2+b^2 = c^2=k^2c'^2 \implies c'^2=a'^2+b'^2\]

Es decir, que podemos dividir todos los elementos de la terna entre $k$ y obtendríamos una nueva terna con la que trabajar. 

Por tanto, vamos a considerar sin pérdida de generalidad que $m.c.d.(a,b,c) = 1$. A las ternas que satisfagan esta condición las llamaremos \textbf{ternas primitivas}.

\item \textbf{Los elementos de una terna primitiva son coprimos dos a dos}

Vamos a demostrar, por reducción al absurdo que $(a,b,c) \in \algb{T} \y m.c.d.(a,b,c)=1 \implies m.c.d(a,b)=1$.
\begin{proof}
Sea $k=m.c.d.(a,b)$, podemos escribir:
\[\begin{array}{l} a=ka' \\ b=kb' \\ c=a^2+b^2 \implies k^2(a'^2+b'^2) \implies m.c.d.(a,b,c)=k \neq 1 \end{array}\]
\end{proof}

De forma equivalente podemos demostrar que dos elementos cualesquiera de una terna pitagórica primitiva son coprimos.

\item \textbf{Paridad de los elementos de la terna}

Sabiendo que se tiene que satisfacer la relación $a^2+b^2=c^2$ vamos a estudiar las diferentes posibilidades:

\begin{itemize}
\item \textbf{a,b pares}

Este caso no puede darse puesto que han de ser coprimos.

\item \textbf{a,b impares}

En este caso tendríamos:
\[a^2+b^2 \text{ par } \implies c^2 \text{ par } \implies c \text{ par}\]

Así tendríamos:
\[\begin{array}{l} a=2m+1 \\ b = 2n+1 \\ c^2 =(2m+1)^2+(2n+1)^2 = 4(n^2+m^2+n+m)+2\end{array} \]

y, puesto que $c$ es par, sabemos que $\frac{c^2}{2}$ también lo será, pero:
\[\frac{c^2}{2} = 2(n^2+m^2+n+m)+1 \text{, que es impar con lo que tendríamos una contradicción}\]

\item \textbf{a impar, b par}

En este caso tendríamos
\[a^2+b^2 \text{ impar } \implies c^2 \text{ impar } \implies c \text{ impar}\]

Podemos ver ahora que las fracciones 
\[\frac{c-a}{b}=\frac{p}{q} \text{ y } \frac{c+a}{b}=\frac{q}{p} \text{ no son irreducibles}\]

\obs Para ver que una fracción es la inversa de la otra (como se intuye al escribirlas como $p/q$ y $q/p$ respectivamente) basta con igualar una con la inversa de la otra y ver que el Teorema de Pitágoras garantiza la igualdad.

Si ahora sumamos y restamos ambas fracciones tenemos:
\[\frac{2c}{b}=\frac{p}{q}+\frac{q}{p} \iff \frac{c}{b}=\frac{p^2+q^2}{2pq}\]
\[\frac{2a}{b}=\frac{p}{q}-\frac{q}{p} \iff \frac{a}{b}=\frac{p^2-q^2}{2pq}\]

De aquí podemos deducir:
\[\left\{ \begin{array}{l} a=p^2-q^2 \\ b=2pq \\ c=p^2-q^2 \end{array} \right.\]

\end{itemize}
Y ya tenemos el resultado buscado, salvo por un detalle, el factor $r$. 

Ahora sólo tenemos que volver la vista al inicio de la demostración y comprobar que estábamos considerando ternas primitivas, para lo que sacamos factor común. Este factor común será la $r$ que necesitamos.
\end{enumerate}
\end{proof}

\subsubsection{Relación entre las ternas pitagóricas y cicunferencia unidad}

La idea se apya e que dado un punto de la circunferencia unidad, la ecuación que lo describe, $x^2+y^2=1$ se asemeja a otra que podemos obtener a partir del Teorema de Pitágoras:
\[a^2+b^2=c^2 \implies \left(\frac{a}{c}\right)^2 + \left(\frac{b}{c} \right)^2 = c^2\]

Por lo que parece razonable pensar que cada coordenada racional de la circunferencia unidad está asociada a una terna pitagórica. Así llegaron al siguiente teorema

\begin{theorem}
$(a,b,c)\in \algb{T} \iff \left( \frac{a}{c},\frac{b}{c}\right)$ es coordenada racional de la circunferencia unidad. 
\end{theorem}
\begin{proof}
La demostración de izquierda a derecha es trivial y se consigue simplemente haciendo cuentas.

Vamos hacer la demostración de derecha a izquierda.

Sea $(x,y)$ una coordenada racional de la circunferencia unidad, podemos escribirla como $(\frac{p}{q},\frac{r}{s})$ y sabemos que
\[\left( \frac{p}{q}\right)^2 + \left( \frac{r}{s}\right)^2 = 1 \iff (sp)^2+(qr)^2=(qs)^2\]
con lo que ya hemos obtenido una terna pitagórica: $(sp,qr,qs)\in \algb{T}$ 
\end{proof}

\subsubsection{Obtención de ternas pitagóricas a partir de la circunferencia unidad}

Ya hemos visto que todas las coordenadas racionales de la circunferencia unidad dan lugar a una terna pitagórica. La idea ahora es ver cómo podemos encontrar estas coordenadas racionales.

La idea es muy sencilla. Si tomo rectas que salgan del $(-1,0)$ con pendientes racionales, al intersecas estas rectas con la circunferencia, obtendremos coordenadas raciones.

La demostración es trivial y sólo requiere construir la recta con pendiente $\frac{p}{q}$ y comprobar que, efectivamente la intersección es un punto racional.

Se deja como ejercicio para el lector.

También es muy sencillo de comprobar que el recíproco es cierto, basándonos en la definición de la pendiente. Si tenemos dos puntos racionales en el plano, la pendiente será un número racional puesto que no hacemos más que restarl y dividir las coordenadas racionales.


\subsection{Demostraciones del Teorema de Pitágoras}
Los griegos mostraron tal fascinación por el Teorema de Pitágoras que idearon numerosas demostraciones del mismo. A continuación veremos algunas de ellas.

\subsubsection{Prueba 1}
Partiendo del siguiente diagrama:
\begin{center}
\includegraphics[width=0.8\linewidth/2]{img/pitagoras1.png}
\end{center}

Podemos comprobar sencillamente que el area del cuadrado menor es igual a la suma de las áreas de los triángulos y del cuadrado menor. 

Así tenemos:
\[(a+b)^2 = c^2 + \frac{ab}{2}\cdot 4 \iff a^2+b^2 +2ab = c^2 +2ab \iff a^2+b^2=c^2\]

\subsubsection{Prueba 2}
Partiendo del siguiente diagrama:
\begin{center}
\includegraphics[width=0.8\linewidth/2]{img/pitagoras1.png}
\end{center}

Podemos encontrar de nuevo una relación entre las áreas de los cuadrados y triángulos pintados:
\[a^2+b^2 +4\frac{ab}{2} = (a+b)^2 \iff a^2+b^2 +2ab = a^2+b^2+2ab \iff a^2+b^2=c^2\]

\subsection{Las distancias}
A raíz del Teorema de Pitágoras, el ejemplo tan trivial que supone un triángulo rectángulo con los dos catetos de longitud $1$ nos lleva a la aparición de nuevos números que no son racionales y que, ante la visión del mundo de los griegos, no son números.

Los griegos consideraban que todos los números eran racionales por lo que, al toparse con lo que hoy en día conocemos como irracionales (como es el caso de $\sqrt{2}$) decidieron denominarlo distancia o magnitud.

Así, los irracionales tenían un sentido geométrico pero no eran considerados como números.

\subsubsection{Demostración de la irracionalidad de $\sqrt{2}$}

La demostración puede hacerse por reducción al absurdo.

Supongamos que es racional, entonces existen $p,q \in \ent$, coprimos entre si, tales que:
\[\sqrt{2} = \frac{p}{q} \implies p^2=2q^2 \implies p^2 \text{ par } \implies p \text{ par } \implies p=2s \text{ con } s \in \nat\]

Así tenemos:
\[(2s)^2 = 2q^2 \implies 2s^2 = q^2 \implies q^2 \text{ par } \implies q \text{ par}\]

con lo que tenemos una contradicción, pues considerábamos que $p$ y $q$ eran coprimos por lo que no pueden ser pares.

\subsubsection{El número π}

Los griegos pronto se dieron cuenta de que no todos los números irracionales podían expresarse como distancias a partir del teorema de pitágoras puesto que no todos eran raiz de algo.

Este es el caso del número π, que pudieron aproximar basándose en las dos relaciones fundamentales de π que todos conocemos, relacionadas con el perímetro y el área de una circunferencia:
\[\left\{ \begin{array}{l} A = πr^2 \\ P =2πr\end{array}\right.\]

\section{Tales de Mileto}
Tales de Mileto era un filósofo que se planteó qué era la naturaleza y dejó tras de si una escuela formada por sucesores como Anaxágoras y Anaximandro, que se dedicaron a la especulación física.

Son numerosos los teoremas y proposiciones que se le atribuyen a tales, veamos algunos de ellos:
\begin{theorem}[Teoremas de Tales]
Los enunciados que se muestran a continuación dependen de los siguientes dibujos:

\begin{minipage}{0.57\textwidth}
\begin{center}
\includegraphics[width=\linewidth/2]{img/tales1.png}
\end{center}
\end{minipage}
\begin{minipage}{0.40\textwidth}
\begin{center}
\includegraphics[width=\linewidth/2]{img/tales2.png}
\end{center}
\end{minipage}

\begin{enumerate}
\item Dadas dos rectas paralelas cortadas por una secante (ver figura de la izquierda) siempre se cumple:
\[m=p, \;\; n=q, \;\; o=s, \;\; \tilde{n}=r\]

\item Los ángulos opuestos por el vértice son iguales. Es decir:
\[m=\tilde{n}, \;\; n=o\]

\item La suma de los ángulos de un triángulo es π

\obs Esta afirmación puede demostrarse a partir de las dos anteriores

\item Todo diámetro divide a la circunferencia en dos partes iguales

\item Dos lados/ángulos iguales en un triángulos implican dos ángulos/lados iguales repectivamente.

\item \textbf{Semejanza}. Si tenemos dos triángulos proporcionales, como se muestra en la figura de la derecha, se cumple que:
\[\frac{\overline{AD}}{\overline{AB}} = \frac{\overline{AE}}{\overline{AC}} =\frac{\overline{DE}}{\overline{BC}}\]
\end{enumerate}
\end{theorem}


\subsection{Area del triángulo}
Es bien conocido por todos que la fórmula para el área del triángulo dice:
\[A= \frac{\text{base}\cdot\text{altura}}{2}\]
lo importante son las diferentes demostraciones existentes de esta fórmula, que veremos a lo largo de esta sección.

\section{Euclides}
Euclides viió en torno al año 300 a.C. en Alejandría, aunque su existencia es dudosa, pues no hay pruebas de la existencia de ninguna persona cuya vida coincida con la conocida de Euclides.

Vivó en tiempos de Ptolomeo, sucesor de Alejandro Magno en Egipto. En esta época, Alejandría contaba con la mayor biblioteca de su tiempo.

La obra más importante de Euclides está constituida por 13 libros conocidos como \textbf{Los elementos}, que resumen prácticamente todo el saber de la época en cuanto a tres temas fundamentales:
\begin{itemize}
\item Geometría plana
\item Aritmética
\item Geometría en el espacio
\end{itemize}

Los griegos, en su afán por descubrir propiedades acerca de los números y desarrollar nuevas teorías comenzaron a trabajar con los números poligonales.

\subsection{Números poligonales}

\begin{defn}[Números figurados]
Los números figurados son aquellos número enteros formados por un conjunto de puntos equidistantes, formando una figura geométrica. Si la representación es un polígono regular se denominan \textbf{números poligonales}. Es el caso de los números triangulares, cuadrados o hexagonales.
\end{defn}

Los \textbf{números poligonales}\footnote{En concreto los griegos, y por tanto nosotros, trabajaban con los números poligonales cnetrados en el origen, que se corresponde con el punto rojo mostrado en los dibujos} pueden representarse gráficamente como:
\begin{center}
\includegraphics[width=\textwidth]{img/numeros_poligonales.png}
\end{center}

La idea de estos números, más bien de estos esquemas, es que nos permiten representar sucesiones de números que, en cierto modo, serán fáciles de estudiar.

\subsubsection{Números triangulares}

\subsubsection{Números pentagonales}

\chapter{Ejercicios}
\section{Ejercicios mandados en clase}
\begin{problem}[1]
Demostrar que cada uno de los ángulos de un triángulo equilátero es de $\frac{π}{3}$.
\solution
\end{problem}

\begin{problem}[2]
Demostrar que cada una semicircunferencia, cualquier punto de la misma nos sirve para construir un triángulo rectángulo (con el ángulo recto en el punto dado) que tiene como base el diámetro de la circunferencia.
\solution
\end{problem}

\begin{problem}[3]
Demostrar que dado el triángulo del ejercicio anterior, el ángulo formado entre base y el segmento que une el centro con el punto de la semicircunferencia es el doble del ángulo izquierdo del triángulo.
\solution
\end{problem}

%% Apéndices (ejercicios, exámenes)
% \appendix





% -*- root: ../HistoriaMatematicas.tex -*-
\section{Ejercicios mandados en clase}
\begin{problem}[1]
Demostrar que cada uno de los ángulos de un triángulo equilátero es de $\frac{π}{3}$.
\solution
\end{problem}

\begin{problem}[2]
Demostrar que cada una semicircunferencia, cualquier punto de la misma nos sirve para construir un triángulo rectángulo (con el ángulo recto en el punto dado) que tiene como base el diámetro de la circunferencia.
\solution
\end{problem}

\begin{problem}[3]
Demostrar que dado el triángulo del ejercicio anterior, el ángulo formado entre base y el segmento que une el centro con el punto de la semicircunferencia es el doble del ángulo izquierdo del triángulo.
\solution
\end{problem}

\printindex
\end{document}
