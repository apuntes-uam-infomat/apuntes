\documentclass{apuntes}

\title{HistoriaMatematicas}
\author{Pedro Valero}
\date{15/16 C1}

% Paquetes adicionales

% --------------------

\begin{document}
\pagestyle{plain}
\maketitle

\begin{abstract}
Este documento resume de modo esquemático las ideas vistas en la asignatura de Historia de las Matemáticas. Desde el punto de vista matemático, este resumen pretende ser completo aunque no desde el punto de vista histórico.

Aunque se realizarán algunas menciones a determinados detalles contextuales, se recomienda encarecidamente la lectura de los apuntes proporcionados por el profesor para una mejor comprensión del contenido histórico de la asignatura.
\end{abstract}

\tableofcontents
\newpage
% Contenido.

\chapter{La antigua Grecia}

\section{Pitágoras}
El teorema de Pitágoras, bien conocido por todos, llamó enormemente la atención de los griegos desde el momento de su descubrimiento.

Una vez planteada la identidad que reza el teorema hay dos cosas fundamentales que hacer a continuación:
\begin{enumerate}
\item Demostrar el enunciado del Teorema
\item Generar triángulos rectángulos que satisfagan el Teorema \textbf{con longitudes enteras}
\end{enumerate}


\begin{defn}[Ternas Pitagóricas]
Dada una tupla de tres números enteros $(a,b,c)$, diremos que es una \textbf{terna pitagórica}, es decir, $(a,b,c) \in \algb{T}$ si satisface:
\[a^2+b^2=c^2\]
\end{defn}

Para calcularlas podemos emplear el Teorema de Euclides y Diofanto que veremos a continuación:
\begin{theorem}[Teorema de Euclides y Diofanto (300-250 a.c.)]
Dada una terna $(a,b,c)$,
\[(a,b,c) \in \algb{T} \iff \left\{ \begin{array}{l} a = (p^2-q^2)\cdot r \\ b = 2\cdot p\cdot q\cdot r \\ c = (p^2+q^2)\cdot r\end{array}\right.\]

para algunos valores de $p,q,r \in \nat$.
\end{theorem}

\begin{proof}
Demostrar la implicación de derecha a izquierda es trivial, basta con operar.

Veamos cómo hicieron los griegos la demostración de la existencia de los $p,q,r$ que aparecen en el teorema. La demostración se basa en ir dando pequeños pasos que nos permiten acercarnos poco a poco al resultado deseado:

\begin{enumerate}
\item \textbf{Ternas primitivas}

Sea $k=m.c.d.(a,b,c)$, podemos escribir:
\[\begin{array}{l} a=ka' \\ b=kb' \\ c=kc' \end{array}  \implies K^2(a'^2+b'^2)=a^2+b^2 = c^2=k^2c'^2 \implies c'^2=a'^2+b'^2\]

Es decir, que podemos dividir todos los elementos de la terna entre $k$ y obtendríamos una nueva terna con la que trabajar. 

Por tanto, vamos a considerar sin pérdida de generalidad que $m.c.d.(a,b,c) = 1$. A las ternas que satisfagan esta condición las llamaremos \textbf{ternas primitivas}.

\item \textbf{Los elementos de una terna primitiva son coprimos dos a dos}

Vamos a demostrar, por reducción al absurdo que $(a,b,c) \in \algb{T} \y m.c.d.(a,b,c)=1 \implies m.c.d(a,b)=1$.
\begin{proof}
Sea $k=m.c.d.(a,b)$, podemos escribir:
\[\begin{array}{l} a=ka' \\ b=kb' \\ c=a^2+b^2 \implies k^2(a'^2+b'^2) \implies m.c.d.(a,b,c)=k \neq 1 \end{array}\]
\end{proof}

De forma equivalente podemos demostrar que dos elementos cualesquiera de una terna pitagórica primitiva son coprimos.

\item \textbf{Paridad de los elementos de la terna}

Sabiendo que se tiene que satisfacer la relación $a^2+b^2=c^2$ vamos a estudiar las diferentes posibilidades:

\begin{itemize}
\item \textbf{a,b pares}

Este caso no puede darse puesto que han de ser coprimos.

\item \textbf{a,b impares}

En este caso tendríamos:
\[a^2+b^2 \text{ par } \implies c^2 \text{ par } \implies c \text{ par}\]

Así tendríamos:
\[\begin{array}{l} a=2m+1 \\ b = 2n+1 \\ c^2 =(2m+1)^2+(2n+1)^2 = 4(n^2+m^2+n+m)+2\end{array} \]

y, puesto que $c$ es par, sabemos que $\frac{c^2}{2}$ también lo será, pero:
\[\frac{c^2}{2} = 2(n^2+m^2+n+m)+1 \text{, que es impar con lo que tendríamos una contradicción}\]

\item \textbf{a impar, b par}

En este caso tendríamos
\[a^2+b^2 \text{ impar } \implies c^2 \text{ impar } \implies c \text{ impar}\]
 
 TO BE CONTINUED...

\end{itemize}

\end{enumerate}


\end{proof}




%% Apéndices (ejercicios, exámenes)
% \appendix





% -*- root: ../HistoriaMatematicas.tex -*-
\section{Ejercicios mandados en clase}
\begin{problem}[1]
Demostrar que cada uno de los ángulos de un triángulo equilátero es de $\frac{π}{3}$.
\solution
\end{problem}

\begin{problem}[2]
Demostrar que cada una semicircunferencia, cualquier punto de la misma nos sirve para construir un triángulo rectángulo (con el ángulo recto en el punto dado) que tiene como base el diámetro de la circunferencia.
\solution
\end{problem}

\begin{problem}[3]
Demostrar que dado el triángulo del ejercicio anterior, el ángulo formado entre base y el segmento que une el centro con el punto de la semicircunferencia es el doble del ángulo izquierdo del triángulo.
\solution
\end{problem}

\printindex
\end{document}
