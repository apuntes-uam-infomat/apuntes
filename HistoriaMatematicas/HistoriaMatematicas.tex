\documentclass{apuntes}

\usepackage{tikztools}

\title{HistoriaMatematicas}
\author{Pedro Valero}
\date{15/16 C1}

% Paquetes adicionales

% --------------------

\begin{document}
\setcounter{tocdepth}{4}

\pagestyle{plain}
\maketitle

\begin{abstract}
Este documento resume de modo esquemático las ideas vistas en la asignatura de Historia de las Matemáticas. Desde el punto de vista matemático, este resumen pretende ser completo aunque no desde el punto de vista histórico.

Aunque se realizarán algunas menciones a determinados detalles contextuales, se recomienda encarecidamente la lectura de los apuntes proporcionados por el profesor para una mejor comprensión del contenido histórico de la asignatura.
\end{abstract}

\tableofcontents
\newpage
% Contenido.

\chapter{La antigua Grecia}

\section{Pitágoras}
El teorema de Pitágoras, bien conocido por todos, llamó enormemente la atención de los griegos desde el momento de su descubrimiento.

Una vez planteada la identidad que reza el teorema hay dos cosas fundamentales que hacer a continuación:
\begin{enumerate}
\item Demostrar el enunciado del Teorema
\item Generar triángulos rectángulos que satisfagan el Teorema \textbf{con longitudes enteras}
\end{enumerate}


\begin{defn}[Ternas Pitagóricas]
Dada una tupla de tres números enteros $(a,b,c)$, diremos que es una \textbf{terna pitagórica}, es decir, $(a,b,c) \in \algb{T}$ si satisface:
\[a^2+b^2=c^2\]
\end{defn}

\subsection{Generación de ternas pitagóricas}
\subsubsection{Teorema de Euclides y Diofanto}
Para calcularlas podemos emplear el Teorema de Euclides y Diofanto que veremos a continuación:
\begin{theorem}[Teorema de Euclides y Diofanto (300-250 a.c.)]
Dada una terna $(a,b,c)$,
\[(a,b,c) \in \algb{T} \iff \left\{ \begin{array}{l} a = (p^2-q^2)\cdot r \\ b = 2\cdot p\cdot q\cdot r \\ c = (p^2+q^2)\cdot r\end{array}\right.\]

para algunos valores de $p,q,r \in \nat$.
\end{theorem}

\begin{proof}
Demostrar la implicación de derecha a izquierda es trivial, basta con operar.

Veamos cómo hicieron los griegos la demostración de la existencia de los $p,q,r$ que aparecen en el teorema. La demostración se basa en ir dando pequeños pasos que nos permiten acercarnos poco a poco al resultado deseado:

\begin{enumerate}
\item \textbf{Ternas primitivas}

Sea $k=m.c.d.(a,b,c)$, podemos escribir:
\[\begin{array}{l} a=ka' \\ b=kb' \\ c=kc' \end{array}  \implies K^2(a'^2+b'^2)=a^2+b^2 = c^2=k^2c'^2 \implies c'^2=a'^2+b'^2\]

Es decir, que podemos dividir todos los elementos de la terna entre $k$ y obtendríamos una nueva terna con la que trabajar. 

Por tanto, vamos a considerar sin pérdida de generalidad que $m.c.d.(a,b,c) = 1$. A las ternas que satisfagan esta condición las llamaremos \textbf{ternas primitivas}.

\item \textbf{Los elementos de una terna primitiva son coprimos dos a dos}

Vamos a demostrar, por reducción al absurdo que $(a,b,c) \in \algb{T} \y m.c.d.(a,b,c)=1 \implies m.c.d(a,b)=1$.
\begin{proof}
Sea $k=m.c.d.(a,b)$, podemos escribir:
\[\begin{array}{l} a=ka' \\ b=kb' \\ c=a^2+b^2 \implies k^2(a'^2+b'^2) \implies m.c.d.(a,b,c)=k \neq 1 \end{array}\]
\end{proof}

De forma equivalente podemos demostrar que dos elementos cualesquiera de una terna pitagórica primitiva son coprimos.

\item \textbf{Paridad de los elementos de la terna}

Sabiendo que se tiene que satisfacer la relación $a^2+b^2=c^2$ vamos a estudiar las diferentes posibilidades:

\begin{itemize}
\item \textbf{a,b pares}

Este caso no puede darse puesto que han de ser coprimos.

\item \textbf{a,b impares}

En este caso tendríamos:
\[a^2+b^2 \text{ par } \implies c^2 \text{ par } \implies c \text{ par}\]

Así tendríamos:
\[\begin{array}{l} a=2m+1 \\ b = 2n+1 \\ c^2 =(2m+1)^2+(2n+1)^2 = 4(n^2+m^2+n+m)+2\end{array} \]

y, puesto que $c$ es par, sabemos que $\frac{c^2}{2}$ también lo será, pero:
\[\frac{c^2}{2} = 2(n^2+m^2+n+m)+1 \text{, que es impar con lo que tendríamos una contradicción}\]

\item \textbf{a impar, b par}

En este caso tendríamos
\[a^2+b^2 \text{ impar } \implies c^2 \text{ impar } \implies c \text{ impar}\]

Podemos ver ahora que las fracciones 
\[\frac{c-a}{b}=\frac{p}{q} \text{ y } \frac{c+a}{b}=\frac{q}{p} \text{ no son irreducibles}\]

\obs Para ver que una fracción es la inversa de la otra (como se intuye al escribirlas como $p/q$ y $q/p$ respectivamente) basta con igualar una con la inversa de la otra y ver que el Teorema de Pitágoras garantiza la igualdad.

Si ahora sumamos y restamos ambas fracciones tenemos:
\[\frac{2c}{b}=\frac{p}{q}+\frac{q}{p} \iff \frac{c}{b}=\frac{p^2+q^2}{2pq}\]
\[\frac{2a}{b}=\frac{p}{q}-\frac{q}{p} \iff \frac{a}{b}=\frac{p^2-q^2}{2pq}\]

De aquí podemos deducir:
\[\left\{ \begin{array}{l} a=p^2-q^2 \\ b=2pq \\ c=p^2-q^2 \end{array} \right.\]

\end{itemize}
Y ya tenemos el resultado buscado, salvo por un detalle, el factor $r$. 

Ahora sólo tenemos que volver la vista al inicio de la demostración y comprobar que estábamos considerando ternas primitivas, para lo que sacamos factor común. Este factor común será la $r$ que necesitamos.
\end{enumerate}
\end{proof}

\subsubsection{Relación entre las ternas pitagóricas y cicunferencia unidad}

La idea se apya e que dado un punto de la circunferencia unidad, la ecuación que lo describe, $x^2+y^2=1$ se asemeja a otra que podemos obtener a partir del Teorema de Pitágoras:
\[a^2+b^2=c^2 \implies \left(\frac{a}{c}\right)^2 + \left(\frac{b}{c} \right)^2 = c^2\]

Por lo que parece razonable pensar que cada coordenada racional de la circunferencia unidad está asociada a una terna pitagórica. Así llegaron al siguiente teorema

\begin{theorem}
$(a,b,c)\in \algb{T} \iff \left( \frac{a}{c},\frac{b}{c}\right)$ es coordenada racional de la circunferencia unidad. 
\end{theorem}
\begin{proof}
La demostración de izquierda a derecha es trivial y se consigue simplemente haciendo cuentas.

Vamos hacer la demostración de derecha a izquierda.

Sea $(x,y)$ una coordenada racional de la circunferencia unidad, podemos escribirla como $(\frac{p}{q},\frac{r}{s})$ y sabemos que
\[\left( \frac{p}{q}\right)^2 + \left( \frac{r}{s}\right)^2 = 1 \iff (sp)^2+(qr)^2=(qs)^2\]
con lo que ya hemos obtenido una terna pitagórica: $(sp,qr,qs)\in \algb{T}$ 
\end{proof}

\subsubsection{Obtención de ternas pitagóricas a partir de la circunferencia unidad}

Ya hemos visto que todas las coordenadas racionales de la circunferencia unidad dan lugar a una terna pitagórica. La idea ahora es ver cómo podemos encontrar estas coordenadas racionales.

La idea es muy sencilla. Si tomo rectas que salgan del $(-1,0)$ con pendientes racionales, al intersecas estas rectas con la circunferencia, obtendremos coordenadas raciones.

La demostración es trivial y sólo requiere construir la recta con pendiente $\frac{p}{q}$ y comprobar que, efectivamente la intersección es un punto racional.

Se deja como ejercicio para el lector.

También es muy sencillo de comprobar que el recíproco es cierto, basándonos en la definición de la pendiente. Si tenemos dos puntos racionales en el plano, la pendiente será un número racional puesto que no hacemos más que restarl y dividir las coordenadas racionales.


\subsection{Demostraciones del Teorema de Pitágoras}
Los griegos mostraron tal fascinación por el Teorema de Pitágoras que idearon numerosas demostraciones del mismo. A continuación veremos algunas de ellas.

\subsubsection{Prueba 1}
Partiendo del siguiente diagrama:
\begin{center}
\includegraphics[width=0.8\linewidth/2]{img/pitagoras1.png}
\end{center}

Podemos comprobar sencillamente que el area del cuadrado menor es igual a la suma de las áreas de los triángulos y del cuadrado menor. 

Así tenemos:
\[(a+b)^2 = c^2 + \frac{ab}{2}\cdot 4 \iff a^2+b^2 +2ab = c^2 +2ab \iff a^2+b^2=c^2\]

\subsubsection{Prueba 2}
Partiendo del siguiente diagrama:
\begin{center}
\includegraphics[width=0.8\linewidth/2]{img/pitagoras1.png}
\end{center}

Podemos encontrar de nuevo una relación entre las áreas de los cuadrados y triángulos pintados:
\[a^2+b^2 +4\frac{ab}{2} = (a+b)^2 \iff a^2+b^2 +2ab = a^2+b^2+2ab \iff a^2+b^2=c^2\]

\subsubsection{Prueba 3 (Euclides)}
Partiendo del siguiente diagrama:
\begin{center}
\inputtikz{pitagoras_prueba_3}
\end{center}

Podemos observar que las dos líneas son azules, por triangulación.

TO BE CONTINUED...


\subsection{Las distancias}
A raíz del Teorema de Pitágoras, el ejemplo tan trivial que supone un triángulo rectángulo con los dos catetos de longitud $1$ nos lleva a la aparición de nuevos números que no son racionales y que, ante la visión del mundo de los griegos, no son números.

Los griegos consideraban que todos los números eran racionales por lo que, al toparse con lo que hoy en día conocemos como irracionales (como es el caso de $\sqrt{2}$) decidieron denominarlo distancia o magnitud.

Así, los irracionales tenían un sentido geométrico pero no eran considerados como números.

\subsubsection{Demostración de la irracionalidad de $\sqrt{2}$}

La demostración puede hacerse por reducción al absurdo.

Supongamos que es racional, entonces existen $p,q \in \ent$, coprimos entre si, tales que:
\[\sqrt{2} = \frac{p}{q} \implies p^2=2q^2 \implies p^2 \text{ par } \implies p \text{ par } \implies p=2s \text{ con } s \in \nat\]

Así tenemos:
\[(2s)^2 = 2q^2 \implies 2s^2 = q^2 \implies q^2 \text{ par } \implies q \text{ par}\]

con lo que tenemos una contradicción, pues considerábamos que $p$ y $q$ eran coprimos por lo que no pueden ser pares.

\section{Tales de Mileto}
Tales de Mileto era un filósofo que se planteó qué era la naturaleza y dejó tras de si una escuela formada por sucesores como Anaxágoras y Anaximandro, que se dedicaron a la especulación física.

Son numerosos los teoremas y proposiciones que se le atribuyen a tales, veamos algunos de ellos:
\begin{theorem}[Teoremas de Tales]
Los enunciados que se muestran a continuación dependen de los siguientes dibujos:

\begin{minipage}{0.57\textwidth}
\begin{center}
\includegraphics[width=\linewidth/2]{img/tales1.png}
\end{center}
\end{minipage}
\begin{minipage}{0.40\textwidth}
\begin{center}
\includegraphics[width=\linewidth/2]{img/tales2.png}
\end{center}
\end{minipage}

\begin{enumerate}
\item Dadas dos rectas paralelas cortadas por una secante (ver figura de la izquierda) siempre se cumple:
\[m=p, \;\; n=q, \;\; o=s, \;\; \tilde{n}=r\]

\item Los ángulos opuestos por el vértice son iguales. Es decir:
\[m=\tilde{n}, \;\; n=o\]

\item La suma de los ángulos de un triángulo es π

\obs Esta afirmación puede demostrarse a partir de las dos anteriores

\item Todo diámetro divide a la circunferencia en dos partes iguales

\item Dos lados/ángulos iguales en un triángulos implican dos ángulos/lados iguales repectivamente.

\item \textbf{Semejanza}. Si tenemos dos triángulos proporcionales, como se muestra en la figura de la derecha, se cumple que:
\[\frac{\overline{AD}}{\overline{AB}} = \frac{\overline{AE}}{\overline{AC}} =\frac{\overline{DE}}{\overline{BC}}\]
\end{enumerate}
\end{theorem}


\subsection{Area del triángulo}
Es bien conocido por todos que la fórmula para el área del triángulo dice:
\[A= \frac{\text{base}\cdot\text{altura}}{2}\]
lo importante son las diferentes demostraciones existentes de esta fórmula, que veremos a lo largo de esta sección.

\begin{lemma}
El área de un triángulo rectángulo es proporcional al lado y la constante depende del ángulo
\end{lemma}
\begin{proof}
Dado el siguiente dibujo
\begin{center}
\inputtikz{thales_triangulo}
\end{center}
Sabemos, por el \textbf{teorema de proporcionalidad}, que:
\[\frac{a'}{a} = \frac{b'}{c} = \frac{b'}{c} = λ\]
Atendiendo a las áreas de los triángulos representados tenemos:
\[\left\{ \begin{array}{l} S= \frac{1}{2}ac \\ S' = \frac{1}{2} a'c'\end{array}\right. \implies \frac{S'}{S} =\frac{a'c'}{ac} = λ^2\]

Por otro lado tenemos
\[S' = S\left( \frac{b'}{b}\right)^2 \iff \frac{S'}{S} = \frac{b'^2}{b^2} \implies \left\{ \begin{array}{l} S' = rb'^2 \\ S= r b^2 \end{array}\right.\]
donde 
\[r=\frac{1}{2} \sin(α)\cos(α)\]
\end{proof}

\section{Euclides}
Euclides viió en torno al año 300 a.C. en Alejandría, aunque su existencia es dudosa, pues no hay pruebas de la existencia de ninguna persona cuya vida coincida con la conocida de Euclides.

Vivó en tiempos de Ptolomeo, sucesor de Alejandro Magno en Egipto. En esta época, Alejandría contaba con la mayor biblioteca de su tiempo.

La obra más importante de Euclides está constituida por 13 libros conocidos como \textbf{Los elementos}, que resumen prácticamente todo el saber de la época en cuanto a tres temas fundamentales:
\begin{itemize}
\item Geometría plana
\item Aritmética
\item Geometría en el espacio
\end{itemize}

Los griegos, en su afán por descubrir propiedades acerca de los números y desarrollar nuevas teorías comenzaron a trabajar con los números poligonales.

\subsection{Números poligonales}

\begin{defn}[Números figurados]
Los números figurados son aquellos número enteros formados por un conjunto de puntos equidistantes, formando una figura geométrica. Si la representación es un polígono regular se denominan \textbf{números poligonales}. Es el caso de los números triangulares, cuadrados o hexagonales.
\end{defn}

Los \textbf{números poligonales}\footnote{En concreto los griegos, y por tanto nosotros, trabajaban con los números poligonales cnetrados en el origen, que se corresponde con el punto rojo mostrado en los dibujos} pueden representarse gráficamente como:
\begin{center}
\includegraphics[width=\textwidth]{img/numeros_poligonales.png}
\end{center}

La idea de estos números, más bien de estos esquemas, es que nos permiten representar sucesiones de números que, en cierto modo, serán fáciles de estudiar. 


\subsubsection{Números triangulares}
En el caso de los números triangulares, podemos ver que si contamos los puntos que añadimos en cada nivel al trabajar con los números triangulares tenemos que el número será:
\[S=\sum_{i=1}^ni\]
y tenemos una forma muy sencilla de calcular este número.

Suponemos un cuadrado de dimensión $n$. El total de puntos del cuadrado será $n^2$. Ahora tenemos en cuenta que para nuestro triángulo sólo estamos utilizando la mitad del cuadrado por lo que tendremos $\frac{n^2}{2}$ puntos. Pero al quitar la mitad de puntos estamos quitando toda la parte superior de la diagonal (lo cual es correcto) y la mitad de la diagonal, que debemos recuperar. En total tenemos:
\[S=\frac{n^2}{2}+\frac{n}{2} = \frac{n(n+1)}{2}\]

\subsubsection{Números pentagonales}
 De forma equivalente a lo que hicimos con los números triangulares, podemos trabajar con los números pentagonales obteniendo:
\[S = 1+4+7+10 + ... = \sum_{i=1}^n1+3(i-1) = \frac{n(3n-1)}{2}\]

La demostración de esta fórmula queda como ejercicio para el lector.

\subsection{Números perfectos}
\begin{defn}{Números perfectos}
Un número perfecto es aquel que es igual a la suma de sus divisores.

Son ejemplos de números perfectos el $6=1+2+3$, $28=1+2+4+7+14$.
\end{defn}

Euler llegó a comporbar que los cuatro primeros números perfectos vienen dados por la fórmula: $2^{n-1}(2^n-1)$.

Aunque los griegos dieron mucha importancia a estos números, pues siempre estaban buscando la perfección y relaciones nuevas, a día de hoy no tienen importancia alguna y han quedado sólo como un dato anecdótico.

\subsection{Teoría de divisores}
Euclies fue el primero en dejar constancia de su estudio sobre los números primos en su libro \textbf{Los Elementos}. En el libro prueba diversos teoremas sobre ellos, define los conceptos de máximo común divisor y mínimo común múltiplo y define el algoritmo de Euclides, aún empleado hoy dia.

Uno de los resultados más importantes es el siguiente teorema
\begin{theorem}[Teorema de Euclides]
Existe \textbf{infinitos} números primos
\end{theorem}
\begin{proof}
Consideramos una serie de números primos distintos ordenados de menor a mayor (no tienen por que ser consecutivos): $p1<p_2<p_2<...<p_n$

A partir de estos números calculamos $p_{n+1}$ como el producto de todos más 1, es decir:
\[p_{n+1}=p_1\cdot p_2 \cdot p_3 ... \cdot p_n+1\]

Una vez tenemos calculado $p_{n+1}$ tenemos dos opciones:
\begin{itemize}
\item \textbf{Es primo}

En este caso acabamos de encontrar un nuevo número primo.

\item \textbf{No es primo}

En este caso tendrá factores: $p_{n+1}=q_1\cdot q_2 ... \cdot q_k$, pero podemos observar que:
\[\frac{p_{n+1}}{p_i}=m_i \text{ con resto } r= 1\]
Sin embargo
\[\frac{p_{n+1}}{q_i}=l_i \text{ con resto } r= 0\]

Por tanto es claro que uno de esos $q_i \neq p_j \ \forall i,j$ será un primo nuevo.

\end{itemize}

Es decir, dados $n$ primos siempre podemos encontrar un número primo nuevo. Por tanto queda claro que hay infinitos números primos.

\end{proof}

No obstante, podemos ver que, aunque haya infinitos números primos, la densidad de los mismos va disminuyendo. Así podemos representar la siguiente funciones.
\[ρ(n)=\frac{\#\{q \ : \ 1 < q < n, \ q \text{ es primo }\}}{n}\]
\begin{center}
\includegraphics[width=0.8\linewidth]{img/density.png}
\end{center}

\subsubsection{Criba de Erastótenes}
\begin{defn}[Criba de Erastótenes]
La criba de Eratóstenes es un algoritmo que permite hallar todos los números primos menores que un número natural dado n. Se forma una tabla con todos los números naturales comprendidos entre 2 y n, y se van tachando los números que no son primos de la siguiente manera: Comenzando por el 2, se tachan todos sus múltiplos; comenzando de nuevo, cuando se encuentra un número entero que no ha sido tachado, ese número es declarado primo, y se procede a tachar todos sus múltiplos, así sucesivamente. El proceso termina cuando el cuadrado del mayor número confirmado como primo es mayor que n.
\end{defn}

\subsubsection{Máximo común divisor}
Euler nos proporcionó un algoritmo para calcular le máximo común divisor de dos números que aún hoy en día sigue empleándose.

Este método se basa en el siguiente lemma:
\begin{lemma}
Un número $c$ es divisor común a dos números $a>b$ si y sólo si $c$ es divisor común de $a$ y de $r=a\mod b$ 
\end{lemma}
\begin{proof}
Trabajando sobre una cuadrícula, dibujamos un rectángulo de $a$ cuadros de longitud por $b$ de altura.

Una vez tenemos el rectángulo, quitamos del mismo tantos \textbf{cuadrados} de lado $b$ como sea posible. 

Tras esto nos quedará un rectángulo de altura $b$ y base $r$.

\begin{center}
\inputtikz{divisor_comun}
\end{center}

Si tuviésemos un número $c$ que divide a $a$ y a $b$, podríamos haber cubierto todo el rectángulo inicial con cuadrados de tamaño $c$ y, además, podríamos haber cubierto los cuadrados de lado $b$ con cuadrados de tamaño $c$. 

Por tanto, la región sobrante $r\times b$ también se podría cubrir por cuadrados de tamaño $c$ y, por tanto, $c$ tiene que dividir a $r$ también.
\end{proof}

Una vez tenemos claro este lema, el algoritmo de Euclides para calcular el máximo común divisor de dos números consiste en iterar sobre el lema reduciendo el problema a algo cada vez más sencillo.

La idea consiste en que, dados dos números $a>b$ tenemos:
\[c.d.(a,b)=c.d.(b,r_1)=c.d.(r_1,r_2)...\]
\begin{example}
Los divisores comunes de 75 y 28 pueden calcularse como:
\[c.d.(75,28)=c.d.(28,19) = c.d.(19,9)=c.d.(9,1) = c.d.(1,0)\]

Paramos en el momento en que el resto sea 0 con lo que obtenemos que los divisores comunes que estamos buscando son los divisores de 0 y de otro número $c$, es decir, los divisores comunes serán todos los divisores de $c$.
\end{example}

\obs El número $c$ que obtenemos, tal que todos sus divisores son los divisores comunes de $a$ y $b$, es el máximo común divisor de estos números.

Así, acabamos de encontrar un algoritmo para calcular el máximo común divisor entre dos números, conocido como \concept{Algoritmo de Euclides}

Podemos ver el \textbf{algoritmo} trabajando con $a>b>1 \in \nat^+$:
\begin{verbatim}
a = q*b + r;

Mientras r != 0"
    Si r = 0:
        mcd(a,b) = b;
    Si r > 0:
        a = b;
        b = r;
        a = q*b + r;
\end{verbatim}

\subsubsection{Factorizando irracionales}

Veamos qué ocurre si empezamos a trabajar con los números irracionales.

Si dibujamos un cuadrado de base $\sqrt{2}$ y de altura 1, podemos ver que dentro sólo cabe un cuadrado de lado 1 y nos queda un rectángulo de dimensiones $(\sqrt{2}-1)\times 1$.

Puesto que el divisor común que buscamos debe dividir a $\sqrt{2}-1$, por el lemma anterior, entonces podremos dibujar un rectángulo con este valor como lado, que podrá rellenarse con cuadrados de lado $c = c.d.(\sqrt{2},1)$.

Este rectángulo puede dividirse en un cuadrado de lado $\sqrt{2}-1$ y un rectángulo de dimensiones: $(\sqrt{2}-1) \times (1-(\sqrt{2}-1) ) = (\sqrt{2}-1) \times (2 - \sqrt{2})$ que podemos ver que es proporcional al rectángulo con el que empezamos y que debería poder rellenarse con cuadrados de lado $c$.

Por tanto este procedimiento nunca acabará, ya que siempre obtenemos un subrectángulo proporcional al inicial de modo que nunca encontraremos $c$.

Esto supone \textbf{una prueba más de la irracionalidad de $\sqrt{2}$}.

\subsubsection{Métodos de aproximación}
El intento de factorización anterior, si bien no nos aportó información nueva a priori, si que nos ha proporcionado un método de aproximación de irracionales, como $\sqrt{2}$.

Si consideramos el mismo dibujo anterior, con $x = \sqrt{2}$ el lado inferior del rectángulo y $y=x+1$, lo que resta de base al quitar un cuadrado unidad del rectángulo inicial, vemos que obtenemos la iteración:

\[x = 1+y \implies x = 1 + \frac{1}{2+y} = 1 + \frac{1}{1 + \frac{1}{2+y}} = ... \]

Esto nos da un método de aproximación de $x=\sqrt{2}$, para lo que basta tomar un $y$ cualquiera (podemos tomar 0 por comodidad) y tomar la fracción que deseemos. 

Estas fracciones que se prolongan infinitamente son conocidas como \concept{fracciones continuas}

\section{Arquímedes}
\textbf{Arquímedes de Siracusa}, nacido en torno al año 287 a.C. fue un físico, ingeniería|ingeniero, inventor, astrónomo y matemática helénica|matemático griego. Aunque se conocen pocos detalles de su vida, es considerado uno de los científicos más importantes de la Antigüedad clásica. 

Entre sus avances en física se encuentran sus fundamentos en hidrostática, estática (mecánica)|estática y la explicación del principio de la palanca. Es reconocido por haber diseñado innovadoras máquinas, incluyendo arma de asedio|armas de asedio y el tornillo de Arquímedes, que lleva su nombre. 

Experimentos modernos han probado las afirmaciones de que Arquímedes llegó a diseñar máquinas capaces de sacar barcos enemigos del agua o prenderles fuego utilizando una serie de espejos. 


\subsection{Cálculo de π}
Arquímedes desarrolló un procedimiento para aproximar el número π con precisión de tantos decimales como deseemos. No obstante, la mala notación matemática de los griegos hacía los cálculos muy complicados con lo que, a pesar de tener un algoritmo, no llegó a calcular demasiados decimales de π.


La idea consiste en aproximar el perímetro y el área de una circunferencia por medio de polígonos inscritos y circunscritos en la misma.

Vamos a partir de una circunferencia de radio 1 y vamos a considerar los hexágonos inscritos y circunscritos en la misma.

\begin{center}
\includegraphics[width=0.3\linewidth]{img/hexagonos.png}
\end{center}

\obs Empleamos el hexágono y no otra figura geométrica por la particularidad de que el radio es igual al lado, lo que nos facilita los cálculos.

\obs En los cálculos que siguen emplearemos letras mayúsculas para referirnos al hexágono mayor (el circunscrito) y minúsculas para el menor (el inscrito).

\obs A lo largo de todo este procedimiento nos apoyaremos en el Teorema de Pitágoras, construyendo triángulos apoyados en medio lado del polígono y con hipotenusa igual al radio.

Podemos escribir las siguientes relaciones, donde $p$ representa el semiperímetro\footnote{Empleamos el semiperímetro en lugar del perímetro puesto que queremos aproximar π directamente, no su doble}:
\[\left\{ \begin{array}{l} 
p_6 = 6x_6 = 6 \frac{1}{2} = 3\\
X_6^2+H_6^2=(2X_6)^2 \implies X_6 = \frac{1}{\sqrt{3}}\\
P_6 = 6X_6 = \frac{6}{\sqrt{3}} = 2\sqrt{3}
\end{array}\right.\]

Podemos tratar de establecer una relación entre los lados del polígono inscrito y el circunscrito.

\[\frac{X_6}{x_6} = 2\sqrt{3} = \frac{H_6}{h_6} = \frac{1}{\sqrt{1-x_6^2}}\]

De forma general, deducimos la fórmula:
\[X_n=\frac{x_n}{\sqrt{1-x_n^2}}\]

Ahora deberíamos tomar un polígono con más lados de forma que tengamos mayor precisión. Lo más cómodo será considerar un dodecaedro.

Empezamos buscando la relación entre la longitud del lado del hexágono y del dodecaedro de forma general. 

\[\left\{ \begin{array}{l}
x_n^2 + (1-h_n)^2 = (2x_{2n})^2 \\
x_n^2+(1-\sqrt{1-x_n^2})^2 = 4x_{2n}^2\\
x_n^2+1+1-x_n^2 - 2 \sqrt{1-x_n^2} = 4x_{2n}^2\\
\end{array}\right. \implies 2x_{2n}^2 = \frac{x_n^2}{1+\sqrt{1-x_n^2}}\]

Ahora tomamos la relación entre $x_n$ y $X_n$ y escribimos:
\[X_{2n} = \frac{x_{2n}}{\sqrt{1-x_{2n}^2}} = \frac{x_n}{1+\sqrt{1-x_n^2}}\]

Una vez tenemos escritos los lados, podemos calcular los semiperímetros.
\[p_n = n x_n\]
\[P_n =nX_n = n \frac{x_n}{1-x_n^2}\]
\[p_{2n}=2nx_{2n} = n\left(\frac{1}{2} \frac{x_n^2}{1+\sqrt{1-x_n}}\right)^{\frac{1}{2}}\]
\[P_{2n} = 2nX_{2n} = 2n \frac{x_n}{1+\sqrt{1-x_n}}\]

Jugando con estas ecuaciones podemos llegar a:
\[P_{2n}=\frac{2P_n\cdot p_n}{P_n + p_n}, \ \ \ p_{2n}=\sqrt{p_nP_{2n}}\]

A partir de estas relaciones podemos iterar, comenzando con unos hexágonos cuyos perímetros sabemos calcular, avanzando hacia una cota cada vez más aproximada de π.

Un problema importante que tenían los griegos al llegar a este punto es que no conocían π, por lo que no era sencillo saber cómo de buena era su aproximación.

Una posible froma de acotar el error consiste en tomar como aproximación la media aritmética de $p$ y $P$, con lo que sabemos que el error cometido será, como mucho, igual a $\frac{P-p}{2}$.

%% Apéndices (ejercicios, exámenes)
% \appendix





\chapter{Ejercicios}
% -*- root: ../HistoriaMatematicas.tex -*-
\section{Ejercicios mandados en clase}
\begin{problem}[1]
Demostrar que cada uno de los ángulos de un triángulo equilátero es de $\frac{π}{3}$.
\solution
\end{problem}

\begin{problem}[2]
Demostrar que cada una semicircunferencia, cualquier punto de la misma nos sirve para construir un triángulo rectángulo (con el ángulo recto en el punto dado) que tiene como base el diámetro de la circunferencia.
\solution
\end{problem}

\begin{problem}[3]
Demostrar que dado el triángulo del ejercicio anterior, el ángulo formado entre base y el segmento que une el centro con el punto de la semicircunferencia es el doble del ángulo izquierdo del triángulo.
\solution
\end{problem}

\printindex
\end{document}
