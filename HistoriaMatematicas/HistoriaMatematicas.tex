\documentclass{apuntes}

\usepackage{tikztools}
\usetikzlibrary{shapes.geometric}
\usetikzlibrary{intersections}
\usepackage{tkz-euclide}
\usetkzobj{all}

\title{Historia de las Matemáticas}
\author{Pedro Valero}
\date{15/16 C1}

% Paquetes adicionales

% --------------------

\begin{document}
\setcounter{tocdepth}{3}

\pagestyle{plain}

\begin{abstract}
Este documento resume de modo esquemático las ideas vistas en la asignatura de Historia de las Matemáticas. Desde el punto de vista matemático, este resumen pretende ser completo aunque no desde el punto de vista histórico.

Aunque se realizarán algunas menciones a determinados detalles contextuales, se recomienda encarecidamente la lectura de los apuntes proporcionados por el profesor para una mejor comprensión del contenido histórico de la asignatura.
\end{abstract}

\maketitle

\tableofcontents
\newpage
% Contenido.

\chapter{La antigua Grecia}

\section{Pitágoras}
El teorema de Pitágoras, bien conocido por todos, llamó enormemente la atención de los griegos desde el momento de su descubrimiento.

Una vez planteada la identidad que reza el teorema hay dos cosas fundamentales que hacer a continuación:
\begin{enumerate}
\item Demostrar el enunciado del Teorema
\item Generar triángulos rectángulos que satisfagan el Teorema \textbf{con longitudes enteras}
\end{enumerate}


\begin{defn}[Ternas Pitagóricas]
Dada una tupla de tres números enteros $(a,b,c)$, diremos que es una \textbf{terna pitagórica}, es decir, $(a,b,c) \in \algb{T}$ si satisface:
\[a^2+b^2=c^2\]
\end{defn}

\subsection{Generación de ternas pitagóricas}
\subsubsection{Teorema de Euclides y Diofanto}
Para calcularlas podemos emplear el Teorema de Euclides y Diofanto que veremos a continuación:
\begin{theorem}[Teorema de Euclides y Diofanto (300-250 a.c.)]
Dada una terna $(a,b,c)$,
\[(a,b,c) \in \algb{T} \iff \left\{ \begin{array}{l} a = (p^2-q^2)\cdot r \\ b = 2\cdot p\cdot q\cdot r \\ c = (p^2+q^2)\cdot r\end{array}\right.\]

para algunos valores de $p,q,r \in \nat$.
\end{theorem}

\begin{proof}
Demostrar la implicación de derecha a izquierda es trivial, basta con operar.

Veamos cómo hicieron los griegos la demostración de la existencia de los $p,q,r$ que aparecen en el teorema. La demostración se basa en ir dando pequeños pasos que nos permiten acercarnos poco a poco al resultado deseado:

\begin{enumerate}
\item \textbf{Ternas primitivas}

Sea $k=m.c.d.(a,b,c)$, podemos escribir:
\[\begin{array}{l} a=ka' \\ b=kb' \\ c=kc' \end{array}  \implies K^2(a'^2+b'^2)=a^2+b^2 = c^2=k^2c'^2 \implies c'^2=a'^2+b'^2\]

Es decir, que podemos dividir todos los elementos de la terna entre $k$ y obtendríamos una nueva terna con la que trabajar.

Por tanto, vamos a considerar sin pérdida de generalidad que $m.c.d.(a,b,c) = 1$. A las ternas que satisfagan esta condición las llamaremos \textbf{ternas primitivas}.

\item \textbf{Los elementos de una terna primitiva son coprimos dos a dos}

Vamos a demostrar, por reducción al absurdo que $(a,b,c) \in \algb{T} \y m.c.d.(a,b,c)=1 \implies m.c.d(a,b)=1$.
\begin{proof}
Sea $k=m.c.d.(a,b)$, podemos escribir:
\[\begin{array}{l} a=ka' \\ b=kb' \\ c=a^2+b^2 \implies k^2(a'^2+b'^2) \implies m.c.d.(a,b,c)=k \neq 1 \end{array}\]
\end{proof}

De forma equivalente podemos demostrar que dos elementos cualesquiera de una terna pitagórica primitiva son coprimos.

\item \textbf{Paridad de los elementos de la terna}

Sabiendo que se tiene que satisfacer la relación $a^2+b^2=c^2$ vamos a estudiar las diferentes posibilidades:

\begin{itemize}
\item \textbf{a,b pares}

Este caso no puede darse puesto que han de ser coprimos.

\item \textbf{a,b impares}

En este caso tendríamos:
\[a^2+b^2 \text{ par } \implies c^2 \text{ par } \implies c \text{ par}\]

Así tendríamos:
\[\begin{array}{l} a=2m+1 \\ b = 2n+1 \\ c^2 =(2m+1)^2+(2n+1)^2 = 4(n^2+m^2+n+m)+2\end{array} \]

y, puesto que $c$ es par, sabemos que $\frac{c^2}{2}$ también lo será, pero:
\[\frac{c^2}{2} = 2(n^2+m^2+n+m)+1 \text{, que es impar con lo que tendríamos una contradicción}\]

\item \textbf{a impar, b par}

En este caso tendríamos
\[a^2+b^2 \text{ impar } \implies c^2 \text{ impar } \implies c \text{ impar}\]

Podemos ver ahora que las fracciones
\[\frac{c-a}{b}=\frac{p}{q} \text{ y } \frac{c+a}{b}=\frac{q}{p} \text{ no son irreducibles}\]

\obs Para ver que una fracción es la inversa de la otra (como se intuye al escribirlas como $p/q$ y $q/p$ respectivamente) basta con igualar una con la inversa de la otra y ver que el Teorema de Pitágoras garantiza la igualdad.

Si ahora sumamos y restamos ambas fracciones tenemos:
\[\frac{2c}{b}=\frac{p}{q}+\frac{q}{p} \iff \frac{c}{b}=\frac{p^2+q^2}{2pq}\]
\[\frac{2a}{b}=\frac{p}{q}-\frac{q}{p} \iff \frac{a}{b}=\frac{p^2-q^2}{2pq}\]

De aquí podemos deducir:
\[\left\{ \begin{array}{l} a=p^2-q^2 \\ b=2pq \\ c=p^2-q^2 \end{array} \right.\]

\end{itemize}
Y ya tenemos el resultado buscado, salvo por un detalle, el factor $r$.

Ahora sólo tenemos que volver la vista al inicio de la demostración y comprobar que estábamos considerando ternas primitivas, para lo que sacamos factor común. Este factor común será la $r$ que necesitamos.
\end{enumerate}
\end{proof}

\subsubsection{Relación entre las ternas pitagóricas y cicunferencia unidad}

La idea se apya e que dado un punto de la circunferencia unidad, la ecuación que lo describe, $x^2+y^2=1$ se asemeja a otra que podemos obtener a partir del Teorema de Pitágoras:
\[a^2+b^2=c^2 \implies \left(\frac{a}{c}\right)^2 + \left(\frac{b}{c} \right)^2 = c^2\]

Por lo que parece razonable pensar que cada coordenada racional de la circunferencia unidad está asociada a una terna pitagórica. Así llegaron al siguiente teorema

\begin{theorem}
$(a,b,c)\in \algb{T} \iff \left( \frac{a}{c},\frac{b}{c}\right)$ es coordenada racional de la circunferencia unidad.
\end{theorem}
\begin{proof}
La demostración de izquierda a derecha es trivial y se consigue simplemente haciendo cuentas.

Vamos hacer la demostración de derecha a izquierda.

Sea $(x,y)$ una coordenada racional de la circunferencia unidad, podemos escribirla como $(\frac{p}{q},\frac{r}{s})$ y sabemos que
\[\left( \frac{p}{q}\right)^2 + \left( \frac{r}{s}\right)^2 = 1 \iff (sp)^2+(qr)^2=(qs)^2\]
con lo que ya hemos obtenido una terna pitagórica: $(sp,qr,qs)\in \algb{T}$
\end{proof}

\subsubsection{Obtención de ternas pitagóricas a partir de la circunferencia unidad}

Ya hemos visto que todas las coordenadas racionales de la circunferencia unidad dan lugar a una terna pitagórica. La idea ahora es ver cómo podemos encontrar estas coordenadas racionales.

La idea es muy sencilla. Si tomo rectas que salgan del $(-1,0)$ con pendientes racionales, al intersecas estas rectas con la circunferencia, obtendremos coordenadas raciones.

La demostración es trivial y sólo requiere construir la recta con pendiente $\frac{p}{q}$ y comprobar que, efectivamente la intersección es un punto racional.

Se deja como ejercicio para el lector.

También es muy sencillo de comprobar que el recíproco es cierto, basándonos en la definición de la pendiente. Si tenemos dos puntos racionales en el plano, la pendiente será un número racional puesto que no hacemos más que restarl y dividir las coordenadas racionales.


\subsection{Demostraciones del Teorema de Pitágoras}
Los griegos mostraron tal fascinación por el Teorema de Pitágoras que idearon numerosas demostraciones del mismo. A continuación veremos algunas de ellas.

\subsubsection{Prueba 1}\label{sec:Prueba1}
Partiendo del siguiente diagrama:
\begin{center}
\includegraphics[width=0.8\linewidth/2]{img/pitagoras1.png}
\end{center}

Podemos comprobar sencillamente que el area del cuadrado menor es igual a la suma de las áreas de los triángulos y del cuadrado menor.

Así tenemos:
\[(a+b)^2 = c^2 + \frac{ab}{2}\cdot 4 \iff a^2+b^2 +2ab = c^2 +2ab \iff a^2+b^2=c^2\]

\subsubsection{Prueba 2}\label{sec:Prueba2}
Partiendo del siguiente diagrama:
\begin{center}
\includegraphics[width=0.8\linewidth/2]{img/pitagoras1.png}
\end{center}

Podemos encontrar de nuevo una relación entre las áreas de los cuadrados y triángulos pintados:
\[a^2+b^2 +4\frac{ab}{2} = (a+b)^2 \iff a^2+b^2 +2ab = a^2+b^2+2ab \iff a^2+b^2=c^2\]

\subsubsection{Prueba 3 (Euclides)}\label{sec:Prueba3}
Partiendo del siguiente diagrama:
\begin{center}
\inputtikz{pitagoras_prueba_3}
\end{center}

Podemos observar que las dos líneas son azules, por triangulación.

TO BE CONTINUED...


\subsection{Las distancias}
A raíz del Teorema de Pitágoras, el ejemplo tan trivial que supone un triángulo rectángulo con los dos catetos de longitud $1$ nos lleva a la aparición de nuevos números que no son racionales y que, ante la visión del mundo de los griegos, no son números.

Los griegos consideraban que todos los números eran racionales por lo que, al toparse con lo que hoy en día conocemos como irracionales (como es el caso de $\sqrt{2}$) decidieron denominarlo distancia o magnitud.

Así, los irracionales tenían un sentido geométrico pero no eran considerados como números.

\subsubsection{Demostración de la irracionalidad de $\sqrt{2}$}

La demostración puede hacerse por reducción al absurdo.

Supongamos que es racional, entonces existen $p,q \in \ent$, coprimos entre si, tales que:
\[\sqrt{2} = \frac{p}{q} \implies p^2=2q^2 \implies p^2 \text{ par } \implies p \text{ par } \implies p=2s \text{ con } s \in \nat\]

Así tenemos:
\[(2s)^2 = 2q^2 \implies 2s^2 = q^2 \implies q^2 \text{ par } \implies q \text{ par}\]

con lo que tenemos una contradicción, pues considerábamos que $p$ y $q$ eran coprimos por lo que no pueden ser pares.

\section{Tales de Mileto}
Tales de Mileto era un filósofo que se planteó qué era la naturaleza y dejó tras de si una escuela formada por sucesores como Anaxágoras y Anaximandro, que se dedicaron a la especulación física.

Son numerosos los teoremas y proposiciones que se le atribuyen a tales, veamos algunos de ellos:
\begin{theorem}[Teoremas de Tales]
Los enunciados que se muestran a continuación dependen de los siguientes dibujos:

\begin{minipage}{0.57\textwidth}
\begin{center}
\includegraphics[width=\linewidth/2]{img/tales1.png}
\end{center}
\end{minipage}
\begin{minipage}{0.40\textwidth}
\begin{center}
\includegraphics[width=\linewidth/2]{img/tales2.png}
\end{center}
\end{minipage}

\begin{enumerate}
\item Dadas dos rectas paralelas cortadas por una secante (ver figura de la izquierda) siempre se cumple:
\[m=p, \;\; n=q, \;\; o=s, \;\; \tilde{n}=r\]

\item Los ángulos opuestos por el vértice son iguales. Es decir:
\[m=\tilde{n}, \;\; n=o\]

\item La suma de los ángulos de un triángulo es π

\obs Esta afirmación puede demostrarse a partir de las dos anteriores

\item Todo diámetro divide a la circunferencia en dos partes iguales

\item Dos lados/ángulos iguales en un triángulos implican dos ángulos/lados iguales repectivamente.

\item \textbf{Semejanza}. Si tenemos dos triángulos proporcionales, como se muestra en la figura de la derecha, se cumple que:
\[\frac{\overline{AD}}{\overline{AB}} = \frac{\overline{AE}}{\overline{AC}} =\frac{\overline{DE}}{\overline{BC}}\]
\end{enumerate}
\end{theorem}


\subsection{Area del triángulo}
Es bien conocido por todos que la fórmula para el área del triángulo dice:
\[A= \frac{\text{base}\cdot\text{altura}}{2}\]
lo importante son las diferentes demostraciones existentes de esta fórmula, que veremos a lo largo de esta sección.

\begin{lemma}
El área de un triángulo rectángulo es proporcional al lado y la constante depende del ángulo
\end{lemma}
\begin{proof}
Dado el siguiente dibujo
\begin{center}
\inputtikz{thales_triangulo}
\end{center}
Sabemos, por el \textbf{teorema de proporcionalidad}, que:
\[\frac{a'}{a} = \frac{b'}{c} = \frac{b'}{c} = λ\]
Atendiendo a las áreas de los triángulos representados tenemos:
\[\left\{ \begin{array}{l} S= \frac{1}{2}ac \\ S' = \frac{1}{2} a'c'\end{array}\right. \implies \frac{S'}{S} =\frac{a'c'}{ac} = λ^2\]

Por otro lado tenemos
\[S' = S\left( \frac{b'}{b}\right)^2 \iff \frac{S'}{S} = \frac{b'^2}{b^2} \implies \left\{ \begin{array}{l} S' = rb'^2 \\ S= r b^2 \end{array}\right.\]
donde
\[r=\frac{1}{2} \sin(α)\cos(α)\]
\end{proof}

\section{Euclides}
Euclides viió en torno al año 300 a.C. en Alejandría, aunque su existencia es dudosa, pues no hay pruebas de la existencia de ninguna persona cuya vida coincida con la conocida de Euclides.

Vivó en tiempos de Ptolomeo, sucesor de Alejandro Magno en Egipto. En esta época, Alejandría contaba con la mayor biblioteca de su tiempo.

La obra más importante de Euclides está constituida por 13 libros conocidos como \textbf{Los elementos}, que resumen prácticamente todo el saber de la época en cuanto a tres temas fundamentales:
\begin{itemize}
\item Geometría plana
\item Aritmética
\item Geometría en el espacio
\end{itemize}

Los griegos, en su afán por descubrir propiedades acerca de los números y desarrollar nuevas teorías comenzaron a trabajar con los números poligonales.

\subsection{Números poligonales}

\begin{defn}[Números poligonales]
Los números poligonales son aquellos número enteros formados por un conjunto de puntos equidistantes, formando una figura geométrica. Si la representación es un polígono regular se denominan \textbf{números poligonales}. Es el caso de los números triangulares, cuadrados o hexagonales.
\end{defn}

Los \textbf{números poligonales}\footnote{En concreto los griegos, y por tanto nosotros, trabajaban con los números poligonales centrados en el origen, que se corresponde con el punto rojo mostrado en los dibujos} pueden representarse gráficamente como:
\begin{center}
\includegraphics[width=\textwidth]{img/numeros_poligonales.png}
\end{center}

La idea de estos números, más bien de estos esquemas, es que nos permiten representar sucesiones de números que, en cierto modo, serán fáciles de estudiar.


\subsubsection{Números triangulares}
En el caso de los números triangulares, podemos ver que si contamos los puntos que añadimos en cada nivel al trabajar con los números triangulares tenemos que el número será:
\[S=\sum_{i=1}^ni\]
y tenemos una forma muy sencilla de calcular este número.

Suponemos un cuadrado de dimensión $n$. El total de puntos del cuadrado será $n^2$. Ahora tenemos en cuenta que para nuestro triángulo sólo estamos utilizando la mitad del cuadrado por lo que tendremos $\frac{n^2}{2}$ puntos. Pero al quitar la mitad de puntos estamos quitando toda la parte superior de la diagonal (lo cual es correcto) y la mitad de la diagonal, que debemos recuperar. En total tenemos:
\[S=\frac{n^2}{2}+\frac{n}{2} = \frac{n(n+1)}{2}\]

\subsubsection{Números pentagonales}
 De forma equivalente a lo que hicimos con los números triangulares, podemos trabajar con los números pentagonales obteniendo:
\[S = 1+4+7+10 + ... = \sum_{i=1}^n1+3(i-1) = \frac{n(3n-1)}{2}\]

La demostración de esta fórmula queda como ejercicio para el lector.

\subsection{Números perfectos}
\begin{defn}{Números perfectos}
Un número perfecto es aquel que es igual a la suma de sus divisores.

Son ejemplos de números perfectos el $6=1+2+3$, $28=1+2+4+7+14$.
\end{defn}

Euler llegó a comporbar que los cuatro primeros números perfectos vienen dados por la fórmula: $2^{n-1}(2^n-1)$.

Aunque los griegos dieron mucha importancia a estos números, pues siempre estaban buscando la perfección y relaciones nuevas, a día de hoy no tienen importancia alguna y han quedado sólo como un dato anecdótico.

\subsection{Teoría de divisores}
Euclies fue el primero en dejar constancia de su estudio sobre los números primos en su libro \textbf{Los Elementos}. En el libro prueba diversos teoremas sobre ellos, define los conceptos de máximo común divisor y mínimo común múltiplo y define el algoritmo de Euclides, aún empleado hoy dia.

Uno de los resultados más importantes es el siguiente teorema
\begin{theorem}[Teorema de Euclides]\label{theorem:infinitud_primos}
Existen \textbf{infinitos} números primos
\end{theorem}
\begin{proof}
Consideramos una serie de números primos distintos ordenados de menor a mayor (no tienen por que ser consecutivos): $p1<p_2<p_2<...<p_n$

A partir de estos números calculamos $p_{n+1}$ como el producto de todos más 1, es decir:
\[p_{n+1}=p_1\cdot p_2 \cdot p_3 ... \cdot p_n+1\]

Una vez tenemos calculado $p_{n+1}$ tenemos dos opciones:
\begin{itemize}
\item \textbf{Es primo}

En este caso acabamos de encontrar un nuevo número primo.

\item \textbf{No es primo}

En este caso tendrá factores: $p_{n+1}=q_1\cdot q_2 ... \cdot q_k$, pero podemos observar que:
\[\frac{p_{n+1}}{p_i}=m_i \text{ con resto } r= 1\]
Sin embargo
\[\frac{p_{n+1}}{q_i}=l_i \text{ con resto } r= 0\]

Por tanto es claro que uno de esos $q_i \neq p_j \ \forall i,j$ será un primo nuevo.

\end{itemize}

Es decir, dados $n$ primos siempre podemos encontrar un número primo nuevo. Por tanto queda claro que hay infinitos números primos.

\end{proof}

No obstante, podemos ver que, aunque haya infinitos números primos, la densidad de los mismos va disminuyendo. Así podemos representar la siguiente funciones.
\[ρ(n)=\frac{\#\{q \ : \ 1 < q < n, \ q \text{ es primo }\}}{n}\]
\begin{center}
\includegraphics[width=0.8\linewidth]{img/density.png}
\end{center}

\subsubsection{Criba de Erastótenes}
\begin{defn}[Criba de Erastótenes]
La criba de Eratóstenes es un algoritmo que permite hallar todos los números primos menores que un número natural dado n. Se forma una tabla con todos los números naturales comprendidos entre 2 y n, y se van tachando los números que no son primos de la siguiente manera: Comenzando por el 2, se tachan todos sus múltiplos; comenzando de nuevo, cuando se encuentra un número entero que no ha sido tachado, ese número es declarado primo, y se procede a tachar todos sus múltiplos, así sucesivamente. El proceso termina cuando el cuadrado del mayor número confirmado como primo es mayor que n.
\end{defn}

\subsubsection{Máximo común divisor}\label{sec:maximo_comun_divisor}
Euler nos proporcionó un algoritmo para calcular le máximo común divisor de dos números que aún hoy en día sigue empleándose.

Este método se basa en el siguiente lemma:
\begin{lemma}
Un número $c$ es divisor común a dos números $a>b$ si y sólo si $c$ es divisor común de $a$ y de $r=a\mod b$
\end{lemma}
\begin{proof}
Trabajando sobre una cuadrícula, dibujamos un rectángulo de $a$ cuadros de longitud por $b$ de altura.

Una vez tenemos el rectángulo, quitamos del mismo tantos \textbf{cuadrados} de lado $b$ como sea posible.

Tras esto nos quedará un rectángulo de altura $b$ y base $r$.

\begin{center}
\inputtikz{divisor_comun}
\end{center}

Si tuviésemos un número $c$ que divide a $a$ y a $b$, podríamos haber cubierto todo el rectángulo inicial con cuadrados de tamaño $c$ y, además, podríamos haber cubierto los cuadrados de lado $b$ con cuadrados de tamaño $c$.

Por tanto, la región sobrante $r\times b$ también se podría cubrir por cuadrados de tamaño $c$ y, por tanto, $c$ tiene que dividir a $r$ también.
\end{proof}

Una vez tenemos claro este lema, el algoritmo de Euclides para calcular el máximo común divisor de dos números consiste en iterar sobre el lema reduciendo el problema a algo cada vez más sencillo.

La idea consiste en que, dados dos números $a>b$ tenemos:
\[c.d.(a,b)=c.d.(b,r_1)=c.d.(r_1,r_2)...\]
\begin{example}
Los divisores comunes de 75 y 28 pueden calcularse como:
\[c.d.(75,28)=c.d.(28,19) = c.d.(19,9)=c.d.(9,1) = c.d.(1,0)\]

Paramos en el momento en que el resto sea 0 con lo que obtenemos que los divisores comunes que estamos buscando son los divisores de 0 y de otro número $c$, es decir, los divisores comunes serán todos los divisores de $c$.
\end{example}

\obs El número $c$ que obtenemos, tal que todos sus divisores son los divisores comunes de $a$ y $b$, es el máximo común divisor de estos números.

Así, acabamos de encontrar un algoritmo para calcular el máximo común divisor entre dos números, conocido como \concept{Algoritmo de Euclides}

Podemos ver el \textbf{algoritmo} trabajando con $a>b>1 \in \nat^+$:
\begin{verbatim}
a = q*b + r;

Mientras r != 0"
    Si r = 0:
        mcd(a,b) = b;
    Si r > 0:
        a = b;
        b = r;
        a = q*b + r;
\end{verbatim}

\subsubsection{Factorizando irracionales}

Veamos qué ocurre si empezamos a trabajar con los números irracionales.

Si dibujamos un cuadrado de base $\sqrt{2}$ y de altura 1, podemos ver que dentro sólo cabe un cuadrado de lado 1 y nos queda un rectángulo de dimensiones $(\sqrt{2}-1)\times 1$.

Puesto que el divisor común que buscamos debe dividir a $\sqrt{2}-1$, por el lemma anterior, entonces podremos dibujar un rectángulo con este valor como lado, que podrá rellenarse con cuadrados de lado $c = c.d.(\sqrt{2},1)$.

Este rectángulo puede dividirse en un cuadrado de lado $\sqrt{2}-1$ y un rectángulo de dimensiones: $(\sqrt{2}-1) \times (1-(\sqrt{2}-1) ) = (\sqrt{2}-1) \times (2 - \sqrt{2})$ que podemos ver que es proporcional al rectángulo con el que empezamos y que debería poder rellenarse con cuadrados de lado $c$.

Por tanto este procedimiento nunca acabará, ya que siempre obtenemos un subrectángulo proporcional al inicial de modo que nunca encontraremos $c$.

Esto supone \textbf{una prueba más de la irracionalidad de $\sqrt{2}$}.

\subsubsection{Fracciones continuas}
El intento de factorización anterior, si bien no nos aportó información nueva a priori, si que nos ha proporcionado un método de aproximación de irracionales, como $\sqrt{2}$.

Si consideramos el mismo dibujo anterior, con $x = \sqrt{2}$ el lado inferior del rectángulo y $y=x+1$, lo que resta de base al quitar un cuadrado unidad del rectángulo inicial, vemos que obtenemos la iteración:

\[x = 1+y \implies x = 1 + \frac{1}{2+y} = 1 + \frac{1}{1 + \frac{1}{2+y}} = ... \]

Esto nos da un método de aproximación de $x=\sqrt{2}$, para lo que basta tomar un $y$ cualquiera (podemos tomar 0 por comodidad) y tomar la fracción que deseemos.

Estas fracciones que se prolongan infinitamente son conocidas como \concept{fracciones continuas}

\begin{example}
Supongamos que queremos encontrar la fracción continua que nos lleva a $\sqrt{3}$.

Con los métodos actuales es sencillo pues la calculadora nos da $\sqrt{3} = 1 +0.7320$ y a partir de aquí podemos iterar:
\[\sqrt{3}=1+0.7320 = 1 +\frac{1}{1 + 0.366025} = 1 + \frac{1}{1+ \frac{1}{2+0.7320}}+ ...\]

Puesto que el último decimal que hemos obtenido coincide con el que empezamos, podemos ver que la fracción continua que representa este número sería:
\[\sqrt{3} = [1,1,2,1,1,2,...]\]

Aunque estamos trabajando con una aproximación de $\sqrt{3}$, nos basta para ver que no obtenemos una fracción continua finita y que, por tanto, no es racional.

\obs Los números calculados en el desarrollo de fracciones se han calculado arrastrando tantos decimales como nos permitió la calculadora del profesor. Si trabajásemos únicamente con los números que se ven, estaríamos trabajando con $1.7320$ que, evidentemente, si que puede expresarse como un número racional.
\end{example}

Pero este ejemplo no termina de ser del todo claro, pues estamos trabajando con aproximaciones y con la calculadora, cosa que no termina de encajar con la realidad griega. Sin embargo, el siguiente ejemplo nos muestra cómo se pudo demostrar la irracionalidad de $\sqrt{3}$ en el mundo griego.

\begin{example}
\[x^2=3=1+2 \iff x^2-1 = 2 \iff (x+1)(x-1) = 2 \iff x=1+\frac{2}{x+1}\]

Renombrando $x-1=y$ tenemos
\[y = \frac{2}{2+y} = \frac{1}{\frac{2+y}{2}} =\frac{1}{1+\frac{y}{2}} = \frac{1}{1+\frac{1}{2+y}}\]

Como hemos vuelto a obtener $2+y$, que es con lo que empezamos, vemos que este procedimiento nunca termina.

\[\sqrt{3}=[1,1,2,1,2,1,2...]\]

\end{example}

De forma un poco más general podemos ver que si queremos expresar $x=\sqrt{n^2+1}$ como una fracción continua podemos escribir:
\[x^2=n^2+1 \iff x^2-n^2 = 1 \iff (x-n)(x+n) = 1 \iff \]
\[\iff x = n + \frac{1}{n+x}=n+\frac{1}{n+n+\frac{1}{n+x}} = n + \frac{1}{2n + \frac{1}{x+n}}\]

A partir de aquí podemos ver que la iteración nos lleva a:
\[\sqrt{n^2+1} = [n;\overline{2n}]\]

De la misma forma podemos ver que:
\[\sqrt{n^2+2} = [n,\overline{n,2n}]\]
\[\sqrt{n^2+n} = [n,\overline{2,2n}]\]
\[\sqrt{n^2+2n+1} = [n+1, 0]\]

Hay un teorema que fue demostrado por Lagrange y que dice:
\begin{theorem}
Toda raíz cuadrada de un número entero puede expresarse como función continua de período finito.
\end{theorem}

\subsubsection{Teorema del número primo}
\begin{theorem}[Teorema del número primo]\label{theorem:numero_primo}
Todo número entero positivo es un producto finito de números primos. La expresión $n=p_1\cdot p_2 ... \cdot p_n$ es única
\end{theorem}

\begin{proof}
Vamos a dividir la demostración en dos partes

\begin{enumerate}
\item \textbf{Demostración de existencia de la factorización}

En esta parte queremos probar que:
\[\forall n , \exists p_1,p_2,...,p_n \tq n=p_1\cdot p_2 \cdot ... \cdot p_n\]

La forma de demostrarlo es mediante un procedimiento de descenso basado en el siguiente algoritmo:
\begin{verbatim}
1. Intento escribir n como producto de dos números
    (obviamente menores que n)
2. Si no puedo hacerlo es que n es primo
3. Mientras sea posible, tomo un factor y repito el paso 1 con él.
\end{verbatim}

Este procedimiento funciona y fué el empleado por los griegos pero, si pretendemos escribirlo de manera formal con el lenguaje actual, resulta demasiado extensa.

Veamos otra forma de demostralo más cómoda.

\textbf{Demostración de Euclides}

Supongamos que el conjunto $S=\{n \ t.q.$ no es primo ni puede escribirse como producto de primos$\}$ no es vacío

Tomo el elemento menor de este conjunto (puesto que el conjunto de los naturales está acotado inferiormente por el 0, es seguro que existirá este mínimo.

Este mínimo será un número no primo y compuesto, por lo que podrá escribirse como $n=a\cdot b $ siendo $a,b < n$ por lo que $a$ y $b$ serán primos o podrán escribirse como producto de primos (puesto que están fuera del conjunto $S$) con lo que ya tenemos la factorización.

\item \textbf{Demostración de la unicidad de la factorización}

Supongamos que $n=p_1\cdot...\cdot p_n=q_1\cdot ... \cdot q_n$.

En este caso, $p_1|n$ y por tanto $p_1|q_1\cdot ... \cdot q_n$ lo que implica que se cumple una de las siguientes dos opciones:
\begin{enumerate}
\item $p_1=q_1$

En este caso hemos terminado y vamos con $p_2$

\item $p_1\neq q_1$

Puesto que ambos son primos, entonces $p_1 \nmid q_1 \implies p_1 | \overbrace{q_2\cdot .... \cdot q_n}^r)$ y podemos repetir este procedimiento teniendo ahora una cadena más corta.
\end{enumerate}

Al final del procedimiento habremos emparejado cada $p_i$ con un $p_j$ en un tiempo finito puesto que con cada $p_i$ el algoritmo a seguir es finito (necesitamos $n$ iteraciones) y tendremos un máximo $n$ valores de $p_i$.

\begin{mdframed}
\textbf{Salto histórico}

La valided del segundo paso del procedimiento puede probarse también mediante el Lema de Bezout (cuya demostración se deja como ejercicio para el lector).

Este lemma nos dice que si $m.c.d(p,q)=1$ entonces $\exists n,m\in \ent $ tales que $mp+nq =1$

Si tomamos $n=q_1r$ tendremos:
\[(q_1,p_1) = 1 \ \implies \ mp_1+nq_1 = 1 \implies mp_1r +nq_1r = r\]
Puesto que $p_1$ divide a los dos sumandos de la izquierda, es obvio que también divide a $r$.
\end{mdframed}
\end{enumerate}
\end{proof}

\section{Arquímedes}
\textbf{Arquímedes de Siracusa}, nacido en torno al año 287 a.C. fue un físico, ingeniería|ingeniero, inventor, astrónomo y matemática helénica|matemático griego. Aunque se conocen pocos detalles de su vida, es considerado uno de los científicos más importantes de la Antigüedad clásica.

Entre sus avances en física se encuentran sus fundamentos en hidrostática, estática (mecánica)|estática y la explicación del principio de la palanca. Es reconocido por haber diseñado innovadoras máquinas, incluyendo arma de asedio|armas de asedio y el tornillo de Arquímedes, que lleva su nombre.

Experimentos modernos han probado las afirmaciones de que Arquímedes llegó a diseñar máquinas capaces de sacar barcos enemigos del agua o prenderles fuego utilizando una serie de espejos.


\subsection{Cálculo de π}
\subsubsection{Aproximación por polígonos}
Arquímedes desarrolló un procedimiento para aproximar el número π con precisión de tantos decimales como deseemos. No obstante, la mala notación matemática de los griegos hacía los cálculos muy complicados con lo que, a pesar de tener un algoritmo, no llegó a calcular demasiados decimales de π.


La idea consiste en aproximar el perímetro y el área de una circunferencia por medio de polígonos inscritos y circunscritos en la misma.

Vamos a partir de una circunferencia de radio 1 y vamos a considerar los hexágonos inscritos y circunscritos en la misma como muestra la siguiente figura:


\begin{minipage}{0.47\textwidth}
\begin{center}
\inputtikz{hexagonos_ins_circuns_crito}
\end{center}
\end{minipage}
\begin{minipage}{0.52\textwidth}
\obs Empleamos el hexágono y no otra figura geométrica por la particularidad de que el radio es igual al lado, lo que nos facilita los cálculos.

\obs A lo largo de todo este procedimiento nos apoyaremos en el Teorema de Pitágoras, construyendo triángulos apoyados en medio lado del polígono y con hipotenusa igual al radio.
\end{minipage}


Podemos escribir las siguientes relaciones, donde $p$ representa el semiperímetro\footnote{Empleamos el semiperímetro en lugar del perímetro puesto que queremos aproximar π directamente, no su doble}:
\[\left\{ \begin{array}{l}
p_6 = 6x_6 = 6 \frac{1}{2} = 3\\
X_6^2+H_6^2=(2X_6)^2 \implies X_6 = \frac{1}{\sqrt{3}}\\
P_6 = 6X_6 = \frac{6}{\sqrt{3}} = 2\sqrt{3}
\end{array}\right.\]

Podemos tratar de establecer una relación entre los lados del polígono inscrito y el circunscrito.

\[\frac{X_6}{x_6} = 2\sqrt{3} = \frac{H_6}{h_6} = \frac{1}{\sqrt{1-x_6^2}}\]

De forma general, deducimos la fórmula:
\[X_n=\frac{x_n}{\sqrt{1-x_n^2}}\]

Ahora deberíamos tomar un polígono con más lados de forma que tengamos mayor precisión. Lo más cómodo será considerar un dodecaedro.

Empezamos buscando la relación entre la longitud del lado del hexágono y del dodecaedro de forma general.

\[\left\{ \begin{array}{l}
x_n^2 + (1-h_n)^2 = (2x_{2n})^2 \\
x_n^2+(1-\sqrt{1-x_n^2})^2 = 4x_{2n}^2\\
x_n^2+1+1-x_n^2 - 2 \sqrt{1-x_n^2} = 4x_{2n}^2\\
\end{array}\right. \implies 2x_{2n}^2 = \frac{x_n^2}{1+\sqrt{1-x_n^2}}\]

Ahora tomamos la relación entre $x_n$ y $X_n$ y escribimos:
\[X_{2n} = \frac{x_{2n}}{\sqrt{1-x_{2n}^2}} = \frac{x_n}{1+\sqrt{1-x_n^2}}\]

Una vez tenemos escritos los lados, podemos calcular los semiperímetros.
\[p_n = n x_n\]
\[P_n =nX_n = n \frac{x_n}{1-x_n^2}\]
\[p_{2n}=2nx_{2n} = n\left(\frac{1}{2} \frac{x_n^2}{1+\sqrt{1-x_n}}\right)^{\frac{1}{2}}\]
\[P_{2n} = 2nX_{2n} = 2n \frac{x_n}{1+\sqrt{1-x_n}}\]

Jugando con estas ecuaciones podemos llegar a:
\[P_{2n}=\frac{2P_n\cdot p_n}{P_n + p_n}, \ \ \ p_{2n}=\sqrt{p_nP_{2n}}\]

A partir de estas relaciones podemos iterar, comenzando con unos hexágonos cuyos perímetros sabemos calcular, avanzando hacia una cota cada vez más aproximada de π.

Un problema importante que tenían los griegos al llegar a este punto es que no conocían π, por lo que no era sencillo saber cómo de buena era su aproximación.

Una posible froma de acotar el error consiste en tomar como aproximación la media aritmética de $p$ y $P$, con lo que sabemos que el error cometido será, como mucho, igual a $\frac{P-p}{2}$.

\subsubsection{Aproximación trigonométrica (Taylor)}
Nos trasladamos repentinamente al año 1700 donde pasan a conocerse las Series de Taylor.

Vamos a reducirnos a una pequeña parte del dibujo anterior:

\begin{minipage}{0.47\textwidth}
\begin{center}
\inputtikz{angle_pi}
\end{center}
\end{minipage}
\begin{minipage}{0.52\textwidth}
\[
x_n=\sin\left(\frac{π}{n}\right) \implies p_n=n\sin\left(\frac{π}{n}\right) = n\left(\frac{π}{2}-\frac{ε}{3!}\right)\]
\[X_n = \tg\left(\frac{π}{n}\right) \implies P_n = n\tg\left(\frac{π}{n}\right)=π-\frac{nε^3}{3!}\]
con $ε\in \left(0, \frac{π}{2}\right)$, el término de error de la serie de Taylor
\end{minipage}


Con esta aproximación tenemos que el error se ve acotado como sigue:
\[e_n = |p_n-π| \leq \max_ε \frac{n|ε|^3}{3!} \text{ con } ε \in \left( 0, \frac{π}{2} \right) = \frac{π^3}{6n^2}\]

Pero esto nos da otro problema y es que estamos consiguiendo acotad π pero la cota depende de π. Bueno, no es un gran problema puesto que podemos sustituir π por el un valor conocido mayor que él, como puede ser $P_6$ que ya calcularon los griegos. Así nos queda:
\[e_n \leq \frac{π^3}{6n^2} \leq \frac{2^33^{3/2}}{6n^2}=\frac{4\sqrt{3}}{n^2}\]

Con esto ya hemos rducido el error. Con la aproximación de los griegos teníanos un error lineal, $O\left( \frac{1}{n}\right)$. Sin embargo, con nuestra nueva aproximación llegamos a un error $O\left( \frac{1}{n^2}\right)$.

Vamos a intentar hacer nuestro error aún más pequeño.

Si nos fijamos en el desarrollo de Taylor del seno y la tangente tenemos:
\[\sin(x)=x-\frac{x^3}{3!}+\frac{x^5}{5!}+...\]
\[\tan(x)=x+\frac{x^3}{3!}-\frac{x^5}{5!}+...\]

Vamos a buscar la forma de combinarlas linealmente de modo que el resultado conserve el elemento $x$ y caiga el $x^3$. Para ello basta con tomar
\[\frac{2}{3}\sin(x)+\frac{1}{3}\tan(x)\]

Así podemos encontrar una nueva aproximación de π:
\[Q_n = \frac{2p_n+P_n}{3} = π + \frac{nε^5}{5!} \implies |Q_n-π| \leq c\frac{1}{n^4}=O\left( \frac{1}{n^4}\right)\]
donde en la última desigualdad nos hemos vuelto a basar en que $ε\in\left( 0, \frac{π}{2} \right)$

Este método aún puede mejorarse, puesto que podemos combinar $p_n,P_n$ y $Q_n$ para obtener una nueva aproximación que tampoco tenga término $x^5$ y así sucesivamente.

\chapter{Ecuaciones de Pell}

\section{Origen de las ecuaciones de Pell}
Unos de los temas estudiados por los griegos que más llamó la atención fue el intento de racionalizar las raíces, en parte por que aún no se había llegado a comprender bien el concepto de los irracionales.

Si tratamos de aproximar $\sqrt{2}$ como una fracción vemos que:
\[\frac{x}{y} = \sqrt{2} \iff \frac{x^2}{y^2} = 2 \iff ε \approx \frac{x^2}{y^2} - 2 = \frac{x^2-2y^2}{y^2}\]

puesto que pretendemos que el error, ε, sea lo más pequño posible necesitamos que el numerador sea pequeño y el denominador grande. Es decir nuestro problema de escribir raíces como racionales (o aproximarlas al menos) se ha convertido en la búsqueda de elementos del siguiente conjunto.
\[\{(x,y) \tq x^2-2y^2 = 1 \ \ y >> 1\}\]

La solución de la ecuación que define el conjunto se conoce como un \textbf{problema diofántico} en honor a \concept{Diofanto}, algebrista griego que vivió en la última parte de Grecia.

A raíz de este problema surgen las ecuaciones de Pell.

\begin{defn}[Ecuaciones de Pell]
Son un método de generación de pares $(x,y)$ que satisfagan $x^2-ny^2=1$ (\textbf{ecuación conocida como ecuación de Pell}) con $|y|>>1$.

\textbf{Dato:} Pell fue un inglés que vivió en torno al año 1700 y que no merece que su nombre aparezca aquí pues sólo hizo algunos trabajillos relacionados con estas ecuaciones.
\end{defn}

La idea base para encontrar estos números es emplear un método generador y una solución trivial. La solución más pequeña, es decir, aquella con los valores de $x$ e $y$ más pequeños es fácil de encontrar.

Veamos este método generador con el caso de la $\sqrt{2}$
\begin{example}
En esta ocasión la solución ``mínima'' es:
\[\sqrt{2} \approx \frac{3}{2}\]

Si escribimos
\[x^2-2y^2 = (x-\sqrt{2}y)(x+\sqrt{2}y) = 1\]
podemos ver que estamos trabajando dentro del anillo $\ent(\sqrt{2})$, es decir, $(x,y)$ son solución de la ecuación de Pell si y sólo si $(x,y)\in U(\ent(\sqrt{2}))$

\obs Por teoría de anillos sabemos que podemos sumar y multiplicar elementos del anillo sin salirnos de él. Además:
\[z_1\cdot z_2 = (a,b)(c,d) = (ac+2bd,ad+bc) = z_3\]

\begin{prop}
Las unidades de $\ent(\sqrt{2})$ es un grupo y por tanto:
\[x_1,z_2 \in U(\ent(\sqrt{2})) \implies z_1\cdot z_2 \in U(\ent(\sqrt{2}))\]
\end{prop}

Llegados a este punto es sencillo ver el procedimiento a seguir. Una vez tenemos dos soluciones las multiplicamos y vamos combinando (multiplicando) este resultado por el anterior hasta ir obteniendo cada vez números más grandes.
\end{example}

Paralelamente al desarrollo de esta teoría se desarrolló un teorema indio que venía a decir lo mismo.

\begin{theorem}[Teorema indio]
Si tenemos dos soluciones $(x_1,y_1)$, $(x_2,y_2)$ de una ecuación del tipo
\[x^2-ny^2=1 \text{ donde } n \text{ es conocido }\]
tenemos que
\[\left\{ \begin{array}{l} x= x_1x_2+2y_1y_2 \\ y=x_1y_2+x_2y_1 \end{array}\right.\]
es otra solución.
\end{theorem}
\begin{corol}
Una ecuación del tipo $x^2-ny^2=1$ tiene infinitas soluciones
\end{corol}

\obs En las sucesivas aproximaciones que podemos hacer apoyándonos en este teorema cometemos un error:
\[\abs{\frac{x_n}{y_n} - \sqrt{n}} = \frac{x_n-\sqrt{n}y_n}{y_n} = \frac{x_n^2-ny_n^2}{y_n(x_n + \sqrt{n}y_n)}=\frac{1}{2\sqrt{n}y_n^2}\]

Con nuestros conocimientos actuales podemos ver que la ecuación $x^2-ny^2 =1$ representa una hipérbola y que lo que estamos haciendo es encontrar puntos enteros en la misma.

En general, las geometrías hiperbólicas dan lugar al problema de Pell

\obs Si aplicamos este procedimiento con $n=4$ tendremos que
\[x^2-4y^2 = 1 \iff (x-2y)(x+2y) = 1 \left\{ \begin{array}{l} x+2y=x-2y = 1 \text{ solución } (x,y)=(1,0) \\ x+2y=-1=x-2y \text{ solución } (x,y)=(-1,0)\end{array}\right.\]

\begin{theorem}
Si $n$ tiene raíz exacta la ecuación de Pell sólo tiene dos soluciones triviales. En caso contrario tendrá infinitas soluciones
\end{theorem}

\begin{theorem}
Para toda solución $(x,y)$ existe un $m$ tal que
\[x+\sqrt{n}y = \left( x_1+\sqrt{n}y_1\right)^m\]
donde $(x_1,y_1)$ es la solución mínima.
\end{theorem}
\begin{proof}
Debemos observar que $U^+(\ent[\sqrt{n}])$ es un grupo cíclico con generador $(x_1,y_1)$.

\[(x_1+\sqrt{n}y)^m = (x_m,y_m)\]
siendo
\[x_m = x_1^m+ {m \choose 2 } x_1^{m-2}\cdot n \cdot y_1^2 + ...\]
\[y_m = {m \choose 1}x_1^{m-1}\sqrt{n}y_1 + ...\]
donde ambos valores, $x_m,y_m$ tienden a infinito de forma potencial.
\end{proof}

El problema de Pell puede generalizarse forzando al numerador a alcanzar distintos valores.

\begin{defn}[Ecuación de Pellk]
Las ecuaciones de Pellk son de la forma
\[x^2-ny^2 = k\]
que \textbf{siempre} tienen solución
\end{defn}

\begin{example}
Si tenemos $N=225$, $x=31$, $y=2$ tendríamos una ecuación de Pellk con $k=89$.

Pero con ingeniosas variaciones podríamos encontrar una $k$ distinta. El truco para variar la $k$ pasa por escribir $N(x,y)=x^2-Dy^2$ con lo que podemos escribir
\[(x_1+\sqrt{D}y_1)(x_2+\sqrt{D}y_2) = x_3 + \sqrt{D}y_3 \equiv N_1\cdot N_2 = N_3 \implies k_1\cdot k_2=k_3\]

Aunque este procedimiento parece que hace la $k$ cada vez mayor, también nos permitirá derivar otro algoritmo que nos lleva a obtener $k$ más pequeñas.
\end{example}

\begin{mdframed}
\textbf{Caso de Arquímedes}\\

Arquímedes escribió (y Lenstra lo generalizó) el problema:
\[h^2=dl^2+1 \text{ equivalente a } x^2-Dy^2 = 1\]

Ya Arquímedes fué capaz de encontrar soluciones al problema para todo $D$ y aún no sabemos cómo
\end{mdframed}

\section{Fracciones continuas}
Ya hemos visto en otras ocasiones cómo acabamos calculando ecuaciones continuas a partir de una ecuación de Pell pero, ¿podemos hacer justo lo contrario?.

Para hacerlo vamos a rellenar la siguiente tabla con los numeradores y denominadores de las fracciones continuas para el caso concreto $\sqrt{2}$

\begin{center}
\begin{tabular}{|c|c|c|c|c|c|c|}
\hline
$\sqrt{2}$ & $a_n$  & $p_n$ & $a_n$ & $\frac{p_n}{a_n}$ & $\left(\frac{p_n}{a_n}\right)^2$ & $n-$ norma \\
\hline
\hline
0 & 1 & 1 & 1 & 1 & 1 & -1 \\
\hline
1 & 2 & 3 & 2 & 1.5 & 2.25  & 1 \\
\hline
2 & 2 & 7 & 5 & 1.4 & 1.96 & \\
\hline
3 & 2 & 17 & 12 & 1.416 & 2.006 & \\
\hline
4 & 2 & 41 & 29 & 1.413 & 1.94 & \\
\hline
\end{tabular}
\end{center}

La norma del segundo paso por fracciones continuas es la solución de la ecuación de Pell con $k=1$.

\begin{example}
Vamos a ver que
\[\sqrt{7} = [2; \overline{1,1,1,4}]\]

Para ello, sabiendo que $\sqrt{7} \approx 2,$ escribimos:
\[7 =4+3 \implies (\sqrt{7}-2)(\sqrt{7}+2) = 3 \implies \sqrt{7} = 2+\frac{3}{\sqrt{7}+2}\]

Como nos va a interesar poder escribir la fracción como $\frac{1}{\text{algo}}$ vamos a calcular su inversa.
\[\frac{\sqrt{7}+2}{3} = 1 +\frac{\sqrt{7}-1}{3}\]
puesto que
\[\sqrt{7}-1 = \frac{7-1}{\sqrt{7}+1}\]
podemos escribir
\[\frac{\sqrt{7}+2}{3} = 1 +\frac{\sqrt{7}-1}{3} = 1 +\frac{2}{\sqrt{7}+1}\]
Por otro lado
\[\frac{\sqrt{7}+1}{2} = 1+\frac{\sqrt{7}-1}{2} = 1+\frac{3}{\sqrt{7}+1}\]
Finalmente, volviendo a la fórmula inicial tenemos
\[\sqrt{7} =  2+\frac{3}{\sqrt{7}+2} = 2+ \frac{1}{1+\frac{2}{\sqrt{7}+1}}=2+\frac{1}{1+\frac{1}{1+\frac{3}{\sqrt{7}+1}}}=...\]
\end{example}

\section{Versión de Pell}
Podemos estudiar las diferentes fracciones que se van obteniendo con el procedimiento visto en el ejemplo anterior y ver que estas convergen al resultado esperado por arriba y por debajo de forma alterna.

Es decir, en cada paso obtenemos un valor un poco más cercano al que queremos aproximar.

\[\sqrt{7} = 2, \underbrace{2+1}_{3}, \underbrace{2+\frac{1}{1+1}}_{\frac{5}{2}}, \underbrace{2+\frac{1}{1+\frac{1}{1+1}}}_{\frac{8}{3}},  \underbrace{2+\frac{1}{1+\frac{1}{1+\frac{1}{1+1}}}}_{\frac{37}{14}},  \underbrace{2+\frac{1}{1+\frac{1}{1+\frac{1}{4}}}}_{\frac{45}{17}}\]

\begin{theorem}[Teorema experimental]
Siendo $x_n = \frac{p_n}{q_n}$ se cumple
\[x_{n+1}-x_n = \frac{(-1)^n}{q_1 \cdot q_{n+1}}\]
\end{theorem}

Este teorema no pertenece a Pell, más bien es un resultado numérico experimental.

La idea de Pell era ligeramente diferente. Él buscaba, como ya hemos visto, valores de $x,y$ con $x^2-ny^2 = 1$.

Supongamos que tenemos $n=7$, entonces:
\[\begin{array}{l}
x_0=2, y_0 =1, N_0 = x_0^2-7y_0^2 = -3 \\
x_1=5, y_1 =2, N_0 = x_1^2-7y_1^2 = 2 \\
x_2=5, y_2 =2, N_0 = x_2^2-7y_2^2 = -3 \\
x_2=8, y_2 =3, N_0 = x_2^2-7y_2^2 = 1
\end{array}\]

A partir de aquí, como ya hemos obtenido el generador, las soluciones de Pell son
\[x_n + \sqrt{7}y_n = \left(8 + 3 \sqrt{7}\right)^n\]

\begin{defn}[Número áureo]
El número aureo es la fracción continua
\[\phi=[1;\overline{1}]\]
que satisface, por definición:
\[\frac{1}{\phi} = \phi -1 \implies \phi = \frac{1+\sqrt{5}}{2}\]
\end{defn}

\chapter{Geometría}
Antes de empezar a estudiar este tema recordemos que los griegos no conocían los números negativos. A la hora de hacer cuentas eran capaces (lógicamente) de entender el concepto de una deuda, aunque no tenían un concepto ni la notación necesarias para representar los números negativos.

En ``Los Elementos'' Euclides describe toda la geometría conocida en el momento partiendo de las definiciones más elementales. Evidentemente conceptos como punto, plano o recta han de ser intuitivos.

Si alguien no entiende estos conceptos, le rogamos encarecidamente deje de leer este libro y se cambie a magisterio.

Una vez establecemos una serie de conceptos como elementales o intuitivos, podemos empezar a construir la geometría, empezando desde abajo.

\section{Construcciones elementales}
\subsection{Perpendicular}
Una de las primeras cosas que debemos hacer es aprender a construir la perpendicular a partir de una recta dada.

El procedimiento es sencillo y queda ilustrado en la siguiente imagen:

\begin{minipage}{0.4\textwidth}
\begin{center}
\inputtikz{build_perp}
\end{center}
\end{minipage}
\begin{minipage}{0.57\textwidth}
\begin{enumerate}
\item Queremos trazar la recta perpendicular a $\overline{AB}$ pasando por $P$.
\item Desde $P$ trazamos un arco de radio cualquier y obtenemos los puntos $A'$ y $B'$
\item Desde estos puntos, con un arco cualquiera mayor que $\frac{\norm{\overline{A'B'}}}{2}$ trazamos los dos arcos verdes que se cortarán en $P$ y en $Q$.
\item La recta que contiene al segmento $\overline{PQ}$ es la que buscamos
\end{enumerate}
\end{minipage}

Si quisiéramos trazar la perpendicular a la recta por un punto dado de la recta, con un radio cualquiera trazamos un arco centrado en el punto $P$. Los dos puntos de corte con la recta serán los puntos $A'$ y $B'$ respectivamente.

A partir de aquí la construcción es igual a la ya explciada.

\subsection{Paralela}

\begin{minipage}{0.4\textwidth}
\begin{center}
\inputtikz{build_parallel}
\end{center}
\end{minipage}
\begin{minipage}{0.57\textwidth}
\begin{enumerate}
\item Queremos trazar la recta paralela a $\overline{AB}$ pasando por $P$.
\item Tomamos dos puntos al azar $C1$, $C2$ en la recta dada.
\item Tomando el radio $\overline{C1P}$ trazamos un arco centrado en $C2$.
\item Con radio $\overline{C1C2}$ trazamos un arco centrado en $P$
\item La recta buscada es la que pasa por $P$ y el nuevo punto determinado.
\end{enumerate}
\end{minipage}

\subsection{Cuadrado}
Una vez hemos explicado el procedimiento necesario para construir rectas paralelas y perpendiculares es totalmente trivial el proceso de construcción de un cuadrado a partir de un lado.

\begin{enumerate}
\item Construimos dos perpendiculares al lado en los dos vértices que tomemos.
\item Conociendo el lado, midiendo sobre las perpendiculares que acabamos de dibujar podemos encontrar otros dos vértices.
\item Los unimos y completamos el cuadrado
\end{enumerate}

\section{Los Elementos}
El conjunto de libros escritos por Euclides conocidos como ``Los Elementos'' consta de unos trece libros de los que los 4 primeros tratan sobre la geometría plana.

En estos volúmenes, Euclides definió  una serie de postulados que eran tomados como base a la hora de desarrollar toda la geometría.

Estos \concept{postulados geométricos de Euclides} consistuyen una base elemental, que no requiere ser probada, sobre la geometría a partir de la cual se pueden desarrollar conceptos más complejos. Estaban basados en el saber de la época y se relacionan con las construcciones elementales mencionadas en la sección anterior.

Los postulados son:
\begin{enumerate}
\item Dos puntos cualesquiera determinan un segmento de recta
\item Un segmento de recta puede prolongarse indefinidamente en una línea recta
\item Se puede trazar una dicrcunferencia dados un punto y un radio cualquiera
\item Dos ángulos rectos son iguales entre si
\item Dada una recta y un punto exterior a esta sólo puede trazarse una paralela que pase por el punto.
\end{enumerate}

\section{Problemas de áreas}
Es sencillo calcular el área de un cuadrado de lado 1 unidad, por la propia definición de área se establece el área de dicho cuadrado como una unidad cuadrada.

\subsection{Área del rectángulo}

Todos sabemos cuál es la fórmula para el área de un rectángulo pero, ¿Cómo podemos demostrar esta fórmula?

La idea que empleaban los griegos para calcular las areas consistía en transofmar la figura en algo cuyo área ya supiesen calcular.

En el caso del rectángulo, lo que aremos será convertirlo en un cuadrado como muestra el siguiente dibujo.

\begin{minipage}{0.4\textwidth}
\begin{center}
\inputtikz{area_rectangulo}
\end{center}
\end{minipage}
\begin{minipage}{0.55\textwidth}
El teorema de la altura nos dice que
\[h^2=l_1\cdot l_2\]

Por tanto ya tenemos el área del cuadrado verde que sabemos es igual a la del rectángulo original porque...
\end{minipage}

\subsection{Áreas de triángulos}

Para calcular el área de un triángulo recto lo trivial es completarlo a un rectángulo y ver que el área del triángulo es la mitad de la del rectángulo.

Dado un triángulo isósceles podemos realizar un trabajo similar, dividiendo previamente el isósceles en dos triángulos rectángulos.

Finalmente, en el caso de un triángulo escaleno deberemos trabajar algo más, pero siempre intentando compararlo con un cuadrado o rectángulo.

Los siguientes dibujos muestran cómo pueden realizarse estas extensiones.

\begin{minipage}{0.33\textwidth}
\begin{center}
\inputtikz{area_triangulo_rectangulo}
\end{center}
\end{minipage}
\begin{minipage}{0.33\textwidth}
\begin{center}
\inputtikz{area_triangulo_isosceles}
\end{center}
\end{minipage}
\begin{minipage}{0.33\textwidth}
\begin{center}
\inputtikz{area_triangulo_escaleno}
\end{center}
\end{minipage}

El caso del triángulo escaleno, aún sin implicar ninguna idea diferente al resto, merece algún comentario.

La forma de calcular el área de este tríangulo consistiría en calcular el area del rectángulo que lo engloba todo y dividir este área entre 2.

Con eso ya hemos calculado el área del triángulo \textbf{más} el triángulo rectángulo que comparte ángulo con el inferior derecho del rectángulo. Pero sabemos calcular el área de este triángulo y sólo tenemos que restársela al primer resultado.

\subsection{Áreas en otros polígonos}
En general lo que podemos hacer es dividir el polígono en triángulos, cuyo área sabemos calcular.

Esta división puede hacerse pensando un poco, de forma que reducimos la cantidad de triángulos con los que trabajar, o dibujando todas las diagonales posibles y enfrentándonos al monstruito que obtengamos.

\section{Estudio de polígonos regulares}
\subsection{Triángulo equilátero}
Dado un triángulo equilátero de lado 1, el más trivial posible, parece razonable tratar de estudiar sus propiedades.

Uno de los primeros dibujos que podemos hacer sobre un triángulo equilátero es el de su altura, con lo que obtenemos:
\begin{center}
\inputtikz{triangulo_equilatero}
\end{center}

Pero como el triángulo es equilátero, podemos ver que aunque lo apoyemos en cualquier otro lado, podemos dibujar diferentes alturas.

\begin{center}
\inputtikz{triangulo_equilatero_alturas}
\end{center}

donde el punto $O$ es el \concept{ortocentro} del triángulo que se encuentra a distancia $\frac{l}{3}$ de cada lado y a distancia $\frac{2\cdot l}{3}$ de cada vértice.

\begin{theorem}[Teorema de las alturas]
Las alturas de un triángulo se cortan en el ortocentro.
\end{theorem}

Una vez sabemos la relación existente entre las distancias respecto a lados y ángulos desde el ortocentro, podemos ver que, siendo $R$ el radio de la circunferencia circunscrita y $r$ el de la inscrita en un triángulo, se satisface la relación:
\[r = \frac{1}{2} R\]

\subsection{Cuadrado}
Imitando lo realizado en la sección anterior, parece intuitivo dibujar las dos diagonales de un cuadrado con lo que obtenemos 4 triángulos isósceles, y calcular la altura de los mismos.

\begin{center}
\inputtikz{cuadrado_alturas}
\end{center}

En esta ocasión, con la misma nomenclatura del apartado anterior, vemos que la relación entre los radios de las circunferencias inscrita y circunscrita es:
\[r = \frac{1}{\sqrt{2}}R\]

\begin{theorem}
Sean $R(n)$, $r(n)$ los radios de las circunferencias circunscrita e inscrita, respectivamente, en un polígono regular de $n$ lados, se cumple que:
\[\lim_{n\to \infty} \frac{R(n)}{r(n)}=1\]
\end{theorem}
\begin{proof}
La demostración puede verse considerando un polígono con $n$ lados para un $n$ grande y observando el triángulo formado por dos vértices consecutivos y el centro del polígono.

En este caso observamos que cuando mayor sea $n$ más pequeña será la separación entre los arcos de las circunferencias inscrita y circunscrita.
\end{proof}

\subsection{Pentágono regular}

Si dibujamos un pentágono regular con todas sus diagonales obtenemos un nuevo pentágono en el interior del inicial, como muestra el siguiente dibujo:


\begin{minipage}{0.56\textwidth}
\begin{center}
\inputtikz{pentagon}
\end{center}
\end{minipage}
\begin{minipage}{0.43\textwidth}
Por semejanza de triángulos podemos ver que
\[\frac{d-1}{1} = \frac{1}{d}\]
que nos da el número áureo.
\end{minipage}

\subsection{Otros polígonos}

Los griegos fueron capaces de comprobar, para ciertos valores de $n$ pequeños, si era posible o no construir polígonos regulares de $n$ lados.

No obstante, no fueron capaces de encontrar un razonamiento general para cualquier $n$, pregunta que quedó abierta hasta Gauss, quien dijo que podía dibujarse con regla y compás un polígono de $n$ lados siempre y cuando existieran dos enteros $m_1,m_2$ tales que
\[n = (2^{m_1}(2^{m_2}+1))\]

Un número de esta forma se conoce como \concept{número de Fermat}

\section{Estudio de polígonos no regulares}

De forma casi anecdótica comentamos el caso de un triángulo escaleno en el que podemos dibujar las medianas, que se cortan en el \concept{baricentro}.

\section{Estudio de figuras tridimensionales}
Sólo existen 5 posibles poliedros regulares en los que todas sus caras sean polígonos regulares.

A partir del estudio de estos poliedros se fué aumentando el número de caras estudiadas. Si llevamos este procedimiento al límite llegamos a toparnos con la esfera, que podría asociarse a un poliedro con infinitas caras.

A partir de aquí surge el estudio de figuras tridimensionales que no sólo se componen de polígonos sino también de elementos curvos, como son el cono y el cilindro.

Una vez se encontraron con el cono, empezaron a estudiar las secciones del mismo con planos lo que llevó al desarrollo de la geometría analítica (que Descartes empezó) con el estudio de eclipses, parábolas e hipérbolas.

\obs Recordemos que hasta que no llegamos a \textbf{Descartes} no existía el plano cartesiano ni las nociones de representación gráfica que resultan triviales hoy día. Por tanto, no existía la representación analítica de curvas, lo que dificultaba muchísimo su estudio.

\section{Construcciones imposibles}
Hay tres construcciones imposibles con las que se ha tratado de lidiar desde hace tiempo y que estudiaremos en esta sección.

\subsection{Duplicación del cubo}

Dado un segmento, constuir otro con longitud doble es algo absolutamente trivial con escuadra y cartabón.

Si tenemos un cuadrado y queremos obtener otro con el doble de área podemos dibujar 4 veces el mismo cuadrado, formando uno más grande y tomar el cuadrado formado por las diagonales de los 4 cuadrados más pequeños.

Así el área de este nuevo cuadrado será igual a la diagonal del inicial al cuadrado. Por tanto si el primer cuadrado tenía lado $l$, su diagonal era $\sqrt{2}l$ y el nuevo cuadrado tendrá área $2l^2$.

Sin embargo, si queremos realizar este procedimiento con un cubo y obtener un cubo con el doble de volumen del inicial, tenemos un problema.

El problema formalizado es:

\textbf{Dad un cubo de lado $l=1$, construir otro de lado $L$ tal que $L^3=2$}

La aproximación numérica por fuerza bruta está fundamentada en Eudoxo y Arquímedes: Existen sucesiones de racionales $l_n^1,l_n^2$ tales que sus cubos aproximan a dos por arriba y por debajo.

\textbf{Los griegos no encontraron la solución a este problema} y no fue hasta 1837 cuando Wantzell demostró que no podía resolverse ya que si existiera un $L$ como el buscado, entonces la ecuación
\[x^3=2\]
tendría solución.

Sin embargo, por Teoría de Galois sabemos que con regla y compás se obtienen soluciones de ecuaciones cuadráticas y las soluciones de polinomios cúbicos que no se factorizan.

\obs Hay una equivalencia entre fracciones continuas periódicas y soluciones de ecuaciones cuadráticas.

\subsubsection{Solución griega sin regla y compás}
La idea de esta solución es que el corte entre una parábola y una hipérbola es $\sqrt[3]{2}$, es decir:
\[\begin{array}{l}y=x^2 \\ y=\frac{2}{x}\end{array} \implies \frac{2}{x} = x^3 \implies x^3 = 2\]

Puesto que los griegos conocían las parábolas y las hipérbolas podían llegar a estas interesecciones pero, puesto que no podrían escribir las ecuaciones de las mismas, no sabían que se cortaban en $\sqrt[3]{2}$.

\subsection{Trisección del ángulo}
Es muy sencillo calcular la bisectriz de un ángulo con la ayuda de regla y compás como todos hemos hecho en bachillerato.

Sin embargo no resulta tan trivial el proceso de dividir el ángulo en tres partes iguales.

Una posible aproximación a la solución del problema pasa por hacer uso de la trigonometría y los números complejos

\[e^{3iβ} = (e^{iβ})^3 = (\cos (β) + i \sin (β))^3 = \cos^3(β)-3\cos(β)\sin^2(b)+i(\text{algo})\]

Si tratamos de conocer la parte real, que nos permitiría conocer el ángulo ``trisector'', tendremos:
\[\cos(α)=\cos^3(β)-3\cos(β)\sin^2(b)\]
ecuación que sigue sin solución.

\subsection{Cuadratura del círculo}
Este famoso problema consiste en la construcción de un cuadrado con igual área que el círculo unidad, lo que implicaría construir un cuadrado de lado $\sqrt{π}$.

En 1882 Lindenmann demostró que π es un número trascendente, es decir, no es solución de ningún polinomio (es decir, no es algebraico).

Aunque los griegos estudiaron este problema, no llegaron a comprobar que $\sqrt{π}$ no es racional.

\begin{theorem}\label{theorem:e_irracional}
El número $e$ es irracional
\end{theorem}
\begin{proof}
Sabemos que
\[e = 1+1+\frac{1}{2!}+\frac{1}{3!}+...+\frac{1}{n!} + R_n \approx 2.71823\]

Supongamos que es racional, es decir $\exists p, q \in \ent \tq e = \frac{p}{q}$.

Si multiplicamos por $qn!$ a ambos lados de la igualdad tenemos:
\[pn! = qn!+qn!+q\frac{n!}{2} +q \frac{n!}{3!} + ... + q + qR_nn!\]

Puesto que todos los sumandos son enteros salvo el último, podemos concluir que
\[R_n \in \ent\]

Pero sabemos que
\[R_n = \frac{f^{n+1}(ε)}{(n+1)!} \leq \frac{3}{(n+1)!}<1\]

Con lo que tendríamos que $R_n \in [-1,1], \ R_n \in \ent, \ R_n \neq 0$ lo que nos lleva a una contradicción.
\end{proof}

Para demostrar que π es irracional podríamos ver que las fracciones continuas no tienen período finito.

\chapter{El Renacimiento}
Durante la Edad Media se produjo un parón completo en el desarrollo de las matemáticas debido a la gran influencia de la religión cristiana que no comulgaba en absoluto con las ideas de la ciencia.

No obstante, durante esta etapa el mundo islámico siguió desarrollándo la ciencia, que habían heredado de los indios y, aunque no hicieron grandes aportaciones, permitieron que esta volviera a occidente en el Renacimiento.

%% Apéndices (ejercicios, exámenes)
\appendix
\chapter{Ejercicios}
% -*- root: ../HistoriaMatematicas.tex -*-
\section{Ejercicios mandados en clase}
\begin{problem}[1]
Demostrar que cada uno de los ángulos de un triángulo equilátero es de $\frac{π}{3}$.
\solution
\end{problem}

\begin{problem}[2]
Demostrar que cada una semicircunferencia, cualquier punto de la misma nos sirve para construir un triángulo rectángulo (con el ángulo recto en el punto dado) que tiene como base el diámetro de la circunferencia.
\solution
\end{problem}

\begin{problem}[3]
Demostrar que dado el triángulo del ejercicio anterior, el ángulo formado entre base y el segmento que une el centro con el punto de la semicircunferencia es el doble del ángulo izquierdo del triángulo.
\solution
\end{problem}


\printindex
\end{document}

