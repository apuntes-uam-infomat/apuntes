\chapter{Códigos lineales}
\section{¿Por qué códigos lineales?}
Como ya se ha observado en el capítulo anterior hay una gran ventaja en el empleo de códigos lineales y es que la distancia mínima puede calcularse de forma mucho más sencilla, pues podemos calcular los pesos de cada palabra para luego tomar el mínimo en lugar de calcular todas las distancias entre posibles pares de palabras antes de tomar el mínimo.

Otra ventaja de los códigos lineales es que a la hora de definirlos no es necesario listar todas las palabras que componenen el código sino que basta con dar una base del espacio lineal que define el código.

Esta ventaja también se puede aprovechar proporcionando las ecuaciones que definen el código, de nuevo sin tener que listar todas las palabras del mismo. No obstante se tendrán que proporcionar más ecuaciones que elementos de la base aunque en cualquier caso estamos teniendo que redactar muchísima menos información que si diésemos todas las palabras válidas.

Por último, pero no por ello menos improtante, otra ventaja de los códigos lineales es que existen algoritmos rápidos de codificación y descodificación.

No obstante, los códigos lineales también tienen algunas \textbf{desventajas}. En general los códigos lineales son ``mejores'' que los demás aunque esto no siempre es así y nos encontraremos con excepciones en las que es mejor emplear códigos no lineales.

\begin{defn}[.$(n,k,d)$-código q-nario]
Un $[n,k,d]$-código lineal q-nario es un subespacio vectorial $\algb{C} \subset F_q^n$ tal que $dim(\algb{C})=k$ y $d(\algb{C})=d$.
\end{defn}

\begin{defn}[B$_q(n,d)$]
\[B_q(n,d)=max \{q^k \tq \ \exists [n,k,d]\text{-código linea q-nario}\}\]
\end{defn}

\obs $B_q(n,d)\leq A_q(n,d)$ y en ocasiones se tiene la desigualdad escrita.

\begin{defn}[Matriz generadora]
Sea $\algb{C}$ un [n,k,d]-código lineal q-nario decimos que una matriz $M$ es generadora de $\algb{C}$ si y sólo si las filas de $M$ son una base de $\algb{C}$.

Esta matriz deberá ser de dimensión $k\times n$.
\end{defn}

Decimos que $H \in \text{Mat}(F_q)$ será \concept{matriz controladora de paridad} para $\algb{C}$ si dado $x\in F_q^n$ se cumple
\[x \in \algb{C} \iff Hx^t = 0 \]