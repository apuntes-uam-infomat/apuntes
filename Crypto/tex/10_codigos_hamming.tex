\chapter{Códigos de Hamming}

\section{Códigos de Hamming binarios}
Trabajando con $q=2$ y $d=3$ vamos a trabajar con códigos Ham(r,2). Fijando $r$ vamos a buscar códigos con el mayor $n$ posible.

Para garantizar que la distancia mínima sea mayor o igual que 3, simplemente necesitamos:
\[\forall i \ H_i \neq 0 \ \ \ \forall i,j \ H_i \neq H_j\]
Es decir, basta con tomar cualquier conjunto $\{H_1,...,H_n\}\in F_2^r-\{0\}$ con $n$ vectores distintos.

Si queremos que el $n$ sea el mayor posible, tendremos un total de $n=2^r-1$.

Una vez tenemos $n$, sabiendo que la matriz controladora de paridad, $H$, tendrá $r$ filas y $n$ columnas (y sabiendo que la distancia del código es 3) podemos concluir que nos encontramos ante un $(n=q^r-1,q^{n-r},3)$-código $q$-ario.

\begin{example}
Si tenemos un código $Ham(4,2)$ con distancia mínima $d=3$, estamos ante un $(15,2^{11},3)$-código binario.

\end{example}

En general, si tenemos un código Ham(r,q) con distancia mínima $d=3$ queremos tener $q=p^s$ sobre $F_q$. Pedimos que $q$ sea potencia de un primo para poder trabajar con álgebra de primero puesto que de no ser $F_q$ un cuerpo el concepto de linealmente independiente quedaría bastante más complejo.

Para tener distancia mínima $d \geq 3$ necesitamos que ninguna columna de $H$ sea nula ni múltiplo de otra.

En el cuerpo $F_q^r-\{0\}$ tenemos un total de $q^r-1$ posibles columnas \footnote{Ya estamos descontando la columna nula}. Sin embargo, por cada columna $H_i$ que fijamos, nos estamos eliminando $q-1$ opciones, ya que el resto de columnas no podrán ser múltiplo de la primera.

Por tanto, tenemos como máximo $n=\frac{q^r-1}{q-1}$.

Lo más fácil es exigir que la coordenada más alta sea distinta de 0, por ejemplo 1.

\begin{example}
Si queremos encontrar el código Ham(2,3) óptimo con distancia mínima $d=3$ tenemos $n=\frac{3^2-1}{3-1}=4$. En concreto tenemos
\[H=\left( \begin{array}{cccc}
0 & 1 & 1 & 1 \\
1 & 0 & 1 & 2
\end{array}\right)\]
\end{example}
