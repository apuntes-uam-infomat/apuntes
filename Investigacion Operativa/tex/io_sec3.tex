

\section{Teorema fundamental de la programación lineal.}

Por el teorema de representación \ref{thm:representacion}, el problema lineal:


\begin{ioprob}
\goal{$\min c^\top x$}
\restrictions{$Ax=b$}{$x\geq 0$}{}{}{}{}
\end{ioprob}

es equivalente a: 

\begin{ioprob}
\goal{$\min c^\top \left[\sum_{i=1}^k \lambda_i x_i + \sum_{j=1}^\ell \mu_j d_j\right]$}
\restrictions{$\sum_{i=1}^k \lambda_i=1$}{$\lambda_i\geq 0,\ i=1,\ldots,k$}{$\mu_j\geq 0,\ j=1,\ldots,\ell$}{}{}
\end{ioprob}

donde $x_1,\ldots, x_k$ son los puntos extremos del conjunto factible y  $d_1,\ldots,d_\ell$ son sus direcciones extremas.



\[\exists j c^td_j < 0 \to \text{ no existe solución}\]

Por la afirmación anterior, vamos a suponer que $c^td_i ∀i < l$. Entonces, es óptimo fijar $µ_i = 0∀i<l$. De esta manera, obtenemos el problema:
\begin{ioprob}
\goal{$\min c^\top \left[\sum_{i=1}^k \lambda_i x_i \right]$}
\restrictions{$\sum_{i=1}^k \lambda_i=1$}{$\lambda_i\geq 0,\ i=1,\ldots,k$}{$\mu_j\geq 0,\ j=1,\ldots,\ell$}{}{}
\end{ioprob}

Sea $j$ tal que $c^tx_j = \min{c^tx_i : i = 1,..., k}$. Entonces, es óptimo $λ_j = 1, λ_i = 0 ∀i≠j$
 
\begin{theorem}[Teoremafundamental\IS de la programación lineal]

\end{theorem}
  
\section{Algoritmo del simplex.}

  
  
\subsection{Ejemplos.}
  
  
\subsection{La tabla del simplex. Pivoteo.}
  
  
\subsection{Método de las dos fases.}
  
\section{Optimización lineal con {\tt R}}
 