% -*- root: ../InvestigacionOperativa.tex -*-


\section{Conjuntos convexos}

Vamos a ver qué es un conjunto convexo como caso particular de conjuntos afines.  
Dados 2 puntos $x_1,x_2$, y $\theta$, tomamos 
\[
y = \theta x_1 + (1-\theta)x_2 = x_2 + \theta (x_1-x_2),\ \ \  x_1,x_2\in\mathbb{R}^n.
\]

Vamos a pensar qué ocurre diferentes valores de $\theta$:


\begin{figure}
\begin{center}
\includegraphics[scale=0.8]{tex/berrendero/tema2/combinacion}
\caption{Ejemplo de combinaciones lineales de 2 puntos para ilustrar conjuntos afin y convexo.}
\label{sec2:comb}
\end{center}
\end{figure}


En la figura $\ref{sec2:comb}$, vemos que la recta entera sería el conjunto afín entero, mientras que sólo el intervalo $\theta \in [0,1]$ sería el conjunto convexo. Formalmente,

\begin{defn}[Conjunto\IS afín]
Dados $x_1,x_2\in S$, S es un conjunto afín si $\theta x_1 + (1-\theta)x_2\in S$, para todo $\theta\in \mathbb{R}$.
\end{defn}

\begin{defn}[Conjunto\IS convexo]
Dados $x_1,x_2\in S$, S es un conjunto convexo si $\theta x_1 + (1-\theta)x_2\in S$, para todo $\theta\in [0,1]$.
\end{defn}





Vamos a ver ejemplos de subconjuntos convexos.
\begin{itemize}
\item \textbf{Hiperplanos}: $S=\{x:\, p^\top x = \alpha\}$, donde $p\in\mathbb{R}^n$, $\alpha\in\mathbb{R}$. 

\item \textbf{Semiespacios}: $S=\{x:\, p^\top x \leq \alpha\}$, donde $p\in\mathbb{R}^n$, $\alpha\in\mathbb{R}$. 

\item \textbf{Intersección arbitraria} de convexos: Si $S_i$ es convexo para todo $i\in I$, entonces $S=\bigcap_{i=1}^I S_i$ es un conjunto convexo.

\item Un \textbf{poliedro} (intersección finita de semiespacios) es un conjunto convexo. Por ejemplo, $S=\{x:\, Ax\leq b,\ x\geq 0\}$ es un conjunto convexo.

\item Una \textbf{bola} $B(\bar x,r)=\{x\in\mathbb{R}^n:\, \|x-\bar x\|<r\}$ es un conjunto convexo (para cualquier norma).

\end{itemize}



\subsection{Combinaciones convexas y afines}

Hemos visto lo que son los conjuntos convexos, pero a la hora de trabajar, puede que el conjunto de datos con el que trabajamos, no sea convexo. ¿Tiene sentido hablar del "mínimo conjunto convexo" que contiene al conjunto con el que estamos trabajando? Vamos a verlo, pero para ello necesitamos definir qué es una combinación.

\begin{defn}[Combinación\IS afín]

Sean $x_1,\ldots,x_k \in\mathbb{R}^n$. Una combinación afín de $\{x_i\}$ es
\[
y = \lambda_1 x_1+\cdots +\lambda_k x_k,
\]
donde $\lambda_1+\cdots +\lambda_k=1$.
\end{defn}


En el caso de combinaciones convexas, tenemos alguna restricción más, ya que los conjuntos convexos son subconjuntos de conjuntos afines.

\begin{defn}[Combinación\IS convexa] 
Sean $x_1,\ldots,x_k \in\mathbb{R}^n$. Una combinación convexa de $\{x_i\}$ es
\[
y = \lambda_1 x_1+\cdots +\lambda_k x_k,
\]
donde $\lambda_1+\cdots +\lambda_k=1$ y $\lambda_i \geq 0$, para todo $i=1,\ldots,n$.
\end{defn}


Ya tenemos los ingredientes para construir los \textbf{cierres}, es decir, los mínimos conjuntos convexos/afines que contienen a un conjunto.


\begin{defn}[Cierre\IS afín] 
Definimos $\afin{S}$, el cierre afín de un conjunto $S$ como
\[
\afin{S} = \left\{\sum_{i=1}^k \lambda_i x_i:\, x_i\in S,\  \sum_{i=1}^k \lambda_i = 1\right\}.
\]
\end{defn}

\paragraph{Propiedades:}
\begin{itemize}
\item Un conjunto es afín si y solo $S=\afin{S}$.

\item Un conjunto es afín si y solo si es la traslación de un subespacio vectorial (único)

\item La dimensión afín de un conjunto es la dimensión de su cierre afín (que a su vez es la dimensión del correspondiente subespacio vectorial).
\end{itemize}




\begin{defn}[Cierre convexo]
Definimos $\convx{S}$, el cierre convexo de un conjunto $S$ como
\[
\convx{S} = \left\{\sum_{i=1}^k \lambda_i x_i:\, x_i\in S,\ \lambda_i\geq 0,\ \sum_{i=1}^k \lambda_i = 1\right\}.
\]
\end{defn}

\paragraph{Propiedades:}
\begin{itemize}
\item  S es convexo si y sólo si $S = \convx{S}$

\begin{proof} Vamos a separar las implicaciones:


$\impliedby)$:\\ $S = \convx{S} \implies S$ convexo es trivial, vamos a demostrar la otra implicación:

$\implies)$:\\ Queremos demostrar que si $S$ es convexo, $S\subset \convx{S}$ y $\convx{S}\subset S$. Es trivial que $S\subset\convx{S}$, asique vamos a ver la otra inclusión y vamos a demostrarlo por inducción sobre $k$. Sea

\[x = \sum^k λ_ix_ki \text{ con } x_i \in S,λ\geq 0, \sum λ = 1\]

\subparagraph{Base:} $k=1$ es trivial.

\subparagraph{Paso:} Supongamos, sin pérdida de generalidad que $λ_{k+1} < 1$ y tomamos:
\[\sum^{k+1} λ_ix_i = (1-λ_{k+1})\underbrace{\left(\frac{λ_1}{1-λ_{k+1}}x_1 + ... + \frac{λ_k}{1-λ_{k+1}}x_k \right)}_{(1)}+λ_{k+1}x_{k+1}\]
$(1)$ es una combinación convexa, con lo que este paréntesis $\in S$. 

Por otro lado, $(1-λ_{k+1})a + λ_{k+1}x_{k+1}$ es una combinación convexa, ya que $a,x_{k+1}\in S$ y ambos coeficientes cumplen las restricciones.

Con ello, vemos que $\convx{S}\subset S$.
\end{proof}


\item $\convx{S}$ es el menor conjunto convexo que contiene a $S$

\end{itemize}


\begin{theorem}[Teorema\IS de Carathéodory]
Sea $S\subset \mathbb{R}^n$. Si $x\in\mbox{conv}(S)$

\textbf{Entonces} $x=\sum_{i=1}^{n+1}\lambda_i x_i$, con $\sum_{i=1}^{n+1}\lambda_i=1$, $\lambda_i\geq 0$, $x_i\in S,\; \forall i=1,\ldots, n+1$.
\end{theorem}

La demostración es interesante porque tiene un razonamiento que utilizaremos a menudo. Este razonamiento es: \textit{Perturbar un conjunto de coeficientes para que alguno llegue a ser 0, pero sin llegar a obtener ningún coeficiente negativo.
}

\begin{proof}
Sea $x\in\mbox{conv}(S)$. 

Vamos a tomar una combinación convexa con $k$ elementos. Entonces $x=\sum_{i=1}^{k}\lambda_i x_i$, con $\sum_{i=1}^{k}\lambda_i=1$, $\lambda_i> 0$, $x_i\in S$. 


Si $k\leq n+1$, hemos terminado, ya que tenemos expresado $x$ como combinación convexa de, a lo más, $n+1$ componentes. 

Ahora vamos a ver que si $k > n+1$, entonces podemos expresarlo como combinación convexa de $k-1$ elementos. 

Por ser $k>n+1$, tenemos que $x_2-x_1,\ldots, x_k-x_1$ son linealmente dependientes, entonces existen $\mu_i$ (alguno estrictamente positivo) tales que $\sum_{i=1}^{k}\mu_i x_i=0$.

 Tenemos que $x=\sum_{i=1}^k (\lambda_i-\alpha\mu_i)x_i$, donde
\[
\alpha = \min\{\lambda_j/\mu_j:\, \mu_j>0\}:=\lambda_r/\mu_r > 0.
\]

Este $\alpha$ lo obtenemos para mantener que sea una combinación convexa. Necesitaremos $λ_i - \alpha μ_i \geq 0$ y $\sum (λ_i -\alpha μ_i) = 0$.

Hemos conseguido expresar $x$ como combinación convexa de $k-1$ elementos. Volvemos al paso 1. En un número finito de iteraciones habremos llegado a $k = n+1$, con lo que habremos terminado la demostración.
\end{proof}



Sea $S\subset\mathbb{R}^n$ un convexo cuyo cierre afín es $\mbox{afin}(S)$. El \textbf{interior relativo} de $S$ se define como el conjunto de puntos $x\in S$ tales que existe $r>0$ con $B(x,r)\cap \mbox{afin}(S) \subset S$.

\begin{itemize}
\item ¿Cuál es el interior relativo de $\{x\in \mathbb{R}^3:\, -1\leq x_1\leq 1,\, -1\leq x_2\leq 1,\, x_3=0\}$?
\end{itemize}



\begin{theorem} 
Un conjunto convexo no vacío en $\mathbb{R}^n$ tiene interior relativo no vacío.
\end{theorem}


\begin{theorem} 
El interior de un conjunto convexo  en $\mathbb{R}^n$ es vacío si y solo si el conjunto está contenido en un hiperplano de $\mathbb{R}^n$.
\end{theorem}

\begin{proof}
$\impliedby)$:\\ Si el conjunto no está contenido en un hiperplano, entonces $\afin{S} = \real^n \to int(S) = intrel(S)$ pero $ intrel(S) ≠ \emptyset$, por el teorema anterior.

$\implies)$:\\

\end{proof}


Vamos a ver un lema que podemos demostrar con estos teoremas, para seguir relacionando estos conceptoss.


\begin{lemma}
Sea $S\subset\mathbb{R}^n$ un conjunto convexo con $\mbox{int}(S)\neq\emptyset$. Sea $x_1\in \bar{S}$, $x_2\in\mbox{int}(S)$. Entonces, $\theta x_1+ (1-\theta) x_2 \in\mbox{int}(S)$, para todo $\theta\in [0,1)$.
\end{lemma}

\begin{proof}

Sea $y=\theta x_1+ (1-\theta)x_2$.

\begin{enumerate}
\item Existe $\epsilon>0$ tal que $B(x_2,\epsilon)\subset S$.
\item Sea $\tilde{y}\in B(y,\eta)$, con $\eta=\epsilon(1-\theta)$.
\item Como $x_1\in\bar{S}$, existe $\tilde{x}_1\in S$ tal que 
\[
\|x_1-\tilde{x}_1\| < \frac{\eta-\|\tilde{y}-y\|}{\theta}
\Leftrightarrow \|y-\tilde{y}\| + \theta \|x_1 - \tilde{x}_1\| < \eta.
\]
\item Sea $\tilde{x}_2=(\tilde{y} - \theta \tilde{x}_1)/(1-\theta) \Leftrightarrow 
\tilde{y}=\theta \tilde{x}_1 + (1-\theta) \tilde{x}_2$.
\item Se verifica
\[
\|x_2-\tilde{x}_2\| = \frac{\|y-\theta x_1 - \tilde{y}+\theta\tilde{x}_1\|}{1-\theta}\leq  
\frac{1}{1-\theta}(\|y-\tilde{y}\| + \theta \|x_1 - \tilde{x}_1\|)<\frac{\eta}{1-\theta} = \epsilon.
\]
\item Por 1 y 5, $\tilde{x}_2\in S$. Por 4, $\tilde{y}\in S$. Como $B(y,\eta)\subset S$, $y\in \mbox{int}(S)$.


\end{enumerate}
\end{proof}

\begin{figure}[h]
\centering
\begin{tikzpicture}
\draw (2,2) circle (1.5cm);
\draw (6,2) circle (2cm);
\draw (4,2) ellipse (5cm and 3 cm);
\draw (2,2) -- (4,2) -- (6,2);
\draw (6,2) node[point,above]{$y$};
\draw (6.5,2.5) node[point,above]{$\tilde{y}$};
\draw (2,2) node[point,above]{$x_2$};
\draw (2.5,2.5) node[point,above]{$\tilde{x_2}$};
\node[point,above] at (2,2.5) {$x_1$} ;
\draw (-1,2) node[point,left]{$\tilde{x_1}$};
\end{tikzpicture}
\end{figure}


A continuación, vamos a repasar un par de conceptos topológicos para demostrar:

\begin{corol}
Si $S\subset\mathbb{R}^n$ es convexo, entonces tanto $\mbox{int}(S)$ como $\bar{S}$ son conjuntos convexos.
\end{corol}
\begin{proof}
\textcolor{red}{Por hacer.}
\end{proof}
\begin{corol}
Si $S\subset\mathbb{R}^n$ es convexo, entonces $\mbox{int}(S)=\mbox{int}(\bar{S})$. 
Si además $\mbox{int}(S)\neq \emptyset$, entonces $\bar{S}=\overline{\mbox{int}(S)}$.
\end{corol}
\begin{proof}
\textcolor{red}{Por hacer.}
\end{proof}

\begin{corol}
Si $S\subset\mathbb{R}^n$ es convexo, entonces $\partial S=\partial \bar{S}$.
\end{corol}
\begin{proof}
\textcolor{red}{Por hacer.}
\end{proof}


\begin{theorem}[Teorema\IS de la proyección]
 Sea $S\subset\mathbb{R}^n$ un conjunto convexo, no vacío y cerrado. Sea $y\in \mathbb{R}^n$. Existe un \textbf{único} $\bar x\in S$ (la proyección de $y$ sobre $S$) tal que 
\[
\|y-\bar x\| \leq \|y - x\|, \ \ \mbox{para todo}\ x\in S.
\]
Además, $\bar x$ es la proyección de $y$ sobre $S$ si y solo si
\begin{equation}
\label{eq.proyeccion}
(y-\bar x)^\top (x - \bar{x}) \leq 0.
\end{equation}
\end{theorem}

\begin{proof}
$\gor{x} + λ(x-\gor{x})\in S$ por convexidad, $∀λ\in(0,1]$.

Además, tenemos que 

\[
||y-\vx||^2 \leq ||y-\vx - λ(x-\vx)||^2 = \norm{y-\vx}^2 + \norm{λ(x-\vx)}^2 + 2λ(x-\vx)^{t}(y-\vx)
\]

Reordenando y haciendo $λ\to 0$, entonces \[ (x-\vx)^{t}(y-\vx) \leq 0 \]

Hasta ahora, hemos encontrado que existe un punto y además, que cumple la condición. Ahora vamos a ver que es \textbf{único}. Es interesante que este truco, que se utilizará en algunos ejercicios.

Supongamos que existe $z\in S$ que cumple la condición, es decir: \[ (x-z)^{t}(y-z) \leq 0 \; ∀x\in S\]

El truco está en tomar $x = \gor x$, el punto que encontramos antes. Vamos a escribir las ecuaciones que tenemos.

\[ 
\begin{array}{c}
(y-\vx)^{t}(x-\vx) \leq 0 \\
(y-z)^{t}(x-z) \leq 0 \\
(z-y)^{t}(z-\gor{x}) \leq 0
\end{array}
\]


Ahora, si sumamos las 3, por alguna razón obtenemos: 

\[(z-\vx)^t(z-\vx) = \norm{z-\vx} \leq 0 \to z = \vx\]

\end{proof}

Vamos a profundizar un poco en el resultado de este teorema:
\textbf{¿Cómo queda la condición (\ref{eq.proyeccion}) cuando $S$ es un conjunto afín?}
La aplicación $P:\mathbb{R}^n\to S$, que a cada $y$ le hace corresponder su proyección $P(y)$ sobre $S$, es continua.

\subsection{Teoremas de separación}

\begin{theorem}[Teorema\IS del hiperplano separador]

Sea $S\subset\mathbb{R}^n$ un conjunto convexo, no vacío y cerrado. Sea $y\notin S$.
Entonces $\exists p\in\mathbb{R}^n$, $p\neq 0$, y $\exists\alpha\in\mathbb{R}$ tales que el punto está a un lado, y el conjunto al otro, es decir $p^\top x\leq \alpha$, para todo $x\in S$ y $p^\top y > \alpha$.

\end{theorem}


\begin{proof}
Sea $\gor{x}$ la proyección de $y$ sobre $S$ y definimos $p = y-\gor{x}$ y $\alpha = p^t\gor{x}$.

\[
p^tx = \underbrace{(y-\gx)^t(x-\gx)}_{\leq 0} + \underbrace{p^t\gx}_{0} \leq \alpha
\]

Por el otro lado,

\[
p^tx = \underbrace{(y-\gx)^t(y-\gx)}_{\norm{y-\gx}^2 > 0} + \underbrace{p^t\gx}_{\alpha} > \alpha
\]


Como vemos, esta demostración se basa en el teorema de la proyección.
\end{proof}


Sobre este teorema hay diversas versiones, dependiendo de las condiciones impuestas. Si en vez de considerarlo cerrrado lo consideráramos abierto, obtendríamos separación no estrcita. En este curso, utilizaremos únicamente esta versión del teorema.


\begin{theorem}[Teorema\IS hiperplano separador]
Sea $S\subset\mathbb{R}^n$ un conjunto convexo\footnote{En las diapositivas de la asignatura se incluye la condición de interior no vacío. Esta condición, en realidad no es necesaria porque cuando el interior es vacío, es un poco absurdo de aplicar, pero el teorema es cierto}. 
Sea $\bar{x}\in \partial S$, la frontera de $S$. 

Entonces existe $p\in\mathbb{R}^n$, $p\neq 0$, tal que $p^\top (x-\bar{x})\leq 0$, para todo $x\in S$.

\end{theorem}



\begin{proof}

Vamos a aplicar el teorema anterior. Como el punto $x$ está en la frontera, podemos acercarnos a él desde fuera. 
Si tenemos una sucesión de puntos $y_i$ que convergen a $x$, tenemos un hiperplano separador en cada $y_i$. Tomando límite, ese será el hiperplano soporte.

Sea $\gx \in \partial S = \partial\gor{S}$. Además, $\int{S} = \int{\gor{S}}$

\[∀k\in \nat \exists y_k \in B\left(\gx,\frac{1}{k}\right) \cap \gor{S}^C\]
Por el teorema del hiperplano separador (\ref) existe $p_k$ con $\norm{p_k}$ tal que $p^t y_k > p^t x$ por el teorema anterior, $∀x\in S$.

Como $\{ p : \norm{p}\}$ es un conjunto compacto, existe una subsucesión  de $p_k$  convergente a $p$.

Haciendo $k\to \infty$, perdemos la desigualdad estricta y obtenemos:

\[p^t\gx \geq p^tx \dimplies p^t(x-\gx) \leq 0\]

\end{proof}


\begin{theorem}[Teorema\IS de separación entre 2 convexos]

Sean $S_1,S_2\subset \real^n$ convexos no vacíos y tal que $S_1\cap S_2  =\emptyset$.

\textbf{Entonces} existe $p\in\real^n$, con $p≠0$ tal que:

\[
\inf{p^tx: x\in S_1} \geq \sup\{p^tx : x\in S_2\}
\]

\end{theorem}

\begin{proof}
La demostración es un ejercicio de la hoja.

Tomando $S = \{x\in\real^n: x = x_1 - x_2, x_i \in S_i\}$.

$S$ es convexo y $0 \not\in S$.

Si $0\in\gor{S}$, aplicamos el primer teorema. Si $0\in\partial{S}$, aplicamos el segundo teorema
\end{proof}


\obs Aunque $S_1$ y $S_2$ sean cerrados, la desigualdad no tiene porqué ser estricta.




\subsection{Teoremas de la alternativa}

Ahora que ya hemos visto muchos conceptos generales, vamos a volver a los problemas de optimización.
Una buena herramienta son estos teoremas, que nos ayudan a saber si un sistema tiene solución o no.

\begin{lemma}[Lema\IS de Gordan]
Sea $A$ una matriz $m\times n$. Entonces, uno y \textbf{sólamente uno} de los siguientes sistemas tiene solución.

\begin{equation}
Ax < 0\;,\; x\in\real^n
\label{eq:Gordan_1}
\end{equation}

\begin{equation}
A^tp = 0\;,\; p > 0
\label{eq:Gordan_2}
\end{equation}
\end{lemma}


Vamos a razonar geométricamente, qué tiene que ocurrir para que los 2 sistemas tuvieran solución. Sea $A = (a_1^t,a_2^t)^t$.

\begin{figure}[h]
\centering
\begin{tikzpicture}
\draw (-2,0) -- (2,0);
\draw (0,-2) -- (0,2);
\draw[->] (0,0) -- (1,0) node[above] { $a_1$ }; 
\draw[->] (0,0) -- (-1,0) node[above] { $a_2$ }; 
\end{tikzpicture}
\end{figure}

En este caso, tenemos \[Ax < 0 \dimplies \left.\begin{array}{c}a_1^tx < 0\\a_2^tx<0\end{array}\right\} \to a_1 = -\alpha a_2\]


Por alguna razón que no se muy bien, vemos que $A^tp$ no tiene solución.


Antes de meternos con la demostración del teorema, vamos a ver qué relación tiene este resultado con los problemas de optimización que se supone que queremos aprender a resolver en esta asignatura.
Pensemos en el problema:
\begin{ioprob}
\goal{$\min f(x)$}
\restrictions{$g(x) \leq 0$}{}{}{}{}{}
\end{ioprob}

$\gor{x}$ es el mínimo local del problema $\dimplies$
\[\left.\begin{array}{c}\not\exists d \tq \grad f(\gx)^td < 0 \\ \grad g(\gx)^td < 0\end{array}\right\} \to \exists p = \begin{pmatrix}p_1\\p_2\end{pmatrix} \tq p_1\grad f(\gx) + p_2 \grad g(\gx) = 0, p_i \geq 0\]

\begin{proof}
La parte fácil es la demostración de: \ref{eq:Gordan_1} tiene solución, entonces \ref{eq:Gordan_2} no tiene solución.

Sea $\gx$ la solución de \ref{eq:Gordan_1} y supongamos que $\gor{p}$ es la solución de \ref{eq:Gordan_2}.

Consideremos:
\[\gor{p}^t A \gx = \underbrace{\gor{p}^t(A\gor{x})}_{<0} = \underbrace{(A^t\gor{p})^t\gor{x}}_{=0}\]

No puede ser igual y menor que 0 al mismo tiempo.


Vamos a ver que si \ref{eq:Gordan_1} no tiene solución, entonces \ref{eq:Gordan_2} sí la tiene. Aquí es donde entran los teoremas de separación.

Sean \[S_1 = \{z\in\real^m: z=Ax, x\in\real^n\}\subset\real^m\]
 \[S_2 = \{z\in\real^m: z<0\}\subset\real^m\]

Estos 2 conjuntos son convexos\footnote{¿Por qué?} y por otro lado, $S_1 \cap S_2 = \emptyset$, ya que \ref{eq:Gordan_1} no tiene solución.

Vemos que $\exists p ≠ 0$ tal que $p^tAx \geq p^tz, \forall x\in\real^n,z<0,z\in\real^m$


$p$ resuelve \ref{eq:Gordan_2}. Supongamos que $p\not\geq0$ (por ejemplo $p_j < 0$).

Consideramos: $z = -\lambda (0,\cdots,0,1,0,\cdots,0)$, con $\lambda > 0$.

Entonces, $p^tAx \geq -\lambda p_j \convs[λ\to\infty]\infty$ y esto se da $\forall x\in\real^n$

$x = -A^tp \to p^tAx = -\norm{A^tp}^2 \leq 0$

Pero haciendo $z\to 0$, resulta que $p^tAx \geq 0 \forall x \implies -\norm{A^tp}\geq 0$.

Como $\norm{A^tp} \geq 0$ y $\norm{A^tp} \leq 0$, deducimos que $\norm{A^tp} = 0$, con lo que \ref{eq:Gordan_2} tiene solución.
\end{proof}
























