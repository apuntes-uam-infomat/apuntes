 -*- root: ../InvestigacionOperativa.tex -*-

\section{Hoja 1}

\begin{problem}[1]

Una empresa de reciclaje usa papel y tela desechados para fabricar dos tipos distintos de papel reciclado.
Cada tanda de papel reciclado de clase A requiere 20 kg de tela y 180 kg de papel y produce un beneficio de 500 euros, mientras que cada tanda de papel reciclado de clase B requiere 10 kg de tela y 150 kg de papel y produce un beneficio de 250 euros. 
La compañía dispone de 100 kg de tela y 660 kg de papel. ¿Cuántas tandas debe fabricar de cada tipo?

\solution

\begin{center}
\begin{tabular}{c|ccc}
& A & B & Disp. \\\hline
Tela & 20 & 10 & 100\\
Papel & 180 & 150 & 660\\
Beneficio & 500 & 250 & 
\end{tabular}
\end{center}

Variables $x = $ calidad de A e $y = $ calidad de B.

\begin{ioprob}
\goal{$\max 500x_1 + 250 x_2$}
\restrictions{$20x_1 + 10x_2 \leq 100$}{$180x_1 + 160x_2 \leq 660$}{$x_i \geq 0$}{}{}{}
\end{ioprob}

Vamos a resolverlo gráficamente en la figura \ref{ej:1.1.a}. Vamos a reprensentar el conjunto del plano que cumple las 3 restricciones.


\begin{figure}[hbtp]
\centering
\begin{tikzpicture}[scale=0.6]
\draw[thick,->] (-1,0) -- (10,0) node[anchor=west] {$x$};
\draw[thick,->] (0,-1) -- (0,10) node[anchor=east] {$y$};
\draw[thick,-] (5,0) -- node[anchor=north west] {\text{ }$20x + 10 y = 100$} (0,10);
\draw[thick,-] (3.8,0) -- node[anchor=west] {$180x + 150y = 660$} (0,4);
\filldraw[fill=blue!40!white, pattern=north west lines, pattern color=blue] (0,0) -- (0,4) -- (3.8,0);
\draw[thick,-,color=red] (2,-2) -- (-1,4);
\draw (3.8,0) node[anchor=north]  {$(3.8,0)$};
\end{tikzpicture}
\label{ej:1.1.a}
\caption{Representamos las 2 rectas fronteras y el vector gradiente de la función objetivo.}
\end{figure}


La idea es mover la recta roja en su dirección perpendicular todo lo que podamos. En este caso, no podremos alejarnos más que el punto que del punto $(3.8,0)$, con lo que será el óptimo.

\end{problem}


\begin{problem}[2]

La empresa Animales Salvajes S.A. cría faisanes y perdices para repoblar el bosque y dispone de sitio para criar 100 pájaros durante la temporada.
Criar un faisán cuesta 20 euros y criar una perdiz cuesta 30 euros. 
La fundación Vida Animal paga a Animales Salvajes S.A. por los pájaros de forma que se obtiene un beneficio de 14 euros por cada faisán y 16 euros por cada perdiz. 
La empresa dispone de 2400 euros para cubrir costes. ¿Cuántas perdices y cuántos faisanes debe criar?


\solution 


\begin{center}
\begin{tabular}{c|cccc}
&Faisán & Perdiz & Disp. & \# pájaros \\\hline
Coste&20&30&2400&100\\
Beneficio&14&16&
\end{tabular}
\end{center}

Las variables utilizadas son $x$ para Faisán e $y$ para Perdiz.

El problema a resolver sería:

\begin{ioprob}
\goal{$\max 14x + 16y$}
\restrictions{$x+y\leq 100$}{$20x + 30y \leq 2400$}{$x,y > 0$}{}{}{}
\end{ioprob}

En la figura \ref{ej:1.2.a} encontramos la solución gráfica del problema. Vemos que la solución es la intersección de las rectas, así que calculamos la intersección: 
\[
\left.
	\begin{array}{cc}
		x+y = 100 \\ 
		20x+30y = 2400
	\end{array}
\right\} 
\to (x,y) = (40,60)
\]

\begin{figure}[h]
\centering
\begin{tikzpicture}[scale=0.6]
\draw[thick,->] (-1,0) -- (12.5,0) node[anchor=west] {$x$};
\draw[thick,->] (0,-1) -- (0,12.5) node[anchor=east] {$y$};
\draw[thick,-] (10,0) --  node[anchor=west] {$x+y=100$} (0,10);
\draw[thick,-] (8,0) -- (0,12) node[anchor=west] {$20x + 30y = 2400$};
\filldraw[fill=blue!40!white, pattern=north west lines, pattern color=blue] (0,0) -- (0,10) -- (4,6) -- (8,0);
\draw[thick,-,color=red] (3.5,0) -- (0,4);
\draw[thick,-,color=green] (9,0) -- (0,11);
\end{tikzpicture}
\label{ej:1.2.a}
\caption{Representamos las 2 rectas fronteras y el vector gradiente de la función objetivo. En rojo la dirección del gradiente y en verde la solución óptima.}
\end{figure}





\end{problem}


\begin{problem}[3]

La siguiente tabla da el porcentaje de proteínas, grasas y carbohidratos, para cinco alimentos
básicos, A, B, C, D y E :


Los precios por 100 g de estos alimentos (dados en el mismo orden de la tabla) son 5, 17, 37, 10,
15. Si una persona necesita consumir como mínimo 75 gramos de proteínas, 90 de grasas y 300 de
hidratos de carbono, plantea el problema de minimización para calcular la dieta alimenticia de mínimo coste.
\solution


\end{problem}


\begin{problem}[9]

Una propiedad conocida de la mediana de un conjunto de datos $y_i$ es que minimiza en $\theta$ el valor de $\sum y_i-\theta$. Plantea el problema de optimización como un problema lineal en forma estándar.
\solution

Queremos minimizar 
\[\sum_{i=1}^n |y_i - \theta|\]

El procedimiento habitual podría ser derivar e igualar a 0, pero en este caso no podemos ir por ese camino, ya que no es derivable.
Vamos a ver que la solución es la mediana y vamos a demostrarlo de 2 maneras distintas. 
Primero, formalmente y después, planteándolo como un problema de optimización lineal.

\begin{proof}

Tomando una muestra de tamaño 2 y un punto interior de $\theta$ el objetivo a minimizar es:
\[\sum |y_i - \theta| = \theta - y_{(1)} + y_{(2)}-\theta = y_{(2)} - y_{(1)}\]
es decir, la longitud del intervalo.

Vamos a tomar una muestra de tamaño $n$. 
En esa muestra,tTomamos una serie de intervalos contenidos de la siguiente manera:

\[ [y_{(1)},y_{(n)}] \supset  [y_{(2)},y_{(n-1)}] \supset [y_{(3)},y_{(n-2)}] \supset ... \supset [y_{(m)},y_{(n-m+1)}]\]

Donde \[m=\left\{ \begin{array}{cc} \frac{n}{2} & n\text{ par}\\ ?? & n\text{ impar} \end{array}\right.\]
Este es un razonamiento casi geométrico de porque la mediana minimiza.
\end{proof}


Ahora, vamos a contestar al enunciado, planteándolo como un problema de regresión.

Definimos $x_i \equiv y_i - \theta = x_i^+ - x_i^-$ donde $x_i^+ = \max\{x_i,0\}$ y $x_i^- = \max\{-x_i,0\}$

De esta manera, $\abs{x_i} = x_i^+ + x_i^-$. Con este truco, hemos conseguido modificar la función objetivo, que de esta manera es una función lineal.

Nuestra función objetivo es:

\[\min \sum_{i=1}^n x_1^+ + x_i^-\]

Y el precio a pagar, es que necesitamos incluir una restricciones, con lo que:


\begin{ioprob}
\goal{\[\min \sum_{i=1}^n x_1^+ + x_i^-\]}
\restrictions{$y_i = x_i^+ - x_i^- + \theta^+ + \theta^-$, $i=1,...,n$}{$x_i^+ \geq 0$}{$x_i^- \geq 0$}{$\theta_i^+ \geq 0$}{$\theta_i^+ \geq 0$}{}
\end{ioprob}


Vamos a escribirlo matricialmente.
Las \textbf{variables de decisión} son $(x_1^+,...,x_n^+,x_1^-,...,x_n^-,\theta^+,\theta^-$. 

\[c = (\underbrace{1,...,1}_{n},\underbrace{1,...,1}_{n},0,0)\]
\[b = (y_1,...,y_n) \]
\[A = \left( I | -I | 1_n | -1_n\right)\to \begin{array}{c}1_n = \begin{pmatrix}1\\1\\\vdots\\1\end{pmatrix}\\ I = \text{ identidad} \end{array}\]




\obs{}
¿Qué utilidad puede tener esto?
Al tomar un modelo de regresión (una recta que pase por una nube de puntos) se suele tomar el criterio de "mínimos cuadrados", que es minimizar $\sum y_i-(β_0+β_1x_i)$ (siendo $β_0+β_1x$ el modelo de regresión). 
Podríamos plantear un modelo de regresión con otro criterio, por ejemplo, el de minimizar el valor absoluto.




\end{problem}