\documentclass{beamer}

%\usepackage{beamerthemetreebars}
\usepackage{graphicx}
%\usepackage{beamerthemesplit}
%\beamertemplateshadingbackground{red!10}{blue!10}

%----------------------------------------------------------------------
% Para que aparezcan varias transparencias en la misma p\'{a}gina
\usepackage{pgfpages}
%\pgfpagesuselayout{4 on 1}[a4paper,border shrink=5mm,landscape]
%----------------------------------------------------------------------
% Suprime los s\'{\i}mbolos de navegaci\'{o}n
\setbeamertemplate{navigation symbols}{}
%---------------------------------------------------------------------

\usepackage[utf8]{inputenc}
%----------------------------------------------------------------
\newcommand{\ep}{\epsilon}
\newcommand{\real}{{\rm I\kern-.17em R}}
\newcommand{\pro}{\mbox{P}}
%-----------------------------------------------------------------

\title[Estad\'{\i}stica: Tema 3]{Tema 3\\
 Optimización lineal. Algoritmo del simplex}
\author[Berrendero]
{Jos\'{e} R. Berrendero}
\date{}
\institute{Departamento de Matem\'{a}ticas\\
 Universidad Aut\'{o}noma de Madrid}

%------------------------------------------------------------

\begin{document}

\frame{\titlepage}

\begin{frame}[plain]
\frametitle{Contenidos del tema 3}

\begin{itemize}
 
  \item Teorema fundamental de la programación lineal.
  
\  
  
  \item Algoritmo del simplex.

\  
  
  \item Ejemplos.
  
\  
  
  \item La tabla del simplex. Pivoteo.
  
\  
  
  \item Método de las dos fases.
  
  \
  
  \item Optimización lineal con {\tt R}
 
\end{itemize}


\end{frame}
%----------------------------------------------
\begin{frame}
\frametitle{Teorema fundamental de la programación lineal}

Por el teorema de representación, el problema lineal:

\begin{center}
\begin{tabular}{lr}
minimizar & $c^\top x$ \\
s.a. & $Ax=b$   \\
	 & $x\geq 0$
\end{tabular}
\end{center}

es equivalente a: 

\begin{center}
\begin{tabular}{ll}
minimizar & $c^\top \left[\sum_{i=1}^k \lambda_i x_i + \sum_{j=1}^\ell
 \mu_j d_j\right]$ \\
s.a. & $\sum_{i=1}^k \lambda_i=1$   \\
	 & $\lambda_i\geq 0,\ i=1,\ldots,k$\\
	 & $\mu_j\geq 0,\ j=1,\ldots,\ell$,
\end{tabular}
\end{center}
donde $x_1,\ldots, x_k$ son los puntos extremos del conjunto factible y  $d_1,\ldots,d_\ell$ son sus direcciones extremas.


\end{frame}
%----------------------------------------------
\begin{frame}
\frametitle{Teorema fundamental de la programación lineal}

{\bf Teorema:} Consideremos un problema de optimización lineal en forma estándar.
Sean $x_1,\ldots, x_k$ los puntos extremos del conjunto factible y sean $d_1,\ldots,d_\ell$ sus direcciones extremas. El problema tiene solución factible óptima  si y solo si $c^\top d_j\geq 0$, para todo $j=1,\ldots,\ell$. Si esta condición se cumple, existe un punto extremo que es  solución factible óptima del problema.


\

\

\begin{itemize}
\item Interpretación de la condición de existencia de solución.

\item ¿Puede haber exactamente dos soluciones factibles óptimas?

\item  Para resolver un problema lineal se podría comprobar que tiene solución, evaluar $c^\top x_i$ para todos los puntos extremos y elegir el mejor de ellos. En la práctica este método no es útil porque el número de puntos extremos puede ser muy grande.
\end{itemize}

\end{frame}
%----------------------------------------------
\begin{frame}
\frametitle{El algoritmo del simplex}

\begin{itemize}
\item Es un método sistemático para pasar de un punto extremo a otro de manera que siempre mejore el objetivo.

\

\item En cada paso se puede detectar si ya hemos llegado al óptimo o aún tenemos que pasar a otro punto extremo. También se puede detectar si el problema no tiene solución óptima.

\

\item Pasar de un punto extremo a otro corresponde a cambiar la base $B$ por una base nueva $\hat{B}$. 

\

\item En el simplex ambas bases difieren en un único vector de modo que las operaciones del cambio de base son relativamente sencillas.
\end{itemize}


\end{frame}
%----------------------------------------------
\begin{frame}
\frametitle{El algoritmo del simplex}

\begin{itemize}
\item \textbf{Solución factible básica inicial:} 
\[
\bar{x}=(x_1,\ldots,x_m,0,\ldots,0)^\top=(\bar{x}_B^\top,\bar{x}_N^\top)^\top={B^{-1}b\choose 0}:={\bar{b}\choose 0}.
\]

\item  $\bar{b}$ es el vector de coordenadas de $b$ respecto a la base $B$.

\item Valor objetivo inicial: $\bar{z}=c^\top_B B^{-1}b=c^\top_B \bar{b}$.



\item Valor objetivo en cualquier otro punto factible $x=(x_B^\top,x_N^\top)^\top$:
\begin{align*}
z &= c_B^\top (\bar{b}-B^{-1} N x_N)+c^\top_N x_N =
 \bar{z}-(c_B^\top B^{-1} N - c_N^\top)x_N\\
  &=\bar{z}- \sum_{j\in N} (z_j-c_j)x_j,
\end{align*}
donde $z_j = c_B^\top B^{-1} a_j:= c_B^\top y_j$.



\item $y_j=B^{-1} a_j\Leftrightarrow a_j = y_{1j} a_1+\cdots + y_{mj}a_m$, es decir, $y_j$ es el vector de coordenadas de la columna no básica $a_j$ respecto a la base $B$.




\end{itemize}

\end{frame}
%----------------------------------------------
\begin{frame}
\frametitle{El algoritmo del simplex}

\[
z = \bar{z}- \sum_{j\in N} (z_j-c_j)x_j.
\]

\begin{itemize}
\item ¿Qué ocurre si $z_j-c_j\leq 0$, para todo $j\in N$?

\

\item Supongamos que existe $k\in N$ con $z_k-c_k>0$.

\

\item Vamos a pasar de $\bar{x}$ a una nueva solución factible básica:
\[
\hat{x} = (\hat{x}_1,\ldots,\overset{(r)}{0},\ldots,\hat{x}_m,0,\ldots,\overset{(k)}{\alpha},\ldots,0)^\top,\ \ \ \alpha>0.
\]


\

\item El nuevo valor objetivo es: $\hat{z} =  \bar{z} - (z_k-c_k)\alpha$.




\end{itemize}

\end{frame}
%----------------------------------------------
\begin{frame}
\frametitle{El algoritmo del simplex}



\begin{itemize}
\item \textbf{Criterio de entrada:} Entra a la base la variable $k$ tal que
\[
z_k - c_k = \max_{j\in N}\{z_j-c_j:\, z_j-c_j>0\}.
\]

\


\item Hay que aumentar $\alpha$ tanto como sea posible sin salirnos del conjunto factible: $A\hat{x} = b$ es equivalente a
\[
\hat{x} = \bar{x} + \alpha{-y_k\choose e_k} \geq 0,\ \mbox{donde}\ y_k=B^{-1}a_k.
\]

\

\item ¿Qué ocurre si $y_k\leq 0$?

\

\item Supongamos que $y_k\nleq 0$.


\end{itemize}

\end{frame}
%----------------------------------------------
\begin{frame}
\frametitle{El algoritmo del simplex}



\begin{itemize}
\item Para que $\hat{x}$ sea factible también hace falta $\hat{x}\geq 0$:
\[
\hat{x}_B\geq 0\Leftrightarrow\bar{b}-\alpha y_k\geq 0\Leftrightarrow \alpha \leq \frac{\bar{b}_i}{y_{ik}},
\]
para todo $i=1,\ldots,m$ tal que $y_{ik}>0$.

\

\item El mayor valor posible de $\alpha$ es:
\[
\alpha = \min\left\{\frac{\bar{b}_i}{y_{ik}}:\, y_{ik}>0 \right\}:= \frac{\bar{b}_r}{y_{rk}}
\]


\


\item \textbf{Criterio de salida:} sale de la base la variable $r$ en la que se alcanza el mínimo anterior.


\end{itemize}

\end{frame}
%----------------------------------------------
\begin{frame}
\frametitle{Ejemplo}

\begin{center}
\begin{tabular}{lr}
maximizar & $3x_1 + x_2 + 2x_3$ \\
	 &  \\
s.a. & $2x_1+x_2+x_3 \leq 2$    \\
	 & $x_1 + 2x_2 + 3x_3\leq 5$  \\
	 & $2x_1+2x_2+x_3\leq 6$ \\
	 & $x_1\geq 0,\ x_2\geq 0,\ x_3\geq 0$
\end{tabular}
\end{center}

\

Pasamos primero a la forma estándar:

\begin{center}
\begin{tabular}{lr}
minimizar & $-3x_1 - x_2 - 2x_3$ \\
s.a. & $2x_1+x_2+x_3 + x_4 = 2$    \\
	 & $x_1 + 2x_2 + 3x_3 +  x_5 = 5$  \\
	 & $2x_1+2x_2+x_3 + x_6 = 6$ \\
	 & $x_i\geq 0,\ i=1,\ldots,6.$
\end{tabular}
\end{center}


\end{frame}
%----------------------------------------------
\begin{frame}
\frametitle{Ejemplo}

\textbf{Solución factible básica inicial:} $\bar{x}=(0,0,0,2,5,6)^\top$, para la que el objetivo es $\bar{z}=0$.

\
\

\begin{itemize}

\item Escribe los valores de: $B$, $\bar{b}$, $c_B$, $y_j$ y $z_j-c_j$, para todo $j\in N$.

\

\item ¿Es $\bar{x}$ la solución factible óptima del problema?

\

\item ¿Qué variable $k$ entra en la base? ¿Qué $y_k$ le corresponde?

\

\item ¿Qué variable $r$ sale de la base?

\

%\item ¿Cuál es la nueva solución factible básica?
\end{itemize}


\end{frame}
%----------------------------------------------
\begin{frame}
\frametitle{Pivoteo}

Necesitamos expresar los vectores $a_j$ y $b$ respecto a la nueva base $\hat{B}=\{a_1,a_5,a_6\}$. A esta operación se le llama \textbf{pivoteo}.

\

El \textbf{pivote} es el coeficiente $y_{rk}$ correspondiente a la fila de la variable que sale y la columna de la que entra.

\

¿Cuál es el pivote en el ejemplo?

\

\begin{align*}
2x_1+x_2+x_3 + x_4 &= 2   \\
x_1 + 2x_2 + 3x_3 +  x_5 &= 5  \\
2x_1+2x_2+x_3 + x_6 &= 6 \\
\end{align*}





\end{frame}
%----------------------------------------------
\begin{frame}
\frametitle{Pivoteo}

\begin{align*}
2x_1+x_2+x_3 + x_4 &= 2   \\
x_1 + 2x_2 + 3x_3 +  x_5 &= 5  \\
2x_1+2x_2+x_3 + x_6 &= 6 \\
\end{align*}


\begin{itemize}
\item La fila $r$ se divide por el pivote para que el coeficiente de $x_k$ sea 1.
\item Al resto de filas se les resta la fila $r$  multiplicada por el valor adecuado para que $x_k$ ya no aparezca en esa fila.
\end{itemize}

\begin{align*}
x_1 + x_2/2 + x_3/2 + x_4/2 &= 1   \\
     3x_2/2 + 5x_3/2 - x_4/2 +  x_5 &= 4  \\
      x_2 - x_4 + x_6 &= 4 \\
\end{align*}

\end{frame}
%----------------------------------------------
\begin{frame}
\frametitle{Ejemplo}

\textbf{Solución factible básica actual:} $\hat{x}=(1,0,0,0,4,4)^\top$, para la que el objetivo es $\hat{z}=-3$.

\
\

\begin{itemize}

\item Escribe los valores de:  $c_B$, $y_j$ y $z_j-c_j$, para todo $j\in N$.

\

\item ¿Es $\bar{x}$ la solución factible óptima del problema?

\

\item ¿Qué variable $k$ entra en la base? ¿Qué $y_k$ le corresponde?

\

\item ¿Qué variable $r$ sale de la base?


\end{itemize}


\end{frame}
%----------------------------------------------
\begin{frame}
\frametitle{Convergencia}

Si en cada paso encontramos $\bar{b}=B^{-1}b >0$, entonces $\bar{x}$ y $\hat{x}$ son puntos extremos distintos.

\

Como hay un número finito de puntos extremos, el algoritmo converge en un número finito de iteraciones.

\

Si en algún paso $\bar{b}_r=0$, entonces $\alpha=0$. Cambia la base, pero el punto extremo es el mismo.

\

Esto podría ocurrir infinitas veces y entonces el algoritmo del simplex no converge (se dice que ha ocurrido un ciclo).

\

Hay criterios de entrada y salida para evitar los ciclos: regla de Bland


\end{frame}
%----------------------------------------------
\begin{frame}
\frametitle{Tabla simplex}

Los elementos para efectuar cada iteración se suelen disponer ordenadamente en forma de tabla:

\

\begin{center}
\begin{tabular}{c|c|c}
$c$ & $c_B^\top$  & $c^\top_N$ \\ \hline
Variables & $x_B^\top$ &  $x^\top_N$ \\ \hline
$x_B=\bar{b}$ &    $\mathbb{I}_{m\times m}$ & $B^{-1}N$ \\ \hline
$z-c$   & $0$   & $c_B^\top B^{-1}N - c^\top_N$
\end{tabular}
\end{center}

\

\begin{itemize}
\item Las columnas corresponden a variables básicas y no básicas.

\item Las filas corresponden a las variables básicas.

\end{itemize}


\end{frame}
%----------------------------------------------
\begin{frame}
\frametitle{Ejemplo}

\begin{center}
\begin{tabular}{lr}
minimizar & $-4x_1 - 3x_2$ \\
s.a. & $-x_1+x_2+x_3  = 2$    \\
	 & $x_1 + 2x_2 + x_4 = 6$  \\
	 & $2x_1+x_2 + x_5 = 6$ \\
	 & $x_i\geq 0,\ i=1,\ldots,5.$
\end{tabular}
\end{center}

\

\textbf{Tabla inicial:}  $B=(a_3,a_4,a_5) = \mathbb{I}_{3\times 3}$



\begin{center}
\begin{tabular}{r | rrrrr}
$c$ & -4 & -3 & 0 & 0 & 0 \\ \hline
Variables & $x_1$ & $x_2$ & $x_3$ & $x_4$ & $x_5$ \\ \hline
$x_3 = 2$ & -1 & 1 & 1 & 0 & 0 \\
$x_4=6$   & 1 & 2 & 0 & 1 & 0  \\
$x_5=6$ & 2 & 1 & 0 & 0 & 1 \\ \hline
$z_j-c_j$ & 4 & 3 & 0 & 0 & 0 
\end{tabular}
\end{center}

\

Aplica los criterios de entrada y salida a la base.

\end{frame}
%----------------------------------------------
\begin{frame}
\frametitle{Ejemplo (primera iteración)}


\begin{center}
\begin{tabular}{r | rrrrr}
$c$ & -4 & -3 & 0 & 0 & 0 \\ \hline
Variables & $x_1$ & $x_2$ & $x_3$ & $x_4$ & $x_5$ \\ \hline
$x_3 = 2$ & -1 & 1 & 1 & 0 & 0 \\
$x_4=6$   & 1 & 2 & 0 & 1 & 0  \\
$x_5=6$ & \textbf{2} & 1 & 0 & 0 & 1 \\ \hline
$z_j-c_j$ & 4 & 3 & 0 & 0 & 0 
\end{tabular}
\end{center}


\

\begin{center}
\begin{tabular}{r | rrrrr}
$c$ & -4 & -3 & 0 & 0 & 0 \\ \hline
Variables & $x_1$ & $x_2$ & $x_3$ & $x_4$ & $x_5$ \\ \hline
$x_3=5$ & 0 & 3/2 & 1 & 0 & 1/2 \\
$x_4=3$   & 0 & 3/2 & 0 & 1 & -1/2  \\
$x_1=3$ & 1 & 1/2 & 0 & 0 & 1/2 \\ \hline
$z_j-c_j$ &  0 & 1 & 0 & 0 & -2 
\end{tabular}
\end{center}

\end{frame}
%----------------------------------------------
\begin{frame}
\frametitle{Ejemplo (segunda iteración)}

\begin{center}
\begin{tabular}{r | rrrrr}
$c$ & -4 & -3 & 0 & 0 & 0 \\ \hline
Variables & $x_1$ & $x_2$ & $x_3$ & $x_4$ & $x_5$ \\ \hline
$x_3=5$ & 0 & 3/2 & 1 & 0 & 1/2 \\
$x_4=3$   & 0 & \textbf{3/2} & 0 & 1 & -1/2  \\
$x_1=3$ & 1 & 1/2 & 0 & 0 & 1/2 \\ \hline
$z_j-c_j$ &  0 & 1 & 0 & 0 & -2 
\end{tabular}
\end{center}

\

\begin{center}
\begin{tabular}{r | rrrrr}
$c$ & -4 & -3 & 0 & 0 & 0 \\ \hline
Variables & $x_1$ & $x_2$ & $x_3$ & $x_4$ & $x_5$ \\ \hline
$x_3=2$ & 0 & 0 & 1 & -1 & 1 \\
$x_2=2$   & 0 & 1 & 0 & 2/3 & -1/3  \\
$x_1=2$ & 1 & 0 & 0 & -1/3 & 2/3 \\ \hline
$z_j-c_j$ &  0 & 0 & 0 & -2/3 & -5/3 
\end{tabular}
\end{center}

\

La solución factible óptima es $(2,2,2,0,0)^\top$ y el valor objetivo óptimo es -14.


\end{frame}
%----------------------------------------------
\begin{frame}
\frametitle{Actualización de la tabla}

\textbf{Columna de la izquierda}

\begin{align*}
\hat{x}_i &= \bar{b}_i - \alpha y_{ik} = \bar{b}_i - \frac{\bar{b}_r}{y_{rk}} y_{ik}\\
\hat{x}_k &= \alpha = \frac{\bar{b}_r}{y_{rk}}.
\end{align*}

\

\textbf{Valores $y_{ij}$}

\begin{align*}
a_k &= y_{1k}a_1 +\cdots + y_{rk}a_r + \cdots + y_{mk} a_m\\
a_r &= -\frac{y_{1k}}{y_{rk}}a_1-\cdots + \frac{1}{y_{rk}}a_k - \cdots - \frac{y_{1k}}{y_{rk}}a_m\\
a_j &= \left(y_{1j} - \frac{y_{1k}}{y_{rk}} y_{rj}\right)a_1 + \cdots + \frac{y_{rj}}{y_{rk}}a_k + \cdots + \left(y_{mj} - \frac{y_{mk}}{y_{rk}} y_{rj}\right)a_m
\end{align*}




%\textbf{Última fila:}




\end{frame}
%----------------------------------------------
\begin{frame}
\frametitle{Actualización de la tabla}


\textbf{Valores $y_{ij}$}

\begin{align*}
\hat{y}_{ij} &= y_{ij} - \frac{y_{rj}}{y_{rk}}y_{ik}, \ \ \mbox{si}\ i\neq r, \\
\hat{y}_{rj} &= \frac{y_{rj}}{y_{rk}}
\end{align*} 

\textbf{Última fila:}

\begin{align*}
\hat{z}_j - \hat{c}_j &= \sum_{r\neq i=1}^m c_i \hat{y}_{ij} + c_k\hat{y}_{rj} - c_j=
\sum_{i=1}^m c_i \left(y_{ij} - \frac{y_{rj}}{y_{rk}}y_{ik}\right) + c_k\frac{y_{rj}}{y_{rk}} - c_j \\
&= \sum_{i=1}^m c_i y_{ij} - c_j -  \frac{y_{rj}}{y_{rk}} \sum_{i=1}^m c_i y_{ik} + c_k\frac{y_{rj}}{y_{rk}} = (z_j-c_j) - \frac{y_{rj}}{y_{rk}} (z_k-c_k)
\end{align*}


\end{frame}
%----------------------------------------------
\begin{frame}
\frametitle{Actualización de la tabla}

En resumen:

\begin{itemize}
\item La fila del pivote (fila $r$) se divide por el pivote ($y_{rk}$). Así se consigue que $\hat{y}_{rk}=1$.

\

\item A la fila $i$ se les resta la fila  $r$ actualizada y multiplicada por $y_{ik}$. Así se consigue que $\hat{y}_{ik}=0$.

\

\item A la última fila se le resta la fila  $r$ actualizada y multiplicada por $z_k-c_k$. Así se consigue que $\hat{z}_k - \hat{c}_k=0$
\end{itemize}


\end{frame}
%-----------------------------------------------------------
\begin{frame}
\frametitle{Método de las dos fases}

Es un método útil para:

\begin{itemize}
\item Encontrar una solución factible básica inicial.

\item Detectar si el conjunto factible es  vacío.

\item Detectar si hay restricciones redundantes.
\end{itemize}


\


\textbf{Fase 1:} Se introducen variables artificiales y se minimiza su suma.

\

\textbf{Fase 2:} Si la suma óptima no es cero entonces el problema original no es factible. En caso contrario las variables artificiales habrán abandonado la base y dispondremos de una base inicial de variables \textit{legítimas}.

\end{frame}
%----------------------------------------------
\begin{frame}
\frametitle{Fase 1}

Si $e=(1,\ldots,1)^\top$, se resuelve el problema:

\begin{center}
\begin{tabular}{lr}
minimizar & $e^\top x^a$ \\
s.a. & $Ax + \mathbb{I}x^a = b$   \\
	 & $x\geq 0,\ x^a\geq 0$
\end{tabular}
\end{center}


\begin{itemize}
\item Variables artificiales son diferentes a variables de holgura.
\item Este problema tiene una solución factible básica obvia en la que las variables básicas son las artificiales.
\item Sea $(\bar{x},\bar{x}^a)$ el óptimo al final de la fase 1.
\end{itemize}

\end{frame}
%----------------------------------------------
\begin{frame}
\frametitle{Fase 2}

\textbf{Caso 1:} $\bar{x}^a\neq 0$, el problema original no es factible (¿por qué?).

\

\textbf{Caso 2:} $\bar{x}^a= 0$, pueden ocurrir a su vez dos casos

\

\begin{itemize}
\item \textbf{Ninguna variable artificial es básica.} En este caso se eliminan de la tabla las columnas de las variables artificiales. Se calculan los valores $z_j-c_j$ y se continúa como en el método simplex habitual.

\item \textbf{Hay alguna variable artificial en la base al nivel 0 (degeneración).} Se busca en la fila un pivote para poder sustituirla por una variable \textit{legítima}. Se calculan los valores $z_j-c_j$ y se continúa como en el método simplex habitual.
\end{itemize}



\end{frame}
%----------------------------------------------
\begin{frame}
\frametitle{Ejemplo}

\textbf{Problema original:}

\begin{center}
\begin{tabular}{lr}
minimizar & $4x_1 + x_2 + x_3$ \\
	 &  \\
s.a. & $2x_1+x_2+2x_3 =4 $    \\
	 & $3x_1 + 3x_2 + x_3= 3$  \\
	 & $x_1\geq 0,\ x_2\geq 0,\ x_3\geq 0$
\end{tabular}
\end{center}

\

\textbf{Problema a resolver en la fase 1:}

\begin{center}
\begin{tabular}{lr}
minimizar & $x^a_1 + x^a_2$ \\
	 &  \\
s.a. & $2x_1+x_2+2x_3 + x_1^a=4 $    \\
	 & $3x_1 + 3x_2 + x_3 + x_2^a = 3$  \\
	 & $x_i\geq 0,\ x_i^a\geq 0$
\end{tabular}
\end{center}

\end{frame}
%----------------------------------------------
\begin{frame}
\frametitle{Fase 1}

\begin{center}
\begin{tabular}{r | rrrrr}
$c$ & 0 & 0 & 0 & 1 & 1 \\ \hline
Variables & $x_1$ & $x_2$ & $x_3$ & $x^a_1$ & $x^a_2$ \\ \hline
$x_1^a=4$ & 2 & 1 & 2 & 1 & 0 \\
$x_2^a=3$   &\textbf{3} & 3 & 1 & 0 & 1  \\ \hline 
$z_j-c_j$ &  5 & 4 & 3 & 0 & 0 
\end{tabular}
\end{center}

\begin{center}
\begin{tabular}{r | rrrrr}
$c$ & 0 & 0 & 0 & 1 & 1 \\ \hline
Variables & $x_1$ & $x_2$ & $x_3$ & $x^a_1$ & $x^a_2$ \\ \hline
$x_1^a=2$ & 0 & -1 & \textbf{4/3} & 1 & -2/3 \\
$x_1=1$   & 1 & 1 & 1/3 & 0 & 1/3  \\ \hline 
$z_j-c_j$ &  0 & -1 & 4/3 & 0 & -5/3 
\end{tabular}
\end{center}


\begin{center}
\begin{tabular}{r | rrrrr}
$c$ & 0 & 0 & 0 & 1 & 1 \\ \hline
Variables & $x_1$ & $x_2$ & $x_3$ & $x^a_1$ & $x^a_2$ \\ \hline
$x_3=3/2$ & 0 & -3/4 & 1 & 3/4 & -1/2 \\
$x_1=1/2$   & 1 & 5/4 & 0 & -1/4 & 1/2  \\ \hline 
$z_j-c_j$ &  0 & 0 & 0 & -1 & -1 
\end{tabular}
\end{center}

\end{frame}
%----------------------------------------------
\begin{frame}
\frametitle{Fase 2}

\begin{itemize}
\item Partimos de la solución factible básica en la última tabla de la fase 1.

\item Se eliminan las variables artificiales.

\item Se actualizan la primera y la última fila de la tabla.
\end{itemize}

\

\begin{center}
\begin{tabular}{r | rrr}
$c$ & 4 & 1 & 1  \\ \hline
Variables & $x_1$ & $x_2$ & $x_3$  \\ \hline
$x_1=1/2$ & 1 & \textbf{5/4} & 0  \\
$x_3=3/2$   & 0 & -3/4 & 1   \\ \hline 
$z_j-c_j$ &  0 & 13/4 & 0 
\end{tabular}
\end{center} 

\


Para las variables básicas $z_j-c_j=0$. Además,
\[
z_2-c_2 = (4,1){5/4 \choose -3/4}- 1 = 13/4.
\]


\end{frame}
%----------------------------------------------
\begin{frame}
\frametitle{Fase 2}



\begin{center}
\begin{tabular}{r | rrr}
$c$ & 4 & 1 & 1  \\ \hline
Variables & $x_1$ & $x_2$ & $x_3$  \\ \hline
$x_2=2/5$ & 4/5 & 1 & 0  \\
$x_3=9/5$   & 3/5 & 0 & 1   \\ \hline 
$z_j-c_j$ &  -13/5 & 0 & 0 
\end{tabular}
\end{center} 

\




La solución factible óptima del problema viene dada por\\ $\bar{x}_1=0$, $\bar{x}_2=2/5$ y $\bar{x}_3=9/5$\\ y el valor objetivo óptimo es $\bar{z}=11/5$.


\end{frame}
%----------------------------------------------
\begin{frame}
\frametitle{Ejemplo}

Aplica el método de las dos fases para resolver:

\begin{center}
\begin{tabular}{lr}
minimizar & $-x_1 + 2x_2 -3x_3$ \\
	 &  \\
s.a. & $x_1+x_2+x_3 =6 $    \\
	 & $-x_1 + x_2 + 2x_3  = 4$  \\
	 & $2x_2 + 3x_3  = 10$  \\
	 & $x_3  \leq 2$  \\
	 & $x_i\geq 0$
\end{tabular}
\end{center}


\end{frame}
%----------------------------------------------------------------------
\end{document}
%----------------------------------------------------------------------


%----------------------------------------------
\begin{frame}
\frametitle{}

\end{frame}
%--------------------------------------------