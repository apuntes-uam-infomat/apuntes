\documentclass{beamer}

%\usepackage{beamerthemetreebars}
\usepackage{graphicx}
%\usepackage{beamerthemesplit}
%\beamertemplateshadingbackground{red!10}{blue!10}

%----------------------------------------------------------------------
% Para que aparezcan varias transparencias en la misma p\'{a}gina
\usepackage{pgfpages}
%\pgfpagesuselayout{4 on 1}[a4paper,border shrink=5mm,landscape]
%----------------------------------------------------------------------
% Suprime los s\'{\i}mbolos de navegaci\'{o}n
\setbeamertemplate{navigation symbols}{}
%---------------------------------------------------------------------

\usepackage[utf8]{inputenc}
%----------------------------------------------------------------
\newcommand{\ep}{\epsilon}
\newcommand{\real}{{\rm I\kern-.17em R}}
\newcommand{\pro}{\mbox{P}}
%-----------------------------------------------------------------

\title[Estad\'{\i}stica: Tema 2]{Tema 2\\
Conjuntos convexos}
\author[Berrendero]
{Jos\'{e} R. Berrendero}
\date{}
\institute{Departamento de Matem\'{a}ticas\\
 Universidad Aut\'{o}noma de Madrid}

%------------------------------------------------------------
\begin{document}
%-------------------------------------------------------------------



\frame{\titlepage}
%----------------------------------------------------------------------
\begin{frame}[plain]
\frametitle{Contenidos del tema 2}

\begin{itemize}
 

  \item Conjuntos convexos. Propiedades básicas y ejemplos.
  \item Cierre e interior de un conjunto convexo.
  \item Teorema de la proyección.
  \item Teoremas de separación.
  \item Caracterización de puntos extremos  y direcciones extremas de  $S=\{x: Ax=b, x\geq 0\}$.
  \item Teorema de representación.
 
\end{itemize}


\end{frame}

%---------------------------------------------------------------------
\begin{frame}
\frametitle{Conjuntos afines y convexos}

\[
y = \theta x_1 + (1-\theta)x_2 = x_2 + \theta (x_1-x_2),\ \ \  x_1,x_2\in\mathbb{R}^n.
\]


\begin{center}
\includegraphics[scale=0.8]{combinacion}
\end{center}

\begin{itemize}
\item 
\textbf{Conjunto afín}: si $x_1,x_2\in S$, entonces $\theta x_1 + (1-\theta)x_2\in S$, para todo $\theta\in \mathbb{R}$.
\item 
\textbf{Conjunto convexo}: si $x_1,x_2\in S$, entonces $\theta x_1 + (1-\theta)x_2\in S$, para todo $\theta\in [0,1]$.
\end{itemize}

\end{frame}
%---------------------------------------------------------------------
\begin{frame}
\frametitle{Ejemplos de conjuntos convexos}

\begin{itemize}
\item \textbf{Hiperplanos}: $S=\{x:\, p^\top x = \alpha\}$, donde $p\in\mathbb{R}^n$, $\alpha\in\mathbb{R}$. 

\

\item \textbf{Semiespacios}: $S=\{x:\, p^\top x \leq \alpha\}$, donde $p\in\mathbb{R}^n$, $\alpha\in\mathbb{R}$. 

\

\item \textbf{Intersección arbitraria} de convexos: Si $S_i$ es convexo para todo $i\in I$, entonces $S=\bigcap_{i=1}^I S_i$ es un conjunto convexo.

\

\item Un \textbf{poliedro} (intersección finita de semiespacios) es un conjunto convexo. Por ejemplo, $S=\{x:\, Ax\leq b,\ x\geq 0\}$ es un conjunto convexo.


\

\item Una \textbf{bola} $B(\bar x,r)=\{x\in\mathbb{R}^n:\, \|x-\bar x\|<r\}$ es un conjunto convexo (para cualquier norma).

%\item El conjunto de \textbf{matrices $n\times n$ simétricas y semidefinidas positivas} $\mathbb{S}^n_+$. 

\end{itemize}
\end{frame}
%---------------------------------------------------------------------
\begin{frame}
\frametitle{Combinaciones afines}

\textbf{Combinación afín} de $x_1,\ldots,x_k \in\mathbb{R}^n$:
\[
y = \lambda_1 x_1+\cdots +\lambda_k x_k,
\]
donde $\lambda_1+\cdots +\lambda_k=1$.

\

\textbf{Cierre afín} de $S$:
\[
\mbox{afin}(S) = \left\{\sum_{i=1}^k \lambda_i x_i:\, x_i\in S,\  \sum_{i=1}^k \lambda_i = 1\right\}.
\]

\begin{itemize}
\item Un conjunto es afín si y solo $S=\mbox{afin}(S)$.

\item Un conjunto es afín si y solo si es la traslación de un subespacio vectorial (único)

\item La dimensión afín de un conjunto es la dimensión de su cierre afín (que a su vez es la dimensión del correspondiente subespacio vectorial).
\end{itemize}


\end{frame}
%---------------------------------------------------------------------
\begin{frame}
\frametitle{Combinaciones convexas y cierre convexo}

\textbf{Combinación convexa} de $x_1,\ldots,x_k \in\mathbb{R}^n$:
\[
y = \lambda_1 x_1+\cdots +\lambda_k x_k,
\]
donde $\lambda_1+\cdots +\lambda_k=1$ y $\lambda_i \geq 0$, para todo $i=1,\ldots,n$.

\




\textbf{Cierre convexo} de $S$:
\[
\mbox{conv}(S) = \left\{\sum_{i=1}^k \lambda_i x_i:\, x_i\in S,\ \lambda_i\geq 0,\ \sum_{i=1}^k \lambda_i = 1\right\}.
\]

\

\begin{itemize}
\item Un conjunto $S$ es convexo si y solo si $S=\mbox{conv}(S)$.
\item $\mbox{conv}(S)$ es el menor conjunto convexo que contiene a $S$.
\end{itemize}


\end{frame}
%---------------------------------------------------------------------
\begin{frame}
\frametitle{Cierre convexo}

\begin{center}
\includegraphics[scale=0.55]{CierreConvexo}
\end{center}

\end{frame}
%---------------------------------------------------------------------
\begin{frame}
\frametitle{Teorema de Carathéodory}

Cualquier punto del cierre convexo de un conjunto de $\mathbb{R}^n$ se puede poner como combinación lineal convexa de a lo más $n+1$ puntos del conjunto.

\



\textbf{Teorema:} Sea $S\subset \mathbb{R}^n$. Si $x\in\mbox{conv}(S)$, entonces $x=\sum_{i=1}^{n+1}\lambda_i x_i$, con $\sum_{i=1}^{n+1}\lambda_i=1$, $\lambda_i\geq 0$, $x_i\in S$, para todo $i=1,\ldots, n$.

\

{\scriptsize
\textbf{Demostración:} Sea $x\in\mbox{conv}(S)$, entonces $x=\sum_{i=1}^{k}\lambda_i x_i$, con $\sum_{i=1}^{k}\lambda_i=1$, $\lambda_i> 0$, $x_i\in S$.

\begin{enumerate}
\item Si $k\leq n+1$, hemos terminado.
\item Si $k > n+1$, $x_2-x_1,\ldots, x_k-x_1$ son linealmente dependientes.
\item Existen $\mu_i$ (alguno estrictamente positivo) tales que $\sum_{i=1}^{k}\mu_i x_i=0$.
\item Tenemos que $x=\sum_{i=1}^k (\lambda_i-\alpha\mu_i)x_i$, donde
\[
\alpha = \min\{\lambda_j/\mu_j:\, \mu_j>0\}:=\lambda_r/\mu_r > 0.
\]
\item Hemos conseguido expresar $x$ como combinación convexa de $k-1$ elementos. Volvemos al paso 1. En un número finito de iteraciones habremos terminado.
\end{enumerate}
}
\end{frame}
%--------------------------------------------------------------------
\begin{frame}
\frametitle{Interior relativo}

Sea $S\subset\mathbb{R}^n$ un convexo cuyo cierre afín es $\mbox{afin}(S)$. El \textbf{interior relativo} de $S$ se define como el conjunto de puntos $x\in S$ tales que existe $r>0$ con $B(x,r)\cap \mbox{afin}(S) \subset S$.

\begin{itemize}
\item ¿Cuál es el interior relativo de $\{x\in \mathbb{R}^3:\, -1\leq x_1\leq 1,\, -1\leq x_2\leq 1,\, x_3=0\}$?
\end{itemize}

\

\textbf{Teorema:} Un conjunto convexo no vacío en $\mathbb{R}^n$ tiene interior relativo no vacío.

\

\textbf{Teorema:} El interior de un conjunto convexo  en $\mathbb{R}^n$ es vacío si y solo si el conjunto está contenido en un hiperplano de $\mathbb{R}^n$.


\end{frame}
%------------------------------------------------------------------------------
\begin{frame}
\frametitle{Cierre e interior de un conjunto convexo}


\textbf{Lema:} Sea $S\subset\mathbb{R}^n$ un conjunto convexo con $\mbox{int}(S)\neq\emptyset$. Sea $x_1\in \bar{S}$, $x_2\in\mbox{int}(S)$. Entonces, $\theta x_1+ (1-\theta) x_2 \in\mbox{int}(S)$, para todo $\theta\in [0,1)$.

\

{\scriptsize
\textbf{Demostración:} Sea $y=\theta x_1+ (1-\theta)x_2$.

\begin{enumerate}
\item Existe $\epsilon>0$ tal que $B(x_2,\epsilon)\subset S$.
\item Sea $\tilde{y}\in B(y,\eta)$, con $\eta=\epsilon(1-\theta)$.
\item Como $x_1\in\bar{S}$, existe $\tilde{x}_1\in S$ tal que 
\[
\|x_1-\tilde{x}_1\| < \frac{\eta-\|\tilde{y}-y\|}{\theta}
\Leftrightarrow \|y-\tilde{y}\| + \theta \|x_1 - \tilde{x}_1\| < \eta.
\]
\item Sea $\tilde{x}_2=(\tilde{y} - \theta \tilde{x}_1)/(1-\theta) \Leftrightarrow 
\tilde{y}=\theta \tilde{x}_1 + (1-\theta) \tilde{x}_2$.
\item Se verifica
\[
\|x_2-\tilde{x}_2\| = \frac{\|y-\theta x_1 - \tilde{y}+\theta\tilde{x}_1\|}{1-\theta}\leq  
\frac{1}{1-\theta}(\|y-\tilde{y}\| + \theta \|x_1 - \tilde{x}_1\|)<\frac{\eta}{1-\theta} = \epsilon.
\]
\item Por 1 y 5, $\tilde{x}_2\in S$. Por 4, $\tilde{y}\in S$. Como $B(y,\eta)\subset S$, $y\in \mbox{int}(S)$.



\end{enumerate}
}
\end{frame}
%----------------------------------------------
\begin{frame}
\frametitle{Cierre e interior de un conjunto convexo}

%{\bf Algunos corolarios:}


\begin{enumerate}

\item Si $S\subset\mathbb{R}^n$ es convexo, entonces tanto $\mbox{int}(S)$ como $\bar{S}$ son conjuntos convexos.

\

\item Si $S\subset\mathbb{R}^n$ es convexo, entonces $\mbox{int}(S)=\mbox{int}(\bar{S})$. Si además $\mbox{int}(S)\neq \emptyset$, entonces $\bar{S}=\overline{\mbox{int}(S)}$.


\

\item Si $S\subset\mathbb{R}^n$ es convexo, entonces $\partial S=\partial \bar{S}$.


\end{enumerate}


\end{frame}
%----------------------------------------------
\begin{frame}
\frametitle{Teorema de la proyección}

{\bf Teorema:} Sea $S\subset\mathbb{R}^n$ un conjunto convexo, no vacío y cerrado. Sea $y\in \mathbb{R}^n$. Existe un \textbf{único} $\bar x\in S$ (la proyección de $y$ sobre $S$) tal que 
\[
\|y-\bar x\| \leq \|y - x\|, \ \ \mbox{para todo}\ x\in S.
\]
Además, $\bar x$ es la proyección de $y$ sobre $S$ si y solo si
\begin{equation}
\label{eq.proyeccion}
(y-\bar x)^\top (x - \bar{x}) \leq 0.
\end{equation}


\

\begin{itemize}
\item ¿Cómo queda la condición (\ref{eq.proyeccion}) cuando $S$ es un conjunto afín?
\item La aplicación $P:\mathbb{R}^n\to S$, que a cada $y$ le hace corresponder su proyección $P(y)$ sobre $S$, es continua.
\end{itemize}

\end{frame}
%----------------------------------------------
\begin{frame}
\frametitle{Teorema del hiperplano separador}

{\bf Teorema:} Sea $S\subset\mathbb{R}^n$ un conjunto convexo, no vacío y cerrado. Sea $y\notin S$. Entonces existe $p\in\mathbb{R}^n$, $p\neq 0$, y $\alpha\in\mathbb{R}$ tales que
\begin{itemize}
\item $p^\top x\leq \alpha$, para todo $x\in S$.
\item $p^\top y > \alpha$ 
\end{itemize}

\

\textbf{Demostración:} 

\begin{enumerate}
\item Sea $\bar x$ la proyección de $y$ sobre $S$.

\item Se verifica $(y-\bar{x})^\top (x-\bar{x})\leq 0$,   para todo $x\in S$.

\item Definimos $p=y-\bar{x}$ y $\alpha=p^\top\bar{x}$. Comprobamos que cumplen las condiciones requeridas:
\begin{itemize}
\item Si $x\in S$, $p^\top x = (y-\bar{x})^\top (x-\bar{x}) + \alpha \leq \alpha$.
\item $p^\top y = (y-\bar{x})^\top (y-\bar{x}) + \alpha =\|y-\bar x\|^2 +\alpha > \alpha$.
\end{itemize}

\end{enumerate}



\end{frame}
%----------------------------------------------
\begin{frame}
\frametitle{Teorema del hiperplano soporte}

{\bf Teorema:} Sea $S\subset\mathbb{R}^n$ un conjunto convexo con interior no vacío. Sea $\bar{x}\in \partial S$, la frontera de $S$. Entonces existe $p\in\mathbb{R}^n$, $p\neq 0$, tal que $p^\top (x-\bar{x})\leq 0$, para todo $x\in S$.

\

{\scriptsize
\textbf{Demostración:} 

\begin{enumerate}
\item $\bar{x}\notin \mbox{int}(S)=\mbox{int}(\bar{S})$.

\item Para todo $k\in\mathbb{N}$, existe $y_k\in B(\bar{x}, 1/k) \cap \bar{S}^c$

\item Existe $p_k\in\mathbb{R}^n$, $\|p_k\|=1$, con $p^\top_k y_k > p_k^\top x$ para todo $x\in \bar{S}$.

\item Tenemos $y_k\to \bar{x}$. Para una subsucesión, $p_k\to p$ con $\|p\|=1$.

\item Tomando límites en 3, $p^\top (x-\bar{x})\leq 0$, para todo $x\in \bar{S}$.

\end{enumerate}
}
\end{frame}
%----------------------------------------------
\begin{frame}
\frametitle{Separación de dos convexos disjuntos}

{\bf Teorema:} Sean $S_1,S_2\subset\mathbb{R}^n$ conjuntos convexos, no vacíos, tales que $S_1\cap S_2=\emptyset$.  Entonces existe $p\in\mathbb{R}^n$, $p\neq 0$, tal que
\[
\inf\{p^\top x:\, x\in S_1\} \geq \sup \{p^\top x:\, x\in S_2\}.
\]

\

{\bf Demostración:} La idea es considerar
\[
S = \{x\in \mathbb{R}^n:\, x = x_1-x_2,\ x_1\in S_1,\ x_2\in S_2\}
\]
y aplicar los teoremas de separación.

\end{frame}
%----------------------------------------------
\begin{frame}
\frametitle{Teoremas de la alternativa}

{\bf Teorema (Gordan):} Sea $A$ una matriz $m\times n$. Uno y solo uno de los sistemas siguientes tiene solución:
\begin{itemize}
\item $Ax<0$ para algún $x\in\mathbb{R}^n$.
\item $A^\top p=0$, $p\geq 0$ y $p\neq 0$ para algún $p\in\mathbb{R}^m$.
\end{itemize} 

\

{\scriptsize

{\bf Demostración:} 

\begin{itemize}
\item (1) tiene solución $\Rightarrow$ (2) no la tiene.

\

Sea $\bar{x}$ la solución de (1) y supongamos que $\bar{p}$ fuese una solución de (2). Consideramos $\bar{p}^\top A\bar{x}$.

\

\item (1) no tiene solución $\Rightarrow$ (2) sí la tiene.

\

Los conjuntos $S_1=\{z\in\mathbb{R}^m:\, z=Ax,\ x\in\mathbb{R}^n\}$ y $S_2=\{z\in\mathbb{R}^m:\, z<0\}$ son convexos, no vacíos y disjuntos. Existe $p\neq 0$ con $p^\top Ax\geq p^\top z$ para $x\in\mathbb{R}^n$ y $z<0$. Se demuestra que $p$ resuelve (2).

\end{itemize}

}

\end{frame}
%----------------------------------------------
\begin{frame}
\frametitle{Teoremas de la alternativa}


\textbf{Ejemplo:} ¿Existen $x_1$, $x_2$ y $x_3$ tales que 
$x_1+x_2+x_3<0$, $x_1>0$ y $2x_1-x_2-x_3<0$?


\


{\bf Teorema (Farkas):} Sea $A$ una matriz $m\times n$ y $c\in \mathbb{R}^n$. Uno y solo unos de los sistemas siguientes tiene solución:
\begin{enumerate}
\item $Ax\leq 0$ y $c^\top x>0$  para algún $x\in\mathbb{R}^n$.
\item $A^\top y=c$, $y\geq 0$  para algún $y\in\mathbb{R}^m$.
\end{enumerate} 

\

{\scriptsize

{\bf Demostración:} Ejercicio

Probar que (1) no tiene solución cuando (2) sí la tiene es muy fácil. Cuando (2) no tiene solución considera $S=\{x\in\mathbb{R}^n:\, x=A^\top y,\ y\geq 0\}$ y observa que $c\notin S$.
}






\end{frame}
%----------------------------------------------
\begin{frame}
\frametitle{Teorema de representación de poliedros}



Los resultados hasta el final del tema se refieren al poliedro $S=\{x:\, Ax=b, x\geq 0\}$, el conjunto factible de un problema de optimización lineal (en forma estándar).

\

Suponemos que $A$ es una matriz $m\times n$ ($m<n$) con rango $r(A)=m$. 


\

Vamos a representar los puntos de $S$ en términos de los puntos extremos (vértices) de $S$ y sus direcciones extremas.


\

\textbf{Cualquier punto de $S$ se puede expresar como una combinación convexa de sus puntos extremos más una combinación lineal positiva de sus direcciones extremas.}




\end{frame}
%----------------------------------------------
\begin{frame}
\frametitle{Soluciones factibles básicas}

Podemos dividir las columnas de $A$ es dos grupos $B$ y $N$, donde $B$ es $m\times m$ con $r(B)=m$.

\

\[
Ax=b \Leftrightarrow (B,N) {x_B \choose x_N}=b \Leftrightarrow B x_B + Nx_n = b 
\]

\

Si hacemos $x_N=0$, $x_B=B^{-1}b$ y se verifica $B^{-1}b\geq 0$, obtenemos unos puntos especiales de $S$ que se llaman  \textbf{soluciones factibles básicas}.

\


Algunas soluciones del sistema compatible indeterminado $Ax=b$, con $r(A)=r(A,b)=m<n$ se consiguen fijando $n-m$ incógnitas como 0 y despejando las $m$ incógnitas restantes. Si estas soluciones son no negativas, son soluciones factibles básicas.
 
\end{frame}
%----------------------------------------------
\begin{frame}
\frametitle{Puntos extremos}

\textbf{Definición:} Sea $S\subset\mathbb{R}^n$ un conjunto convexo no vacío. Se dice que $x\in S$ es un \textbf{punto extremo} de $S$ si $x=\lambda x_1 + (1-\lambda) x_2$, con $x_1,x_2\in S$, $\lambda\in (0,1)$, implica que $x=x_1=x_2$.

\




\

Es decir, $x$ es un punto extremo si no está en el interior (relativo) del segmento definido por otros dos puntos del conjunto.

\

\

Piensa ejemplos de conjuntos convexos: con un único punto extremo, con un número finito mayor que uno de puntos extremos, con infinitos puntos extremos, sin puntos extremos.





\end{frame}
%----------------------------------------------
\begin{frame}
\frametitle{Direcciones extremas}


\textbf{Definición:} Sea $S\subset\mathbb{R}^n$ un conjunto convexo no vacío. Se dice que $d\in \mathbb{R}^n$, $d\neq 0$, es una \textbf{dirección} de $S$ si para todo $x\in S$ y para todo $\lambda\geq 0$, se cumple $x + \lambda d \in S$. 

\

¿Qué condiciones caracterizan las direcciones de $S=\{x:\, Ax=b, x\geq 0\}$?

\

\textbf{Definición:} Sea $S\subset\mathbb{R}^n$ un conjunto convexo no vacío. Se dice que $d\in \mathbb{R}^n$, $d\neq 0$, es una \textbf{dirección extrema } de $S$ si $d=\lambda_1 d_1 + \lambda_2 d_2$, con $d_1,d_2$ direcciones de $S$, $\lambda_1,\lambda_2 > 0$, implica que $d_1= \alpha d_2$, para algún $\alpha>0$.


\



Piensa ejemplos de conjuntos convexos: con dos direcciones extremas, con una única dirección extrema, sin direcciones extremas. 


\end{frame}


%----------------------------------------------
\begin{frame}
\frametitle{Caracterización de puntos extremos}

{\bf Teorema:} $x$ es un punto extremo de $S$ si y solo si $x$ es una solución factible básica.
 
\

{\scriptsize
{\bf Demostración:} $(\Leftarrow)$ Ejercicio. Demostramos $(\Rightarrow)$.

\begin{enumerate}
\item  Reordenando las coordenadas $x=(x_1,\ldots,x_k,0,\ldots,0)^\top$, con $x_i>0$. Spg. $k\geq 1$ (el caso $x=0$ es trivial).

\

\item Reordenamos igual las columnas de $A$: $A=(a_1,\ldots,a_k,\mbox{resto de columnas})$.

\

\item Veamos que $a_1,\ldots,a_k$ son linealmente independientes. Si no fuera así existiría un vector $\lambda \neq 0$ tal que $A\lambda=0$.

\

\item Para $\alpha> 0$ suficientemente pequeño, $\bar x_1=x+\alpha\lambda$ y $\bar x_2=x-\alpha\lambda$ pertenecen a $S$ $\Rightarrow$ $x=(\bar x_1+\bar x_2)/2$ no es punto extremo.

\

\item Si $k < m$, añadimos columnas de manera que las columnas de $B=(a_1,\ldots,a_k,a_{k+1},\ldots,a_m)$ formen una base.
\end{enumerate}
} 
\end{frame}
%----------------------------------------------
\begin{frame}
\frametitle{Caracterización de puntos extremos}

La demostración implica que un punto extremo  no puede tener más de $m$ coordenadas estrictamente positivas. El recíproco no es cierto (veremos algún ejemplo más adelante en los ejercicios).

\

\

Si una solución factible básica tiene $k<m$ coordenadas estrictamente positivas se llama \textbf{solución factible básica degenerada}.

\

\

En el caso degenerado puede haber dos bases distintas $B$ y $B'$ que representen el mismo punto extremo (¿por qué?).


\end{frame}
%----------------------------------------------
\begin{frame}
\frametitle{Caracterización de puntos extremos}


{\bf Teorema:} El número de puntos extremos de $S$ es finito 

(¿Por qué?)

\

{\bf Teorema:} $S$ tiene al menos un punto extremo. 

(La demostración es muy parecida a la del teorema de Carathéodory.)


\



\

{\bf Ejemplo:} Obtener las soluciones factibles básicas del problema:

\begin{center}
\begin{tabular}{lr}
minimizar & $-x_1 + 2x_2$ \\
	 &  \\
s.a. & $x_1+2x_2 \leq 2$    \\
	 & $2x_1 + x_2\leq 1$  \\
	 & $x_1\geq 0$\\
	 & $x_2\geq 0$
\end{tabular}
\end{center}

\end{frame}
%----------------------------------------------
\begin{frame}
\frametitle{Caracterización de direcciones extremas}

Es fácil demostrar que $d$ es una dirección de $S$  si y solo si $Ad=0$ y $d\geq 0$. El resultado siguiente caracteriza las direcciones extremas:

\

{\bf Teorema:} $d\in \mathbb{R}^n$ es una dirección extrema de $S$ si y solo si $A=(B,N)$, donde $B$ es una matriz $m\times m$ básica, de manera que $B^{-1}a_j\leq 0$ para alguna de las columnas de $N$ y $d=\alpha {-B^{-1}a_j\choose e_j}$ para algún $\alpha>0$, donde $e_j\in\mathbb{R}^{n-m}$ es un vector de ceros salvo un 1 en la posición correspondiente a $a_j$. 

\

{\scriptsize
{\bf Demostración:} 



$(\Leftarrow)$ Ejercicio. 

$(\Rightarrow)$ No la hacemos. Es totalmente análoga a la del teorema de caracterización de puntos extremos.
}

\

{\bf Ejemplo:} Calcula las direcciones extremas del conjunto 
\[
S = \{(x_1,x_2)\in\mathbb{R}^2:\, x_2-x_1\leq 1,\, x_2\leq 2,\, x_1\geq 0,\, x_2\geq 0\}.
\]

\end{frame}
%----------------------------------------------
\begin{frame}
\frametitle{Teorema de representación}

{\bf Teorema:} Sean $x_1,\ldots,x_k$ los puntos extremos de $S$ y sean $d_1,\ldots,d_\ell$ sus direcciones extremas. Entonces,
\[
x\in S \Leftrightarrow x = \sum_{i=1}^k \lambda_i x_i + \sum_{j=1}^\ell
 \mu_j d_j,
\]
donde $\lambda_i\geq 0$, $\mu_j\geq 0$, $\sum_{i=1}^k \lambda_i=1$.

\

{\bf Consecuencias:}

\begin{itemize}
\item Si $S$ es acotado, cualquier punto de $S$ es combinación lineal convexa de sus puntos extremos.


\item $S$ tiene al menos una dirección extrema si y solo si $S$ no es acotado.
\end{itemize}

\

¿Es el número de direcciones extremas siempre un número finito?
 



\end{frame}

%----------------------------------------------
\begin{frame}
\frametitle{Demostración del teorema de representación}

{\scriptsize
\[
C = \{x\in\mathbb{R}^n:\, x = \sum_{i=1}^k \lambda_i x_i + \sum_{j=1}^\ell
 \mu_j d_j,\ \lambda_i\geq 0,\ \mu_j\geq 0,\ \sum_{i=1}^k \lambda_i=1\}
\]

\begin{itemize}
\item $C\neq\emptyset$
\item $C$ es convexo y cerrado.
\item $C\subset S$

\

\item Para probar $S\subset C$, supongamos que existe $z\in S$ con $z\notin C$.

\

\item Existe $p\in\mathbb{R}^n$ ($p\neq 0$) y $\alpha \in\mathbb{R}$ tales que $p^\top z>\alpha$ y $p^\top x\leq \alpha$ para todo $x\in C$.
\item Como consecuencia, $p^\top d_j \leq 0$ y $p^\top x_i\leq \alpha \Rightarrow
p^\top z > p^\top x_i,\ i=1,\ldots,k$.
\item Sea $\bar{x}$ el punto extremo tal que $p^\top \bar{x} =\max_i p^\top x_i$.
\item Consideramos la partición $A=(B,N)$ en parte básica y no básica correspondiente  a $\bar{x}$.
\item Se comprueba $0 < p^\top z- p^\top \bar{x}=(p_N^\top - p_B^\top B^{-1} N) z_N$, lo que implica que para algún $j>m$ se tiene $z_j>0$ y $p_j-p_B^\top B^{-1} a_j>0$.
\item Se demuestra  que $y_j:= B^{-1}a_j\nleq 0$. (Por reducción al absurdo ya que si $y_j\leq 0$, entonces $d_j= {-y_j\choose e_j}$ es una dirección extrema y por tanto $p_j-p_B^\top B^{-1}a_j\leq 0$.)
\end{itemize}
}

\end{frame}
%----------------------------------------------
\begin{frame}
\frametitle{Demostración del teorema de representación}

{\scriptsize

\begin{itemize}

\item Definimos el vector
\[
x = {\bar{b}\choose 0} + \alpha {-y_j\choose e_j},
\]
donde $\bar{b} = B^{-1}b$ y $\alpha =\min\left\{\frac{\bar{b}_i}{y_{ij}}:\, y_{ij}>0 \right\}:= \frac{\bar{b}_r}{y_{rj}}>0$.

\

\item Comprobamos $x\in S$, $x_r=0$, $x_j>0$. 

\item Las columnas $a_1,a_2,\ldots,a_{r-1},a_{r+1},a_j$
son linealmente independientes, es decir, $x$ es un punto extremo de $S$.


\

\item Pero $p^\top x = p^\top \bar{x} + \alpha (p_j-p_B^\top B^{-1}a_j)> p^\top \bar{x}$.

\item Una contradicción ya que habíamos supuesto $p^\top \bar{x} =\max_i p^\top x_i$.
\end{itemize}
}

\end{frame}
%----------------------------------------------------------------------
\end{document}
%----------------------------------------------------------------------


%----------------------------------------------
\begin{frame}
\frametitle{}

\end{frame}
%--------------------------------------------