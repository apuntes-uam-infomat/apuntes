
\documentclass[11pt,a4paper,twoside]{article}


%-------------------------------------------------------------

\usepackage[utf8]{inputenc}
\usepackage{graphicx,hyperref,amsmath,natbib,bm,url,microtype}
\usepackage{paralist}
\usepackage[spanish]{babel}
\usepackage[a4paper,text={16.5cm,25.2cm},centering]{geometry}
\usepackage[compact,small]{titlesec}
\setlength{\parskip}{1.2ex}
\setlength{\parindent}{0em}
\clubpenalty = 10000
\widowpenalty = 10000
\usepackage{kpfonts}
\usepackage[T1]{fontenc}
\pagestyle{empty}

\newcounter{problem} \setcounter{problem}{1}
\newcommand{\ex}{\noindent {\sf \bf \theproblem}\addtocounter{problem}{1}.\ }

%-------------------------------------------------------------------
\begin{document}
%-------------------------------------------------------------------
% Encabezamiento



%-----------------------------------
\hrule
\bigskip
\textbf{Investigación Operativa   \hfill Curso 2015/2016\\
Grado en Matemáticas}
\bigskip
\hrule
%--------------------------------

\


\begin{center}  {\bf \large
Relaci\'{o}n 1 de problemas}
\end{center}
\bigskip


\ex Una empresa de reciclaje usa papel y tela
desechados para fabricar dos tipos distintos de papel
reciclado. Cada tanda de papel reciclado de clase A
requiere 20 kg de tela y 180 kg de papel y produce un beneficio
de 500 euros, mientras que
cada tanda de papel reciclado de clase B requiere 10 kg
de tela y 150 kg de papel y produce un beneficio
de 250 euros. La compa\~n\'{\i}a dispone de 100 kg
de tela y 660 kg de papel. ?`Cu\'antas tandas debe fabricar
de cada tipo?

\

\ex La empresa {\it Animales Salvajes S.A.} cr\'{\i}a faisanes y
perdices para repoblar el bosque y dispone de sitio para
criar 100 p\'ajaros durante la temporada. Criar un fais\'an
cuesta 20 euros y criar una perdiz cuesta 30 euros. La
fundaci\'on {\it Vida Animal} paga a {\it Animales Salvajes
S.A.} por los p\'ajaros de forma que se obtiene un beneficio
de 14 euros por cada fais\'an y 16 euros por cada perdiz. La
empresa dispone de 2400 euros para cubrir costes. ?`Cu\'antas
perdices y cu\'antos faisanes debe criar?

\

\ex    La siguiente tabla da el porcentaje de prote\'{\i}nas, grasas y
carbohidratos, para cinco alimentos
b\'asicos, A, B, C, D y E :



\begin{center}
\begin{tabular}{r|r|r|r}
 & Prote\'{\i}nas & Grasas & Carbohidratos \\ \hline 
A&8.6&1.1&56.4\\
B&25.4&35.4&0.0\\ C&30.0&7.5&0.0\\ D&22.1&7.0&0.0\\ E&2.5&0.0&40.7\\
\end{tabular}
\end{center}




Los precios por 100 g de estos alimentos (dados en el mismo orden
de la tabla) son 5, 17, 37, 10, 15. Si una persona necesita
consumir como m\'{\i}nimo 75 gramos de prote\'{\i}nas, 90 de grasas y 300 de
hidratos de carbono, plantea el problema de minimizaci\'on  para
calcular la dieta alimenticia de m\'{\i}nimo coste.

\

\ex La siguiente tabla indica los requerimientos
m\'{\i}nimos de personal de enfermer\'{\i}a en un hospital en distintos
per\'{\i}odos del d\'{\i}a.



\begin{center}  \begin{tabular}{c|c}
 Per\'{\i}odo del d\'{\i}a&N\'umero de enfermeros/as requerido\\ \hline
8:00-12:00&140\\ 12:00-16:00&120\\ 16:00-20:00&160\\
20:00-24:00&90\\ 24:00-4:00&30\\ 4:00-8:00&60\\ 
\end{tabular}
\end{center}



El trabajo est\'a organizado en se\-is tur\-nos de ocho ho\-ras
ca\-da uno. Ca\-da cua\-tro ho\-ras co\-mien\-za un nue\-vo turno.
Por tanto, cada turno coincide durante las cuatro primeras horas
con el turno anterior y durante las cuatro \'ultimas con el turno
siguiente. Plantea el problema de optimizaci\'on para determinar
cu\'antos enfermeros/as deben formar parte de cada turno, de forma
que el n\'umero total sea m\'{\i}nimo y  se cumplan los requerimientos
m\'{\i}nimos de personal.

\


\ex Un pastelero dispone de 150 kg de harina, 22 kg de az\'ucar y 27.5
 kg de mantequilla para elaborar dos tipos de pasteles ($A$ y
 $B$). Cada caja de pasteles de tipo $A$ requiere 3 kg de harina,
 1 kg de az\'ucar y 1 kg de mantequilla y su venta le reporta un
 beneficio de 20 euros. Cada caja de pasteles
  de tipo $B$ requiere 6 kg de harina,
 0.5 kg de az\'ucar y 1 kg de mantequilla y su venta le reporta un
 beneficio de 30 euros.
 \begin{compactitem}
 \item[(a)] ?`Cu\'antas cajas de cada
 tipo debe elaborar el pastelero de manera que se maximicen sus
 ganancias? Resuelve el problema gr\'aficamente.
 \item [(b)] Supongamos que la
 cantidad de harina disponible aumenta en un kg. ?`Cu\'anto aumenta
 el beneficio del pastelero? Contesta a la misma cuesti\'on para un
 aumento de un kg en la cantidad de az\'ucar y mantequilla.
 \end{compactitem}
 
 \
 
 \ex Resuelve por separado cada uno de los cuatro problemas de
programaci\'on lineal que pueden escribirse al sustituir en la siguiente
formulaci\'on

\begin{tabular}{ll}
Maximizar & $3x + 2y$ \\
sujeto a  & \\
& $ 2x + 3y \bigodot 6  $\\
& $ 2x +  y \bigoplus 4  $\\
& $ x \ge 0,\; y\ge 0 $
\end{tabular}

los s\'{\i}mbolos $\bigodot$ y $\bigoplus$ por todas las combinaciones
posibles de signos $\ge$ y $\le$.

\

\ex Considera el siguiente problema  lineal:

\begin{tabular}{ll}
Minimizar & $-3x_1-2x_2$ \\
sujeto a & \\
&$-2x_1+x_2\leq 2$\\
&$x_1-2x_2\leq 2$\\
&$x_1\geq 0,\ x_2\geq 0$
\end{tabular}

\begin{compactitem}
\item[(a)] Comprueba gr\'aficamente que el problema tiene soluciones factibles
pero no tiene soluci\'on \'optima.
\item[(b)]  Supongamos que se incorpora la restricci\'on
$
x_1 + x_ 2 \ge 10.
$
Encuentra una soluci\'on \'optima para el nuevo problema o comprueba que no existe.
\end{compactitem}

\

\ex Considera el siguiente problema  lineal:

\begin{tabular}{ll}
Maximizar & $x_1+x_2$ \\
sujeto a & \\
&$-2x_1+2x_2\leq 1$\\
&$16x_1-14x_2\leq 7$\\
&$x_1\geq 0,\ x_2\geq 0$
\end{tabular}


\begin{compactitem}
\item[(a)] Encuentra gr\'aficamente una soluci\'on \'optima $(x_1,x_2)$.
\item[(b)] Supongamos adem\'as que cada variable est\'a restringida a
tomar valores enteros. ?`Se obtiene un punto factible al redondear
cada componente de la soluci\'on \'optima al entero m\'as pr\'oximo?
\item[(c)] Encuentra gr\'aficamente una soluci\'on \'optima entera.
\end{compactitem}


\

\ex Una propiedad conocida de la mediana de un conjunto de datos $y_1,\ldots,y_n$ es que  minimiza en $\theta$ el valor de $\sum_{i=1}^n |y_i-\theta |$. Plantea este problema de optimización como un problema  lineal en forma estándar. 


%------------------------------------------------------------------------
\end{document}
%-------------------------------------------------------------------------

