
\documentclass[12pt,a4paper,twoside]{article}


%-------------------------------------------------------------

\usepackage[utf8]{inputenc}
\usepackage{graphicx,hyperref,amsmath,natbib,bm,url,microtype}
\usepackage{paralist}
\usepackage[spanish]{babel}
\usepackage[a4paper,text={16.5cm,25.2cm},centering]{geometry}
\usepackage[compact,small]{titlesec}
\setlength{\parskip}{1.2ex}
\setlength{\parindent}{0em}
\clubpenalty = 10000
\widowpenalty = 10000
\usepackage{kpfonts}
\usepackage[T1]{fontenc}
\pagestyle{empty}

\newcounter{problem} \setcounter{problem}{1}
\newcommand{\ex}{\noindent {\sf \bf \theproblem}\addtocounter{problem}{1}.\ }
\usepackage[normalem]{ulem}

%-------------------------------------------------------------------
\begin{document}
%-------------------------------------------------------------------
% Encabezamiento



%-----------------------------------
\hrule
\bigskip
\textbf{Investigación Operativa   \hfill Curso 2015/2016\\
Grado en Matemáticas}
\bigskip
\hrule
%--------------------------------

\


\begin{center}  {\bf \large
Relaci\'{o}n 2 de problemas}
\end{center}
\bigskip


\ex Sea $S\subset\mathbb{R}^n$. Dado $\epsilon\geq 0$,la dilatación de $S$ se define como $S_\epsilon=\{x:\, d(x,S)\leq \epsilon\}$, donde $d(x,S)=\inf_{y\in S} \|x-y\|$. La erosión de $S$ se define como $S_{-\epsilon}=\{x:\, B(x,\epsilon)\subset S \}$, donde $B(x,\epsilon)$ es la bola cerrada con centro $x$ y radio $\epsilon$. Demuestra que si $S$ es convexo, entonces tanto $S_\epsilon$ como $S_{-\epsilon}$ son conjuntos convexos. 


\

\ex Sean $x_0,\ldots,x_k\in \mathbb{R}^n$. Considera los puntos que están más cerca de $x_0$ que de otro de los puntos $x_i$, es decir,
\[
V=\{x\in\mathbb{R}^n:\, \|x-x_0\|\leq \|x-x_i\|,\, i=1,\ldots, k\}.
\]
El conjunto $V$ se llama región de Voronoi de $x_0$ respecto de $x_1,\ldots,x_k$.
\begin{compactitem}
\item[(a)] Demuestra que $V$ es un poliedro. Determina una matriz $A$ y un vector $b$ tales que $V=\{x\in\mathbb{R}^n:\, Ax\leq b\}$.
\item[(b)] Recíprocamente, dado un poliedro $P$ con interior no vacío, determina $x_0,\ldots,x_k$ de manera que $P$ sea la  región de Voronoi de $x_0$ respecto de $x_1,\ldots,x_k$.
\end{compactitem}


\



 \ex Sea $S\subset \mathbb{R}^n$ un conjunto convexo no vacío y sean $\lambda_1>0$ y $\lambda_2>0$. 
\begin{compactitem}
\item[(a)] Demuestra que $(\lambda_1+\lambda_2)S=\lambda_1S + \lambda_2S$.
\item[(b)] Determina razonadamente si es cierta o no la propiedad del apartado anterior cuando el conjunto $S$ no es convexo.
\end{compactitem}
% Examen: 14 de mayo de 2015


\

\ex Sea $S\subset \mathbb{R}^n$ un conjunto cerrado tal que si $x_1,x_2\in S$, entonces $(x_1+x_2)/2\in S$. Demuestra que $S$ es convexo.


\

\ex Sea $X_1,\ldots, X_n$ una muestra de $n$ vectores independientes e idénticamente distribuidos, con distribución uniforme en el cuadrado unidad $S=[0,1]^2$. Consideramos la variable aleatoria $N_n$ correspondiente al número de vértices del cierre convexo de $X_1,\ldots, X_n$. 
\begin{compactitem}
\item[(a)] Escribe una función  en {\tt R} que  dos argumentos $n$ y $B$, y que dé como resultado un vector con $B$ realizaciones de la variable $N_n$.  
\item[(b)] Genera $B=10000$ realizaciones de la v.a. $N_n$ para $n=100$. Calcula la media y la desviación típica de los valores obtenidos y representa el correspondiente  histograma. ¿A qué distribución se parece el histograma obtenido?
\end{compactitem}


\

\ex Demuestra que si $S\subset \mathbb{R}^n$ es un conjunto convexo, entonces  $\mbox{int}(S)$ y $\bar{S}$ también son conjuntos convexos.



\

\ex Sea $S\subset \mathbb{R}^n$  un conjunto convexo con $\mbox{int}(S)\neq \emptyset$. \sout{Demuestra $\mbox{int}(\bar{S})=\mbox{int}(S)$.} Demuestra $\bar{S}=\overline{\mbox{int}(S)}$.


\

\ex Da un ejemplo para probar que el cierre convexo, $\mbox{conv}(S)$, de un conjunto cerrado no es necesariamente cerrado. Utiliza el teorema de Caratheodory para probar que si $S$ es compacto, entonces $\mbox{conv}(S)$ es compacto.
%Da una condición suficiente que permita asegurar que $\mbox{conv}(S)$ es cerrado cuando $S$ lo es.



\

\ex  Encuentra un ejemplo que muestre que la implicaci{\'o}n
$$
S_1 \hbox{ y } S_2 \hbox{ son convexos cerrados } \Rightarrow
S_1+S_2 \hbox { es cerrado,}
$$
no es cierta en general. Prueba que esta  implicaci{\'o}n es cierta
cuando al menos uno de los dos conjuntos $S_1$ o $S_2$ se supone
compacto.


\

\ex Demuestra que un conjunto convexo cerrado es igual a la intersección de todos los semiespacios cerrados que lo contienen.



\





\ex Sean $S_1$ y $S_2$ dos conjuntos convexos no vac{\'\i}os de
$\mathbb{R}^n$ tales que $S_1\cap S_2=\emptyset$. Demuestra que existe
$p\in\mathbb{R}^n$, $p\neq 0$, tal que
\[
\inf\,\{p^\top x:\, x\in S_1\}\geq \sup\,\{p^\top x:\, x\in S_2\}.
\]
 Si,
adem{\'a}s, los conjuntos son cerrados y uno de ellos es acotado,
demostrar que existe $p\in\mathbb{R}^n$, $p\neq 0$, y $\,\epsilon>0$
tales que
\[
\inf\,\{p^\top x:\, x\in S_1\}\geq \epsilon+\sup\,\{p^\top x:\, x\in S_2\}.
\]
 {\em  Sugerencia: Para la primera parte, considera $S=\{x_1-x_2:\, x_1\in
S_1,\ x_2\in S_2\}$ y aplica alg{\'u}n teorema de separaci{\'o}n.}

\

\ex La función soporte de un conjunto compacto $C\subset\mathbb{R}^n$ se define de la forma siguiente:
\[
S_C(p) = \sup\{p^\top x:\, x\in C\}.
\]
Sean $C$ y $D$ dos conjuntos convexos y compactos. Demuestra que $C=D$ si y solo si las correspondientes funciones soporte son iguales.



\

\ex Demuestra que si $x$ es una solución factible básica de $S=\{x:\, Ax=b, x\geq 0\}$, entonces es un punto extremo de $S$.



\

\ex Halla los puntos extremos de los siguientes conjuntos:
\begin{compactitem}
\item[(a)] $S=\{(x_1,x_2)\in \mathbb{R}^2 :\, x_1+2x_2\geq 2,\,
-x_1+x_2=4,\, x_1,x_2\geq 0\}$.
\item[(b)] $S=\{(x_1,x_2,x_3)\in \mathbb{R}^3 :\, x_1+x_2+x_3\leq
10,\, -x_1+2x_2=4,\, x_1,x_2,x_3\geq 0\}$.

\end{compactitem}


%------------------------------------------------------------------------
\end{document}
%-------------------------------------------------------------------------


