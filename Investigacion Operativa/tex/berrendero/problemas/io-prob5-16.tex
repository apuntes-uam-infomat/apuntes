
\documentclass[12pt,a4paper,twoside]{article}


%-------------------------------------------------------------

\usepackage[utf8]{inputenc}
\usepackage{graphicx,hyperref,amsmath,natbib,bm,url,microtype}
\usepackage{paralist}
\usepackage[spanish]{babel}
\usepackage[a4paper,text={16.5cm,25.2cm},centering]{geometry}
\usepackage[compact,small]{titlesec}
\setlength{\parskip}{1.2ex}
\setlength{\parindent}{0em}
\clubpenalty = 10000
\widowpenalty = 10000
\usepackage{kpfonts}
\usepackage[T1]{fontenc}
\pagestyle{empty}

\newcounter{problem} \setcounter{problem}{1}
\newcommand{\ex}{\noindent {\sf \bf \theproblem}\addtocounter{problem}{1}.\ }

%-------------------------------------------------------------------
\begin{document}
%-------------------------------------------------------------------
% Encabezamiento



%-----------------------------------
\hrule
\bigskip
\textbf{Investigación Operativa   \hfill Curso 2015/2016\\
Grado en Matemáticas}
\bigskip
\hrule
%--------------------------------

\


\begin{center}  {\bf \large
Relaci\'{o}n 5 de problemas}
\end{center}




\ex Considera el problema max $x_1+x_2^2+x_3$ sujeto a $x_1^2+x_2^2+x_3^2\leq b$, donde $0<b<1/2$.
\begin{compactitem}
\item[(a)] Resuélvelo en función de $b$ utilizando las condiciones KKT.
\item[(b)] Sea $F(b)$ la función que da el valor objetivo óptimo en función de $b$. ¿Qué relación existe entre esta función y los multiplicadores de KKT?
\end{compactitem}


\

\ex Sean $f$ y $g_i$ ($i=1,\ldots,m$) funciones con derivadas
parciales continuas en $\mathbb{R}^n$. Consideremos el problema
\[
\mbox{(P)}\ \ \min f(x) \ \ \mbox{s.a.}\ \ g_i(x)=0,\ \ i=1,\ldots, m.
\]
Sea $\overline{x}\in\mathbb{R}^n$ un punto factible para (P)
tal que existen n\'umeros reales $\lambda_1,\ldots,\lambda_m$
para los  que
\[
\nabla f(\overline{x})+\sum_{i=1}^m\lambda_i
\nabla g_i(\overline{x})=0.
\]
Supongamos que $f$ es convexa, $g_i$ es convexa si $\lambda_i>0$
y $g_i$ es c\'oncava si $\lambda_i<0$. Demuestra que
$\overline{x}$ es un m\'{\i}nimo global de (P).


\

\ex Considera el problema 
\[
\mbox{min}\ x_1+x_2-x_3 \ \mbox{s.a.}\ \ x_1^2+x_2^2+x_3^2\leq 27,\ \ x_1+x_2\leq 10.
\]
\begin{compactitem}
\item[(a)] Calcula todos los puntos que satisfacen las condiciones KKT correspondientes al problema anterior.
\item[(b)] ¿Podemos asegurar que los puntos calculados en el apartado anterior son mínimos globales del problema? ¿Existe algún mínimo global que no verifique las condiciones KKT?
\end{compactitem}



\ex Considera el problema 
\[
\mbox{max}\ x^2_1+2x^2_2-6x_1 \ \mbox{s.a.}\ \ x_1+x_2\geq 1.
\]
\begin{compactitem}
\item[(a)] Calcula todos los puntos que satisfacen las condiciones KKT correspondientes al problema anterior.
\item[(b)] ¿Podemos asegurar que los puntos calculados en el apartado anterior son máximos globales del problema? ¿Existe algún máximo global que no verifique las condiciones KKT?
\end{compactitem}


\


\ex Sea $f:\mathbb{R}^2\to\mathbb{R}$ convexa y diferenciable. Considera el problema de minimizar $f(x_1,x_2)$ sujeto a $x_1\geq x_2$. Supongamos que el problema tiene solución factible óptima.
\begin{compactitem}
\item[(a)] Escribe las condiciones de Karush-Kuhn-Tucker (KKT) correspondientes al problema anterior. ¿Qué condiciones debe cumplir $\nabla f(x_1,x_2)$ para que el punto $(x_1,x_2)$ verifique las condiciones KKT? Distingue los casos $x_1=x_2$ y $x_1\neq x_2$.
\item[(b)] Determina razonadamente si son verdaderas o falsas las  afirmaciones siguientes: 
	\begin{compactitem}
		\item[(1)] Si un punto verifica las condiciones KKT entonces es la solución factible óptima de este problema.
		\item[(2)] La solución factible óptima de este problema podría no verificar las condiciones KKT.
	\end{compactitem}
\item[(c)] Si $f(x_1,x_2)=x_1^2+x_2^2+\alpha x_1 + \beta x_2$, donde $\alpha,\beta\in\mathbb{R}$, determina todos los valores de $\alpha$ y $\beta$ para los que la solución del problema tiene sus dos coordenadas iguales. 
\end{compactitem}



\

\ex Considera el problema de minimizar $f(x)$ s.a. $f_i(x)\leq 0$, $i=1,\ldots,m$, donde todas las funciones son convexas y diferenciables. Sean $\bar{x}\in\mathbb{R}^n$ y $\bar{u}\in\mathbb{R}^m$ tales que satisfacen las condiciones KKT. Demuestra que $\nabla f(\bar{x})^\top (x-\bar{x})\geq 0$, para cualquier punto factible $x$ del problema. (Sabemos que esta condición implica que $\bar{x}$ es la solución óptima.)


\

\ex Sea $c\in\mathbb{R}^n$, $c\neq 0$. Considera el problema de maximizar $f(x)=c^\top x$ s.a. $x^\top x\leq 1$. Calcula los puntos que verifican las condiciones KKT del problema y determina su máximo global. 




\


\ex Demuestra la desigualdad de dualidad débil ($\bar{d}\leq \bar{p}$) en los casos $\bar{p}=-\infty$ (el problema primal es no acotado) y $\bar{d}=\infty$ (el problema dual es no acotado).

\


\ex Sean $a_i\in\mathbb{R}^n$, $a_i\in\mathbb{R}^n$, $i=1,\ldots,m$. Considera el problema de minimizar una función lineal a trozos:
\[
\mbox{Minimizar}\ \max_{i=1,\ldots,m} (a_i^\top x + b_i),\ \ x\in\mathbb{R}^n.
\]
Definiendo $y_i=a_i^\top x + b_i$, este problema se puede expresar de forma equivalente como
\begin{center}
\begin{tabular}{ll}
Minimizar & $\displaystyle{\max_{i=1,\ldots,m}} y_i$ \\
sujeto a & \\
& $a_i^\top x + b_i=y_i,\ \ i=1,\ldots,m$
\end{tabular}
\end{center}
Demuestra que el correspondiente problema dual es
\begin{center}
\begin{tabular}{ll}
Maximizar & $b^\top u$ \\
sujeto a & \\
& $A^\top u=0$ \\
& $\mathbf{1}^\top u = 1$\\
& $u\geq 0$,
 \end{tabular}
\end{center}
donde $A$ es una matriz cuyas filas son los vectores $a_1,\ldots,a_m$.
(Conclusión: la minimización de una función lineal a trozos se puede reducir a un problema de optimización lineal.)








\newpage

\ex  Considera el problema
\begin{center}
\begin{tabular}{ll}
Minimizar & $x_1 + x_2$ \\
sujeto a & \\
& $2x_1 + x_2 \geq 4$\\
& $x_1+7x_2 \geq 7$\\
&$x_i\geq 0,\ i=1,2$
\end{tabular}
\end{center}


\begin{compactitem}
\item[(a)] Plantea el método de las dos fases con variables artificiales
y lleva a cabo una iteración del mismo.
\item[(b)] Resuelve el problema utilizando el algoritmo simplex-dual. 
\end{compactitem}




\

\ex Se dispone de dos complejos vitam\'{\i}nicos (marcas 1 y 2)
cuyos costes por unidad de peso son 30 y 40 euros,
respectivamente. Se desea asegurar la ingesta de un m\'{\i}nimo de 36
unidades de vitamina A al d\'{\i}a, 28 unidades de vitamina C y 32 de
vitamina D. Supongamos que la marca 1 proporciona (por unidad de
peso) 2 unidades de vitamina A, 2 de vitamina C y 8 de vitamina D.
La marca 2 proporciona 3, 2 y 2 unidades respectivamente.

\begin{compactitem}
\item[(a)] Plantea el problema de optimizaci\'on para calcular la
combinaci\'on de coste m\'as bajo que garantice la ingesta m\'{\i}nima
diaria de las tres vitaminas.
\item[(b)] Lleva a cabo una iteraci\'on del algoritmo simplex-dual para resolver este
problema.
\end{compactitem}





\


\ex Considera el siguiente problema de optimización lineal:

\begin{center}
\begin{tabular}{ll}
Maximizar & $4x_1 + x_2 + 3x_3$ \\
sujeto a & \\
& $x_1 + 4x_2 \leq 1 $\\
& $3x_1 - x_2 + x_3 \leq 4$\\
&$x_i\geq 0,\ i=1,\ldots,3$
\end{tabular}
\end{center}

\begin{compactitem}
\item[(a)] Escribe el problema en forma estándar. Escribe la tabla inicial del algoritmo simplex y lleva a cabo una iteración del algoritmo. ¿Se ha alcanzado con esta iteración la solución factible óptima?
\item[(b)] Escribe el problema dual, resuélvelo gráficamente y utiliza la solución del dual para calcular la solución del primal. 
\item[(c)] Escribe la primera tabla del algoritmo simplex-dual y lleva a cabo una iteración del algoritmo. ¿Se ha alcanzado con esta iteración la solución factible óptima?
\end{compactitem}




\



\ex Considera el problema
\[
\mbox{Minimizar}\ x^2+1,\ \ \mbox{s.a.}\ (x-2)(x-4)\leq 0.
\]
\begin{compactitem}
\item[(a)] Determina el conjunto factible, el valor óptimo y la solución factible óptima del problema.
\item[(b)] Calcula la correspondiente función dual.
\item[(c)] Plantea y resuelve el problema dual. ¿Hay dualidad fuerte? ¿Se verifica la condición de Slater?
\end{compactitem}



%------------------------------------------------------------------------
\end{document}
%-------------------------------------------------------------------------




