
\documentclass[11pt,a4paper,twoside]{article}


%-------------------------------------------------------------

\usepackage[utf8]{inputenc}
\usepackage{graphicx,hyperref,amsmath,natbib,bm,url,microtype}
\usepackage{paralist}
\usepackage[spanish]{babel}
\usepackage[a4paper,text={16.5cm,25.2cm},centering]{geometry}
\usepackage[compact,small]{titlesec}
\setlength{\parskip}{1.2ex}
\setlength{\parindent}{0em}
\clubpenalty = 10000
\widowpenalty = 10000
\usepackage{kpfonts}
\usepackage[T1]{fontenc}
\pagestyle{empty}

\newcounter{problem} \setcounter{problem}{1}
\newcommand{\ex}{\noindent {\sf \bf \theproblem}\addtocounter{problem}{1}.\ }

%-------------------------------------------------------------------
\begin{document}
%-------------------------------------------------------------------
% Encabezamiento



%-----------------------------------
\hrule
\bigskip
\textbf{Investigación Operativa   \hfill Curso 2015/2016\\
Grado en Matemáticas}
\bigskip
\hrule
%--------------------------------

\


\begin{center}  {\bf \large
Relaci\'{o}n 4 de problemas}
\end{center}

\


\ex Sea  $D\subset\mathbb{R}^n$ un
conjunto convexo, y sea $f:D\to \mathbb{R}^n$ una funci{\'o}n
definida sobre {\'e}l. 
\begin{compactitem}
\item[(a)] Demuestra que una condici{\'o}n necesaria, pero no suficiente,
para que $f$ sea convexa es que, para cada n{\'u}mero real $\alpha$,
el conjunto $\{x\in D:\ f(x)\leq \alpha\}$ sea convexo.
\item[(b)] Demuestra que una condici{\'o}n necesaria y suficiente para que
$f$ sea convexa es que el epigrafo de $f$ sea un conjunto convexo.
\end{compactitem}


\

\ex Se dice que una función $f:\mathbb{R}^n\to\mathbb{R}$ es cuasiconvexa si los conjuntos $S_\alpha=\{x\in\mathbb{R}^n:\, f(x)\leq \alpha\}$ son convexos para todo $\alpha\in\mathbb{R}$. Demuestra que  $f$ es cuasiconvexa si y solo si para todo $x,y\in\mathbb{R}^n$ y $\lambda\in [0,1]$ se verifica $f((1-\lambda)x+\lambda y)\leq \max\{f(x),f(y)\}.$ 
% Ex. junio 2015.


\




\ex Comprueba si las siguientes funciones son convexas o cóncavas (o ni convexas ni cóncavas) en su dominio de definición:
\begin{compactitem}
\item[(a)] $f(x_1,x_2,x_3)=(x_1+2x_2-3x_3)^2$.
\item[(b)] $f(x_1,x_2)=8x_1-6x_2-2x_1^2-3x_2^2+4x_1x_2$.
\item[(c)] $f(x_1,x_2)=\min\{x_1,x_2\}$.
\item[(d)] $f(x_1,x_2)=(x_2-x_1^2)^2$.
\end{compactitem}


\

\ex Sea $I\subset \mathbb{R}$ un intervalo y $f:I\to\mathbb{R}$ una función convexa.
\begin{compactitem}
\item[(a)] (Lema de las tres cuerdas)  Sean $x,y,z\in I$ con $x<y<z$, demuestra
\[
\frac{f(y)-f(x)}{y-x}\leq \frac{f(z)-f(x)}{z-x}\leq \frac{f(z)-f(y)}{z-y}.
\]
Deduce que la función $g(x)=(f(x)-f(a))/(x-a)$, definida para $x\neq a$, es creciente.
\item[(b)] Sea $c$ un punto del interior de $I$. Demuestra que existen las derivadas por la derecha y por la izquierda de $f$ en $c$ y verifican $f'_-(c)\leq f'_+(c)$.
\item[(c)] Sean $a$ y $b$ puntos del interior de $I$ con $a<b$ entonces, por el apartado (b), existen las derivadas por la derecha y por la izquierda de $f$ en $a$ y $b$. Demuestra que se verifica:
\[
f'_-(a)\leq f'_+(a)\leq \frac{f(b)-f(a)}{b-a} \leq f'_-(b)\leq f'_+(b).
\]
\item[(d)] Si $f$ es derivable, demuestra que $f$ es convexa si y solo si $f'$ es creciente.
\item[(e)] Si $f$ es derivable dos veces, demuestra que $f$ es convexa si y solo si $f''(x)\geq 0$, para todo $x\in I$.
\end{compactitem}




\
 
\ex Utiliza la desigualdad de Jensen (aplicada a una función convexa adecuada) para demostrar que si $x_1,\ldots,x_m \in\mathbb{R}$ y $\lambda_1,\ldots,\lambda_m>0$ con $\lambda_1+\cdots + \lambda_m=1$, entonces
\[
\prod_{i=1}^m x_i^{\lambda_i} \leq \sum_{i=1}^m \lambda_i x_i.
\]


\

\ex Una función $f$ es log-convexa si es positiva y $\log f$ es convexa. Demuestra que toda función log-convexa es convexa. Demuestra que la función gamma, $\Gamma(x)=\int_0^\infty t^{x-1} e^{-t} dt$, $x>0$, es log-convexa (\textit{Indicación: usa la desigualdad de Hölder}).


\
 
\ex Sea $f:[0,\infty)\to\mathbb{R}$ convexa y derivable. Demuestra que la función que da los promedios de $f$ sobre los intervalos $[0,x]$, es decir $F(x)=(1/x)\int_0^x f(t)dt$, también es convexa.  (\textit{Indicación: calcula $F''(x)$ y exprésala en la forma $F''(x)=(2/x^3)\int_0^x g(x,t)dt$, para cierta función $g(x,t)$}).

\
 
\ex Considera el problema $\min f(x)$ s.a. $x\geq 0$, donde $f:\mathbb{R}^n\to \mathbb{R}$ es convexa y diferenciable. Demuestra que $\bar{x}$ es la solución factible óptima de este problema si y solo si  para todo $i=1\ldots,n$ se cumple $\bar{x}_i\geq 0$,  $f'_i(\bar{x})\geq 0$ y $\bar{x}_if'_i(\bar{x})=0$.

\

 \ex Sean $f_1,\ldots,f_m$ funciones convexas definidas sobre un dominio $D\subset\mathbb{R}^n$ convexo y supongamos que no existe $x\in D$ tal que $f_1(x)<0,\ldots,f_m(x)<0$.
\begin{compactitem}
\item[(a)] Demuestra que el conjunto 
\[
S=\{(y_1,\ldots,y_m)\in\mathbb{R}^m:\, \mbox{existe}\ x\in D\ \mbox{con}\ f_1(x)<y_1,\ldots,f_m(x)<y_m\}
\]
es convexo.
\item[(b)] Demuestra que existen escalares no negativos $\lambda_1,\ldots,\lambda_m$, no todos nulos, tales que $\lambda_1 f_1(x)+\cdots + \lambda_m f_m(x)\geq 0$ para todo $x\in D$.
\end{compactitem}


\

\ex Un subgradiente de una función $f:D\to\mathbb{R}$ en el punto $\bar{x}\in D\subset \mathbb{R}^n$ es un vector $u\in\mathbb{R}^n$ tal que $f(x)\geq f(\bar{x}) + u^\top (x-\bar{x})$, para todo $x\in D$. 
\begin{compactitem}
\item[(a)] Demuestra que el conjunto de subgradientes de una función en un punto (la subdiferencial de la función en ese punto) es un conjunto convexo. 
\item[(b)] Demuestra que si $f:D\to\mathbb{R}$ es convexa y diferenciable en un conjunto abierto y convexo $D$, entonces el único subgradiente de $f$ en cualquier $\bar{x}\in D$ es $\nabla f(\bar{x})$.
\item[(c)] Determina los subgradientes de la función $f(x)=\|x\|$, donde $\|\cdot\|$ es cualquier norma procedente de un producto escalar.
\item[(d)] Sea $f:S\to \mathbb{R}$, con $S\subset\mathbb{R}^n$ convexo. Considera el problema $\min f(x)$ s.a. $x\in S$. Demuestra que $\bar{x}\in S$ es la solución de este problema si $f$ tiene un subgradiente $u$ en $\bar{x}$ tal que $u^\top (x-\bar{x})\geq 0$, para todo $x\in S$. (Si $f$ es convexa el recíproco también es cierto pero la demostración, basada en los teoremas de separación de conjuntos convexos, es más complicada.)  
\end{compactitem}

%------------------------------------------------------------------------
\end{document}
%-------------------------------------------------------------------------

