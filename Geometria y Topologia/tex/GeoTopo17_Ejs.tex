\section{Hoja 1}


\begin{problem}[1]Responde brevemente a las siguientes preguntas:
	\ppart Si $T=T(\overline{x},\overline{y})$ y $S=S(\overline{x},\overline{y})$ son tensores, ¿lo es $T(\overline{x},\overline{y})\cdot S(\overline{x},\overline{y})$?¿y $T(\overline{x},\overline{y})+S(\overline{x},\overline{y})$?
	\ppart ¿Es $T(\overline{x},\overline{y})=\overline{x}+\overline{y}$ una aplicación bilineal?
	\ppart ¿Es un tensor la aplicación que dados dos vectores de $ℝ^{3}$ les asigna la primera coordenada de su producto vectorial?
	\ppart ¿Es un tensor la aplicación que a cada par de vectores de $ℝ^{2}$ con la base canónica les asigna el área del paralelogramo que determinan?
	
	\solution
	\doneby{Jose} \\
	\spart Apartado a
	
	\spart Apartado b
	
	\spart Apartado c
	
	\spart Apartado d
	

	
\end{problem}
\begin{problem}[2]
	Demuestra que, fijada una base, cualquier tensor dos veces covariante es de la forma $T(\overline{x},\overline{y})=\overline{x}^{T}\cdot A\cdot\overline{y}$, donde A es una matriz cuadrada.
\end{problem}[2]