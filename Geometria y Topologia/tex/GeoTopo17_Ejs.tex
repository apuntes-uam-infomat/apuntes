% -*- root: ../GeoTopo17.tex -*-
\section{Hoja 1}
\begin{problem}[1]Responde brevemente a las siguientes preguntas:
	\ppart Si $T=T(\overline{x},\overline{y})$ y $S=S(\overline{x},\overline{y})$ son tensores, ¿lo es $T(\overline{x},\overline{y})\cdot S(\overline{x},\overline{y})$?¿y $T(\overline{x},\overline{y})+S(\overline{x},\overline{y})$?
	\ppart ¿Es $T(\overline{x},\overline{y})=\overline{x}+\overline{y}$ una aplicación bilineal?
	\ppart ¿Cuántas componentes tiene un tensor (r,s) con $V=ℝ^{m}$?
	\ppart ¿Es un tensor la aplicación que dados dos vectores de $ℝ^{3}$ les asigna la primera coordenada de su producto vectorial?
	\ppart ¿Es un tensor la aplicación que a cada par de vectores de $ℝ^{2}$ con la base canónica les asigna el área del paralelogramo que determinan?
	
	\solution
	\doneby{Jose}\\
	\spart Tenemos $$\appl{T}{ℝ^{n}×ℝ^{n}}{ℝ};\tab\appl{S}{ℝ^{n}×ℝ^{n}}{ℝ};\tab\text{ambos multilineales}$$\indent Es fácil observar que $T\cdot S(\overline{x},\overline{y})=T(\overline{x},\overline{y})\cdot S(\overline{x},\overline{y})$ no es multilineal , ya que $$T\cdot S(\alpha\cdot\overline{x},\overline{y})=\alpha^2\cdot T\cdot S(\overline{x},\overline{y})$$ \indent luego no es tensor.\newline
	\indent Si ahora nos fijamos en $T+S(\overline{x},\overline{y})=T(\overline{x},\overline{y})+S(\overline{x},\overline{y})$ es inmediato comprobar que es \indent un tensor 2 veces covariante:
	$$T+S(\alpha\cdot\overline{x},\overline{y})=T(\alpha\overline{x},\overline{y})+S(\alpha\overline{x},\overline{y})=\alpha\cdot(T+S(\overline{x},\overline{y}))$$
	$$T+S(\overline{x}_1+\overline{x}_2,\overline{y})=T(\overline{x}_1,\overline{y})+S(\overline{x}_1,\overline{y})+T(\overline{x}_2,\overline{y})+S(\overline{x}_2,\overline{y})=(T+S(\overline{x}_1,\overline{y}))+(T+S(\overline{x}_2,\overline{y}))$$
	\spart \indent Inmediato comprobar que no es bilineal multiplicando una variable por un escalar: $$T(\alpha\overline{x},\overline{y})=\alpha\overline{x}+\overline{y}\neq\alpha(\overline{x}+\overline{y})$$
	\newpage
	\spart $$\appl{T}{\underbrace{(ℝ^{m})^{*}×\cdots×(ℝ^{m})^{*}}_{\text{r veces}}×\underbrace{ℝ^{m}×\cdots×ℝ^{m}}_{\text{s veces}}}{ℝ}$$ \indent luego habrá $m^{r+s}$ componentes.
	
	\spart  Hay dos formas, una es considerar:
	\begin{align*}
		\appl{T}{ℝ^{3}×ℝ^{3}&}{ℝ} \\
		T\left(\begin{pmatrix}x_1\\x_2\\x_3\end{pmatrix},\begin{pmatrix}y_1\\y_2\\y_3\end{pmatrix}\right) &\longmapsto{x_2\cdot y_3-x_3\cdot y_2}
	\end{align*}
 	y comprobar que efectivamente se cumplen las condiciones de multilinealidad. \\
 	La segunda es considerar:
 	\begin{align*}
 		\appl{T}{ℝ^{3}×ℝ^{3}&}{ℝ} \\
 		T\left(\begin{pmatrix}x_1\\x_2\\x_3\end{pmatrix},\begin{pmatrix}y_1\\y_2\\y_3\end{pmatrix}\right) &\longmapsto{(\overline{x}×\overline{y})\cdot\overline{e}_1=\begin{vmatrix}
 				1 & 0 &  0 \\ 
 				x_1 & x_2 & x_3 \\ 
 				y_1 & y_2 & y_3 \\ 
 		\end{vmatrix}}
 	\end{align*}
	y como vimos que el determinante es multilineal, pues ya está demostrado porque es un determinante.
	
	\spart 
		\begin{align*}
		\appl{T}{ℝ^{2}×ℝ^{2}&}{ℝ} \\
		T(\overline{x},\overline{y}) &\longmapsto{A=\text{área}}
	\end{align*}
\indent El área siempre es $\geq 0$, luego si multiplico por $\lambda=-1$ tenemos $T(\lambda\overline{x},\overline{y})\neq\lambda T(\overline{x},\overline{y})$
	
\end{problem}

\begin{problem}[3] Halla cuántas componentes nulas y cuántas componentes no nulas tiene el tensor determinante en $ℝ^{n}$. Estudia cuántas son positivas.
	
	\solution Sea el tensor n veces covariante en $ℝ^{n}$ y la base canónica $\base = \{ \overline{e}_1,...,\overline{e}_n \}$ de $ℝ^{n}$:
	
\begin{align*}
	\appl{D}{\underbrace{(ℝ^{n})×\cdots×(ℝ^{n})}_{\text{n veces}}&}{ℝ} \\
	D(\overline{x}_1,\cdots,\overline{x}_n) &\longmapsto{\begin{vmatrix}
							\overline{x}_1 & \cdots &  \overline{x}_n \\ 
						\end{vmatrix}}
\end{align*}
		El tensor tienen $n^n$ componentes:
		$$D_{1\space1\cdots1}=\begin{vmatrix}
		1 & 1 &\cdots & 1 \\ 
		0& 0 &\cdots & 0 \\ 
		\vdots & \vdots &\ddots & \vdots \\ 
		0& 0 &\cdots & 0 \\ 
		\end{vmatrix}$$
	En cuanto el determinante contenga a dos $\overline{e}_i$ que tengan la misma $i$ ya da 0 (por ser determinante de elementos linealmente dependientes). Luego para que no sea nulo ha de tener todas los índices distintos. Esto nos dice que el número de componentes no nulas son las permutaciones de $n$ elementos ($n!$), y las nulas serían entonces $n^n - n!$.\\
	Ahora, de las que no son nulas vamos a ver cuales son positivas. Probando descubrimos que con esta base hay dos valores posibles del determinante que son $1$ y $-1$. Las componentes positivas son aquellas que valen $1$ y esto ocurre cuando hay un número par de $\overline{e}_i$ cambiadas de posición (recordemos que todas las $i$ son ahora distintas porque estamos en el caso no nulo). \\
	La conclusión es que la cantidad de las que valen $1$ es la cantidad de permutaciones pares de n elementos, que es $\frac{n!}{2}$.
\end{problem}
\begin{problem}[9] Enunciado
	
	\solution El espín es un vector $\overline{v}=a_1\overline{e}_1+a_2\overline{e}_2$ para $a_1,a_2\in\mathbb{C}$...
\end{problem}

