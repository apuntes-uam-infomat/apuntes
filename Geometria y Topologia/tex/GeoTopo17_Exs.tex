% -*- root: ../GeoTopo17.tex -*-
\section{Parcial 1 - 28/02/2017}
\begin{problem}[1][3 puntos]Decide si son verdaderas o falsas las siguientes afirmaciones, dando una pequeña explicación:
	\ppart Las ecuaciones de Euler-Lagrange para el Lagrangiano $L=L(\dot{q},q,t)=\frac 12 m\dot{q}^2-\frac 12 Kq^2$ con $m,K\in\mathbb{R}^+$, equivalen a $\ddot{q}+\frac{K}{m}q=0$.
	\ppart En $ℝ^2$ con la carta trivial (coordenadas cartesianas), se cumple $dx\otimes dy=dy\otimes dx$.
	\ppart Si $f:M\longrightarrow N$ es una función $C^\infty$ entre variedades y $f$ es inyectiva, la aplicación tangente $T_p(M)\longrightarrow T_{f(p)}(N)$ es inyectiva para cada  $p\in M$. 
	\solution 
	\spart Verdadero $$\frac{\partial L}{\partial \dot{q}}
	=m\dot{q}\ \Rightarrow\ \frac{d}{dt}\Big(\frac{\partial L}{\partial \dot{q}}\Big)=m\ddot{q},\qquad \frac{d}{dt}\Big(\frac{\partial L}{\partial \dot{q}}\Big)=\frac{\partial L}{\partial {q}}\ \Leftrightarrow\ m\ddot{q}= -K q\ \Leftrightarrow\ \ddot{q}+\frac{K}{m}q=0.$$
	\spart Falso $$dx\otimes dy
	\Big(\frac{\partial}{\partial x},\frac{\partial}{\partial y}\Big)
	=
	dx
	\Big(\frac{\partial}{\partial x}\Big)
	dy
	\Big(\frac{\partial}{\partial y}\Big)
	=1\cdot 1,
	$$$$
	dy\otimes dx
	\Big(\frac{\partial}{\partial x},\frac{\partial}{\partial y}\Big)
	=
	dy
	\Big(\frac{\partial}{\partial x}\Big)
	dx
	\Big(\frac{\partial}{\partial y}\Big)
	=0\cdot 0.$$
	\spart Falso. Uno de los contraejemplos más sencillos es $M=N=ℝ$ con $f(x)=x^3$. En el origen la derivada, y por tanto la aplicación tangente, es nula. 
\end{problem}
\newpage
\begin{problem}[2][3 puntos] Expresa el campo de vectores tangentes $(x^2+y^2)\dfrac{\partial}{\partial x}$ de $ℝ^2$ (con las coordenadas cartesianas) en la carta en coordenadas polares. 
	
	\solution Usando las fórmulas $r=\sqrt{x^2+y^2}$, $\theta=\arctan(y/x)$ y las reglas de transformación de los vectores, que son tensores de tipo $(1,0)$, se tiene: $$(x^2+y^2)\frac{\partial}{\partial x}
	=
	r^2\Big(\frac{\partial r}{\partial x}
	\frac{\partial}{\partial r}
	+\frac{\partial \theta}{\partial x}
	\frac{\partial}{\partial \theta}\Big)
	=
	r^2\frac{x}{\sqrt{x^2+y^2}}
	\frac{\partial}{\partial r}
	-r^2\frac{y}{{x^2+y^2}}
	\frac{\partial}{\partial \theta}
	=
	r^2\cos\theta 
	\frac{\partial}{\partial r}
	-r\sen\theta
	\frac{\partial}{\partial \theta}.$$
	
\end{problem}
\begin{problem}[3][3 puntos] Si $T$ es un tensor de tipo $(1,1)$ en una variedad, se define su traza en cada punto como~$T_i^i$. Demuestra que no depende de la carta elegida.
	
	\solution Por las reglas de transformación de los tensores, sabemos que que si $T^i_j$ y $\widetilde{T}^i_j$ son las componentes en dos cartas con funciones coordenadas $(x^1,\dots, x^n)$ y $(y^1,\dots, y^n)$, respectivamente, $${T}^i_j
	=
	\frac{\partial x^i}{\partial y^k}
	\frac{\partial y^l}{\partial x^j}
	\widetilde{T}^k_l
	\ \Rightarrow\ 
	{T}^i_i
	=
	\frac{\partial x^i}{\partial y^k}
	\frac{\partial y^l}{\partial x^i}
	\widetilde{T}^k_l
	\ \Rightarrow\ 
	{T}^i_i
	=
	\frac{\partial y^l}{\partial y^k}
	\widetilde{T}^k_l
	=
	\delta^l_k
	\widetilde{T}^k_l
	=
	\widetilde{T}^k_k.$$
	Para la primera implicación se toma $i=j$ (se suma en $i=j$). La segunda es la regla de la cadena. 
\end{problem}
\begin{problem}[4][1 punto] ¿Qué dimensión tiene la variedad (subvariedad de $ℝ^9$) formada por las matrices simétricas reales~$3\times 3$ con polinomio característico $\lambda^2(\lambda-1)$? Puedes razonar intuitivamente. 
	
	\solution Cada matriz $A$ de la variedad diagonaliza en base ortonormal $\{\vec{u},\vec{v}_1,\vec{v}_2\}$ con $A\vec{u}=\vec{u}$ y $A\vec{v}_j=\vec{0}$ para $j=1,2$. Por tanto, la aplicación lineal $L$ que corresponde a $A$ es $L(\vec{x})=\text{proy}_{\vec{u}}(\vec{x})=(\vec{u}\cdot \vec{x})\vec{u}$ que está biunívocamente determinada por la dirección de  $\vec{u}$, lo que da dimensión dos (un punto en $S^2$ identificando opuestos).   
\end{problem}
\subsection{Criterios y errores frecuentes}
\begin{itemize}
	\item Decir verdadero o falso no cuenta nada si no va acompañado de una explicación. 
	
	
	b) Algunos confunden $dx\otimes dy$ con la forma diferencial $dx\wedge dy$ que todavía no ha aparecido en el curso. 
	
	c) Algunos creen que $g:ℝ^n\longrightarrow ℝ^n$ inyectiva implica $Dg:ℝ^n\longrightarrow ℝ^n$ inyectiva y les parece que se deduce del teorema del rango o algo similar pero eso no es cierto.
	\item Cosas como $\frac{\partial r}{\partial x}=\big(\frac{\partial x}{\partial r}\big)^{-1}$ además de falsas son muy, muy feas.  El teorema de la función inversa dice que la matriz jacobiana de la inversa es la inversa de la jacobiana pero una matriz no se invierte tomando el inverso de cada elemento. 
	\item Aquí hay un error que después de recapacitar no he penalizado pero que es importante tener claro. Lo ilustro primero con un ejemplo. Si multiplicamos  $\sum_i a_i x^i$ por $\sum_i b_i y^i$ el resultado es  $\sum_{i,j} a_ib_j x^iy^j$ que es bien distinto de $\sum_{i} a_ib_i x^iy^i$. El convenio de sumación nos debería proteger de este error porque $a_ib_i x^iy^i$ no tiene un sentido definido, debo sumar en subíndices y superíndices iguales ¿pero cómo los agrupo? Si uno lo ``define'' como sumar en todos al tiempo, al multiplicar $a_ix^i$ por $b_iy^i$ no se obtiene $a_ib_i x^iy^i$. Volviendo al problema, varios escriben 
	$\frac{\partial x^i}{\partial y^k}
	\frac{\partial y^k}{\partial x^i}
	\widetilde{T}^k_k$
	en lugar de 
	$\frac{\partial x^i}{\partial y^k}
	\frac{\partial y^l}{\partial x^i}
	\widetilde{T}^k_l$. 
	\item Más o menos ha habido dos enfoques del problema, uno con proyecciones y autovectores (como en la solución anterior) y otro con las ecuaciones correspondientes a los coeficientes del polinomio característico. Todos menos una persona se han equivocado pero como considero que los del primer enfoque se quedan cerca de la solución  tienen un 0.75. Con el segundo era bastante más difícil obtener el resultado. Algunos llegan a las ecuaciones correctas pero no se dan cuenta de que restringen más de lo que parece, tienen 0.5. Típicamente una ecuación quita un grado de libertad pero cosas como $x^2+y^2+z^2=0$ pasan de tres variables a ninguna. 
\end{itemize}
\section{Parcial 2 - 4/04/2017}
\begin{problem}[1]Decide si son verdaderas o falsas las siguientes afirmaciones, dando una breve explicación.
	\ppart (1 punto) Si todos los símbolos de Christoffel se anulan, entonces $\nabla_X Y= \big(X^i\partial_i Y^j\big) \partial_j$. 
	\ppart (1 punto) Si $V(t)\ne 0$ es un transporte paralelo a lo largo de una curva, entonces $e^tV(t)$ no es trasporte paralelo.
	
	\solution 
	\spart Verdadero. Es consecuencia de la igualdad general
	\[
	\nabla_XY
	=
	\nabla_{X^i\partial_i}Y
	=
	X^i\nabla_{i}Y
	=
	X^i\big(\partial_iY^k+\Gamma_{il}^kY^l\big)\partial_k.
	\]
	\spart Verdadero. Se tiene
	\[
	\frac{DV}{dt}
	=
	\frac{D\big(e^tV(t)\big)}{dt}
	\overset{(*)}{=}
	\frac{de^t}{dt}V(t)+e^t\frac{DV}{dt}
	=
	e^tV(t)\ne 0.
	\]
	La justificación detallada de (*) es 
	$\frac{d(e^tV)}{dt}+\Gamma_{ij}^ke^tV^i\frac{d x^j}{dt}
	= \frac{de^t}{dt}V(t)+e^t
	\Big(
	\frac{dV}{dt}+\Gamma_{ij}^kV^i\frac{d x^j}{dt}\big)$.
\end{problem}
\newpage
\begin{problem}[2]En $\mathbb{R}^+\times \mathbb{R}^+$ se considera la métrica $x^{-2}dx^2+x^{-1}dy^2$. 
	\ppart (3 puntos) Calcula el símbolo de Christoffel $\Gamma_{11}^1$.
	\ppart (2 puntos) Sabiendo que hay una geodésica horizontal con punto inicial $p=(1,1)$ y vector inicial $2\partial_1$, halla su ecuación explícita.  {\sf Nota:} No se pide demostrar la existencia de tal geodésica, se puede dar por supuesta. 
	
	\solution 
	\spart El lagrangiano es $L=x^{-2}\dot{x}^2+x^{-1}\dot{y}^2$. Entonces
	\[
	\frac{\partial L}{\partial x}
	=
	-2x^{-3}\dot{x}^2-x^{-2}\dot{y}^2
	\qquad\text{y}\qquad
	\frac{d}{dt}
	\Big(
	\frac{\partial L}{\partial \dot{x}}
	\Big)
	=
	\frac{d}{dt}
	\big(
	2x^{-2}\dot{x}
	\big)
	=
	-4x^{-3}\dot{x}^2+2x^{-2}\ddot{x}.
	\]
	Igualando, se sigue $\ddot{x}-x^{-1}\dot{x}^2+\frac 12 \dot{y}^2=0$. El símbolo $\Gamma_{11}^1$ es el coeficiente de $\dot{x}^2$, esto es, $-x^{-1}$.\\
	\spart Por la conservación de la energía el lagrangiano es constante a lo largo de las geodésicas: $x^{-2}\dot{x}^2=\textsf{cte}$, de aquí $x^{-1}\dot{x}=\textsf{cte}$ que implica $x(t)=Ae^{Bt}$. De las condiciones iniciales $x(0)=1$ y $\dot{x}(0)=2$ se sigue $A=1$ y $B=2$. En definitiva $\big(x(t),y(t)\big)=(e^{2t},1)$.  
\end{problem}
\begin{problem}[3]Sea $F:M\longrightarrow N$ con $dF|_p:T_p(M)\longrightarrow T_{F(p)}(N)$ inyectiva para cada $p\in M$ y $G$ una métrica definida positiva en~$N$. 
	\ppart (2 puntos) Prueba que el \emph{pullback} $F^*G$ es una métrica en~$M$.
	{\sf Nota:} Recuerda la definición de \emph{pullback}  $F^*G(V,W)=G\big(dF(V), dF(W)\big)$.
	\ppart (1 punto) ¿Es cierto en general que $F$ transforma geodésicas de $M$ en geodésicas de $N$? 
	
	\solution 
	\spart Como $G(X,Y)=G(Y,X)$, la simetría de $F^*G$ está asegurada. Si la matriz de componentes de $F^*G$ fuera singular en $p$, existiría $V\in T_p(M)$ tal que $F^*G(V,V)=0$ (escogiéndolo del núcleo) y entonces $X=dF|_p(V)\in T_{F(p)}(N)$ sería un vector no nulo (por la inyectividad) tal que $G(X,X)=G\big(dF|_p(V), dF|_p(V)\big)=0$, lo que contradice que la métrica sea definida positiva.\\
	\spart No, por ejemplo la inclusión $i:\mathbb{S}^1\hookrightarrow\mathbb{R}^2$ tiene como imagen una circunferencia y los arcos de circunferencia no son geodésicas de $\mathbb{R}^2$ (con ninguna parametrización) pero sí lo son en $\mathbb{S}^1$ con la métrica usual. 
	
\end{problem}
\section{Parcial 3 - 11/05/2017}