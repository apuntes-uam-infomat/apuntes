\documentclass[palatino, bibnumbers]{apuntes}

\title{Geometría y Topología}
\author{Jose Antonio García del Saz}
\date{16/17 C2}

% Paquetes adicionales
\usepackage{enumitem}
\usepackage{kpfonts}
\usepackage{tikztools}
\usepackage{fancysprefs}
\usepackage{tikz-3dplot}
\usepackage{xfrac}
\usepackage{wrapfig}
\usepackage{fastbuild}
\usepackage{tikz-cd}

\usetikzlibrary{arrows}
\usetikzlibrary{patterns}
\usetikzlibrary{intersections}
\usetikzlibrary{calc}
\usetikzlibrary{fadings}

\tikzset{
	snake/.style={
		rounded corners,
		to path={
			-- ([xshift=1em]\tikztostart.east)
			-- ([xshift=1em]\tikztostart.south east)
			-- ([xshift=-1em]\tikztotarget.north west)
			-- ([xshift=-1em]\tikztotarget.west)
			-- (\tikztotarget)
		}
	},
	snake up/.style={
		rounded corners,
		to path={
			-- ([xshift=-1em]\tikztostart.west)
			-- ([xshift=-1em]\tikztostart.north west)
			-- ([xshift=1em]\tikztotarget.south east)
			-- ([xshift=1em]\tikztotarget.east)
			-- (\tikztotarget)
		}
	}
}

\setlist{itemsep=1pt, topsep=5pt}
\bibliographystyle{alpha}
% --------------------

%\precompileTikz

\newcommand{\Id}{\mop{Id}}
\newcommand{\cln}{\colon\!}

\setcounter{tocdepth}{3}

\begin{document}
\pagestyle{plain}

% http://tex.stackexchange.com/a/14243
\relpenalty=9999
\binoppenalty=9999

\begin{abstract}
Estos son los apuntes del curso de Geometría y Topología, del profesor Fernando Chamizo.
\end{abstract}

\maketitle

\tableofcontents
\newpage
% Contenido.

\chapter{Álgebra Tensorial}

\section{Tensores en  $ℝ^{n}$}

Estudiar los tensores en $ℝ^n$ es en realidad estudiar el Álgebra Lineal pero en varias variables. En primer curso (Álgebra I) estudiamos las aplicaciones lineales, las cuales eran de la forma:
\begin{align*}
	\appl{π}{ℝ^{n}&}{ℝ^{m}} \\
	\overline{x} &\longmapsto[\overline{y}=A\cdot \overline{x}]
\end{align*}

\begin{defn}[Aplicación\IS Lineal] Sea $f$ una aplicación entre dos espacios vectoriales $V$, $W$ sobre el mismo cuerpo $K$. Decimos que $f$ es una \textbf{aplicación lineal} si se cumplen las siguientes propiedades:
	\begin{enumerate}
		\item $f(λ\overline{x})=λ*f(\overline{x})$ .
		\item $f(\overline{x_1}+\overline{x_2})=f(\overline{x_1})+f(\overline{x_2})$
	\end{enumerate}
\end{defn}

\begin{defn}[Aplicación\IS Bilineal] Sea 
	\begin{align*}
	\appl{f}{ℝ^{n}×ℝ^{n}&}{ℝ} \\
	\overline{x},\overline{y} &\longmapsto{f(\overline{x},\overline{y})}
	\end{align*}
una aplicación, decimos que es \textbf{bilineal} si es una aplicación lineal en cada una de las dos variables, es decir:
\begin{enumerate}
	\item $f(λ\overline{x},\overline{y})=λ\cdot f(\overline{x},\overline{y})$; $f(\overline{x_1}+\overline{x_2},\overline{y})=f(\overline{x_1},\overline{y})+f(\overline{x_2},\overline{y})$
	\item $f(\overline{x},λ\overline{y})=λ\cdot f(\overline{x},\overline{y})$; $f(\overline{x},\overline{y_1}+\overline{y_2})=f(\overline{x},\overline{y_1})+f(\overline{x},\overline{y_2})$
\end{enumerate}
\end{defn}
\textbf{Observación:} todas las aplicaciones bilineales entre dos espacios se pueden escribir de la siguiente manera:
$$f(\overline{x},\overline{y})=\overline{x}^{T}A\overline{y}$$ con A una matriz n×n.
\newpage
\begin{defn}[Aplicación\IS Multilineal] Decimos que una aplicación es \textbf{multilineal} si es lineal en cada una de sus variables. 
\end{defn}

\begin{defn}[Tensor\IS n veces covariante] Es cualquier aplicación multilineal
	$\appl{T}{\varprod_{i=1}^n V}{ℝ}$, siendo V un espacio vectorial de dimensión finita sobre $ℝ$ (que como sabemos de otros cursos son isomorfos a $ℝ^{n}$).
\end{defn}

\begin{example} Sea
	\begin{align*}
		\appl{T}{ℝ^{3}×ℝ^{3}&}{ℝ} \\
		T\left(\begin{pmatrix}x_1\\x_2\\x_3\end{pmatrix},\begin{pmatrix}y_1\\y_2\\y_3\end{pmatrix}\right) &\longmapsto{x_1\cdot y_3}
	\end{align*}
es obvio que T es multilineal, luego T es un tensor 2 veces covariante en $ℝ^{3}$.
\end{example}
\begin{example} Sea
	\begin{align*}
		\appl{T}{ℝ^{3}×ℝ^{3}&}{ℝ} \\
		T\left(\begin{pmatrix}x_1\\x_2\\x_3\end{pmatrix},\begin{pmatrix}y_1\\y_2\\y_3\end{pmatrix}\right) &\longmapsto{x_1\cdot x_3}
	\end{align*}
	se ve rápidamente que \underline{no} es una aplicación lineal respecto de la variable $\overline{x}$.
\end{example}
\begin{example} Sea
	\begin{align*}
		\appl{T}{ℝ^{3}×ℝ^{3}×ℝ^{3}&}{ℝ} \\
		T\left(\begin{pmatrix}x_1\\x_2\\x_3\end{pmatrix},\begin{pmatrix}y_1\\y_2\\y_3\end{pmatrix},\begin{pmatrix}z_1\\z_2\\z_3\end{pmatrix}\right) &\longmapsto{\begin{vmatrix}
			x_1 & y_1 &  z_1 \\ 
			x_2 & y_2 & z_2 \\ 
			x_3 & y_3 & z_3 \\ 
		\end{vmatrix}}
	\end{align*}
	la propiedad de linealidad del producto por un escalar es obvia por las propiedades de los determinantes. La propiedad de linealidad que conserva la adición se demuestra fácilmante desarrollando el determinante por adjuntos en la primera columna.
\end{example}

%% Apéndices (ejercicios, exámenes)
\appendix


\chapter{Ejercicios}
% -*- root: ../GeoTopo17.tex -*-
\section{Hoja 1}
\begin{problem}[1]Responde brevemente a las siguientes preguntas:
	\ppart Si $T=T(\overline{x},\overline{y})$ y $S=S(\overline{x},\overline{y})$ son tensores, ¿lo es $T(\overline{x},\overline{y})\cdot S(\overline{x},\overline{y})$?¿y $T(\overline{x},\overline{y})+S(\overline{x},\overline{y})$?
	\ppart ¿Es $T(\overline{x},\overline{y})=\overline{x}+\overline{y}$ una aplicación bilineal?
	\ppart ¿Cuántas componentes tiene un tensor (r,s) con $V=ℝ^{m}$?
	\ppart ¿Es un tensor la aplicación que dados dos vectores de $ℝ^{3}$ les asigna la primera coordenada de su producto vectorial?
	\ppart ¿Es un tensor la aplicación que a cada par de vectores de $ℝ^{2}$ con la base canónica les asigna el área del paralelogramo que determinan?
	
	\solution
	\doneby{Jose}\\
	\spart Tenemos $$\appl{T}{ℝ^{n}×ℝ^{n}}{ℝ};\tab\appl{S}{ℝ^{n}×ℝ^{n}}{ℝ};\tab\text{ambos multilineales}$$\indent Es fácil observar que $T\cdot S(\overline{x},\overline{y})=T(\overline{x},\overline{y})\cdot S(\overline{x},\overline{y})$ no es multilineal , ya que $$T\cdot S(\alpha\cdot\overline{x},\overline{y})=\alpha^2\cdot T\cdot S(\overline{x},\overline{y})$$ \indent luego no es tensor.\newline
	\indent Si ahora nos fijamos en $T+S(\overline{x},\overline{y})=T(\overline{x},\overline{y})+S(\overline{x},\overline{y})$ es inmediato comprobar que es \indent un tensor 2 veces covariante:
	$$T+S(\alpha\cdot\overline{x},\overline{y})=T(\alpha\overline{x},\overline{y})+S(\alpha\overline{x},\overline{y})=\alpha\cdot(T+S(\overline{x},\overline{y}))$$
	$$T+S(\overline{x}_1+\overline{x}_2,\overline{y})=T(\overline{x}_1,\overline{y})+S(\overline{x}_1,\overline{y})+T(\overline{x}_2,\overline{y})+S(\overline{x}_2,\overline{y})=(T+S(\overline{x}_1,\overline{y}))+(T+S(\overline{x}_2,\overline{y}))$$
	\spart \indent Inmediato comprobar que no es bilineal multiplicando una variable por un escalar: $$T(\alpha\overline{x},\overline{y})=\alpha\overline{x}+\overline{y}\neq\alpha(\overline{x}+\overline{y})$$
	\newpage
	\spart $$\appl{T}{\underbrace{(ℝ^{m})^{*}×\cdots×(ℝ^{m})^{*}}_{\text{r veces}}×\underbrace{ℝ^{m}×\cdots×ℝ^{m}}_{\text{s veces}}}{ℝ}$$ \indent luego habrá $m^{r+s}$ componentes.
	
	\spart  Hay dos formas, una es considerar:
	\begin{align*}
		\appl{T}{ℝ^{3}×ℝ^{3}&}{ℝ} \\
		T\left(\begin{pmatrix}x_1\\x_2\\x_3\end{pmatrix},\begin{pmatrix}y_1\\y_2\\y_3\end{pmatrix}\right) &\longmapsto{x_2\cdot y_3-x_3\cdot y_2}
	\end{align*}
 	y comprobar que efectivamente se cumplen las condiciones de multilinealidad. \\
 	La segunda es considerar:
 	\begin{align*}
 		\appl{T}{ℝ^{3}×ℝ^{3}&}{ℝ} \\
 		T\left(\begin{pmatrix}x_1\\x_2\\x_3\end{pmatrix},\begin{pmatrix}y_1\\y_2\\y_3\end{pmatrix}\right) &\longmapsto{(\overline{x}×\overline{y})\cdot\overline{e}_1=\begin{vmatrix}
 				1 & 0 &  0 \\ 
 				x_1 & x_2 & x_3 \\ 
 				y_1 & y_2 & y_3 \\ 
 		\end{vmatrix}}
 	\end{align*}
	y como vimos que el determinante es multilineal, pues ya está demostrado porque es un determinante.
	
	\spart 
		\begin{align*}
		\appl{T}{ℝ^{2}×ℝ^{2}&}{ℝ} \\
		T(\overline{x},\overline{y}) &\longmapsto{A=\text{área}}
	\end{align*}
\indent El área siempre es $\geq 0$, luego si multiplico por $\lambda=-1$ tenemos $T(\lambda\overline{x},\overline{y})\neq\lambda T(\overline{x},\overline{y})$
	
\end{problem}

\begin{problem}[3] Halla cuántas componentes nulas y cuántas componentes no nulas tiene el tensor determinante en $ℝ^{n}$. Estudia cuántas son positivas.
	
	\solution
	
\end{problem}	


\bibliography{../Apuntes}{}
\printindex
\end{document}
