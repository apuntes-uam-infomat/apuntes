\documentclass[palatino]{apuntes}

\title{Geometría y Topología}
\author{}
\date{15/16 C2}

% Paquetes adicionales
\usepackage{enumitem}
\usepackage{tikztools}

\setlist{itemsep=1pt, topsep=5pt}
\bibliographystyle{plainnat}
% --------------------

\begin{document}
\pagestyle{plain}

\begin{abstract}
Estos son los apuntes del curso de Geometría y Topología, del profesor Gabino González.
\end{abstract}

\maketitle


\tableofcontents
\newpage
% Contenido.

\chapter{Conceptos básicos}

En Geometría, los objetos que estudiamos se llaman ``variedades''. Veremos de distintos tipos (por ejemplo, en Geometría Diferencial (\cite{ApuntesGeoDif}) veíamos variedades diferenciables), aunque nosotros empezaremos con las topológicas.

\begin{defn}[Variedad\IS topológica] Una variedad topológica es un espacio topológico $M$ con las siguientes propiedades:
\begin{enumerate}
\item $M$ es $T_2$ (esto es, \concept{Hausdorff}: dos puntos distintos tienen entornos disjuntos).
\item $∀p ∈ M$ admite un entorno $U$ y un homeomorfismo $\appl{φ_u}{U}{ℝ^N}$ (o $\bola^N$).

Al par $(U,φ_u)$ se le llama \concept{Carta} para $p$. Si $φ_u(p) = 0$, se dice que la carta está centrada en $p$. A la colección de cartas se le llamará \concept{Atlas}.
\item Si para cualquier par de cartas $(U,φ_u)$ y $(V,φ_v)$ la aplicación $$\appl{φ_v ○ \inv{φ_u}}{φ_u(U∩V)}{φ_v(U∩V)}$$ es difeomorfismo, estamos entonces ante una \concept{Variedad\IS diferenciable}.
\end{enumerate}
\end{defn}

\begin{figure}[thbp]
\centering
\inputtikz{Cartas}
\caption{Un esquema de las cartas de una variedad $M$ y cómo se comportan en la intersección.}
\label{fig:Cartas}
\end{figure}

La dimensión de la variedad está dada por la dimensión de $ℝ$ a la que son homeomorfas las cartas. La cuestión es que no tenemos claro si eso está bien definido. En el caso diferenciable, la condición de difeomorfismo para la intersección de cartas implica que la dimensión de ambas cartas ha de ser la misma. En el caso topológico también está bien definido, aunque es más difícil de demostrar ya que dependemos de que no exista un homeomorfismo entre $ℝ^n$ y $ℝ^m$ con $n ≠ m$, que no es trivial.

Un ejemplo sencillo de variedad es $M = \bola^N$, con $φ$ la identidad. Otro ejemplo es $\crc$ (la circunferencia), que es una variedad de dimensión 1, que no se puede dar con sólo una carta (la circunferencia no es homeomorfa a $ℝ$)\footnote{$\crc$ es compacta y $\real$ no, y como compacidad es propiedad topológica y los homeomorfismos las preservan, no pueden ser homeomorfas.}. Podríamos darla tomando las dos mitades superior e inferior usando senos y cosenos, y también podríamos hacer la proyección estereográfica (\ref{fig:ProyEstereo}) desde los polos norte y sur $(0,1), (0,-1)$ respectivamente sobre la recta real. En este caso, tendríamos las siguientes cartas: \[
\begin{matrix}
	\appl{α_1}{V_1 = \crc\setminus\set{(0,1)}&}{&ℝ} \\
	p=(s,t) &\longmapsto& \frac{s}{1-t}
\end{matrix}
\qquad
\begin{matrix}
	\appl{α_1}{V_2 = \crc\setminus\set{(0,-1)}&}{&ℝ} \\
	p=(s,t) &\longmapsto& \frac{s}{1+t}
\end{matrix}\]

\begin{figure}[hbtp]
\inputtikz{ProyeccionCirc}
\caption{Proyección estereográfica de la circunferencia.}
\label{fig:ProyEstereo}
\end{figure}

Para comprobar si este atlas es diferenciable, tendríamos que mirar qué ocurre con $α_2 ○ \inv{α_1}$. Después de un montón de cuentas, nos sale que efectivamente lo es ($α_2 ○ \inv{α_1} = \frac{1}{x}$) en el dominio en el que está definido (el cero no es un problema porque no está dentro del dominio).


\chapter{Espacios tangente y cotangente}

\chapter{Cohomología de De Rham}

%% Apéndices (ejercicios, exámenes)
\appendix

\chapter{Ejercicios}
% -*- root: ../GeometriaTopologia.tex.tex -*-
\section{Hoja 1}

\begin{problem}[1] Considérese en $S^1$ las 2 cartas siguientes:

\[
U_1 = S^1 \setminus {(1,0)} \; \appl{φ_1^{-1}}{U_1}{(0,2π)} \text{ con } φ_1^{-1}(θ) = (\cos θ,\sin θ)
\]
\[
U_2 = S^1 \setminus {(-1,0)} \; \appl{φ_2^{-1}}{U_2}{(-π,π)} \text{ con } φ_2^{-1}(θ) = (\cos θ,\sin θ)
\]

\ppart Comprobar que estas 2 cartas definen un atlas en $S^1$.

\ppart Comprobar que el atlas anteriror es equivalente a los atlas definidos mediante la proyección estereográfica y la proyección a los ejes respectivamente.

\solution

\doneby{Dejuan}

\spart ¿$\{φ_1,φ_2\}$ es un atlas de $S^1$ si $\img(φ_1) \cup \img(φ_2) = S^1$? En caso de ser así, sólo hay que escribir las cosas con cuidado.

\spart 2 atlas son equivalentes si sus cartas son compatibles. Vamos a tomar $\{\Psi_i\}$ las cartas de las proyección estereográfica. La condición de compatibilidad es si $φ_1 \circ \Psi_1^{-1}$ es diferenciable.

Para más información, este ejercicio está esbozado ejemplificando la \ref{def::atlas_compatibles}

\end{problem}



\begin{problem}[5] Demostrar que las variedades $\quot{ℝ^2}{ℤ^2}$, $\projp^n$ y $\projcp^n$ son compactas.

\solution

\doneby{Guille}


\end{problem}

\begin{problem} Probar que si una variedad $M$ posee un atlas que consiste de dos cartas cuya intersección es conexa entonces $M$ es orientable. Deducir que $\crc[n]$ es orientable.

\solution

Según la \fref{def:VariedadOrientable}, una variedad es orientable si existe un atlas $A = \set{(U_i,φ_i)}_{i∈I}$ tal que $\abs{\Dif (φ_j○\inv{φ_i})} > 0$ para cualesquiera $j,i ∈ I$. Más generalmente, nos vale con que la diferencial de los difeomorfismos de cambio de carta tengan el mismo signo.

Por ser el cambio de carta un difeomorfismo, su diferencial es continua y además no se anula nunca. La única forma de que cambiase de signo es que la intersección de cartas no fuese conexa (podría saltar de valor entre componentes conexas), pero por hipótesis la intersección es conexa, luego la diferencial mantiene signo y por lo tanto la variedad es orientable.

En $\crc[n]$ se puede dar siempre un atlas con dos cartas con intersección conexa a través de la proyección estereográfica, luego es orientable.

\end{problem}

\begin{problem}[8] Sean $M$ y $N$ dos variedades orientables, y sea $\appl{f}{M}{N}$ una aplicación diferenciable.

\ppart Define el concepto de morfismo que preserva la orientación y pon un ejemplo de uno (y de otro que no lo sea).

\ppart Demostrar que una variedad cociente $\quot{M}{G}$ en la que $M$ es orientable y los elementos de $G$ preservan la orientación es orientable.

\ppart Deducir que los espacios proyectivos de dimensión impar son orientables.

\solution

\spart

La definición de orientación de Geometría Diferencial \citep[Def. IV.7]{ApuntesGeoDif} era bastante cómoda, ya que sólo dependía de la existencia de una $n$-forma de volumen que no se anulase. La cuestión es que dudo bastante que podamos usar eso aquí, así que toca ir a la definición fea, dependiente de las cartas.

\begin{defn}[Aplicación\IS compatible con la orientación] Sean $M$ y $N$ dos variedades orientables, y sea $\appl{f}{M}{N}$ una aplicación diferenciable. Diremos que $f$ es compatible con la orientación (o que preserva la orientación) si y sólo si, para dos cartas cualesquiera $(U_i, φ_i)$ y $(U_j, φ_j)$ de $M$ y $N$ respectivamente, el jacobiano dado por \[ \Dif (φ_i ○ f ○ \inv{φ_j})\] es positivo en la región en la que esté definido.
\end{defn}

Los ejemplos no se me ocurren ahora.

\spart

\end{problem}


\bibliography{../Apuntes}{}
\printindex
\end{document}
