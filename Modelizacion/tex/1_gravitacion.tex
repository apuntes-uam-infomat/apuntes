\chapter{Gravitación y leyes de kepler}
\section{Leyes de Kepler}

Kepler, a principios del siglo XVII, enunció unas leyes experimentales (con datos de Tycho Brahc):

\begin{lemma}[1ª Ley de Kepler]
	Las órbitas de los planetas son elipses con el Sol en uno de los focos.
\end{lemma}

\begin{lemma}[2ª Ley de Kepler]
	La linea que une un planeta y el Sol barre áreas iguales en tiempos iguales.
\end{lemma}

\begin{lemma}[3ª Ley de Kepler]
	El cuadrado del periodo orbital es proporcional al cubo del semieje mayor de la órbita.
\end{lemma}

\section{Newton}

Gracias a Newton, en 1985, las leyes de Kepler se explican matemáticamente suponiendo que la fuerza es $ \frac{-GMm}{r^2} $ en la dirección del radio vector.



$$ \vec{F} = m\ga$$

Que se deriva de la posición de una párticula: $\vr = (x(t), y(t), z(t))$. De lo que se deriva $ \ga = \frac{d^2\vr}{d t^2} $.

Esta fórmula describe, por ejemplo, la atracción que ejerce el Sol sobre la Tierra, por ello el signo negativo. En el caso de el Sol y la Tierra, la fuerza recíproca no se tiene en cuenta ya que el Sol es demasiado pesado como para ser influido apreciablemente por la gravedad de la Tierra. Hay que hacer mediciones muy finas para poder detectar estas perturbaciones.

\textbf{Despreciamos la fuerza de los planetas sobre el Sol y de los planetas entre ellos.}

Por lo tanto, modelamos tomando solo un planeta y asumiendo un Sol fijo.


Expresado en forma vectorial:

$$ \frac{-GMm}{{||r||}^2} \frac{\vr}{{||r||}^2}  =  m \frac{d^2\vr}{d t^2}$$


Tenemos que probar que las soluciones $\vr = \vr(t)$
 de la siguiente ecuación diferencial satisfacen las leyes de Kepler (con las condiciones iniciales de los planetas)


%Esto es un sistema con una llave
$$
\begin{cases}
 x'' = \frac{-GMx}{(x^2 + y^2 + z^2)^{3/2}}\\
 y'' = \frac{-GMy}{(x^2 + y^2 + z^2)^{3/2}}\\
 z'' = \frac{-GMz}{(x^2 + y^2 + z^2)^{3/2}}\\
\end{cases}
$$


 Como problema matemático (sin saber trucos de Física) no es facil probar características de la elipse. 

 \begin{obs}
 No se puede calcular explicitamente $x$,$y$ y $z$, en función de $t$. $\vr = \vr(t)$ no tiene una fórmula cerrada (en términos de funciones elementales\footnote{Las de la calculadora (sin, cos...)}). Aun así se puede demostrar que la fórmula es una elipse.
 \end{obs}


 %Un método es un truco que sirve varias veces
Vamos a probar que basta demostrar que la curva $\vr = \vr(t)$ está contenida en un plano. 

Un posible método es usar que una curva es plana $\Leftrightarrow \tau \text{(torsión)} = 0$ y aplicar la fórmula de la torsión. Dicha fórmula involucra un determinante con una fila $r$ y otra $r''$, como son proporcionales¿? la torsión sale 0.


Nuestro método más sencillo se basa en tomar $L(t) = \vr(x) \times \frac{d\vr}{dt}$.

$$\vec{L}(t) = \frac{d\vr}{dt} \times \frac{d\vr}{dt} + \vr(t) + \frac{d^2\vr}{dt^2} = \vec{0}$$ 

$\Rightarrow \vec{L}$ es un vector constante $\Rightarrow \vr \perp \vec{L} \Rightarrow \vr(t)$ está contenida en el plano $\vec{L}(x,y,z) = 0 $.
(pensar $\vec{L} = \vec{0}$). Transformamos además $\vr \rightarrow G\vr$, siengo G un giro (matriz ortogonal).

$$\frac{d^2\vr}{dt^2} = \text{cte}\frac{\vr}{||\vr||^3} \Leftrightarrow \frac{d^2(G\vr)}{dt^2} = \text{cte}\frac{G\vr}{||G\vr||^3}$$

En definitiva "girando la cabeza" (aplicando el cambio $\vr \rightarrow G\vr$) podemos suponer que la curva $\vr = \vr(t)$ está contenida en el plano $z = 0 \Rightarrow$ Suponesmos $z(t) = 0$.


%Esto es un sistema con una llave
$$
\begin{cases}
 x'' = k\frac{x}{(x^2 + y^2)^{3/2}}\\
 y'' = k\frac{y}{(x^2 + y^2)^{3/2}}\\
\end{cases}
$$

Ahora hay que probar las leyes de kepler para $t \rightarrow (x(t), y(t))$.





