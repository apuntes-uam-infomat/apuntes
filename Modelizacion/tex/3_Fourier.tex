
\chapter{Aplicaciones del análisis de Fourier}
\section{Desarrollo de Fourier}
\textbf{Idea del análisis de Fourier:} "toda" señal se puede descomponer en "tonos puros" (armónicos) de frecuencia fija (típicamente $\sin(\alpha n x)$ , $\cos(\alpha n x)$  $ n \in \ent$).

Dibujo con caption ($\frac{1}{4} - \frac{2}{\pi^2} \sum_{n impares} \frac{1}{n^2} \cos(2\pi n x)$)

\begin{example}
	Cogemos $x=0$:
	$$0 = \frac{1}{4} - \frac{2}{\pi^2} \sum_{\text{n impares}} \frac{1}{n^2}$$ 
	$$1 + \frac{1}{3^2} + \frac{1}{5^2} + \ldots = \frac{\pi^2}{8}$$
\end{example}

\subsection{Aplicaciones}
\begin{itemize}
	\item Muchas aplicaciones de ingeniería están basadas en estas ideas. (JPEG,(MP3) MPEG, telecomunicaciones)
	
	Hay tonos puros (frecuencias) que se eliminan o modifican porque n tienen mucha influencia o son ruido.
	
	Menos frecuencia $\rightarrow$ Menos información $\rightarrow$ Compresión (pérdidas)
	
	En ingeniería una aplicación muy común es utilizar esto como filtro, en el caso del MP3 se eliminan las frecuencias que no oimos.
	
	\item En matemáticas y física: Hay problemas difíciles para funciones generales y fáciles para "tonos puros" (senos y cosenos).
	
		\textbf{Interpretación de Copenhague} : Las partículas tienen funciones de ondas y en los experimentos solo se detectan los tonos puros que componen estas funciones con una probabilidad que depende de su amplitud.
		
	Vamos a ver aplicaciones de esto:
	
	\textbf{\textit{Pasar de analógico a digital}}
	
	Vamos a hacer análisis de Fourier discreto con ondas digitalizadas.
	
	\begin{example}
		Vamos a estudiar la función $\sin(\frac{2\pi}{T}\cdot x)$
		
		Dibujo
		
		que es un caso particular de $\sin(\frac{2\pi}{T}\cdot k x)$ (oscila k veces en [0,T]).
		
		Dibujo
		
		Si digitalizamos la función, hacemos que x solo tenga valores discretos: $\sin(\frac{2\pi}{T}\cdot n)$
		
		Dibujo
		
		LLamamos $f(n)$ a la función discretizada, $n \in \ent$ ; Con N periódica $f(n + N) = f(n)$
		
		Matemáticamente pensamos $f(n)$ como :
		$$f : \ent/N\ent \rightarrow \mathbb{C} $$
	\end{example} 
	\obs Hemos puesto que f va a $\mathbb{C}$ porque $e^{ix} = \cos x * i\sin x$ permite escribir senos y cosenos al mismo tiempo.
	
	$$\cos x = \frac{e^{ix} + e^{-ix}}{2}$$
	$$\sin x = \frac{e^{ix} - e^{-ix}}{2i}$$
	
\end{itemize}

\begin{prop}[Análisis de Fourier en $\ent/N\ent$]
	
	Cualquier $f : \ent/N\ent \rightarrow \mathbb{C}$ se puede escribir como :
	$$f(n) = \frac{1}{N}\sum_{m\in \ent/N\ent}\widehat{f}(m)\cdot e(\frac{nm}{N})$$
	donde $e(x) = e^{2\pi ix}$
	
	$\widehat{f}(n)$ es la \textbf{transformada de Fourier discreta}
	$$\widehat{f}(n) = \sum_{m\in \ent/N\ent}\widehat{f}(m)\cdot e(\frac{-nm}{N})$$
	
\end{prop}
\obs Hay un algoritmo (FFT) para calcular los $\widehat{f}(m)$

\begin{proof}
	Definimos la función $\delta : \ent/N\ent \rightarrow \mathbb{C}$ como :
	$$\delta (n) = \begin{cases}
	1 \rightarrow n=0\\
	0 \rightarrow n\neq 0 \\
	\end{cases}$$
	Y decimos que también se puede escribir como:
	$$\deta (n) = \frac{1}{N} \sum_{m_0}^{N-1} e(\frac{nm}{N})$$
	Donde $e(\frac{nm}{N}) = e^{2\pi inm/N}$
	
	Es claro que $\deta(0) = 1$ , pero ¿se cumple que si $n\neq 0 \implies \delta (n) = 0$?
	
	Si se cumple, porque si lo pensamos como una progresión geométrica 
\end{proof}