% -*- root: ../LogicaMatematica.tex -*-
\section{Hoja 5}

Asumimos gen\'erica e informalmente la no trivialidad. Al escribir f\'ormulas, se permiten
las abreviaciones razonables. 

\begin{problem} 
Probar que en ZFC no existe una cadena infinita descendente $z_1 \ni z_2 \ni z_3 \ni \dots \ $.

\solution
% Solución de Wikipedia: https://en.wikipedia.org/wiki/Axiom_of_regularity

La demostración de este hecho se basa en \textbf{el axioma de Fundación}, que dice:
\[\forall x \left[(x\neq \emptyset)\implies (\exists y \in x (\forall z \in x (z\notin y)))\right]\]

Supongamos que existe una cadena infinita descendente: $z_1 \ni z_2 \ni z_3 \ni \dots \ $ entonces tendremos una serie de funciones $\appl{f}{\nat}{z_i}$ tal que $f(i)=z_i$.

Estas funciones, por definición cumplen que $f(n+1) \in f(n)$.

Definimos ahora el conjunto $S=\{f(n) \text{ con } n\in \nat \}$, que representa el rango de $f$, que sabemos que puede ser visto como un conjunto debido al \textbf{axioma de reemplazo}.

El axioma de fundación nos garantiza la existencia de $B \in S$ tal que $B \cap S = \emptyset$. 

Por definición de $S$, tenemos que $\exists n_0$ tal que $B=f(n_0)$ pero hemos dicho, desde el principio, que $f(n_0)$ contiene a $f(n_0+1)$, que también es un elemento de $S$.

Por tanto, llegamos a una contradicción, pues empezamos suponiendo que $B \cap S = \emptyset$, por lo que es imposible que exista la cadena infinita descendente que originó este razonamiento.

\end{problem}


\begin{problem} 
Sea $T$ la teor\'{\i}a (en un lenguaje de primer orden) 
que se obtiene al a$\operatorname{\tilde{n}}$adir a los tres axiomas de la teor\'{\i}a de  grupos,
un cuarto axioma $\forall x\forall y (x = y)$. Probar que $T$ es completa.
\solution


Recordamos que una teoría será completa, por definición, si es consistente y $\forall p, \ T \vdash p \Or T\vdash \neg p$, no puede ser que no podamos deducir ninguna.

Juntando los tres axiomas vistos en clase y el que añade el enunciado tenemos:
\begin{enumerate}
\item $\varphi_1: \forall x\forall y\forall z ((x\cdot y)\cdot z) = (x\cdot(y\cdot z))$
\item $\varphi_2: \forall x(x\cdot e = e\cdot x = x)$
\item $\varphi_3: \forall x\exists y (x\cdot y = e)$
\item $\varphi_4: \forall x\forall y (x=y)$
\end{enumerate}

El último axioma que hemos añadido hace que sólo haya un modelo de esta teoría, es decir, sólo hay un grupo que satisface todos estos axiomas simultáneamente y es el grupo trivial formado por el elemento neutro: $G=\{id\}$. Por tener un modelo, y aplicando el Teorema de Completitud (versión 2) tenemos que T es consistente.

Puesto que sólo tenemos un modelo, es obvio cualquier FBF $\varphi$ describirá una propiedad para la que pueden ocurrir dos opciones:
\begin{enumerate}
\item $G$ satisface $\varphi$ por lo que $G(\phi)=\top \implies  T \vDash \varphi $. Evidentemente si la propiedad se satisface para el modelo de la teoría (el grupo $G$), su contrario no lo hará, es decir, en este caso tendríamos $G \nvDash \neg \varphi \implies G \nvdash \neg \varphi$


\item $G$ no satisface $\varphi$, por lo que $G(\varphi) = \perp \implies G(\neg \varphi) = T$, es decir, $G$ satisface su opuesto. Por tanto tendríamos $G \vDash \varphi \implies G \vdash \varphi$
\end{enumerate}

La idea clave de esta explicación es que nuestra teoría tiene un único grupo que es modelo lo que evita la posibilidad de que ciertos grupos de la teoría satisfagan $\varphi$ y otros $\neg \varphi$, como ocurría con esta misma teoría antes de añadir el cuarto axioma.


\end{problem}

\begin{problem} Sea $T$ la teor\'{\i}a, 
 en un lenguaje de primer orden  $L$, que se obtiene al a$\operatorname{\tilde{n}}$adir a los
axiomas de Peano la coleccion $\{ \psi_n: n\in\mathbb{N}\}$, donde
$\psi_n$ es la fbf $\exists x > \underline{n}$,  y  por cada n\'umero natural $n$,  $ \underline{n}$ es una constante
en $L$ cuya interpretaci\'on es $n$.
 Decidir razonadamente si 
$T$ es consistente. Decidir razonadamente si $T$ tiene un modelo. 
\obs N\'otese que  cualquier
$L$-estructura debe contener a $\mathbb{N}$.
\solution


Como hemos visto en clase, cualquier interpretación (en este caso la L-estructura) que satisfaga los axiomas de Peano debe ser $\nat$ (en este sentido, $\nat$ es único).
 
En efecto vemos que $\nat$ es un modelo de $T$ pues satisface los axiomas de Peano y por ser infinito numerable satisface $\psi_n$ con $n\in\mathbb{N}$.

Puesto que tiene un modelo sabemos también que es consistente por el teorema de completitud.

\end{problem}