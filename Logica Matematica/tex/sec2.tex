\section{Lógica de primer orden (o de predicados)}
\subsection{Exposición informal y ejemplos}
Asumimos que todos los conjuntos usados para definir el lenguaje son disjuntos. Usaremos:
\begin{itemize}
	\item \textbf{Símbolos} lógicos y de puntuación:
	$$\{\neg, \vee, ),(, ``," \}$$
	También usaremos los demás símbolos logicos como abreviaciones.
	\item Tendremos un conjunto infinito numerable de \textbf{variables}:
	$$Var = \{v_1, v_2, \hdots\}$$
	Aunque en la práctica usaremos $x,y,z$.
	\item Un conjunto de constantes, que podría ser vacío:
	$$Con = \{C_0, C_1, \hdots\}$$
	\item Funciones:
	$$Fun = \{f_0, f_1, \hdots\}$$
	\item Relaciones o predicados:
	$$Rel = \{R_1, R_2, \hdots\}$$
	\item Cuatificadotes:
	$$\{\forall, \exists\}$$
	Podemos considerar $\exists x(P(x))$ como abreviación de $\neg (\forall x(\neg P(x)))$.
\end{itemize} 

\begin{example}[Expresiones de primer orden] Sabemos que $P(x) \implies P(x)$ con $x$ es libre es cierto. Por otro lado $\forall x(P(x)\implies P(x))$ tiene el mismo significado, pero $x$ ahora es una variable ligada al cuantificador.
\end{example}

\begin{example}[Expresion de segundo orden] La expresión $\forall P \forall x (P(x)\implies P(x))$ es una expresión de segundo orden al estar cuantificando sobre predicados.
\end{example}

\begin{mdframed}
	\textbf{Principio de inducción matemática en $\mathbb{N}$}:
	$$\forall P (\left[P(0)\y (\forall n(P(n)\implies P(n+1)))\right]\implies \forall n(P(n)))$$
	Es una fórmula de segundo orden.
\end{mdframed}
En una lógica de primer orden, en vez de ``principio'', tenemos un ``esquema de principios'': Un principio de inducción por cada $P$.

\begin{example}[Axiomas de la teoría de grupos: Lengiaje $L_G$]\mbox{}

\begin{itemize}

	\item Axiomas:
	\begin{itemize}
		\vspace{-3mm}
		\item $\varphi_1: \forall x\forall y\forall z ((x\cdot y)\cdot z) = (x\cdot(y\cdot z))$
		\item $\varphi_2: \forall x(x\cdot e = e\cdot x = x)$
		\item $\varphi_3: \forall x\exists y (x\cdot y = e)$
	\end{itemize}

	\item Funciones:
	$$Fun = \{\cdot\}$$
	
	\item Relaciones:
	$$Rel = \{=\}$$
	
	\item Constantes:
	$$Con = \{e\}$$
\end{itemize}

Tenemos la teoría $T_G = \{\varphi_1, \varphi_2, \varphi_3\}$. Un grupo es cualquier modelo de $T_G$. Queremos ver si podemos demostrar, a partir de la misma $\psi: \forall x\forall y(x\cdot y = y\cdot x)$. Es decir, si $T_G\vdash \psi$. Vemos que si $G$ es no abeliano, entonces $G\vDash \psi$, como conclusión tenemos $T_G\vdash \psi$.

Esto es un ejemplo de que el teorema de validez también es cierto en lógica de primer orden.
$$T_G\vdash \psi \implies T_G \vDash \psi$$

Equivalentemente
$$T_G\nvDash \psi \implies T_G \nvdash \psi$$

¿Podemos probar entonces que $T_G\vdash \neg \psi$? No, porque si $G_2$ es abeliano, entonces $G_2\nvDash \neg\psi$ y por tanto, por validez, $T_G\nvdash \neg \psi$.

Como conclusión, tenemos que $T_G$ no es una teoría completa.
\end{example}

\begin{defn}[Clausura universal]
	La clausura universal de $\phi$ es $\forall x_1, \forall x_2,\hdots, \forall x_n \phi$, donde $x_1, x_2, \hdots, x_n$ son todas las variables que aparecen libres en $\phi$.
	
	\textbf{Abreviación}: $\forall x\in z$ $(\phi(x))$ es abreviación de $\forall x ((x\in z) \y \phi (x))$.
	
	\textbf{Abreciación}: $\exists ! x(\phi (x))$ abrevia $\exists x(\phi (x) \y (\forall y (\phi \implies y = x)))$.
\end{defn}

\begin{example}
	La clausura universal de $x < 5 + y$ es $$\forall x\forall y (x < 5 + y)$$ que tiene valor de verdad \textbf{falso}.
\end{example}

\begin{example}[Axiomas de ZFC (Zermelo-Fraenkel con Elección)]
	Todos son conjuntos, hereditarios y bien fundados.
	No hay ``urelementos''\footnote{Se utiliza el prefijo alemán ``ur'' que significa primordial} o átomos.
	\begin{itemize}
		\item \textbf{Axioma 1:} Existe un conjunto vacío:
		$$\exists x (\forall x (x\notin z))$$
		(En el lenguaje de la teoría de conjuntos $L_S$, tenemos las relaciones binarias $\{\in, =\}$. Por otro lado $Fun: \emptyset$, $Cons: \emptyset$).
		\item \textbf{Axioma 2: Extensionalidad}.
		Dos conjuntos con los mismos elementos son iguales.
		$$\forall x\forall y(\forall z (z\in x\iff z\in y)\implies x = y)$$
		\begin{corol}
			El conjunto vacío es único.
		\end{corol}
		
		\item (Esquema de) \textbf{Axioma}(s) \textbf{3: Compresión}. Las FBF permiten definir subconjuntos. Hablar de subconjuntos permite evitar la paradoja de Russel.
		
		Tenemos un axioma por cada $\phi\in FBF(L_S)$. El axioma es la clausura universal
		$$\forall z \exists y \forall x(x\in y\iff((x\in z)\y \phi))$$
		
		Si no tuvieramos $z$, el ``axioma'' quedaría así.
		
		$$\exists y \forall x (x\in y\iff \phi)$$
		
		Si tomamos $\phi: x\notin x (\neg (x\in x))$
		$$\exists y \forall x (x\in y\iff x\notin x)$$
		Sabemos que $\exists y$, digamos $y_0$.
		$$\forall x (x\in y_0 \iff x\notin x)$$ 
		Como tenemos el cuantificador $\forall x$, podemos sustituir $x$ por $y_0$, obteniendo
		$$y_0\in y_0 \iff y_0\notin y_0$$
		Que es una contradicción. Esto se soluciona añadiendo $z$ como se ha hecho.
		
		\item \textbf{Axioma 4: Emparejamiento.} Si $x$ e $y$ son conjuntos, el par $\{x, y\}$ es un conjunto. Si $z$ es un conjunto, $\{\{x,y\},z\}$ es un conjunto. Si $x = y$, entonces obtenemos $\{x,x\} = \{x\}$. Formalmente
		$$\forall x\forall y\exists z\forall u(u\in z\iff \left[(u=x)\Or (u=y)\right])$$
	\end{itemize}
	
	
\end{example}

