% -*- root: ../LogicaMatematica.tex -*-
\section{Incompletitud}

Los principales enunciados básicos correspondientes a esta sección ya los hemos visto. Lo que haremos será añadir ciertas definiciones y varios resultados/teoremas importantes.

\textbf{Recordatorio de lógica proposicional}
\[\{p,q\}\vdash r \iff \vdash (p\y q) \to r\]
Para demostrar esto empezábamos apoyándonos en el teorema de la deducción:
\[\{p,q\} \vdash r \implies \{p\} \vdash q \to r\]
Aplicando de nuevo el teorema de la deducción obteníamos:
\[\{p\} \vdash q \to r \implies \vdash p \to (q \to r) \equiv (p \y q ) \to r\]

\begin{theorem}
Consideramos el lenguaje de la aritmética formal, definido como:
\[L=\{0,S_i,+_1,\cdot_1, <_1\}\]
y la teoría $\underline{N}=\{N_1,...,N_9\}$ correspondiente a los axiomas de Peano (sin incluir inducción) y definida en el libro en la página 91.

en este lenguaje, $L$, y con la teoría $\underline{N}$, toda función recursiva total $\appl{f}{\nat^k}{\nat}$ es representable para una fórmula de primer orden $\varphi_f$.

Lo mismo ocurre con los conjuntos recursivos.
\end{theorem}

\begin{prop}[Codificación]
Las fórmulas pueden codificarse usando números naturales.

Las demostraciones también.

\end{prop}

Para llevar a cabo estas codificaciones debemos definir símbolos del lenguaje $n \in \nat$ y asociar a cada variable $v_i$ el número $n(v_i)=2i$ con lo que empleamos solamente números pares.

Una vez tenemos esto podemos asignar a las operaciones los números impares como sigue:
\[n(\top) = 1, \ \ n(\perp)=3, \ \ n(\neg) = 5, \ \ n(\Or) = 7, \ \ n(\y)=9, \ \ n(=)=11\]
\[n(\exists)=13, \ \ n(\forall) =15, \ \ n('(')=17, \ \ n(')')=19, \ \ n(0)=21, \ \ n(S_1)=23\]
\[n(+_1)=25, \ \ n(\cdot_1)=27, \ \ n(<_1) = 29\]

El problema que nos deriva de esta codificación es que no es claro cómo descifrar el predicado ``230'' puesto que puede interpretarse como el símbolo asociado al 23 seguido de la variable asociada al 0, o como la variable asociada al 2 seguida de la variable asociada al 15, etc.

Empleamos la factorización única para resolver este problema. Para ello definimos la función primo($K$) que nos devuelve el primero que ocupa la posición $K+1$.

Así el número de Gödel de una palabra se calcula como:
\[g(u_0,u_1,...,u_k)=2^{n(u_0)}\times 2^{n(u_1)}\times ... \times \text{primo}(k)^{n(u_k)}\]

De esta forma, al leer el número asociado a una fórmula basta con factorizarlo, lo que nos permite obtener la fórmula original de forma unívoca.

%TODO completar. codificacion de vectores usando el dibujito ese que recorre todo.

\begin{defn}[Teoría recursiva]
Una teoría Σ es recursiva si el conjunto de números de Gödel
\[\#Σ = \{g(σ) \tq σ \in Σ\}\]
es recursivo.
\end{defn}

Esto nos dice que, dada cualquier σ, podemos decidir algorítmicamente si $σ \in Σ$ ó $σ \in Σ^c$.

\begin{defn}[Teoría de Σ (Teor(Σ))]
Al conjunto de teoremas de Σ se le denomina Teoría de Σ.
\[\text{Teor}(Σ) = \{σ \tq Σ \vdash σ\}\]
Definimos también
\[\#\text{Teor}(Σ)=\{g(σ)\tq Σ \vdash σ\}\]
\end{defn}

\begin{defn}[Teoría deducible]
Decimos que Σ es deducible si $\#\text{Teor}(Σ)$ es recursiva, es decir, existe un algoritmo que dado cualquier σ nos dice es un teorema o no.
\end{defn}

\begin{example}
Podemos comprobar que el caso $Σ = \emptyset$ es recursiva de manera trivial.

También puede verse que $\underline{N}$ es recursiva.
\end{example}

\obs Cualquier teoría finita es recursiva.

\begin{theorem}
La lógica proposicional es deducible.

De hecho, puede demostrarse que el conjunto de los números de Gödel de las tautologías, es decir $\{g(p) \tq p \text{ es tautología}\}$, es recursivo primitivo.
\end{theorem}

\begin{theorem}
El conjunto de axiomas de la lógica de primer orden es recursivo (de hecho es recursivo primitivo).
\end{theorem}

\begin{theorem}
Sea Σ una teoría recursiva, el conjunto:
\[\text{Dem}(Σ)=\{(m,n)\tq m=g(σ) \text{ y } n=g(\text{alguna demostración formal de σ})\}\]
es recursivo.
\end{theorem}
\begin{proof}
Para llevar a cabo esta demostración nos basta con encontrar un algoritmo que nos permita determinar si el par de valores (m,n) pertenece, o no, a Dem(Σ). Para ello tendremos que descodificar $m$ y $n$.

Si $g^{-1}(m)=σ$ y $g^{-1}(n)$ es una demostración de σ, entonces $(m,n)\in \text{Dem}(Σ)$.

En caso contrario $(m,n)\in \text{Dem}(Σ)$.
\end{proof}

\obs Como $\ind_{\text{Dem}(Σ)}$ es una función recursiva total, existe una fórmula en el lenguaje formal, digamos $\varphi_{\text{Dem}(Σ)}$, que la representa.

Puede demostrarse que las proyecciones sobre las coordenadas de conjuntos recursivos son recursivamente enumerables.

\begin{corol}
Si una teoría es recursiva entonces el conjunto de sus teoremas es recursivamente enumerable, es decir, existe un algoritmo que encuentra todos los teoremas de Σ.
\end{corol}
\begin{proof}
Sabemos que la proyección sobre los ejes de un conjunto recursivo es recursivamente enumerable por lo que:
\[π\left(\ind_{\text{Dem}(Σ)}\right) \text{ es recursivamente enumerable}\]

que es justo lo que necesitamos probar.
\end{proof}

\begin{corol}
\label{completaRecursivaImplicaDecidible}
Si Σ es recursiva y completa entonces es decidible.
\end{corol}
\begin{proof}
Como $\#\text{Teor}(Σ)$ es recursivamente enumerable, basta probar que su complementario es recursivamente enumerable.

Dada σ, enunciamos los teoremas de Σ sistematicamente. Como Σ es completa, tras un número finito de pasos aparecerá σ o $\neg σ$.
\end{proof}

Vamos a ver ahora los teoremas de incompletitud.

\begin{theorem}
Sea $Σ$ una teoría consistente y recursiva que contiene a $\underbar{N} = \{N_1,...,N_9\}$ (los 9 axiomas de la aritmética).

\textbf{Entonces} $Σ$ no es decidible.
\end{theorem}

\begin{proof}
Idea: Si $Σ$ fuera decidible, podríamos determinar de manera algorítmica si cualquier $σ$ es un teorema o no.

Ello nos permitiría resolver problemas insolubes que se pueden codificar en $L$ utilizando $\underbar{N}$, por ejemplo el $X$ problema de Hilbert (\ref{problema:Hilbert}), con alguna salvedad.

El problema de Hilbert pregunta por soluciones en los enteros y aquí sólo tenemos naturales, pero puede demostrarse que no tiene solución algorítmica, cuando consideramos soluciones estrictamente positivas.
%
También basta utilizar coeficientes es $ℕ$, por ejemplo, decidir si $x^n + y^n - z^n = 0$ tiene soluciones en los naturales positivos es equivalente a decidirlo para $x^n + y^n = z^n$

\end{proof}

\begin{corol}
La lógica de primer orden no es decidible, es decir, el conjunto de números de Gödel que pueden deducirse de los axiomas no es recursivo.
\end{corol}

\begin{proof}
En este caso, $Σ = Ø$. No estamos añadiendo nada a los axiomas lógicos.

Utilizando el teorema de la deducción y Modus Ponens,
\[\underbar{N} \vdash σ \dimplies \vdash  (N_1 \y N2 \y ... \y N9) \to σ\]

Si la lógica de primer orden fuera decidible, la teoría $\underbar{N}$ también lo sería y podríamos resolver problemas insolubles.
\end{proof}

\begin{corol}
Sea $Σ$ una teoría consistente y recursiva que contiene a $\underbar{N}$. Entonces $Σ$ no es completa.
\end{corol}

\obs No es completa ni completable. Si añadiéramos una nueva sentencia que no podemos probar, mantendríamos la consistencia, pero aparecerían nuevas sentencias que no podemos probar.

\begin{proof}
Si $Σ$ fuera completa, sería decidible (por el corolario \ref{completaRecursivaImplicaDecidible})
\end{proof}



Vamos a recordar algo visto con anterioridad, sobre la codificación. Teníamos $n(\perp) = 3$ y $g(\perp) = 2^3$.
%
Informalmente, $\underbar{N} \vdash 2+2=4$ es cierto. Formalmente, escribiríamos:
\[\underbar{N} \vdash S_1^2(\underbar{0}),+,\vdash S_1^2(\underbar{0}) = \vdash S_1^4(\underbar{0})\]

Ahora nos preguntamos, ¿Podemos demostrar $\underbar{N} \vdash 2+2 = 5$? \textbf{No.}
%
¿Y podríamos probar $\underbar{N} \vdash \left( \underbar{N} \not \vdash 2+2=5 \right)$?
\textbf{Tampoco}, por el segundo teorema de incompletitud.

La segunda parte del recordatorio es que $Dem(Σ)$ es recursivo. Ello implicaba que $\exists φ_{Dem(Σ)}$ que ``representaba'' a $Dem(Σ)$, es decir, si $$(m,n) \in Dem(Σ), Σ\vdash φ_{Dem(Σ)}(S_1^m(\underbar{0},S_1^n(\underbar{0}))$$
Por otro lado: $$(m,n) \not \in Dem(Σ), Σ\vdash \neg φ_{Dem(Σ)}(S_1^m(\underbar{0},S_1^n(\underbar{0}))$$.

Y utilizando esto, se puede formalizar qué es la consistencia.

\begin{defn}[Consistencia (CON)]
\[CON(Σ) := \neg ∃ v_0 \left( φ_{Dem(Σ)}(\perp,v_0) \right)\]
Aunque $\perp$ no puede aparecer ahí (porque estamos codificando todo en números), sino que tiene que aparecer su número de Gödel, es decir $g(\perp) = 2^3 = 8$
\[CON(Σ) := \neg ∃ v_0 \left( φ_{Dem(Σ)}(S_1^8(\underbar{0}),v_0) \right)\]
\end{defn}

\begin{theorem}[Teorema\IS de incompletitud (Segundo)]
Si $Σ$ es recursiva, consistente y contiene a $\underbrace{N}$, entonces $Σ\not\vdash CON(Σ)$
\end{theorem}

Este es el teorema que implica que \[\underbar{N} \vdash \left( \underbar{N} \not \vdash 2+2=5 \right)\]
Demostrar esa afirmación, sería demostrar la consistencia de $Σ$ y eso no es demostrable como hemos visto en el segundo teorema de incompletitud.


\paragraph{Comentario sobre la hoja 8}. Esta hoja está puesta por la poca confianza que hay en el teorema de compacidad.

Definimos:

\[F = \{A\subset ℕ : A^c \text{ es finito}\}.\]
\[F\subset U^{\text{ultrafiltro}}\]
Entonces, podemos definir $ℕ^ℕ$, con la siguiente inclusión $f(n) : ℕ^ℕ \to ℕ$ construida de la siguiente manera $f(n) = (n,n, ... ,n)$

Por lo que: $(a_n)_{n=1}^{∞} \sim (b_n)_{n=1}^{∞}$ si $\{a_n = b_n\} \in U$ \footnote{Es decir, 2 sucesiones son iguales si son iguales en casi todo punto (iguales salvo en conjuntos de probabilidad 0 y los elementos que tienen probabilidad 0 son los del ultrafiltro (o algo así))}

Definimos también $\mathcal{U} := ℕ^ℕ / \sim$

\textbf{Afirmación} \[c = \left\{\left[ (n)_{n=1}^{∞} \right] > \left[ (k)_{n=1}^{∞} \right] ∀k∈ℕ\right\}\]

Tenemos también una inclusión de $ℕ$ en $\mathcal{U}$.

Siento esta mierda de comentario, he hecho lo que he podido.
