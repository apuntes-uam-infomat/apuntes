% -*- root: ../LogicaMatematica.tex -*-
\section{Hoja 8}

\


Para el Lunes 14/12/2015.
Se pueden entregar ejercicios individualmente o en grupo. Hacerlo
en grupo no penaliza.


\

Los 7  primeros problemas en esta hoja proporcionan una ``construcción" moderadamente explicita (Zorn) de modelos
no estándar de los naturales. Consideramos probabilidades $P$ finitamente aditivas, que sólo toman los valores
$0$ y $1$.

\begin{problem}
Sea $P$ una probabilidad finitamente aditiva, definida en un álgebra de subconjuntos de $\mathbb{N}$,
con valores en $\{0,1\}$. Los conjuntos con probabilidad 1 forman un {\em filtro}. Definir {\em filtro} sin utilizar
probabilidades, y probar que las dos definiciones coinciden.

\solution

\end{problem}

\begin{problem}
Probar que todo filtro de subconjuntos de un conjunto $S$,
 puede extenderse a un ultrafiltro (a un filtro maximal con respecto a la inclusión) en $S$.
Sugerencia: Zorn.

\solution

\end{problem}

\begin{problem}
Probar que si $U$ es un ultrafiltro en $S$, para todo $A\subset S$, o $A\in U$ o $A^c\in U$.

\solution

\end{problem}

\begin{problem}
Probar que si $U$ es un ultrafiltro en $S$, y  $A_1 \cup \cdots \cup A_n = S$, entonces existe una $j\in \{ 1, \dots, n\}$ tal que
$A_j\in U$.

\solution

\end{problem}

\begin{problem}
Sea  $\prod_{n\in \mathbb{N}} X_n$ un producto cartesiano de conjuntos no vacios, y sea $U$ un ultrafiltro en $\mathbb{N}$.
Para $x,y \in  \prod_{n\in \mathbb{N}}\ X_n$, definimos $x\sim y$ si $x$ e $y$ son iguales casi seguro, es
decir si $\{n\in \mathbb{N}: x_n = y_n\}\in U$. Probar que $\sim$ es una relación de equivalencia.
Al conjunto $\mathcal{U}:=\prod_{n\in \mathbb{N}} X_n/\sim$ se le denomina {\em ultraproducto}.

\solution

\end{problem}

\begin{problem}
Tomando $X_n = \mathbb{N}$ para todo $n$ en la construcción anterior, probar que $\mathcal{U}$ es un modelo de los axiomas de Peano. Sugerencia: Buscar en la literatura el Teorema Fundamental de los Ultraproductos, de {\L}o\'s, e invocarlo.

\solution

\end{problem}

\begin{problem}
Probar que si el ultrafiltro $U$ contiene a todos los subconjuntos de $\mathbb{N}$ con complemento finito, entonces
existe una $c\in \mathcal{U}$ tal que para todo $n\in \mathbb{N}, c>n$. Sugerencia: tomar $c_n = n$.

\solution

\end{problem}

\begin{problem}
Dada la MT con alfabeto $\{B, 1, 2\}$, definida por $\{(q_0 \ 1 \ 2 \ q_0), (q_0 \ 2 \ D \ q_0),  (q_0 \ B \ D \ q_0)\}$,
determinar su output cuando el input es $1 \ 1  \ B \ B \ 1 \  B  \ 2  \ B \  $ (el resto de la cinta son $B's$). Describir en general
que hace esta MT, y si se detiene el algún momento o no.

\solution

\end{problem}


