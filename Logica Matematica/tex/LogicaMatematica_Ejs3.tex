\section{Hoja 3}

\begin{problem}[1]
Sea $\sigma$ una valoraci\'on Booleana. Expresar $\sigma(p\vee q)$ y
$\sigma(p\wedge q)$ en t\'erminos de
$\sigma(p)$ y $\sigma(q)$,
usando la suma y el producto en $\mathbb{Z}_2$.
\solution

\end{problem}

\begin{problem}[2]
Demostrar que los siguientes conjuntos de conectivas son completos:
$\{\neg, \vee\}$,  $\{\neg, \wedge\}$,  $\{\neg, \to\}$, $\{\neg, \leftrightarrow\}$,
$\{\to, \perp\}$.
\solution

\end{problem}



\begin{problem}[3]
Demostrar que el  conjunto de conectivas $\{\vee, \wedge\}$ no es completo.
Sugerencia: usar monoton\'{\i}a.
\solution

\end{problem}



\begin{problem}[4]
Leer con cuidado el Lema 2.1.2 p. 16, y demostrar la columna derecha de los apartados
4-7 (distributividad, absorci\'on, De Morgan, tercero exclu\'{\i}do).
\solution

\end{problem}



\begin{problem}[5]
Demostrar el Lema 2.1.3 p. 17.
\solution

\end{problem}



\begin{problem}[6]
Comprobar que los tres axiomas en el ejercico 1 de la hoja 2 son tautolog\'{\i}as.
\solution

\end{problem}



\begin{problem}[7]
Comprobar que los ocho axiomas en las p\'aginas 20-21 del libro son tautolog\'{\i}as.
\solution

\end{problem}



\begin{problem}[8]
Reescribir  los ocho axiomas en las p\'aginas 20-21 del libro
usando s\'olo las conectivas en $\{\neg, \to\}$,  y  comprobar que dichos axiomas
se deducen de los   tres axiomas en el ejercico 1 de la hoja 2.
\solution

\end{problem}

\begin{problem}[9]
Obtener $\vdash (p\to p)$ a partir del Lema de la Deducci\'on.
\solution

\end{problem}

