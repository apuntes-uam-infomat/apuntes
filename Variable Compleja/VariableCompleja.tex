\documentclass{apuntes}

\usepackage{tikz}
\usepackage{tikz-3dplot}
\usepackage{pgfplots}

\title{Variable Compleja}
\author{Pedro Valero Mejía}
\date{14/15 C2}

\usetikzlibrary{arrows,calc,shapes}
\usetikzlibrary{calc,fadings,decorations.pathreplacing}

\newcommand\pgfmathsinandcos[3]{%
  \pgfmathsetmacro#1{sin(#3)}%
  \pgfmathsetmacro#2{cos(#3)}%
}
\newcommand\LongitudePlane[3][current plane]{%
  \pgfmathsinandcos\sinEl\cosEl{#2} % elevation
  \pgfmathsinandcos\sint\cost{#3} % azimuth
  \tikzset{#1/.estyle={cm={\cost,\sint*\sinEl,0,\cosEl,(0,0)}}}
}
\newcommand\LatitudePlane[3][current plane]{%
  \pgfmathsinandcos\sinEl\cosEl{#2} % elevation
  \pgfmathsinandcos\sint\cost{#3} % latitude
  \pgfmathsetmacro\yshift{\cosEl*\sint}
  \tikzset{#1/.estyle={cm={\cost,0,0,\cost*\sinEl,(0,\yshift)}}} %
}
\newcommand\DrawLongitudeCircle[2][1]{
  \LongitudePlane{\angEl}{#2}
  \tikzset{current plane/.prefix style={scale=#1}}
   % angle of "visibility"
  \pgfmathsetmacro\angVis{atan(sin(#2)*cos(\angEl)/sin(\angEl))} %
  \draw[current plane] (\angVis:1) arc (\angVis:\angVis+180:1);
  \draw[current plane,dashed] (\angVis-180:1) arc (\angVis-180:\angVis:1);
}
\newcommand\DrawLatitudeCircle[2][1]{
  \LatitudePlane{\angEl}{#2}
  \tikzset{current plane/.prefix style={scale=#1}}
  \pgfmathsetmacro\sinVis{sin(#2)/cos(#2)*sin(\angEl)/cos(\angEl)}
  % angle of "visibility"
  \pgfmathsetmacro\angVis{asin(min(1,max(\sinVis,-1)))}
  \draw[current plane] (\angVis:1) arc (\angVis:-\angVis-180:1);
  \draw[current plane,dashed] (180-\angVis:1) arc (180-\angVis:\angVis:1);
}

%% document-wide tikz options and styles

\tikzset{%
  >=latex, % option for nice arrows
  inner sep=0pt,%
  outer sep=2pt,%
  mark coordinate/.style={inner sep=0pt,outer sep=0pt,minimum size=3pt,
    fill=black,circle}%
}

%Libro con problemas resueltos:
%  Pestana, D., Rodriguez, J.M. Marcellan, F., curso práctico de variable compljea y teoría de transformadas. Editorial Síntesis 1999

% Paquetes adicionales

% --------------------

\begin{document}
\pagestyle{plain}
\maketitle

\tableofcontents
\newpage
% Contenido.
\chapter{Introducción}
Se trata de una de las ramas más bonitas del análisis. Puesto que nos reduciremos a trabajar con un conjunto de funciones selecto al que le exigimos más propiedades, obtendremos resultados más rígidos que en el análisis habitual en $\real$. Se estudiarán las funciones analíticas que son generalización de los polinomios en z.

\begin{defn}[Función analítica\IS]
Una funcíón $\appl{f}{\Omega \subset \cplex}{\cplex}$ es análítica en $z_0\in\Omega$ si y sólo si:
\[\exists B(z_0,R)\subset \Omega \tq f(x)=\sum_{n=0}^{\infty}a_n(z-z_0) \ \forall z \in B(z_0, R)\]
\end{defn}

Hasta ahora siempre hemos estudiado el concepto de derivadas en un punto siendo ese punto un punto de los reales pero, ¿qué ocurre si ese punto está en los complejos?. Bien, para eso tenemos una nueva definición

\begin{defn}[Función holomorfa]
Una función es holomorfa en $z_0$ si:
\[\exists f'(x_0)=\lim_{h \to 0}\frac{f(z_0+h)-f(z_0)}{h}, \ h \in \cplex\]
\end{defn}

Estas dos definiciones son equivalentes, como veremos más adelante.

Trabajando con este tipo de funciones, tenemos que si una función tiene una derivada, entonces tiene todas las derivadas. Estas funciones, en general, tienen muchas propiedades similares a las de los polinomios.

Algunas \textbf{propiedades} son:
\begin{itemize}
\item Los ceros de una función holomorfa (no constante) son aislados.

De esta propiedad podemos deducir que si dos funciones holomorfas coinciden en una curva, su diferencia a lo largo de esa curva coincidirá con la función 0 y, puesto que los ceros son aislados, tendremos que la diferencia ha de ser la función constante 0.

Por tanto, dos funciones holomorfas que coinciden en una curva son iguales en todos punto.

\item Si $f$ es holomorfa (entera) en $\cplex$ y acotada entonces $f$ es constante. Esta propiedad es el resultado del \textbf{Teorema de Louville}

De esta propiedad podemos deducir el Teorema Fundamental de Álgebra.

Supongamos que $P(z)$ es un polinomio no constante, sin ninguna raíz. En ese caso tendríamos que $f(z)=\frac{1}{p(z)}$ estaría bien definido, estaría acotada y pertenecería a los complejos.

Aplicando la propiedad tendríamos que $f(z)$ es una constante lo que indicaría que $P(z)$ es una constante y llegamos a contradicción.

Otra consecuencia sería que $P(\cplex)=\cplex$

\item Si $f$ es holomorfa en $\cplex$  (entera) y no constante, entonces $f(\cplex)$ omite, como mucho, un punto. Esta propiedad es el resultado del \textbf{Teorema pequeño de Picard}

Un ejemplo sería la función $f(z)=e^z$.

\item Si $f$ es holomorfa en $\Omega\setminus \{z_0\}$ con $z_0$ una singularidad esencial entonces $f(\Omega\setminus \{z_0\})$ es $\cplex$ sin, como mucho, dos puntos. Esta propiedad es el resutlado del \textbf{Teorema grande de Picard}.

Los dos teoremas que originan estas propiedades se estudiarán con detenimiento en el curso de Variable Compleja 2.
\end{itemize}

\chapter{Números complejos y funciones}
\section{Operaciones aritméticas en el cuerpo de los números complejos}
En primer lugar, vamos a ver cómo y por qué se construye el cuerpo de los complejos.
Partimos de las siguientes propiedades de los reales:

\begin{itemize}
\item $(\mathbb{R}, +, \cdot)$ es un cuerpo.
\item $\mathbb{R}$ tiene una ordenación.
\item Todo conjunto en $\mathbb{R}$ acotado superiormente tiene un supremo (axioma del supremo).
\item En $\mathbb{R}$, la ecuación

\begin{equation} \label{eq:basic}
x^2+1=0
\end{equation}

no tiene solución pues $\alpha^2+1>0 \ \forall \alpha \in \mathbb{R}$
\end{itemize}
Buscamos el menor cuerpo que contenga a $(\mathbb{R}, +, \cdot)$ como subcuerpo y donde \eqref{eq:basic} tenga solución.

\begin{defn}[Número complejo]
Un número complejo es un par ordenado $(a,b)$ con $a,b\in\mathbb{R}$ que satisface
\[(a,b)=(c,d) \iff a=c\ \text{y} \ b=d\]

Al conjunto de los números complejos lo denotamos por $\mathbb{C}$ y lo dotamos de las siguientes operaciones.
\[(a,b)+(c,d) = (a+c, b+d)\]
\[(a,b)\cdot(c,d) = (ac-bd, bc+ad)\]
\end{defn}

\subsection{Inverso multiplicativo}

Dado que $(\mathbb{C}, +, \cdot)$ es un cuerpo, ha de tener inverso multiplicativo. El inverso multiplicativo de $(a,b)$ se tiene a partir de

\begin{align*}
    (a,b)\cdot(c,d) & = (1,0)\\
    (ac-bd, bc+ad) & = (1,0)
\end{align*}

De donde se obteiene el sistema:
\[
\left\{
\begin{array}{cc}
    ac-bd & = 1\\
    bc+ad & = 0\\
\end{array}
\right.
\]

Vamos a contemplar dos casos.

\begin{itemize}
\item \textbf{$a \neq 0$}\\

Tenemos
\[c = \frac{1+bd}{a} \]
\[b(\frac{1+bd}{a})+ad = 0 \iff (a^2+b^2)d = -b\]

De donde obtenemos

\begin{align*}
d & = \frac{-b}{a^2+b^2}\\
c & = \frac{a}{a^2+b^2}\\
\end{align*}

\item \textbf{$a=0$}

$b \neq 0$ porque el elemento $(0,0)$ no tiene inverso.

Tenemos por tanto:
\begin{align*}
d & = \frac{-1}{b}\\
c & = 0\\
\end{align*}
\end{itemize}

Una vez definidas todas las operaciones del cuerpo $\cplex$, podemos considerar el cuerpo $\cplex_0=\{(a,0): a \in \real\} \subset \cplex$ y la aplicación

\begin{equation} \label{eq:isom}
\appl{f}{\real}{\cplex_0}
\end{equation}

tal que $f(a)=(a,0)$ es un isomorfismo de cuerpos por lo que podemos identificar $\real$ con $\cplex_0$.

El número complejo $(0,1)$ es solución de $x^2+1=0$ pues
\[(0,1)\cdot(1,0)+(1,0)=(-1,)+(1,0)=(0,0)\]

Ahora definimos la \textbf{notación} $(0,1)=i$ tal que $i^2=-1$ y, basándonos en el isomorfismo \eqref{eq:isom}, identificamos $(a,0)=a$ obteniendo así la notación conocida para números complejos:
\[(a,b)=a+bi\]

Con esta \textbf{notación} los complejos obedecen las mismas reglas aritméticas que cumplían los reales.

Para calcular el inverso multiplicativo de un número complejo procedemos multiplicando y dividiendo por su conjugado:
\[\frac{1}{a+bi}=\frac{a-bi}{(a+bi)(a-bi)} = \frac{a-bi}{a^2+b^2} = \frac{a}{a^2+b^2}-\frac{bi}{a^2+b^2}\]

Gauss demostró el Teorema Fundamental del Álgebra que nos garantiza que cualquier polinomio con coeficientes complejos de grado n tiene n soluciones complejas.

\section{Conjugación}
Vamos a ver algunas definiciones importantes y necesarias a la hora de trabajar con números complejos:

\begin{defn}[Parte real (Re(z))]
Es la parte del número complejo que no aparece multiplicada por $i$ en la forma
\[z = a + bi\]
\end{defn}

\begin{defn}[Parte imaginaria (Im(z))]
Es la parte del número complejo que aparece multiplicada por $i$ en la forma
\[z = a + bi\]
\end{defn}

\begin{defn}[Conjugado]
Dado un número complejo $z=a+bi$ su conjugado es el que se obtiene cambiando de signo la parte imaginaria:
\[\bar{z}=a-bi\]

Evidentemente, si tomamos un número complejo sin parte imaginaria (un número real) tendremos que el número y su conjugado coinciden.
\end{defn}

\begin{defn}[Módulo]
Sea $z=a+bi$ un número complejo, su módulo es la raíz cuadrada de la suma de los cuadrados de la parte real y la parte imaginaria.
\[|z|=\sqrt{a^2+b^2}\]

Podemos comprobar fácilmente que
\[|z|^2=z \cdot \overline{z}\]
y que si el número $z$ es distinto de 0, multiplicando por $z^{-1}$ a ambos lados de la igualdad anterior y despejando, tenemos que
\[z^{-1}=\frac{\overline{z}}{|z|^2}\]
\end{defn}

Una vez vistas estas definiciones, veamos una serie de propiedades que resultarán muy útiles a la hora de trabajar con complejos:
\begin{enumerate}
\item $Re(z)=\frac{z+\bar{z}}{2}$
\item $Im(z)=\frac{z-\bar{z}}{2i}$
\item $\overline{z+w}=\bar{z}+\bar{w}$
\item $\overline{zw}=\bar{z}\bar{w}$
\item $\overline{\frac{z}{w}}=\frac{\bar{z}}{\bar{w}}$
\item $|z|=|\bar{z}|$
\end{enumerate}

Prácticamente cualquier otra relación entre las partes de los números complejos puede deducirse a partir de estas propiedades.

\section{Representación polar}
Podemos considerar los números complejos como vectores en $\real^{2}$ siendo el eje $0X$ la parte real y el eje $0Y$ la parte imaginaria.

Una vez aquí, podemos definir una nueva notación para los números complejos: la \textbf{representación polar}

\begin{defn}[Representación polar]
Consiste en definir los números complejos como vectores en el plano. Para ello sólo es necesario dar su módulo y el ángulo (\textbf{argumento}) que forman respecto a la horizontal.
\end{defn}

Una vez definida esta notación, tenemos otra serie de propiedades de gran utilidad:
\begin{enumerate}
\item $\arg{z\cdot w} = \arg{z} + \arg{w}$
\item $\arg{\frac{z}{w}} = \arg{z}-\arg{w}$
\end{enumerate}

Un mismo único número complejo tiene infinitos argumentos válidos con la siguiente relación: Si $α_0 \in \arg{z} \implies \arg{z}=\{α_0+2πk \ k \in \ent\}$

En un intervalo en $\real$ de tipo $(a,a+2π]$ sólo puede haber un argumento de $z$.

Para $z\in \cplex \setminus \{x\leq 0\}$ fijamos el intervalo (-π,π) y llamamos argumento principal de z al argumento de z que esté en (-π,π). Lo denotaremos \textbf{Arg(z)}

\begin{example}
Para el entero $z=1-i$ tenemos que $|z|=\sqrt{2}$, $Arg(z)=-\frac{π}{4}$

Por tanto, podríamos escribir este número como:
\[z=\sqrt{2}\left(\cos(\frac{-π}{4}+2πk)+i\sen(\frac{-π}{4}+2πk)\right)\]
\end{example}

Esta nueva representación de los números complejos resulta muy útil pues se comporta bien respecto a la multiplicación. Es decir, con esta representación, podemos multiplicar dos complejos basándonos en las operaciones con las ecuaciones trigonométricas habituales.

\section{Desigualdad triangular}
El proceso de sumar el número complejo $z_1 = x_1 + iy_1$ al número complejo $z_2 = x_2 + iy_2$ tiene una interpretación simple en términos vectoriales.

El vector que representa la suma de los números complejos $z_1$ y $z_2$ se obtiene sumando vectorialmente los vectores de $z_1$ y $z_2$ ; es decir, empleando la regla del paralelogramo. La desigualdad triangular se puede obtener a partir de este esquema
geométrico. La longitud de un lado cualquiera de un triángulo es menor o igual que la suma de las longitudes de los otros lados. La longitud correspondiente a $z_1 + z_2$ es $|z_1 + z_2 |$, que debe ser menor o igual que la suma de las longitudes, $|z_1 | + |z_2 |$.

\begin{prop}[Desigualdad Triangular]
Si $z_1$ y $z_2$ son números complejos, entonces
\[|z_1+z_2|\leq |z_1|+|z_2|\]
\end{prop}
\begin{proof}
\begin{align*}
|z_1+z_2|^2 & = (z_1+z_2)\overline{(z_1+z_2)} \\
& = (z_1+z_2)(\bar{z_1}+\bar{z_2}) \\
& = z_1\bar{z_1}+z_1\bar{z_2}+z_2\bar{z_1}+z_2\bar{z_2} \\
& = |z_1|^2+z_1\bar{z_2}+z_2\bar{z_1}+|z_2|^2 \\
& = |z_1|^2+z_1\bar{z_2}+\overline{\bar{z_2}z_1}+|z_2|^2
\end{align*}

pero

\[z_1\bar{z_2}+\overline{\bar{z_2}z_1} = 2 Re(z_1\bar{z_2}) \leq 2|z_1\bar{z_2}| = 2 |z_1||\bar{z_2}| = \textcolor{red}{2}|z_1||z_2|\]

luego

\[|z_1+z_2|^2 \leq |z_1|^2+2|z_1||z_2|+|z_2|^2 = (|z_1|+|z_2|)^2\]

de donde se deduce que
\[|z_1+z_2| \leq |z_1|+|z_2|\]
con lo cual queda establecida la desigualdad triangular
\end{proof}

\section{Raíces y potencias}
Para calcular las potencias de un número complejo resulta muy cómodo apoyarnos en la representación polar de los mismos.

\begin{example}
Dados una serie de números complejos $z_j=r_j(\cos(α_j)+i\sin(α_j))$ tenemos que
\[\prod_{i=0}^n z_i = \prod_{i=0}^nr_i\left(\cos\left(\sum_{i=0}^n α_j\right)+i\sin\left(\sum_{i=0}^n α_j\right)\right)\]

de donde podemos deducir que
\[z_0^n = r_0^n\left(\cos(nα_0)+i\sin(nα_0)\right)\]
\end{example}

Supongamos ahora que tenemos $z_0=r_0(\cos(α_0)+i\sin(α_0))$ y queremos resolver $z^n=z_0$.

Si $z=r(\cos(α)+i\sin(α))$, como $z^n=r^n(\cos(nα)+i\sin(nα))$ tenemos que, igualando a $z_0$:
\[r=r_0^{\frac{1}{n}} \ \ \ α=\frac{α_0+2πk}{n}\]

Por tanto las n raíces n-ésimas de $z_0$ son
\[\sqrt[n]{z_0}=r_0^{\frac{1}{n}}\left(\cos\left(\frac{α_0+2πk}{n}\right)+i\sin\left(\frac{α_0+2πk}{n}\right)\right)\]

Cuando definamos $e^z$ tendremos que $e^{iα}=\cos(α)+i\sin(α)$.

Así, tendríamos que :
\[z=|z|e^{iα} \text{ con } z^n=|z|^ne^{inα} \text{ y } \sqrt[n]{z}=\sqrt[n]{|z|}e^{\frac{α+2kπ}{n}}\]
\section{Topología del plano complejo}
\textit{La profesora pareció ignorar esta parte para saltar directamente al plano complejo extendido. Así que estas líneas vienen por cortesía de Wikipedia}

En matemáticas, el plano complejo es una forma de visualizar y ordenar el conjunto de los números complejos. Puede entenderse como un plano cartesiano modificado, en el que la parte real está representada en el eje de abscisas y la parte imaginaria en el eje de ordenadas. El eje de abscisas también recibe el nombre de eje real y el eje de ordenadas el nombre de eje imaginario.

El plano complejo a veces recibe el nombre de plano de Argand a causa de su uso en diagramas de Argand. Su creación se atribuye a Jean-Robert Argand, aunque fue inicialmente descrito por el encuestador y matemático Noruego-danés Caspar Wessel.

El concepto de plano complejo permite interpretar geométricamente los números complejos. La suma de números complejos se puede relacionar con la suma con vectores, y la multiplicación de números complejos puede expresarse simplemente usando coordenadas polares, donde la magnitud del producto es el producto de las magnitudes de los términos, y el ángulo contado desde el eje real del producto es la suma de los ángulos de los términos.

Los diagramas de Argand se usan frecuentemente para mostrar las posiciones de los polos y los ceros de una función en el plano complejo.

El análisis complejo, la teoría de las funciones complejas, es una de las áreas más ricas de la matemática, que encuentra aplicación en muchas otras áreas de la matemática así como en física, electrónica y muchos otros campos

\section{Esfera de Riemann}
\begin{defn}[Plano complejo extendido]
\[\widehat{\cplex}=\cplex \cup \{\infty\}\]
Añadir el infinito a los complejos nos resultará muy ventajoso, como veremos más adelante.
\end{defn}

Lo primero que debemos hacer para trabajar en este plano complejo es definir una distancia. Para ello vamos a basarnos en la proyección esterográfica.

La idea es sencilla. Consideramos el plano complejo y construimos una esfera a la que el plano corte por la mitad. Cada punto del plano lo unimos con el polo norte de la esfera y tomamos como imagen la intersección de ese segmento con la esfera.

El dibujo de la proyección es:

\input{tikz/ProyeccionEstereografica.tex}

Así, todos los puntos del plano están asociados de manera única a un punto de la esfera y viceversa. El único punto sin imagen sería el polo norte al cual le asociamos el infinito.

Por tanto, podemos identificar el plano complejo extendido con una esfera (\textbf{la esfera de Riemann}) mediante la proyección estereográfica:
\[\appl{\psi}{\widehat{\cplex}}{S}\]

\begin{example}
Veamos algunos ejemplos de cómo se comporta esta proyección, a fin de entenderla mejor:
\[\psi(\{z \in \cplex : |z|<1\})=\{(x_1,x_2,x_3) \in S: x_3<0\}\]
\[\psi(\{z \in \cplex : |z|=1\})=\{(x_1,x_2,0)\in S\}\]
\[\psi(\{z \in \cplex : |z|>1\})=\{(x_1,x_2,x_3) \in S: x_3>0\}\]
\end{example}

Vamos a calcular explícitamente esta aplicación $\psi$.

Sea $z=x+iy$ lo identificamos con $(x,y,0)$. La recta que pasa por este punto y el polo norte será
\[\{((1-t)x, (1-t)y, t) \tq t \in \real\}\]
y su intersección con la esfera corresponde al $t$ tal que
\[(1-t)^2x^2+(1-t)^2y^2+t^2=1 \iff (1-t)^2|z|^2=1-t^2 \iff (1-t)|z|^{\textcolor{red}{2}}=1+t\]
Despejando llegamos a
\[t = \frac{|z|^2-1}{|z|^2+1}\]

Por tanto, la proyección estereográfica se define como:
\[\psi(z)=\left(\frac{2x}{1+|z|^2},\frac{2y}{1+|z|^2},\frac{|z|^2-1}{|z|^2+1}\right)\]

\obs Para $(x_1,x_2,x_3)\in S\setminus \{(0,0,0)\}$ tenemos que
\[\psi^{-1}((x_1,x_2,x_3))=\frac{x_1+ix_2}{1-x_3}\]

Por último definimos la distancia $\widehat{d}$ en $\widehat{\cplex}$ como la distancia entre las imágenes de esos puntos por $\psi$ en $\real^3$. Es decir:
\[\widehat{d}(z,w)=||\psi(w)-\psi(z)||\]

\begin{example}
Sean dos complejos $z,w$ vamos a calcular su distancia.

\[\widehat{d}(z,w)=||\psi(w)-\Psi(z)||\]
\[(\widehat{d}(z,w))^2 = \left( \frac{z+\bar{z}}{1+|z|^2}-\frac{w+\bar{w}}{1+|w|^2}\right)^2-\left( \frac{z-\bar{z}}{1+|z|^2}-\frac{w-\bar{w}}{1+|w|^2}\right)^2 + \left( \frac{|z|^2-1}{1+|z|^2}-\frac{|w|² -1}{1+|w|^2}\right)^2\]

de donde podemos deducir
\[\widehat{d}(z,w)=\frac{2|z-w|}{\sqrt{(1+|z|^2)(1+|w|^2)}}\]
\end{example}

Una vez que tenemos definida una distancia podemos hablar de límites y continuidad.

\section{Funciones complejas: límites y continuidad}
En $\cplex$ definimos $d(z,w)=|z-w|$

\begin{defn}[Convergencia]
Dada una sucesión $\{z_n\}\in \cplex$
\[\lim_{n \to \infty} z_n = z \iff \forall ε \ \exists N>0 \tq |z_n-z| < ε \ \forall n > N\]

Basándonos en la desigualdad triangular podemos ver que $z_n=a_n+b_ni$ converge a $z=a+bi$ si y sólo si $a_n$ converge a $a$ y $b_n$ converge a $b$.
\end{defn}

En definitiva esta convergencia es equivalente a la convergencia que conocemos en $\real^2$. Por tanto, no nos resultará extraña la siguiente proposición:
\begin{prop}
$(\cplex, d)$ y $(\widehat{\cplex}, \widehat{d})$ son espacios completos.
\end{prop}

\begin{defn}[Continuidad]
Sea $\Omega$ un dominio (abierto y conexo) en $\cplex$ y sea $\appl{f}{\Omega}{\cplex}$ decimos que:
\[\lim_{n\to \infty}f(z)=a \iff \forall ε > 0 \ \exists δ>0 \tq \text{ si } 0 < |z-z_0| < δ \implies |f(z)-a| < ε\]

Decimos que una función $f$ es \textbf{continua} en $z_0$ si $f(z_0)=a$
\end{defn}

En el espacio complejo conservamos las propiedades de suma, producto y cociente de límites. La prueba de estas propiedades sería idéntica a la realizada trabajando en $\real^2$ y se deja como ejercicio para el lector desconfiado.

De la misma forma, la suma, producto, cociente y composición de funciones continuas es continua. Esto puede demostrarse de manera trivial una vez probadas las propiedades indicadas en el párrafo anterior.

\begin{problem}[1]
Sean $z_0, w_0 \in \cplex$, demostrar:
\ppart \[\lim_{z\to z_0} f(z) = \infty \iff \lim_{z\to z_0}\frac{1}{f(z)}=0\]
\ppart \[\lim_{z\to \infty} f(z) = w_0 \iff \lim_{z\to 0}\frac{1}{f(\frac{1}{z})}=\frac{1}{w_0}\]
\ppart \[\lim_{z\to \infty} f(z) = \infty \iff \lim_{z\to 0}\frac{1}{f(1/z)}=0\]

\solution
\textcolor{blue}{Hecho por mi. No fiarse al 100\%}

En definitiva todos los apartados se resuelven aplicando las propiedades para el cociente de límites.

\spart
Puesto que
\[\lim_{z\to z_0}\frac{1}{f(z)}=\frac{\lim_{z\to z_0}1}{\lim_{z\to z_0}f(z)}=\frac{1}{\lim_{z\to z_0}f(z)}=0\]
tenemos que la única posibilidad es que $\lim_{z\to z_0}f(z)=\infty$

\spart
\[\lim_{z\to 0}\frac{1}{f(\frac{1}{z})}=\frac{\lim_{z\to 0}1}{\lim_{z\to 0}f(\frac{1}{z})}=\frac{1}{f(\lim_{z\to 0}\frac{1}{z})}=\frac{1}{f(\lim_{z\to \infty}z)}=\frac{1}{\lim_{z\to \infty}f(z)}=w_0\]
Despejando llegamos a
%TODO comprobar que el enunciado sea correcto
\[\lim_{z\to \infty}f(z)=\frac{1}{w_0}\]

\spart
\[\lim_{z\to 0}\frac{1}{f(1/z)}=\frac{\lim_{z\to 0}1}{\lim_{z\to 0}f(1/z)}=\frac{1}{f(\lim_{z\to 0}1/z)} = \frac{1}{f(\lim_{z\to \infty}z)} = \frac{1}{\lim_{z\to \infty}f(z)}=0\]
Una vez más, despejando llegamos a
\[\lim_{z\to \infty} f(x) = \frac{1}{0}=\infty\]
\end{problem}


\chapter{Funciones holomorfas}
\begin{defn}[Función holomorfa\IS]
Sea $\Omega \subset \cplex$ abierto y $\appl{f}{\Omega}{\cplex}$ $f$ es \textbf{holomorfa} en $z_0 \in \Omega$ si y sólo si existe
\[\lim_{z\to z_0} \frac{f(z)-f(z_0)}{z-z_0}= f'(z_0)\]

Una función será \textbf{holomorfa} si lo es en todos sus puntos.
\end{defn}

\begin{defn}[Función entera\IS]
Una función es \textbf{entera} si es holomorfa en todo $\cplex$
\end{defn}

\begin{prop}
Si una función $f$ es holomorfa en $z_0$ entonces será continua en ese mismo punto
\end{prop}
\begin{proof}
Basta con considerar
\[f(z)=\frac{f(z)-f(z_0)}{z-z_0}(z-z_0)+f(z_0)\]
sabemos que, por ser $f$ holomorfa,
\[\lim_{z \to z_0}\frac{f(z)-f(z_0)}{z-z_0} = f'(z_0)\]
por lo que
\[\lim_{z \to z_0} f(z) = f'(z_0)\cdot0+f(z_0)=f(z_0)\]
\end{proof}

\section{Derivada compleja}
\begin{itemize}
\item Si f,g son holomorfas en $z_0$ entonces su suma, producto y cociente\footnote{Siempre que el denominador sea no nulo} son holomorfas. Además se conservan las reglas de derivación para estas operaciones.

\item Si $g$ es holomorfa en $z_0$ y $f$ es holomorfa en $g(z_0) \implies$ $f(g(z))$ es holomorfa en $z_0$ y se conserva la regla de la cadena para derivar la composición de funciones

\item Los polinomios son funciones enteras y podemos calcular su derivada con normalidad

\item Veremos que si $\exists f'(z_0) \implies \exists f^{(n)}(z_0) \forall n \in \nat$
\end{itemize}

Con la última propiedad hemos obtenido algo realmente nuevo, que no se cumple en las funciones de variable real. A continuación veremos qué le estamos pidiendo a las funciones holomorfas que nos gartantiza esto y que no lo pedimos a las funciones de variable real.


\section{Ecuaciones de Cauchy-Riemann}
Sea $\appl{f}{\Omega}{\cplex}$ holomorfa en $z_0=x_0+iy_0$ y sean $u(x,y)=Re((f(x+iy)))$, $v(x,y)=Im((f(x+iy)))$, entonces podemos escribir $f(x,y)=u(x,y)+iv(x,y)$. Básicamente, hemos definido las funciones $u,v$ que nos dan la parte real e imaginaria respectivamente de la función $f$.

Tenemos que
\[f'(z_0)=\lim_{h \to 0} \frac{f(z_0+h)-f(z_0)}{h}\]
si el límite existe, en cualquier dirección en que me acerque a $z_0$ debo obtener el mismo límite\footnote{Recordad lo que hacíamos en Cálculo II aproximándonos por rectas al trabajar en $\real^2$}

Esta vez vamos a acercarnos en concreto por dos direcciones, las de los ejes.
\begin{itemize}
\item \textbf{Eje x}
\[f'(z_0)=\lim_{h \to 0} \frac{u(x_0+h,y_0)+iv(x_0+h,y_0)-u(x_0, y_0)-iv(x_0,y_0)}{h}=\]
\[= \lim_{h \to 0}\frac{u(x_0+h,y_0)-u(x_0, y_0)}{h}+i\frac{v(x_0+h,y_0)-v(x_0,y_0)}{h}=\frac{du}{dx}(x_0,y_0)+i\frac{dv}{dx}(x_0,y_0)\]
en la última igualdad sabemos que las derivadas parciales existen puesto que el límite de una función compleja puede calcularse como límites de parte real y parte imaginaria

\item \textbf{Eje y}
\[f'(z_0)=\lim_{k \to 0} \frac{u(x_0,y_0+k)+iv(x_0,y_0+k)-u(x_0, y_0)-iv(x_0,y_0)}{ik}=\]
\[=\frac{1}{i}\left(\frac{du}{dy}(x_0,y_0)+i \frac{dv}{dy}(x_0, y_0)\right)=\frac{1}{i}\frac{df}{dy}\]
\end{itemize}

Como ya indicamos, para que la función sea diferenciable, las dos derivadas que acabamos de calcular deberían coincidir. Igualando obtenemos
\[\frac{df}{dx}(x_0,y_0)=\frac{df}{dy}(x_0,y_0) = -i \frac{df}{dy} \text{ en } (x_0,y_0)\]
es decir
\[\frac{du}{dx}(x_0,y_0)+i\frac{dv}{dx}(x_0,y_0)=-i\left(\frac{du}{dy}(x_0,y_0)+\frac{dv}{dy}(x_0,y_0)\right)=\frac{dv}{dy}(x_0,y_0)-i\frac{du}{dy}(x_0,y_0)\]


Igualando obtenemos las ecuaiones de \textbf{Cauchy-Riemann}.

Por tanto si existe $f'$ existen las 4 derivadas parciales con las que hemos estado trabajando y verifican las ecuaciones de \textbf{Cauchy-Riemann}

\begin{defn}[Ecuaciones de Cauchy-Riemann]
Las \textbf{ecuaciones de Cauchy-Riemann} son dos ecuaciones diferenciales parciales que son básicas en el análisis de funciones complejas de variable compleja, debido a que su verificación constituye una condición necesaria (aunque no suficiente) para la derivabilidad de este tipo de funciones.

Estas ecuaciones son:
\[\frac{du}{dx}(x_0,y_0) = \frac{dv}{dy}(x_0,y_0)\]
\[\frac{dv}{dx}(x_0,y_0) = -\frac{du}{dy}(x_0,y_0)\]
\end{defn}

\section{Funciones armónicas}

Si $u,v\in \cplex^2(\Omega)$ y satisfacen las ecuaciones de Cauchy-Riemann(C-R), entonces, derivando a ambos lados de las ecuaciones C-R, obtenemos:
\[\frac{d^2u}{dx²}=\frac{d^2v}{dxdy}, \ \ \frac{d^2v}{dy²}=\frac{d^2u}{dydx}\]
\[\frac{d^2u}{dy²}=-\frac{d^2v}{dydx}, \ \ \frac{d^2v}{dx²}=-\frac{d^2u}{dxdy}\]

Por tanto, puesto que el orden de derivación no influye, tenemos que
\[\frac{d^2u}{dx^2}+\frac{d^2u}{dy^2} = 0 \ \ \frac{d^2v}{dx^2}+\frac{d^2v}{dy^2} = 0\]

\begin{defn}[Laplaciano]
El laplaciano u operador de Laplace se aplica sobre funciones y consiste en sumar las segundas derivadas respecto de cada variable. Es decir:
\[\vartriangle (f(x,y)) = \frac{d^2f}{dx^2}+\frac{d^2f}{dy^2}\]
siendo $f(x,y)=\underbrace{Re(x,y)}_{u(x,y)}+\underbrace{Im(x,y)}_{v(x,y)}i$
\end{defn}

\begin{defn}[Funciones armónicas]
Se dice que una función es \textbf{armónica} si su laplaciano es 0.
\end{defn}

\begin{prop}
Si $u,v \in \cplex^2(\Omega)$ y verifican las ecuaciones de Cauchy-Riemann entonces su parte real y su parte imaginaria son armónicas.
\end{prop}

Como veremos que la derivada de una función holomorfa es holomofa, entonces $u,v$ tendrán derivadas parciales de todos los órdenes.

Por tanto, si $f=u+iv$ es holomorfa en $\Omega$ entonces $u,v$ son \textbf{armónicas} en $\Omega$

\obs Si $\Omega$ es simplemente conexo y $u$ es arḿónica en $\Omega$, entonces existirá una función $f$ holomorfa en $\Omega$ tal que $f=u+iv$

\begin{prop}
Supongamos que tenemos dos funciones $u,v$ diferenciables en $(x_0,y_0)$ (por ejemplo, existen las derivadas parciales en ese punto y son continuas en un entorno del mismo)\footnote{Recordamos que si era diferencialbe teníamos esta propiedad pero no al revés.} y que verifican las ecuaciones de Cauchy-Riemann, entonces $f=u+iv$ es holomorfa en $z_0=x_0+iy_0$
\end{prop}
\begin{proof}
Tenemos las dos funciones $\appl{u,v}{\real^2}{\real}$ diferenciables. Por ser diferenciables en $(x_0,y_0)$ sabemos que
\[u(x_0+α,y_0+β)-u(x_0,y_0)=\frac{du}{dx}(x_0,y_0)α+\frac{du}{dy}(x_0,y_0)β + o(||(α,β)||)\]
\[v(x_0+α,y_0+β)-v(x_0,y_0)=\frac{dv}{dx}(x_0,y_0)α+\frac{dv}{dy}(x_0,y_0)β + o(||(α,β)||)\]

Para que $f$ fuese holomorfa necesitamos que exista la derivada. Vamos a verlo:
\[\underbrace{f(x_0+iv_0+α+βi)}_{f(z_0+μ)}-\underbrace{f(x_0+iy_0)}_{f(z_0)} = u(x_0+α,y_0+β)-u(x_0,y_0) + i \left( v(x_0+α,y_0+β)-v(x_0,y_0)\right) = \]
\[= \frac{du}{dx}(x_0,y_0)α+\frac{du}{dy}(x_0,y_0)β + o(||(α,β)||)+i\left( \frac{dv}{dx}(x_0,y_0)α+\frac{dv}{dy}(x_0,y_0)β + o(||(α,β)||)\right)\]

Pero, puesto que tanto $u$ como $v$ verifican las ecuaciones de Cauchy-Riemann, podemos escribir:
\[\frac{du}{dy}(x_0,y_0)=-\frac{dv}{dx}(x_0,y_0), \ \ \frac{dv}{dy}(x_0,y_0)=\frac{du}{dx}(x_0,y_0) \]
y aplicando estos cambios en la igualdad anterior tendríamos:
\[\frac{du}{dx}(x_0,y_0)α-\frac{dv}{dx}(x_0,y_0)β + o(||(α,β)||)+i\left( \frac{dv}{dx}(x_0,y_0)α+\frac{du}{dx}(x_0,y_0)β + o(||(α,β)||)\right)=\]
\[=\left(\frac{du}{dx}(x_0,y_0)+i\frac{dv}{dx}(x_0,y_0)\right)(α+iβ)+o(|α+iβ|)\]
Para garantizar que $f$ es holomorfa deberíamos ver que:
\[\lim_{h\to 0} \frac{f(z_0+h)-f(z_0)}{h} = \lim_{α,β->0} \frac{\left(\frac{du}{dx}(x_0,y_0)+i\frac{dv}{dx}(x_0,y_0)\right)(α+iβ)+o(|α+iβ|)}{α+βi} = \]
\[= \left(\frac{du}{dx}(x_0,y_0)+i\frac{dv}{dx}(x_0,y_0)\right)\]

Así, queda claro que la derivada de $f$ existe por lo que $f$ es holomorfa.
\end{proof}

Veamos ahora algo de notación.

\begin{defn}[Notación: $\partial$]
Sea una función $\appl{f}{\cplex}{\cplex}$ definimos
\[\partial f = \partial_z f = \frac{1}{2}(d_x f - i d_y f)\]
\end{defn}

\begin{defn}[Notación: $\bar{\partial}$]
Sea una función $\appl{f}{\cplex}{\cplex}$ definimos
\[\bar{\partial} f = \partial_{\bar{z}} f = \frac{1}{2}(d_x f + i d_y f)\]
\end{defn}

Formalmente, si consideramos a $f$ como una función de $z$ y $\bar{z}$ tenemos que
\[x = \frac{z+\bar{z}}{2}, \ \ y = \frac{z-\bar{z}}{2i}\]
de donde, derivando, obtenemos que
\[\frac{dx}{dz}=\frac{1}{2}, \ \frac{dx}{d\bar{z}}=\frac{1}{2}, \ \frac{dy}{dz}=\frac{1}{2i}, \ \frac{dy}{d\bar{z}}=\frac{1}{2i},\]
Ahora podemos aplicar la regla de la cadena y calcular
\[\partial_z f = \partial_x f \frac{1}{2}+\partial_u f \frac{1}{2i}\]
\[\partial_{\bar{z}} f = \partial_x f \frac{1}{2}+\partial_u f \left(-\frac{1}{2i}\right)\]

\begin{prop}
\[f\text{ es holomorfa } \iff \bar{\partial}f=0\]
\end{prop}
\begin{proof}
Si la función $f$ es holomorfa, sabemos que verifica las ecuaciones de Cauchy-Riemann. Así
\[\bar{\partial}_z f = \frac{1}{z}(d_x f + i d_y f) = \frac{1}{z} \left(\frac{df}{dx}+i\frac{df}{dy} \right) = \frac{1}{z} \left( \frac{du}{dx}+i\frac{dv}{dx} + i \frac{du}{dy} - \frac{dv}{dy}\right) \underbrace{=}_{C-R}\]
\[=\frac{1}{z} \left( \frac{dv}{dy}-i\frac{du}{dy} + i \frac{du}{dy} - \frac{dv}{dy}\right) = 0\]
\end{proof}

\begin{prop}
\[f(z)-f(z_0)=\partial_xf(z_0)\left(\frac{z+\bar{z}-z_0-\bar{z}_0}{2}\right)-i\partial_y f (z_0)\left( \frac{z-\bar{z}-z_0+\bar{z}_0}{2}\right) + o (|z-z_0|)\]
y aplicando las definiciones (notación) dadas anteriormente, tenemos
\[f(z)-f(z_0)= \partial f(z_0)(z-\bar{z}_0)+\bar{\partial}f(z_0)(\bar{z}-\bar{z}_0)+o|z-z_0|\]

Es decir, podemos escribir $f$ en función de $z$ su derivada en función de las definiciones $\partial$ y $\bar{\partial}$.
\end{prop}

\begin{example}
\begin{enumerate}
\item
\[f(z)=\frac{z^2}{(z^2+1)^2}\]
Esta función es holomorfa en $\cplex\setminus\{\pm i\}$. Para verlo nos basamos en que es el cociente de dos funciones holomorfas por lo que $f(z)$ será holomorfa en todo $\cplex$ salvo los puntos en que se anula el denominador.

Si calculamos la derivada obtenemos
\[f'(z)=\frac{2z(z^2+1)^2-z^22(z^2+1)2z}{(z^2+1)^4}\]
y vemos que, efectivamente, existe para todo $z \in \cplex\setminus\{\pm i\}$

\item
\[f(z)=z\cdot Re(z)\]
Vamos a reescribir esta función de la forma:
\[f(z)=(x+iy)x = x^2 + ixy\]
Si escribimos $f(x)=u(x,y) + i v(x,y)$ tenemos que $ u(x,y)=x^2$ y $v(x,y)=xy$. Vamos a ver si satisfacen las ecuaciones de Cauchy-Riemann.

\begin{align*}
\partial_x f &= -i \partial_yf \\
u_x+iv_x &= -i (u_y+iv_y) \\
\end{align*}

Para que se cumplan las ecuaciones de Cauchy-Riemann necesitamos que $u_x=v_y$ y que $u_y=v_x$, ecuaciones que se satisfacen únicamente en $(0,0)$, punto en el que son diferenciables tanto $u$ como $v$.

\obs Podríamos haber comprobado que la función es holomorfa mediante la definición o escribiendo:
\[f(z)=z\left( \frac{z+\bar{z}}{2}\right)=\frac{z^2}{2}+\frac{z}{2}\bar{z}\]
Tenemos pues que $\bar{\partial}f=\frac{z}{2}$ que sólo se anulará en $z=0$.

\item
\[f(z) = \left\{ \begin{array}{lcc}
   \frac{z^5}{|z|^4} & si & z \neq 0 \\
   \\ 0 & si & x = 0 \end{array} \right.\]

En primer lugar vemos que la función es contínua en $z=0$ pues
\[\lim_{z \to 0} \frac{z^5}{|z|^4} = 0 = f(0)\]
Sabemos que el límite es 0 pues aplicando la definición directamente obtenemos que
\[\forall ε > 0 \exists δ > 0 \tq \text{ si } 0<|z|<δ \implies |\frac{z^5}{|z|^4}| = |z| < ε\].

Veamos ahora si es una función holomorfa. Para ello tenemos que calcular el límite
\[\lim_{z \to 0} \frac{f(z)}{z}\]
En este caso, vamos a ver que no existe y para ello vamos a acercarnos por dos rectas diferentes: el eje real y la recta $z=|z|e^{iα}=re^{iα}$.

Obtenemos los siguientes resultados
\[\lim_{x \to 0} \frac{x^5}{|x|^4} = \lim_{x \to 0} 1 = 1\]
\[\lim_{r \to 0} \frac{r^4 e^{i4α}}{r^4} = e^{i4α} \neq 1\]

Por tanto, el límite no existe y la función \textbf{no es holomorfa}. Sin embargo, las ecuaciones de Cauchy-Riemann si se satisfacen para esta función, con lo que queda claro que las ecuaciones de Cauchy-Riemann no son condición suficiente para que la función sea holomorfa.
\end{enumerate}
\end{example}

\section{Teorema de la función inversa}
\begin{theorem}[Teorema de la función inversa]
Sea $\appl{f}{\Omega\subset \cplex}{\cplex}$ holomorfa siendo $\Omega$ un abierto con $z_0\in \Omega$.

Existe un entorno $U_{z_0}\subset \Omega$ de $z_0$ tal que $f|_{U_{z_0}}$ es un isomorfismo holomorfo (biyectiva, holmorfa y con inversa holomorfa) si y sólo si $f'(z_0)=0$
\end{theorem}
\begin{proof}
Vamos a demostrarlo apoyándonos en el ya conocido teorema de la función inversa para $\real^2$.

Sea $f(x,y)=u(x,y)+iv(x,y)$ definimos la función
\[\appl{g}{\real^2}{\real^2}\]
que nos lleva el punto $(x,y)$ a $\left( u(x,y), v(x,y)\right)$

Esta función tendrá inversa en el punto $(x_0,y_0)$ si su Jacobiano es distinto de 0 en ese punto. Vamos a calcularlo
\[ J_g(x_0,y_0) = \left| \begin{array}{cc}
u_x(x_0,y_0) & u_y(x_0,y_0)\\
v_x(x_0,y_0) & v_y(x_0,y_0) \end{array} \right| = u_x(x_0,y_0)v_y(x_0,y_0)-u_y(x_0,y_0)v_x(x_0,y_0) \underbrace{=}_{C-R}\]
\[\underbrace{=}_{C-R} \left(u_x(x_0,y_0)\right)^2 +\left(v_x(x_0,y_0)\right)^2 = \left(u_y(x_0,y_0)\right)^2 +\left(v_y(x_0,y_0)\right)^2 = |f'(z_0)|^2\]

Puesto que $f$ es un isomorfismo holomorfo, sabemos que es un isomorfismo diferneciable pues veremos un resultado que garantiza que siendo $f$ holomorfa, sabemos que $f'$ también lo es y por tanto $u,v \in \cplex^{\infty}$.

Por tanto ya tenemos que, siendo $f$ una función holomorfa que tiene función inversa, es necesario que $f'(z_0)\neq 0$.

Para probar la implicación contraria, si $f'(z_0) \neq 0$ tenemos, por el teorema de la función inversa en los reales, que existe $g^{-1}(h,k)$, la función inversa de $g$ en un entorno de $(x_0,y_0)$. Simplemente nos quedaría ver que $h+ik$ es holomorfa.

Sea $f(z)=α+βi$ la matriz Jacobiana en $(α,β)$ es
\[\left( \begin{array}{cc}
h_x(x_0,y_0) & h_y(x_0,y_0)\\
k_x(x_0,y_0) & k_y(x_0,y_0) \end{array} \right) = \left( \begin{array}{cc}
u_x(x_0,y_0) & u_y(x_0,y_0)\\
v_x(x_0,y_0) & v_y(x_0,y_0) \end{array} \right)^{-1}=\]
\[=\frac{1}{|f'(z)|^2}\left| \begin{array}{cc}
v_y(x_0,y_0) & -u_y(x_0,y_0)\\
-v_x(x_0,y_0) & v_x(x_0,y_0) \end{array} \right| \underbrace{=}_{C-R} \frac{1}{|f'(z)|^2}\left| \begin{array}{cc}
u_x(x_0,y_0) & -u_y(x_0,y_0)\\
-u_x(x_0,y_0) & u_y(x_0,y_0) \end{array} \right| \]
De ahí podemos extraer que $h(α,β)$ y $k(α,β)$ verifican las ecuaciones de Cauchy-Riemann.

Además,
\[g^{-1}(f(z))=h_x(f(z))+ik_x(f(z)) = \frac{1}{|f'(z)|^2}\left( u_x(x,y)-iv_x(x,y)\right) = \frac{f'(z)}{|f'(z)|^2} = \frac{1}{f'(z)}\]

\end{proof}
\section{La función logaritmo}

\section{Series de potencias}
Sea $\{z_n=x_n+iy_n\}$ una sucesión en $\cplex$,
\[\lim_{n \to \infty}z_n=z \iff \forall ε > 0 \ \exists N \tq |z_n-z| \leq ε \ \forall n \geq N\]

Además, como
\[|x_n-x|<|z_n-z|<|x_n-x|+|y_n-y|\]
la sucesión $z_n$ convergerá sii lo hacen las sucesiones $x_n$ e $y_n$.

Siempre que exista ese número complejo, $z$, diremos que la sucesión \concept{converge}. Si no existe este límite, diremos que la sucesión \concept{diverge}. En particular, si $\lim_{n \to \infty}|z_n|=\infty$ diremos que \concept{diverge a $\infty$}.

Esta divergencia a infinito tiene sentido cuando trabajamos con el plano complejo extendido (recordemos que era el plano complejo al que habíamos añadido el infinito). Para que la \textbf{divergencia a infinito} tenga auténtico sentido debemos observar la sucesión en la esfera de Riemann donde veríamos que esta ``va hacia el polo norte''

\begin{defn}[Serie]
En matemáticas, una \textbf{serie} es la generalización de la noción de suma a los términos de una sucesión infinita. Informalmente, es el resultado de sumar los términos: $a_1 + a_2 +a_3 + a_4 + a_5 + a_6 \dots $ lo cual suele escribirse en forma más compacta como $\sum_{1\le n} a_n$.

El estudio de las series consiste en la evaluación de la suma de un número finito n de términos sucesivos y, mediante un pasaje al límite, identificar el comportamiento de la serie a medida que n crece indefinidamente.
\end{defn}

Una serie $\sum z_n$ converge a $w$ si y sólo si $\lim_{n \to \infty}\sum_{k=1}^n z_k = w$. En este caso escribimos
\[\sum_{k=1}^{\infty}z_k = w\]

Veamos algunas propiedades acerca de las series
\begin{itemize}
\item Si $\sum z_n$ converge, entonces
\[\lim_{n \to \infty} z_n = 0\]

Esto es sencillo de entender puesto que si una suma infinita converge a un número necesitamos que los sumandos convergan a 0. De lo contrario siempre estaríamos sumando algo no despreciable y la suma infinita tendería a infinito.

\item Sea $s_n=\sum_{k=1}^{n} z_n$,

\begin{defn}[Sucesión de Cauchy]
\[\{s_n\} \text{ es de Cauchy} \iff \forall δ > \ \exists N \tq \text{ si } n,m \geq N \implies |s_n-s_m| < ε\]
\end{defn}

\item Si $\sum | z_n| $ converge entonces $\sum z_n$ también converge.

\begin{proof}
Sean $s_n = \sum_{k=1}^n z_k$ y $s_n'=\sum_{k=1}^n |z_k|$ vamos a probar que $s_n$ converge. Para ver que converge vamos a probar que es de Cauchy.

\[|s_m - s_n| = \left|\sum_{k=m+1}^m z_k\right| \leq \sum_{k=n+1} ^m | z_k| = |s_m'-s_n'| < ε\]

Puesto que sabemos que $s_n'$ es de Cauchy queda claro que la sucesión $s_n$ también lo es, pues la distancia entre dos elementos de esta segunda sucesión es siempre menor que la diferencia entre dos elementos de la primera sucesión, que es de Cauchy.

\end{proof}

\end{itemize}

\begin{defn}[Convergencia absoluta]
Si la serie $\sum |z_n|$ converge diremos que $\sum z_n$ converge absolutamente.
\end{defn}

\begin{defn}[Convergencia puntual]
Sea $\{\appl{f_n}{\Omega\subset \cplex}{\cplex}\}$ una sucesión de funciones, decimos que $\{f_n\}$ converge puntualmente a $\appl{f}{\Omega\subset\cplex}{\cplex}$ sii converge punto a punto. Es decir
\[f_n(z) \rightarrow f(z) \iff \lim_{n\to\infty}f_n(z)=f(z) \forall z \in \Omega \iff\]
\[\iff \forall ε > 0 \forall z \in \Omega \exists N=N(ε,z) \tq \forall n \geq N |f_n(z)-f(z)| < ε \]
\end{defn}

\begin{defn}[Convergencia uniforme]
Sea $\{\appl{f_n}{\Omega\subset \cplex}{\cplex}\}$ una sucesión de funciones, decimos que $\{f_n\}$ converge uniformemente a $\appl{f}{\Omega\subset\cplex}{\cplex}$ si ``tenemos convergencia puntual con un único N(z) para todo z''. Es decir
\[f_n \rightrightarrows f \iff \forall ε >0 \ \exists N=N(ε) \tq \forall n \geq N \ |f_n(z)-f(z)| < ε \forall z \in \Omega\]
\end{defn}

\begin{prop}[Criterio de M de Weierstrass]
Sea $\appl{f_n}{\Omega}{\cplex}$ funciones tales que $|f_n(z)| \leq M_n$ para todo $z \in \Omega$, si $\sum M_n$ converge, entonces $\sum f_n$ converge \textbf{absoluta} y \textbf{uniformemente}.
\end{prop}

\begin{proof}
Primero debemos determinar quién es ese límite antes de poder llevar a cabo la demostración.

Sea $g_n(z)=\sum_{k=1}^{n}f_k(z)$ podemos ver que es una sucesión de Cauchy, pues
\[|g_m(z)-g_n(z) | = \left|\sum_{k=n+1}^m f_k(z)\right| \leq \sum_{k=n+1}^m | f_k(z)| \leq \sum_{k=n+1}^m M_k = |s_m-s_n|\footnote{siendo $s_n = \sum_{i=1}^n M_i$} < ε\]

La última desigualdad la obtenemos del hecho de que $M_n$ es convergente y por tanto es de Cauchy.

Por tanto, tenemos que la sucesión $g_n$ también es de Cauchy por lo que converge. Así, podemos definir límite como $g(z)=\lim_{n\to \infty}g_n(z)$ pero, si nos fijamos, tenemos que
\[g(z)=\lim_{n\to \infty}g_n(z) = \lim_{n\to \infty}\sum_{k=1}^n f_k(z) = \sum_{k=1}^{\infty} f_k(z)\]
por lo que, efectivamente, hemos probado que la suma de las $f_n$ converge.

Ahora nos queda probar que la convergencia es uniforme pues, con los pasos realizados hasta ahora, sólo podemos garantizar la convergencia puntual.

Vamos a ello
\[|g_n(z)-g(z)| = \lim_{k \to \infty}|g_n(z)-g_{n+k}(z)|\]
Dado ε > 0 $\exists N$ t.q.  $\forall n \geq N$, $|g_n(z)-g_{n+k}(z)| < ε \ \forall z \in \Omega$.

Así, queda probada la convergencia uniforme.
%TODO completar por que el final de la demostración no dice nada
\end{proof}

\begin{defn}[Serie centrada]
Una series de potencias \textbf{centrada en $z_0 \in \cplex$} es una serie de la forma
\[\sum_{n=0}^{\infty} a_n(z-z_0)^n\]
\end{defn}

\begin{example}
Dada la serie geométrica $\sum_{n=0}^{\infty}z_n^n$ podemos ver que
\[(1-z)(1+z+z^2+...z^n)=1-z^{n+1}\]
para $z \neq 1$ de donde podemos deducir que
\[1+z+z^2+...+z^n = \frac{1-z^{n+1}}{1-z}\]
(en los ejercicios podemos encontran una demostración por inducción de esta fórmula)

Vamos a analizar ahora el resultado obtenido viendo el límite de estas series.
\begin{itemize}
\item \textbf{Si |z| < 1}
\[\lim_{n \to \infty} z^n = 0\]
\[\sum z^n =\lim_{n \to \infty}\sum_{i=0}^{\infty} \frac{1}{1-z}\]
\item \textbf{Si |z| > 1}
\[\lim_{n \to \infty} z^n = \infty\]
La serie diverge, luego su suma será infinita.
\end{itemize}
\end{example}

\begin{prop}
Consideremos la serie de potencias
\[\sum_{n=0}^{\infty} a_n (z-z_0)^n\]
y supongamos que $\exists ε \in \cplex$ tal que
\[|a_n||ε|^n \leq M \ \forall n\]
con $M$ constante.

Entonces, para todo $\rho \tq 0 < \rho < |ε|$ la serie converge absoluta y uniformemente en $\{z \tq |z-z_0| < \rho\}$
\end{prop}
\begin{proof}
Si $|z-z_0 | \leq \rho < |ε|$ tenemos que
\[|a_n(z-z_n)^n| = |a_n||z-z_0|^n=|a_n||ε|^n\frac{|z-z_0|^n}{|ε|^n}\leq M \left( \frac{\rho}{|ε|}\right)^n\]
Como $\sum \left( \frac{\rho}{|ε|}\right)$ converge obtenemos el resultado buscado por el criterio M de Weierstrass
\end{proof}

Básicamente ese ε nos va a dar el tamaño de la bola centrada en $z_0$ que nos permite garantizar la convergencia y, obviamente, estamos interesados en localizar el mayor ε. Siendo α$=(z-z_0)$, tenemos dos posibilidades:
\begin{enumerate}
\item \[\forall α > 0 \ \sup_n|a_n|α^n < \infty\]
En este caso tenemos convergencia en el discto $|z-z_0|<α \ \forall α$.

Se trata de un caso \textbf{ideal} en el que la serie converge en todo $\cplex$

\item \[\exists α > 0 \tq \sup_n|a_n|α^n = \infty\]
En este caso nos gustaría encontrar el mayor α posible sin que el supremo indicado se haga infinito.

\obs $\phi(α)=\sum_n|a_n|α^n$ es una función creciente. Por tanto, dada una sucesión $α_1<α_2,...$ sabemos que si $\phi(α_2) < \infty \implies \phi(α_1) < \infty$ y que si $\phi(α_2)=\infty \implies \phi(α_3) = \infty$.

Por tanto, queremos buscar justo el punto en que esta función pasa de ser finita a ser infinita.

\begin{defn}[Radio de convergencia]
El \textbf{radio de convergencia de la serie} se define como
\[R=\inf \{α > 0\tq \sup_n|a_n|α^n = \infty\}\]
de forma totalmente equivalente podemos definir este radio de la siguiente forma
\[R=\sup \{α > 0\tq \inf_n|a_n|α^n < \infty\}\]
\end{defn}

Evidentemente, por la propia definición del radio de convergencia, la serie converge en el interior de la bola $|z-z_0|<R$ y diverge fuera de la misma.

La frontera ($|z-z_0|=R$) es un caso extremo que habría que analizar detenidamente en cada caso particular.
\end{enumerate}

Hasta ahora hemos dejado claro la importancia de este radio de convergencia y la teoría acerca de cómo calcularlo, pero no es viable calcular un ínfimo sobre todos los posibles complejos que cumplen una cierta condición.

Necesitamos una fórmula para calcular este radio y el siguiente teorema nos la da:

\begin{theorem}
Sea $R$ el radio de convergencia de la serie de potencias $\sum_{n=0}^{\infty} a_n (z-z_0)^n$ se tiene que
\begin{enumerate}
\item \[R = \frac{1}{\limsup_{n \to \infty} |a_n|^{1/n}}\]
\item
\[\text{ Si } a_n \neq 0 \forall n \text{ y }\exists \lim_{n \to \infty} \left|\frac{a_n}{a_{n+1}}\right| \text{ (pudiendo ser infinito) } \implies R = \lim_{n \to \infty}\frac{|a_n|}{|a_{n+1}|}\]
\item
La serie converge uniformemente en $\{z \tq |z-z_0| < r\}$ con $r \leq R$ y converge absolutamente en $\{z \tq |z-z_0|<R\}$
\end{enumerate}
\end{theorem}
\begin{proof}

\begin{enumerate}
\item Lo primero que debemos haceer es comprobar que esta fórmula para $R$ coincide con la definición del mismo dada anteriormente.

Supongamos que $0<R<\infty$ (calculando $R$ según su definición). Entonces
\[\limsup_{n \to \infty} |a_n|^{1/n} \leq \frac{1}{R}\]

Supongamos ahora que existe un $0<r<R$ tal que $\sup_n |a_n|<\infty$. En este caso existiría un $M>0$ t.q. $|a_n|r^n < M \ \forall n$.

Así, tendríamos que $|a_n|^{1/n} < \frac{M^{1/n}}{r}$ y en el infinito veríamos que
\[\lim_{n \to \infty} |a_n|^{1/n} \leq \frac{1}{r} \ \forall 0<r < R \implies \limsup_{n \to \infty}|a_n|^{1/n} \leq \frac{1}{R}\]

Si conseguimos probar ahora la desigualdad contraria habremos logrado demostrar la igualdad, es decir, habremos demostrado que la fórmula dada para el cálculo de $R$ encaja con la definición del mismo.

Vamos a probar esta desigualdad por reducción al absurdo.
\[\limsup_{n \to \infty}|a_n|^{1/n} < \frac{1}{R} \implies \exists N > 0 \tq \forall n \geq N \ \sup_{k \geq N} |a_n|^{1/n} < \frac{1}{r} < \frac{1}{R} \implies\]
\[\implies \forall k \geq N \ |a_k|^{1/k} < \frac{1}{r} \implies |a_n| r^n < M \forall n \implies \sup_{n\to\infty}|a_n|r^n< \infty\]
pero teníamos que $R > r$ lo que nos lleva a contradicción, pues la R marca el límite a partir del cual la serie deja de ser convergente.
\item
Sea β$= \lim_{n \to \infty}\left|\frac{a_n}{a_{n+1}} \right|$

Si vemos que la serie converge en $\{z: |z-z_0| < r^-\} \ \forall r^-<β$, tendremos que β $\leq R$.

Como $r^- < β$ existe un $N$ t.q. $r^-<\left|\frac{a_n}{a_{n+1}} \right| \forall n \geq N$, es decir:
\[a_{n+1} < \frac{|a_n|}{r^-}\]
desplazándonos un término a la izquierda en la sucesión ($n-1$) y multiplicando a ambos lados por $(r^-)^n$ obtenemos
\[|a_n|(r^-)^n < \frac{|a_{n-1}|(r^-)^n}{r^-} = |a_{n-1}|(r^-)^{n-1}<\frac{|a_{n-2}|(r^-)^{n-1}}{r^-}=|a_{n-2}|(r^-)^{n-2} \leq ... \leq M\]

Así, podemos escribir
\[\sum_n |a_n(z-z_0)^n|=\sum_n |a_n|(r^-)^n\left(\frac{|z-z_0|}{r^-}\right)^n \leq M \sum \left( \frac{|z-z_0|}{r^-}\right)^n\]
que es convergente para $|z-z_0|<r^-$

Por tanto, tenemos que $β \leq R$. Sólo nos queda demostrar la desigualdad contraria para poder concluir la igualdad.

Si vemos que la serie divertge en $|z-z_0|>r^+ \ \forall r^+ > β$, tendremos que $r^+\geq R$. Vamos a ello

Como $r^+ > β$ existe un $N$ t.q. $r^+>\left|\frac{a_n}{a_{n+1}} \right| \forall n \geq N$, es decir:
\[|a_{n+1}| r^+ \geq |a_n|\]
y reptiendo las cuentas del apartado anterior podemos llegar a
\[|a_n|(r^+)^n \geq M\]

Así, podemos escribir
\[\sum_n |a_n(z-z_0)^n|=\sum_n |a_n|(r^+)^n\left(\frac{|z-z_0|}{r^+}\right)^n \geq M \sum \left( \frac{|z-z_0|}{r^+}\right)^n\]
que es divergente para $|z-z_0|>r^+$.

Por tanto, obtenemos claramente que $β \geq R$ por lo que, finalmente, podemos concluir
\[β=R=\lim_{n \to \infty}\frac{|a_n|}{|a_{n+1}|}\]

\item La demostración es idéntica a una realizada anteriormente y se deja como ejercicio para el lector.
\end{enumerate}
\end{proof}


Veamos algunas muestras de la relación de este resultado con nuestros conocimientos sobre los reales. Puesto que los reales son parte de los complejos, deberán cumplirse las condiciones de convergencia sobre ellos
\begin{itemize}
\item
Sea $\sum a_n$ una serie compleja y sea $c=\limsup_{n \to \infty} |a_n|^{1/n}$; si $c<1$ la serie converge y si $c > 1$ diverge.

Esta $c$ aparecía al definir el radio de convergencia como el inverso del mismo por lo que esta resultado puede escribirse como; $R>1$ implica que la serie converge y si $R<1$ diverge.

\item
Recordemos de los reales que si $\exists \lim_{n \to \infty} \frac{a_{n+1}}{|a_n|} = D \implies$ si $D > 1$ la serie diverge y si es menor que 1, converge.

Nuevamente esto puede probarse con el teorema anterior considerando que $D=\frac{1}{R}$. Por tanto, $D>1 \implies R < 1\implies$ divergencia y viceversa.

\end{itemize}

Veamos algunos ejemplos del estudio de convergencia de series
\begin{example}
\begin{enumerate}
\item
Tomemos la serie
\[\sum_{n=1}^{\infty} \frac{(2n)!}{(n)!}z^n\]
en este caso
\[R = \lim_{n \to \infty} \frac{|a_n|}{|a_{n+1}|} = \lim_{n \to \infty} \frac{(2n)!((n+1)!)^2}{(n!)^2(2(n+1))!} = \lim_{n \to \infty} \frac{(n+1)^2}{(2n+2)(2n+1)} = \frac{1}{4}\]

\item
Tomemos la serie
\[\sum_{n=0}^{\infty} z^{n!} = \sum_{k=0}^{\infty}a_k z^k \text{ con } a_k = 1 \iff k=n!\]

En este caso
\[\limsup|a_k|^{1/k} = 1 \implies R = 1\]

\end{enumerate}
\end{example}

Vamos a estudiar ahora cómo se relacionan los radios de convergencia para la suma y el producto mediante un ejemplo.
\begin{example}[Comportamiento de los radios de convergencia frente a la suma y el producto]
Sea $\sum a_nz^n$ con radio de convergencia $R_1$ y $\sum b_n z^n$ con radio de convergencia $R_2$ tenemos:
\begin{enumerate}
\item
\[\sum_{n=0}^{\infty}(a_n\pm b_n)z^n \implies R = \min\{R_1, R_2\}\]
esto se debe a que podemos escribir la suma como
\[\sum_{n=0}^{\infty}(a_n\pm b_n)z^n = \lim_{n \to \infty} \sum_{i=0}^n (a_i\pm b_i)z^i\]
y ese límite podrá descomponerse en suma de límites cuando ambos límites existan. La forma de garantizar que esto ocurra es tomar el mínimo de los radios de convergencia.

\item
\[\sum_{n=0}^{\infty} a_nb_nz^n \implies \frac{1}{R}=\limsup_{ n \to \infty}|a_nb_n|^{1/n} = \limsup_{n\to \infty}|a_n|^{1/n}|b_n|^{1/n} \leq\]
\[\leq \limsup_{n\to \infty} |a_n|^{1/n}\cdot \limsup_{n\to \infty} |b_n|^{1/n} = \frac{1}{R_1}\frac{1}{R_2}\]

Por tanto tenemos que $R \geq R_1R_2$ y tendremos la igualdad cuando exista al menos uno de los límites superiores: $\limsup_{n\to \infty} |a_n|^{1/n}$ ó $\limsup_{n\to \infty}|b_n|^{1/n}$.

Para afirmar esto nos hemos basado en el siguiente lema:
\begin{lemma}
Sean $x_n,y_n \geq 0$,
\[\exists \lim_{n \to \infty} x_n \implies \limsup_{n \to \infty }(x_ny_n)=\lim_{n \to \infty}x_n \cdot \limsup_{n \to \infty} y_n\]
\end{lemma}

\item
\[\sum_{n = 0}^{\infty} \frac{a_n}{b_n}z^n \text{ con } b_n \neq 0\]
Podemos calcular su radio de convergencia de la siguiente forma:
\[\frac{1}{R}=\limsup_{n \to \infty} |a_n|^{1/n} = \limsup_{n \to \infty} \frac{|a_n|^{1/n}}{|b_n|^{1/n}}|b_n|^{1/n} \leq \limsup_{n \to \infty} \frac{|a_n|^{1/n}}{|b_n|^{1/n}}\limsup_{n \to \infty} |b_n|^{1/n}=\frac{1}{R}\cdot \frac{1}{R_2}\]

Así hemos llegado a que $R \leq \frac{R_1}{R_2}$, teniendo la igualdad en caso de que exista el límite de $|b_n|^{1/n}$

\item \textbf{Producto de Cauchy}
\[\sum_{n=0}^{\infty}c_nz^n \text{ con } c_n=\sum_{k=0}^{n}a_kb_{n-k}\]
Vamos a jugar un poco con esta fórmula
\[\sum_{i=0}^{m}c_iz^i = \sum_{k=0}^m a_kz^k\left(\sum_{l=0}^{m-k} b_lz^l \right)=\]
La igualdad es cierta por puro juego aritmético. La mejor forma de que el lector se convenza de ello y entienda un poco qué hemos hecho es agrupar de nuevo los sumatorios de la derecha y comprobar que, efectivamente, obtenemos el sumatorio de la izquierda de la igualdad.

Sigamos ahora con nuestro cálculo:
\[= \sum_{k=0}^m a_k z^k \left( \sum_{n}^{\infty}b_nz^n-\sum_{l=m-k+1}^{\infty} b_l z^l\right)=\footnote{Para poder aplicar la igualdad estamos considerando $|z| < R_2$ para garantizar la convergencia de la serie}\sum_{k=0}^m a_k z^k \sum_n^{\infty} b_n z^n - \sum_{k=0}^m a_kz^k\sum_{l=m-k+1}^{\infty}b_lz^l = \]
\[= \sum_{k=0}^m a_k z^k \sum_n^{\infty} b_n z^n - \sum_{k=0}^m a_kz^k \cdot 0\footnote{por que la cola de una serie convergente tiende a 0}\]
En definitiva, hemos llegado a que, para $|z| < \min \{R_1, R_2\}$ se cumple que
\[\sum_{i=0}^{m}c_iz^i = \sum_{k=0}^m a_k z^k \sum_n^{\infty} b_n z^n\] donde la sucesión converge por ser producto de convergentes.

Es decir: $R = \min\{R_1, R_2\}$
\end{enumerate}
\end{example}

\begin{prop}
Si la serie $\sum a_n z^n$ tiene radio de convergencia $R$, entonces la serie $\sum na_nz^{n-1}$ también tiene radio de convergencia $R$.
\end{prop}
\begin{proof}
Empecemos viendo que, puesto que la suma es infinita, su convergencia no depende de los primeros términos, de modo que
\[\sum_n na_nz^{n-1} \text{ converge } \iff \sum_n a_n z^n \text{ converge }\]
Vamos a calcular ahora sus radios de convergencia.

Sea $R'$ el radio de convergencia de $\sum_n a_n z^n$ tenemos que
\[\frac{1}{R'}=\limsup_{n \to \infty}n^{1/n}|a_n|^{/n} = \limsup_{n \to \infty}n^{1/n} \cdot \limsup_{n \to \infty}|a_n|^{1/n} = \limsup_{n \to \infty}|a_n|^{1/n} = \frac{1}{R}\]
de modo que queda probada la igualdad

\obs Aplicando el mismo resultado repetidas veces tenemos que para cualquier derivada\footnote{Abuso de notación puesto que aún no hemos definido la derivada de una serie} el radio de convergencia coincide con el de la función original.
\end{proof}

\section{Principio de los ceros aislados}
\begin{defn}[Función analítica]
Sea $\appl{f}{\Omega \subset \cplex}{\cplex}$, decimos que $f$ es \textbf{anaĺitica} en $\Omega$ si para todo $z_0 \in \Omega$ existe una bola $B(z_0, r)=\{z: |z-z_0|<r\}$ donde $F$ coincide con una serie de potencias centrada en $z_0$, es decir
\[f(x)=\sum_{n=0}^{\infty} a_n(z-z_0)^n\]
\end{defn}

\begin{prop}
Si $f(z)=\sum_n a_n(z-z_0)^n$ con radio de convergencia $R$, entonces $f$ es holomorfa en $\{z: |z-z_n|<R\}$ y $f'=\sum_n n a_n z^{n-1}$ con el mismo radio de convergencia.

Es decir, que si una función es analítica entonces será homeomorfa. Más adelante veremos la implicación contraria con lo que tendremos la equivalencia entre estas dos definiciones, como ya adelantamos a principios de curso.
\end{prop}
\begin{proof}
Sin pérdida de generalidad, podemos considerar $z_0 = 0$
\[\left|\frac{f(z+h)-f(z)}{h}- \sum_{n=1}^{\infty}na_nz^{n-1}\right|=\frac{1}{n}\left|\sum_{n=1}^{\infty} a_n \left[(z+h)^n-z-hnz^{n-1}\right] \right| \leq\]
aplicamos ahora la desigualdad triangular que nos lleva a:
\[\leq \frac{1}{n} \sum_{n=1}^{\infty}|a_n|\left| (z+h)^n-z-hnz^{n-1} \right |\]
Pero $(z+h)^n -z^n-hnz^{n-1} = \sum_{k=2}^{n}{n \choose k} z^{n-k}h^k$, y su valor absoluto, aplicando la desigualdad triangular cumpliría que
\[\left|\sum_{k=2}^{n}{n \choose k} z^{n-k}h^k \right| \leq |h|^2 \sum_{k=2}^{\infty}{n \choose k}|z|^{n-k}|h|^{k-2} \leq \frac{|h|^2}{δ^2} \left( |z| + δ\right)^n\]
para la última desigualdad nos basamos en tomar un $|h|<δ$

Volviendo a nuestro sumatorio tenemos
\[\left|\frac{f(z+h)-f(z)}{h}- \sum_{n=1}^{\infty}na_nz^{n-1}\right| \leq  \frac{|h|^2}{δ^2} \sum_{n=1}^{\infty} a_n\left( |z| + δ\right)^n\]
si tomamos ahora $|z|+δ \leq |w| < R$ entonces la serie converge. Para ello basta con tomar un δ(z) apropiado.

Es decir, para cada punto $z$ tendremos un δ que nos garantiza que el valor absoluto de la desigualdad anterior será menor que un cierto ε, con lo que queda claro que la función es holomorfa.

%TODO tratar de aclarar esto que lo ha dado todo deprisa y corriendo
\end{proof}

\obs Si $a_n = \frac{1}{n!}f^{(n)}$ entonces
\[f^{(n)}(z)=\sum_{n=k}^{\infty}n(n-1)\cdots (n-k+1)a_n(z-z_0)^{n-k}\]

\begin{prop}[Principio de los ceros aislados]
Veamos dos enunciados equivalentes de esta proposición
\begin{itemize}
\item Supongamos que existe una sucesión $\{w_k\}$ que converge a $z_0$ sin llegar a alcanzarlo y tal que $f(w_k)=0 \ \forall k$. Entonces $f(z)=0$ para $\{z: |z-z_0|<R\}$ pues
\[a_0=f(z_0)=\lim_{k \to \infty} f(w_k) \overbrace{=}^{\text{por continuidad}} 0 \]
\item Si $\sum a_n(z-z_0)^n$ y $\sum b_n(z-z_0)^n$ convergen y coinciden en una sucesión de puntos que se acumulan en $z_0 \ \implies$ $a_n=b_n \ \forall n$
\end{itemize}
\end{prop}
\begin{proof}
Puesto que $f(z)=0$ tenemos que:
\[\frac{f(z)}{z-z_0}=a_1+a_2(z-z_0)+a_3(z-z_0)^2+... \implies a_0=0\]
y por el mismo argumento aplicado a la función $g(z)=\frac{f(z)}{z-z_0}$ obtenemos $a_1=0$ y así sucesivamente.

Es decir, vemos que si tenemos una serie de ceros de la función que no son aislados, sino que tienen un punto de acumulación, entonces estamos ante un caso trivial en el que la función es nula.
\end{proof}

\begin{example}
Desarrollar las siguientes funciones en potencias de la función que se indica y calcular su radio de convergencia.
\begin{enumerate}
\item
\[\frac{1}{az+b} \text{ con } a,b \in \cplex \ b \neq 0 \ \text{ en potencias de } z\]
Sacando factor común a la $b$ tenemos
\[\frac{1}{az+b} =  \frac{1}{b} \frac{1}{1+frac{az}{b}} = \frac{1}{b}\frac{1}{1-\left(-\frac{az}{b}\right)} = \frac{1}{b}\sum_{n=0}^{\infty}(-1)^n\frac{a^n}{b^n}\]
para que esta serie converga necesitamos
\[\left| \frac{a}{b}z\right| < 1 \implies |z| < \left| \frac{b}{a} \right|\]

\item
\[\frac{6z}{z^2-4z+13} \text{ en potencias de } z\]
Vamos a separarlo en la suma de dos fracciones (como se hace al integrar un cociente de polinomios)
\[\frac{6z}{z^2-4z+13} = \frac{6z}{(z-z_1)(z-z_2)} = \frac{-iz_1}{z-z_1} + \frac{iz_2}{z-z_2} = \]
\[= \frac{i}{1-\frac{z}{z_1}} - \frac{i}{1-\frac{z}{z_2}} = i\sum_{n=0}^{\infty}\left(\frac{z}{z_1}\right)^n - i\sum_{n=0}^{\infty}\left(\frac{z}{z_2}\right)^n = i\sum_{n=0}^{\infty}\left( \frac{1}{z_1^n}-\frac{1}{z_2^n}\right)z^n\]
y podemos ver que su radio de convergencia es
\[|z| = \min \{|z_1|, |z_2|\}= \sqrt{3}\]

\item
\[\frac{z^2}{(1+z)^2} \text{ en potencias de } z\]
Para poder resolver este ejercicio debemos darnos cuenta de que esa función se parece a la derivada de
\[\frac{1}{1+z^2} = \sum_{n=0}^{\infty}(-1)^n z^n\]
salvo que aparece multiplicada por un número: $-z^2$

Derivando a ambos lados tenemos
\[\frac{-1}{(1+z)^2}=\sum_{n=0}^{\infty}(-1)^nnz^{n-1}\]
y multiplicando por $-z^2$ obtenemos
\[\frac{z^2}{(1+z)^2} = \sum_{n=0}^{\infty}(-1)^{n+1}nz^{n+1} = \sum_{n=1}^{\infty}(-1)^n(n-1)z^{n}\]

\item
\[\frac{2z+3}{z+1} \text{ en potencias de } z-1\]
Vemos que podemos realizar la divisón cómodamente con lo que obtenemos
\[\frac{2z+3}{z+1}=2+\frac{1}{z+1} = 2 + \frac{1}{2+z-1} = 2 + \frac{1}{2\left( 1+\frac{z-1}{2}\right)} = 2+\frac{1}{2}\sum_{n=0}^{\infty}(-1)^n\left(\frac{z-1}{2} \right)^n\]
y podemos ver que su radio de convergencia sería
\[|z-1| < 2\]

\end{enumerate}
\end{example}

\section{La función exponencial}
\begin{defn}[Función exponencial]
La serie $\sum_{n=0}^{\infty}\frac{z^n}{n!}$ tiene radio de convergencia $R=\infty$ pues
\[\lim_{n \to \infty}\left| \frac{a_n}{a_{n+1}}\right| = \lim_{n \to \infty} \frac{(n+1)!}{n!} = \infty\]
Definimos la \textbf{función exponencial} como $e^z=exp(z)=\sum_{n=0}^{\infty}\frac{z^n}{n!}$.

Esta función es holomorfa en $\cplex$ y es su propia derivada.
\end{defn}

Por que nos será útil saberlo más adelante, veamos una serie de propiedades de las funciones holomorfas:
\begin{prop}
Sea $f$ holomorfa en $\Omega$, entonces:
\begin{enumerate}
\item Si $f'=0$ en $\Omega \implies $ $f$ es constante
\item Si $Re(f)$ es constante en $\Omega \implies \ f$ es constante
\item Si $Im(f)$ es constante en $\Omega \implies \ f$ es constante
\item Si $|f|$ es constante en $\Omega \implies \ f$ es constante
\end{enumerate}
\end{prop}
\begin{proof}
\begin{enumerate}
\item Si $f'=0$ en $\Omega$, tenemos que
\[u_x+iv_x = 0 \text{ en }\Omega\]
pero por las ecuaciones de Cauchy-Riemann, $u_x=v_y$ y $u_y=-v_x$.

Combinando esto con la ecuación obtenida al igualar la derivada a 0 y con la que podríamos obtener derivando respecto a $y$, tenemos que las cuatro derivadas parciales coinciden y son 0 por lo que la función ha de ser constante.

\item Si $u$ es constante tenemos que $u_x=u_y=0$ y, por las ecuaciones de Cauchy-Riemann tenemos que las derivadas respecto de $y$ también son 0.

\item Si $v$ es constante tenemos que $v_x=v_y=0$ y, por las ecuaciones de Cauchy-Riemann tenemos que las derivadas respecto de $y$ también son 0.

\item S $|f|$ es constante tenemos que $u^2+v^2$ es constante

Derivando obtenemos
\[2uu_x+2vv_x=0\]
\[2uu_y+2vv_u=0\]
y, por las ecuaciones de Cauchy-Riemann tenemos
\[uu_x+vv_x =0\]
\[-uv_x+vux = 0\]
por lo que $(u,v)$ es ortogonal a $(u_x,v_x)$ y a $(-v_x,u_x)$, que son linealmente independientes si son distintos del vector nulo.

\end{enumerate}
\end{proof}

Volvamos ya al estudio de nuestra función exponencial.
\begin{prop}
Para todo $z_1$, $z_2 \in \cplex \ e^{z_1+z_2}=e^{z_1}e^{z_2}$
\end{prop}
\begin{proof}
Fijo $w \in \cplex$ y definimos $g(z)=e^z e^{w-z} \ \forall z \in \cplex$

Derivando obtenemos que
\[g'(z)=e^z e^{w-z}-e^ze^{w-z}=0 \implies g(z)=cte\]
por ser constante tendrá el mismo valor en todos sus puntos, de modo que
\[g(z)=g(0)=e^w \implies e^w=e^ze^{w-z}\]
\end{proof}

Veamos algunas propiedades que se deducen de lo que acabamos de estudiar:
\begin{enumerate}
\item \[e^ze^{-z}=e^{z-z}=e^0=1 \implies e^z=\frac{1}{e^{-z}}\]
\item \[e^{\bar{z}}=\overline{e^z} \implies |e^z|^2=e^z\overline{e^z} = e^z e^{\bar{z}}=e^{z+\bar{z}} e^{2 Re(z)}\]

Es decir, nos queda que
\[|e^z|=e^{Re(z)}\]
\end{enumerate}


\section{Funciones trigonométricas e hiperbólicas}
Por analogía con el caso real, podemos escribir:
\[\cos(z)=1-\frac{z²}{2!}+\frac{z⁴}{4!}+... = \sum_{k=0}^{\infty}(-1)^k\frac{z^{2k}}{(2k)!}\]
\[\sin(z)=z-\frac{z^3}{3!}+\frac{z^5}{5!}+... = \sum_{k=0}^{\infty}(-1)^k\frac{z^{2k+1}}{(2k+1)!}\]

Ambas series tienen radio de convergencia infinito y son enteras (son holomorfas en $\cplex$)

Además podemos ver que si derivamos formalmente estas series obtenemos la relación habitual entre las derivadas de estas funciones cuando trabajamos en los reales. Es decir, tenemos que
\[(\cos(z))'=-\sin(z) \ \text{ y } (\sin(z))'=\cos(z)\]

Si jugamos un poco con las series del seno y el coseno, podríamos escribirlos como:
\[\cos(z)=\frac{1}{2}\left( e^{iz}+e^{-iz}\right)\]
\[\sin(z)=\frac{1}{2i}\left( e^{iz}-e^{-iz}\right)\]
de donde podemos sacar la \textbf{fórmula de Euler}:
\[e^{iz}=\cos(z)+i\sin(z)\]

Así mismo, podemos ver que se siguen cumpliendo varias propiedades-definiciones asociadas a estas funciones.
\begin{enumerate}
\item
\[\sin^2(z)+\cos^2(z) =1\]
\item
\[\cos(z_1+z_2)=\cos(z_1)\cos(z_2)-\sin(z_1)\sin(z_2)\]
\item
\[\sin(z_1+z_2)=\sin(z_1)\cos(z_2)+\sin(z_2)\cos(z_1)\]
\end{enumerate}
\begin{proof}
La estructura de la demostraión es la misma en los tres casos. Vamos a realizar la primera y a dejar el resto como ejercicio para el lector.

Ya sabemos por el análisis en variable real, $\sin^2(x)+\cos^2(x)=1 \ \forall x \in \cplex$

Entonces estamos teniendo una serie infinita de ceros de la función
$g(z)=\sin^2(z)+\cos^2(z)-1$
pero por tratarse de una función holomorfa, los ceros han de ser aislados a menos que la función sea constante nula.

Por tanto, es claro que con los complejos mantenemos la propiedad.
\end{proof}

Tanto la tangente como los senos y cosenos hiperbólicos conservan su definición de $\real$, por lo que tenemos:
\[\tan(z)=\frac{\sin(z)}{\cos(z)}=-i \frac{e^iz-e^{-iz}}{e^{iz}+e^{-iz}}\]

\[\sinh(z)=\frac{e^z-e^{-z}}{2}\]
\[\cosh(z)=\frac{e^z+e^{-z}}{2}\]

\begin{example}
Vamos a comprobar que los ceros de $\sin(z)$ y $\cos(z)$ son reales (no añadimos ninguno al trabajar con complejos).

Para ello vamos a descomponer el seno de un número complejo como sigue:
\[\sin(z)=\sin(x+iy)=\sin(x)\cos(iy)+\sin(iy)\cos(x)=\sin(x)\cosh(y)+i\sinh(y)\cos(z)\]
Si tuviésemos que $z$ es un cero de la función, tendríamos que se cumplen las ecuaciones en los reales
\[\sin(x)\cosh(y)=0 \overbrace{\implies}^{\cosh > 1} \sin(x)=0 \implies x = 0\]
y
\[\sinh(y)\cos(x)=0 \overbrace{\implies}^{\sin(x)=0} \sinh(y)=0 \implies y=0\]
con lo que podemos ver que, efectivamente no se ha añadido ningún punto $z$ distinto de los que ya teníamos la trabajar con los reales.
\end{example}

Se deja como ejercicio para el lector la comprobación de que los ceros de $\sinh$ y $\cosh$ son imaginarios puros.

\begin{defn}[Función periódica]
Una función $f$ es periódica de período c $\neq 0$ sii
\[f(z+c)=f(z) \ \forall z \in \cplex\]
\end{defn}

\chapter{Fórmula integral de Cauchy y sus aplicaciones}
\section{Fórmula de Green}
\section{Teorema de Cauchy}
\section{Teorema de Liouville}
\section{Teorema de Morera}
\section{La función primitiva en un dominio simplemente conexo}
\section{Fórmula integral de Cauchy}
\section{Equivalencia entre holomorfía y analiticidad}


\chapter{Cálculo de residuos}
\section{Singularidades aisladas}
\section{Teorema de la singularidad evitable de Riemann}
\section{Series de Laurent}
\section{Teorema de los residuos}
\section{Aplicaciones al cálculo de integrales}


\chapter{Algunos teoremas fundamentales de la variable compleja}
\section{Teorema de Rouché}
\section{Principio del argumento}
\section{Teorema de la aplicación abierta}
\section{Principio del módulo máximo}
\section{Lema de Schwarz}


\chapter{Introducción a la representación (transformación) conforme}
\section{Transformaciones de Möbius}
\section{Automorfismos del disco}
\section{Enunciado del teorema de representación conforme de Riemann}
\section{Aplicaciones conformes entre distintos dominios simplemente conexos en el plano}


%% Apéndices (ejercicios, exámenes)
\appendix

\chapter{Ejercicios}
% -*- root: ../VariableCompleja.tex -*-

\section{Hoja 1}
%
\begin{problem}[1]
Realices las operaciones indicadas
\ppart
\[\frac{1}{i}+\frac{1}{1+i}\]
\ppart
\[\frac{2}{(1-3i)^2}\]
\ppart
\[(1+i\sqrt{3})^3\]
\ppart
\[(\overline{1-i})^2+\overline{2+i}\]

\solution

\spart
\[\frac{1}{i}+\frac{1}{1+i} = \frac{1+i+i}{i-1} = -\frac{2i}{1-i} = \frac{2i(1+i)}{(1-i)(1+i)} = \frac{2i-2}{1+1}=i-1\]
\spart
\[\frac{2}{(1-3i)^2} = \frac{2}{1-9-6i} = \frac{2}{-8-6i}=\frac{2(-8+6i)}{(-8-6i)(-8+6i)} = \frac{2(-8+6i)}{64+36} = \frac{-4+3i}{25}\]
\spart
\[(1+i\sqrt{3})^3\ = 1 +3i\sqrt{3}-3\cdot 3-i3^{\frac{3}{2}} = -8 +i (3\sqrt{3}-3\sqrt{3}) = -8\]
\spart
\[(\overline{1-i})^2+\overline{2+i} = \overline{1-1-2i}+2-i = 2i+2-i=2+i\]
\end{problem}

\begin{problem}[2]
Calcule los valores de
\ppart
\[\sum_{k=1}^{2015}i^k\]
\ppart
\[(1+i)^n+(1-i)^n\]
\ppart
\[\left( \cos \left( \frac{\pi}{12} \right) + i \sin \left( \frac{\pi}{12} \right)\right)^{20}\]
\ppart
\[\left(\frac{1+i}{1-i}\right)^{2014}\]
\solution

\spart
Por tratarse de una sucesión geométrica de razón $i$ sabemos que:
\[\sum_{k=1}^{2015}i^k = \frac{1-i^{2016}}{1-i} -1\]

\spart
Primero debemos observar que
\large
\[(1+i) = 2^{\frac{1}{2}}e^{\frac{\pi}{4}i}\]
\normalsize
por tanto
\[ (1+i)^n= 2^{\frac{n}{2}}e^{\frac{\pi}{4}in} = 2^{\frac{n}{2}}\left(\cos\left(\frac{\pi}{4}n\right)+i \sin\left(\frac{\pi}{4}n\right)\right)\]

Teniendo en cuenta esta relación, podemos resolver el ejercicio:
\[(1+i)^n+(1-i)^n = (1+i)^n+\overline{(1+i)^n} = 2 Re((1+i)^n)=2^{\frac{n}{2}+1}\cos\left(\frac{\pi}{4}n\right)\]

\spart
\[\left( \cos \left( \frac{\pi}{12} \right) + i \sin \left( \frac{\pi}{12} \right)\right)^{20}=\left(e^{\frac{\pi}{12}}\right)^20 = e^{\frac{\pi}{12}\cdot 20} = \cos \left( \frac{\pi\cdot 20}{12} \right) + i \sin \left( \frac{\pi\cdot 20}{12} \right) =\]
\[=\cos \left( \frac{\pi\cdot 5}{3} \right) + i \sin \left( \frac{\pi \cdot 5}{3} \right)\]

\spart
Primero vamos a trabajar con el interior del paréntesis para convertirlo en un número complejo en su expresión habitual, sin fracciones.
\[\frac{1+i}{1-i}=\frac{(1+i)^2}{(1-i)(1+i)} = \frac{1-1+2i}{1+1} = i\]
y puesto que el exponente es par, tenemos que
\[\left(\frac{1+i}{1-i}\right)^{2014}=i^{2014}=1\]
\end{problem}

\begin{problem}[3]
Sea $z=x+iy \in \cplex$. Demuestre que $|x|+|y|\leq \sqrt{2}|z|$, y que sólo hay igualdad si $|x|=|y|$.

\textbf{Ayuda:} Si $a,b \in \real$, entonces $2ab \leq a^2 + b^2$ (con igualdad sólo si $a=b$)

\solution

Si calculamos el módulo de z vemos que
\[|z|=\sqrt{x^2+y^2}\]
si $|x|=|y|$ fácilmente vemos que
\[|z|=\sqrt{x^2+x^2}=\sqrt{2x^2}=\sqrt{2}|x| \implies \sqrt{2}|z|=2|x|=|x|+|y|\]

Veamos ahora el caso en que no son iguales. En esta ocasión, nos apoyamos en al ayuda del enunciado y vemos que
\[|z|=\sqrt{x^2+y^2} \geq \sqrt{2xy} \iff \sqrt{2}|z| \geq 2\sqrt{xy}\]
%TODO terminar esto

\end{problem}

\begin{problem}[4]
Compruebe la identidad
\[|z\bar{w}+1|^2+|z-w|^2 = (1+|z|^2)(1+|w|^2)\]
donde $z,w \in \cplex$

\solution

Llamando a $z=a+bi$ y $w=c+di$ tenemos
\[|(a+bi)(c-di)+1|^2+|a+bi+c-di|^2=|ac-adi+bci+bd+1|^2+|a+bi-c+di|^2 = \]
\[=(ac+bd+1)^2+(bc-ad)^2+(a-c)^2+(b+d)^2 =\]
\[= 1 + a²c²+b²d²+2acbd + 2ac+2bd+b²c²+a²d²-2bcad+a^2+c²-2ac+b²+d²-2bd =\]
\[= 1+a²c²+b²d²+b²c²+a²d²+a²+b²+c²+d² = (1+a²+b²)(1+c²+d^2)\]

\end{problem}

\begin{problem}[5]
Demuestra las siguientes afirmaciones
\ppart
\[\text{Si } |z|=1, \text{ entonces para todos } a,b \in \cplex \text{ con } \bar{b}z+\bar{a} \neq 0\text{ se cumple } \left| \frac{az+b}{\bar{b}z+\bar{a}}\right| = 1\]

\ppart
\[\text{Si } |a| < 1, \text{ entonces } |z| <1 \text{ es equivalente a } \left| \frac{z-a}{1-\bar{a}z}\right|<1\]

\solution

\spart
%Vamos a simplificar el complejo dado
%\[\frac{az+b}{\bar{b}z+\bar{a}}= \frac{(az+b)\bar{z}}{(\bar{b}z+\bar{a})\bar{z}}=\frac{a|z|^2+b\bar{z}}{\bar{b}|z|^2+\bar{a}\bar{z}} = \frac{a+b\bar{z}}{\bar{b}+\bar{z}\bar{a}}\]

%Llegados a este punto, podemos observar que $\overline{\bar{b}+\bar{z}\bar{a}}=b+za$
%\[\frac{az+b}{\bar{b}z+\bar{a}} = \frac{(a_1+ia_2)(z_1+iz_2)+b_1+ib_2}{}\]

\spart

\[\left| \frac{z-a}{1-\bar{a}z}\right|<1 \iff |z-a| < |1-\bar{a}z| \iff \underbrace{|z-a|^2}_{(z-a)(\bar{z}-\bar{a})} < \underbrace{|1-\bar{a}z|^2}_{(1-\bar{a}z)(1-a\bar{z})} \]

Por lo que nos queda que debe cumplirse
\[|z|^2-a\bar{z}-\bar{a}z+|a|^2 < 1-\bar{a}z+a\bar{z}+|a|^2|b|^2 \iff |z|^2+|a|^2-2\cdot Re(z\bar{a}) < 1 + |a|^2|z|^2-2\cdot Re(z\bar{a}) \iff\]

\[\iff |z|^2 + |a|^2 < 1 +|a|^2|z|^2 \iff |z|^2(1-|a|^2) < 1 - |a|^2 \iff |z|^2 < 1 \iff |z| < 1\]

\end{problem}

\begin{problem}[6]
Usando la fórmula de A. de Moivre, demuestre que
\ppart
$\sin(3x)=3\sin(x)-4 \sin^3(x)$, para todod $x \in \real$

\ppart
Para todo $n \in \nat$ par, la función $\cos(n \phi)$ es un polinomio de grado $n$ de $\cos(\phi)$.

\solution

\spart
Aquí hay que tener algo de idea feliz, aunque sabiendo que estamos trabajando con complejos, tampoco es demasiado raro de pensar.

Vamos a elevar el complejo $\cos(x)+i \sin(x)$ al cubo de dos formas distintas y a igualar los resultados.

\begin{enumerate}
\item
\[\left( \cos(x)+i \sin (x) \right)^3 = (e^{ix})^3 = e^{3ix} = \left( \cos(3x)+i \sin (3x) \right)\]
\item
\[\left( \cos(x)+i \sin (x) \right)^3 = . . . = \cos^3(x)-3\cos(x)\sin^2(x) + i \left( 3\cos^2(x)\sin(x)-\sin^3(x)\right)\]
\end{enumerate}
Ahora, puesto que deben ser iguales las dos representaciónes del cubo calculado, debemos igualar las partes reales y las imaginarias.

En este caso, en cuanto forzamos la igualdad de las partes imaginarias obtenemos la igualdad buscada.
\[3\cos^2(x)\sin(x)-\sin^3(x) = \sin(3x) \iff 3\sin(x) - 3 \sin^3(x) - \sin^3(x)=\sin(3x) \iff\]
\[\iff 3\sin(x) - 4 \sin^3(x)=\sin(3x) \]

\spart
El procedimiento a seguir es prácticamente igual que en el caso anterior. Vamos a calcular $\left(\cos(\phi)+i\sin(\phi)\right)^n$ de dos formas distintas
\begin{enumerate}
\item
\[\left(\cos(\phi)+i\sin(\phi)\right)^n = \cos(n\phi)+i\sin(n\phi)\]
\item
\[\left(\cos(\phi)+i\sin(\phi)\right)^n = \sum_{k=0}^n { n \choose k} \cos(\phi)\left( i \sin (\phi)\right)^{n-k}\]
\end{enumerate}

Atendiendo al sumatorio, vemos que vamos a obtener reales siempre que $k$ sea par. En otro caso tendremos siempre un múltiplo de $i$. La suma de esos múltiplos de $i$ acabará siendo $\sin (n\phi)$.

Aplicando esto llegamos a:
\[\cos(n\phi) = \sum_{0 \leq k \leq n} { n \choose k} \cos^k(\phi)(-1)^{\frac{n-k}{2}}\sin^{n-k}(\phi)\]

pero, si nos fijamos en el seno, tenemos que

\[\sin^{n-k}(\phi) = \left(\sin^2(\phi)\right)^{\frac{n-k}{2}} = \left(1-\cos^2(\phi)\right)^{\frac{n-k}{2}}\]

y aplicando esta relación a la igualdad anterior, obtenemos
\[\cos(n\phi) = \sum_{0 \leq k \leq n} (n,k) \cos^k(\phi)(-1)^{\frac{n-k}{2}}\left(1-\cos^2(\phi)\right)^{\frac{n-k}{2}}\]

que, efectivamente, se trata de un polinomio de grado $n$ de $\cos(\phi)$
\end{problem}

\begin{problem}[7]
Demuestre que
\[\left( \frac{1+i\tan(\phi)}{1-i\tan(\phi)}\right)^n = \frac{1+i\tan(n\cdot\phi)}{1-i\tan(n\cdot\phi)}\]

\solution

Vamos a autoconvencernos de que la igualdad es cierta con $n=2$
\[\left( \frac{1+i\tan(\phi)}{1-i\tan(\phi)}\right)^2 = \frac{(1+i\tan(\phi))^2}{(1-i\tan(\phi))^2} = \frac{1-\tan(\phi)^2+2i\tan(\phi)}{1-\tan(\phi)^2-2i\tan(\phi)} = \frac{\cos(\phi)^2-\sin(\phi)^2+2i\sin(\phi)\cos(\phi)}{\cos(\phi)^2-\sin(\phi)^2-2i\sin(\phi)\cos(\phi)}=\]
\[=\frac{\cos(\phi)^2-\sin(\phi)^2+i\sin(2\phi)}{\cos(\phi)^2-\sin(\phi)^2-i\sin(2\phi)}\]
Ahora dividimos entre $\cos(\phi)^2-\sin(\phi)^2$ y, sabiendo que
\[\tan(2 \phi)=\frac{2\sin(\phi)\cos(\phi)}{\cos(\phi)^2-\sin(\phi)^2}\]
tenemos que
\[\frac{\cos(\phi)^2-\sin(\phi)^2+i\sin(2\phi)}{\cos(\phi)^2-\sin(\phi)^2-i\sin(2\phi)}=\frac{1+i\tan(2\cdot\phi)}{1-i\tan(2\cdot\phi)}\]

Ahora vamos a aplicar inducción. Suponemos que la igualdad es cierta para $n$ y vamos a ver qué ocurre con $n+1$.
\[\left( \frac{1+i\tan(\phi)}{1-i\tan(\phi)}\right)^{n+1} =  \frac{\left(1+i\tan(n\cdot\phi)\right)\left(1+i\tan(\phi)\right)}{\left(1-i\tan(n\cdot\phi)\right)\left(1-i\tan(\phi)\right)} = \frac{1-\tan(n\phi)\tan(\phi)+i\left(\tan(\phi)+\tan(n\phi)\right)}{1-\tan(n\phi)\tan(\phi)-i\left(\tan(\phi)+\tan(n\phi)\right)} = \]
multiplicando y dividiendo por $\cos(n\phi)\cos(\phi)$ llegamos a
\[=\frac{\cos(n\phi)\cos(\phi)-\sin(n\phi)\sin(\phi)+i\left(\sin(\phi)\cos(n\phi) + \cos(\phi)\sin(n\phi)\right)}{\cos(n\phi)\cos(\phi)-\sin(n\phi)\sin(\phi)-i\left(\sin(\phi)\cos(n\phi) + \cos(\phi)\sin(n\phi)\right)} =\]
\[=\frac{\cos(n\phi)\cos(\phi)-\sin(n\phi)\sin(\phi)+i\left(\sin((n+1)\phi)\right)}{\cos(n\phi)\cos(\phi)-\sin(n\phi)\sin(\phi)-i\left(\sin((n+1)\phi)\right)}\]
Al igual que hicimos en el caso particular de $n=2$, ahora multiplicamos y dividimos por $\cos(n\phi)\cos(\phi)-\sin(n\phi)\sin(\phi)$ y, sabiendo que
\[\tan((n+1)\phi)=\frac{\sin((n+1)\phi)}{\cos((n+1)\phi)}=\frac{\sin(\phi)\cos(n\phi) + \cos(\phi)\sin(n\phi)}{\cos(n\phi)\cos(\phi)-\sin(n\phi)\sin(\phi)}\]
obtenemos directamente el resultado.

\[\frac{\cos(n\phi)\cos(\phi)-\sin(n\phi)\sin(\phi)+i\left(\sin((n+1)\phi)\right)}{\cos(n\phi)\cos(\phi)-\sin(n\phi)\sin(\phi)-i\left(\sin((n+1)\phi)\right)} =  \frac{1+i\tan((n+1)\cdot\phi)}{1-i\tan((n+1)\cdot\phi)}\]
\obs La última igualdad indicada se obtiene calculando $\sen(α+β)$ y $\cos(α+β)$ con las fórmulas habituales, considerando $α=n\phi$ y $β=\phi$

\end{problem}

\begin{problem}[8]
Sin realizar cálculo alguno, razónese que no es posible que alguno de los valores de $\sqrt[1928]{1+i}$ sea $\frac{1-i}{2}$

\solution
Lo más fácil, en este caso, es ver que los módulos no coinciden. Para ello escribimos
\[1+i = 2^{\frac{1}{2}}e^{(\frac{\pi}{4}+2\pi k)i}\]
y al calcular la raíz obtenemos
\[(1+i)^{\frac{1}{1928}} = 2^{\frac{1}{2\cdot 1928}}e^{(\frac{\pi}{4}+2\pi k)\frac{i}{1928}}\]

Llegados a este punto, podemos ver que los módulos no coinciden, pues
\[2^{\frac{1}{1928}}\neq \left|\frac{1-i}{2}\right| = \sqrt{\frac{1}{2}}\]
\end{problem}

\begin{problem}[9]
Demuestre las siguientes afirmaciones
\ppart
Si $z\neq 1$ entonces
\[\sum_{i=0}^n z^i = \frac{1-z^{n+1}}{1-z}\]

\ppart
Si $w\neq 1$ es una raíz n-ésima de la unidad, entonces
\[\sum_{i=0}^{n-1} w^i = \sum_{i=1}^n w^i= 0\]
y
\[\sum_{i=0}^{n-1} i w^i = \frac{n}{w-1}\]

\ppart
si $\sin\left(\frac{\phi}{2}\right)$, entonces
\[\sum_{i=0}^n \cos(i\phi) = \frac{1}{2}\left(1+\frac{\sin((n+\frac{1}{2})\phi)}{\sin\left( \frac{\phi}{2}\right)}\right)\]

y

\[\sum_{i=1}^n \sin(\phi) = \frac{\sin(\frac{n}{2}\phi)\sin(\frac{n+1}{2}\phi)}{\sin(\frac{\phi}{2})}\]

\textbf{Ayuda:} Use el apartado a) con $z=e^{i\phi}$

\solution
\spart
Vamos a demostrarlo por inducción. En este caso, el caso base es trivial, pues sería $n=1$ con lo que tendríamos
\[1+z=\frac{1-z^2}{1-z}=\frac{(1-z)(1+z)}{1-z} = 1+z\]
Ahora suponemos que la fórmula es válida para $n$ y vamos a ver qué ocurre para $n+1$.
\[\sum_{i=0}^{n+1} z^i = \sum_{i=0}^n z^i + z^{n+1} = \frac{1-z^{n+1}}{1-z} + z^{n+1} = \frac{1-z^{n+1}+z^{n+1}-z^{n+2}}{1-z} = \frac{1+z^{n+2}}{1-z}\]
por lo que queda probado que si la ecuación se cumple para $n$ se cumple también para $n+1$ y, puesto que se cumple para 1, podemos concluir que la ecuación es válida.

\spart
Este caso resulta muy sencillo y rápido sin los apoyamos en el anterior y sabemos que $w^n=1$ y que $w^{n+1}=w$ siendo $w$ una raíz n-ésima de la unidad.

Por el apartado anterior sabemos que
\[\sum_{i=0}^{n+1} z^i = \sum_{i=0}^n z^i + z^{n+1} = \frac{1-w^{n+1}}{1-w}-1 = \frac{1-w}{1-w}-1 = 0\]

Para la segunda igualdad símplemente tenemos que darnos cuenta de que
\[\sum_{i=0}^{n-1} i w^i  = \frac{\partial}{\partial w} \sum_{i=0}^{n}w^i = \frac{\partial}{\partial w} \frac{1-w^{n+1}}{1-w} = \frac{(-n-1)(1-w)+(1-w)}{(1-w)^2} = \frac{-n-1+1}{1-w}\]
y llegamos a
\[\sum_{i=0}^{n-1} i w^i = \frac{-n}{1-w}=\frac{n}{w-1}\]
\spart
%TODO completar "Lo hice" en clase


\end{problem}

\begin{problem}[10]
Calcule todos los valores de
\ppart
\[\left(-\sqrt{2}-i\sqrt{2}\right)^{1/3}\]
\ppart
\[\sqrt{1-i\sqrt{3}}\]
\ppart
\[\sqrt[4]{1-i}\]
\ppart
\[\left(\sqrt{-i}\right)^{1/3}\]

\solution

\spart
\[\left(-\sqrt{2}-i\sqrt{2}\right)^{1/3}=\left(\sqrt{2}(-1-i)\right)^{1/3}=(\sqrt{2}\sqrt{2})^{1/3}\left(e^{\frac{-3\pi}{4}+2k\pi}\right)^{1/3}2^{1/3}e^{\frac{-\pi}{4}+\frac{2}{3}k\pi}\]

\spart
\[\sqrt{1-i\sqrt{3}} = \sqrt{4e^{(\pi/3+2k\pi)i}}=2e^{(\pi/6+k\pi)i}\]
\spart
\[\sqrt[4]{1-i} = \sqrt[4]{2e^{7\pi/8+2k\pi}} = \sqrt[4]{2}e^{(7\pi/32 + k\pi /2)i}\]

\spart

\[\left(\sqrt{-i}\right)^{1/3} = \left(1e^{(\pi+2k\pi)i}\right)^{1/6} = e^{(\pi/6+k\pi/3)i}\]
\end{problem}

\begin{problem}[11]
En este ejercicio, consideramos sólo el \textit{valor principal de la raíz cuadrada}, definido como
\[\sqrt[(p)]{z}=\sqrt{r}\left(\cos\frac{\phi}{2}+i\sin\frac{\phi}{2}\right)\]
cuando $z=r(\cos\phi+i\sin\phi)$ con $-\pi < \phi < \pi$. Claramente, $\left( \sqrt[(p)]{z} \right)^2=z$
\ppart Demuestra que las soluciones en $\cplex$ de la ecuación $az^2+bz+c=0$, con $a\neq 0$, son
\[z=\frac{-b\pm \sqrt[(p)]{b^2-4ac}}{2a}\]
\ppart
Calcule
\[\sqrt[(p)]{\left(\sqrt[(p)]{i}\right)^5} \text { y } \sqrt[(p)]{1+\sqrt[(p)]{i}}\]
\solution

\spart
Para resolver este apartado basta son sustituir la fórmula que nos dan para la $z$ en la ecuación dada y comprobar que, efectivamente, la ecuación se verifica.

\spart
La raíz principal puede sonar a algo exótico pero consiste, simplemente, en tomar la raíz del número dado y, en lugar de considerar los varios ángulos posibles, tomamos el menor posible (siempre positivo).

A efectos legales esto nos hace ahorrarnos el típico $+2k\pi$. Veamos a modo e ejemplo los radicales que nos pide calcular el enunciado
\[\sqrt[(p)]{\left( \sqrt[(p)]{i}\right)^5} = \sqrt[(p)]{e^{-\frac{3\pi}{4}i}} = e^{-\frac{3\pi}{8}}i\]
\end{problem}

\begin{problem}[12]
Resuelve las siguientes ecuaciones:
\ppart
\[(z+1)^4+i=0\]
\ppart
\[Re(z^2+5)=0\]
\ppart
\[Re(z+5)=Im(z-i)\]
\solution

\spart
Despejando como hemos hecho siempre tenemos que
\[z=\sqrt[4]{-i}-1 = e^{\pi/4+\pi k / 2} -1 \]

\spart
Considerando $z=x+iy$ tenemos que $z^2=x^2-y^2+2xyi$ con lo que llegamos a
\[x^2-y^2 = -5\]
que nos da una hipérbola
\spart
Considerando $z=x+iy$ tenemos
\[Re(z+5)=Im(z-i) \iff x+5=y-1\]
obteniendo como resultado una recta.

\end{problem}

\begin{problem}[13]
\ppart
Demuestra que si $w$ es solución de $z^n=μ$ (con $μ\in\cplex$ fijo), entonces todas las soluciones son $ww_i$ con $i=0,1,...,n-1$ donde $w_i$ son las raíces n-ésimas de la unidad
\ppart
Encuentre razonadamente las soluciones de $z^6-8=0$
\solution
\textcolor{blue}{Hecho por mi. No fiarse al 100\%}

\spart
Es sencillo e ver que los números de la forma $ww_i$ son soluciones, puesto que
\[(ww_1)^n = w^n w_i^n= μ \cdot 1 = μ\]
Cualquier otra hipotética solución deberá cumplir que al elevarla a $n$ obtengamos μ, por lo que deberá ser $w$ multiplicado por algo que, al elevarlo a $n$ nos de 1. Es decir, no habrá más posibilidades que las indicadas

\spart
Siguiente lo indicado en el apartado anterior las soluciones serán de la forma:
\[z=\sqrt[6]{8}w_i \text{ con } i=0,1,2,3,4,5 \text{ y } w \text{ raíz n-ésima de la unidad}\]

\end{problem}

\begin{problem}[14]
¿Cuándo son colineales tres puntos $z_1,z_2,z_3$ distintos dos a dos?
\solution
Para verlo hacemos como en bachillerato con los reales: escribimos la recta que pasa por dos de esos puntos y forzamos a que el tercero se contenga en dicha recta.

La recta que pasa por $z_1$ y $z_2$ sería:
\[L=\{z_1+t(z_2-z_1) t \in \real \}\]
Si $z_3 \in L \implies \exists t \tq z_3=z_1 + t(z_2-z_1)$ es decir:
\[t = \frac{z_3-z_1}{z_2-z_1}\in \real\]
para que sea real ese resultado necesitamos que el numerador y el denominador tengan el mismo argumento.

Intuitivamente representa que uniendo $z_3$ con $z_1$ obtenemos la misma recta que uniendo $z_2$ con $z_1$
\end{problem}

\begin{problem}[15]

\ppart
Compruebe que la ecuación
\[Re(az+b) = 0 \text{ con } a,b\in \cplex, \ a \neq 0\]
define una recta en el plano y que, recíprocamente, cada recta viene descrita por una ecuación de este tipo

\ppart
Encuentre los números $a,b$ para que la recta pase por dos puntos dados $z_1, z_2 \in \cplex$

\ppart
Demuestre que las rectas determinadas por las ecuaciones $Re(az+b)=0$ y $Re(cz+d)=0$ respectivamente, son perpendiculares si y sólo si $Re(a\bar{c})=0$

\ppart
Demuestre que la ecuación de una recta que pasa por dos puntos dados $z_1$ y $z_2$ puede escribirse de la forma
\[ \left| \begin{array}{ccc}
z  & \bar{z} & 1 \\
z_1 & \bar{z_1}&  1 \\
z_2 & \bar{z_2} & 1 \end{array} \right| = 0\]

\solution
\textcolor{blue}{Hecho por mi. No fiarse al 100\%}

\spart

Siendo cada número complejo $x\in \cplex = x_r+ix_i$, la ecuación que nos dan se traduce en
\[a_rz_r-a_iz_i+b_r=0 \equiv z_i = \frac{a_rz_r+b_r}{a_i}\]
siendo $x_i = y$ t $z_r = x$ obtenemos la ecuación de una recta en el plano.

\spart

Basta cons sutituir en la ecuación los valores $z_1=z_1r+iz_1i$ y $z_2=z_2r+iz_2i$ y obtenemos un sistema de 4 ecuaciones e 4 incógnitas que podremos resolver.

\spart

Basándonos en el apartado a), podemos ver que las pendientes de esas rectas son, respectivamente, $\frac{a_r}{a_i}$ y $\frac{c_r}{c_i}$.

Para que sean perpendiculares, debemos tener
\[\frac{a_r}{a_i}= - \frac{c_i}{c_r} \implies a_rc_r = -a_ic_i \implies Re\left((a_r+ia_i)(c_r-ic_i)\right) = 0\]

\spart
%TODO por hacer

\end{problem}

\begin{problem}[16]
Describa el conjunto del plano complejo determinado por las siguientes relaciones
\ppart
\[|z-2|-|z+2| > 3\]
\ppart
\[Re(z)+Im(z) < 1\]
\ppart
\[|2z|>|1+z^2|\]

\solution
\textcolor{blue}{Hecho por mi. No fiarse al 100\%}

\spart
Si tuviéramos una igualdad, estaríamos hablando de los puntos del plano cuya diferencia de distancias a los puntos $(2,0)$ y $(-2,0)$ es constante. Es decir, tendríamos una hipérbola.

Al tener una desigualdas, estamos cogiendo aquellos puntos situados a la derecha de la hipérbola.

\spart

Esta ecuación representa aquellos puntos del plano que quedan a la izquierda de la recta $y=-x+1$.

\spart


\end{problem}

\begin{problem}[17]
Determine las ecuaciones complejas:
\ppart de la parábola con foco i y directriz $Im(z)=-1$
\ppart de la elipse con focos $\pm 1$ que pasa por $i$
\ppart de la hipérbola con focos $\pm 1$ que pasa por $i+1$

\solution

Este ejercicio es bastante semejante a los apartados b) y c) del ejercicio 1.10
\spart
Recordemos que una parábola se definía a partir del foco y la directriz como el conjunto de puntos del plano que equidistaban de ellos.

Para escribir la ecuación, simplemente aplicamos la definición y vemos a que ecuación nos lleva.

Sea un punto cualquiera $z=x+iy$ su distancia al foco $i$ sería $|z-i|$ mientras que la distancia a la directriz sería $1+Re(z)$

Igualando tenemos la ecuación buscada
\[|z-i|=Re(z)+1\]

\spart
Recordemos que, por definición, la elipse es el conjunto de puntos del plano con suma de distancias a los focos constante.

Conocemos los focos lo que nos lleva a:
\[|z-1|+|z+1|=cte\]

Para determinar la constante nos basamos en que pasa por $i$, lo que nos lleva a concluir que la constante es $2\sqrt{2}$ es decir, nos queda la ecuación
\[|z-1|+|z+1|=2\sqrt{2}\]

\spart
La hipérbola tenía definición similar a la de la elipse salvo que en este caso considerábamos constante la diferencia de distancias a los pocos en lugar de la suma.

De aquí obtenemos que la ecuación buscada será de la forma
\[|z-1|-|z+1|=cte\]
Sabiendo que pasa por el punot $i+1$ podemos calcular la constante
%TODO completar

\end{problem}

\begin{problem}[18]
Esboce el conjunto de puntos $z \in \cplex$ que satisfacen
\ppart \[Re\left( \frac{z}{1+i}\right) = 0\]
\ppart \[|z^2-4z+4| = 4\]
\ppart \[|z^2-2z-1|=1\]

\solution

\spart
Vamos a jugar un poco con el número que nos dan. Siendo $z=x+iy$ tenemos
\[\frac{x+iz}{1+i}\cdot\frac{1-i}{1-i} = \frac{x-y+i(y-x)}{2} \implies Re\left( \frac{z}{1+i}\right) = \frac{x-y}{2}\]

Por tanto, obtenemos la recta $y=x$, la bisectriz del primer cuadrante.

\spart
\[z^2-4z+4| = 4 \iff |z-2|^2 = 4 \]
con lo que tenemos una circunferencia

\spart
Se deja como ejercicio para el lector, que deberá pasar a cordenadas polares con el objetivo de poder esbozar el dibujo pedido.

\textbf{consejo:} Acordarnos de la Lemniscata
\end{problem}

\begin{problem}[19]
\ppart Sea $a \in \cplex$ un número fijo. Encuentre el máximo de $|z^{12}-a|$ cuando $z$ es cualquier número complejo tal que $|z|\leq 1$
\ppart Halle razonadamente el supremo y el ínfimo del siguiente conjunto de números reales
\[\{Re(iz^4+1) \tq |z|<\sqrt{2}\}\]

\solution
\textcolor{blue}{Hecho por mi. No fiarse al 100\%}

\spart
Si abordamos el ejercicio como un problema en $\real^2$, lo que tenemos es que nos dan un punto cualquiera del plano y debemos buscar el punto del círculo unidad que más diste de él.

Para ello basta con unir el punto dado con el centro y prolongar el segmento que los une hasta que corte a la circunferencia.

Es decir, dado el punto $a=r\left(\cos(\theta)+i\sin(\theta)\right)$, el punto $b$ del círculo unidad más alejado de $a$ es $b=1\left(\cos(\theta +2π)+i\sin(\theta +2π)\right)$

(Posiblemente habría que tener cuidado si el punto $a$ no pertenece al primer cuadrante)

Para calcular ahora el número $z$ pedido, basta con tomar z=$\sqrt[12]{b}$

\spart
Siendo $z=x+iy$ tenemos
\[Re(iz^4+1)=x^4+y^4-6x^2y^2+1\]

Como aprendimos a hacer en Cálculo I, tenemos que calcular el gradiende de esa función e igualarlo a 0 y posteriormente estudiar el comportamiento de la función en la frontera del conjunto que estamos estudiando.

Vamos con el gradiente
\[\nabla \left(Re(iz^4+1)\right) = \left(4x^3-12xy^2, 4y^3-12x^2y\right)\]
al igualarlo a 0 tenemos que los puntos extremos son: el origen y los puntos que satisfacen a la vez las ecuaciones:
\[x^2-3y^2=0\]
\[y^2-3x^2=0\]
que viene a no decir nada y a dejarnos igualmente restringidos al origen

Para estudiar el comportamiento de la función en la frontera del conjunto estudiado tenemos
\[|z| = \sqrt{2} \implies x^2+y^2 =2 \implies y=\sqrt{2-x^2} \]
sustituyendo en la fórmula estudiada tenemos:
\[x^4+4+x^4-4x^2-12x^2+6x^4 +1 =0\]
derivando y simplificando tenemos
\[32x^3-32x=0 \implies x^2-1=0 \implies x=\pm 1 \implies y = \pm \sqrt{3}\]
Los puntos de máximo y mínimo son $(\pm 1, \pm \sqrt{3})$

\end{problem}

\begin{problem}[20]
Describa geométricamente el conjunto de los puntos $w \in \cplex$ que se escriben en la forma $w=iz^2+1$, para $z=x+iy$ con $x>0, y>0, \ x^2+y^2<1$.

\solution

Operando, tenemos que estamos trabajando con el conjunto de números complejos de la forma:
\[w=i(x^2-y^2)-2xy+1\]


\end{problem}

\begin{problem}[21]
Demuestre que, dados $a,c \in \cplex$, la condición necesaria y suficiente para que exista $z \in \cplex$ que verifique $|z+a|+|z-a|=2|c|$ es que sea $|a|\leq|c|$

\textbf{Ayuda:} Si λ>0, el conjunto $ \{z \in \cplex \tq |z+a|+|z-a|=2λ\}$ es una elipse si $λ > |a|$, un segmento si $λ=|a|$ y el conjunto vacío si $λ<|a|$

\solution

Basándonos en la indicación dada es obvio que $|a|\leq|c|$ es condición necesaria y suficiente para que podamos hablar de la solución de la ecuación ya que, en caso contrario, tendríamos el vacío.

Si tenemos $|a|\leq|c|$ el conjunto de puntos solución de la ecuación constituirán una recta o una elipse (según el caso) pero en ambos casos son conjuntos válidos que nos dan solución para la ecuación.
\end{problem}

\begin{problem}[22]
He aquí algunas interpretaciones geométricas de ciertas operaciones con números complejos.

\ppart Si $z=x+iy \in \cplex$ sea $α(z)$ el vector de tres dimensiones $(x,y,0)$. Verifique que para cada $z,w \in \cplex$ se cumple que $α(z)α(w)=Re(z\bar{w})$ y $α(z)\times α(w)=(0,0,Im(\bar{z}w))$

\ppart Si $0,z,w$ son los vértices de un triángulo $T$, compruebe que $Area(T)=\frac{1}{2}|Im(\bar{z}w)|$

\ppart
Si $z_1, z_2,...z_n$ son los vértices de un polígono $P$ que contiene a 0 en su interior, demuestra que $Area(P)=\frac{1}{2}\left|Im\left( \sum_{j=1}^n \bar{z}_jz_{j+1}\right)\right|$, donde se toma $z_{n+1}=z_1$

\solution
\textcolor{blue}{Hecho por mi. No fiarse al 100\%}

\spart
\[α(z)α(w)=z_xw_x+z_yw_y\]
Por otro lado
\[Re(zw)=Re\left(z_xw_x+z_yw_y+i(z_xw_y-z_yw_x)\right) = z_xw_x+z_yw_y \]

Si calculamos el producto vectorial que se nos pide, tenemos que
\[α(z)\times α(w) = (0,0,-z_xw_y+z_yw_x)\]
que podemos comprobar que coincide con la parte imaginaria de
\[Im(\bar{z}w) = Im \left( z_xw_x-z_yw_y+i(-z_xw_y+z_yw_x)\right)\]

\spart

Si ya hemos visto que el producto vectorial coincide con la parte imaginaria, es trivial ver que un medio de esa parte imaginaria nos dará el área del triángulo, pues el producto vectorial nos da el área del paralelogramo generado por los dos vectores.

\spart
Con imagen del producto que se nos da (tras multiplicar por 1/2) obtenemos el área del triángulo formado por los dos puntos dados y el origen.

Puesto que el origen se contiene en la figura cuyo área estamos calculando, al hacer esta operación con todos los vértices tenemos el área de la figura.

\end{problem}

\begin{problem}
Demuestre que la condición necesaria y suficiente para que $\{z_1, z_2, z_3\}$ sea el conjunto de los vértices de un triángulo equilátero es que
\[z_1z_2+z_2z_3+z_3z_1=z_1^2+z_2^2+z_3^2\]
\textbf{Ayuda:} Considere el triángulo $\{z_2, z_3,z_1\}$

\solution

\end{problem}


%%%%%%%%%%%%%%%%%%%%%%%%%%%%%%%%%%%%%%%%%%%%%%%%%%%%%%%%%%%%%%%%%%%%%%%%
%%%%%%%%%%%%%%%%%%%%%%%%%%%%%%%%%%%%%%%%%%%%%%%%%%%%%%%%%%%%%%%%%%%%%%%%
%%                                                                    %%
%%                            HOJA 2                                  %%
%%                                                                    %%
%%%%%%%%%%%%%%%%%%%%%%%%%%%%%%%%%%%%%%%%%%%%%%%%%%%%%%%%%%%%%%%%%%%%%%%%
%%%%%%%%%%%%%%%%%%%%%%%%%%%%%%%%%%%%%%%%%%%%%%%%%%%%%%%%%%%%%%%%%%%%%%%%
\newpage
\section{Hoja 2}
\begin{problem}[1]
(\textit{Esfera de Riemann}) Se considera $\widehat{\cplex} = \cplex \cup \{\infty\}$ y se definen los entornos de $\infty$ como aquellos que contienen un conjunto de la forma $\{z \in \cplex \tq |z|>R\}$ para algún $R > 0$

Con estos entornos $z_n \to \infty$ quiere decir que
\[\forall R > 0 \exists N \tq |z_n| > R \ \forall n > N\]

De manera similar se definen $\lim_{z \to b} f(z)= \infty$ y $\lim_{z \to \infty}f(z)=\infty$.

Sean $\mathbb{S}= \{p \in \real^3 : p_1^2+p_2^2+p_3^2\}$ y consideramos la proyección estereográfica:
\[\appl{\pi}{\mathbb{S}}{\widehat{\cplex}}, \pi(p) = \left\{
\begin{array}{lcc}
    \frac{(p_1+ip_2)}{1-p_3} & si & p \neq N = (0,0,1) \\
 \\ \infty & si & p = N
 \end{array} \right.\]

 \ppart
 Compruebe que
 \[\pi^{-1}\left( \frac{2Re(z)}{|z|^2+1}, \frac{2Im(z)}{|z|^2+1}, \frac{|z|^2-1}{|z|^2+1}\right)\]

 \ppart
 Sea $\rho(z,w)$= distancia (en $\real^3$) entre $\pi^{-1}(z)$ y $\pi^{-1}(w)$ para $z,w \in \widehat{\cplex}$. Entonces:
 \[z_n \to z \text{ en } \widehat{\cplex} \implies \rho(z_n,z) \to 0\]

 \ppart
 Demuestre que
 \[\lim_{n \to \infty}\frac{z^n}{n} = \infty \text{ si } |z| > 1\]

\solution
\textcolor{blue}{Hecho por mi. No fiarse al 1000\%}

\spart
Dado un número complejo $c=a+ib$ podemos verlo en el plano como un punto $c=(a,b)$.

Para poder calcular la inversa de la proyección estereográfica debemos trazar la recta que une este punto con el polo norte ($(0,0,1)$) y calcular la intersección de esta recta con la esfera.

La recta nos queda de la forma:
\[ \left\{
\begin{array}{l}
    x = a+ta\\
 \\ y = b+tb \\
 \\ z = -t
 \end{array} \right.\]

Su intersección con la esfera será aquel punto que cumpla la ecuación de la esfera, es decir:
\[x^2+y^2+z^2=1\]
y sustituyendo tenemos:
\[a^2+t^2a^2+a^2t + b^2+t^2b^2+tb^2+t^2 = 1 \iff\]
\[\iff t^2(a^2+b^2+1)+t(a^2+b^2)+a^2+b^2 -1 \implies \]
\[\implies t = \frac{-(a^2+b^2)\pm \sqrt{a^4+b^4 + 2a^2b^2-4(a^2+b^2+1)(a^2+b^2-1)}}{2(a^2+b^2)}=\]
\[=\frac{-a^2-b^2 \pm \sqrt{-3a^4-3b^4-6a^2b^2}}{2a^2+2b^2} = \frac{-(a^2+b^2) \pm \sqrt{-3(a^2+b^2)^2}}{2(a^2+b^2)} = \frac{-1\pm \sqrt{-3}}{2}\]

\textcolor{blue}{Y en algún punto he metido la pata por que no me salen las cuentas. Una vez tenemos la $t$ sustituimos en la ecuación de la recta y lo tenemos.}

\spart

Ya sabemos que si una sucesión de complejos converge a un complejo dado es por que tanto la parte real como la imaginaria lo hacen por separado. Así mismo, eso implica directamente que la sucesión de los módulos converge al módulo del límite.

Una vez visto esto es obvio ver que la implicación dada es correcta.

\spart

La forma más sencilla de ver esto es trabajando sobre la inversa de la proyección estereográfica. Puesto que es un homeomorfismo (ya lo estudiamos en Análisis) sabemos que $π^{-1}$ es continua por lo que la imagen límite de una sucesión es el límite de las imágenes.

Nos llevamos por tanto los puntos $z_n=\frac{z^n}{n}$ a la esfera y vemos que van creciendo en módulo. Cuando el módulo tiende a infinito, $π^{-1}$ tiene al $(0,0,1)$

\textcolor{blue}{Explicación guarrísima. Trataré de mejorarla.}

\end{problem}

\begin{problem}[2]
\ppart
Demuestre que, mediante la proyección estereográfica, las circunferencias sobre la esfera se transforman en circunferencias o rectas del plano. ¿Cuáles son las circunferencias sobre la esfera que se transforman en rectas?

\ppart
¿Qué corresponde en la esfera de Riemann a una familia de rectas paralelas del plano?

\ppart
Halle, en la esfera de Riemann, las imágenes de los conjuntos definidos por las siguientes desigualdades:
\begin{enumerate}
\item $Im(z) > 0$
\item $Re(z) < 1$
\item $|z| < 1$
\item $|z| > 2$
\end{enumerate}

\solution

\textcolor{blue}{La profesora escribió algunas cuentas pero no me han parecido muy útiles ni novedosas. Aquí doy la idea del ejercicio.}

\spart

Si la circunferencia no pasa por el polo norte, al hacer la proyección estereográfica estamos construyendo un cono e intersecando el mismo con el plano por lo que obtendremos una circunferencia.

No obstante, si hacemos esto mismo con una circunferencia que contiene al polo norte, lo que estamos construyendo es un plano e interescando dos planos, por lo que obtendremos una recta.

Se transforman en circunferencias en el plano aquellas en la esfera que no pasan por el polo norte.

\spart
Las circunferencias en la esfera que pasan por el polo norte se convierten en rectas

\spart
\begin{enumerate}
\item $Im(z) > 0$
Del cuarto de la esfera que se encuentra en la zona compleja del plano

\item $Re(z) < 1$
De un cuarto de la esfera.

\item $|z| < 1$
Son los puntos de la esfera que se encuentran en la semiesfera inferior (por debajo del plano complejo).

\item $|z| > 2$
Son los puntos de la esfera que se encuentran a una altura mayor que $\frac{1}{5}$
\end{enumerate}

\end{problem}

\begin{problem}[3]
Decida si las sucesiones $z_n= \left(\frac{1-2i}{3}\right)^n, \ w_n = \left( \frac{3-4i}{5}\right)^n$ tienen límite (finito) o no

\solution
Para que tengan límite necesitamos que su módulo converja y para ello necesitamos que este sea menor que 1.

En este caso tenemos:
\[|z_n| = \frac{\sqrt{5}}{3} \implies \lim_{n \to \infty} |z_n|^n = 0 \implies \lim_{n\to\infty} z_n = 0\]
\newpage
\[|w_n| = \frac{5}{5} \implies \lim_{n \to \infty} |w_n|^n = 1\]

Esto causa que la sucesión no tenga límite, pues tendremos puntos con el mismo módulo pero diferente ángulo por lo que no converge.

\end{problem}

\begin{problem}[4]
Decida razonadamente si las siguientes funciones tienen límite (finito) o no en el punto indicado

\ppart
\[f(x) = \frac{|z|^2}{z} (\text{ para z}\neq0 )\text{ en el punto } z=0\]

\ppart
\[f(z)= \frac{z^3-8i}{z+2i} ( \text{ para z} \neq -2i) \text{ en el punto } z=-2i\]

\solution

Al calcular este timpo de límites debemos seguir el procedimiento que hacíamos con los reales: probamos a sustituir directamente, nos dará indeterminación y jugamos con el número para evitarla.

\spart
\[\lim_{z \to 0} f(z) = \lim_{z \to 0} \frac{|z|^2}{z} = \lim_{z \to 0}\frac{z \bar{z}}{z}=\lim_{z \to 0} \bar{z} = 0\]

\spart

\[\lim_{z \to -2i} f(z) = \lim_{z \to -2i} \frac{z^3-8i}{z+2i} = \lim_{z \to -2i} z^2-21z-4 = -12\]
\end{problem}

\begin{problem}[5]
Demuestre las siguientes afirmaciones
\ppart

Si $P(z)=a_nz^n+\cdots + a_0$ y $Q(z)=b_mz^m+\cdots b_0$ son polinomios con $a_n \neq 0 \neq b_m$ entonces se tiene
\[\lim_{z \to \infty} \frac{P(z)}{Q(z)} = \left\{
\begin{array}{lcc}
    0& si & n < m \\
    \\ \frac{a_n}{b_m} & si & n=m \\
 \\ \infty & si & n > m
 \end{array} \right.\]

\ppart
No existe $\lim_{z \to \infty}e^z$
\solution
\textcolor{blue}{Hecho por mi. No fiarse al 100 \%}

\spart
Tenemos que calcular
\[\lim_{z \to \infty}\frac{a_nz^n+\cdots + a_0}{b_mz^m+\cdots b_0}\]
\begin{itemize}
\item Si $n<m$
\[\lim_{z \to \infty}\frac{a_nz^n+\cdots + a_0}{b_mz^m+\cdots b_0} = \lim_{z \to \infty}\frac{z^n(a_n+\frac{a_{n-1}}{z}\cdots + \frac{a_0}{z^n})}{z^n(b_mz^{m-n}+\cdots \frac{b_0}{z^n}}=\lim_{z \to \infty}\frac{(a_n+\frac{a_{n-1}}{z}\cdots + \frac{a_0}{z^n})}{(b_mz^{m-n}+\cdots \frac{b_0}{z^n})} = \]
\[=\lim_{z \to \infty}\frac{a_0}{b_mz^n} = 0\]

\item
Si $n=m$
\[\lim_{z \to \infty}\frac{a_nz^n+\cdots + a_0}{b_mz^m+\cdots b_0} = \lim_{z \to \infty}\frac{z^n(a_n+\frac{a_{n-1}}{z}\cdots + \frac{a_0}{z^n})}{z^n(b_mz^{m-n}+\cdots \frac{b_0}{z^n}}=\lim_{z \to \infty}\frac{(a_n+\frac{a_{n-1}}{z}\cdots + \frac{a_0}{z^n})}{(b_m+\cdots \frac{b_0}{z^n})} = \]
\[=\lim_{z \to \infty}\frac{a_0}{b_m} = \frac{a_n}{b_m}\]

\item Si $n>m$
\[\lim_{z \to \infty}\frac{a_nz^n+\cdots + a_0}{b_mz^m+\cdots b_0} = \lim_{z \to \infty}\frac{z^m(a_nz^{n-m}+\cdots + \frac{a_0}{z^m})}{z^m(b_m+\cdots \frac{b_0}{z^m})}=\lim_{z \to \infty}\frac{(a_n+\frac{a_{n-1}}{z}\cdots + \frac{a_0}{z^n})}{(b_mz^{m-n}+\cdots \frac{b_0}{z^n})} = \]
\[=\lim_{z \to \infty}\frac{a_0z^n}{b_m} = \infty\]
\end{itemize}

\spart

Debemos fijarnos en que
\[e^z=e^{x+iy}=e^xe^{iy}=e^x\left(\cos(y)+i\sin(y)\right)\]
cuando $z$ tiende a infinito, así lo hacen su parte real y su parte imaginaria.

Podemos observar que el módulo del número complejo aquí representado crece hasta infinito y su argumento oscila constantemente de modo que no tiene límite.

\end{problem}

\begin{problem}[6]
Halle los puntos de continuidad de las funciones:
\ppart
\[f(z)=\left\{
\begin{array}{lcc}
    \frac{z^4-1}{z-i}& si & z \neq i \\
 \\ 4i & si & z=i
 \end{array} \right.\]
\ppart
\[g(z)=\left\{
\begin{array}{lcc}
    z & si & |z| \leq 1 \\
 \\ |z|^2 & si & |z| > 1
 \end{array} \right.\]

 \solution
\spart
El único punto con posibles problemas y que deberíamos estudiar es el $z=i$. Vamos a estudiar cuánto vale el límite en ese punto para ver si la función es continua o no:
\[\lim_{z \to i} \frac{z^4}{z-i}=\lim_{z \to i}\frac{(z^2-1)(z-i)(z+i)}{z-i} = \lim_{z \to i} (z^2-1)(z+i) = -4i\]

Al no coincidir con el valor de la función en ese punto, podemos concluir que la función no es continua en ese punto.

\spart
El único lugar donde podemos tener problemas es en los puntos con $[z|=1$.

Para hacernos una idea de lo que podemos esperar de este límite, vamos a observar el caso concreto de $z=e^{iα}$ vemos que
\[\lim_{z \to e^{iα}} g(z) =\left\{
\begin{array}{lcc}
    e^iα & si & |z| \leq 1 \\
 \\ 1 & si & |z| > 1
 \end{array} \right. \]

 Por lo general, vemos que esta función no es continua en los puntos con módulo igual a 1 salvo en el punto $z=1$.

 En general es bastante sencillo ver que esta función no es continua, puesto que todos los puntos del círculo unidad se quedan fijos y los demás van a la recta real según su módulo.

\end{problem}


\begin{problem}[7]
¿Dónde son holomorfas las siguientes funciones?
\ppart $f(x,y)=x^2-y^2+ixy$
\ppart $f(z)=g(\bar{z})$, donde $g$ es holomorfa en $\Omega$
\ppart $f(z)=\overline{g(z)}$, donde $g$ es holomorfa en $\Omega$
\ppart $f(z)=\overline{g(\bar{z})}$, donde $g$ es holomorfa en $\Omega$
\ppart $f(z)=|g(z)|$, donde $g$ es holomorfa en $\Omega$

\textbf{Ayuda:} en los apartados b)-e) basta con usar la definición de derivada.

\solution

Para ver si son holomorfas las funciones, vamos a comprobar si se cumplen las ecuaciones de Cauchy-Riemann:
\[\partial_x f = -i \partial_y f\]
\spart
\[2x+iy=-i\left( -2y +ix\right) \iff 2x+iy = -2yi+x \iff (x,y)=(0,0)\]
Con esto no nos basta para garantizar que la función sea holomorfa en ese punto pero, pueso que tanto la parte real como la imaginaria de $f$ son diferenciables ya sí podemos garantizar que la función es holomorfa en el origen.

\textit{En algunos libros podremos ver que esta función no es coniderada holomorfa, puesto que sólo cumple la propiedad en un único punto y, tal y como se hace en variable real, una función sería diferenciable en un entorno del punto, no en un único punto. En general no nos encontraremos con este tipo de funciones en este curso.}

\spart
Tanto en este apartado como en el \textbf{d)}, para que estén bien definidas las $f(z)$ necesitamos que si un punto $z\in Ω$ entonces $z \in Ω$, es decir, el conjunto es simétrico con respecto al eje imaginario.

Vamos a considerar $g=u+iv$ y $f=U+iV$ con
\[U(x,y)=u(x,-y)\]
\[V(x,y)=v(x,-y)\]

%TODO completar estas cuentas por que me he liado mientras copiaba
Una vez visto esto, podemos derivar:
\[U_x = u_x; \;\;\; U_y=-u_y; \;\;\; V_x=v_x; \; \; \; V_y=-v_y;\]
Ahora estamos en condiciones de comprobar si se satisfacen las ecuaciones de Cauchy-Riemann:
\[U_x+iV_x=-i\left(U_y+iV_y\right) \iff u_x = -v_y\; \& \; v_x=u_y\]
pero, por ser $g$ Cauchy-Riemann sabemos que cumple las ecuaciones:
\[u_x+iv_x=-i(u_y+iv_y) \implies u_x=v_y\\; \& \; v_x=-u_y\]

y ambas condiciones sólo se darán en caso de que todas las derivadas sean iguales a 0.

\textbf{Aplicando la definición de derivada}
\[g \text{ holomorfa en }Ω \iff \frac{\partial g}{\partial \bar{w}}(w)=0\ \forall w \in Ω \iff \frac{\partial g}{\partial z}(\bar{z}))0 \ \forall z \in Ω \iff \frac{\partial f}{\partial z}(z)=0 \ \forall z \in Ω\]

Es decir, nos queda que $g$ es holomorfa en Ω si y sólo si $f$ es anti-holomorfa en Ω

\spart
\spart
Basta con ver que
\[\frac{\partial f}{\partial\bar{z}}=\overline{\frac{\partial g}{\partial \bar{z}}}\]
y que si una es holomorfa también lo es la otra.

Ahora nos queda comprobar que es cierta esta fórmula.
\spart

\end{problem}


\begin{problem}[8]
¿Dónde son holomorfas las siguientes funciones? ¿Cuál es su derivada?
\ppart
\[z + \frac{1}{z}\]

\solution

\end{problem}

\begin{problem}[9]
Sea $\appl{T}{\real^2}{\real^2}$ dada por
\[T(x,y) = \left(u(x,y), v(x,y)\right)\]
Definimos la derivada de $T$ en la dirección $\overrightarrow{w}=(a,b)$ como:

\[D_{\overrightarrow{w}} T = \lim_{t \to 0} \frac{T(x+ta, y+tb) - T(x,y)}{t}\]
Observe que
\[D_{\overrightarrow{w}} T = \left(D_{\overrightarrow{w}} u, D_{\overrightarrow{w}} v\right)\]

Dada la función compljea $f(z)=u(x,y)+iv(xy), \ z=x+iy$, demuestre que si $f$ es holomorfa, entonces
\[D_{\overrightarrow{w}} f = f'(z)w \text{ donde } w = a+ib\]

\solution

\end{problem}

\begin{problem}[10]
Sea $f$ una función holomorfa en un dominio $Ω \subset \cplex$. Demuestre que si $|f|$ es constante en Ω, entonces $f$ es constante.

\solution

\end{problem}

\begin{problem}[11]
Demuestre las siguientes afirmaciones:
\ppart Si $h$ es una fucnión de $\real^2$ en $\real$ de clase $\algb{C}^2$ y $f$ es holomorfa, entonces $\nabla(h\circ f) = (\nabla h \circ f)|f'|^2$

\ppart Si $f$ es holomorfa en un dominio $Ω \subset \cplex$ y $f(z) \neq 0 \forall z \in Ω$, entonces
\[\nabla(|f|)=\frac{|f'|^2}{|f|}\]

\ppart Si $f,g$ son holomorfas en un dominio Ω, y si $|f|+|g|$ es constante en Ω y $f$ y $g$ no se anulan en Ω, entonces $f$ y $g$ son constantes.

\solution

\end{problem}

\begin{problem}[13]
Supongamos que los radios de convergencia de las series $\sum_{n=0}^{\infty} a_nz^n$ y $\sum_{n=0}^{\infty}b_nz^n$ son iguales a $r_1$ y $r_2$ respectivamente. ¿Qué se puede decir respecto a los radios de convergencia de las series:
\ppart
\[\sum_{n=0}^{\infty} (a_n\pm b_n)z^n\]

\ppart
\[\sum_{n=0}^{\infty}a_nb_nz^n\]

\ppart
\[\sum_{n=0}^{\infty}\frac{a_n}{b_n}z^n\]

\solution

\end{problem}

\begin{problem}[14]
Pruebe que para todo $z \in \cplex$ tal que $|z|<1$, se verifican las identidades:
\ppart
\[\frac{1}{1-z}=\sum_{n=0}^{\infty}z^n\]
\ppart
\[\left(\frac{1}{1-z}\right)^n = \sum_{n=0}^{\infty} nz^{n-1}\]
\solution
\end{problem}

\begin{problem}[15]
Desarrole las siguientes funciones en series de potencias del tipo indicado
\ppart
\[\frac{z}{z^2-5z+6} \text{ y } \frac{z}{(z-1)^2} \text{ en potencias de } z\]
\ppart
\[\frac{2z+3}{z+1} \text{ y } \frac{2z+3}{(z+1)^2} \text{ en potencias de } z-1\]
\solution
\end{problem}

\begin{problem}[16]
Calcule el radio de convergencia y la suma de
\ppart
\[\sum_{n=0}^{\infty}\frac{z^{2n}}{n!}\]
\ppart
\[\sum_{n=0}^{\infty}n(m-1)z^n\]
\ppart
\[\sum_{n=0}^{\infty}(-1)^n \frac{(z-2πi)^n}{n!}\]
\solution
\end{problem}

\begin{problem}[17]
Si $f(z)=\sum_{n=0}^{\infty}a_nz^n$ ¿qué representa $\sum_{n=1}^{\infty}n^2a_nz^n$ en términos de $f$?
\solution
\end{problem}

\begin{problem}[18]
¿Para qué valores de $z$ convergen las siguientes series?
\ppart
\[\sum_{n=0}^{\infty} \left( \frac{z}{1+z}\right)^n\]

\ppart
\[\sum_{n=0}^{\infty} ne^{-nz}\]

\ppart
\[\sum_{n=0}^{\infty} \frac{\sin(nz)}{n^2}\]

\ppart
\[\sum_{n=0}^{\infty} \frac{\sin(nz)}{2^n}\]

\ppart
\[\sum_{n=0}^{\infty} \frac{z^n}{1+z^2n}\]

\solution
\end{problem}
\printindex
\end{document}
