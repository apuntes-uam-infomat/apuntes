\documentclass[paper=a4, fontsize=11pt]{scrartcl}

\usepackage[T1]{fontenc}
\usepackage{fourier}
\usepackage[spanish]{babel}
\usepackage[utf8]{inputenc}
\usepackage{amsmath,amsfonts,amsthm}
\usepackage{graphicx}
\usepackage{hyperref}

\usepackage{sectsty}
\usepackage{xcolor}
\definecolor{sectioncolor}{rgb}{0.2, 0.2, 0.6}
\allsectionsfont{\color{sectioncolor}\centering \normalfont\scshape}

\numberwithin{equation}{section}
\numberwithin{figure}{section}
\numberwithin{table}{section}

\setlength\parindent{0pt}

\newcommand{\horrule}[1]{\rule{\linewidth}{#1}}

\setcounter{tocdepth}{1}


\hypersetup{
    colorlinks=true,       % false: boxed links; true: colored links
    linkcolor=sectioncolor,          % color of internal links (change box color with linkbordercolor)
    citecolor=sectioncolor,        % color of links to bibliography
    filecolor=sectioncolor,      % color of file links
    urlcolor=sectioncolor           % color of external links
}



\title{
\normalfont \normalsize
\textsc{Curso 2014/2015} \\ [25pt]
\horrule{2pt} \\[0.5cm]
\huge Variable Compleja I\\
\horrule{2pt} \\[0.5cm]
}

\author{Guillermo Ruiz Álvarez}

\date{\normalsize\today}

\AtBeginDocument{
  \addtocontents{toc}{\small}
}

\begin{document}

\maketitle

Me he dedicado a ir escribiendo cosas de Variable Compleja a medida que he ido leyendo apuntes y haciendo los ejercicios. Aquí hay cero formalismos así que no me peguéis si os chirría algo. Para comprender la asignatura mirad los apuntes de Pedro. Esto es para repasar y tener una visión general de todo (teoremas, resultados, cosas en general). Seguramente haya mil errores, así que sentíos libres de añadir/quitar lo que queráis. Si veis que falta algo y no os apetece cambiar el \texttt{.tex} decidlo a cualquiera nosotros.

Como son veinte páginas no me he atrevido a titularlo resumen.
\pagenumbering{gobble}
\begin{flushright}
\textit{Con hamor}

\textit{Rual}
\end{flushright}

\tableofcontents

\newpage
\pagenumbering{arabic}

\input{tex/VC_Resumen.tex}

\end{document}
