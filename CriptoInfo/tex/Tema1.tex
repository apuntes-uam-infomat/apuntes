\section{Introducción criptografía y seguridad informática}
\subsection{ Definiciones básicas}

 \begin{defn}[Criptografía] Proviene de la palabra griega "oculto". Escribir de forma enigmática.
 \end{defn}
 
 
 \begin{defn}[Texto plano]
 	Texto original.El mensaje que se quiere cifrar está escrito en texto plano
 \end{defn}
 
 
 \begin{defn}[Texto cifrado]
	DIBUJO
 \end{defn}
 
 \begin{defn}[Estenografía]
 	Técnica para esconder mensajes dentro de otro mensaje.
 \end{defn}
 
 
 \subsection{Contexto histórico de la criptografía}
 
 \subsubsection{Principales motores de la criptografía}
 
 \begin{itemize}
 	\item Político\item Bélico \item Económico
 \end{itemize}
 
\subsection{Esquema general de cifrado y servicios proporcionados a la seguridad}

\textbf{Notación:}

\begin{itemize}
	\item $Z_m \rightarrow Z$ mod $m$
	\item $P \rightarrow$ conjunto de textos planos
	\item $C \rightarrow$ conjunto dde textos encriptados
	\item $K \rightarrow$ claves
	\item $E_k(x) \rightarrow$ función de cifrado
	\item $D_k(x) \rightarrow$ función de descifrado
\end{itemize}
\subsubsection{Cifrado de desplazamiento}

\begin{itemize}
	\item $P\in Z_m$
	\item $C\in Z_m$
	\item $K\in Z_m$
	\item $y=E_k(x)= x+k$ mod m, con $x\in Z_m$ y $K \in Z_m$
	\item $D_k(y) = x- k$ mod m
	
\end{itemize}

La complejidad de descifrado depende de $|K|$, en este caso
$$|K| = |Z_m| = m$$
Vemos que $E_k$ y $D_k$ son muy rápidas, por lo que se pueden ejecutar en tiempo real, pero a la vez es muy fácil de descifrar.

\subsubsection{Cifrado de sustitución}

\begin{itemize}
		\item $P\in Z_m$
		\item $C\in Z_m$
		\item $K \rightarrow$ permutación de símbolos
		\item $y=E_k(x)= \Pi(x)$ ($\Pi \rightarrow$ permutación)
		\item $D_k(y) = \Pi^{-1}(y)$
\end{itemize}

Es importante ver que en $Pi$ no puede haber índices repetidos.
\begin{example}
	
	
	\textbf{\\ \\ Bien:}
	
	$\begin{matrix}
		1 & 2 & 3 & 4\\
		2 & 1 & 3 & 4
	\end{matrix}$
	
	\textbf{\\ Mal:}
	
		$\begin{matrix}
		1 & 2 & 3 & 4\\
		1 & 1 & 3 & 4
		\end{matrix}$
	
	
\end{example}

Este cifrado es más complejo ya que $|K| = m!$

\subsubsection{Cifrado afín}

\begin{itemize}
	\item $P\in Z_m$
	\item $C\in Z_m$
	\item $K \in (a,b)$
	\item $y=E_k(x)= ax + b$ mod $m , a\in Z^*_m$ y $b\in Z_m$
	\item $D_k(y) = (y-b)* a^{-1}$ mod $m$
\end{itemize}

Decimos que $ a\in Z^*_m$ ya que si $ a\in Z_m$, entonces $a^{-1}$ no tiene porqué existir. Por esto definimos
$$Z^*_m = \{a\in Z_m | mcd(m,a) = 1\}$$

Esto nos asegura que $\forall a \in Z^*_m $ existe $a^{-1}$

Para saber que $a$ queremos utilizar, tenemos que ver:
\begin{enumerate}
	\item mod(a,m) = 1. Para comprobar esto utilizamos el algoritmo de Euclides.
	
	\item ¿Cómo hallamos $a^{-1}$? . Para esto utilizaremos el algoritmo de Euclides extendido.
\end{enumerate}

\begin{enumerate}
	\item \textbf{mcd(a,m) = mcd(m, a mod m)}\\
	
	
	\underline{EUCLIDES}
	
	Llamamos $r_0 = a$ y $r_1 = m$
	
	$r_0 = q_1 r_1 + r_2$
	
	$r_1 = q_2 r_2 + r_3$
	
	$\dots$
	
	$r_{n-2} = q_{n-1}r_{n-1} + r_n$
	
	$r_{n-1} = q_{n}r_{n}$
	
	Y asi nos queda que
	$$r_n = mcd (a,m)$$
	
	\item \textbf{¿ $a^{-1}$?}
	
	\underline{EUCLIDES EXTENDIDO}
	
	La idea es escribir el algoritmo de Euclides pero escribiendo todos los restos en función de $a$ y $m$.
	
	Llamamos $r_0 = a$ y $r_1 = m$
	\begin{itemize}
		\item $r_0 = q_1 r_1 + r_2 \rightarrow r_2 = r_0 - q_1 r_1 \rightarrow r_2 = a-q_1m$
		
		\item $r_1 = q_2 r_2 + r_3 \rightarrow r_3 = m - q_2 r_2$ , y como $r_2$ está e función de $a$ y $m \rightarrow r_3 = m(1+q_1q_2) - q_2a$
		
		\item $r_4 = r_2 - q_3 r_3$
		
		$\dots$
		
		\item $r_{n} = m\cdot U + a\cdot V \Rightarrow a \cdot V = 1$ mod $m \Rightarrow a^{-1} = V$
		
	\end{itemize}
	
	Ejercicio: ¿Cómo calculamos esa $V$?
	
\end{enumerate}

\subsection{Tipos de ataques}
\subsection{Modelos y estándares de seguridad informática , auditoría y certificación}

