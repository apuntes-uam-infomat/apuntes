\documentclass{apuntes}

\title{Topología I}
\author{Guillermo Julián Moreno \\ Cristina Kasner Tourné}
\date{14/15 C1}
% Paquetes adicionales
\usepackage{tikztools}
\usepackage{fastbuild}

\usetikzlibrary{arrows}
% --------------------

\precompileTikz

\begin{document}
\pagestyle{plain}
\maketitle

\tableofcontents
\newpage

\chapter{Conceptos básicos}

\section{Introducción}

En Topología buscamos extender conceptos importantes como continuidad o convergencia. Si partimos del concepto de continuidad en los reales, teníamos que

\begin{defn}[Continuidad] Dada $\appl{f}{(a,b)}{ℝ}$, se dice que es continua en $x_0 ∈ (a,b)$ si $∀ ε > 0 \; ∃δ>0 $ tal que $\abs{x-x_0} < δ \implies \abs{f(x) - f(x_0)} < ε$.
\end{defn}

¿Cómo podemos extender esto a conjuntos que no sean $ℝ$? Lo primero es que necesitamos una distancia. Y la propiedad central de la distancia debería ser \[ \abs{x+y} ≤ \abs{x} + \abs{y} \]. Esta propiedad es la desigualdad triangular, y es ciertamente natural. La extensión de la distancia la tendremos en los espacios métricos.

\begin{defn}[Espacio\IS métrico]
Un espacio métrico es un par $(X, d)$, con $X$ un conjunto y $d$ una aplicación $\appl{d}{X×X}[0, ∞)$ tal que 

\begin{enumerate}
\item $\dst(y,y) = 0\;∀y∈X$.
\item $\dst(x,y) ≥ 0\;∀x,y∈X$.
\item $\dst(x,y) = 0 \dimplies x=y$.
\item \concept{Desigualdad\IS triangular}: $\dst(x,z) = \dst(x,y) + d(y,z)\; ∀x,y,z∈X$.
\end{enumerate}
\end{defn}

Tenemos varios ejemplos de distancias:

Por ejemplo, en $ℝ^m$, tenemos $\dst (x,y) = \md{\vx-\vy} = \sqrt{(x_1-y_1)^2 + \dotsb + (x_m-y_m)^2}$.


Si consideramos el conjunto de funciones continuas $C([0,1]) \equiv \{ \appl{f}{[0,1]}{ℝ}, \text{f continua} \}$, $\md{f} ≝ \max_{x∈[0,1]} \abs{f(x)}$. Con esta noción, el conjunto de funciones continuas se comporta de forma similar a $ℝ^m$ con dimensión infinita, y podemos hacer cosas parecidas a las del espacio euclídeo.

Así, podemos definir $\dst(f,g) ≝ \md{f-g}$, y llegar a una definición de convergencia uniforme: $f_n \to f$ en esa distancia implica una convergencia uniforme en $[0,1]$.

Podemos definir una distancia algo artificial. Sea $X$ un conjunto cualquiera, definimos
\[ \dst(x,y) ≝ \begin{cases}
0 & \text{si}\; x = y \\
1 & \text{si}\; x ≠ y \\
\end{cases} \] 
que claramente cumple las condiciones para ser una distancia.

\begin{defn}[Bola] Dado $(X,\dst)$ un espacio métrico, con $x∈X$ y $r∈(0,∞)$, definimos la bola $\bola$ centrada en $x$ de radio $r$ como 

\[ \bola(x,r) ≝ \{ y∈X \tq \dst(x,y) < r \} \]

En ocasiones querremos especificar la distancia ($\bola_{\dst}$) o el conjunto ($\bola_X$) con el subíndice.
\end{defn}

Las bolas tienen ciertas propiedades muy sencillas. Dados \sdst, $x∈X$, $r>0$, $y∈\bola(x,r)$ entonces $\bola(y,r-\dst(x,y)) ⊆ \bola(x,r)$.

\begin{wrapfigure}{r}{0.4\textwidth}
\inputtikz{I_BolaContenida}
\caption{La bola verde ($\bola(y, r-\dst(x,y)$) contenida dentro de $\bola(x, r)$.}
\label{figBolaContenida}
\end{wrapfigure}

Esto se puede demostrar con un dibujo (\ref{figBolaContenida}), pero tenemos que demostrarlo más formalmente: 

\begin{proof}
$∀z∈\bola(x, r-\dst(x,y))$ tenemos que $\dst(x,z) ≤ \dst(x,y) + \dst(y,z) < \dst(x,y) + r - \dst(x,y) = r$, y por lo tanto $z ∈ \bola(x,r)$.
\end{proof}

Por supuesto, el dibujo es una guía. Si tomásemos la distancia rara de antes que sólo tomaba valores 1 ó 0, la bola no sería una bola como en $ℝ$.

Vamos a definir ahora el cierre, aunque sólo como notación:

\begin{defn}[Cierre] Dado \sdst espacio métrico, $x∈X$, $r≥0$, definimos \[ \overline{\bola}(x,r) ≝ \{ y∈ X\tq \dst(x,y) ≤ r\} \] como la bola cerrada de centro $x$ y radio $r$.\end{defn}

\begin{defn}[Conjunto\IS abierto] Sea \sdst un espacio métrico. Entonces damos dos definiciones

\begin{enumerate}
\item $A⊆X$ es abierto en \sdst si $∀x∈A\; ∃δ=δ_x > 0$ tal que $\bola(x,δ_x) ⊆ A$.
\item La familia de abiertos es $\topl_d \equiv \{ A ⊆ X \tq A\, \text{abierto} \}$.
\end{enumerate}
\end{defn}

La familia de abiertos que acabamos de definir es una \concept[Topología]{topología}, y cumple las siguientes propiedades.

\begin{enumerate}
\item $\emptyset, X ∈ \topl_d$.
\item $A,B ∈ \topl_d \implies A \cap B ∈ \topl_d$.
\item $A_j ∈ \topl_d\; ∀j∈ J \implies \bigcup_{j∈J} A_j ∈ \topl_d$
\end{enumerate}

Demostremos las dos últimas propiedades:

\begin{proof} \paragraph{Prop. 2} Sea $x∈A\cap B$. Entonces $x∈A$ y $x∈B$, luego existen $δ_x^A, δ_x^B$ tales que $\bola(x, δ_x^A) ⊆ A$ y $\bola(x, δ_x^B) ⊆ B$ respectivamente. Sea ahora $δ=\min(δ_x^A, δ_x^B)$. Entonces $\bola(x,δ) ⊆ A\cup B$.

\paragraph{Prop. 3} La propiedad es equivalente a la pregunta de, si dado $x∈\bigcup_{j∈J}$, se cumple que $∃δ > 0 \tq \bola(x,δ)⊆\bigcup_{j∈J} A_j$.

Es obvio que $∃j_x ∈ J\tq x∈ A_{j_x}$, luego \[ ∃ δ > 0 \tq \bola(x,δ) ⊆ A_{j_x} ⊆ \bigcup_{j∈J} A_j \]
\end{proof}

Por otra parte, también podemos hacer una observación: por inducción, la intersección de una familia \textit{finita} de conjuntos también es un abierto.

\subsection{Topologías}

Una vez hecho esto, ya podemos pasar a definir qué es un espacio topológico y una topología.

\begin{defn}[Topología]\label{defTopología}
Sea $X$ un conjunto. Entonces una familia $\topl$ de subconjuntos de $X$ es una topología en $X$ si y sólo si cumple las tres propiedades que acabamos de ver:

\begin{enumerate}
\item $\emptyset, X ∈ \topl$.
\item $A,B ∈ \topl \implies A \cap B ∈ \topl$.
\item $A_j ∈ \topl\; ∀j∈ J \implies \bigcup_{j∈J} A_j ∈ \topl$
\end{enumerate}
\end{defn}

\begin{defn}[Espacio\IS topológico] Un espacio topológico es un par $(X, \topl)$ donde $\topl$ es una topología en $X$.
\end{defn}

Los elementos de \topl son los \textit{abiertos} de la topología. 

También podemos definir el conjunto cerrado:

\begin{defn}[Conjunto\IS cerrado]
Dado un espacio topológico \stopl, $F⊆X$ es cerrado si y sólo si $F^C \equiv X \setminus F$ es abierto.
\end{defn}

Podemos definir dos topologías "comunes", por así decirlo, las obvias para cualquier conjunto. Tenemos la \concept[Topología!trivial]{topología\IS trivial} (el mínimo) dada por \[ \topl_{triv.} = \{ \emptyset, X \} \], y la \concept[Topología!discreta]{topología\IS discreta}, que sería el máximo: \[ \topl_{disc.} = \parts{X} \].

Y volviendo a nuestro bonito mundo de los reales, tenemos las \concept[Topología\IS usual]{topologías usuales} en $ℝ^m$ o $\topl_{ℝ^m}$. Para $m=1$, diremos que \[ A ∈ \topl_ℝ ≝ ∀x∈A\; ∃a,b∈ℝ \tq x∈ (a,b) ⊆ A \]. Esto es, que siempre podemos encontrar un intervalo contenido en $A$ que a su vez contenga a $x$. Equivalentemente, $A$ será una unión de intervalos abiertos.

Para dimensión $m>1$, definimos su topología usual como \[	ℝ^m ⊇ A ∈ \topl_{R^m} ≝ ∀x∈A\; ∃ \begin{matrix} a_1, \dotsc, a_m \\ b_1, \dotsc, b_m \end{matrix} ∈ ℝ \] tales que $ x∈ (a_1, b_1) × \dotsb × (a_m, b_m) ⊆ A$.


\subsubsection{Topologías metrizables}

\begin{defn}[Espacio\IS topológico metrizable] Dado \stopl un espacio topológico, se dice que es metrizable si existe una distancia $\dst$ en $X$ tal que $\topl = \topl_{\dst}$. 

$\dst$ no es necesariamente única.
\end{defn}

Por ejemplo, $\topl_ℝ$ es metrizable. Coincide $\topl= \topl_{\dst}$ con $\dst(x,y) = \abs{x- y}$.

Expandiendo un poco más sobre lo que significa que una topología coincide con otra, o lo que significa que una topología sea \textit{sea inducida} por una aplicación. 

\begin{defn}[Topología\IS inducida] Dado un espacio topológico \stopl, entonces la topología inducida por una función $f$ es \[ \topl_f = \{ \inv{f} (A) \tq A ∈ \topl \} \]
\end{defn} 

¿Qué significa entonces que una topología sea igual a otra? Si nos remitimos a la definición de topología (\ref{defTopología}), vemos que es un conjunto de subconjuntos de $X$. Luego dos topologías son iguales o equivalentes si y sólo si tienen los mismos elementos. Es decir, que si un conjunto es abierto en $X$ según una topología, también lo es según la otra y viceversa.

Volviendo al caso concreto, una topología inducida por la distancia es igual a otra topología si los abiertos según la distancia (esto es, las bolas) son también abiertos según la otra topología que estemos considerando.

Hagamos algunos ejemplos sobre topologías son metrizables. ¿Son $\topl_{disc.}, \topl_{triv.}$ metrizables para $X$ un conjunto cualquiera?

En el caso de $\topl_{disc}$ sí lo es. Definimos la distancia como \[ \dst (x,y) = \begin{cases} 0 & x = y \\ 1 & x ≠ y \end{cases} \], luego $\bola(x, 1/2) = \{ x \}$, luego $\{ x \}$ es abierto en $\topl_{\dst}$. Entonces, si $A ⊆ X$, entonces $A= \bigcup_{x∈A} \{ x \}$ es abierto también.

La topología trivial es más interesante de estudiar. Si $\card{X}≥2$, entonces $\topl_{triv} ≠ \topl_{\dst}$ para cualquier distancia $\dst(x,y)$. ¿Por qué?

En la topología trivial sólo hay dos abiertos (vacío y total). Sin embargo, en la topología inducida por la distancia, los abiertos son las bolas. 

Si hay más de dos elementos en $X$, existen $x,y∈X$ distintos, y por lo tanto existe una distancia $r=\dst(x,y) > 0$. Con $δ=\frac{r}{2}$, las bolas $\bola(x,δ), \bola(y,δ)$ son distintas y disjuntas. Ninguna de ellas es el vacío y el total así que no son abiertos en $\topl_{triv}$, pero sí que son abiertos en $\topl_{\dst}$. Por lo tanto, tenemos que $\topl_{\dst} = \topl_{triv}$.

\subsubsection{Topologías generadas por una base}

Además de por la distancia, podemos considerar las \textbf{topologías generadas por una base}.

Recordemos cómo definíamos una topología en $\topl_ℝ$. Decíamos que $A∈\topl_ℝ$ si y sólo si $∀x∈A\; ∃(a,b)$ tales que $x∈(a,b) ⊆ A$. 

\begin{defn}[Base]
Sea $X$ un conjunto y $\base$ una familia de subconjuntos de $X$ (e.d. $\base ⊆ \parts{X}$). Entonces $\base$ es una base si y sólo si 

\begin{enumerate}
\item $∀x∈X\;∃B∈\base \tq x∈B$. Dicho de otra forma, $\bigcup_{B∈\base} B = X$.
\item $∀B_1,B_2∈\base;\, ∀x∈B_1 ∩ B_2$, existe $B_3 ∈ \bola \tq x∈B_3 ⊆ B_1∩B_2$.
\end{enumerate}
\end{defn}


\begin{defn}[Topología\IS generada por una base] La topología $\topl_\base$ se define por \[ A ∈ \topl_\base \iff ∀x ∈ A\; B∈\base \tq x∈B⊆A \]
\end{defn}

Tenemos que demostrar, eso sí, que eso que hemos definido ahí es realmente una topología.

\begin{prop} $\topl_\base$ es una topología en $X$.\end{prop}

\begin{proof} Tenemos que comprobar las tres propiedades de una topología (\ref{defTopología}). Sabemos que $\emptyset ∈ \topl_\base$. Además, tal y como hemos definido la topología generada, también sabemos que $X∈ \topl_\base$.

\paragraph{Propd. 2} Tenemos que demostrar que $A_1, A_2 ∈ \topl_\base \implies A_1∩A_2 ∈ \topl_\base$. Si $x∈ A_1∩A_2$, entonces $∃B_1, B_2 ∈ \base$ tales que $x∈B_1⊆A_1$ y $x∈B_2⊆A_2$ respectivamnete. 

Según la segunda propiedad de la base, existe un $B_3∈\base$ tal que $x∈B_3 ⊆ B_1∩B_2 ⊆A_1∩A_2$, luego $A_1∩A_2 ∈ \topl_\base$.

\paragraph{Propd. 3} Demostramos que $A_j ∈ \topl_\base\; ∀j∈J \implies \bigcup_{j∈J} A_j ∈ \topl_\base$. Si $x∈\bigcup_{j∈J} A_j \implies ∃i=i_x∈J\tq x∈A_i$. Luego como $A_i ∈ \topl_\base$ tenemos que $\exists B∈ \base$ tal que $x∈B ⊆ A_i ⊆  \bigcup A_j$.
\end{proof}

Nos fijamos que en la demostración de la tercera propiedad no hemos usado nada sobre cómo hemos definido la base. Es decir, que siempre que defininamos una topología $\topl$ como \[ A ∈ \topl \iff ∀x∈ A\;∃U ∈ \mathcal{F} \tq x∈ U ⊆ A \], donde $\mathcal{F}$ es una familia de subconjuntos de $X$, la propiedad tercera de la definición de topología \textbf{está garantizada}. Es para la primera y segunda propiedad para las que se necesita que $\mathcal{F}$ cumpla algún tipo de propiedad.

\begin{defn}[Topología\IS fina]
Dado un espacio $X$ y dos topologías $\topl_1, \topl_2$, si $\topl_1⊆\topl_2$ (todo abierto de $\topl_1$ es abierto de $\topl_2$) se dice que $\topl_2$ es \textbf{más fina} que $\topl_1$.
\end{defn}

\begin{prop} Sea $X$ un espacio topológico y $\base$ una base. Entonces 

\begin{enumerate}
\item $\base⊆\topl_\base$ (todos los elementos de $\base$ son abiertos en $\topl_\base$.
\item $A∈\topl_\base$ si y sólo si $A$ es unión de elementos de $\base$.
\end{enumerate}
\end{prop}

\begin{proof}
\paragraph{1)} Recordamos que \[ V ∈ \topl_\base ≝ ∀x∈V \; ∃B=B_x∈\base\tq  x∈B⊆V \]. Sea $M∈\base$, quiero demostrar que $M∈ \topl_\base$. He de comprobar que \[ ∀x ∈ M\; ∃B∈\base \tq x∈ B ⊆ M \], lo cual es obvio si tomamos $B=M$, ya que los elementos de la base son siempre abiertos.

\paragraph{2)} Partiendo de la afirmación de antes, sabemos que si $A ∈ \topl_\base$, entonces $∀x∈A\; ∃B_x∈\base$ tal que $x∈ B_x⊆A$. Como cada uno de esos conjuntos está en $A$, su unión también lo está. Y por otra parte, dado que consideramos todos los puntos $x$ de $A$, nos queda que \[ A = \bigcup_{x∈A}B_x \], demostrando así el primer lado de la implicación.

La implicación a la izquierda se resuelve por la primera parte de esta proposición: si $B_j∈B$, entonces $B_j∈\topl_\base$ y por la tercera propiedad de la topología (\ref{defTopología}), nos queda que \[ \bigcup_{j∈J} B_j ∈ \topl_\base \]

\end{proof}

Ahora que ya sabemos cómo generar una topología a partir de una base, podemos hacernos una pregunta. Consideramos una serie de conjuntos que queremos que sean abiertos en nuestro espacio. Obviamente, la topología discreta cumple lo que buscamos, pero, ¿hay una topología más pequeña? ¿Cuál es la topología \textit{"mínima"}?

\begin{prop} Sea $X$ un conjunto. \label{propTopologiaMinima}

\begin{enumerate}
\item Si $\topl_k$ es una topología en $X$, $∀k∈K$ entonces \[ \topl ≝\bigcap_{k∈K} \topl_k \].

\item Sea $D$ una familia de subconjuntos de $X$ ($D⊆\parts{X}$) y sea \[ \topl_D ≝ \bigcap_{D⊆\topl} \topl \] donde $\topl$ es una topología en $X$. 

Entonces $\topl_D$ es una topología en $X$, $D⊆\topl_D$ y $\topl_D$ es la topología menos fina que cumple $D⊆\topl_D$.
\end{enumerate}
\end{prop}

\begin{proof}
\paragraph{1)} La primera propiedad de la topología es trivial. Vamos con la segunda. Si $V_1V_2 ∈ \topl$, tenemos que $V_1, V_2 ∈ \topl_k\,∀k∈K$. Luego como $\topl_k$ es topología, $V_1∩ V_2 ∈\topl_k\, ∀k∈K$, y entonces $V_1∩V_2 ∈ \bigcap \topl_k = \topl$.

\paragraph{2)} Sabemos que $\topl_D$ es topología por lo que acabamos de demostrar. Ahora bien, ¿es la más pequeña? Es obvio, viendo que es la intersección de todas las topologías que contienen a $D$.\footnote{Relacionado con el ejercicio 9-c.}
\end{proof}

Tenemos que tener cuidado cuando $D$ es una base: hay que asegurarse de que la topología coincida en ese caso.\footnote{Y nos preocupamos nosotros de eso.} 

\subsubsection{Topología del orden}

Hasta ahora hemos visto cómo generar topologías a partir de la distancia, y también tratando de extrapolar el concepto de los intervalos de $ℝ$ con las bases. Ahora vamos a ver cómo hacerlos a través de otra visión de los intervalos como elementos de orden. Recordemos brevemente qué es un orden total: a grandes rasgos es uno donde podemos comparar todos los elementos.

\begin{defn}[Orden\IS total] Dado un conjunto $X$, un orden total en $X$ es una relación $x < y$ tal que 

\begin{enumerate}
\item $x<y, y < z\implies x < z$.
\item $∀x∈X$, $x < x$ es falso.
\item $∀x,y∈X$ con $x≠y$ entonces se cumple una y sólo una de $x< y$ ó $y<x$. 
\end{enumerate}
\end{defn}

Dado un conjunto $X$ y un orden total $<$ se puede construir una topología $\topl_<$ de la misma forma que en $ℝ$: intervalos $(a,b)$. Empecemos con ejemplos.

\paragraph{Orden lexicográfico en $ℝ^2$} Este ejemplo es una topología muy visual (ver la figura \ref{figOrdenLex}), importante y rara. Empezamos definiendo qué es ese orden

\begin{defn}[Orden\IS lexicográfico] Denotamos como $<_{Lex}$ al orden que, dado $x=(x_1,x_2), y=(y_1, y_2)$ ambos en $ℝ^2$, se dice que $x<_{Lex} y$ si $x_1 < y_1$ o bien, si $x_1 = y_1$, entonces $x_2 < y_2$.
\end{defn}

\begin{wrapfigure}{r}{0.4\textwidth}
\inputtikz{I_OrdenLexicografico}
\caption{Ilustración del orden lexicográfico en $ℝ^2$. Cualquier punto en $r_2$ es mayor que todos los de $r_1$. En la misma vertical, tenemos que $a<_{Lex}b$.}
\label{figOrdenLex}
\end{wrapfigure}

A partir de esto podemos definir el intervalo lexicográfico de la forma obvia:

\begin{defn}[Intervalo\IS lexicográfico] Si $a,b∈ℝ^2$ con $a<_{Lex}b$ entonces
\[ (a,b)_{Lex} ≝ \{  x ∈ℝ^2\tq a <_{Lex} x <_{Lex} b \} \]
\end{defn}

\begin{prop} \[ \base_{Lex}=\{ (a,b)_{Lex} \tq a,b∈ℝ^2, a<_{Lex}b \} \] es una base para una topología en $ℝ^2$.\end{prop}

\begin{proof}
Como ejercicio, pero queda claro que si $B_1,B_2∈\base$ entonces $B_1∩B_2∈\base$, lo que es todavía mejor que la definición de base.
\end{proof}

\begin{defn}[Topología\IS lexicográfica en $ℝ^2$] En $ℝ^2$, definimos la topología lexicográfica como 

\[ \topl_{Lex} ≝ \topl_{\base_{Lex}} \]
\end{defn}

Podemos ver un ejemplo, considerando en $[0,1]^2$ el intervalo acotado lexicográficamente por $\mathbbold{0} = (0,0)$ y $\mathbbm{1} = (1,1)$, luego
\[ [0,1] × [0,1] = [\mathbbold{0}, \mathbbm{1}]_{Lex} \] 

En esta topología, el abierto más sencillo que contiene a un punto "en el medio" (por ejemplo, el $(0.5, 0.5)$) sería un intervalo "vertical". Sin embargo, para un punto en el borde superior o inferior, el abierto más sencillo sería un rectángulo que se expande hacia la derecha (ver figura \ref{figIntervalosLex}).

\begin{figure}[hbtp]
\centering
\begin{subfigure}[b]{0.4\textwidth}
\inputtikz{I_OrdenLex_AbiertoVertical}
\caption{En un punto dentro del cuadrado, el abierto más sencillo es un intervalo vertical}
\end{subfigure}
~
\begin{subfigure}[b]{0.4\textwidth}
\inputtikz{I_OrdenLex_AbiertoTop}
\caption{En un punto en un borde del cuadrado, el abierto más sencillo es un rectángulo, el intervalo $(a,b)_{Lex}$.}
\end{subfigure}

\caption{Intervalos más sencillos en $[\mathbbold{0}, \mathbbm{1}]_{Lex}$.}
\label{figIntervalosLex}
\end{figure}

Más ejemplos: ¿una bola $A$ en el sentido habitual (ver figura \ref{figBolaLex}) de $ℝ^2$ es abierto en esta topología? Efectivamente: podemos expresarlo como unión de abiertos de la topología, las líneas verticales. 

\begin{figure}[hbtp]
\centering
\inputtikz{I_OrdenLex_Bola}
\caption{La bola $A$ es abierto en la topología lexicográfica si la expresamos como unión de intervalos verticales}
\label{figBolaLex}
\end{figure}

Si lo expresamos de forma simbólica también llegamos a lo mismo. Tomamos

\[ A = \{ (x,y) \tq \left(x-\frac{1}{2}\right)^2 + \left(y-\frac{1}{2}\right)^2 < \frac{1}{100} \]

Así, tendríamos que 

\[ A = \bigcup (a_x, b_x)_{Lex} \]

con 

\begin{gather*}
a_x = \left(x, \frac{1}{2} - \sqrt{\frac{1}{100} - \left(x-\frac{1}{2}\right)^2}\right) \\
b_x = \left(x, \frac{1}{2} + \sqrt{\frac{1}{100} - \left(x-\frac{1}{2}\right)^2}\right)
\end{gather*}

\subsection{Convergencia}

Ahora sigamos con más definiciones.

\begin{defn}[Entorno\IS abierto] Dado \stopl un espacio topológico y $x∈X$, un entorno abierto de $x$ es un abierto $U∈\topl$ tal que $x∈U$.
\end{defn}

\begin{defn}[Convergencia\IS de sucesiones] Sea \stopl un espacio topológico y $\{x_n\}_{n∈ℕ}$ una sucesión en $X$, y $x∈X$. Se dice que $x_n$ converge a $x$ si todo entorno de $x$ contiene todos los términos de la sucesión a partir de un índice determinado.

Dicho simbólicamente

\[ ∀U∈\topl,\; x∈U\; ∃n_U∈ℕ \tq x_n ∈ U \,∀n≥n_U \]
\end{defn}

\paragraph{Ejercicio} En un espacio métrico \sdst con la topología $\topl_{\dst}$ inducida por la distancia, demuestra que \[ x_n \convs x \iff \dst(x,x_n)\convs 0 \iff ∀ε> 0\, ∃n_ε\tq \dst(x,x_n) < ε \]

\paragraph{Ejemplo 1} Veamos ejemplos de convergencia en topologías raras. Tomemos $ℝ$ con $\topl_{[, )}$, y la sucesión $x_n= \frac{-1}{n}$ para $n≥1$. 

Esta sucesión converge en el sentido usual (euclídeo) a cero, pero no en esta topología. Y es que existe un intervalo que es entorno de $0$ (por ejemplo, $[0, 1)$ que no contiene a ningún punto de la sucesión.

\paragraph{Ejemplo 2} Tomemos la sucesión $x_n=\left(\frac{1}{n}, n\right)$ para $n≥1$ en el espacio topológico $(ℝ^2, \topl_{Lex})$. Converge en la topología usual a $(0,1)$, pero no en la lexicográfica: un entorno vertical del $(0,1)$ no contiene a puntos de la sucesión.


Ahora vamos a ver algunos conceptos en relación a los cerrados, que recordemos eran los complementarios de los abiertos. 

\begin{prop}
\begin{enumerate}
\item $\emptyset, X$ son cerrados.
\item La unión finita de cerrados es cerrado.
\item La intersección de una familia de cerrados es cerrado.
\end{enumerate}
\end{prop}

\paragraph{Ejemplos}  Si tomamos $R_a ≝ \{ (a,y) \tq y ∈ ℝ \}$, es abierto y cerrado en $(ℝ^2, \topl_{Lex})$. 

Es cerrado porque $R_a^c$ es abierto: para todo punto puedo coger un entorno abierto contenido en el conjunto. También podemos verlo como que $R_a^c = \bigcup_{b∈ℝ\setminus \{a\}} R_b$, que es unión de abiertos.

Otro conjunto interesante es $[a,b)$ en $\topl_{[,)}$, que también es abierto (es un elemento de la base) y cerrado (su complementario es $(-∞, a) ∪ [b, ∞)$, ambos abiertos (podemos expresarlos como unión de conjuntos de la base).

Curiosamente, ese conjunto en $(ℝ, \topl_ℝ)$ no es ni abierto ni cerrado: no hay ningún abierto que contenga a $a$ y que esté contenido en $[a,b)$ por lo que no es abierto; y tampoco es cerrado porque $[a,b)^c=(-∞,a) ∪ [b,∞)$ que no es abierto.

\seprule

Vamos a hacer ahora ciertas observaciones sobre convergencia en topologías definidas por una base, para darnos cuenta de los interesantes que pueden resultar al permitirnos probar cosas mirando sólo los elementos de la base.

\begin{prop} Dado un conjunto $X$ y $\base$ una base para una topología en $X$ $\topl_\base$, entonces 

\[ x_n\convs x \iff ∀B∈\base \tq x ∈ B,\; ∃n_B \tq  x_n∈ B ∀ n ≥ n_B \].

Es decir, basta comprobar la definición para entornos de $x$ que son elementos de la base.\end{prop}

\begin{proof}
La implicación a la derecha es trivial: si se cumple para todos los abiertos, se cumple para algunos en particular.

Ahora tenemos que demostrar la implicación a la izquierda. Si $U∈ \topl_\base$ y $x∈U$, entonces por la definición de $\topl_\base$ $∃B∈ \base$ tal que $x∈B⊆U$. Por hipótesis, $∃n_B$ tal que $x_n∈B\; ∀n≥ n_B$, y como $B⊆U$ entonces $x_N∈U \; ∀n≥ n_B$.
\end{proof}

\subsection{Interior, adherencia y frontera de conjuntos}

Sea \stopl un espacio topológico y $W⊆X$ un conjunto cualquiera. Entonces, vamos a definir varios conceptos

\begin{defn}[Interior] Decimos que $x ∈ \mathop{Int}(W)$ si existe un entorno $U$ de $x$ tal que $U⊆W$. Es decir, existe un $U$ abierto tal que $x∈U ⊆ W$.

El interior de un conjunto $W$ se denota como $\intr{W}$.
\end{defn}

Por ejemplo, $\intr{ℚ}$ es vacío tanto en la topología usual como en $\topl_{[,)}$.

El intervalo $[a,b]$ en la topología usual tiene como interior el abierto $(a,b)$. Pero, ¿y en la topología $\topl_{[,)}$? En este caso es: $\intr{[a,b]} = [a,b)$.

Razonando, si $a≤x<b$, entonces $x∈[a,b) ⊆ [a,b]$, luego $x$ está en el interior. Si $x = b$, entonces no existe un intervalo abierto $U$ con $b∈U⊆[a,b]$, porque si $b∈U$ con $U$ abierto entonces existiría un $[α,β)$ con $b∈[α,β) ⊆ U$. En ese caso, el punto medio $\frac{b+β}{2} ∈ [α,β)$, luego entonces también pertenecería a $U$. Sin embargo, es claro que $\frac{b+β}{2} > b$, así que no puede estar en $U$.

\begin{figure}[hbtp]
\centering
\inputtikz{I_ConjuntoWLex}
\caption{Conjunto $W$ en la topología lexicográfica.}
\label{figConjuntoWLex}
\end{figure}

Otro ejemplo: tomemos $([0,1]^2, \topl_{Lex})$. ¿Cuál es el interior del conjunto $W$ que aparece en la figura \ref{figConjuntoWLex}?

Está claro que los puntos de dentro $p_i$ son del interior, ya que siempre podemos encontrar un abierto. También están dentros los puntos $p_b$ de los bordes inferior y superior (con $0≤x<b$), ya que podemos encontrar una banda (como la banda azul) contenida en $W$. Los puntos del lateral izquierdo $p_l$ también están en el interior.

Sin embargo, los puntos $p_d$ en el borde de la diagonal o en el borde inferior con $b ≥ x$ no están en el interior: cualquier abierto que cojamos será una banda como la roja, que se sale de $W$. Tampoco están los puntos del borde lateral derecho.

En definitiva, en la figura \ref{figConjuntoWLex} el interior sería el interior de naranja con los bordes naranjas, excluyendo los marcados en rojo.

\appendix
\chapter{Ejercicios}

\section{Hoja 1}


\begin{problem}[6] Sea $\appl{g}{X}{Y}$ una aplicación entre dos conjuntos.

\ppart Demostrar que si $\topl$ es una topología en $X$ entonces \[ \mathcal{S} = \{ E ⊆ Y \tq \inv{g}(E) ∈ \topl \} \] es una topología en $Y$.
\ppart Demostrar que si $\mathcal{S}$ es una topología en $Y$ entonces \[ \mathcal{U} = \{ \inv{g}(E) \tq E ∈ \mathcal{S} \} \]es una topología en $X$.

\solution
\spart Vamos a demostrar que es una topología, para lo cual tenemos que comprobar las 3 propiedades (ver \ref{defTopología}):

\begin{enumerate}
\item $Y\in \mathcal{S} \dimplies g^{-1}(x)\in \mathcal{T}$, por ser $g^{-1}(y)=x$

El razonamiento de porqué $\emptyset \in \mathcal{S}$ es igual.

\item $A,B \in \mathcal{S} \dimplies g^{-1}(A),g^{-1}(B) \in \mathcal{T}$.

$A\cap B \in\mathcal{S} \dimplies g^{-1}(A\cap B) \in \mathcal{T}$.

Hemos llegado a que para demostrar la segunda propiedad, tenemos que demostrar $g^{-1}(A),g^{-1}(B) \in \mathcal{T} \implies g^{-1}(A\cap B) \in \mathcal{T}$.

Para ello: $g^{-1}(a\cap B) = g^{-1}(A)\cap g^{-1}(B)$. No es difícil convencernos de esta igualdad. En caso de tener dudas, demostrar las 2 inclusiones (una en cada sentido). Esto para imágenes directas no funciona.

\item Demostramos ahora que la unión de abiertos está en la topología. Si $A, B ∈ \mathcal{S}$, entonces $\inv{g}(A), \inv{g}(B) ∈ \topl$. Como $\topl$ es topología, tenemos que $\inv{g}(A) ∪ \inv{g}(B) ∈ \topl$, lo que implica de forma obvia que $\inv{g}(A∪B) ∈ \topl$ y por lo tanto $A∪B ∈ \mathcal{S}$.
\end{enumerate}

\spart

\end{problem}

\begin{problem}[9] Se consideran las siguientes familias de conjuntos en $ℝ$:

\begin{gather*}
\base_{\leftarrow} = \{ (-∞, b) \tq b ∈ ℝ \} \\
\base_{\rightarrow} = \{ (a,∞) \tq  ∈ ℝ \} 
\end{gather*}

\ppart Demostrar que cada familia es una base de una topología sobre $ℝ$.
\ppart Comparar esas topologías.
\ppart Demostrar que la topología generada por $\base_{\leftarrow} ∪ \base_{\rightarrow}$ es la usual.
\solution
\spart Si añadimos $\emptyset$ y el total, entonces tenemos una topología generada por $\base_{\rightarrow}$ y otra generada por $\base_{\leftarrow}$.

\spart  

\spart
\end{problem}

\begin{problem}[11]
Sea $\topl_j$, $j∈J$ una familia de topologías sobre $X$. Demostrar que existe una topología que contiene a todas las $\topl_j$, para $j∈J$ y además es la menos fina de todas las que verifican esta propiedad. 
\solution

Aplicamos directamente la proposición \ref{propTopologiaMinima}: la topología que contiene a todas ellas es \[ \topl = \bigcap_{j∈J} \topl_j \]
\end{problem}

\paragraph{Observación útil para el 5 y el 12:}  
\begin{enumerate}
\item $x \in C(x,\varepsilon)$
\item $\varepsilon_1 > \varepsilon_2 \implies C(x,\varepsilon_2) \subset C(x,\varepsilon_1)$
\end{enumerate}

Y podemos aplicar la propiedad:
\[
A\in\topl \dimplies \forall a\in A \exists \varepsilon > 0 \tlq C(x,\varepsilon)\subseteq A
\]

Haciendo caso al enunciado y haciendo el dibujo vemos que se cumplen las propiedades de base.

Esta topología contiene a la usual pero al revés no, porque para el punto de intersección de las diagonales no existe un abierto de la usual que le contenga.


\paragraph{Pistas para espacios métricos}

(16) Si tengo $d$, una distancia no acotada, puedo definir $d'=\frac{d}{1+d}$, que sigue siendo una distancia, parecida y además acotada.

(17) $\sum \frac{1}{2n} \leq 1$. La clave está en aplicar la desigualdad triangular a cada término del sumatorio. La clave para este problema es el 16.

\printindex

\end{document}
