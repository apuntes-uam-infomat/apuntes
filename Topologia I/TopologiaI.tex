\documentclass{apuntes}

\title{Topología I}
\author{Guillermo Julián Moreno \\ Cristina Kasner Tourné}
\date{14/15 C1}
% Paquetes adicionales
\usepackage{tikztools}
\usepackage{fastbuild}
% --------------------

\begin{document}
\pagestyle{plain}
\maketitle

\tableofcontents
\newpage

\chapter{Conceptos básicos}

\section{Introducción}

Muy útilies los conceptos de convergencia y continuidad ;)

En Topología buscamos extender conceptos importantes como continuidad o convergencia. Si partimos del concepto de continuidad en los reales, teníamos que

\begin{defn}[Continuidad] Dada $\appl{f}{(a,b)}{ℝ}$, se dice que es continua en $x_0 ∈ (a,b)$ si $∀ ε > 0 \; ∃δ>0 $ tal que $\abs{x-x_0} < δ \implies \abs{f(x) - f(x_0)} < ε$.
\end{defn}

¿Cómo podemos extender esto a conjuntos que no sean $ℝ$? Lo primero es que necesitamos una distancia. Y la propiedad central de la distancia debería ser \[ \abs{x+y} ≤ \abs{x} + \abs{y} \]. Esta propiedad es la desigualdad triangular, y es ciertamente natural. La extensión de la distancia la tendremos en los espacios métricos.

\begin{defn}[Espacio\IS métrico]
Un espacio métrico es un par $(X, d)$, con $X$ un conjunto y $d$ una aplicación $\appl{d}{X×X}[0, ∞)$ tal que 

\begin{enumerate}
\item $\dst(x,y) = 0\;∀x∈X$.
\item $\dst(x,y) ≥ 0\;∀x,y∈X$.
\item $\dst(x,y) = 0 \dimplies x=y$.
\item \concept{Desigualdad\IS triangular}: $\dst(x,z) = \dst(x,y) + d(y,z)\; ∀x,y,z∈X$.
\end{enumerate}
\end{defn}

Tenemos varios ejemplos de distancias:

Por ejemplo, en $ℝ^m$, tenemos $\dst (x,y) = \md{\vx-\vy} = \sqrt{(x_1-y_1)^2 + \dotsb + (x_m-y_m)^2}$.


Si consideramos el conjunto de funciones continuas $C([0,1]) \equiv \{ \appl{f}{[0,1]}{ℝ}, \text{f continua} \}$, $\md{f} ≝ \max_{x∈[0,1]} \abs{f(x)}$. Con esta noción, el conjunto de funciones continuas se comporta de forma similar a $ℝ^m$ con dimensión infinita, y podemos hacer cosas parecidas a las del espacio euclídeo.

Así, podemos definir $\dst(f,g) ≝ \md{f-g}$, y llegar a una definición de convergencia uniforme: $f_n \to f$ en esa distancia implica una convergencia uniforme en $[0,1]$.

Podemos definir una distancia algo artificial. Sea $X$ un conjunto cualquiera, definimos
\[ \dst(x,y) ≝ \begin{cases}
0 & \text{si}\; x = y \\
1 & \text{si}\; x ≠ y \\
\end{cases} \] 
que claramente cumple las condiciones para ser una distancia.

\begin{defn}[Bola] Dado $(X,\dst)$ un espacio métrico, con $x∈X$ y $r∈(0,∞)$, definimos la bola $\bola$ centrada en $x$ de radio $r$ como 

\[ \bola(x,r) ≝ \{ y∈X \tq \dst(x,y) < r \} \]

En ocasiones querremos especificar la distancia ($\bola_{\dst}$) o el conjunto ($\bola_X$) con el subíndice.
\end{defn}

Las bolas tienen ciertas propiedades muy sencillas. Dados \sdst, $x∈X$, $r>0$, $y∈\bola(x,r)$ entonces $\bola(y,r-\dst(x,y)) ⊆ \bola(x,r)$.

\begin{wrapfigure}{r}{0.4\textwidth}
\inputtikz{I_BolaContenida}
\caption{La bola verde ($\bola(y, r-\dst(x,y)$) contenida dentro de $\bola(x, r)$.}
\label{figBolaContenida}
\end{wrapfigure}

Esto se puede demostrar con un dibujo (\ref{figBolaContenida}), pero tenemos que demostrarlo más formalmente: 

\begin{proof}
$∀z∈\bola(x, r-\dst(x,y))$ tenemos que $\dst(x,z) ≤ \dst(x,y) + \dst(y,z) < \dst(x,y) + r - \dst(x,y) = r$, y por lo tanto $z ∈ \bola(x,r)$.
\end{proof}

Por supuesto, el dibujo es una guía. Si tomásemos la distancia rara de antes que sólo tomaba valores 1 ó 0, la bola no sería una bola como en $ℝ$.

Vamos a definir ahora el cierre, aunque sólo como notación:

\begin{defn}[Cierre] Dado \sdst espacio métrico, $x∈X$, $r≥0$, definimos \[ \overline{\bola}(x,r) ≝ \{ y∈ X\tq \dst(x,y) ≤ r\} \] como la bola cerrada de centro $x$ y radio $r$.\end{defn}

\begin{defn}[Conjunto\IS abierto] Sea \sdst un espacio métrico. Entonces damos dos definiciones

\begin{enumerate}
\item $A⊆X$ es abierto en \sdst si $∀x∈A\; ∃δ=δ_x > 0$ tal que $\bola(x,δ_x) ⊆ A$.
\item La familia de abiertos es $\topl_d \equiv \{ A ⊆ X \tq A \text{abierto} \}$.
\end{enumerate}
\end{defn}

La familia de abiertos que acabamos de definir es una \concept[Topología]{topología}, y cumple las siguientes propiedades.

\begin{enumerate}
\item $\emptyset, X ∈ \topl_d$.
\item $A,B ∈ \topl_d \implies A \cap B ∈ \topl_d$.
\item $A_j ∈ \topl_d\; ∀j∈ J \implies \bigcup_{j∈J} A_j ∈ \topl_d$
\end{enumerate}

Demostremos las dos últimas propiedades:

\begin{proof}
\paragraph{Prop. 2} Sea $x∈A\cap B$. Entonces $x∈A$ y $x∈B$, luego existen $δ_x^A, δ_x^B$ tales que $\bola(x, δ_x^A) ⊆ A$ y $\bola(x, δ_x^B) ⊆ B$ respectivamente. Sea ahora $δ=\min(δ_x^A, δ_x^B)$. Entonces $\bola(x,δ) ⊆ A\cup B$.

\paragraph{Prop. 3} La propiedad es equivalente a la pregunta de, si dado $x∈\bigcup_{j∈J}$, se cumple que $∃δ > 0 \tq \bola(x,δ)⊆\bigcup_{j∈J} A_j$.

Es obvio que $∃j_x ∈ J\tq x∈ A_{j_x}$, luego \[ ∃ δ > 0 \tq \bola(x,δ) ⊆ A_{j_x} ⊆ \bigcup_{j∈J} A_j \]
\end{proof}

Por otra parte, también podemos hacer una observación: por inducción, la intersección de una familia \textit{finita} de conjuntos también es un abierto.

\subsection{Topologías}

Una vez hecho esto, ya podemos pasar a definir qué es un espacio topológico y una topología.

\begin{defn}[Topología]\label{defTopología}
Sea $X$ un conjunto. Entonces una familia $\topl$ de subconjuntos de $X$ es una topología en $X$ si y sólo si cumple las tres propiedades que acabamos de ver:

\begin{enumerate}
\item $\emptyset, X ∈ \topl$.
\item $A,B ∈ \topl \implies A \cap B ∈ \topl$.
\item $A_j ∈ \topl\; ∀j∈ J \implies \bigcup_{j∈J} A_j ∈ \topl$
\end{enumerate}
\end{defn}

\begin{defn}[Espacio\IS topológico] Un espacio topológico es un par $(X, \topl)$ donde $\topl$ es una topología en $X$.
\end{defn}

Los elementos de \topl son los \textit{abiertos} de la topología. 

También podemos definir el conjunto cerrado:

\begin{defn}[Conjunto\IS cerrado]
Dado un espacio topológico \stopl, $F⊆X$ es cerrado si y sólo si $F^C \equiv X \setminus F$ es abierto.
\end{defn}

Podemos definir dos topologías "comunes", por así decirlo, las obvias para cualquier conjunto. Tenemos la \concept[Topología!trivial]{topología\IS trivial} (el mínimo) dada por \[ \topl_{triv.} = \{ \emptyset, X \} \], y la \concept[Topología!discreta]{topología\IS discreta}, que sería el máximo: \[ \topl_{disc.} = \parts{X} \].

Y volviendo a nuestro bonito mundo de los reales, tenemos las \concept[Topología\IS usual]{topologías usuales} en $ℝ^m$ o $\topl_{ℝ^m}$. Para $m=1$, diremos que \[ A ∈ \topl_ℝ ≝ ∀x∈A\; ∃a,b∈ℝ \tq x∈ (a,b) ⊆ A \]. Esto es, que siempre podemos encontrar un intervalo contenido en $A$ que a su vez contenga a $x$. Equivalentemente, $A$ será una unión de intervalos abiertos.

Para dimensión $m>1$, definimos su topología usual como \[	ℝ^m ⊇ A ∈ \topl_{R^m} ≝ ∀x∈A\; ∃ \begin{matrix} a_1, \dotsc, a_m \\ b_1, \dotsc, b_m \end{matrix} ∈ ℝ \] tales que $ x∈ (a_1, b_1) × \dotsb × (a_m, b_m) ⊆ A$.


\subsubsection{Topologías metrizables}

\begin{defn}[Espacio\IS topológico metrizable] Dado \stopl un espacio topológico, se dice que es metrizable si existe una distancia $\dst$ en $X$ tal que $\topl = \topl_{\dst}$. 

$\dst$ no es necesariamente única.
\end{defn}

Por ejemplo, $\topl_ℝ$ es metrizable. Coincide $\topl= \topl_{\dst}$ con $\dst(x,y) = \abs{x- y}$.

Expandiendo un poco más sobre lo que significa que una topología coincide con otra, o lo que significa que una topología sea \textit{sea inducida} por una aplicación. 

\begin{defn}[Topología\IS inducida] Dado un espacio topológico \stopl, entonces la topología inducida por una función $f$ es \[ \topl_f = \{ \inv{f} (A) \tq A ∈ \topl \} \]
\end{defn} 

¿Qué significa entonces que una topología sea igual a otra? Si nos remitimos a la definición de topología (\ref{defTopología}), vemos que es un conjunto de subconjuntos de $X$. Luego dos topologías son iguales o equivalentes si y sólo si tienen los mismos elementos. Es decir, que si un conjunto es abierto en $X$ según una topología, también lo es según la otra y viceversa.

Volviendo al caso concreto, una topología inducida por la distancia es igual a otra topología si los abiertos según la distancia (esto es, las bolas) son también abiertos según la otra topología que estemos considerando.

Hagamos algunos ejemplos sobre topologías son metrizables. ¿Son $\topl_{disc.}, \topl_{triv.}$ metrizables para $X$ un conjunto cualquiera?

En el caso de $\topl_{disc}$ sí lo es. Definimos la distancia como \[ \dst (x,y) = \begin{cases} 0 & x = y \\ 1 & x ≠ y \end{cases} \], luego $\bola(x, 1/2) = \{ x \}$, luego $\{ x \}$ es abierto en $\topl_{\dst}$. Entonces, si $A ⊆ X$, entonces $A= \bigcup_{x∈A} \{ x \}$ es abierto también.

La topología trivial es más interesante de estudiar. Si $\card{X}≥2$, entonces $\topl_{triv} ≠ \topl_{\dst}$ para cualquier distancia $\dst(x,y)$. ¿Por qué?

En la topología trivial sólo hay dos abiertos (vacío y total). Sin embargo, en la topología inducida por la distancia, los abiertos son las bolas. 

Si hay más de dos elementos en $X$, existen $x,y∈X$ distintos, y por lo tanto existe una distancia $r=\dst(x,y) > 0$. Con $δ=\frac{r}{2}$, las bolas $\bola(x,δ), \bola(y,δ)$ son distintas y disjuntas. Ninguna de ellas es el vacío y el total así que no son abiertos en $\topl_{triv}$, pero sí que son abiertos en $\topl_{\dst}$. Por lo tanto, tenemos que $\topl_{\dst} = \topl_{triv}$.

\subsubsection{Topologías generadas por una base}

Además de por la distancia, podemos considerar las \textbf{topologías generadas por una base}.

Recordemos cómo definíamos una topología en $\topl_ℝ$. Decíamos que $A∈\topl_ℝ$ si y sólo si $∀x∈A\; ∃(a,b)$ tales que $x∈(a,b) ⊆ A$. 

\begin{defn}[Base]
Sea $X$ un conjunto y $\base$ una familia de subconjuntos de $X$ (e.d. $\base ⊆ \parts{X}$). Entonces $\base$ es una base si y sólo si 

\begin{enumerate}
\item $∀x∈X\;∃B∈\base \tq x∈B$. Dicho de otra forma, $\bigcup_{B∈\base} B = X$.
\item $∀B_1,B_2∈\base;\, ∀x∈B_1 ∩ B_2$, existe $B_3 ∈ \bola \tq x∈B_3 ⊆ B_1∩B_2$.
\end{enumerate}
\end{defn}


\begin{defn}[Topología\IS generada por una base] La topología $\topl_\base$ se define por \[ A ∈ \topl_\base \iff ∀x ∈ A\; B∈\base \tq x∈B⊆A \]
\end{defn}

Tenemos que demostrar, eso sí, que eso que hemos definido ahí es realmente una topología.

\begin{prop} $\topl_\base$ es una topología en $X$.\end{prop}

\begin{proof} Tenemos que comprobar las tres propiedades de una topología (\ref{defTopología}). Sabemos que $\emptyset ∈ \topl_\base$. Además, tal y como hemos definido la topología generada, también sabemos que $X∈ \topl_\base$.

\paragraph{Propd. 2} Tenemos que demostrar que $A_1, A_2 ∈ \topl_\base \implies A_1∩A_2 ∈ \topl_\base$. Si $x∈ A_1∩A_2$, entonces $∃B_1, B_2 ∈ \base$ tales que $x∈B_1⊆A_1$ y $x∈B_2⊆A_2$ respectivamnete. 

Según la segunda propiedad de la base, existe un $B_3∈\base$ tal que $x∈B_3 ⊆ B_1∩B_2 ⊆A_1∩A_2$, luego $A_1∩A_2 ∈ \topl_\base$.

\paragraph{Propd. 3} Demostramos que $A_j ∈ \topl_\base\; ∀j∈J \implies \bigcup_{j∈J} A_j ∈ \topl_\base$. Si $x∈\bigcup_{j∈J} A_j \implies ∃i=i_x∈J\tq x∈A_i$. Luego como $A_i ∈ \topl_\base$ tenemos que $\exists B∈ \base$ tal que $x∈B ⊆ A_i ⊆  \bigcup A_j$.
\end{proof}

Nos fijamos que en la demostración de la tercera propiedad no hemos usado nada sobre cómo hemos definido la base. Es decir, que siempre que defininamos una topología $\topl$ como \[ A ∈ \topl \iff ∀x∈ A\;∃U ∈ \mathcal{F} \tq x∈ U ⊆ A \], donde $\mathcal{F}$ es una familia de subconjuntos de $X$, la propiedad tercera de la definición de topología \textbf{está garantizada}. Es para la primera y segunda propiedad para las que se necesita que $\mathcal{F}$ cumpla algún tipo de propiedad.

\appendix
\chapter{Ejercicios}

\section{Hoja 1}


\begin{problem}[6] Sea $\appl{g}{X}{Y}$ una aplicación entre dos conjuntos.

\ppart Demostrar que si $\topl$ es una topología en $X$ entonces \[ \mathcal{S} = \{ E ⊆ Y \tq \inv{g}(E) ∈ \topl \} \] es una topología en $Y$.
\ppart Demostrar que si $\mathcal{S}$ es una topología en $Y$ entonces \[ \mathcal{U} = \{ \inv{g}(E) \tq E ∈ \mathcal{S} \} \]es una topología en $X$.

\solution
\spart Vamos a demostrar que es una topología, para lo cual tenemos que comprobar las 3 propiedades (ver \ref{defTopología}):

\begin{enumerate}
\item $Y\in \mathcal{S} \dimplies g^{-1}(x)\in \mathcal{T}$, por ser $g^{-1}(y)=x$

El razonamiento de porqué $\emptyset \in \mathcal{S}$ es igual.

\item $A,B \in \mathcal{S} \dimplies g^{-1}(A),g^{-1}(B) \in \mathcal{T}$.

$A\cap B \in\mathcal{S} \dimplies g^{-1}(A\cap B) \in \mathcal{T}$.

Hemos llegado a que para demostrar la segunda propiedad, tenemos que demostrar $g^{-1}(A),g^{-1}(B) \in \mathcal{T} \implies g^{-1}(A\cap B) \in \mathcal{T}$.

Para ello: $g^{-1}(a\cap B) = g^{-1}(A)\cap g^{-1}(B)$. No es difícil convencernos de esta igualdad. En caso de tener dudas, demostrar las 2 inclusiones (una en cada sentido). Esto para imágenes directas no funciona.

\item Demostramos ahora que la unión de abiertos está en la topología. Si $A, B ∈ \mathcal{S}$, entonces $\inv{g}(A), \inv{g}(B) ∈ \topl$. Como $\topl$ es topología, tenemos que $\inv{g}(A) ∪ \inv{g}(B) ∈ \topl$, lo que implica de forma obvia que $\inv{g}(A∪B) ∈ \topl$ y por lo tanto $A∪B ∈ \mathcal{S}$.
\end{enumerate}

\spart

\end{problem}

\begin{problem}[9] Se consideran las siguientes familias de conjuntos en $ℝ$:

\begin{gather*}
\base_{\leftarrow} = \{ (-∞, b) \tq b ∈ ℝ \} \\
\base_{\rightarrow} = \{ (a,∞) \tq  ∈ ℝ \} 
\end{gather*}

\ppart Demostrar que cada familia es una base de una topología sobre $ℝ$.
\ppart Comparar esas topologías.
\ppart Demostrar que la topología generada por $\base_{\leftarrow} ∪ \base_{\rightarrow}$ es la usual.
\solution
\spart Si añadimos $\emptyset$ y el total, entonces tenemos una topología generada por $\base_{\rightarrow}$ y otra generada por $\base_{\leftarrow}$.

\spart  

\spart
\end{problem}

\begin{problem}[11]
Sea $\topl_j$, $j∈J$ una familia de topologías sobre $X$. Demostrar que existe una topología que contiene a todas las $\topl_j$, para $j∈J$ y además es la menos fina de todas las que verifican esta propiedad. 
\solution

Aplicamos directamente la proposición \ref{propTopologiaMinima}: la topología que contiene a todas ellas es \[ \topl = \bigcap_{j∈J} \topl_j \]
\end{problem}

\paragraph{Observación útil para el 5 y el 12:}  
\begin{enumerate}
\item $x \in C(x,\varepsilon)$
\item $\varepsilon_1 > \varepsilon_2 \implies C(x,\varepsilon_2) \subset C(x,\varepsilon_1)$
\end{enumerate}

Y podemos aplicar la propiedad:
\[
A\in\topl \dimplies \forall a\in A \exists \varepsilon > 0 \tlq C(x,\varepsilon)\subseteq A
\]

Haciendo caso al enunciado y haciendo el dibujo vemos que se cumplen las propiedades de base.

Esta topología contiene a la usual pero al revés no, porque para el punto de intersección de las diagonales no existe un abierto de la usual que le contenga.


\paragraph{Pistas para espacios métricos}

(16) Si tengo $d$, una distancia no acotada, puedo definir $d'=\frac{d}{1+d}$, que sigue siendo una distancia, parecida y además acotada.

(17) $\sum \frac{1}{2n} \leq 1$. La clave está en aplicar la desigualdad triangular a cada término del sumatorio. La clave para este problema es el 16.

\printindex

\end{document}
