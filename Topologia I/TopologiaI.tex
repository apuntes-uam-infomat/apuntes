\documentclass{apuntes}

\title{Topología I}
\author{Guillermo Julián Moreno \\ Cristina Kasner Tourné}
\date{14/15 C1}
% Paquetes adicionales
\usepackage{tikztools}
\usepackage{fastbuild}

\usetikzlibrary{arrows}
% --------------------

%\precompileTikz

\begin{document}
\pagestyle{plain}
\maketitle

\tableofcontents
\newpage

\chapter{Conceptos básicos}

\section{Introducción}

En Topología buscamos extender conceptos importantes como continuidad o convergencia. Si partimos del concepto de continuidad en los reales, teníamos que

\begin{defn}[Continuidad] Dada $\appl{f}{(a,b)}{ℝ}$, se dice que es continua en $x_0 ∈ (a,b)$ si $∀ ε > 0 \; ∃δ>0 $ tal que $\abs{x-x_0} < δ \implies \abs{f(x) - f(x_0)} < ε$.
\end{defn}

¿Cómo podemos extender esto a conjuntos que no sean $ℝ$? Lo primero es que necesitamos una distancia. Y la propiedad central de la distancia debería ser \[ \abs{x+y} ≤ \abs{x} + \abs{y} \]. Esta propiedad es la desigualdad triangular, y es ciertamente natural. La extensión de la distancia la tendremos en los espacios métricos.

\begin{defn}[Espacio\IS métrico] \label{defEspacioMetrico}
Un espacio métrico es un par $(X, d)$, con $X$ un conjunto y $d$ una aplicación $\appl{d}{X×X}[0, ∞)$ tal que 

\begin{enumerate}
\item $\dst(x,x) = 0\;∀x∈X$.
\item $\dst(x,y) ≥ 0\;∀x,y∈X$.
\item $\dst(x,y) = 0 \dimplies x=y$.
\item \concept{Desigualdad\IS triangular}: $\dst(x,z) \leq \dst(x,y) + d(y,z)\; ∀x,y,z∈X$.
\end{enumerate}
\end{defn}

Tenemos varios ejemplos de distancias:

Por ejemplo, en $ℝ^m$, tenemos $\dst (x,y) = \md{\vx-\vy} = \sqrt{(x_1-y_1)^2 + \dotsb + (x_m-y_m)^2}$.


Si consideramos el conjunto de funciones continuas $C([0,1]) \equiv \{ \appl{f}{[0,1]}{ℝ}, \text{f continua} \}$, $\md{f} ≝ \max_{x∈[0,1]} \abs{f(x)}$. Con esta noción, el conjunto de funciones continuas se comporta de forma similar a $ℝ^m$ con dimensión infinita, y podemos hacer cosas parecidas a las del espacio euclídeo.

Así, podemos definir $\dst(f,g) ≝ \md{f-g}$, y llegar a una definición de convergencia uniforme: $f_n \to f$ en esa distancia implica una convergencia uniforme en $[0,1]$.

Podemos definir una distancia algo artificial. Sea $X$ un conjunto cualquiera, definimos
\[ \dst(x,y) ≝ \begin{cases}
0 & \text{si}\; x = y \\
1 & \text{si}\; x ≠ y \\
\end{cases} \] 
que cumple las 3 primeras propiedades de distancia y su comprobación es trivial y para la comprobación de la desigualdad triangular, basta comprobarlo por casos.

\begin{defn}[Bola] Dado $(X,\dst)$ un espacio métrico, con $x∈X$ y $r∈(0,∞)$, definimos la bola $\bola$ centrada en $x$ de radio $r$ como 

\[ \bola(x,r) ≝ \{ y∈X \tq \dst(x,y) < r \} \]

En ocasiones querremos especificar la distancia ($\bola_{\dst}$) o el conjunto ($\bola_X$) con el subíndice.
\end{defn}

Las bolas tienen ciertas propiedades muy sencillas. Dados \sdst, $x∈X$, $r>0$, $y∈\bola(x,r)$ entonces $\bola(y,r-\dst(x,y)) ⊆ \bola(x,r)$.

\begin{wrapfigure}{r}{0.4\textwidth}
\inputtikz{I_BolaContenida}
\caption{La bola verde ($\bola(y, r-\dst(x,y)$) contenida dentro de $\bola(x, r)$.}
\label{figBolaContenida}
\end{wrapfigure}

Esto se puede demostrar con un dibujo (\ref{figBolaContenida}), pero tenemos que demostrarlo más formalmente: 

\begin{proof}
$∀z∈\bola(x, r-\dst(x,y))$ tenemos que $\dst(x,z) ≤ \dst(x,y) + \dst(y,z) < \dst(x,y) + r - \dst(x,y) = r$, y por lo tanto $z ∈ \bola(x,r)$.
\end{proof}

Por supuesto, el dibujo es una guía. Si tomásemos la distancia rara de antes que sólo tomaba valores 1 ó 0, la bola no sería una bola como en $ℝ$.

Vamos a definir ahora el cierre, aunque sólo como notación:

\begin{defn}[Cierre] Dado \sdst espacio métrico, $x∈X$, $r≥0$, definimos \[ \overline{\bola}(x,r) ≝ \{ y∈ X\tq \dst(x,y) ≤ r\} \] como la bola cerrada de centro $x$ y radio $r$.\end{defn}

\begin{defn}[Conjunto\IS abierto] Sea \sdst un espacio métrico. Entonces damos dos definiciones

\begin{enumerate}
\item $A⊆X$ es abierto en \sdst si $∀x∈A\; ∃δ=δ_x > 0$ tal que $\bola(x,δ_x) ⊆ A$.
\item La familia de abiertos es $\topl_d \equiv \{ A ⊆ X \tq A\, \text{abierto} \}$.
\end{enumerate}
\end{defn}

La familia de abiertos que acabamos de definir es una \concept[Topología]{topología}, y cumple las siguientes propiedades.

\begin{enumerate}
\item $\emptyset, X ∈ \topl_d$.
\item $A,B ∈ \topl_d \implies A \cap B ∈ \topl_d$.
\item $A_j ∈ \topl_d\; ∀j∈ J \implies \bigcup_{j∈J} A_j ∈ \topl_d$
\end{enumerate}

Demostremos las dos últimas propiedades:

\begin{proof} \paragraph{Propiedad 2} Sea $x∈A\cap B$. Entonces $x∈A$ y $x∈B$, luego existen $δ_x^A, δ_x^B$ tales que $\bola(x, δ_x^A) ⊆ A$ y $\bola(x, δ_x^B) ⊆ B$ respectivamente. Sea ahora $δ=\min(δ_x^A, δ_x^B)$. Entonces $\bola(x,δ) ⊆ A\cup B$.

\paragraph{Propiedad 3} La propiedad es equivalente a la pregunta de, si dado $x∈\bigcup_{j∈J}$, se cumple que $∃δ > 0 \tq \bola(x,δ)⊆\bigcup_{j∈J} A_j$.

Es obvio que $∃j_x ∈ J\tq x∈ A_{j_x}$, luego \[ ∃ δ > 0 \tq \bola(x,δ) ⊆ A_{j_x} ⊆ \bigcup_{j∈J} A_j \]
\end{proof}

Por otra parte, también podemos hacer una observación: por inducción, la intersección de una familia \textit{finita} de conjuntos también es un abierto.

\subsection{Topologías}

Una vez hecho esto, ya podemos pasar a definir qué es un espacio topológico y una topología.

\begin{defn}[Topología]\label{defTopología}
Sea $X$ un conjunto. Entonces una familia $\topl$ de subconjuntos de $X$ es una topología en $X$ si y sólo si cumple las tres propiedades que acabamos de ver:

\begin{enumerate}
\item $\emptyset, X ∈ \topl$.
\item $A,B ∈ \topl \implies A \cap B ∈ \topl$.
\item $A_j ∈ \topl\; ∀j∈ J \implies \bigcup_{j∈J} A_j ∈ \topl$
\end{enumerate}
\end{defn}

\begin{defn}[Espacio\IS topológico] Un espacio topológico es un par $(X, \topl)$ donde $\topl$ es una topología en $X$.
\end{defn}

Los elementos de \topl son los \textit{abiertos} de la topología. 

También podemos definir el conjunto cerrado:

\begin{defn}[Conjunto\IS cerrado]
Dado un espacio topológico \stopl, $F⊆X$ es cerrado si y sólo si $F^C \equiv X \setminus F$ es abierto.
\end{defn}

Podemos definir dos topologías "comunes", por así decirlo, las obvias para cualquier conjunto. Tenemos la \concept[Topología!trivial]{topología\IS trivial} (el mínimo) dada por \[ \topl_{triv.} = \{ \emptyset, X \} \], y la \concept[Topología!discreta]{topología\IS discreta}, que sería el máximo: \[ \topl_{disc.} = \parts{X} \].

Y volviendo a nuestro bonito mundo de los reales, tenemos las \concept[Topología\IS usual]{topologías usuales} en $ℝ^m$ o $\topl_{ℝ^m}$. Para $m=1$, diremos que \[ A ∈ \topl_ℝ ≝ ∀x∈A\; ∃a,b∈ℝ \tq x∈ (a,b) ⊆ A \] Esto es, que siempre podemos encontrar un intervalo contenido en $A$ que a su vez contenga a $x$. Equivalentemente, $A$ será una unión de intervalos abiertos.

Para dimensión $m>1$, definimos su topología usual como \[	ℝ^m ⊇ A ∈ \topl_{R^m} ≝ ∀x∈A\; ∃ \begin{matrix} a_1, \dotsc, a_m \\ b_1, \dotsc, b_m \end{matrix} ∈ ℝ \] tales que $ x∈ (a_1, b_1) × \dotsb × (a_m, b_m) ⊆ A$.


\subsubsection{Topologías metrizables}

\begin{defn}[Espacio\IS topológico metrizable] Dado \stopl un espacio topológico, se dice que es metrizable si existe una distancia $\dst$ en $X$ tal que $\topl = \topl_{\dst}$. 

$\dst$ no es necesariamente única.
\end{defn}

Por ejemplo, $\topl_ℝ$ es metrizable. Coincide $\topl= \topl_{\dst}$ con $\dst(x,y) = \abs{x- y}$.

Expandiendo un poco más sobre lo que significa que una topología coincide con otra, o lo que significa que una topología \textit{sea inducida} por una aplicación. 

\begin{defn}[Topología\IS inducida] Dado un espacio topológico \stopl, entonces la topología inducida por una función $f$ es \[ \topl_f = \{ \inv{f} (A) \tq A ∈ \topl \}\footnote{La prueba de porqué es topología se encuentra en los ejercicios (H1.E6)} \]
\end{defn} 

¿Qué significa entonces que una topología sea igual a otra? Si nos remitimos a la definición de topología (\ref{defTopología}), vemos que es un conjunto de subconjuntos de $X$. Luego dos topologías son iguales o equivalentes si y sólo si tienen los mismos elementos. Es decir, que si un conjunto es abierto en $X$ según una topología, también lo es según la otra y viceversa.

Volviendo al caso concreto, una topología inducida por la distancia es igual a otra topología si los abiertos según la distancia (esto es, las bolas) son también abiertos según la otra topología que estemos considerando.

Hagamos algunos ejemplos sobre topologías metrizables. ¿Son $\topl_{disc.}, \topl_{triv.}$ metrizables para $X$ un conjunto cualquiera?

En el caso de $\topl_{disc}$ sí lo es. Definimos la distancia como \[ \dst (x,y) = \begin{cases} 0 & x = y \\ 1 & x ≠ y \end{cases} \], luego $\bola(x, 1/2) = \{ x \}$, luego $\{ x \}$ es abierto en $\topl_{\dst}$. Entonces, si $A ⊆ X$, entonces $A= \bigcup_{x∈A} \{ x \}$ es abierto también.

La topología trivial es más interesante de estudiar. Si $\card{X}≥2$, entonces $\topl_{triv} ≠ \topl_{\dst}$ para cualquier distancia $\dst(x,y)$. ¿Por qué?

En la topología trivial sólo hay dos abiertos (vacío y total). Sin embargo, en la topología inducida por la distancia, los abiertos son las bolas. 

Si hay más de dos elementos en $X$, existen $x,y∈X$ distintos, y por lo tanto existe una distancia $r=\dst(x,y) > 0$. Con $δ=\frac{r}{2}$, las bolas $\bola(x,δ), \bola(y,δ)$ son distintas y disjuntas. Ninguna de ellas es el vacío y el total así que no son abiertos en $\topl_{triv}$, pero sí que son abiertos en $\topl_{\dst}$. Por lo tanto, tenemos que $\topl_{\dst} = \topl_{triv}$.

\subsubsection{Topologías generadas por una base}

Además de por la distancia, podemos considerar las \textbf{topologías generadas por una base}.

Recordemos cómo definíamos una topología en $\topl_ℝ$. Decíamos que $A∈\topl_ℝ$ si y sólo si $∀x∈A\; ∃(a,b)$ tales que $x∈(a,b) ⊆ A$. 

\begin{defn}[Base]\label{defBase}
Sea $X$ un conjunto y $\base$ una familia de subconjuntos de $X$ (e.d. $\base ⊆ \parts{X}$). Entonces $\base$ es una base si y sólo si 

\begin{enumerate}
\item $∀x∈X\;∃B∈\base \tq x∈B$. Dicho de otra forma, $\bigcup_{B∈\base} B = X$.
\item $∀B_1,B_2∈\base;\, ∀x∈B_1 ∩ B_2$, existe $B_3 ∈ \bola \tq x∈B_3 ⊆ B_1∩B_2$.
\end{enumerate}
\end{defn}


\begin{defn}[Topología\IS generada por una base] \label{TopologiaGeneradaBase} La topología $\topl_\base$ se define por \[ A ∈ \topl_\base \iff ∀x ∈ A\; \exists B∈\base \tq x∈B⊆A \]
\end{defn}

Tenemos que demostrar, eso sí, que eso que hemos definido ahí es realmente una topología.

\begin{prop} $\topl_\base$ es una topología en $X$.\end{prop}

\begin{proof} Tenemos que comprobar las tres propiedades de una topología (\ref{defTopología}). Sabemos que $\emptyset ∈ \topl_\base$. Además, tal y como hemos definido la topología generada, también sabemos que $X∈ \topl_\base$.

\paragraph{Propd. 2} Tenemos que demostrar que $A_1, A_2 ∈ \topl_\base \implies A_1∩A_2 ∈ \topl_\base$. Si $x∈ A_1∩A_2$, entonces $∃B_1, B_2 ∈ \base$ tales que $x∈B_1⊆A_1$ y $x∈B_2⊆A_2$ respectivamnete. 

Según la segunda propiedad de la base, existe un $B_3∈\base$ tal que $x∈B_3 ⊆ B_1∩B_2 ⊆A_1∩A_2$, luego $A_1∩A_2 ∈ \topl_\base$.

\paragraph{Propd. 3} Demostramos que $A_j ∈ \topl_\base\; ∀j∈J \implies \bigcup_{j∈J} A_j ∈ \topl_\base$. Si $x∈\bigcup_{j∈J} A_j \implies ∃i=i_x∈J\tq x∈A_i$. Luego como $A_i ∈ \topl_\base$ tenemos que $\exists B∈ \base$ tal que $x∈B ⊆ A_i ⊆  \bigcup A_j$.
\end{proof}

Nos fijamos que en la demostración de la tercera propiedad no hemos usado nada sobre cómo hemos definido la base. Es decir, que siempre que defininamos una topología $\topl$ como \[ A ∈ \topl \iff ∀x∈ A\;∃U ∈ \mathcal{F} \tq x∈ U ⊆ A \], donde $\mathcal{F}$ es una familia de subconjuntos de $X$, la propiedad tercera de la definición de topología \textbf{está garantizada}. Es para la primera y segunda propiedad para las que se necesita que $\mathcal{F}$ cumpla algún tipo de propiedad.

\begin{defn}[Topología\IS fina]
Dado un espacio $X$ y dos topologías $\topl_1, \topl_2$, si $\topl_1⊆\topl_2$ (todo abierto de $\topl_1$ es abierto de $\topl_2$) se dice que $\topl_2$ es \textbf{más fina} que $\topl_1$.
\end{defn}

\begin{prop} Sea $X$ un espacio topológico y $\base$ una base. Entonces 

\begin{enumerate}
\item $\base⊆\topl_\base$ (todos los elementos de $\base$ son abiertos en $\topl_\base$.
\item $A∈\topl_\base$ si y sólo si $A$ es unión de elementos de $\base$.
\end{enumerate}
\end{prop}

\begin{proof}
\paragraph{1)} Recordamos que \[ V ∈ \topl_\base ≝ ∀x∈V \; ∃B=B_x∈\base\tq  x∈B⊆V \]. Sea $M∈\base$, quiero demostrar que $M∈ \topl_\base$. He de comprobar que \[ ∀x ∈ M\; ∃B∈\base \tq x∈ B ⊆ M \], lo cual es obvio si tomamos $B=M$, ya que los elementos de la base son siempre abiertos.

\paragraph{2)} Partiendo de la afirmación de antes, sabemos que si $A ∈ \topl_\base$, entonces $∀x∈A\; ∃B_x∈\base$ tal que $x∈ B_x⊆A$. Como cada uno de esos conjuntos está en $A$, su unión también lo está. Y por otra parte, dado que consideramos todos los puntos $x$ de $A$, nos queda que \[ A = \bigcup_{x∈A}B_x \], demostrando así el primer lado de la implicación.

La implicación a la izquierda se resuelve por la primera parte de esta proposición: si $B_j∈B$, entonces $B_j∈\topl_\base$ y por la tercera propiedad de la topología (\ref{defTopología}), nos queda que \[ \bigcup_{j∈J} B_j ∈ \topl_\base \]

\end{proof}

Ahora que ya sabemos cómo generar una topología a partir de una base, podemos hacernos una pregunta. Consideramos una serie de conjuntos que queremos que sean abiertos en nuestro espacio. Obviamente, la topología discreta cumple lo que buscamos, pero, ¿hay una topología más pequeña? ¿Cuál es la topología \textit{"mínima"}?

\begin{prop} Sea $X$ un conjunto. \label{propTopologiaMinima}

\begin{enumerate}
\item Si $\topl_k$ es una topología en $X$, $∀k∈K$ entonces \[ \topl ≝\bigcap_{k∈K} \topl_k \].

\item Sea $D$ una familia de subconjuntos de $X$ ($D⊆\parts{X}$) y sea \[ \topl_D ≝ \bigcap_{D⊆\topl} \topl \] donde $\topl$ es una topología en $X$. 

Entonces $\topl_D$ es una topología en $X$, $D⊆\topl_D$ y $\topl_D$ es la topología menos fina que cumple $D⊆\topl_D$.
\end{enumerate}
\end{prop}

\begin{proof}
\paragraph{1)} La primera propiedad de la topología es trivial. Vamos con la segunda. Si $V_1V_2 ∈ \topl$, tenemos que $V_1, V_2 ∈ \topl_k\,∀k∈K$. Luego como $\topl_k$ es topología, $V_1∩ V_2 ∈\topl_k\, ∀k∈K$, y entonces $V_1∩V_2 ∈ \bigcap \topl_k = \topl$.

\paragraph{2)} Sabemos que $\topl_D$ es topología por lo que acabamos de demostrar. Ahora bien, ¿es la más pequeña? Es obvio, viendo que es la intersección de todas las topologías que contienen a $D$.\footnotemark
\end{proof}
\footnotetext{Relacionado con el ejercicio 9-c.}

Tenemos que tener cuidado cuando $D$ es una base: hay que asegurarse de que la topología coincida en ese caso.\footnote{Y nos preocupamos nosotros de eso.} 

\subsubsection{Topología del orden}

Hasta ahora hemos visto cómo generar topologías a partir de la distancia, y también tratando de extrapolar el concepto de los intervalos de $ℝ$ con las bases. Ahora vamos a ver cómo hacerlos a través de otra visión de los intervalos como elementos de orden. Recordemos brevemente qué es un orden total: a grandes rasgos es uno donde podemos comparar todos los elementos.

\begin{defn}[Orden\IS total] Dado un conjunto $X$, un orden total en $X$ es una relación $x < y$ tal que 

\begin{enumerate}
\item $x<y, y < z\implies x < z$.
\item $∀x∈X$, $x < x$ es falso.
\item $∀x,y∈X$ con $x≠y$ entonces se cumple una y sólo una de $x< y$ ó $y<x$. 
\end{enumerate}
\end{defn}

Dado un conjunto $X$ y un orden total $<$ se puede construir una topología $\topl_<$ de la misma forma que en $ℝ$: intervalos $(a,b)$. Empecemos con ejemplos.

\paragraph{Orden lexicográfico en $ℝ^2$} Este ejemplo es una topología muy visual (ver la figura \ref{figOrdenLex}), importante y rara. Empezamos definiendo qué es ese orden

\begin{defn}[Orden\IS lexicográfico] Denotamos como $<_{Lex}$ al orden que, dado $x=(x_1,x_2), y=(y_1, y_2)$ ambos en $ℝ^2$, se dice que $x<_{Lex} y$ si $x_1 < y_1$ o bien, si $x_1 = y_1$, entonces $x_2 < y_2$.
\end{defn}

\begin{wrapfigure}{r}{0.4\textwidth}
\inputtikz{I_OrdenLexicografico}
\caption{Ilustración del orden lexicográfico en $ℝ^2$. Cualquier punto en $r_2$ es mayor que todos los de $r_1$. En la misma vertical, tenemos que $a<_{Lex}b$.}
\label{figOrdenLex}
\end{wrapfigure}

A partir de esto podemos definir el intervalo lexicográfico de la forma obvia:

\begin{defn}[Intervalo\IS lexicográfico] Si $a,b∈ℝ^2$ con $a<_{Lex}b$ entonces
\[ (a,b)_{Lex} ≝ \{  x ∈ℝ^2\tq a <_{Lex} x <_{Lex} b \} \]
\end{defn}

\begin{prop} \[ \base_{Lex}=\{ (a,b)_{Lex} \tq a,b∈ℝ^2, a<_{Lex}b \} \] es una base para una topología en $ℝ^2$.\end{prop}

\begin{proof}
Como ejercicio, pero queda claro que si $B_1,B_2∈\base$ entonces $B_1∩B_2∈\base$, lo que es todavía mejor que la definición de base.
\end{proof}

\begin{defn}[Topología\IS lexicográfica en $ℝ^2$] En $ℝ^2$, definimos la topología lexicográfica como 

\[ \topl_{Lex} ≝ \topl_{\base_{Lex}} \]
\end{defn}

Podemos ver un ejemplo, considerando en $[0,1]^2$ el intervalo acotado lexicográficamente por $\mathbbold{0} = (0,0)$ y $\mathbbm{1} = (1,1)$, luego
\[ [0,1] × [0,1] = [\mathbbold{0}, \mathbbm{1}]_{Lex} \] 

En esta topología, el abierto más sencillo que contiene a un punto "en el medio" (por ejemplo, el $(0.5, 0.5)$) sería un intervalo "vertical". Sin embargo, para un punto en el borde superior o inferior, el abierto más sencillo sería un rectángulo que se expande hacia la derecha (ver figura \ref{figIntervalosLex}).

\begin{figure}[hbtp]
\centering
\begin{subfigure}[b]{0.4\textwidth}
\inputtikz{I_OrdenLex_AbiertoVertical}
\caption{En un punto dentro del cuadrado, el abierto más sencillo es un intervalo vertical}
\end{subfigure}
~
\begin{subfigure}[b]{0.4\textwidth}
\inputtikz{I_OrdenLex_AbiertoTop}
\caption{En un punto en un borde del cuadrado, el abierto más sencillo es un rectángulo, el intervalo $(a,b)_{Lex}$.}
\end{subfigure}

\caption{Intervalos más sencillos en $[\mathbbold{0}, \mathbbm{1}]_{Lex}$.}
\label{figIntervalosLex}
\end{figure}

Más ejemplos: ¿una bola $A$ en el sentido habitual (ver figura \ref{figBolaLex}) de $ℝ^2$ es abierto en esta topología? Efectivamente: podemos expresarlo como unión de abiertos de la topología, las líneas verticales. 

\begin{figure}[hbtp]
\centering
\inputtikz{I_OrdenLex_Bola}
\caption{La bola $A$ es abierto en la topología lexicográfica si la expresamos como unión de intervalos verticales}
\label{figBolaLex}
\end{figure}

Si lo expresamos de forma simbólica también llegamos a lo mismo. Tomamos

\[ A = \{ (x,y) \tq \left(x-\frac{1}{2}\right)^2 + \left(y-\frac{1}{2}\right)^2 < \frac{1}{100} \]

Así, tendríamos que 

\[ A = \bigcup (a_x, b_x)_{Lex} \]

con 

\begin{gather*}
a_x = \left(x, \frac{1}{2} - \sqrt{\frac{1}{100} - \left(x-\frac{1}{2}\right)^2}\right) \\
b_x = \left(x, \frac{1}{2} + \sqrt{\frac{1}{100} - \left(x-\frac{1}{2}\right)^2}\right)
\end{gather*}

\subsection{Convergencia}

Ahora sigamos con más definiciones.

\begin{defn}[Entorno\IS abierto] Dado \stopl un espacio topológico y $x∈X$, un entorno abierto de $x$ es un abierto $U∈\topl$ tal que $x∈U$.
\end{defn}

\begin{defn}[Convergencia\IS de sucesiones] Sea \stopl un espacio topológico y $\{x_n\}_{n∈ℕ}$ una sucesión en $X$, y $x∈X$. Se dice que $x_n$ converge a $x$ si todo entorno de $x$ contiene todos los términos de la sucesión a partir de un índice determinado.

Dicho simbólicamente

\[ ∀U∈\topl,\; x∈U\; ∃n_U∈ℕ \tq x_n ∈ U \,∀n≥n_U \]
\end{defn}

\paragraph{Ejercicio} En un espacio métrico \sdst con la topología $\topl_{\dst}$ inducida por la distancia, demuestra que \[ x_n \convs x \iff \dst(x,x_n)\convs 0 \iff ∀ε> 0\, ∃n_ε\tq \dst(x,x_n) < ε \]

\paragraph{Ejemplo 1} Veamos ejemplos de convergencia en topologías raras. Tomemos $ℝ$ con $\topl_{[, )}$, y la sucesión $x_n= \frac{-1}{n}$ para $n≥1$. 

Esta sucesión converge en el sentido usual (euclídeo) a cero, pero no en esta topología. Y es que existe un intervalo que es entorno de $0$ (por ejemplo, $[0, 1)$ que no contiene a ningún punto de la sucesión.

\paragraph{Ejemplo 2} Tomemos la sucesión $x_n=\left(\frac{1}{n}, 1\right)$ para $n≥1$ en el espacio topológico $(ℝ^2, \topl_{Lex})$. Converge en la topología usual a $(0,1)$, pero no en la lexicográfica: un entorno vertical del $(0,1)$ no contiene a puntos de la sucesión.


Ahora vamos a ver algunos conceptos en relación a los cerrados, que recordemos eran los complementarios de los abiertos. 

\begin{prop} $ $
\begin{enumerate}
\item $\emptyset, X$ son cerrados.
\item La unión finita de cerrados es cerrado.
\item La intersección de una familia de cerrados es cerrado.
\end{enumerate}
\end{prop}

\paragraph{Ejemplos}  Si tomamos $R_a ≝ \{ (a,y) \tq y ∈ ℝ \}$, es abierto y cerrado en $(ℝ^2, \topl_{Lex})$. 

Es cerrado porque $R_a^c$ es abierto: para todo punto puedo coger un entorno abierto contenido en el conjunto. También podemos verlo como que $R_a^c = \bigcup_{b∈ℝ\setminus \{a\}} R_b$, que es unión de abiertos.

Otro conjunto interesante es $[a,b)$ en $\topl_{[,)}$, que también es abierto (es un elemento de la base) y cerrado (su complementario es $(-∞, a) ∪ [b, ∞)$, ambos abiertos (podemos expresarlos como unión de conjuntos de la base).

Curiosamente, ese conjunto en $(ℝ, \topl_ℝ)$ no es ni abierto ni cerrado: no hay ningún abierto que contenga a $a$ y que esté contenido en $[a,b)$ por lo que no es abierto; y tampoco es cerrado porque $[a,b)^c=(-∞,a) ∪ [b,∞)$ que no es abierto.

\seprule

Vamos a hacer ahora ciertas observaciones sobre convergencia en topologías definidas por una base, para darnos cuenta de los interesantes que pueden resultar al permitirnos probar cosas mirando sólo los elementos de la base.

\begin{prop} Dado un conjunto $X$ y $\base$ una base para una topología en $X$ $\topl_\base$, entonces 

\[ x_n\convs x \iff ∀B∈\base \tq x ∈ B,\; ∃n_B \tq  x_n∈ B ∀ n ≥ n_B \].

Es decir, basta comprobar la definición para entornos de $x$ que son elementos de la base.\end{prop}

\begin{proof}
La implicación a la derecha es trivial: si se cumple para todos los abiertos, se cumple para algunos en particular.

Ahora tenemos que demostrar la implicación a la izquierda. Si $U∈ \topl_\base$ y $x∈U$, entonces por la definición de $\topl_\base$ $∃B∈ \base$ tal que $x∈B⊆U$. Por hipótesis, $∃n_B$ tal que $x_n∈B\; ∀n≥ n_B$, y como $B⊆U$ entonces $x_N∈U \; ∀n≥ n_B$.
\end{proof}

\subsection{Interior, adherencia y frontera de conjuntos}

\subsubsection{Interior}

Sea \stopl un espacio topológico y $W⊆X$ un conjunto cualquiera. Entonces, vamos a definir varios conceptos

\begin{defn}[Interior] Decimos que $x ∈ \mathop{Int}(W)$ si existe un entorno $U$ de $x$ tal que $U⊆W$. Es decir, existe un $U$ abierto tal que $x∈U ⊆ W$.

El interior de un conjunto $W$ se denota como $\intr{W}$.
\end{defn}

Por ejemplo, $\intr{ℚ}$ es vacío tanto en la topología usual como en $\topl_{[,)}$.

El intervalo $[a,b]$ en la topología usual tiene como interior el abierto $(a,b)$. Pero, ¿y en la topología $\topl_{[,)}$? En este caso es: $\intr{[a,b]} = [a,b)$.

Razonando, si $a≤x<b$, entonces $x∈[a,b) ⊆ [a,b]$, luego $x$ está en el interior. Si $x = b$, entonces no existe un intervalo abierto $U$ con $b∈U⊆[a,b]$, porque si $b∈U$ con $U$ abierto entonces existiría un $[α,β)$ con $b∈[α,β) ⊆ U$. En ese caso, el punto medio $\frac{b+β}{2} ∈ [α,β)$, luego entonces también pertenecería a $U$. Sin embargo, es claro que $\frac{b+β}{2} > b$, así que no puede estar en $U$.

\begin{figure}[hbtp]
\centering
\inputtikz{I_ConjuntoWLex}
\caption{Conjunto $W$ en la topología lexicográfica.}
\label{figConjuntoWLex}
\end{figure}

Otro ejemplo: tomemos $([0,1]^2, \topl_{Lex})$. ¿Cuál es el interior del conjunto $W$ que aparece en la figura \ref{figConjuntoWLex}?

Está claro que los puntos de dentro $p_i$ son del interior, ya que siempre podemos encontrar un abierto. También están dentros los puntos $p_b$ de los bordes inferior y superior (con $0≤x<b$), ya que podemos encontrar una banda (como la banda azul) contenida en $W$. Los puntos del lateral izquierdo $p_l$ también están en el interior.

Sin embargo, los puntos $p_d$ en el borde de la diagonal o en el borde inferior con $b ≥ x$ no están en el interior: cualquier abierto que cojamos será una banda como la roja, que se sale de $W$. Tampoco están los puntos del borde lateral derecho.

En definitiva, en la figura \ref{figConjuntoWLex} el interior sería el interior de naranja con los bordes naranjas, excluyendo los marcados en rojo.

\begin{prop} En el caso de una topología generada por una base $\topl_\base$, si $W⊆X$, decimos que $x∈\intr{W}$ si y sólo si existe un elemento $B$ de la base tal que $x∈B⊆W$.
\end{prop}

\begin{proof} Recordamos la definición: \[ x∈ \intr{W} \iff ∃A∈\topl \tq x∈A⊆W \]

La implicación a la izquierda la demostramos diciendo que si $B∈\base$, entonces $B∈\topl_\base$ y $A=B$.

Hacia la derecha, si $x∈\intr{W}$ entonces sabemos que $∃A∈\toplb$ tal que $x∈A⊆W$. Como $A∈\toplb$, entonces $∃B∈\base \tq x∈B⊆A$, y nos queda $x∈B⊆A⊆W$.
\end{proof}

\begin{prop} Sea $\stopl$ un espacio topológico y $W⊆X$. Entonces

\begin{enumerate}
\item $\intr{W} ⊆ W$.
\item $\intr{W} = \bigcup A$ tales que $A∈\topl, A⊆W$ (los abiertos contenidos en $W$.
\item $W$ es abierto si y sólo si $\intr{W} = W$.
\end{enumerate}
\label{propInterior}
\end{prop}

\begin{proof} La primera parte es trivial.

La segunda proposición, si $U$ es abierto y $U⊆W$, entonces $A⊆\intr{U}$, de forma bastante obvia. Por otra parte, si $x∈\intr{W}$, entonces $∃A∈\topl\tq x∈A⊆W$. Entonces $x∈A⊆W⊆\bigcup U$ donde $U$ son abiertos contenidos en $W$.

Y por úlitmo la tercera parte. Como $\intr{W}$ es unión de abiertos (lo acabamos de demostrar) también es abierto. La demostración a la izquierda resulta trivial de esta forma: si $W=\intr{W}$ y $\intr{W}$ es abierto, $W$ es abierto.

Ahora queremos demostrar que si $W$ es abierto, entonces $\intr{W}$ es abierto también. Sabemos que siempre $\intr{W} ⊆ W$, así que sólo nos falta la otra inclusión $W⊆\intr{W}$. Simplemente tenemos que recordar la demostración anterior: si $\intr{W}$ es la unión de todos los abiertos $U$ contenidos en $W$, entonces $W$ es uno de esos abiertos y por lo tanto $W⊆\bigcup U = \intr{W}$.
\end{proof}

\begin{remark}
$\intr{W}$ es el abierto más grande contenido en $W$.
\end{remark}

\subsubsection{Adherencia}

\begin{defn}[Adherencia] Sea \stopl un espacio topológico, $W⊆X$. La adherencia $\adh{W}$ de $W$ se define como todos los entornos de $X$ que "cortan" a $W$. Más formalmente

\[ x ∈ \adh{W} \iffdef A∩W ≠ \emptyset \; ∀ A ∈ \topl \tq x∈A \]
\label{defAdherencia}
\end{defn}

Como ocurría con otras propiedades de conjuntos, si la topología está generada por una base nos vale con comprobar la definición para los elementos de la base.

\begin{prop} En una topología generada por una base \toplb se tiene que \[ x ∈ \adh{W} \iff B∩W≠\emptyset \; ∀B∈\base \tq x∈B \]
\end{prop}

\paragraph{Ejemplos} $ℚ⊆\topl_ℝ$ y en $\topl_{[,)}$. En ambos casos $\adh{ℚ} = ℝ$: cualquier $x∈(a,b)⊆ℝ$ que cojamos se tiene que $(a,b) ∩ ℚ$ no es vacío (y cualquier intervalo $(a,b)$ que escojamos no es vacío porque $x$ está en él). La razón es análoga para $\topl_{[,)}$.

Este ejemplo nos sirve como introducción a la definición de conjunto "denso".

\begin{defn}[Conjunto\IS denso] Dado \stopl espacio topológico y $W⊆X$, se dice que $W$ es denso en $X$ para \topl si $\adh{W} = X$.
\end{defn}

Sigamos con ejemplos. Tomamos $X=[0,1]^2$ con la topología lexicográfica, y $W=\{(x,y) ∈ [0,1]^2 \tq x+y < \frac{3}{2} \}$.

\begin{figure}[hbtp]
\inputtikz{I_AdhConjuntoWLex}
\caption{Adherencia (azul) del conjunto $W$ (naranja) en la topología lexicográfica}
\label{figAdhWLex}
\end{figure}

Está claro que $W⊆\adh{W}$. La diagonal (puntos $p_d$ en la figura \ref{figAdhWLex}) también está en la adherencia: cualquier abierto que cojamos alrededor de ese punto interseca con $W$. Además, cualquier punto $p_b$ del borde superior (salvo la esquina $(1,1)$) está en la adherencia, ya que los abiertos serán bandas como la verde, que intersecan igualmente con $W$.

\begin{prop} Dado \stopl un espacio topológico y $W⊆X$, se tiene que 
\begin{enumerate}
\item $W⊆\adh{W}$.
\item $\adh{W} = \left(\mop{Int}(W^c)\right)^c$. Como consecuencia $\adh{W}$ es cerrado.
\item $\adh{W} = \bigcap_{F⊇W} F$ donde $F$ son todos los cerrados que contienen a $W$.
\item $W$ es cerrado si y sólo si $W = \adh{W}$.\\
\end{enumerate}\end{prop}

Para demostrar estas propiedades, muy similares a las del interior (\ref{propInterior}), usaremos la dualidad abierto-cerrado, unión-intersección e interior-adherencia, que nos será muy útil en el futuro.

\begin{proof}
\begin{enumerate}
\item Claro a partir de la definición (\ref{defAdherencia}).
\item Lo que estamos diciendo es que $\adh{W}^c = \mop{Int}(W^c)$. Empezamos diciendo que si $x∉\adh{W}$, entonces $∃A∈\topl \tq x∈A$ y $A∩W=\emptyset$. Es decir, que $A⊆W^c$, luego $x∈\mop{Int}(W^c)$.
\item Lo deja como ejercicio, usando la segunda definición y las propiedades del interior. 
\item Ídem.
\end{enumerate}
\end{proof}

\subsubsection{Frontera de un conjunto}

\begin{defn}[Frontera] Dado \stopl un espacio topológico y $W⊆X$, se dice que 

\[ x∈\mop{Fr}(W) \iffdef A∩W ≠ \emptyset \y A∩W^c ≠ \emptyset\; ∀A ∈ \topl \tq x∈A \]

Es decir, la frontera son todos los puntos cuyos entornos cortan a $W$ y $W^c$. 
\label{defFrontera}
\end{defn}

Como viene siendo habitual, si tenemos una topología generada por una base, bastan comprobar para los elementos de la base.

\begin{prop} Sea \stopl un espacio topológico y $W⊆X$, entonces

\begin{enumerate}
\item $\mop{Fr}(W) = \adh{W} ∩ \adh{W^c}$, y por lo tanto es cerrado.
\item $\mop{Fr}(W) = \adh{W} \setminus \intr{W}$.
\end{enumerate}
\end{prop}

\begin{proof}
\begin{enumerate}
\item Claro a partir de la definición (\ref{defFrontera}).
\item Antes hemos visto que $\adh{W^c} = \left(\mop{Int}((W^c)^c)\right)^c$, o dicho de otra forma $\adh{W^c} = (\intr{W})^c = X \setminus \intr{W}$. Entonces $\mop{Fr}(W) = \adh{W} ∩ \adh{W^c} = \adh{W} ∩ (X\setminus \intr{W}) = \adh{W} \setminus \intr{W}$.
\end{enumerate}
\end{proof}

\begin{remark} La definición nos permite escribir la adherencia como unión de conjuntos disjuntos: \[ \adh{W} = \intr{W} ∪ \mop{Fr}(W) \]
\end{remark}

\paragraph{Ejemplos} Empezamos con el simple: $ℚ$ en $ℝ$ con la topología usual y con $\topl_{[,)}$. En ambos casos tenemos que \[ \mop{Fr}(ℚ) = \adh{ℚ} \setminus \intr{ℚ} = ℝ \], ya que en cualquier intervalo de longitud mayor que cero hay racionales e irracionales.

Y volvemos a la lexicográfica: sea $W=\{ (x,y) ∈ [0,1]^2 \tq x+y < 3/2 \}$. Sabemos que $\mop{Fr}(W) = \adh{W} \setminus \intr{W}$, y ya habíamos calculado la adherencia (azul en la figura \ref{figFrontWLex}) como

\[ \adh{W} = \left\{ (x,y) ∈ [0,1]^2 \tq x + y ≤ \frac{3}{2} \right\} υ \left\{ (x,1) \tq \frac{1}{2} ≤ x < 1 \right\} \]

Por otra parte (ver figura \ref{figConjuntoWLex}), el interior era

\[ \intr{W} = W \setminus \left\{ (x,0) \tq \frac{1}{2} < x \right\} \]

De esta forma, nos queda que \[ \mop{Fr}(W) = \{ (x,y) ∈ [0,1]^2 \tq \} \]

\begin{figure}[hbtp]
\inputtikz{I_FronteraConjuntoWLex}
\caption{Frontera del conjunto $W$, en verde, con los dibujos de la adherencia (azul, figura \ref{figAdhWLex}) y el interior (rojo, figura \ref{figConjuntoWLex}).}
\label{figFrontWLex}
\end{figure}

Más ejemplos: sea \sdst un espacio métrico y $\topl_{\dst}$ la topología inducida por la distancia. Consideramos $x∈X$ y $r>0$, y $W=\bola(x,r)$. ¿Cuáles son el interior, adherencia y frontera del conjunto?

Está claro que $\intr{W} = \bola(x,r)$ por ser las bolas abiertas. La adherencia es algo más complicada. Es obvio que \[ \adh{W} ⊆ \adh{\bola}(x,r) ≝ \{ y∈X \tq \dst(x,y) ≤ r \} \] (que es la bola cerrada, no la adherencia de la bola), pero pueden ser distintos. En $ℝ^n$ con la distancia euclídea sí son iguales. Pero consideremos la distancia \[ \dst(x,y) = \begin{cases} 0 & x = y \\ 1 & x ≠ y \end{cases} \] que nos genera la topología discreta, en la que todos los conjuntos son a la vez abiertos y cerrados. Si $r=1$, entonces \[ W = \bola(x,1) = \{ x \} \implies \adh{W} = W = \{ x \} \] 

¿Y cuál es la bola cerrada? $\adh{\bola}(x,1) = X$, obviamente no es lo mismo aunque desde luego $\adh{W} ⊆ \adh{\bola}(x,1)$.

Además, en esta topología, tendríamos que $\mop{Fr}(\{ x \} ) = \emptyset$.


\begin{prop} Una proposición trivial: dado \stopl espacio topológico, $A⊆B⊆X$. Entonces se tiene que $\int{A}⊆\int{B}$ y que $\adh{A}⊆\adh{B}$.
\end{prop}

\subsection{Puntos aislados y puntos de acumulación}

\begin{defn}[Punto\IS aislado] Dado \stopl un espacio topológico y $W⊆X$, se dice que $x∈W$ es un punto aislado de $W$ si y sólo si existe un abierto $A$ con $x∈A$ y tal que $A∩W=\{x\}$.
\end{defn}

El punto que no es aislado es un punto de acumulación:

\begin{defn}[Punto\IS de acumulación] Dado \stopl un espacio topológico y $W⊆X$, se dice que $x∈X$ (no es necesario que esté en $W$) es un punto de acumulación de $W$ si y sólo si $∀A∈\topl$ con $x∈A$ se tiene que $A∩(W\setminus\{x\}) ≠ \emptyset$.

El conjunto de todos los puntos de acumulación de $W$ se denota como $W'$. En algún caso (no durante estas clases) se le llamará el \concept[Conjunto\IS derivado]{conjunto derivado} de $W$.
\end{defn}

\paragraph{Ejemplos} Empezamos por lo simple, como siempre. Consideramos a $(a,b)$ en $ℝ$ con la topología usual. No hay ningún punto aislado, y los puntos de acumulación son $[a,b]$.

Si pensamos en $W=(a,b]$ en $ℝ$ con la topología de intervalos $\topl_{[,)}$, vemos que $b$ es un punto aislado. Cualquier abierto  $[b, b + ε)$ sólo interseca en $b$ con $W$. Por otra parte, los puntos de acumulación son $[a,b)$. 

Como siempre, vamos ahora a la topología lexicográfica. Consideramos $W=\{ (x, 0.5) \tq 0≤x ≤ 1\}$. Todos los puntos son aislados. Los puntos de acumulación desde luego no serán puntos del intervalo: son los bordes superiores e inferiores, cuyos entornos son bandas (salvo los que coincidan con bordes laterales). Luego \[ W' = \{ (x,1) \tq x < 1 \} ∪ \{ (x,0) \tq x > 0 \} \] 

Y por último, consideremos $W=\left\{ \frac{1}{n} \tq n ≥ 1, n∈ℕ\right\}$ con la topología usual en $ℝ$. Todos los puntos de $W$ son aislados, y el $0$ es punto de acumulación.

\begin{prop} Sea \stopl un espacio topológico y $W⊆X$. Entonces

\begin{enumerate}
\item $x∈W' \iff x∈ \adh{W\setminus \{x\}}$. Como consecuencia, $W'⊆\adh{W}$.
\item $\adh{W} = W ∪ W'$.
\item $W$ es cerrado si y sólo si $W'⊆W$.
\end{enumerate}
\end{prop}

\subsection{Espacios métricos}

En los espacios métricos con distancias razonables, la adherencia, interior y demás puntos se vuelven más interesantes y fáciles de estudiar.

\begin{prop} Sea \sdst un espacio métrico con la topología inducida por la distancia $\topl_{\dst}$, y sea $W⊆X$. 

\begin{enumerate}
\item $x∈\intr{W} \iff ∃δ> 0 \tq \bola(x,δ) ⊆ W$.
\item $x∈\adh{W} \iff ∀ε>0\; \bola(x,ε) ∩ ≠ \emptyset$.
\item $x∈\mop{Fr}(W) \iff ∀ε>0\; \bola(x,ε) ∩W ≠ \emptyset \y \bola(x, ε) ∩W^c ≠ \emptyset$.
\item $x$ es punto aislado de $W$ si y sólo si $∃δ>0 \tq \bola(x,δ) ∩ W = \{ x\}$.
\item $x∈W' \iff ∀ε>0\; \bolac(x,ε) ∩ W ≠ \emptyset$, donde $\bolac(x,ε) = \bola(x,ε) \setminus \{x\}$.
\item $x∈ \adh{W}$ si y sólo si existe una sucesión $\{x_n\}⊆W\tq x_n\convs x$.
\item $x∈W'$ si y sólo si existe una sucesión $\{x_n\}⊆W$ con $x_n\convs x$ y $x_n≠x\; ∀n∈ℕ$ (excluimos sucesiones constantes a partir de un término.
\end{enumerate}

En las proposiciones 2,3 y 5 basta con comprobarlo para una sucesión $ε_n \to 0$, por ejemplo $ε_n=\frac{1}{n}$.
\end{prop}

\begin{proof}
\begin{enumerate}
\item Queremos demostrar que $x∈\intr{W} \iff ∃δ> 0 \tq \bola(x,δ) ⊆ W$, y para ello recordamos la definición: $x∈\intr{W} \iffdef ∃A∈\topl \tq x∈A ⊆ W$. La implicación hacia la izquierda es sencilla si tomamos simplemente el abierto $A = \bola(x,δ)$.

En la otra dirección, la hipótesis es que $∃A∈\topl \tq x∈A⊆W$. Al ser una topología inducida por la distancia, podemos encontrar una bola $\bola(x,δ)⊆A⊆W$.

\item Son bastante sencillas.
\item 
\item 
\item
\item $x∈\adh{W} \iff ∃x_n∈W \tq x_n\convs x$. Hacia la derecha, podemos ir cogiendo bolas $\bola(x,1/n)∩W ≠ \emptyset$, cada vez más pequeñas, y escoger puntos $x_n∈\bola(x,1/n)$, de tal forma que $\dst(x_n, x) \convs 0$, o de otra forma $x_n \convs x$.

Para el otro lado, es cierto y además para cualquier espacio topológico. Tenemos que $x_n\convs x$ para $x_N∈W$, queremos probar que $x∈\adh{W}$. Por la propia definición de convergencia, podemos encontrar un $n_A$ tal que $x_n∈A$ para todo $n≥n_A$, luego $A∩W ≠ \emptyset$. 

\item Y ya para acabar, hay que demostrar que $x∈W' \iff ∃x_n∈W, x_n≠x \; ∀n∈ℕ$ y $x_n\convs x$. Hacia la izquierda, sabemos que $\bolac(x,\frac{1}{n})∩W ≠ \emptyset$, luego $∃x_n∈W∩\bolac(x, 1/n)$, es decir, que $x_n∈W$ y además $x_n≠x$, luego ya hemos encontrado la sucesión que buscábamos.

La implicación hacia la izquierda es igual en cualquier espacio topológico. $∀A∈\topl$ con $x∈A$, tenemos que $∃n_A \tq x_n∈A \; ∀n≥a$ por definición de convergencia. Luego es claro que $(A\setminus \{ x\}∩W ≠ \emptyset$ y entonces $x∈W'$.
\end{enumerate}
\end{proof}

\begin{remark} Si \sdst es un espacio métrico y $x_n\to x$ y  $x≠z$, entonces $x_n$ no converge a $z$. Es decir, el límite es único.\end{remark}

\begin{proof}
Si $x≠z$, entonces $\dst(x,z)>0$. Sea $δ$ con $0<δ<\frac{\dst(x,z)}{2}$. Por definición de convergencia, existe $n_δ \tq x_n ∈ \bola(x,δ) \; ∀ n≥n_δ$. Por otra parte, $\bola(x,δ) ∩ \bola(x,δ) = \emptyset$, y entonces $x_n∈\bola(z,δ) \,∀n≥n_δ$, luego es imposible que $x_n$ converja a $z$.
\end{proof}

Atención porque esto sólo pasa en espacios métricos: si cogemos un espacio raro el límite puede dejar de ser único. 

Lo que importa realmente del argumento no es que sea métrico, si no que podamos coger dos bolas disjuntas para puntos distintos. Formalicémoslo:


\begin{defn}[Espacio\IS Hausdorff]\label{defHausdorff}
Un espacio topológico \stopl es Hausdorff si $∀x,y \in X$ con $x≠y$ existen $V_x$ entorno de $x$ y $V_y$ entorno de y tal que $V_x∩V_y≠\emptyset$
\end{defn}

\begin{prop} Sea \stopl espacio topológico de Hausdorff. Entonces
\begin{enumerate}
\item Si $x_n\to x$ entonces $x_n$ no converge a $y$ $∀y≠x$ (es decir, el límite de una sucesión, si existe, es único).
\item $\{x\}$ es cerrado $∀x \in X$.
\end{enumerate}
\end{prop}

\begin{proof}
\begin{enumerate}
\item Supongo $x_n\to x$ y $x≠y$

	\begin{enumerate}
	\item Por definición, $\exists V_x,V_y$ entornos de $x$ e $y$ con $V_x∩V_y≠\emptyset$.
	\item Si $x_n\to x$ y $x\in V_x$, por definición de convergencia $\exists n_0$ tal que $x_n\in V_x ∀n ≥n_0$. Luego $x_n\notin V_y ∀n ≥ n_0$ y por lo tanto $ x_n$ no converge a $y$.
	\end{enumerate}

\item Queremos demostrar que  $\adh{\{x\}} = \{x\}$, y lo haremos por doble contenido. El contenido a la izquierda es trivial ($\{x\} ⊆ \adh{\{x\}}$) así que sólo tenemos que hacerlo a la derecha.

Si $y≠x$, entonces existen $V_y$ entorno de $y$ y $V_x$ entorno de $x$ tales que $V_x∩V_y≠\emptyset$, por lo tanto $V_y∩\{x\}=\emptyset$ , de tal forma que $y\notin \adh{\{x\}}$ y $\adh{\{x\}} ⊆ \{x\}$.
\end{enumerate}
\end{proof}

\begin{remark}
La propiedad 2 ($\{x\}$ cerrado $∀x \in X$) es equivalente a decir que $∀ x,y \in X$ con $x≠y$ existen entornos respectivos $V_x$ y $V_y$ tal que $x\notin V_y$ , $y\notin V_x$.
Espacios topológicos con esa propiedad se llaman $T_1$ (Hausdoff es $T_2$)
\end{remark}

Por ejemplo, $(ℝ^m, \topl_{usual})$, $(ℝ^2, \topl_{Lex.})$, o \sdst son espacios Hausdorff.

\subsection{Topología de subespacios}

\begin{defn}[Topología\IS de subespacio]
Dado \stopl espacio topológico y  $S⊆X$, se define la topología de subespacio en $S$ por:
$V\in \topl^{sub} \equiv \topl^{sub}_S \equiv \exists A\in \topl$ tal que $V = A∩S$.
\end{defn}

Ejercicio: comprobar que es una topología.

\begin{prop}
\stopl e.t, $S⊆X$
\begin{enumerate}
\item $C(⊆S)$ es cerrado en $\topl^{sub} \iff \exists F$ cerrado en \stopl tal que $C=F∩S$.
\item Si $\topl = \toplb$, la topología generada por una base $\base$, entonces
	\begin{enumerate}
	\item $\base_S \equiv \{B∩S : B\in \base\}$ es una base para una top en $S$
	\item $\topl^{sub} = \topl_{\base_S}$
	\end{enumerate}
\item Si $\topl = \topl_d$ es la topología inducida por una distancia $\dst(x,y)$ en $X$, entonces $\topl^{sub} = \topl_{\dst|_{S×S}}$, tomando $\dst|_{S×S}$ como la restricción de la distancia al conjunto $S$:
\begin{align*}
	\appl{\dst|_{S×S}}{S×S&}{[0,\infty)} \\
	(x,y)&\longmapsto \dst(x,y)
\end{align*}

\end{enumerate}
\end{prop}

\begin{proof}
\begin{enumerate}
\item $C$ cerrado en $\topl^{sub}$ si y sólo si $S\setminus C ∈ \topl^{sub}$, lo cual es equivalente a a que $∃A∈\topl$  tal que $S\setminus C = A∩S$. 

Si $F≝X\setminus A$, entonces $A=X\setminus F$, luego volviendo a lo que teníamos nos queda que $S\setminus C = (X\setminus F) ∩ S = S\setminus (F∩S)$, equivalente a $C=F∩S$.
\item
\item
\end{enumerate}
\end{proof}

\section{Continuidad} 

En esta sección estudiaremos las funciones $f$ entre dos espacios topológicos $(X,\topl_X)$ y $(Y, \topl_Y)$. Usaremos la forma habitual $\appl{f}{X}{Y}$, aunque para ser más concretos las denotaremos como $\appl{f}{(X,\topl_X)}{(Y, \topl_Y)}$.

\begin{defn}[Función\IS continua]
Sea $x_0∈ X$. Se dice que $f$ es continua en $x_0$ si y sólo si $∀V$ entorno de $f(x_0)$, existe un $U∈\topl_X$ entorno de $x_0$ tal que $f(x)∈V\; ∀x∈U$.

Es decir, que $f(U)⊆ V$ y que $x_0∈ U ⊆\inv{f}(V)$.

Por otra parte, y como hacíamos en otras asignaturas, se dice que una función es continua si es continua en todos los puntos de su dominio.
\end{defn}

\begin{remark} En el caso de espacios métricos $(X, \dst_X), (Y, \dst_Y)$, se dice que $f$ es continua en $x_0$ si y sólo si $∀ε>0\;∃δ>0 \tq \dst_X(x,x_0) < δ \implies \dst_Y(f(x), f(x_0)) < ε$, que es decir de forma más general la forma que hemos puesto de continuidad.

Además, para topologías generadas por bases, basta considerar entornos que están en la base, como siempre.
\end{remark}

Veamos un ejemplo simple sobre continuidad de funciones. Más concretamente, continuidad de funcionales usando sólo la definición de continuidad basada en topologías.

\begin{example} Tomamos $X$ como el conjunto de las funciones continuas $X=C([0,1])≝\{ \appl{f}{[0,1]}{ℝ} \tq f ∈ C^1\}$, y definimos la distancia como
\begin{align*}
\md{f}_{∞} &≝ \max_{x∈[0,1]} \abs{f(x)} \\ 
\dst(f,g) &≝ \md{f-g}_{\infty}
\end{align*}

A partir de esto definimos el siguiente funcional: 
\begin{align*}
\appl{F}{(X, \dst_X)&}{(ℝ, \topl_{usual})} \\
f&\longmapsto F(f) ≝ \int_0^1f(x) \dif x 
\end{align*}

Afirmamos que $F$ es continua es $f_0 ∈ X$. Podemos encontrar $\epsilon$ y $\delta$ de la siguinte forma:

\begin{align*}
\dst_{ℝ}(F(f), F(f_0)) &= \abs{F(f) - F(f_0)} = \abs{\int_0^1 f(x) \dif x - \int_0^1 f_0(x)\dif x} = \\
&= \abs{\int_0^1 f(x) - f_0(x) \dif x} \leq \int_0^1 \abs{f(x)-f_0(x) \dif x} \leq \\
&\leq \int_0^1 \md{f(x) - f_0(x)}_{\infty} \dif x \leq \md{f - f_0}_{\infty} = \dst_{X}(f, f_0) < \delta
\end{align*}

Luego, como $\dst_{\mathbb{R}}(F(f), F(f_0)) < \epsilon$, $\forall \epsilon > 0$ , tenemos que:

\begin{align*}
\delta = \epsilon \wedge dst_{X}(f, f_0) < \delta = \epsilon \implies \dst_{\mathbb{R}}(F(f), F(f_0)) \leq dst_{X}(f, f_0) - \delta
\end{align*} \qed

De hecho, hemos obtenido un resultado más fuerte: el $\delta$ no depende del punto, luego F es \underline{uniformemente contínua}.

\end{example}

A partir del ejemplo, podemos desarrollar nuevas propiedades:

\begin{prop} Sean $(X_1, \topl_1), (X_2, \topl_2)$ espacios topológicos y $\appl{f}{X_1}{X_2}$. Entonces

\begin{enumerate}
\item $f$ es continua si y sólo si la imagen inversa de un abierto en $X_2$ es abierta en $X_1$. Es decir, si y sólo si $∀A∈\topl_2 \; \inv{f}(A) ∈ \topl_1$.
\item $f$ es continua si y sólo si la imagen inversa de un cerrado en $X_2$ es cerrada en $X_2$.
\item $f$ es continua si y sólo si $f(\adh{W}) ⊆ \adh{f(W)}\; ∀W⊆X_1$.
\end{enumerate}
\end{prop}

\begin{proof}
\begin{enumerate}
\item Empezamos tomando como hipótesis que $f$ es continua en $x\; ∀x∈X_1$. Tomamos $A∈\topl_2$ y estudiamos su imagen inversa $\inv{f}(A)$ para ver si está en$\topl_1$. Si $x∈\inv{f}(A)$, entonces $f(x) ∈ A$.  Si $A$ es abierto en $X_2$ y $f(x)∈A$, entonces por la definición de continuidad de $f$, $∃V_x$ entorno de $X$ val que $f(V_x)⊆A$. 

Es decir, que $x∈V_x⊆\inv{f}(A)$. Como esto pasa para todo punto, podemos escribir $\inv{f}(A)$ como la unión de todos los $V_x$, la unión de abiertos es abierta y entonces ya tenemos que $A$ es abierto.

En el otro sentido, tomamos como hipótesis que la imagen inversa de un abierto es abierta. Sea $x∈X$, queremos saber si $f$ es continua en $x$. 

La definición de continuidad nos decía que dado un entorno $W$ cualquiera de $f(x)$, existía un entorno $V_x$ de $x$ tal que $f(V_x) ⊆ W$. Dicho de otra forma, cercanía en el dominio implica cercanía en la imagen.

Entonces, partiendo de dos premisas ($W∈\topl_2 \implies \inv{f}(W) ∈ \topl_1$, $f(x)∈W \iff x∈\inv{f}(W)$) nos queda que $V=\inv{f}(W)∈ \topl_1$, $x∈V$ y $f(V)⊆W$. % No me queda muy claro esto pero lo dejo así.

\item Queremos demostrar que la imagen inversa de cerrados es cerrada, y vamos a usar la propiedad de que el complementario de un cerrado es abierto. Vemos claramente que $\inv{f}(X_2\setminus A ) = X_1 \setminus \inv{f}(A)$. A partir de esto es muy sencillo probarlo tomando complementarios, y no lo voy a copiar.

\item Tomamos como hipótesis que $f$ es continua, y queremos demostrar que $f(\adh{W}) ⊆ \adh{f(W)}\; ∀W⊆X_1$. Está claro que $\adh{f(W)}$ es cerrado, y que al ser $f$ continua entonces $\inv{f}(\adh{f(W)})$ es cerrado también. Además, es seguro que $W⊆\inv{f}(\adh{f(W)})$. Uniendo las dos cosas, tenemos que $\adh{W} ⊆ \inv{f}(\adh{f(W)})$. 
\end{enumerate}
\end{proof}

Vamos a ir ahora a por algunas propiedades sobre la composición de funciones.

\begin{prop} Sean $(X,\topl_X), (Y, \topl_Y), (Z, \topl_Z)$ espacios topológicos. Entonces 

\begin{enumerate}
\item Si $\appl{f}{X}{Y}$ y $\appl{g}{Y}{Z}$  son continuas entonces $\appl{g○f}{X}{Z}$ también lo es.
\item Si $\appl{f}{X}{Y}$ es constante entonces $f$ es continua.
\item Si $\appl{f}{X}{Y}$continua y $S$ subespacio de $X$ (es decir, $S⊆X$ con $\topl^{sub}$), entonces $\appl{f|_S}{S}{Y}$ es continua.
\item Sea $\appl{f}{X}{Y}$ y $W$ tales que $f(X) ⊆ W ⊆ Y$. Denotemos $\appl{f^W}{X}{W}$ con $W$ en la topología de subespacios. Entonces $f$ es continua si y sólo si $f^W$ es continua.
\end{enumerate}
\end{prop}

\begin{proof}
\begin{enumerate}
\item Sea $A$ abierto en $Z$, entonces tenemos que demostrar que $(g○f)^{-1}(A)$ es abierto en $X$. Sabemos que $\inv{(g○f)}(A) = \inv{f} ()$ y nosequé.
\end{enumerate}
\end{proof}

\begin{remark} Si tenemos una aplicación $\appl{f}{(X,\topl_X)}{(Y, \topl_Y)}$ donde $\topl_Y = \toplb$ con $\base$ una base, entonces $f$ es continua si y sólo si $\inv{f}(B)∈\topl_X\;∀B∈\base$.
\end{remark}

\begin{defn}[Homeomorfismo] Un homeomorfismo entre espacios topológicos $(X,\topl_X)$, $(Y, \topl_Y)$ es una aplicación biyectiva $\appl{f}{X}{Y}$ y tal que tanto $f$ como $\inv{f}$ son continuas. En este caso se dice que $(X,\topl_X)$, $(Y, \topl_Y)$ son homeomorfos.
\end{defn}

\begin{defn}[Propiedad\IS topológica] Una propiedad de un espacio topológico $(X, \topl_X)$ es topológica si la comparten todos los espacios topológicos homeomorfos a $(X, \topl_X)$.\end{defn}

\begin{remark} La topología de subespacios usual en $\bola(0,1)$ es $\topl_{\dst}$ donde $\dst$ es la distancia euclídea en $\bola(0,1)$. \end{remark}

Podemos demostrar que $ℝ^2$ y $\bola(0,1)$ son homeomorfos, tomando la biyección 

\begin{align*}
\appl{f}{ℝ^2&}{\bola(0,1)} \\
\vx&\longmapsto \frac{\vx}{1+\md{\vx}} = \vy
\end{align*}

Demostramos que es biyectiva: \[ \md{\vy} = \frac{\md\vx}{1 + \md{\vx}} \implies \md{\vx} = \frac{\md{\vy}}{1- \md{\vy}} \], luego \[ \vx = \frac{\vy}{1-\md{\vy}} \]. 

Otro ejemplo: vamos a demostrar que $[0,1]$ y $(0,1)$ no son homeomorfos. La propiedad que nos va a interesar sería la compacidad, pero todavía no tenemos la maquinaria para hacerlo. 

Lo que haremos será ver qué ocurre si quitamos un punto. Cualquier punto que quitemos en $(0,1)$ nos dejará dos trozos no conexos. Ahora bien, si lo hacemos en $[0,1]$, podemos quedarnos con un trozo conexo si el punto es $0$ o $1$.

Vamos a demostrar que no son homeomorfos viendo que si $\appl{f}{[0,1]}{(0,1)}$ es continua, entonces no es sobreyectiva.

Un pequeño aparte: tenemos que entender que cada intervalo está con la topología del subespacio. En el caso de $\topl_{[0,1]}^{sub}$, la topología del subespacio sería la generada por los elementos de la base de $\topl_ℝ$ intersección $[0,1]$. Es decir, la base es \begin{multline*} \base = \{ (a,b) ∩ [0,1] \tq a < b, a,b∈ℝ \} \equiv \\ \equiv \left\{ [0,1], [0,b), (a, 1], (c,d) \tq 0 < b,a < 1,\; 0<c<d<1 \right\} \end{multline*}

Volviendo a lo que íbamos: no tendremos problema por parte de la imagen inversa. En este caso, la imagen inversa de un abierto $(0,1)$ es un abierto: $[0,1]$ es abierto en la topología de su subespacio (es el total).

Donde sí vamos a lograr algo es viendo que $f$ continua en $[0,1]$ alcanza un máximo. Es decir, $∃x_0 ∈ [0,1]$ tal que $f(x) ≤ f(x_0)\; ∀x∈[0,1]$. 

Por otra parte, es obvio que $f(x_0)∈(0,1)\implies f(x_0) < 1$. Juntando estas dos conclusiones llegamos a que $(f(x_0), 1) ∩ f([0,1]) = \emptyset$. Es decir, $f$ no es sobreyectiva (por ejemplo, $\frac{1+f(x_0)}{2}$ no está en la imagen).

Otra forma sería ver que si $f$ fuese inyectiva, tendría que ser monótona creciente o decreciente. Podríamos suponer sin pérdida de generalidad que fuese creciente, y entonces $f(0) < f(x)\; ∀x∈(0,1]$. Sin embargo, siempre podríamos encontrar un $y ∈ (0, f(0))$ por lo que no sería sobreyectiva.

\paragraph{Propiedad topológica Hausdorff} Recordamos los espacios Hausdorff (ver \ref{defHausdorff}).  Queremos demostrar que si $(X,\topl_X)$ es Hausdorff y  $(X,\topl_X)$ es homeomorfo a $(Y, \topl_Y)$, entonces $(Y, \topl_Y)$ es Hausdorff igualmente.

Tomamos $x_1∈V_1$ y $x_2∈V_2$, ambos en $X$, con $y_j=f(x_j)$. Hay que comprobar que $y_j∈f(V_j)$, que $f(V_j)$ es abierto en $Y$ y que la intersección $f(V_1) ∩ F(V_2) = \emptyset$.

\begin{remark} Si $f$ es un homemorfismo entonces $\inv{f}$ es continua. Dicho de otra forma, la imagen inversa de la inversa de un abierto es abierto. % WTF.

También se puede decir que $f$ es homeomorfismo si $f$ es biyectiva, y que la imagen inversa de $\inv{f}$ de un abierto es abierto, y que $f$ de un abierto es abierto también.
\end{remark}

\begin{defn}[Función\IS abierta] Sean $(X,\topl_X)$, $(Y, \topl_Y)$ espacios topológicos y $\appl{f}{X}{Y}.$ Se dice que $f$ es abierta si y sólo si $∀A∈\topl_X$ se tiene que $f(A) ∈ \topl_Y$.

Por otra parte, se dice que una función es una \concept[Función! cerrada]{función cerrada} si y sólo si para todo $C$ cerrado en $X$, $f(C)$ es cerrado en $Y$.
\end{defn}

Por ejemplo, si $Y$ es Hausdorff (\ref{defHausdorff}) y $f$ es constante, entonces es cerrada. La razón es que en un espacio Hausdorff, un único punto siempre es un conjunto cerrado.

Las dos definiciones no son excluyentes. Un homeomorfismo es aplicación abierta y cerrada, por ejemplo. También podríamos construir una aplicación $f$ abierta y cerrada sin que sea necesario homeomorfismo. Podemos coger una $f$ constante e $Y$ con la topología discreta, o con $\topl_Y = \{ \emptyset, \{p\}, \{p\}^c, Y\}$.

\section{Topología producto}

Ahora consideraremos dos espacios topoloógicos $(X_1, \topl_1), (X_2, \topl_2)$, y estudiaremos la topología en $X_1×X_2$.

\begin{prop} Sea $\base = \{ V_1 × V_2 \tq V_i ∈ \topl_i \}$. $\base$ es una base para una topología en $X_1 × X_2$. $\topl_\base$ es la topología producto y se denota $\topl_1 \otimes \topl_2$ (habitualmente $\topl_1× \topl_2$).
\end{prop}

\begin{proof}
Vamos a demostrar que efectivamente es una topología. Está claro que $\bigcup_{B∈\base} B = X_1 × X_2$, ya que $X_1$ y $X_2$ son el total y están en sus respectivas topologías, por lo que $X_1× X_2∈\base$.

Por otra parte, queremos demostrar que la intersección también está.  Si tenemos $B, C ∈ \base$ y  $x∈B∩C$, entonces existe un $\tilde{B}∈\base$ tal que $X∈\tilde{B}⊆B∩C$.

Para ello usaremos que $(V_1×V_2) ∩ (\hat{V}_1 × \hat{V}_2) = (V_1 ∩ \hat{V}_1) × (V_2 ∩ \hat{V}_2)$, luego $B∩C ∈ \base$. Podría copiar la demostración pero no me da tiempo.
\textcolor{red}{Entonces es que no podias -.-}

\end{proof}

La topología producto no se limita a sólo dos espacios topológicos.

\begin{prop} Más generalmente, si $\{X_j,\topl_j\}_{j=1,\dotsc, m}$ es un conjunto de espacios topológicos, entonces $\base = \{ V_1× V_2 × \dotsb × V_m \tq V_j ∈ \topl_j\}$ es una base para una topología producto $\topl_1 \otimes \topl_2 \dotsb \otimes \topl_m$.
\end{prop}

Como ejercicio, podemos considerar tres espacios topológicos, y estudiar si la topología producto es asociativa, es decir, si 

\[ \topl_1 \otimes \topl_2 \otimes \topl_3 = (\topl_1 \otimes \topl_2)\otimes \topl_3 = \topl_1 \otimes (\topl_2 \otimes \topl_3 ) \]

Otro ejemplo, ¿son iguales $\topl_{ℝ^2} = \topl_ℝ \otimes \topl_ℝ$? En el primer caso, la base $\base_1$ de $ℝ^2$ son elementos de la forma $(a_1,b_1) × (a_2, b_2)$, es decir, rectángulos. La segunda base $\base_2$ está formada por los elementos $V_1 × V_2$ donde $V_i ∈ \topl_ℝ$.

Para demostrarlo vamos a ver lo de siempre: las bases molan. Si tenemos $\base$ una base y $\topl$ una topología con $B⊆\topl$, entonces $\topl_\base⊆\topl$. Es decir, $\topl_\base ⊆ \topl \iff \base ⊆ \topl$. 

Como consecuencia de esa observación, si $\base_1,\base_2$ son bases entonces $\topl_{\base_1} ⊆ \topl_{\base_2} \iff \base_1 ⊆ \topl_{\base_2}$ y, por tanto \[ \topl_{\base_1} = \topl_{\base_2} \iff \base_1 ⊆ \topl_{\base_2} \y \base_2 ⊆ \topl_{\base_1} \]

Con esto ya podemos volver a nuestra vida normal y demostrar que $\topl_{ℝ^2} = \topl_ℝ \otimes \topl_ℝ$. Si $B∈\base_1$ con $B=(a_1,b_1) × (a_2, b_2)$, entonces como $(a_1,b_1)$ y $(a_2, b_2)$ son abiertos en $\topl_ℝ$, $B∈\base_2$ y $\base_1 ⊆ \base_2 ⊆ \topl_{\base_2}$. Con esto hemos demostrado el contenido hacia la izquierda ( $\topl_{ℝ^2} ⊆ \topl_ℝ \otimes \topl_ℝ$).

Vamos ahora para el otro lado. Si $B∈\base_2$ con $B=V_1 ×V_2$, con $V_i ∈ \topl_ℝ$, tenemos que $V_1 = \bigcup_{j∈ J_1}(a_j, b_j)$ y análogamente con $V_2 = \bigcup_{k∈K_2} (c_k, d_k)$. Entonces 

\[ V_1 × V_2 = \bigcup_{\substack{j∈J_1 \\ k∈K_2}} (a_j, b_j) ×(c_k, d_k) \], por lo tanto $V_1×V_2$ es unión de elementos de $\base_1$ y por lo tanto está en $\topl_{\base_1}$, luego $\base_2 ⊆ \topl_{\base_1}$ y tenemos la inclusión para el otro lado, y entonces $\topl_{\base_1} = \topl_{\base_2}$.

Pero la justificación de esto no sólo nos vale para $ℝ$, sino que puede ser mucho más general si no escribimos explícitamente los intervalos de $ℝ$, que no son más que abiertos de la topología.

\begin{prop} Sean $(X, \topl_{\base_X})$, $(Y, \topl_{\base_Y})$. Entonces \[ \topl_{\base_X} \otimes  \topl_{\base_Y} = \topl_\base \] donde \[ \base =\left\{ B × \hat{B} \tq B ∈ \base_X, \hat{B} ∈ \base_Y \right\} \].
\end{prop}

\begin{prop} Sean $(X_1, \dst_1)$ y $(X_2, \dst_2)$ espacios métricos. Entonces $\topl_{\dst_1} \otimes \topl_{\dst_2} = \topl_{\dst}$ donde \[ \dst\left((x_1, x_2), (y_1, y_2)\right) = \dst_1(x_1, y_1) + \dst(x_2, y_2)\]
\end{prop}

La demostración se deja como ejercicio para el lector, aunque es parecido al ejercicio 18 de la hoja 1.

\subsection{Funciones continuas y topología producto}

La gran ventaja de la topología producto nos va a venir a la hora de comprobar continuidad de funciones en espacios producto. Vamos a verlo.

\begin{prop} Sean $(X_1, \topl_1), (X_2, \topl_2)$ espacios topológicos. Entonces

\begin{enumerate}
\item Sean $\appl{P_j}{(X_1×X_2, \topl_1×\topl_2)}{(X_j, \topl_j)}$ con $P_1((x_1, x_2)) = x_1$ y $P_2((x_1, x_2)) = x_2$ las proyecciones. Entonces $P_1$ y $P_2$ son continuas.

\item Sea $\appl{f}{(X, \topl)}{(X_1×X_2, \topl_1 \otimes \topl_2)}$ tal que $x\longmapsto f(x) = (f_1(x), f_2(x))$. Entonces $f$ es continua si y sólo si $f_1, f_2$ son continuas.

\item Si $\appl{f, g}{(X, \topl)}{(ℝ, \topl_ℝ)}$ son continuas entonces $f\pm g$, $fg$ y $\frac{f}{g}$ son continuas, en el último caso suponiendo que $g(x) ≠ 0$.
\end{enumerate}

Las propiedades se mantienen para un producto finito de espacios.
\end{prop}

\begin{proof}
\begin{enumerate}
\item Si $P_1$ es continua, entonces $\inv{P_1}(A_1) ∈ \topl_1 \otimes \topl_2\quad ∀ A_1 ∈ \topl_1$. Tenemos que ver primero qué es $\inv{P_1}(A_1)$. Al ser la proyección, $\inv{P_1}(A_1) = A_1 × X_2$, que es claramente un abierto en la topología producto (es un elemento de la base).

\item Podemos decir que $f_1 = P_1 ○ f$ y $f_2 = P_2 ○ f$. La implicación a la derecha es obvia: si $f$ es continua, como $P_1$ y $P_2$ son continuas entonces $P_1 ○ f$ y $P_2 ○ f$ son continuas.
\end{enumerate}
\end{proof}
 
 %
\begin{prop} Sean $(X_i, \topl_i)$ con $i=1,2$ espacios topológicos, y consideramos las proyecciones $\appl{p_i}{X_1×X_2}{X_i}$ las proyecciones. $p_1$ y $p_2$ son abiertas, suponiendo en $X_1×X_2$ la topología producto.

\end{prop}

\begin{proof}
Tenemos que demostrar que si $A∈ \topl_1 \otimes \topl_2$, entonces $p_1(A) ∈ \topl_1$. Los abiertos de la topología producto pueden ser bastante raros, pero el caso básico ($A = V_1 × V_2$), donde $V_1$ y $V_2$ son abiertos, lo podemos justificar rápidamente porque $p_1(V_1) = V_1$, que es abierto.

Pero consideremos el caso general, en el que $A∈\topl_1 \otimes \topl_2 = \topl_\base$. Entonces \[ A = \bigcup_{j∈J} V_1^j × V_2^j \]. Ahora bien, sabemos que la imagen de la unión de conjuntos es la unión de las imágenes, luego $p_1(A)$
 será \[ p_1(A) = \bigcup_{j∈J} p_1(V_1^j × V_2^j) = \bigcup_{j∈J}V_1^j\], que es abierto.
\end{proof}

\begin{remark} Las proyecciones $p_i$ no son cerradas en general. Por ejemplo, en $ℝ^2$ con la topología usual, tendríamos que escoger un $F$ en $ℝ^2$ que fuese cerrado pero con $p_1(F)$ no cerrado en $ℝ$. Por ejemplo, $F= \{ (x_1, x_2) \tq x_1x_2 ≥ 1\}$, pero $p_1(F) = (0, +∞)$. 

Puede parecer un poco contraintuitivo que $F$ sea cerrado, pero lo es: $f(x_1, x_2) = x_1x_2 - 1$ es continua y $F = \inv{f}([0, ∞))$, y la imagen inversa de un cerrado es cerrado.
\end{remark}

\begin{remark} $\topl_1\otimes\topl_2$ es la más pequeña para la que las proyecciones son continuas. La demostración está en ver qué es $\inv{p_1}(V_1) ∩ \inv{p_2}(V_2)$.
\end{remark}

\chapter{Propiedades de los espacios topológicos}

Hasta ahora hemos estado aprendiendo sólo el lenguaje, lo básico. Ahora vamos a ir a por la topología de verdad, empezando por conexión y compacidad, dos conceptos muy potentes. Hasta ahora hemos podido hablar sólo de continuidad y convergencia, pero queremos más.
\textcolor{red}{oh si papi, dame más}

\section{Conexión}

Empecemos con un ejemplo. Si consideramos $A=(-1, 0) ∪ (1,0)$ y $B=(2,3)$, vemos que no son homeomorfos. Para ello, necesitaríamos encontrar una función $f$ continua e inyectiva, luego tiene que ser monótona. La imagen de un abierto es un abierto, así que deberíamos poder escribir $B$ como unión de dos abiertos disjuntos, pero no podemos: podríamos coger el ínfimo o supremo de cualquiera de ellos y esos no estarían en el intervalo.


\begin{defn}[Conexión]
	Sea $(X, \topl)$ un espeacio topologico. Tenemos 2 definiciones:

	\begin{enumerate}
		\item X es conexo si no existen $V_1, V_2$ abiertos $(V_i \in \topl)$ tales que $V_1 \cap V_2 = \emptyset$, $V_1 \cup V_2 = X$, y $V1 ≠ \emptyset ≠ V_2$\\
		($X$ no conexo $\equiv ∃V_1,V_2, \ldots$)

		\item Sea $W ⊆ X$, $W$ es conexo $\equiv (W, \topl^{sub})$ es conexo.
	\end{enumerate}
\end{defn}


\begin{remark}
	W no conexo quiere decir que $∃V_1, V_2 \in \topl^{sub}$ no vacíos y tal que $V_1 \cap V_2 = \emptyset$, $V_1 \cup V_2 = W$

	$V_1, V_2 \in \topl^{sub} \Leftrightarrow V_i=A_i \cap W, A_i \in \topl$

	Que W sea no conexo quiere decir que $∃A_1, A_2 \in \topl$ no vacíos con $A_1~\cap~W~≠~\emptyset, A_2~\cap~W~≠~\emptyset$, y tal que $ A_1 \cap A_2 \cap W = \emptyset$, $(A_1 \cap W) \cup (A_2 \cap W) = (A_1 \cup A_2) \cap W = W \implies W ⊆ A_1 \cup A_2$
\end{remark}


\begin{remark}
	$A_1$ y $A_2$ se pueden cortar fuera del $W$
\end{remark}

\begin{remark}
	Acerca de la 1 definición dada para conexión:\\
	$V_2=V_1^c \implies$ (siendo $V_1$ abierto) $V_2$ es cerrado en $(X,\topl)$, es decir, es abierto y cerrado. Lo mismo le sucede a $V_1$.
\end{remark}


\begin{remark} 
	En espacios topologicos que no son conexos puede haber conjuntos abiertos y cerrados.
\end{remark}


\begin{prop}
	Sean $(X,\topl_X),(Y,\topl_Y)$ espacios topólogicos y $\appl{f}{X}{Y}$ continua.
	Si $W ⊆ X$ es conexo, entonces $f(W)$ es conexo.
\end{prop}

\begin{corol}
 Ser conexo es una propiedad topologica
\end{corol}

\begin{proof}[Proposición]
	Supongamos que $f(W)$ no es conexo, entonces:\\
	$∃A_1, A_2 ∈ \topl_Y$ no vacíos, con $A_i \cap f(W) ≠ \emptyset, i=1,2$; tales que $A_1 \cap A_2 \cap f(W) = \emptyset$ y $f(W) ⊆ A_1 \cup A_2$

	Sea $B_j = f^{-1}(A_j) j=1,2,\ldots$ entonces:
	\begin{enumerate}
		\item $B_1, B_2 \in \topl_x$ ($f$ continua)

		\item $B_j \cap W ≠ \emptyset$, y $f^{-1}(A_j \cap f(W)) ⊆ B_j \cap W ≠ \emptyset$

		\item $B_1 \cap B_2 \cap W = \emptyset$\\
		si $x ∈ B_1 \cap B_2 \cap W$, entonces $f(x) ∈ f(B_1) \cap f(B_2) \cap f(W)$. Sabemos que $f(B_1) ⊆ A_1$ y $f(B_2) ⊆ A_2$\\
		Entonces $f(x) ∈ A_1 \cap A_2 \cap f(W) ≠ \emptyset$ y llegamos a contradicción, ya que al principio supusimos que $A_1 \cap A_2 \cap f(W) = \emptyset$.
		\begin{remark}
			$f(f^{-1}(A)) \subseteq A$, pueden ser distintos, por ejemplo cuando $f = cte$.\\
			Para que $f(f^{-1}(A))=A$, $f$ ha de ser sobreyectiva 
		\end{remark}

		\item $W ⊆ B_1 \cup B_2$\\
		\underline{Razón}: $f(W) ⊆ A_1 \cup A_2 \implies f^{-1}(f(W)) ⊆ f^{-1}(A_1 \cap A_2) = f^{-1}(A_1) \cup f^{-1}(A_2) = B_1 \cup B_2$\\
		Es decir, $W ⊆ B_1 \cup B_2$
		\begin{remark}
			$M ⊆ f^{-1}(f(M))$ pero pueden ser distintos.\\
			En caso de que $f$ sea inyectiva, $M = f^{-1}(f(M))$.
		\end{remark}
	\end{enumerate}

	1,2,3 y 4 $\implies W$ no es conexo. Por tanto llegamos a una contradicción.
\end{proof}


\begin{proof}[Corolario]
	$(X, \topl_X)$ conexo y $\appl{f}{X}{Y}$ homeomorfismo (y por tanto continua) $\implies Y = f(X)$ es conexo.\\
	Por tanto la conexión es propiedad topológica. 
\end{proof}


\begin{prop}
	Sea $(X,\topl)$ un espacio topologico:

	\begin{enumerate}
		\item $X$ es conexo $\Leftrightarrow ∄F ⊆ X$ tal que $F$ es abierto, cerrado y $F≠ \emptyset, X$

		\item
		\[\left.
			\begin{array}{cc}
				W ⊆ X \text{\ conexo} \\ \\
				W ⊆ A \cup B \\ \\
				A,B \text{\ abiertos disjuntos}
			\end{array}
		\right\} \implies W ⊆ A \text{\ ó \ } W ⊆ B \]
	\end{enumerate}
\end{prop}

\begin{proof}
	\begin{enumerate}
		\item $V_1 = F, V_2 = F^c$ en la definición
		\begin{remark}
			$F ≠ \emptyset$, $X$ por definicion es subconjunto propio.
		\end{remark}

		\item Si no es cierto que $V_1 = A \cap W$, $ V_2 = B \cap W$ muestran que W no es conexo.\\
		\[\left.
			\begin{array}{cc}
				W \nsubseteq A \\ \\
				W \nsubseteq B \\ \\
				W ⊆ A \cup B
			\end{array}
		\right\} \implies
		\left.
			\begin{array}{cc}
				A \cap W ≠ \emptyset \\ \\
				B \cap W ≠ \emptyset
			\end{array}
		\right\} \implies 
		A \cap B = \emptyset \implies A\cap B \cap W = \emptyset
		\]

		Por tanto $W ⊆ A \cup B$
	\end{enumerate}
\end{proof}
 

\begin{example}[$ℝ \setminus \{0\}$, $\topl_{usual}$ NO conexo.\\]
	$ℝ \setminus \{0\} = (-∞, 0) \cup (0, ∞)$\\
	$V_1 = (-∞ , 0)$ y $V_2 = (0, ∞)$ son abiertos en  $\topl^{sub}$ no vacíos, disjuntos y cuya unión es $ℝ \setminus \{0\}$. Esto implica que no es conexo.
\end{example}


\begin{prop}
	$(ℝ, \topl_{usual})$, $W ⊆ ℝ$

	$W$ es conexo $\Leftrightarrow W$ es un intervalo ($I ⊆ ℝ$ es un intervalo $\equiv ∀a,b,c$ con $a,b ∈ I$, $a<c<b$ se tiene $c ∈ I$)
\end{prop}


\begin{proof}
	\begin{enumerate}
		\item $\implies)$\\
		Hipótesis: $W$ es conexo\\
		Si no es intervalo, $∃ a,b,c$ con $a,b ∈ W$, $a<c<b$, $c ∉ W$

		Entonces $a ∈ V_1 = (-∞, c) \cap W$\\
		$b ∈ V_2 = (c, ∞) \cap W$

		Estos dos conjuntos son abiertos en $\topl^{sub}$ no vacíos disjuntos y su unión es $W$. Por tanto llegamos a una contradicción, ya que $W$ es conexo y no se debería poder expresar como unión de abiertos disjuntos.
	\end{enumerate}
\end{proof}

\appendix
\chapter{Ejercicios}

\section{Hoja 1}


\begin{problem}[6] Sea $\appl{g}{X}{Y}$ una aplicación entre dos conjuntos.

\ppart Demostrar que si $\topl$ es una topología en $X$ entonces \[ \mathcal{S} = \{ E ⊆ Y \tq \inv{g}(E) ∈ \topl \} \] es una topología en $Y$.
\ppart Demostrar que si $\mathcal{S}$ es una topología en $Y$ entonces \[ \mathcal{U} = \{ \inv{g}(E) \tq E ∈ \mathcal{S} \} \]es una topología en $X$.

\solution
\spart Vamos a demostrar que es una topología, para lo cual tenemos que comprobar las 3 propiedades (ver \ref{defTopología}):

\begin{enumerate}
\item $Y\in \mathcal{S} \dimplies g^{-1}(x)\in \mathcal{T}$, por ser $g^{-1}(y)=x$

El razonamiento de porqué $\emptyset \in \mathcal{S}$ es igual.

\item $A,B \in \mathcal{S} \dimplies g^{-1}(A),g^{-1}(B) \in \mathcal{T}$.

$A\cap B \in\mathcal{S} \dimplies g^{-1}(A\cap B) \in \mathcal{T}$.

Hemos llegado a que para demostrar la segunda propiedad, tenemos que demostrar $g^{-1}(A),g^{-1}(B) \in \mathcal{T} \implies g^{-1}(A\cap B) \in \mathcal{T}$.

Para ello: $g^{-1}(a\cap B) = g^{-1}(A)\cap g^{-1}(B)$. No es difícil convencernos de esta igualdad. En caso de tener dudas, demostrar las 2 inclusiones (una en cada sentido). Esto para imágenes directas no funciona.

\item Demostramos ahora que la unión de abiertos está en la topología. Si $A, B ∈ \mathcal{S}$, entonces $\inv{g}(A), \inv{g}(B) ∈ \topl$. Como $\topl$ es topología, tenemos que $\inv{g}(A) ∪ \inv{g}(B) ∈ \topl$, lo que implica de forma obvia que $\inv{g}(A∪B) ∈ \topl$ y por lo tanto $A∪B ∈ \mathcal{S}$.
\end{enumerate}

\spart

\end{problem}

\begin{problem}[9] Se consideran las siguientes familias de conjuntos en $ℝ$:

\begin{gather*}
\base_{\leftarrow} = \{ (-∞, b) \tq b ∈ ℝ \} \\
\base_{\rightarrow} = \{ (a,∞) \tq  ∈ ℝ \} 
\end{gather*}

\ppart Demostrar que cada familia es una base de una topología sobre $ℝ$.
\ppart Comparar esas topologías.
\ppart Demostrar que la topología generada por $\base_{\leftarrow} ∪ \base_{\rightarrow}$ es la usual.
\solution
\spart Si añadimos $\emptyset$ y el total, entonces tenemos una topología generada por $\base_{\rightarrow}$ y otra generada por $\base_{\leftarrow}$.

\spart  

\spart
\end{problem}

\begin{problem}[11]
Sea $\topl_j$, $j∈J$ una familia de topologías sobre $X$. Demostrar que existe una topología que contiene a todas las $\topl_j$, para $j∈J$ y además es la menos fina de todas las que verifican esta propiedad. 
\solution

Aplicamos directamente la proposición \ref{propTopologiaMinima}: la topología que contiene a todas ellas es \[ \topl = \bigcap_{j∈J} \topl_j \]
\end{problem}

\paragraph{Observación útil para el 5 y el 12:}  
\begin{enumerate}
\item $x \in C(x,\varepsilon)$
\item $\varepsilon_1 > \varepsilon_2 \implies C(x,\varepsilon_2) \subset C(x,\varepsilon_1)$
\end{enumerate}

Y podemos aplicar la propiedad:
\[
A\in\topl \dimplies \forall a\in A \exists \varepsilon > 0 \tlq C(x,\varepsilon)\subseteq A
\]

Haciendo caso al enunciado y haciendo el dibujo vemos que se cumplen las propiedades de base.

Esta topología contiene a la usual pero al revés no, porque para el punto de intersección de las diagonales no existe un abierto de la usual que le contenga.


\paragraph{Pistas para espacios métricos}

(16) Si tengo $d$, una distancia no acotada, puedo definir $d'=\frac{d}{1+d}$, que sigue siendo una distancia, parecida y además acotada.

(17) $\sum \frac{1}{2n} \leq 1$. La clave está en aplicar la desigualdad triangular a cada término del sumatorio. La clave para este problema es el 16.

\printindex

\end{document}
