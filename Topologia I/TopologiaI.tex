\documentclass{apuntes}

\title{Topología I}
\author{Guillermo Julián Moreno \\ Cristina Kasner Tourné}
\date{14/15 C1}
% Paquetes adicionales
\usepackage{tikztools}
\usepackage{fastbuild}

\usetikzlibrary{arrows}
% --------------------

\precompileTikz

\begin{document}
\pagestyle{plain}
\maketitle

\tableofcontents


\chapter{Conceptos básicos}

\section{Introducción}

En Topología buscamos extender conceptos importantes como continuidad o convergencia. Si partimos del concepto de continuidad en los reales, teníamos que

\begin{defn}[Continuidad] Dada $\appl{f}{(a,b)}{ℝ}$, se dice que es continua en $x_0 ∈ (a,b)$ si $∀ ε > 0 \; ∃δ>0 $ tal que $\abs{x-x_0} < δ \implies \abs{f(x) - f(x_0)} < ε$.
\end{defn}

¿Cómo podemos extender esto a conjuntos que no sean $ℝ$? Lo primero es que necesitamos una distancia. Y la propiedad central de la distancia debería ser \[ \abs{x+y} ≤ \abs{x} + \abs{y} \]. Esta propiedad es la desigualdad triangular, y es ciertamente natural. La extensión de la distancia la tendremos en los espacios métricos.

\begin{defn}[Espacio\IS métrico] \label{defEspacioMetrico}
Un espacio métrico es un par $(X, d)$, con $X$ un conjunto y $d$ una aplicación $\appl{d}{X×X}[0, ∞)$ tal que

\begin{enumerate}
\item $\dst(x,x) = 0\;∀x∈X$.
\item $\dst(x,y) ≥ 0\;∀x,y∈X$.
\item $\dst(x,y) = 0 \dimplies x=y$.
\item \concept{Desigualdad\IS triangular}: $\dst(x,z) \leq \dst(x,y) + d(y,z)\; ∀x,y,z∈X$.
\end{enumerate}
\end{defn}

Tenemos varios ejemplos de distancias:

Por ejemplo, en $ℝ^m$, tenemos $\dst (x,y) = \md{\vx-\vy} = \sqrt{(x_1-y_1)^2 + \dotsb + (x_m-y_m)^2}$.


Si consideramos el conjunto de funciones continuas $C([0,1]) \equiv \{ \appl{f}{[0,1]}{ℝ}, \text{f continua} \}$, $\md{f} ≝ \max_{x∈[0,1]} \abs{f(x)}$. Con esta noción, el conjunto de funciones continuas se comporta de forma similar a $ℝ^m$ con dimensión infinita, y podemos hacer cosas parecidas a las del espacio euclídeo.

Así, podemos definir $\dst(f,g) ≝ \md{f-g}$, y llegar a una definición de convergencia uniforme: $f_n \to f$ en esa distancia implica una convergencia uniforme en $[0,1]$.

Podemos definir una distancia algo artificial. Sea $X$ un conjunto cualquiera, definimos
\[ \dst(x,y) ≝ \begin{cases}
0 & \text{si}\; x = y \\
1 & \text{si}\; x ≠ y \\
\end{cases} \]
que cumple las 3 primeras propiedades de distancia y su comprobación es trivial y para la comprobación de la desigualdad triangular, basta comprobarlo por casos.

\begin{defn}[Bola] Dado $(X,\dst)$ un espacio métrico, con $x∈X$ y $r∈(0,∞)$, definimos la bola $\bola$ centrada en $x$ de radio $r$ como

\[ \bola(x,r) ≝ \{ y∈X \tq \dst(x,y) < r \} \]

En ocasiones querremos especificar la distancia ($\bola_{\dst}$) o el conjunto ($\bola_X$) con el subíndice.
\end{defn}

Las bolas tienen ciertas propiedades muy sencillas. Dados \sdst, $x∈X$, $r>0$, $y∈\bola(x,r)$ entonces $\bola(y,r-\dst(x,y)) ⊆ \bola(x,r)$.

\begin{wrapfigure}{R}{0.4\textwidth}
\inputtikz{I_BolaContenida}
\caption{La bola verde ($\bola(y, r-\dst(x,y)$) contenida dentro de $\bola(x, r)$.}
\label{figBolaContenida}
\end{wrapfigure}

Esto se puede demostrar con un dibujo (\ref{figBolaContenida}), pero tenemos que demostrarlo más formalmente:

\begin{proof}
$∀z∈\bola(x, r-\dst(x,y))$ tenemos que $\dst(x,z) ≤ \dst(x,y) + \dst(y,z) < \dst(x,y) + r - \dst(x,y) = r$, y por lo tanto $z ∈ \bola(x,r)$.
\end{proof}

Por supuesto, el dibujo es una guía. Si tomásemos la distancia rara de antes que sólo tomaba valores 1 ó 0, la bola no sería una bola como en $ℝ$.

Vamos a definir ahora el cierre, aunque sólo como notación:

\begin{defn}[Cierre] Dado \sdst espacio métrico, $x∈X$, $r≥0$, definimos \[ \overline{\bola}(x,r) ≝ \{ y∈ X\tq \dst(x,y) ≤ r\} \] como la bola cerrada de centro $x$ y radio $r$.\end{defn}

\begin{defn}[Conjunto\IS abierto] Sea \sdst un espacio métrico. Entonces damos dos definiciones

\begin{enumerate}
\item $A⊆X$ es abierto en \sdst si $∀x∈A\; ∃δ=δ_x > 0$ tal que $\bola(x,δ_x) ⊆ A$.
\item La familia de abiertos es $\topl_d \equiv \{ A ⊆ X \tq A\, \text{abierto} \}$.
\end{enumerate}
\end{defn}

La familia de abiertos que acabamos de definir es una \concept[Topología]{topología}, y cumple las siguientes propiedades.

\begin{enumerate}
\item $\emptyset, X ∈ \topl_d$.
\item $A,B ∈ \topl_d \implies A \cap B ∈ \topl_d$.
\item $A_j ∈ \topl_d\; ∀j∈ J \implies \bigcup_{j∈J} A_j ∈ \topl_d$
\end{enumerate}

Demostremos las dos últimas propiedades:

\begin{proof} \paragraph{Propiedad 2} Sea $x∈A\cap B$. Entonces $x∈A$ y $x∈B$, luego existen $δ_x^A, δ_x^B$ tales que $\bola(x, δ_x^A) ⊆ A$ y $\bola(x, δ_x^B) ⊆ B$ respectivamente. Sea ahora $δ=\min(δ_x^A, δ_x^B)$. Entonces $\bola(x,δ) ⊆ A\cup B$.

\paragraph{Propiedad 3} La propiedad es equivalente a la pregunta de, si dado $x∈\bigcup_{j∈J}A_j$, se cumple que $∃δ > 0 \tq \bola(x,δ)⊆\bigcup_{j∈J} A_j$.

Es obvio que $∃j_x ∈ J\tq x∈ A_{j_x}$, luego \[ ∃ δ > 0 \tq \bola(x,δ) ⊆ A_{j_x} ⊆ \bigcup_{j∈J} A_j \]
\end{proof}

Por otra parte, también podemos hacer una observación: por inducción, la intersección de una familia \textit{finita} de conjuntos abiertos también es un abierto.

\subsection{Topologías}

Una vez hecho esto, ya podemos pasar a definir qué es un espacio topológico y una topología.

\begin{defn}[Topología]\label{defTopología}
Sea $X$ un conjunto. Entonces una familia $\topl$ de subconjuntos de $X$ es una topología en $X$ si y sólo si cumple las tres propiedades que acabamos de ver:

\begin{enumerate}
\item $\emptyset, X ∈ \topl$.
\item $A,B ∈ \topl \implies A \cap B ∈ \topl$.
\item $A_j ∈ \topl\; ∀j∈ J \implies \bigcup_{j∈J} A_j ∈ \topl$
\end{enumerate}
\end{defn}

\begin{defn}[Espacio\IS topológico] Un espacio topológico es un par $(X, \topl)$ donde $\topl$ es una topología en $X$.
\end{defn}

Los elementos de \topl son los \textit{abiertos} de la topología.

También podemos definir el conjunto cerrado:

\begin{defn}[Conjunto\IS cerrado]
Dado un espacio topológico \stopl, $F⊆X$ es cerrado si y sólo si $F^C \equiv X \setminus F$ es abierto.
\end{defn}

Podemos definir dos topologías "comunes", por así decirlo, las obvias para cualquier conjunto. Tenemos la \concept[Topología!trivial]{topología\IS trivial} (el mínimo) dada por \[ \topl_{triv.} = \{ \emptyset, X \} \], y la \concept[Topología!discreta]{topología\IS discreta}, que sería el máximo: \[ \topl_{disc.} = \parts{X} \].

Y volviendo a nuestro bonito mundo de los reales, tenemos las \concept[Topología\IS usual]{topologías usuales} en $ℝ^m$ o $\topl_{ℝ^m}$. Para $m=1$, diremos que \[ A ∈ \topl_ℝ ≝ ∀x∈A\; ∃a,b∈ℝ \tq x∈ (a,b) ⊆ A \] Esto es, que siempre podemos encontrar un intervalo contenido en $A$ que a su vez contenga a $x$. Equivalentemente, $A$ será una unión de intervalos abiertos.

Para dimensión $m>1$, definimos su topología usual como \[	ℝ^m ⊇ A ∈ \topl_{R^m} ≝ ∀x∈A\; ∃ \begin{matrix} a_1, \dotsc, a_m \\ b_1, \dotsc, b_m \end{matrix} ∈ ℝ \] tales que $ x∈ (a_1, b_1) × \dotsb × (a_m, b_m) ⊆ A$.


\subsubsection{Topologías metrizables}

\begin{defn}[Espacio\IS topológico metrizable] Dado \stopl un espacio topológico, se dice que es metrizable si existe una distancia $\dst$ en $X$ tal que $\topl = \topl_{\dst}$.

$\dst$ no es necesariamente única.
\end{defn}

Por ejemplo, $\topl_ℝ$ es metrizable. Coincide $\topl= \topl_{\dst}$ con $\dst(x,y) = \abs{x- y}$.

Expandiendo un poco más sobre lo que significa que una topología coincide con otra, o lo que significa que una topología \textit{sea inducida} por una aplicación.

\begin{defn}[Topología\IS inducida] Dado un espacio topológico \stopl, entonces la topología inducida por una función $f$ es \[ \topl_f = \{ \inv{f} (A) \tq A ∈ \topl \}\footnote{La prueba de porqué es topología se encuentra en los ejercicios (H1.E6)} \]
\end{defn}

¿Qué significa entonces que una topología sea igual a otra? Si nos remitimos a la definición de topología (\ref{defTopología}), vemos que es un conjunto de subconjuntos de $X$. Luego dos topologías son iguales o equivalentes si y sólo si tienen los mismos elementos. Es decir, que si un conjunto es abierto en $X$ según una topología, también lo es según la otra y viceversa.

Volviendo al caso concreto, una topología inducida por la distancia es igual a otra topología si los abiertos según la distancia (esto es, las bolas) son también abiertos según la otra topología que estemos considerando.

Hagamos algunos ejemplos sobre topologías metrizables. ¿Son $\topl_{disc.}, \topl_{triv.}$ metrizables para $X$ un conjunto cualquiera?

En el caso de $\topl_{disc}$ sí lo es. Definimos la distancia como \[ \dst (x,y) = \begin{cases} 0 & x = y \\ 1 & x ≠ y \end{cases} \], luego $\bola(x, 1/2) = \{ x \}$, luego $\{ x \}$ es abierto en $\topl_{\dst}$. Entonces, si $A ⊆ X$, entonces $A= \bigcup_{x∈A} \{ x \}$ es abierto también.

La topología trivial es más interesante de estudiar. Si $\card{X}≥2$, entonces $\topl_{triv} ≠ \topl_{\dst}$ para cualquier distancia $\dst(x,y)$. ¿Por qué?

En la topología trivial sólo hay dos abiertos (vacío y total). Sin embargo, en la topología inducida por la distancia, los abiertos son las bolas.

Si hay más de dos elementos en $X$, existen $x,y∈X$ distintos, y por lo tanto existe una distancia $r=\dst(x,y) > 0$. Con $δ=\frac{r}{2}$, las bolas $\bola(x,δ), \bola(y,δ)$ son distintas y disjuntas. Ninguna de ellas es el vacío y el total así que no son abiertos en $\topl_{triv}$, pero sí que son abiertos en $\topl_{\dst}$. Por lo tanto, tenemos que $\topl_{\dst} \neq \topl_{triv}$.

\subsubsection{Topologías generadas por una base}

Además de por la distancia, podemos considerar las \textbf{topologías generadas por una base}.

Recordemos cómo definíamos una topología en $\topl_ℝ$. Decíamos que $A∈\topl_ℝ$ si y sólo si $∀x∈A\; ∃(a,b)$ tales que $x∈(a,b) ⊆ A$.

\begin{defn}[Base]\label{defBase}
Sea $X$ un conjunto y $\base$ una familia de subconjuntos de $X$ (e.d. $\base ⊆ \parts{X}$). Entonces $\base$ es una base si y sólo si

\begin{enumerate}
\item $∀x∈X\;∃B∈\base \tq x∈B$. Dicho de otra forma, $\bigcup_{B∈\base} B = X$.
\item $∀B_1,B_2∈\base;\, ∀x∈B_1 ∩ B_2$, existe $B_3 ∈ \bola \tq x∈B_3 ⊆ B_1∩B_2$.
\end{enumerate}
\end{defn}


\begin{defn}[Topología\IS generada por una base] \label{TopologiaGeneradaBase} La topología $\topl_\base$ se define por \[ A ∈ \topl_\base \iff ∀x ∈ A\; \exists B∈\base \tq x∈B⊆A \]
\end{defn}

Tenemos que demostrar, eso sí, que eso que hemos definido ahí es realmente una topología.

\begin{prop} $\topl_\base$ es una topología en $X$.\end{prop}

\begin{proof} Tenemos que comprobar las tres propiedades de una topología (\ref{defTopología}). Sabemos que $\emptyset ∈ \topl_\base$. Además, tal y como hemos definido la topología generada, también sabemos que $X∈ \topl_\base$.

\paragraph{Propd. 2} Tenemos que demostrar que $A_1, A_2 ∈ \topl_\base \implies A_1∩A_2 ∈ \topl_\base$. Si $x∈ A_1∩A_2$, entonces $∃B_1, B_2 ∈ \base$ tales que $x∈B_1⊆A_1$ y $x∈B_2⊆A_2$ respectivamente.

Según la segunda propiedad de la base, existe un $B_3∈\base$ tal que $x∈B_3 ⊆ B_1∩B_2 ⊆A_1∩A_2$, luego $A_1∩A_2 ∈ \topl_\base$.

\paragraph{Propd. 3} Demostramos que $A_j ∈ \topl_\base\; ∀j∈J \implies \bigcup_{j∈J} A_j ∈ \topl_\base$. Si $x∈\bigcup_{j∈J} A_j \implies ∃i=i_x∈J\tq x∈A_i$. Luego como $A_i ∈ \topl_\base$ tenemos que $\exists B∈ \base$ tal que $x∈B ⊆ A_i ⊆  \bigcup A_j$.
\end{proof}

Nos fijamos que en la demostración de la tercera propiedad no hemos usado nada sobre cómo hemos definido la base. Es decir, que siempre que definamos una topología $\topl$ como \[ A ∈ \topl \iff ∀x∈ A\;∃U ∈ \mathcal{F} \tq x∈ U ⊆ A \], donde $\mathcal{F}$ es una familia de subconjuntos de $X$, la propiedad tercera de la definición de topología \textbf{está garantizada}. Es para la primera y segunda propiedad para las que se necesita que $\mathcal{F}$ cumpla algún tipo de propiedad.

\begin{defn}[Topología\IS fina]
Dado un espacio $X$ y dos topologías $\topl_1, \topl_2$, si $\topl_1⊆\topl_2$ (todo abierto de $\topl_1$ es abierto de $\topl_2$) se dice que $\topl_2$ es \textbf{más fina} que $\topl_1$.
\end{defn}

\begin{prop} Sea $X$ un espacio topológico y $\base$ una base. Entonces

\begin{enumerate}
\item $\base⊆\topl_\base$ (todos los elementos de $\base$ son abiertos en $\topl_\base$.
\item $A∈\topl_\base$ si y sólo si $A$ es unión de elementos de $\base$.
\end{enumerate}
\end{prop}

\begin{proof}
\paragraph{1)} Recordamos que \[ V ∈ \topl_\base ≝ ∀x∈V \; ∃B=B_x∈\base\tq  x∈B⊆V \]. Sea $M∈\base$, quiero demostrar que $M∈ \topl_\base$. He de comprobar que \[ ∀x ∈ M\; ∃B∈\base \tq x∈ B ⊆ M \], lo cual es obvio si tomamos $B=M$, ya que los elementos de la base son siempre abiertos.

\paragraph{2)} Partiendo de la afirmación de antes, sabemos que si $A ∈ \topl_\base$, entonces $∀x∈A\; ∃B_x∈\base$ tal que $x∈ B_x⊆A$. Como cada uno de esos conjuntos está en $A$, su unión también lo está. Y por otra parte, dado que consideramos todos los puntos $x$ de $A$, nos queda que \[ A = \bigcup_{x∈A}B_x \], demostrando así el primer lado de la implicación.

La implicación a la izquierda se resuelve por la primera parte de esta proposición: si $B_j∈B$, entonces $B_j∈\topl_\base$ y por la tercera propiedad de la topología (\ref{defTopología}), nos queda que \[ \bigcup_{j∈J} B_j ∈ \topl_\base \]

\end{proof}

Ahora que ya sabemos cómo generar una topología a partir de una base, podemos hacernos una pregunta. Consideramos una serie de conjuntos que queremos que sean abiertos en nuestro espacio. Obviamente, la topología discreta cumple lo que buscamos, pero, ¿hay una topología más pequeña? ¿Cuál es la topología \textit{"mínima"}?

\begin{prop} Sea $X$ un conjunto. \label{propTopologiaMinima}

\begin{enumerate}
\item Si $\topl_k$ es una topología en $X$, $∀k∈K$ entonces \[ \topl ≝\bigcap_{k∈K} \topl_k \].

\item Sea $D$ una familia de subconjuntos de $X$ ($D⊆\parts{X}$) y sea \[ \topl_D ≝ \bigcap_{D⊆\topl} \topl \] donde $\topl$ es una topología en $X$.

Entonces $\topl_D$ es una topología en $X$, $D⊆\topl_D$ y $\topl_D$ es la topología menos fina que cumple $D⊆\topl_D$.
\end{enumerate}
\end{prop}

\begin{proof}
\paragraph{1)} La primera propiedad de la topología es trivial. Vamos con la segunda. Si $V_1V_2 ∈ \topl$, tenemos que $V_1, V_2 ∈ \topl_k\,∀k∈K$. Luego como $\topl_k$ es topología, $V_1∩ V_2 ∈\topl_k\, ∀k∈K$, y entonces $V_1∩V_2 ∈ \bigcap \topl_k = \topl$.

\paragraph{2)} Sabemos que $\topl_D$ es topología por lo que acabamos de demostrar. Ahora bien, ¿es la más pequeña? Es obvio, viendo que es la intersección de todas las topologías que contienen a $D$.\footnotemark
\end{proof}
\footnotetext{Relacionado con el ejercicio 9-c.}

Tenemos que tener cuidado cuando $D$ es una base: hay que asegurarse de que la topología coincida en ese caso.\footnote{Y nos preocupamos nosotros de eso.}

\subsubsection{Topología del orden}

Hasta ahora hemos visto cómo generar topologías a partir de la distancia, y también tratando de extrapolar el concepto de los intervalos de $ℝ$ con las bases. Ahora vamos a ver cómo hacerlos a través de otra visión de los intervalos como elementos de orden. Recordemos brevemente qué es un orden total: a grandes rasgos es uno donde podemos comparar todos los elementos.

\begin{defn}[Orden\IS total] Dado un conjunto $X$, un orden total en $X$ es una relación $x < y$ tal que

\begin{enumerate}
\item $x<y, y < z\implies x < z$.
\item $∀x∈X$, $x < x$ es falso.
\item $∀x,y∈X$ con $x≠y$ entonces se cumple una y sólo una de $x< y$ ó $y<x$.
\end{enumerate}
\end{defn}

Dado un conjunto $X$ y un orden total $<$ se puede construir una topología $\topl_<$ de la misma forma que en $ℝ$: intervalos $(a,b)$. Empecemos con ejemplos.

\paragraph{Orden lexicográfico en $ℝ^2$} Este ejemplo es una topología muy visual (ver la figura \ref{figOrdenLex}), importante y rara. Empezamos definiendo qué es ese orden

\begin{defn}[Orden\IS lexicográfico] Denotamos como $<_{Lex}$ al orden que, dado $x=(x_1,x_2), y=(y_1, y_2)$ ambos en $ℝ^2$, se dice que $x<_{Lex} y$ si $x_1 < y_1$ o bien, si $x_1 = y_1$, entonces $x_2 < y_2$.
\end{defn}

\begin{wrapfigure}{R}{0.4\textwidth}
\inputtikz{I_OrdenLexicografico}
\caption{Ilustración del orden lexicográfico en $ℝ^2$. Cualquier punto en $r_2$ es mayor que todos los de $r_1$. En la misma vertical, tenemos que $a<_{Lex}b$.}
\label{figOrdenLex}
\end{wrapfigure}

A partir de esto podemos definir el intervalo lexicográfico de la forma obvia:

\begin{defn}[Intervalo\IS lexicográfico] Si $a,b∈ℝ^2$ con $a<_{Lex}b$ entonces
\[ (a,b)_{Lex} ≝ \{  x ∈ℝ^2\tq a <_{Lex} x <_{Lex} b \} \]
\end{defn}

\begin{prop} \[ \base_{Lex}=\{ (a,b)_{Lex} \tq a,b∈ℝ^2, a<_{Lex}b \} \] es una base para una topología en $ℝ^2$.\end{prop}

\begin{proof}
Como ejercicio, pero queda claro que si $B_1,B_2∈\base$ entonces $B_1∩B_2∈\base$, lo que es todavía mejor que la definición de base.
\end{proof}

\begin{defn}[Topología\IS lexicográfica en $ℝ^2$] En $ℝ^2$, definimos la topología lexicográfica como

\[ \topl_{Lex} ≝ \topl_{\base_{Lex}} \]
\end{defn}

Podemos ver un ejemplo, considerando en $[0,1]^2$ el intervalo acotado lexicográficamente por $\mathbbold{0} = (0,0)$ y $\mathbbm{1} = (1,1)$, luego
\[ [0,1] × [0,1] = [\mathbbold{0}, \mathbbm{1}]_{Lex} \]

En esta topología, el abierto más sencillo que contiene a un punto "en el medio" (por ejemplo, el $(0.5, 0.5)$) sería un intervalo "vertical". Sin embargo, para un punto en el borde superior o inferior, el abierto más sencillo sería un rectángulo que se expande hacia la derecha (ver figura \ref{figIntervalosLex}).

\begin{figure}[hbtp]
\centering
\begin{subfigure}[b]{0.4\textwidth}
\inputtikz{I_OrdenLex_AbiertoVertical}
\caption{En un punto dentro del cuadrado, el abierto más sencillo es un intervalo vertical}
\end{subfigure}
\begin{subfigure}[b]{0.4\textwidth}
\inputtikz{I_OrdenLex_AbiertoTop}
\caption{En un punto en un borde del cuadrado, el abierto más sencillo es un rectángulo, el intervalo $(a,b)_{Lex}$.}
\end{subfigure}

\caption{Intervalos más sencillos en $[\mathbbold{0}, \mathbbm{1}]_{Lex}$.}
\label{figIntervalosLex}
\end{figure}

Más ejemplos: ¿una bola $A$ en el sentido habitual (ver figura \ref{figBolaLex}) de $ℝ^2$ es abierto en esta topología? Efectivamente: podemos expresarlo como unión de abiertos de la topología, las líneas verticales.

\begin{figure}[hbtp]
\centering
\inputtikz{I_OrdenLex_Bola}
\caption{La bola $A$ es abierto en la topología lexicográfica si la expresamos como unión de intervalos verticales.}
\label{figBolaLex}
\end{figure}

Si lo expresamos de forma simbólica también llegamos a lo mismo. Tomamos

\[ A = \left\{ (x,y) \tq \left(x-\frac{1}{2}\right)^2 + \left(y-\frac{1}{2}\right)^2 < \frac{1}{100} \right\} \]

Así, tendríamos que

\[ A = \bigcup (a_x, b_x)_{Lex} \]

con

\begin{gather*}
a_x = \left(x, \frac{1}{2} - \sqrt{\frac{1}{100} - \left(x-\frac{1}{2}\right)^2}\right) \\
b_x = \left(x, \frac{1}{2} + \sqrt{\frac{1}{100} - \left(x-\frac{1}{2}\right)^2}\right)
\end{gather*}

\subsection{Convergencia}

Ahora sigamos con más definiciones.

\begin{defn}[Entorno\IS abierto] Dado \stopl un espacio topológico y $x∈X$, un entorno abierto de $x$ es un abierto $U∈\topl$ tal que $x∈U$.
\end{defn}

\begin{defn}[Convergencia\IS de sucesiones] Sea \stopl un espacio topológico y $\{x_n\}_{n∈ℕ}$ una sucesión en $X$, y $x∈X$. Se dice que $x_n$ converge a $x$ si todo entorno de $x$ contiene todos los términos de la sucesión a partir de un índice determinado.

Dicho simbólicamente

\[ ∀U∈\topl,\; x∈U\; ∃n_U∈ℕ \tq x_n ∈ U \,∀n≥n_U \]
\end{defn}

\paragraph{Ejercicio} En un espacio métrico \sdst con la topología $\topl_{\dst}$ inducida por la distancia, demuestra que \[ x_n \convs x \iff \dst(x,x_n)\convs 0 \iff ∀ε> 0\, ∃n_ε\tq \dst(x,x_n) < ε \]

\paragraph{Ejemplo 1} Veamos ejemplos de convergencia en topologías raras. Tomemos $ℝ$ con $\topl_{[, )}$, y la sucesión $x_n= \frac{-1}{n}$ para $n≥1$.

Esta sucesión converge en el sentido usual (euclídeo) a cero, pero no en esta topología. Y es que existe un intervalo que es entorno de $0$ (por ejemplo, $[0, 1)$ que no contiene a ningún punto de la sucesión.

\paragraph{Ejemplo 2} Tomemos la sucesión $x_n=\left(\frac{1}{n}, 1\right)$ para $n≥1$ en el espacio topológico $(ℝ^2, \topl_{Lex})$. Converge en la topología usual a $(0,1)$, pero no en la lexicográfica: un entorno vertical del $(0,1)$ no contiene a puntos de la sucesión.


Ahora vamos a ver algunos conceptos en relación a los cerrados, que recordemos eran los complementarios de los abiertos.


\begin{prop} $ $
\begin{enumerate}
\item $\emptyset, X$ son cerrados.
\item La unión finita de cerrados es cerrado.
\item La intersección de una familia de cerrados es cerrado.
\end{enumerate}
\end{prop}

\paragraph{Ejemplos}  Si tomamos $R_a ≝ \{ (a,y) \tq y ∈ ℝ \}$, es abierto y cerrado en $(ℝ^2, \topl_{Lex})$.

Es cerrado porque $R_a^c$ es abierto: para todo punto puedo coger un entorno abierto contenido en el conjunto. También podemos verlo como que $R_a^c = \bigcup_{b∈ℝ\setminus \{a\}} R_b$, que es unión de abiertos.

Otro conjunto interesante es $[a,b)$ en $\topl_{[,)}$, que también es abierto (es un elemento de la base) y cerrado (su complementario es $(-∞, a) ∪ [b, ∞)$, ambos abiertos (podemos expresarlos como unión de conjuntos de la base).

Curiosamente, ese conjunto en $(ℝ, \topl_ℝ)$ no es ni abierto ni cerrado: no hay ningún abierto que contenga a $a$ y que esté contenido en $[a,b)$ por lo que no es abierto; y tampoco es cerrado porque $[a,b)^c=(-∞,a) ∪ [b,∞)$ que no es abierto.

\seprule

Vamos a hacer ahora ciertas observaciones sobre convergencia en topologías definidas por una base, para darnos cuenta de los interesantes que pueden resultar al permitirnos probar cosas mirando sólo los elementos de la base.

\begin{prop} Dado un conjunto $X$ y $\base$ una base para una topología en $X$, $\topl_\base$, entonces

\[ x_n\convs x \iff ∀B∈\base \tq x ∈ B,\; ∃n_B \tq  x_n∈ B ∀ n ≥ n_B \].

Es decir, basta comprobar la definición para entornos de $x$ que son elementos de la base.\end{prop}

\begin{proof}
La implicación a la derecha es trivial: si se cumple para todos los abiertos, se cumple para algunos en particular.

Ahora tenemos que demostrar la implicación a la izquierda. Si $U∈ \topl_\base$ y $x∈U$, entonces por la definición de $\topl_\base$ $∃B∈ \base$ tal que $x∈B⊆U$. Por hipótesis, $∃n_B$ tal que $x_n∈B\; ∀n≥ n_B$, y como $B⊆U$ entonces $x_n ∈ U \; ∀n ≥ n_B$.
\end{proof}

\subsection{Interior, adherencia y frontera de conjuntos}

\subsubsection{Interior}

Sea \stopl un espacio topológico y $W⊆X$ un conjunto cualquiera. Entonces, vamos a definir varios conceptos

\begin{defn}[Interior] Decimos que $x ∈ \mop{Int}(W)$ si existe un entorno $U$ de $x$ tal que $U⊆W$. Es decir, existe un $U$ abierto tal que $x∈U ⊆ W$.

El interior de un conjunto $W$ se denota como $\intr{W}$.\label{defInterior}
\end{defn}

Por ejemplo, $\intr{ℚ}$ es vacío tanto en la topología usual como en $\topl_{[,)}$.

El intervalo $[a,b]$ en la topología usual tiene como interior el abierto $(a,b)$. Pero, ¿y en la topología $\topl_{[,)}$? En este caso es: $\intr{[a,b]} = [a,b)$.

Razonando, si $a≤x<b$, entonces $x∈[a,b) ⊆ [a,b]$, luego $x$ está en el interior. Si $x = b$, entonces no existe un intervalo abierto $U$ con $b∈U⊆[a,b]$, porque si $b∈U$ con $U$ abierto entonces existiría un $[α,β)$ con $b∈[α,β) ⊆ U$. En ese caso, el punto medio $\frac{b+β}{2} ∈ [α,β)$, luego entonces también pertenecería a $U$. Sin embargo, es claro que $\frac{b+β}{2} > b$, así que no puede estar en $U$.

\begin{figure}[hbtp]
\centering
\inputtikz{I_ConjuntoWLex}
\caption{Conjunto $W$ en la topología lexicográfica.}
\label{figConjuntoWLex}
\end{figure}

Otro ejemplo: tomemos $([0,1]^2, \topl_{Lex})$. ¿Cuál es el interior del conjunto $W$ que aparece en la figura \ref{figConjuntoWLex}?

Está claro que los puntos de dentro $p_i$ son del interior, ya que siempre podemos encontrar un abierto. También están dentro los puntos $p_b$ de los bordes inferior y superior (con $0≤x<b$), ya que podemos encontrar una banda (como la banda azul) contenida en $W$. Los puntos del lateral izquierdo $p_l$ también están en el interior.

Sin embargo, los puntos $p_d$ en el borde de la diagonal o en el borde inferior con $b ≥ x$ no están en el interior: cualquier abierto que cojamos será una banda como la roja, que se sale de $W$. Tampoco están los puntos del borde lateral derecho.

En definitiva, en la figura \ref{figConjuntoWLex} el interior sería el interior de naranja con los bordes naranjas, excluyendo los marcados en rojo.

\begin{prop} En el caso de una topología generada por una base $\topl_\base$, si $W⊆X$, decimos que $x∈\intr{W}$ si y sólo si existe un elemento $B$ de la base tal que $x∈B⊆W$.
\end{prop}

\begin{proof} Recordamos la definición: \[ x∈ \intr{W} \iff ∃A∈\topl \tq x∈A⊆W \]

La implicación a la izquierda la demostramos diciendo que si $B∈\base$, entonces $B∈\topl_\base$ y $A=B$.

Hacia la derecha, si $x∈\intr{W}$ entonces sabemos que $∃A∈\toplb$ tal que $x∈A⊆W$. Como $A∈\toplb$, entonces $∃B∈\base \tq x∈B⊆A$, y nos queda $x∈B⊆A⊆W$.
\end{proof}

\begin{prop} Sea $\stopl$ un espacio topológico y $W⊆X$. Entonces

\begin{enumerate}
\item $\intr{W} ⊆ W$.
\item $\intr{W} = \bigcup A$ tales que $A∈\topl, A⊆W$ (los abiertos contenidos en $W$.
\item $W$ es abierto si y sólo si $\intr{W} = W$.
\end{enumerate}
\label{propInterior}
\end{prop}

\begin{proof} La primera parte es trivial.

La segunda proposición, si $U$ es abierto y $U⊆W$, entonces $A⊆\intr{U}$, de forma bastante obvia. Por otra parte, si $x∈\intr{W}$, entonces $∃A∈\topl\tq x∈A⊆W$. Entonces $x∈A⊆W⊆\bigcup U$ donde $U$ son abiertos contenidos en $W$.

Y por último la tercera parte. Como $\intr{W}$ es unión de abiertos (lo acabamos de demostrar) también es abierto. La demostración a la izquierda resulta trivial de esta forma: si $W=\intr{W}$ y $\intr{W}$ es abierto, $W$ es abierto.

Ahora queremos demostrar que si $W$ es abierto, entonces $\intr{W}=W$. Sabemos que siempre $\intr{W} ⊆ W$, así que sólo nos falta la otra inclusión $W⊆\intr{W}$. Simplemente tenemos que recordar la demostración anterior: si $\intr{W}$ es la unión de todos los abiertos $U$ contenidos en $W$, entonces $W$ es uno de esos abiertos y por lo tanto $W⊆\bigcup U = \intr{W}$.
\end{proof}

\begin{remark}
$\intr{W}$ es el abierto más grande contenido en $W$.
\end{remark}

\subsubsection{Adherencia}

\begin{defn}[Adherencia] Sea \stopl un espacio topológico, $W⊆X$. La adherencia $\adh{W}$ de $W$ se define como todos los entornos de $X$ que "cortan" a $W$. Más formalmente

\[ x ∈ \adh{W} \iffdef A∩W ≠ \emptyset \; ∀ A ∈ \topl \tq x∈A \]
\label{defAdherencia}
\end{defn}

Como ocurría con otras propiedades de conjuntos, si la topología está generada por una base nos vale con comprobar la definición para los elementos de la base.

\begin{prop} En una topología generada por una base \toplb se tiene que \[ x ∈ \adh{W} \iff B∩W≠\emptyset \; ∀B∈\base \tq x∈B \]
\end{prop}

\paragraph{Ejemplos} $ℚ⊆\topl_ℝ$ y en $\topl_{[,)}$. En ambos casos $\adh{ℚ} = ℝ$: cualquier $x∈(a,b)⊆ℝ$ que cojamos se tiene que $(a,b) ∩ ℚ$ no es vacío (y cualquier intervalo $(a,b)$ que escojamos no es vacío porque $x$ está en él). La razón es análoga para $\topl_{[,)}$.

Este ejemplo nos sirve como introducción a la definición de conjunto "denso".

\begin{defn}[Conjunto\IS denso] Dado \stopl espacio topológico y $W⊆X$, se dice que $W$ es denso en $X$ para \topl si $\adh{W} = X$.
\end{defn}

Sigamos con ejemplos. Tomamos $X=[0,1]^2$ con la topología lexicográfica, y $W=\{(x,y) ∈ [0,1]^2 \tq x+y < \frac{3}{2} \}$.

\begin{figure}[hbtp]
\inputtikz{I_AdhConjuntoWLex}
\caption{Adherencia (azul) del conjunto $W$ (naranja) en la topología lexicográfica}
\label{figAdhWLex}
\end{figure}

Está claro que $W⊆\adh{W}$. La diagonal (puntos $p_d$ en la figura \ref{figAdhWLex}) también está en la adherencia: cualquier abierto que cojamos alrededor de ese punto interseca con $W$. Además, cualquier punto $p_b$ del borde superior (salvo la esquina $(1,1)$) está en la adherencia, ya que los abiertos serán bandas como la verde, que intersecan igualmente con $W$.

\begin{prop} Dado \stopl un espacio topológico y $W⊆X$, se tiene que
\begin{enumerate}
\item $W⊆\adh{W}$.
\item $\adh{W} = \left(\mop{Int}(W^c)\right)^c$. Como consecuencia $\adh{W}$ es cerrado.
\item $\adh{W} = \bigcap_{F⊇W} F$ donde $F$ son todos los cerrados que contienen a $W$.
\item $W$ es cerrado si y sólo si $W = \adh{W}$.\\
\end{enumerate}\end{prop}

Para demostrar estas propiedades, muy similares a las del interior (\ref{propInterior}), usaremos la dualidad abierto-cerrado, unión-intersección e interior-adherencia, que nos será muy útil en el futuro.

\begin{proof}
\begin{enumerate}
\item Claro a partir de la definición (\ref{defAdherencia}).
\item Lo que estamos diciendo es que $\adh{W}^c = \mop{Int}(W^c)$. Empezamos diciendo que si $x∉\adh{W}$, entonces $∃A∈\topl \tq x∈A$ y $A∩W=\emptyset$. Es decir, que $A⊆W^c$, luego $x∈\mop{Int}(W^c)$.
\item Lo deja como ejercicio, usando la segunda definición y las propiedades del interior.
\item Ídem.
\end{enumerate}
\end{proof}

\subsubsection{Frontera de un conjunto}

\begin{defn}[Frontera] Dado \stopl un espacio topológico y $W⊆X$, se dice que

\[ x∈\mop{Fr}(W) \iffdef A∩W ≠ \emptyset \y A∩W^c ≠ \emptyset\; ∀A ∈ \topl \tq x∈A \]

Es decir, la frontera son todos los puntos cuyos entornos cortan a $W$ y $W^c$.
\label{defFrontera}
\end{defn}

Como viene siendo habitual, si tenemos una topología generada por una base, basta comprobarlo para los elementos de la base.

\begin{prop} Sea \stopl un espacio topológico y $W⊆X$, entonces

\begin{enumerate}
\item $\mop{Fr}(W) = \adh{W} ∩ \adh{W^c}$, y por lo tanto es cerrado.
\item $\mop{Fr}(W) = \adh{W} \setminus \intr{W}$.
\end{enumerate}
\end{prop}

\begin{proof}
\begin{enumerate}
\item Claro a partir de la definición (\ref{defFrontera}).
\item Antes hemos visto que $\adh{W^c} = \left(\mop{Int}((W^c)^c)\right)^c$, o dicho de otra forma $\adh{W^c} = (\intr{W})^c = X \setminus \intr{W}$. Entonces $\mop{Fr}(W) = \adh{W} ∩ \adh{W^c} = \adh{W} ∩ (X\setminus \intr{W}) = \adh{W} \setminus \intr{W}$.
\end{enumerate}
\end{proof}

\begin{remark} La definición nos permite escribir la adherencia como unión de conjuntos disjuntos: \[ \adh{W} = \intr{W} ∪ \mop{Fr}(W) \]
\end{remark}

\paragraph{Ejemplos} Empezamos con el simple: $ℚ$ en $ℝ$ con la topología usual y con $\topl_{[,)}$. En ambos casos tenemos que \[ \mop{Fr}(ℚ) = \adh{ℚ} \setminus \intr{ℚ} = ℝ \], ya que en cualquier intervalo de longitud mayor que cero hay racionales e irracionales.

Y volvemos a la lexicográfica: sea $W=\{ (x,y) ∈ [0,1]^2 \tq x+y < 3/2 \}$. Sabemos que $\mop{Fr}(W) = \adh{W} \setminus \intr{W}$, y ya habíamos calculado la adherencia (azul en la figura \ref{figFrontWLex}) como

\[ \adh{W} = \left\{ (x,y) ∈ [0,1]^2 \tq x + y ≤ \frac{3}{2} \right\} \bigcup \left\{ (x,1) \tq \frac{1}{2} ≤ x < 1 \right\} \]

Por otra parte (ver figura \ref{figConjuntoWLex}), el interior era

\[ \intr{W} = W \setminus \left\{ (x,0) \tq \frac{1}{2} < x \right\} \]

De esta forma, nos queda que \[ \mop{Fr}(W) = \{ (x,y) \in [0,1]^2 \tq \{ x > \frac{1}{2}, y = 0 \} \cup \{ x \geq \frac{1}{2}, y=1\} \cup \{ x+y=\frac{3}{2} \} \} \]

\begin{figure}[hbtp]
\inputtikz{I_FronteraConjuntoWLex}
\caption{Frontera del conjunto $W$, en verde, con los dibujos de la adherencia (azul, figura \ref{figAdhWLex}) y el interior (rojo, figura \ref{figConjuntoWLex}).}
\label{figFrontWLex}
\end{figure}

Más ejemplos: sea \sdst un espacio métrico y $\topl_{\dst}$ la topología inducida por la distancia. Consideramos $x∈X$ y $r>0$, y $W=\bola(x,r)$. ¿Cuáles son el interior, adherencia y frontera del conjunto?

Está claro que $\intr{W} = \bola(x,r)$ por ser las bolas abiertas. La adherencia es algo más complicada. Es obvio que \[ \adh{W} ⊆ \adh{\bola}(x,r) ≝ \{ y∈X \tq \dst(x,y) ≤ r \} \] (que es la bola cerrada, no la adherencia de la bola), pero pueden ser distintos. En $ℝ^n$ con la distancia euclídea sí son iguales. Pero consideremos la distancia \[ \dst(x,y) = \begin{cases} 0 & x = y \\ 1 & x ≠ y \end{cases} \] que nos genera la topología discreta, en la que todos los conjuntos son a la vez abiertos y cerrados. Si $r=1$, entonces \[ W = \bola(x,1) = \{ x \} \implies \adh{W} = W = \{ x \} \]

¿Y cuál es la bola cerrada? $\adh{\bola}(x,1) = X$, obviamente no es lo mismo aunque desde luego $\adh{W} ⊆ \adh{\bola}(x,1)$.

Además, en esta topología, tendríamos que $\mop{Fr}(\{ x \} ) = \emptyset$.


\begin{prop} Una proposición trivial: dado \stopl espacio topológico, $A⊆B⊆X$. Entonces se tiene que $\intr{A}⊆\intr{B}$ y que $\adh{A}⊆\adh{B}$.
\end{prop}

\subsection{Puntos aislados y puntos de acumulación}

\begin{defn}[Punto\IS aislado] Dado \stopl un espacio topológico y $W⊆X$, se dice que $x∈W$ es un punto aislado de $W$ si y sólo si existe un abierto $A$ con $x∈A$ y tal que $A∩W=\{x\}$.
\end{defn}

El punto que no es aislado es un punto de acumulación:

\begin{defn}[Punto\IS de acumulación] Dado \stopl un espacio topológico y $W⊆X$, se dice que $x∈X$ (no es necesario que esté en $W$) es un punto de acumulación de $W$ si y sólo si $∀A∈\topl$ con $x∈A$ se tiene que $A∩(W\setminus\{x\}) ≠ \emptyset$.

El conjunto de todos los puntos de acumulación de $W$ se denota como $W'$. En algún caso (no durante estas clases) se le llamará el \concept[Conjunto\IS derivado]{conjunto derivado} de $W$.
\end{defn}

\paragraph{Ejemplos} Empezamos por lo simple, como siempre. Consideramos a $(a,b)$ en $ℝ$ con la topología usual. No hay ningún punto aislado, y los puntos de acumulación son $[a,b]$.

Si pensamos en $W=(a,b]$ en $ℝ$ con la topología de intervalos $\topl_{[,)}$, vemos que $b$ es un punto aislado. Cualquier abierto  $[b, b + ε)$ sólo interseca en $b$ con $W$. Por otra parte, los puntos de acumulación son $[a,b)$.

Como siempre, vamos ahora a la topología lexicográfica. Consideramos $W=\{ (x, 0.5) \tq 0≤x ≤ 1\}$. Todos los puntos son aislados. Los puntos de acumulación desde luego no serán puntos del intervalo: son los bordes superiores e inferiores, cuyos entornos son bandas (salvo los que coincidan con bordes laterales). Luego \[ W' = \{ (x,1) \tq x < 1 \} ∪ \{ (x,0) \tq x > 0 \} \]

Y por último, consideremos $W=\left\{ \frac{1}{n} \tq n ≥ 1, n∈ℕ\right\}$ con la topología usual en $ℝ$. Todos los puntos de $W$ son aislados, y el $0$ es punto de acumulación.

\begin{prop} Sea \stopl un espacio topológico y $W⊆X$. Entonces

\begin{enumerate}
\item $x∈W' \iff x∈ \adh{W\setminus \{x\}}$. Como consecuencia, $W'⊆\adh{W}$.
\item $\adh{W} = W ∪ W'$.
\item $W$ es cerrado si y sólo si $W'⊆W$.
\end{enumerate}
\end{prop}

\subsection{Espacios métricos}

En los espacios métricos con distancias razonables, la adherencia, interior y demás puntos se vuelven más interesantes y fáciles de estudiar.

\begin{prop} Sea \sdst un espacio métrico con la topología inducida por la distancia $\topl_{\dst}$, y sea $W⊆X$.

\begin{enumerate}
\item $x∈\intr{W} \iff ∃δ> 0 \tq \bola(x,δ) ⊆ W$.
\item $x∈\adh{W} \iff ∀ε>0\; \bola(x,ε) ∩ W ≠ \emptyset$.
\item $x∈\mop{Fr}(W) \iff ∀ε>0\; \bola(x,ε) ∩W ≠ \emptyset \y \bola(x, ε) ∩W^c ≠ \emptyset$.
\item $x$ es punto aislado de $W$ si y sólo si $∃δ>0 \tq \bola(x,δ) ∩ W = \{ x\}$.
\item $x∈W' \iff ∀ε>0\; \bolac(x,ε) ∩ W ≠ \emptyset$, donde $\bolac(x,ε) = \bola(x,ε) \setminus \{x\}$.
\item $x∈ \adh{W}$ si y sólo si existe una sucesión $\{x_n\}⊆W\tq x_n\convs x$.
\item $x∈W'$ si y sólo si existe una sucesión $\{x_n\}⊆W$ con $x_n\convs x$ y $x_n≠x\; ∀n∈ℕ$ (excluimos sucesiones constantes a partir de un términ).
\end{enumerate}

En las proposiciones 2,3 y 5 basta comprobarlo con un $\epsilon$ lo suficientemente pequeño. En particular, basta comprobarlo para una sucesión $ε_n \to 0$, por ejemplo $ε_n=\frac{1}{n}$.
\end{prop}

\begin{proof}
\begin{enumerate}
\item Queremos demostrar que $x∈\intr{W} \iff ∃δ> 0 \tq \bola(x,δ) ⊆ W$, y para ello recordamos la definición: $x∈\intr{W} \iffdef ∃A∈\topl \tq x∈A ⊆ W$. La implicación hacia la izquierda es sencilla si tomamos simplemente el abierto $A = \bola(x,δ)$.

En la otra dirección, la hipótesis es que $∃A∈\topl \tq x∈A⊆W$. Al ser una topología inducida por la distancia, podemos encontrar una bola $\bola(x,δ)⊆A⊆W$.

\item Son bastante sencillas.
\item
\item
\item
\item $x∈\adh{W} \iff ∃x_n∈W \tq x_n\convs x$.

Hacia la derecha, podemos ir cogiendo bolas $\bola(x,1/n)∩W ≠ \emptyset$, cada vez más pequeñas, y escoger puntos $x_n∈\bola(x,1/n)$, de tal forma que $\dst(x_n, x) \convs 0$, o de otra forma $x_n \convs x$.

Para el otro lado, es cierto y además para cualquier espacio topológico. Tenemos que $x_n\convs x$ para $x_N∈W$, queremos probar que $x∈\adh{W}$. Por la propia definición de convergencia, podemos encontrar un $n_A$ tal que $x_n∈A$ para todo $n≥n_A$, luego $A∩W ≠ \emptyset$.

\item Y ya para acabar, hay que demostrar que $x∈W' \iff ∃x_n∈W, x_n≠x \; ∀n∈ℕ$ y $x_n\convs x$.

Hacia la izquierda, sabemos que $\bolac(x,\frac{1}{n})∩W ≠ \emptyset$, luego $∃x_n∈W∩\bolac(x, 1/n)$, es decir, que $x_n∈W$ y además $x_n≠x$, luego ya hemos encontrado la sucesión que buscábamos.

La implicación hacia la izquierda es igual en cualquier espacio topológico. $∀A∈\topl$ con $x∈A$, tenemos que $∃n_A \tq x_n∈A \; ∀n≥a$ por definición de convergencia. Luego es claro que $(A\setminus \{ x\}∩W ≠ \emptyset$ y entonces $x∈W'$.
\end{enumerate}
\end{proof}

\begin{remark} Si \sdst es un espacio métrico y $x_n\to x$ y  $x≠z$, entonces $x_n$ no converge a $z$. Es decir, el límite es único.\end{remark}

\begin{proof}
Si $x≠z$, entonces $\dst(x,z)>0$. Sea $δ$ con $0<δ<\frac{\dst(x,z)}{2}$. Por definición de convergencia, existe $n_δ \tq x_n ∈ \bola(x,δ) \; ∀ n≥n_δ$. Por otra parte, $\bola(x,δ) ∩ \bola(z,δ) = \emptyset$, y entonces $x_n \notin \bola(z,δ) \,∀n≥n_δ$, luego es imposible que $x_n$ converja a $z$.
\end{proof}

Atención porque esto sólo pasa en espacios métricos: si cogemos un espacio raro el límite puede dejar de ser único.

Lo que importa realmente del argumento no es que sea métrico, si no que podamos coger dos bolas disjuntas para puntos distintos. Formalicémoslo:


\begin{defn}[Espacio\IS Hausdorff]\label{defHausdorff}
Un espacio topológico \stopl es Hausdorff si $∀x,y \in X$ con $x≠y$ existen $V_x$ entorno de $x$ y $V_y$ entorno de y, tales que $V_x∩V_y = \emptyset$
\end{defn}

\begin{prop} Sea \stopl espacio topológico de Hausdorff. Entonces
\begin{enumerate}
\item Si $x_n\to x$ entonces $x_n$ no converge a $y$ $∀y≠x$ (es decir, el límite de una sucesión, si existe, es único).
\item $\{x\}$ es cerrado $∀x \in X$.
\end{enumerate}
\end{prop}

\begin{proof}
\begin{enumerate}
\item Supongo $x_n\to x$ y $x≠y$

	\begin{enumerate}
	\item Por definición, $\exists V_x,V_y$ entornos de $x$ e $y$ con $V_x∩V_y=\emptyset$.
	\item Si $x_n\to x$ y $x\in V_x$, por definición de convergencia $\exists n_0$ tal que $x_n\in V_x ∀n ≥n_0$. Luego $x_n\notin V_y ∀n ≥ n_0$ y por lo tanto $ x_n$ no converge a $y$.
	\end{enumerate}

\item Queremos demostrar que  $\adh{\{x\}} = \{x\}$, y lo haremos por doble contenido. El contenido a la izquierda es trivial ($\{x\} ⊆ \adh{\{x\}}$) así que sólo tenemos que hacerlo a la derecha.

Si $y≠x$, entonces existen $V_y$ entorno de $y$ y $V_x$ entorno de $x$ tales que $V_x∩V_y=\emptyset$, por lo tanto $V_y∩\{x\}=\emptyset$ , de tal forma que $y\notin \adh{\{x\}}$ y $\adh{\{x\}} ⊆ \{x\}$.
\end{enumerate}
\end{proof}

\begin{remark}
La propiedad 2 ($\{x\}$ cerrado $∀x \in X$) es equivalente a decir que $∀ x,y \in X$ con $x≠y$ existen entornos respectivos $V_x$ y $V_y$ tal que $x\notin V_y$ , $y\notin V_x$.
Espacios topológicos con esa propiedad se llaman $T_1$ (Hausdoff es $T_2$)
\end{remark}

Por ejemplo, $(ℝ^m, \topl_{usual})$, $(ℝ^2, \topl_{Lex.})$, o \sdst son espacios Hausdorff.

\subsection{Topología de subespacios}

\begin{defn}[Topología\IS de subespacio]
Dado \stopl espacio topológico y  $S⊆X$, se define la topología de subespacio en $S$ por:
$V\in \topl^{sub} \equiv \topl^{sub}_S \equiv \exists A\in \topl$ tal que $V = A∩S$.
\end{defn}

Ejercicio: comprobar que es una topología.

\begin{prop}
\stopl e.t, $S⊆X$
\begin{enumerate}
\item $C(⊆S)$ es cerrado en $\topl^{sub} \iff \exists F$ cerrado en \stopl tal que $C=F∩S$.
\item Si $\topl = \toplb$, la topología generada por una base $\base$, entonces
	\begin{enumerate}
	\item $\base_S \equiv \{B∩S : B\in \base\}$ es una base para una topología en $S$
	\item $\topl^{sub} = \topl_{\base_S}$
	\end{enumerate}
\item Si $\topl = \topl_d$ es la topología inducida por una distancia $\dst(x,y)$ en $X$, entonces $\topl^{sub} = \topl_{\dst|_{S×S}}$, tomando $\dst|_{S×S}$ como la restricción de la distancia al conjunto $S$:
\begin{align*}
	\appl{\dst|_{S×S}}{S×S&}{[0,\infty)} \\
	(x,y)&\longmapsto \dst(x,y)
\end{align*}

\end{enumerate}
\end{prop}

\begin{proof}
\begin{enumerate}
\item $C$ cerrado en $\topl^{sub}$ si y sólo si $S\setminus C ∈ \topl^{sub}$, lo cual es equivalente a a que $∃A∈\topl$  tal que $S\setminus C = A∩S$.

Si $F≝X\setminus A$, entonces $A=X\setminus F$, luego volviendo a lo que teníamos nos queda que $S\setminus C = (X\setminus F) ∩ S = S\setminus (F∩S)$, equivalente a $C=F∩S$.
\item Ejercicio
\item Ejercicio
\end{enumerate}
\end{proof}

\section{Continuidad}

En esta sección estudiaremos las funciones $f$ entre dos espacios topológicos $(X,\topl_X)$ y $(Y, \topl_Y)$. Usaremos la forma habitual $\appl{f}{X}{Y}$, aunque para ser más concretos las denotaremos como $\appl{f}{(X,\topl_X)}{(Y, \topl_Y)}$.

\begin{defn}[Función\IS continua]
Sea $x_0∈ X$. Se dice que $f$ es continua en $x_0$ si y sólo si $∀V$ entorno de $f(x_0)$, existe un $U∈\topl_X$ entorno de $x_0$ tal que $f(x)∈V\; ∀x∈U$.

Es decir, que $f(U)⊆ V$ y que $x_0∈ U ⊆\inv{f}(V)$.

Por otra parte, y como hacíamos en otras asignaturas, se dice que una función es continua si es continua en todos los puntos de su dominio.
\end{defn}

\begin{remark} En el caso de espacios métricos $(X, \dst_X), (Y, \dst_Y)$, se dice que $f$ es continua en $x_0$ si y sólo si $∀ε>0\;∃δ>0 \tq \dst_X(x,x_0) < δ \implies \dst_Y(f(x), f(x_0)) < ε$, que es decir de forma más general la forma que hemos puesto de continuidad.

Además, para topologías generadas por bases, basta considerar entornos que están en la base, como siempre.
\end{remark}

Veamos un ejemplo simple sobre continuidad de funciones. Más concretamente, continuidad de funcionales usando sólo la definición de continuidad basada en topologías.

\begin{example} Tomamos $X$ como el conjunto de las funciones continuas $X=C([0,1])≝\{ \appl{f}{[0,1]}{ℝ} \tq f ∈ C^1\}$, y definimos la distancia como
\begin{align*}
\md{f}_{∞} &≝ \max_{x∈[0,1]} \abs{f(x)} \\
\dst(f,g) &≝ \md{f-g}_{\infty}
\end{align*}

A partir de esto definimos el siguiente funcional:
\begin{align*}
\appl{F}{(X, \dst_X)&}{(ℝ, \topl_{usual})} \\
f&\longmapsto F(f) ≝ \int_0^1f(x) \dif x
\end{align*}

Afirmamos que $F$ es continua en $f_0 ∈ X$. Podemos encontrar $\epsilon$ y $\delta$ de la siguiente forma:

\begin{align*}
\dst_{ℝ}(F(f), F(f_0)) &= \abs{F(f) - F(f_0)} = \abs{\int_0^1 f(x) \dif x - \int_0^1 f_0(x)\dif x} = \\
&= \abs{\int_0^1 f(x) - f_0(x) \dif x} \leq \int_0^1 \abs{f(x)-f_0(x) \dif x} \leq \\
&\leq \int_0^1 \md{f(x) - f_0(x)}_{\infty} \dif x \leq \md{f - f_0}_{\infty} = \dst_{X}(f, f_0) < \delta
\end{align*}

Luego, como $\dst_{\mathbb{R}}(F(f), F(f_0)) < \epsilon$, $\forall \epsilon > 0$ , tenemos que:

\begin{align*}
\delta = \epsilon \wedge \dst_{X}(f, f_0) < \delta = \epsilon \implies \dst_{\mathbb{R}}(F(f), F(f_0)) \leq \dst_{X}(f, f_0) - \delta
\end{align*} \qed

De hecho, hemos obtenido un resultado más fuerte: el $\delta$ no depende del punto, luego F es \concept[Continuidad\IS uniforme]{uniformemente continua}.

\end{example}

A partir del ejemplo, podemos desarrollar nuevas propiedades:

\begin{prop} Sean $(X_1, \topl_1), (X_2, \topl_2)$ espacios topológicos y $\appl{f}{X_1}{X_2}$. Entonces

\begin{enumerate}
\item $f$ es continua si y sólo si la imagen inversa de un abierto en $X_2$ es abierta en $X_1$. Es decir, si y sólo si $∀A∈\topl_2 \; \inv{f}(A) ∈ \topl_1$.
\item $f$ es continua si y sólo si la imagen inversa de un cerrado en $X_2$ es cerrada en $X_2$.
\item $f$ es continua si y sólo si $f(\adh{W}) ⊆ \adh{f(W)}\; ∀W⊆X_1$.
\end{enumerate}
\end{prop}

\begin{proof}
\begin{enumerate}
\item Empezamos tomando como hipótesis que $f$ es continua en $x\; ∀x∈X_1$. Tomamos $A∈\topl_2$ y estudiamos su imagen inversa $\inv{f}(A)$ para ver si está en$\topl_1$. Si $x∈\inv{f}(A)$, entonces $f(x) ∈ A$.  Si $A$ es abierto en $X_2$ y $f(x)∈A$, entonces por la definición de continuidad de $f$, $∃V_x$ entorno de $X$ tal que $f(V_x)⊆A$.

Es decir, que $x∈V_x⊆\inv{f}(A)$. Como esto pasa para todo punto, podemos escribir $\inv{f}(A)$ como la unión de todos los $V_x$, la unión de abiertos es abierta y entonces ya tenemos que $A$ es abierto.

En el otro sentido, tomamos como hipótesis que la imagen inversa de un abierto es abierta. Sea $x∈X$, queremos saber si $f$ es continua en $x$.

La definición de continuidad nos decía que dado un entorno $W$ cualquiera de $f(x)$, existía un entorno $V_x$ de $x$ tal que $f(V_x) ⊆ W$. Dicho de otra forma, cercanía en el dominio implica cercanía en la imagen.

Entonces, partiendo de dos premisas ($W∈\topl_2 \implies \inv{f}(W) ∈ \topl_1$, $f(x)∈W \iff x∈\inv{f}(W)$) nos queda que $V=\inv{f}(W)∈ \topl_1$, $x∈V$ y $f(V)⊆W$. % No me queda muy claro esto pero lo dejo así.

\item Queremos demostrar que la imagen inversa de cerrados es cerrada, y vamos a usar la propiedad de que el complementario de un cerrado es abierto. Vemos claramente que $\inv{f}(X_2\setminus A ) = X_1 \setminus \inv{f}(A)$. A partir de esto es muy sencillo probarlo tomando complementarios, y no lo voy a copiar.

\item Tomamos como hipótesis que $f$ es continua, y queremos demostrar que $f(\adh{W}) ⊆ \adh{f(W)}\; ∀W⊆X_1$. Está claro que $\adh{f(W)}$ es cerrado, y que al ser $f$ continua entonces $\inv{f}(\adh{f(W)})$ es cerrado también. Además, es seguro que $W⊆\inv{f}(\adh{f(W)})$. Uniendo las dos cosas, tenemos que $\adh{W} ⊆ \inv{f}(\adh{f(W)})$.
\end{enumerate}
\end{proof}

Vamos a ir ahora a por algunas propiedades sobre la composición de funciones.

\begin{prop} Sean $(X,\topl_X), (Y, \topl_Y), (Z, \topl_Z)$ espacios topológicos. Entonces

\begin{enumerate}
\item Si $\appl{f}{X}{Y}$ y $\appl{g}{Y}{Z}$  son continuas entonces $\appl{g○f}{X}{Z}$ también lo es.
\item Si $\appl{f}{X}{Y}$ es constante entonces $f$ es continua.
\item Si $\appl{f}{X}{Y}$continua y $S$ subespacio de $X$ (es decir, $S⊆X$ con $\topl^{sub}$), entonces $\appl{f|_S}{S}{Y}$ es continua.
\item Sea $\appl{f}{X}{Y}$ y $W$ tales que $f(X) ⊆ W ⊆ Y$. Denotemos $\appl{f^W}{X}{W}$ con $W$ en la topología de subespacios. Entonces $f$ es continua si y sólo si $f^W$ es continua.
\end{enumerate}
\end{prop}

\begin{proof}
\begin{enumerate}
\item Sea $A$ abierto en $Z$, entonces tenemos que demostrar que $(g○f)^{-1}(A)$ es abierto en $X$. Sabemos que $\inv{(g○f)}(A) = \inv{f} (\inv{g}(A))$.

Como g es contínua $\implies \inv{g}(A)$ es abierto en Y; como f es contínua, $\inv{f} ($abierto$)$ es abierto en X. % QED bitch!
\end{enumerate}
\end{proof}

\begin{remark} Si tenemos una aplicación $\appl{f}{(X,\topl_X)}{(Y, \topl_Y)}$ donde $\topl_Y = \toplb$ con $\base$ una base, entonces $f$ es continua si y sólo si $\inv{f}(B)∈\topl_X\;∀B∈\base$.
\end{remark}


\begin{defn}[Homeomorfismo] Un homeomorfismo entre espacios topológicos $(X,\topl_X)$, $(Y, \topl_Y)$ es una aplicación biyectiva $\appl{f}{X}{Y}$ y tal que tanto $f$ como $\inv{f}$ son continuas. En este caso se dice que $(X,\topl_X)$, $(Y, \topl_Y)$ son homeomorfos.
\end{defn}

\begin{defn}[Propiedad\IS topológica] Una propiedad de un espacio topológico $(X, \topl_X)$ es topológica si la comparten todos los espacios topológicos homeomorfos a $(X, \topl_X)$.\end{defn}

\begin{remark} La topología de subespacios usual en $\bola(0,1)$ es $\topl_{\dst}$ donde $\dst$ es la distancia euclídea en $\bola(0,1)$. \end{remark}

Podemos demostrar que $ℝ^2$ y $\bola(0,1)$ son homeomorfos, tomando la biyección

\begin{align*}
\appl{f}{ℝ^2&}{\bola(0,1)} \\
\vx&\longmapsto \frac{\vx}{1+\md{\vx}} = \vy
\end{align*}

Demostremos que es biyectiva obteniendo su inversa: \[ \md{\vy} = \frac{\md\vx}{1 + \md{\vx}} \implies \md{\vx} = \frac{\md{\vy}}{1- \md{\vy}} \], luego \[ \vx = \frac{\vy}{1-\md{\vy}} \]. Y se deja como ejercicio para el lector demostrar que es continua. % Precioso

Otro ejemplo: vamos a demostrar que $[0,1]$ y $(0,1)$ no son homeomorfos. La propiedad que nos va a interesar sería la compacidad, pero todavía no tenemos la maquinaria para hacerlo.

Lo que haremos será ver qué ocurre si quitamos un punto. Cualquier punto que quitemos en $(0,1)$ nos dejará dos trozos no conexos. Ahora bien, si lo hacemos en $[0,1]$, podemos quedarnos con un trozo conexo si el punto es $0$ o $1$.

Vamos a demostrar que no son homeomorfos viendo que si $\appl{f}{[0,1]}{(0,1)}$ es continua, entonces no es sobreyectiva.

Un pequeño aparte: tenemos que entender que cada intervalo está con la topología del subespacio. En el caso de $\topl_{[0,1]}^{sub}$, la topología del subespacio sería la generada por los elementos de la base de $\topl_ℝ$ intersección $[0,1]$. Es decir, la base es \begin{multline*} \base = \{ (a,b) ∩ [0,1] \tq a < b, a,b∈ℝ \} \equiv \\ \equiv \left\{ [0,1], [0,b), (a, 1], (c,d) \tq 0 < b,a < 1,\; 0<c<d<1 \right\} \end{multline*}

Volviendo a lo que íbamos: no tendremos problema por parte de la imagen inversa. En este caso, la imagen inversa de un abierto $(0,1)$ es un abierto: $[0,1]$ es abierto en la topología de su subespacio (es el total).

Donde sí vamos a lograr algo es viendo que $f$ continua en $[0,1]$ alcanza un máximo. Es decir, $∃x_0 ∈ [0,1]$ tal que $f(x) ≤ f(x_0)\; ∀x∈[0,1]$.

Por otra parte, es obvio que $f(x_0)∈(0,1)\implies f(x_0) < 1$. Juntando estas dos conclusiones llegamos a que $(f(x_0), 1) ∩ f([0,1]) = \emptyset$. Es decir, $f$ no es sobreyectiva (por ejemplo, $\frac{1+f(x_0)}{2}$ no está en la imagen).

Otra forma sería ver que si $f$ fuese inyectiva, tendría que ser monótona creciente o decreciente. Podríamos suponer sin pérdida de generalidad que fuese creciente, y entonces $f(0) < f(x)\; ∀x∈(0,1]$. Sin embargo, siempre podríamos encontrar un $y ∈ (0, f(0))$ por lo que no sería sobreyectiva.

\paragraph{Propiedad topológica Hausdorff} Recordamos los espacios Hausdorff (ver \ref{defHausdorff}).  Queremos demostrar que si $(X,\topl_X)$ es Hausdorff y  $(X,\topl_X)$ es homeomorfo a $(Y, \topl_Y)$, entonces $(Y, \topl_Y)$ es Hausdorff igualmente.

Tomamos $x_1∈V_1$ y $x_2∈V_2$, ambos en $X$, con $y_j=f(x_j)$. Hay que comprobar que $y_j∈f(V_j)$, que $f(V_j)$ es abierto en $Y$ y que la intersección $f(V_1) ∩ f(V_2) = \emptyset$. Se deja como ejercicio probarlo.

\begin{remark} Si $f$ es un homemorfismo entonces $\inv{f}$ es continua. Dicho de otra forma, la imagen inversa de la inversa de un abierto es abierto. % WTF.

También se puede decir que $f$ es homeomorfismo si $f$ es biyectiva, y que la imagen inversa de $\inv{f}$ de un abierto es abierto, y que $f$ de un abierto es abierto también.
\end{remark}

\begin{defn}[Función\IS abierta] Sean $(X,\topl_X)$, $(Y, \topl_Y)$ espacios topológicos y $\appl{f}{X}{Y}.$ Se dice que $f$ es abierta si y sólo si $∀A∈\topl_X$ se tiene que $f(A) ∈ \topl_Y$.

Por otra parte, se dice que una función es una \concept[Función! cerrada]{función cerrada} si y sólo si para todo $C$ cerrado en $X$, $f(C)$ es cerrado en $Y$.
\end{defn}

Por ejemplo, si $Y$ es Hausdorff (\ref{defHausdorff}) y $f$ es constante, entonces es cerrada. La razón es que en un espacio Hausdorff, un único punto siempre es un conjunto cerrado.

Las dos definiciones no son excluyentes. Un homeomorfismo es aplicación abierta y cerrada, por ejemplo. También podríamos construir una aplicación $f$ abierta y cerrada sin que sea necesario homeomorfismo. Podemos coger una $f$ constante e $Y$ con la topología discreta, o con $\topl_Y = \{ \emptyset, \{p\}, \{p\}^c, Y\}$.

\section{Topología producto}

Ahora consideraremos dos espacios topológicos $(X_1, \topl_1), (X_2, \topl_2)$, y estudiaremos la topología en $X_1×X_2$.

\begin{prop} Sea $\base = \{ V_1 × V_2 \tq V_i ∈ \topl_i \}$. $\base$ es una base para una topología en $X_1 × X_2$. $\topl_\base$ es la topología producto y se denota $\topl_1 \otimes \topl_2$ (habitualmente $\topl_1× \topl_2$).
\end{prop}

\begin{proof}
Vamos a demostrar que efectivamente es una topología. Está claro que $\bigcup_{B∈\base} B = X_1 × X_2$, ya que $X_1$ y $X_2$ son el total y están en sus respectivas topologías, por lo que $X_1× X_2∈\base$.

Por otra parte, queremos demostrar que la intersección también está.  Si tenemos $B, C ∈ \base$ y  $x∈B∩C$, entonces existe un $\tilde{B}∈\base$ tal que $X∈\tilde{B}⊆B∩C$.

Para ello usaremos que $(V_1×V_2) ∩ (\hat{V}_1 × \hat{V}_2) = (V_1 ∩ \hat{V}_1) × (V_2 ∩ \hat{V}_2)$, luego $B∩C ∈ \base$ y $\tilde{B} = B \cap C$.

Esto es así porque $(x_1, x_2) = x \in (V_1 \times V_2) \cap (\hat{V_1} \times \hat{V_2}) \iff \begin{cases} x_1 \in V_1 \land x_1 \in \hat{V_1} \\ x_2 \in V_2 \land x_2 \in \hat{V_2} \end{cases} \iff \begin{cases} x_1 \in V_1 \cap \hat{V_1} \\ x_2 \in V_2 \cap \hat{V_2} \end{cases} \iff x \in (V_1 \cap \hat{V_1}) \times (V_2 \cap \hat{V_2})$

\end{proof}

La topología producto no se limita a sólo dos espacios topológicos.

\begin{prop} Más generalmente, si $\{X_j,\topl_j\}_{j=1,\dotsc, m}$ es un conjunto de espacios topológicos, entonces $\base = \{ V_1× V_2 × \dotsb × V_m \tq V_j ∈ \topl_j\}$ es una base para una topología producto $\topl_1 \otimes \topl_2 \dotsb \otimes \topl_m$.
\end{prop}

Como ejercicio, podemos considerar tres espacios topológicos, y estudiar si la topología producto es asociativa, es decir, si

\[ \topl_1 \otimes \topl_2 \otimes \topl_3 = (\topl_1 \otimes \topl_2)\otimes \topl_3 = \topl_1 \otimes (\topl_2 \otimes \topl_3 ) \]

Otro ejemplo, ¿son iguales $\topl_{ℝ^2} = \topl_ℝ \otimes \topl_ℝ$? En el primer caso, la base $\base_1$ de $ℝ^2$ son elementos de la forma $(a_1,b_1) × (a_2, b_2)$, es decir, rectángulos. La segunda base $\base_2$ está formada por los elementos $V_1 × V_2$ donde $V_i ∈ \topl_ℝ$.

Para demostrarlo vamos a ver lo de siempre: las bases molan. Si tenemos $\base$ una base y $\topl$ una topología con $B⊆\topl$, entonces $\topl_\base⊆\topl$. Es decir, $\topl_\base ⊆ \topl \iff \base ⊆ \topl$.

Como consecuencia de esa observación, si $\base_1,\base_2$ son bases entonces $\topl_{\base_1} ⊆ \topl_{\base_2} \iff \base_1 ⊆ \topl_{\base_2}$ y, por tanto \[ \topl_{\base_1} = \topl_{\base_2} \iff \base_1 ⊆ \topl_{\base_2} \y \base_2 ⊆ \topl_{\base_1} \]

Con esto ya podemos volver a nuestra vida normal y demostrar que $\topl_{ℝ^2} = \topl_ℝ \otimes \topl_ℝ$. Si $B∈\base_1$ con $B=(a_1,b_1) × (a_2, b_2)$, entonces como $(a_1,b_1)$ y $(a_2, b_2)$ son abiertos en $\topl_ℝ$, $B∈\base_2$ y $\base_1 ⊆ \base_2 ⊆ \topl_{\base_2}$. Con esto hemos demostrado el contenido hacia la izquierda ( $\topl_{ℝ^2} ⊆ \topl_ℝ \otimes \topl_ℝ$).

Vamos ahora para el otro lado. Si $B∈\base_2$ con $B=V_1 ×V_2$, con $V_i ∈ \topl_ℝ$, tenemos que $V_1 = \bigcup_{j∈ J_1}(a_j, b_j)$ y análogamente con $V_2 = \bigcup_{k∈K_2} (c_k, d_k)$. Entonces

\[ V_1 × V_2 = \bigcup_{\substack{j∈J_1 \\ k∈K_2}} (a_j, b_j) ×(c_k, d_k) \], por lo tanto $V_1×V_2$ es unión de elementos de $\base_1$ y por lo tanto está en $\topl_{\base_1}$, luego $\base_2 ⊆ \topl_{\base_1}$ y tenemos la inclusión para el otro lado, y entonces $\topl_{\base_1} = \topl_{\base_2}$.

Pero la justificación de esto no sólo nos vale para $ℝ$, sino que puede ser mucho más general si no escribimos explícitamente los intervalos de $ℝ$, que no son más que abiertos de la topología.

\begin{prop} Sean $(X, \topl_{\base_X})$, $(Y, \topl_{\base_Y})$. Entonces \[ \topl_{\base_X} \otimes  \topl_{\base_Y} = \topl_\base \] donde \[ \base =\left\{ B × \hat{B} \tq B ∈ \base_X, \hat{B} ∈ \base_Y \right\} \].
\end{prop}

\begin{prop} Sean $(X_1, \dst_1)$ y $(X_2, \dst_2)$ espacios métricos. Entonces $\topl_{\dst_1} \otimes \topl_{\dst_2} = \topl_{\dst}$ donde \[ \dst\left((x_1, x_2), (y_1, y_2)\right) = \dst_1(x_1, y_1) + \dst(x_2, y_2)\]
\end{prop}

La demostración se deja como ejercicio para el lector, aunque es parecido al ejercicio 18 de la hoja 1.

\subsection{Funciones continuas y topología producto}

La gran ventaja de la topología producto nos va a venir a la hora de comprobar continuidad de funciones en espacios producto. Vamos a verlo.

\begin{prop} Sean $(X_1, \topl_1), (X_2, \topl_2)$ espacios topológicos. Entonces

\begin{enumerate}
\item Sean $\appl{P_j}{(X_1×X_2, \topl_1×\topl_2)}{(X_j, \topl_j)}$ con $P_1((x_1, x_2)) = x_1$ y $P_2((x_1, x_2)) = x_2$ las proyecciones. Entonces $P_1$ y $P_2$ son continuas.

\item Sea $\appl{f}{(X, \topl)}{(X_1×X_2, \topl_1 \otimes \topl_2)}$ tal que $x\longmapsto f(x) = (f_1(x), f_2(x))$. Entonces $f$ es continua si y sólo si $f_1, f_2$ son continuas.

\item Si $\appl{f, g}{(X, \topl)}{(ℝ, \topl_ℝ)}$ son continuas entonces $f \pm g$, $f \cdot g$ y $\frac{f}{g}$ son continuas, en el último caso suponiendo que $g(x) ≠ 0$.
\end{enumerate}

Las propiedades se mantienen para un producto finito de espacios.
\end{prop}

\begin{proof}
\begin{enumerate}
\item Si $P_1$ es continua, entonces $\inv{P_1}(A_1) ∈ \topl_1 \otimes \topl_2\quad ∀ A_1 ∈ \topl_1$. Tenemos que ver primero qué es $\inv{P_1}(A_1)$. Al ser la proyección, $\inv{P_1}(A_1) = A_1 × X_2$, que es claramente un abierto en la topología producto (es un elemento de la base).

\item Podemos decir que $f_1 = P_1 ○ f$ y $f_2 = P_2 ○ f$. La implicación a la derecha es obvia: si $f$ es continua, como $P_1$ y $P_2$ son continuas entonces $P_1 ○ f$ y $P_2 ○ f$ son continuas.
\end{enumerate}
\end{proof}

 %
\begin{prop} Sean $(X_i, \topl_i)$ con $i=1,2$ espacios topológicos, y consideramos las proyecciones $\appl{p_i}{X_1×X_2}{X_i}$ las proyecciones. $p_1$ y $p_2$ son abiertas, suponiendo en $X_1×X_2$ la topología producto.

\end{prop}

\begin{proof}
Tenemos que demostrar que si $A∈ \topl_1 \otimes \topl_2$, entonces $p_1(A) ∈ \topl_1$. Los abiertos de la topología producto pueden ser bastante raros, pero el caso básico ($A = V_1 × V_2$), donde $V_1$ y $V_2$ son abiertos, lo podemos justificar rápidamente porque $p_1(V_1) = V_1$, que es abierto.

Pero consideremos el caso general, en el que $A∈\topl_1 \otimes \topl_2 = \topl_\base$. Entonces \[ A = \bigcup_{j∈J} V_1^j × V_2^j \]. Ahora bien, sabemos que la imagen de la unión de conjuntos es la unión de las imágenes, luego $p_1(A)$
 será \[ p_1(A) = \bigcup_{j∈J} p_1(V_1^j × V_2^j) = \bigcup_{j∈J}V_1^j\], que es abierto.
\end{proof}

\begin{remark} Las proyecciones $p_i$ no son cerradas en general. Por ejemplo, en $ℝ^2$ con la topología usual, tendríamos que escoger un $F$ en $ℝ^2$ que fuese cerrado pero con $p_1(F)$ no cerrado en $ℝ$. Por ejemplo, $F= \{ (x_1, x_2) \tq x_1x_2 ≥ 1\}$, pero $p_1(F) = (0, +∞)$.

Puede parecer un poco contraintuitivo que $F$ sea cerrado, pero lo es: $f(x_1, x_2) = x_1x_2 - 1$ es continua y $F = \inv{f}([0, ∞))$, y la imagen inversa de un cerrado es cerrado.
\end{remark}

\begin{remark} $\topl_1\otimes\topl_2$ es la más pequeña para la que las proyecciones son continuas. La demostración está en ver qué es $\inv{p_1}(V_1) ∩ \inv{p_2}(V_2)$.
\end{remark}

\chapter{Propiedades de los espacios topológicos}

Hasta ahora hemos estado aprendiendo sólo el lenguaje, lo básico. Ahora vamos a ir a por la topología de verdad, empezando por conexión y compacidad, dos conceptos muy potentes. Hasta ahora, hemos podido hablar sólo de continuidad y convergencia.

\section{Conexión}

Empecemos con un ejemplo. Si consideramos $A=(-1, 0) ∪ (1,0)$ y $B=(2,3)$, vemos que no son homeomorfos. Para ello, necesitaríamos encontrar una función $f$ continua e inyectiva, luego tiene que ser monótona. La imagen de un abierto es un abierto, así que deberíamos poder escribir $B$ como unión de dos abiertos disjuntos, pero no podemos: podríamos coger el ínfimo o supremo de cualquiera de ellos y esos no estarían en el intervalo.


\begin{defn}[Conexión]
	Sea $(X, \topl)$ un espacio topológico. Tenemos 2 definiciones:

	\begin{enumerate}
		\item $X$ es conexo si no existen $V_1, V_2$ abiertos $(V_i \in \topl)$ tales que $V_1 \cap V_2 = \emptyset$, $V_1 \cup V_2 = X$, y $V_1 ≠ \emptyset ≠ V_2$. Análogamente, $X$ es no conexo si existen esos intervalos $V_1,V_2, \ldots$ que cumplan esas condiciones.

		\item Sea $W ⊆ X$, $W$ es conexo si y sólo si $(W, \topl_X^{sub})$ es conexo.
	\end{enumerate}
\end{defn}


\begin{remark}
	$W⊆X$ no conexo quiere decir que $∃V_1, V_2 \in \topl^{sub}$ no vacíos y tal que $V_1 \cap V_2 = \emptyset$, $V_1 \cup V_2 = W$. Recordamos además la topología de subespacio:

	\[ V_1, V_2 \in \topl^{sub} \iff V_i=A_i \cap W, A_i \in \topl \]

	Entonces, que $W⊆X$ sea no conexo quiere decir que $∃A_1, A_2 \in \topl$ no vacíos con $A_1 \cap W ≠ \emptyset$, $A_2 \cap W ≠ \emptyset$, disjuntos en $W$ (esto es, que $A_1 \cap A_2 \cap W = \emptyset$), y cuya unión contiene a $W$: \[ (A_1 \cap W) \cup (A_2 \cap W) = (A_1 \cup A_2) \cap W = W \implies W ⊆ A_1 \cup A_2 \]

	Hay que tener en cuenta, eso sí, que $A_1, A_2$ se pueden cortar fuera de $W$ sin que afecte a la prueba.
\end{remark}

\begin{remark}
	En espacios topológicos que no son conexos puede haber conjuntos abiertos y cerrados.
\end{remark}

\begin{prop}
	Sean $(X,\topl_X),(Y,\topl_Y)$ espacios topológicos y $\appl{f}{X}{Y}$ continua.
	Si $W ⊆ X$ es conexo, entonces $f(W)$ es conexo.
\end{prop}

\begin{corol}
 Ser conexo es una propiedad topológica
\end{corol}

\begin{proof}[Proposición]
	Supongamos que $f(W)$ no es conexo. Entonces existen $A_1, A_2 ∈ \topl_Y$ no vacíos, con $A_i \cap f(W) ≠ \emptyset$, tales que $A_1 \cap A_2 \cap f(W) = \emptyset$ y $f(W) ⊆ A_1 \cup A_2$

	Sea $B_j = f^{-1}(A_j)$. Entonces, vamos a demostrar que $W$ no es conexo.
	\begin{enumerate}
		\item $B_1, B_2 \in \topl_x$ ya que $f$ es continua.

		\item $B_j \cap W ≠ \emptyset$, porque $f^{-1}(A_j \cap f(W)) ⊆ B_j \cap W ≠ \emptyset$

		\item $B_1 \cap B_2 \cap W = \emptyset$: si $x ∈ B_1 \cap B_2 \cap W$, entonces $f(x) ∈ f(B_1) \cap f(B_2) \cap f(W)$. Sabemos que $f(B_1) ⊆ A_1$ y $f(B_2) ⊆ A_2$.\footnote{Recordemos que en general $f(\inv{f}(A)) ≠ A$, por ejemplo si $f$ es constante.}

		Entonces $f(x) ∈ A_1 \cap A_2 \cap f(W) ≠ \emptyset$ y llegamos a contradicción, ya que al principio supusimos que $A_1 \cap A_2 \cap f(W) = \emptyset$.

		\item $W ⊆ B_1 \cup B_2$. ¿Por qué? Sabemos que $f(W) ⊆ A_1 \cup A_2$, así que \[ W \subseteq f^{-1}(f(W)) ⊆ f^{-1}(A_1 \cup A_2) = f^{-1}(A_1) \cup f^{-1}(A_2) = B_1 \cup B_2 \]

		Es decir, $W ⊆ B_1 \cup B_2$.
	\end{enumerate}

	Uniendo estas cuatro proposiciones, llegamos a que si $f(W)$ no es conexo, entonces tampoco lo es $W$, una contradicción.
\end{proof}

\begin{proof}[Corolario]
	$(X, \topl_X)$ conexo y $\appl{f}{X}{Y}$ homeomorfismo (y por tanto continua) $\implies Y = f(X)$ es conexo.

	Por tanto la conexión es propiedad topológica.
\end{proof}


\begin{prop}
	Sea $(X,\topl)$ un espacio topológico:

	\begin{enumerate}
		\item $X$ es conexo $\iff ∄F ⊆ X$ tal que $F$ es abierto, cerrado y $F≠ \emptyset, X$

		\item
		\[\left.
			\begin{array}{cc}
				W ⊆ X \text{\ conexo} \\ \\
				W ⊆ A \cup B \\ \\
				A,B \text{\ abiertos disjuntos}
			\end{array}
		\right\} \implies W ⊆ A \text{\ ó \ } W ⊆ B \]
	\end{enumerate}
\end{prop}

\begin{proof}
	\begin{enumerate}
		\item $V_1 = F, V_2 = F^c$ en la definición
		\begin{remark}
			$F ≠ \emptyset$, $X$ por definición es subconjunto propio.
		\end{remark}

		\item Si no es cierto que $V_1 = A \cap W$, $ V_2 = B \cap W$ muestran que W no es conexo.\\
		\[\left.
			\begin{array}{cc}
				W \nsubseteq A \\ \\
				W \nsubseteq B \\ \\
				W ⊆ A \cup B
			\end{array}
		\right\} \implies
		\left.
			\begin{array}{cc}
				A \cap W ≠ \emptyset \\ \\
				B \cap W ≠ \emptyset
			\end{array}
		\right\} \implies
		A \cap B = \emptyset \implies A\cap B \cap W = \emptyset
		\]

		Por tanto $W ⊆ A \cup B$
	\end{enumerate}
\end{proof}


\begin{example} Vamos a ver que $ℝ \setminus \{0\}$, con la topología usual $\topl_{usual}$ no es conexo. Recordamos que $ℝ \setminus \{0\} = (-∞, 0) \cup (0, ∞)$.

Los dos subintervalos $V_1 = (-∞ , 0)$ y $V_2 = (0, ∞)$ son abiertos en  $\topl^{sub}$ no vacíos, disjuntos y cuya unión es $ℝ \setminus \{0\}$. Esto implica que no es conexo.
\end{example}


\begin{prop}\label{propConexoIntervalo} Consideramos $(ℝ, \topl_{usual})$ y un conjunto $W ⊆ ℝ$. $W$ es conexo si y sólo si $W$ es un intervalo.

Por aclarar, $I ⊆ ℝ$ es un intervalo si $∀a,b,c$ con $a,b ∈ I$, $a<c<b$ se tiene $c ∈ I$,
\end{prop}


\begin{proof}
	\begin{enumerate}
		\item $\implies)$\\
		Hipótesis: $W$ es conexo\\
		Si no es intervalo, $∃ a,b,c$ con $a,b ∈ W$, $a<c<b$, $c ∉ W$

		Entonces $a ∈ V_1 = (-∞, c) \cap W$\\
		$b ∈ V_2 = (c, ∞) \cap W$

		Estos dos conjuntos son abiertos en $\topl^{sub}$ no vacíos disjuntos y su unión es $W$. Por tanto llegamos a una contradicción, ya que $W$ es conexo y no se debería poder expresar como unión de abiertos disjuntos.

		\item $\impliedby)$ Si $W$ es un intervalo, entonces es conexo.

		Si $W$ no fuese conexo, existirían $A,B$ abiertos en $ℝ$ tales que $A∩W$ y $B∩W$ no son vacíos, que $A∩B∩W = \emptyset$ y que $W⊆ A∪B$. Es claro que existen $a,b$ en $A∩W, B∩W$ respectivamente tales que $a≠b$. Supongamos $a<b$, y definimos \[ M ≝ \{ x∈A \tq x < b\} \]
		Sabemos que $M≠\emptyset$ ya que $a∈M$. Además, por la propia definición $M$ está acotado superiormente por $b$.

		Sea $c = \sup M$. $c$ cumple que $a≤c≤b$, luego $c∈W$. De aquí sacamos varias conclusiones, siendo la primera que $c\notin A$. Si lo fuese, como $A$ es abierto, existiría $δ>0$ tal que $(c-δ, c+δ)⊆A$, por lo que $c$ ya no sería el supremo.

		Por tanto, $c \notin A$.

		Siguiendo el mismo argumento, veremos que $c∉B$, lo que nos lleva a una contradicción, por lo que entonces si $W$ es intervalo, es conexo.
	\end{enumerate}
\end{proof}

Esto que hemos visto en $ℝ$ en la proposición \ref{propConexoIntervalo} nos vale para cualquier conjunto $X$ totalmente ordenado con la topología del orden $\topl_<$ siempre que $(X,<)$ tenga la propiedad del supremo, o lo que es lo mismo, que cualquier subconjunto de $X$ no vacío y acotado superiormente tiene un supremo.

Por ejemplo, $([0,1]×[0,1], <_{Lex})$ tiene la propiedad del supremo. Cojamos un $A⊆[0,1]×[0,1]$ no vacío, y consideramos su proyección \[ A_1 = \{ x∈[0,1] \tq ∃y \tq (x,y) ∈ A\} = P_1(A)\] y el supremo de ese conjunto $α = \sup A_1$. Si $\left(\{ α\} × [0,1]\right) ∩ A = \emptyset$, entonces $\sup A = (α, 0)$. Si no, definimos de la misma manera la proyección $A_2 = P_2(A)$, con $β = \sup A_2$ y entonces $\sup A = (α, β)$.

En general, conexo en $(X, \topl_<)$ implica intervalo en $(X, <)$.

\subsection{Construcción de conjuntos conexos}

Más sobre este tema: ¿qué intervalos son conexos en $(ℝ^2, \topl_{Lex})$?

Por ejemplo, el intervalo $\{x_0\} × ℝ$, ¿es conexo? Desde luego, y podemos justificarlo de varias formas. El conjunto es conexo si y sólo si lo es en la topología del subespacio, y en este caso la topología del subespacio son la de intervalos abiertos verticales, y nuestra recta $x=x_0$ no es más que unión de ellos. También podemos considerar una aplicación \begin{align*}
\appl{f}{(ℝ, \topl_{usual})&}{(ℝ^2, \topl_{Lex})} \\
y &\longmapsto (x_0, y)
\end{align*} continua y, por tanto $A = f(ℝ)$ es conexo. De hecho, $\appl{f}{(ℝ, \topl_{usual})}{(A, \topl_{Lex}^{sub})}$ es un homeomorfismo y, por tanto, los subconjuntos conexos de $A$ son los intervalos en $A$.

De hecho, esos $A$ son todos los subconjuntos conexos. Cojamos otro conjunto $B$ cualquiera y consideremos las rectas verticales en $x$ $ℝ_x = \{ (x,y) \tq y ∈ ℝ\}$. Si $∃x_0, x_1$ tales que $B∩ℝ_{x_0} ≠ \emptyset $ y $B∩ ℝ_{x_1} ≠ \emptyset$, entonces $B$ no es conexo. ¿Por qué?

Podríamos expresar $B$ como la unión de $B∩ℝ_{x_0}$, que es un abierto en $B$ no vacío; y de $B∩(ℝ^2\setminus ℝ_{x_0})$, también abierto en $B$ y no vacío. Además, son disjuntos, por lo que cualquier otro conjunto $B$ que no sea una banda vertical será no vacío.

Aquí hemos usado la definición, pero no suele ser la forma más cómoda de trabajar con ellos. Vamos a buscar más formas de construirlos y estudiarlos.

\begin{prop} Sea $(X, \topl)$ espacio topológico.

\begin{enumerate}
\item Si $A,B$ son conexos y $A∩B ≠ \emptyset$ entonces $A∪B$ es conexo. Mas generalmente, si $C_0$ y $C_j, j∈J$ son conexos y $C_0∩C_j ≠ \emptyset\; ∀j∈J$, entonces $C_0 ∪ \left(\bigcup_{j∈J} C_j\right)$ es conexo.

\item Si $W$ es conexo y $W⊆D⊆\adh{W}$, entonces $D$ es conexo. Es decir, que si a un conexo le añadimos puntos en su adherencia, se obtiene otro conexo.
\end{enumerate} \label{propUnionConexa}
\end{prop}

\begin{proof}
\begin{enumerate}
\item Supongamos que $A∪B$ no es conexo. Entonces $∃V_1, V_2$ abiertos en $X$ que cumplen las tres siguientes propiedades:

\begin{enumerate}
\item $V_j∩ (A∪B) ≠ \emptyset$.
\item $V_1∩V_2∩(A∪B) = \emptyset$.
\item $A∪B ⊆ V_1 ∪ V_2$.
\end{enumerate}

En ese caso, tendríamos que \[ A = (A∩V_1) ∪ (A∩V_2)\], donde ambas partes son abiertos en $A$ y disjuntos. Además, $A∩ V_1∩V_2 = \emptyset$. Como $A$ es conexo, lo único que puede ocurrir es que o bien $A∩V_1$ o $A∩V_2$ es el vacío, es decir, $A⊆V_1$ o $A⊆V_2$.

Digamos que $A⊆V_1$. Por el mismo argumento, $B⊆V_1$ ó $B⊆V_2$. Pero como $A∩B ≠ \emptyset$ por hipótesis, tiene que ser $B⊆V_1$, y entonces $V_2 ∩ (A∪B) = \emptyset$, contradicción con la hipótesis del principio (decíamos que $V_j∩ (A∪B) ≠ \emptyset$).

\item Empezamos de la misma forma que antes. Suponemos que $D$ no es conexo, así que $∃V_1, V_2$ abiertos en $X$ tales que $V_j∩ D ≠ \emptyset$, $V_1∩V_2∩D = \emptyset$ y $D ⊆ V_1 ∪ V_2$.

Como $W$ es conexo, tenemos o bien $W∩V_1 = \emptyset$ o bien $W∩V_2 = \emptyset$. Digamos $W∩V_1 = \emptyset$. Entonces $\adh{W} ∩ V_1 = \emptyset$. ¿Por qué?

Si recordamos, la adherencia de $W$ son todos los puntos tales que todo entorno abierto corta a $W$. Si $x ∈ V_1$, entonces $V_1$ es entorno de $x$, pero como $V_1∩W = \emptyset$ entonces $x∉\adh{W}$. Pero eso nos llevaría a que $V_1∩D ⊆ V_1 ∩ \adh{W} = \emptyset$, y $V_1 ∩ D = \emptyset$, contradicción.
\end{enumerate}
\end{proof}

\begin{wrapfigure}{R}{0.3\textwidth}
\centering
\inputtikz{II_ConjuntoEstrellado}
\caption{Conjunto estrellado, unión de segmentos.}
\label{figEstrellado}
\end{wrapfigure}

Vamos a ver algunos ejemplos de estas construcciones en $ℝ^2$. Por ejemplo, en la figura \ref{figEstrellado}, vemos un conjunto estrellado, que no es más que unión de segmentos. Y como cada segmento es imagen continua de un intervalo en $ℝ$, entonces es conexo.

Por concretar, vamos a definir qué estamos usando. Un segmento $[\vx,\vy] = \{ t\vx + (1-t) \vy \}$ es conexo, ya que podemos construir una aplicación

\begin{align*}
\appl{f}{[0,1]&}{ℝ^m} \\
t & \longmapsto t\vx + (1-t)\vy
\end{align*}

continua y $f([0,1]) = [\vx,\vy]$.

\begin{defn}[Conjunto\IS estrellado] Un conjunto estrellado en $ℝ^m$ es un conjunto $E$ tal que $∃\vec{x_0}∈E$ tal que $∀\vx∈E$ se tiene que $[\vec{x_0}, \vx]⊆E$. \label{defEstrellado}
\end{defn}

Además los convexos también son conexos (son estrellados con respecto a cualquiera de sus puntos). Obviamente, entonces las bolas euclídeas, tanto abiertas como cerradas, son conexas.

\begin{figure}[hbtp]
\centering
\inputtikz{II_Peine}
\caption{Peine topológico.}
\label{figPeine}
\end{figure}

Un ejemplo muy interesante es el del peine (figura \ref{figPeine}). Este conjunto se define como

\[ P = \left(\bigcup_{n=1}^∞ \underbrace{\left(\left\{\frac{1}{2^{n-1}}\right\} × [0,1]\right)}_{C_n}\right) ∪ \underbrace{\left([0,1]×\{0\}\right)}_{C_0} \]

y es conexo, ya que la intersección $C_0 ∩ C_n ≠ \emptyset\; ∀n$, y $C_n$ siempre es un segmento conexo. Como $P$ es una unión, es conexo.

Además, $P∪ \{ (0,1)\}$ también es conexo, ya que $P⊆ P ∪ \{ (0,1) \} ⊆ \adh{P} = P∪(\{0\}×[0,1])$.

Eso sí, hay que tener cuidado porque $[0,1)$ no es homeomorfo a $(0,1)$. Para ello, usaremos el siguiente lema:

\begin{lemma} Sean $(X, \topl_X), (Y, \topl_Y)$ espacios topológicos y $\appl{f}{X}{Y}$ homeomorfismo. Entonces, para todo $W⊆X$ se tiene que la aplicación restringida al subespacio \[ \appl{f|_W}{(W,\topl_X^{sub})}{(f(W), \topl_Y^{sub})} \] es un homeomorfismo. Está claro que en ese caso $W$ y $f(W)$ son homeomorfos.
\end{lemma}

Usando este lema, tenemos que como $W=(0,1) ⊆ [0,1]$ es conexo, si $\appl{f}{[0,1)}{(0,1)}$, entonces podemos construir una aplicación \[ \appl{f|_{(0,1)}}{(0,1)}{(0,1)\setminus\{f(0)\}} \]

Sin embargo, denotando $f(0) = a ∈ (0,1)$, tendríamos que $f((0,1)) = (0,a) ∪ (a,1)$, lo que es imposible porque la imagen por una función continua de un conexo tiene que ser conexa. Luego $[0,1)$ y $(0,1)$ no son homeomorfos.

Vamos a ver más construcciones para conjuntos conexos.

\begin{prop} Sean $(X_j, \topl_j)$ con $j=1,2$ espacios topológicos. Entonces \begin{enumerate}

\item Si $X_1, X_2$ son conexos, entonces $X_1×X_2$ también lo es.
\item Si $A_j$ es conexo en $X_j$ entonces $A_1×A_2$ también lo es en $X_1×X_2$.
\end{enumerate}
\end{prop}

Para la demostración necesitaremos antes el siguiente lema:

\begin{lemma} Si $C⊆X_1×X_2$ es abierto, entonces $∀x_1∈X_1, ∀x_2∈X_2$ se tiene que los conjuntos \begin{gather*}
C_{x_1} = \{y_2 \tq (x_1, y_2) ∈ C\} \\
C_{x_2} = \{ y_1 \tq (y_1, x_2) ∈ C\}
\end{gather*} son abiertos en $X_2$ y $X_1$ respectivamente.
\end{lemma}

\begin{proof}[Lema] Vamos a hacer la demostración en dos casos.

El primero es con $C=V_1×V_2$ con $V_j$ abierto en $X_j$. Entonces, $C_{x_1}$ es o bien el vacío si $x_1 \notin V_1$, o bien $V_2$ si $x_1∈V_1$, en ambas instancias es abierto.

Toca ahora el caso más general. Sea \[ C = \bigcup_{j∈J} V_1^j × V_2^j \] con $V_i^j$ abierto en $X_i$. Entonces \[ C_{x_1} = \left(\bigcup_{j∈J} V_1^j × V_2^j\right)_{x_1} = \bigcup_{j∈J} \left(V_1^j × V_2^j\right)_{x_1} = \bigcup_{j∈J} V_2^j ∈ \topl_2\], por lo que es abierto.
\end{proof}

Ahora vamos a demostrar la proposición que nos interesa.

\begin{figure}[hbtp]
\centering
\inputtikz{II_EspacioProductoConexo}
\caption{Posibilidades para el espacio producto.}
\label{figEspacioProductoConexo}
\end{figure}

\begin{proof}[Proposición]
Si $X_1×X_2$ no es conexo, entonces $∃A,B$ abiertos de $\topl_1\otimes\topl_2$ no vacíos, $A∩B=\emptyset$, $A∪B=X_1×X_2$. ¿Qué casos pueden suceder? Lo vemos en la figura \ref{figEspacioProductoConexo}.

Entonces, en el primer caso, tenemos que $∃x_2∈X_2$ tal que $A_{x_2}≠\emptyset≠B_{x_2}$. Entonces \[ X_1 = (X_1×X_2)_{x_2} = (A∪B)_{x_2} = A_{x_2} ∪ B_{x_2} \], ambos abiertos y no vacíos. Además, $A_{x_2}∩B_{x_2} = \emptyset$, lo que nos daría una contradicción con el hecho de que $X_1$ es conexo.

En el segundo caso, tenemos que $∀x_2∈X_2$ $A_{x_2} = \emptyset$ ó $B_{x_2} = \emptyset$. Sea \[ V_2 = \{ x_2 ∈ X_2 \tq A_{x_2} ≠ \emptyset\}\]. Afirmo que $A= X_1×V_2$ y que $B=X_1 × V_2^c$ y que $X_2 = V_2 ∪ V_2^c$, abiertos ($V_2 = p_2(A), V_2^c=p_2(B)$ y $p_2$ es abierta), disjuntos y no vacíos, contradicción con que $X_2$ es conexo.

\end{proof}

Lo relevante de esta demostración es darse cuenta de que $\topl_{A_1}^{sub}\otimes\topl_{A_2}^{sub} = \topl_{A_1×A_2}^{sub}$.

Por ejemplo, podemos ver que la superficie de un cilindro es conexo. Es producto de una circunferencia en $ℝ^2$, conexa (imagen continua de un intervalo) y un intervalo en $ℝ$, luego es conexo.

\subsection{Componentes conexas}

\begin{defn} \label{defRelacConexion} Sea $(X, \topl)$ un espacio topológico. Definimos la siguiente relación de equivalencia $\rel$ para $x,y∈X$ :

\[ x\rel y ≝ ∃W⊆X \text{ conexo } \tq x,y ∈ W \]
\end{defn}

\begin{prop} $\rel$ es una relación de equivalencia.
\end{prop}

\begin{proof} Demostremos las propiedades:

\begin{enumerate}
\item $x\rel x$, obvio si tomamos $W = \{ x \}$.
\item $x\rel y \implies y \rel x$, también obvio ya que en la definición no hay ningún orden.
\item $x\rel y \y y \rel z \implies z \rel x$. Aquí hay que hacer un poco más pero tampoco mucho. $x,y∈W_1$ y $y,z∈W_2$, ambos conexos. La intersección $W_1∩ W_2$ no es vacía (está $y$ al menos), así que usamos la proposición \ref{propUnionConexa} y entonces la unión es conexa.
\end{enumerate}
\end{proof}

\begin{defn}[Componente\IS conexa] Las clases de equivalencia de $\rel$ son las componentes conexas de $X$.\end{defn}

\begin{prop} La definición de componente conexa a partir de una relación de equivalencia nos permite sacar varias propiedades:


\begin{enumerate}
\item Las componentes conexas forman una partición de $X$.
\item Las componentes conexas son los conexos ``más grandes'' de $X$, es decir:
\begin{enumerate}
\item Una componente conexa es conexo.
\item Si $A$ es conexo existe una componente conexa $C$ tal que $A⊆C$.
\end{enumerate}
\item Las componentes conexas son cerrados en $X$.
\end{enumerate}
\end{prop}

\begin{proof}
\begin{enumerate}
\item $\rel$ es una relación de equivalencia.
\item Sea $C$ componente conexa y $x_0∈C$. Entonces $∀y∈C$, $y≠x_0$, $x_0\rel y$, por tanto $∃W_y$ conexo con $x_0, y ∈ W_y$.

De aquí sacamos que $x_0\rel z$, $\quad ∀z∈W_y$, ya que $x_0, z∈W_y$. Es decir, $W_y ⊆ C$. Queda claro entonces que \[ C = \bigcup_{y∈C\setminus\{x_0\}}W_y \]

Por último, sea $W_{x_0} ≝ \{ x_0\}$, conexo. Es un conexo fijo y una familia de conexos tal que la intersección de cada uno de ellos con el fijo es distinto del vacío ($x_0 ∈ W_{x_0} ∩ W_y \;∀y$), así que podemos usar la proposición \ref{propUnionConexa} y entonces tenemos que \[ C = \left(\bigcup_{y ∈ C\setminus \{x_0\}} W_y \right) ∪ W_{x_0} \] es conexo.

\item Si $C$ es componente conexa, entonces es conexo (lo acabamos de ver). En ese caso, $\adh{C}$ es conexo (ver la proposición \ref{propUnionConexa}), y entonces está contenido en una componente conexa. Y como las componentes conexas son disjuntas y $C∩\adh{C} ≠ \emptyset$, tiene que ser $C=\adh{C}$, luego $C$ es cerrado.

\end{enumerate}
\end{proof}

\begin{prop} Si $A$ es abierto y cerrado en $X$ entonces $A$ es unión de componentes conexas de $X$.
\end{prop}

\begin{proof}
Basta comprobar que si $C$ es una componente conexa y $C∩A≠\emptyset$, entonces $C⊆A$. Podemos escribir \[ C= (C∩A) ∪ (C∩A^c)\], unión de conexos disjuntos. Como $C$ es conexo, uno de los dos tiene que ser vacío. Como $C∩A≠\emptyset$, entonces $C∩A^c=\emptyset$, luego $C⊆A$.
\end{proof}

Vamos a ver varios ejemplos. Por ejemplo, en $(ℚ, \topl_{usual})$, ¿cuáles son las componentes conexas? Recordemos que $W⊆ℚ$ es conexo si y sólo si $(W, \topl_{ℚ}^{sub})$ es conexo, lo que ocurre si y sólo si $W⊆ℝ$ es conexo, es decir, si $W$ es un intervalo. Luego es obvio que $W$ tiene que ser un único punto: las componentes conexas de $ℚ$ son $\{q\}$, con $q∈ℚ$.

Otro ejemplo: tomemos el anillo $A=\{ (x,y) ∈ ℝ \tq 1 < x^2 + y^2 < 4\}$, con la topología del orden lexicográfico. ¿Cómo son las componentes conexas?

En $\topl_{Lex}$, los conexos eran los intervalos en rectas verticales, luego será obvio que en el anillo serán los intervalos abiertos contenidos en él.

\begin{prop} Sean $(X, \topl_X), (Y, \topl_Y)$ espacios topológicos. Si $X$ e $Y$ son homeomorfos, entonces tienen el mismo número de componentes conexas.
\end{prop}

\begin{proof} (Idea). Si $\appl{f}{X}{Y}$ es un homeomorfismo, entonces la imagen de una componente conexa de $X$ es una componente conexa de $Y$.
\end{proof}

Contamos esta proposición para el siguiente ejemplo típico. Cojo un \textit{ocho} y un \textit{cero} dibujados en el plano (ver figura \ref{figOchoCero}). No son homeomorfos.

\begin{figure}[hbtp]
\centering
\inputtikz{II_OchoCero}
\caption{Un ocho y un cero.}
\label{figOchoCero}
\end{figure}

Por ejemplo, podemos quitar dos puntos del ocho sin que deje de ser conexo, pero no podemos hacer lo mismo en el cero: da igual qué dos puntos cojamos que dejará de ser conexo.

También podemos hacerlo por componentes conexas: podemos quitar tres puntos en el ocho (los rojos) y nos quedarían dos componentes conexas. Sin embargo, al hacer lo mismo en el cero nos quedarían tres componentes conexas.

\subsection{Conexión por caminos}

\begin{defn}[Conexión\IS por caminos] Se dice que un espacio topológico $(X, \topl)$ es conexo por caminos (cpc) si y sólo si $\,∀x,y∈X$ existe una aplicación $\appl{φ}{[a,b]}{X}$ continua tal que $φ(a) = x$, $φ(b) = y$.

En el caso de un subespacio, $W⊆X$, $W$ es cpc si y sólo si $(W,\topl^{sub})$ es cpc. En otras palabras, si y sólo si $\,∀x,y∈W$ existe una aplicación continua $\appl{φ}{[a,b]}{X}$ tal que $φ(a) = x$, $φ(b)=y$ y con \[ \mop{Traza}(φ) ≝ \{ φ(t) \tq t ∈ [a,b]\} ⊆ X \].
\end{defn}

\begin{prop} Si un espacio o subespacio es conexo por caminos, entonces es conexo.
\end{prop}

\begin{proof}
Podemos hacerlo de varias formas. Como $\mop{Traza}(φ)$ es conexo (es imagen continua de un intervalo), entonces $x\rel y$ con la relación definida en \ref{defRelacConexion} para componentes conexas para cualquier par $x,y∈ X$. Entonces $X$ tiene una única componente conexa y por lo tanto es conexo.

De otra manera: tomo $x_0∈X$ fijo. Entonces, $∀y∈X\;∃φ_y$ que conecta $x_0$ con $y$. Luego podemos escribir \[ X = \{x_0\} ∪ \left(\bigcup_{y∈X} \mop{Traza}(φ_y)\right)\], que es conexo según la proposición \ref{propUnionConexa}.
\end{proof}

Veamos algunos ejemplos. Cojamos $W⊆ℝ^m$. Si $W$ es convexo, entonces es cpc. Recordemos que $W$ es conexo si \[ ∀x,y∈W,\;[x,y]⊆W\], luego es trivial encontrar la aplicación $φ$ continua para cada par $x,y$ que queramos.

Una consecuencia muy sencilla es que las bolas euclídeas $\bola(x,r)$ son cpc.

Otro ejemplo son los conjuntos estrellados \ref{defEstrellado}. Si $W⊆ℝ^m$ es estrellado, entonces es cpc. Recordemos que si $W$ es estrellado, entonces $∃x_0∈W$ tal que $∀y∈W$ el intervalo $[x_0, y] ⊆ W$. En este caso también es muy sencillo encontrar la aplicación φ que necesitamos:

\begin{gather*}
\appl{φ}{[0,2]}{W} \\
φ(t) = \begin{cases} tx_0 + (1-t)x \\
(t-1)y + (2-t)x_0
\end{cases}\end{gather*}

Obviamente tenemos que $φ(0) = x$, $φ(1) = y$ y además φ es continua.

El peine topológico que habíamos visto antes (\ref{figPeine}) también es conexo por caminos, y podemos encontrar fácilmente la aplicación que buscamos. Eso sí, si quitásemos un punto de una de las barras verticales dejaría de ser cpc, pero seguiría siendo conexo.\footnote{
Salvo que quites un punto de la forma $(x, 1)$, la barra $\{x\} \times [0, 1]$ queda partida en dos trozos, luego $P$ ya no sería cpc. Pero los puntos de la forma $(x, 1)$ pertenecen a su adherencia, y mientras que los puntos que quites sean de la adherencia, el conjunto sigue siendo conexo.}

\begin{prop} Varias propiedades interesantes análogas a las que habíamos visto con conexión.

\begin{enumerate}
	\item La imagen continua de un cpc es cpc.
	\item Si $D_0$ es cpc, $D_j$ es cpc $∀j∈J$ y $D_0∩D_j≠∅\;∀j∈J$, entonces $D_0∪\bigcup_{j∈J} D_j$ es cpc. Como caso particular, la unión de dos $A,B$ cpc no disjuntos es cpc igualmente.
	\item Sean $(X_1,\topl_1),(X_2,\topl_2)$ cpc. Entonces $(X_1×X_2, \topl_1\otimes\topl_2)$ es cpc.
\end{enumerate}
\end{prop}

\begin{proof}
\begin{enumerate}
	\item Tomamos $(X_1,\topl_1),(X_2,\topl_2)$ espacios topológicos. Por hipótesis, $W⊆X_1$ es cpc y tenemos $\appl{f}{X_1}{X_2}$ continua. Entonces tenemos que demostrar que $f(W)$ es cpc.

	Si tomamos $x_2, y_2 ∈ f(W)$, entonces $∃x_1, y_1 ∈ W$ con $f(x_1) = x_2, f(y_1) = y_1$. Como $W$ es cpc, entonces $∃\appl{φ_1}{[a,b]}{X_1}$ continua con $φ_1([a,b])⊆W,\,φ_1(a) = x_1,\,φ_1(b) = x_2$. Entonces $φ_2 = f ○ φ_1$ es continua, $φ_2(a) = x_2, \, φ_2(b) = y_2$ y $φ_2([a,b])⊆f(W)$.

	\item Tenemos $x,y∈D_0∪\bigcup_{j∈J}D_j$. $∃j_x$ (posiblemente 0) tal que $x∈D_{j_x}$. De la misma forma, tenemos un $j_y$ tal que $y∈D_{j_y}$.

	Sabemos también que como $D_{j_x}∩D_0≠∅$ por hipótesis, entonces $∃x_0∈D_0∩D_{j_x}$, y análogamente existe un $y_0$.

	Como $x_0,x∈D_{j_x}$ y es cpc, entonces existe la aplicación de siempre $\appl{φ_x}{x}{x_0}$\footnote{Abuso de notación pero no me ha dado tiempo a copiarlo todo bien.}, y lo mismo con otra aplicación $φ_y$. Y como además $x_0, y_0∈D_0$ cpc, tenemos la aplicación $\appl{φ_0}{x_0}{y_0}$ y entonces podemos ``unirlas''
	\footnote{Más formalmente: dadas $\appl{φ_1}{[a,b]}{Φ_1}$, $\appl{φ_2}{[b,c]}{Φ_2}$, entonces la ``unión'' es $\appl{φ_1\astφ_2}{[a,b+d-c]}{Φ_1∪Φ_2}$ con \[ φ_1\astφ_2(t) = \begin{cases} φ_1(t) & a≤t≤b \\ φ_2(t + c-b) & b ≤ t ≤ b+d-c\end{cases} \].}
	y ya tenemos la aplicación continua que necesitamos.

	\item Si $X_1, X_2$ son cpc, entonces $X_1×X_2$ también lo es. Sean $\vx=(x_1, x_2), \vy=(y_1, y_w) ∈ X_1×X_2$. Entonces existen sendas aplicaciones $φ_1, φ_2$ que llevan el intervalo $[0,1]$ a $X_1$ y $X_2$ respectivamente que nos dan el camino de conexión. Luego la aplicación que necesitamos en el espacio producto es \begin{align*} \appl{φ}{[0,1]&}{X_1×X_2} \\ t & \longmapsto (φ_1(t), φ_2(t)) \end{align*} que es continua y demás cosas que necesitamos.
\end{enumerate}
\end{proof}

\begin{remark}
Si $X_1×X_2$ es cpc, entonces $X_1$, $X_2$ son cpc.
\end{remark}

\begin{figure}[hbtp]
\centering
\inputtikz{II_SenoTopologo}
\caption{La llamada curva seno del topólogo.}
\label{figSenoTopologo}
\end{figure}

Vamos a ver otro ejemplo interesante: el seno del topólogo (figura \ref{figSenoTopologo}), dada por el conjunto \[ S = \underbrace{\left\{ \left(x, \sin \frac{1}{x}\right) \tq 0 < x ≤ 1\right\}}_{S_b} ∪ \{(0,0)\} \]

Está claro que $S_b$ es conexo por ser imagen continua del conexo $(0,1)$. Además, $S_b⊆S⊆\adh{S_b}$, así que todo $S$ es conexo. Sin embargo, no es conexo por caminos, y eso es más delicado. Vamos a demostrarlo diciendo que no existe una aplicación $\appl{φ}{[0,1]}{S}$ continua tal que $φ(0) = (1,\sin 1)$, $φ(1) = (0,0)$, y lo vamos a hacer por reducción al absurdo.

Sea $I = \left\{ s ∈ [0,1] \tq φ_1(t) > 0 \,∀t≤s \right\}$. No es vacío (como $φ_1(0) = 1, 0∈I$) y está acotado superiormente por $1$. Sea $a=\sup I$. Entonces afirmamos que $φ_1(a) = 0$ y $φ_1(t) > 0\, ∀t<a$, y por tanto $φ_2(t) = \sin\left(\frac{1}{φ_1(t)}\right)$. ¿Por qué?

Bien, si tenemos que $φ_1(a)  > 0$, entonces $a<1$, y por continuidad $∃δ>0$ tal que $a+δ<1$, $a-δ>0$ y $φ_1|_{(a-δ, a+δ)} > 0$, luego como $s∈I > 0$ entonces $a+\frac{δ}{2} ∈ I$, contradicción porque habíamos dicho que $a$ era el supremo.

Por otro lado, si $t<a$, entonces $∃s∈I$ luego $φ_1(t) > 0$.

Si hacemos que $t$ converja a $a$ por la izquierda, entonces $φ_1(t) \diagdown φ_1(a) = 0$. Pero si $t<a$, entonces $φ(t) = \left(φ_1(t), \sin\left(\frac{1}{φ_1(t)}\right)\right)$ oscila entre $-1$ y $1$, luego el límite $\lim_{t\to a^-}φ(t)$ no existe.\footnote{Aquí no sé qué puñetas he escrito.}

\subsubsection{Componentes conexas por caminos}

Vamos a definirlas de forma análoga a como lo hicimos con las componentes conexas, primero con una relación de equivalencia.

\begin{defn} Sea $(X, \topl)$ un espacio topológico. Entonces \[ x\rel y ≝ ∃\appl{φ}{[0,1]}{X}\] continua y tal que $φ(0) = x$, $φ(1) = y$.
\end{defn}

Esa $\rel$ que acabamos de definir es relación de equivalencia, y no hace falta demostrarlo porque es muy sencillo.

\begin{defn}[Componente\IS conexa por caminos] Las clases de equivalencia de $\rel$ son las componentes conexas por caminos de $X$.\end{defn}

\begin{prop} Sea $(X, \topl)$ un espacio topológico. Entonces \begin{enumerate}
	\item Las componentes conexas por caminos forman una partición de $X$.
	\item Las componentes conexas cpc son los subconjuntos cpc más grandes de $X$, es decir \begin{itemize}
		\item Una componente cpc es cpc.
		\item Si $A$ es cpc, existe $C$ componente cpc tal que $A⊆C$.
	\end{itemize}
\end{enumerate}
\end{prop}

\begin{proof}
\begin{enumerate}
	\item Es una relación de equivalencia, trivial.
	\item Vamos a demostrar que si $A$ es conexa por caminos tiene que ser por fuerza un subconjunto de una clase de equivalencia $C$. Sea $C$ una componente cpc, luego si $x,y∈C$ entonces $x\rel y$ y existe una φ continua que conecta $x,y$. Necesitamos, eso sí, que todos los valores de φ estén en $C$.

	Afirmamos que $φ_1(a) ∈ C\;∀a∈[0,1]$, y por tanto $C$ es cpc. Lo justificamos construyendo la aplicación $\appl{φ_a}{[0,1]}{X}$ tal que $φ_a(t) = φ(ta)$, que es continua. De esta forma, $φ_a(0) = φ(0) = x$, y $φ_a(1) = φ(a)$, luego $x\rel φ(a)$, y como $C$ es clase de equivalencia entonces $φ(a) ∈ C$.

	Ahora ya vamos a $A$: si $A$ es cpc, entonces $∀x,y∈A$ existe una aplicación $\appl{φ}{[0,1]}{X}$ continua que conecta $x$ e $y$, luego cualquier par de puntos en $A$ están en la misma clase de equivalencia, que será $C$.
\end{enumerate}
\end{proof}

\begin{remark} Las componentes conexas son cerrados, pero las componentes cpc no tienen por qué serlo. \end{remark}

\begin{prop} Sean $(X, \topl), (Y, \topl')$ espacios topológicos y $\appl{f}{X}{Y}$ un homeomorfismo. Entonces \begin{enumerate}
	\item La imagen por $f$ de una componente conexa de $X$ es una componente conexa de $Y$.
	\item La imagen por $f$ de una componente cpc de $X$ es una componente cpc de $Y$.
\end{enumerate}
\end{prop}

\begin{proof}
\begin{enumerate}
	\item Sea $C$ componente conexa de $X$. Entonces $f(C)$ es conexo en $Y$. Si no fuese componente conexa de $Y$, existiría un conexo $W$ en $Y$ conexo tal que $f(C) ⊂ W$. En ese caso, como $\inf{f}$ es continua, $\inv{f}(W)$ es conexo en $X$ y entonces al ser $f$ biyectiva tendríamos que $\inv{f}(W) ⊂ C$, contradicción.
	\item Demostración análoga a lo anterior.
\end{enumerate}
\end{proof}

\begin{corol} Dos espacios homeomorfos tienen el mismo número de componentes conexas y el mismo número de componentes cpc (los dos números no tienen por que ser iguales, podemos tener dos componentes conexas y una cpc).

No tiene por qué cumplirse que dos espacios con el mismo número de componentes conexas y cpc sean homeomorfos.
\end{corol}

\begin{prop} Sea $A$ un abierto de $ℝ^m$. Entonces

\begin{enumerate}
	\item Tanto las componentes conexas de $A$ como las componentes cpc de $A$ son abiertos de $ℝ^m$.
	\item $A$ es conexo si y sólo si $A$ es cpc.
\end{enumerate}
\end{prop}

En la demostración vamos a ver cuál es la propiedad especial de $ℝ^m$ que hace que esto ocurra.

\begin{proof}
\begin{enumerate}
\item Sea $C$ una componente conexa de $A$. ¿Es $C$ abierto? La respuesta es sí, y vamos a justificarlo.

Para ver que es abierto, tenemos que ver que si $x∈C$, entonces $∃ε>0$ tal que $\bola(x,ε)⊆C$. Sabemos que esas bolas existen en $A$ ya que es abierto: $∃ε \tq \bola(x,ε) ⊆ A$.

Además, $\bola(x,ε)$ es conexa y $x∈C∩\bola(x,ε) ≠ ∅$. Como las componentes conexas son los conexos más grandes en $A$ y son disjuntas entre ellas, tiene que ser $\bola(x,ε) ⊆ C$.

La demostración es igual para cpc pues $\bola(x,ε)$ es cpc.

\item Ya hemos demostrado que si $A$ es cpc, entonces es conexo. Hay que demostrarlo al revés: que si $A$ es conexo entonces es cpc.

Sea $x_0 ∈ A$ fijo. Basta demostrar que $∀y∈A\;∃\appl{φ}{[0,1]}{A}$ continua que conecta $x_0$ con $y$. Es decir, sólo tenemos que ser capaces de conectar un punto fijo con cualquier otro.

Sea $B$ el conjunto de puntos de $A$ tales que se pueden conectar con $x_0$. Basta demostrar que
\begin{itemize}
	\item $B$ es abierto en $A$.
	\item $B$ es cerrado en $A$:
	\item $B≠∅$.
\end{itemize}

¿Por qué? Las dos primeras propiedades nos indican que o bien $B$ es el vacío o el total, los dos únicos subconjuntos abiertos y cerrados en un conjunto conexo. Y si luego decimos que no es el vacío, pues es el total.

Este argumento es muy genérico, y nos servirá para demostrar que todos los puntos de un conjunto conexo tienen una cierta propiedad cualquiera.

Vamos a ello. Lo primero es lo último: $B≠∅$ ya que $x_0 ∈ B$ obviamente.

Siguiente paso, ¿por qué es abierto? Sea $y∈B$, como $y∈A$ y $A$ es abierto entonces $∃ε>0\tq \bola(y,ε)⊆A$.

Entonces $∀z∈\bola(y,ε)$, $z$ se puede conectar con $x_0$ pasando por $y$, luego $z∈B\; ∀z∈\bola(y,ε)$. Es decir, $\bola(y,ε) ⊆ B$ y $B$ es abierto en $ℝ^m$.

Ahora vamos a demostrar que también es cerrado en $A$, es decir, $\adh{B} ∩ A = B$. Sea $z∈\adh{B}∩A$, entonces como $A$ es abierto $∃δ>0$ tal que $\bola(x,δ) ⊆ A$. Como $z∈ \adh{B}$, entonces $\bola(x,δ)∩B ≠ ∅$, y podemos encontrar un $y∈\bola(x,δ) ∩ B$, luego $z$ se conecta con $x_0$ a través de $y$, y entonces $∀z∈\adh{B}∩A\; z∈B$.

\end{enumerate}
\end{proof}

\begin{figure}[hbtp]
\centering
\inputtikz{II_ConexoCPC}
\caption{Figura para la demostración de que conexo implica cpc. Cualquier punto $z_c$ en la adherencia está en $B$ (lo conectamos con $x_0$ a través de un $y_c$ en la $\bola(z_c,δ)$, cuya intersección con $B$ es no vacía por definición), luego $B$ es cerrado. Para cualquier punto $y_a∈B$ podemos encontrar una bola contenida en $B$, con cualquier punto $z_a$ en esa bola en $B$ ya que podemos conectarlo con $x_0$ a través de $y_a$, luego $A$ también es abierto.}
\end{figure}

Ahora vamos a ver lo que decíamos antes: ¿cuál es la propiedad especial de $ℝ^m$ que nos permite alcanzar esta conclusión?

Lo único que se ha usado es lo siguiente: que $∀A$ abierto y $∀x∈A$ existe un $V$ entorno de $x$ conexo (cpc respectivamente) tal que $x∈V⊆A$.

Los espacios topológicos con esta propiedad se les llama \concept[Espacio!localmente conexo]{localmente conexos} (o respectivamente \textbf{localmente cpc}\index{Espacio!localmente cpc}).

\begin{corol} Sea $A$ abierto en $ℝ^m$.

\begin{enumerate}
	\item Las componentes conexas de $A$ coinciden con las componentes cpc de $A$.
	\item $A$ es unión de intervalos abiertos disjuntos (las componentes conexas).
\end{enumerate}
\end{corol}

\section{Compacidad}

\begin{defn}[Recubrimiento] Sea $(X, \topl)$ un espacio topológico. Un recubrimiento de $X$ es una familia $\{D_i\}_{i∈I}$ de subconjuntos de $X$ tal que $X ⊆ \bigcup_{i∈I} D_i$.
\end{defn}

\begin{defn}[Recubrimiento\IS abierto] Un recubrimiento abierto de $X$ es un recubrimiento en el que todos los $D_i$ son abiertos.
\end{defn}

\begin{defn}[Subrecubrimiento] Un subrecubrimiento de un recubrimiento $\{D_i\}_{i∈I}$ de $X$ es una subfamilia $\{D_j\}_{j∈J}$ con $J⊆I$.
\end{defn}

\begin{defn}[Compacidad] Sea $(X, \topl)$ un espacio topológico. \begin{enumerate}
	\item $(X,\topl)$ es un compacto si para todo recubrimiento abierto de $X$ se puede extraer un subrecubrimiento finito.
	\item Un subconjunto $W⊆X$ es compacto y si sólo si $(W, \topl^{sub})$ es compacto.
\end{enumerate}
\end{defn}

\begin{prop} Sea \tops un espacio topológico, $W⊆X$. Entonces $W$ es compacto si y sólo si de todo recubrimiento de $W$ por abiertos de $X$ se puede extraer un subrecubrimiento finito.
\end{prop}

\paragraph{Ejemplos de compactos}

Empezamos con uno muy sencillo: $(0,1)$ con la topología usual no es compacto. Es un muy fácil encontrar la familia de intervalos $A_n = \left(\frac{1}{n}, 1\right)$, tal que $\{A_n\}_{n=0}^{∞}$ recubrimiento abierto por infinitos conjuntos. Pero si tomamos una unión finita \[ \bigcup_{j=1}^k A_{n_j} = \left(\frac{1}{m},1\right)\] con $m = \max n_j$ no puede recubrir todo el conjunto. Es decir, no existe un SRF\footnote{\textbf{S}ub\textbf{r}ecubrimiento \textbf{f}inito.}.

Un ejemplo que sí nos va a servir bastante: cualquier sucesión convergente junto con su límite es compacto. Es decir, si tenemos $x_n\convs x$ en \tops, entonces el conjunto \[ W = \{x\} ∪ \{ x_n \tq n = 1,2,\dotsc \}\]

¿Por qué? Tomemos un recubrimiento abierto de $W$ $\{A_j\}_{j∈J}$. Si el límite $x∈W$, entonces $∃j_0 \tq x ∈ A_{j_0}$. Por ser $A_{j_0}$ abierto, entonces $∃n_0$ tal que $x_n ∈ A_{j_0}\;∀n≥n_0$.

Por otro lado, si $1≤m<n_0$, entonces $∃j_m∈J$ tal que $x_m∈A_{j_m}$ y entonces $A_{j_0}, A_{j_1}, \dotsc, A_{j_{n_0-1}}$ es un SRF.

Otro ejemplo más: si $W$ es finito entonces es compacto.

Lo malo es que usar la definición es un poco pesado para demostrar la compacidad de conjuntos, así que vamos a tratar de facilitarnos la vida.

\begin{prop} Sea \tops un espacio topológico, entonces
\begin{enumerate}
	\item Si $X$ es compacto y $W⊆X$ es cerrado, entonces $W$ es compacto.
	\item Si $X$ es Hausdorff y $W⊆X$ compacto, entonces $W$ es cerrado.
	\item Si tenemos \stdf continua y $W⊆X$ compacto, entonces $f(W)$ es compacto.
	\item Si \stdf es continua, biyectiva, $X$ compacto e $Y$ Hausdorff, entonces $f$ es homemorfismo.
	\item Dado $W_j⊆X$ compacto para $j=1,2,\dotsc, n$, entonces $W_1 ∪ \dotsb W_n$ es compacto.
	\item Sean $X_1, X_2$ compactos. Entonces $X_1×X_2$ es compacto. También vale para subconjuntos ($W_j ⊆ X_j$ compacto, entonces $W_1 × W_2$ es compacto en $X_1×X_2$).

	Más generalmente, dados $X_1,X_2,\dotsc, X_n$ compactos entonces $X_1 × \dotsb × X_n$ es compacto.
\end{enumerate}
\end{prop}

\begin{proof}
\begin{enumerate}
	\item Tenemos que demostrar que si $X$ es compacto y $W⊆X$ es cerrado, entonces $W$ es compacto. Sea $\{A_i\}_{i∈I}$ un recubrimiento abierto de $W$, es decir, $W⊆\bigcup_{i∈I}A_i$.

	Como $W^c$ es abierto, entonces \[ \{A_i\}_{i∈I} ∪ \{W^c\}\] es un recubrimiento abierto de $X$. Luego por ser $X$ compacto, existe un SRF: $X⊆A_{i_1} ∪ \dotsb ∪ A_{i_m} ∪ W^c$, que podemos suponer que está (y si no está se añade que no pasa nada, el recubrimiento sigue siendo finito).

	Y aquí ya estamos cerca: como $W∩W^c = ∅$, entonces $W ⊆ A_{i_1} ∪ \dotsb ∪ A_{i_m}$.
	\item Tenemos $W$ compacto, $W⊆X$ y $X$ Hausdorff. Queremos demostrar que entonces $W$ es cerrado. Una primera observación: si $W$ es cerrado, $\adh{W} ⊆ W$ y entonces $W^c ⊆ \adh{W}^c$, o dicho de otra forma, $x∉W \implies x∉\adh{W}$. Esta última afirmación es la más fácil de demostrar y es a por la que vamos a ir.

	Sea $x∉W$. Buscamos un entorno que no corte a $W$. Si cogemos un $y∈W$. Por ser Hausdorff podríamos coger entornos que no se corten, pero no sabemos si el entorno de $x$ cortaría a $W$, y no queremos eso. Tenemos que usar el hecho de que $W$ es compacto.

	Para todo $y∈W$ distinto de $x$, $∃ U_y, V_y$ abiertos tales que $x∈U_y, y∈V_y$. Claramente, $W⊆V_y$. Como $W$ es compacto, $∃y_1, y_2, \dotsc, y_n$ tal que $W⊆V_{y_1} ∪ \dotsb ∪ W_{y_m}$, un SRF. Sea $U = U_{y_1} ∩ U_{y_w} ∩ \dotsb ∩ U_{y_m}$, abierto por ser intersección  de un número finito de abiertos. Como $x∈U$, entonces \[ U ∩ W = U ∩ (V_{y_1} ∪ \dotsb ∪ V_{y_m}) = (U∩V_{y_1}) ∪ \dotsb ∪ (U∩V_{y_m}) = ∅ \] ya que la intersección por pares es vacío. Luego $x∉\adh{W}$ y ya está lo que queríamos demostrar\footnote{Creo.}.

	\item Tenemos \stdf continua y $W⊆X$ compacto. ¿Es $f(W)$ compacto?

	Sea $\{A_i\}_{i∈I}$ recubrimiento por abiertos de $f(W)$. Sea $B_i = \inv{f}(A_i)$. Es claor que $B_i$ es abierto en $X$ por ser $f$ continua, y como $f(W)⊆ \bigcup_{i∈I} A_i$ entonces $W⊆\inv{f}(f(W))$ y \[ \bigcup_{i∈I} = \bigcup_{i∈I} \inv{f}(A_i) = \inv{f}\left(\bigcup_{i∈I}A_i\right)\], luego $\{B_i\}_{i∈I}$ es un recubrimiento por abiertos de $W$.

	Por ser $W$ compacto, $∃i_1, i_2, \dotsc, i_m ∈ I$ tales que $W⊆B_{i_1} ∪ \dotsb ∪ B_{i_m}$, luego $f(W) ⊆ f(B_{i_1} ∪ \dotsb ∪ B_{i_m}) = f(B_{i_1}) ∪ \dotsb ∪ f(B_{i_m}) ⊆ A_{i_1} ∪ \dotsb ∪ A_{i_m}$, y ya tenemos el SRF que necesitábamos.
	\item Tenemos \stdf continua y biyectiva, $X$ compacto, $Y$ Hausdorff, y queremos demostrar que $f$ es un homeomorfismo.

	Sabemos que $\inv{f}$ es continua si y sólo si la imagen inversa por $\inv{f}$ de un cerrado es cerrado.

	Sea $W$ cerrado en $X$, entonces por la propiedad 1, $W$ es compacto, por la 3 $f(W)$ es compacto, y usando la propiedad 2, entonces $f(W)$ es cerrado en $Y$.
	\item Si $\{A_i\}_{i∈I}$ es un recubrimiento abierto de $W_1 ∪ \dotsb W_m$, entonces es claro que $W_k ⊆ W_1 ∪ \dotsb W_m ⊆ \bigcup_{i∈I} A$. Por ser $W_k$ compacto, $∃J_k ⊆ I$ finito tal que $W_k ⊆ \bigcup_{k∈J_k} A_j$.

	Sea $J = J_1 ∪ \dotsb ∪ J_k$. Entonces $\{A_j\}_{j∈J}$ es un SRF, ya que \[ \bigcup_{j∈J} A_j = \underbrace{\left(\bigcup_{j∈J_1} A_j\right)}_{⊆W_1} ∪ \dotsc ∪ \underbrace{\left(\bigcup_{j∈J_n} A_j\right)}_{⊆W_n} \] y entonces la unión $W_1 ∪ \dotsc W_m ⊆ \bigcup_{j∈J} A_j$, con $J$ finito, y ya lo tenemos.
	\item Tenemos $X_1, X_2$ compactos y queremos demostrar que $X_1×X_2$ es compacto. Sea $\{A_i\}_{i∈I}$ un recubrimiento por abiertos de $X_1×X_2$. 

	La primera observación que vamos a hacer es que cada $A_i$ es unión de conjuntos de la base $V_1×V_2$, abiertos en $X_1$ y $X_2$ respectivamente, así que pasamos al recubrimiento abierto formado por esos elementos de la base: $\{U_j×V_j\}_{j∈J}$

	Entonces, si encontramos un SRF de ese segundo recubrimiento entonces obtenemos un SRF del original, esto es, los $A_i$ de los que proceden los elementos de la base

	Para cada $x∈X_1$, $\{x\}×X_2$ es compacto (es $f(X_2)$ con $f(y) = (x,y)$ continua), y $\{U_j×V_j\}_{j∈J}$ es un recubrimiento abierto de $\{x\}×X_2$. Al ser compacto, entonces existe un SRF $\{U_j^x×V_j^x\}_{j∈J_x}$. 

	Sea $U_x = \bigcap_{j∈J_x} U_j$. Es un entorno de $x$ y \[ U_x×V_2 ⊆ \bigcup_{j∈J_x}U^x_j × V_j^x \] 

	Es decir, que $U_x$ no sólo cubre la línea recta $\{x\}×X_2$ sino que cubre una banda alrededor (ver figura \ref{figBandaCompactoProducto}), así que entonces $\{U_x\}_{x∈X_1}$ es un recubrimiento abierto de $X_1$. Por ser $X_1$ compacto, $∃x_1, x_2,\dotsc, x_l$ tal que \[ X_1⊆ \bigcup_{n=1}^l U_{x_n}\]

	Como consecuencia, hay un número finito de bandas y para cada una de ellas hay un SRF, luego juntándolas hay un SRF para la unión de las bandas, que no es más que $X_1×X_2$.
\end{enumerate}
\end{proof}


\begin{figure}[hbtp]
\centering
\inputtikz{II_BandaCompactoProducto}
\caption{Si los rectángulos naranjas son los elementos de la base que forman el recubrimiento de $\{x\}×X_2$, entonces la banda verde es intersección de todos los $U_{j_x}$ y cubre una banda alrededor de $\{x\} × X_2$.}
\label{figBandaCompactoProducto}
\end{figure}

Vamos a ver ciertas propiedades de $ℝ^m$ y sus compactos.

\begin{prop} \pbreak
\begin{enumerate}
	\item $[a,b]$ es compacto en $ℝ$. Más generalmente, $[a_1, b_2] × \dotsb × [a_m, b_m]$ es compacto en $ℝ^m$.
	\item $K⊆ℝ^m$ es compacto si y sólo si $K$ es cerrado y acotado. 
	\item Si $K⊆ℝ$ es compacto, entonces $K$ tiene un máximo y un mínimo.
\end{enumerate}
\end{prop}

\begin{proof}
\begin{enumerate}
	\item Por reducción al absurdo: si existe $\{A_i\}_{i∈I}$ recubrimiento abierto de $[a,b]$ sin SRF. Entonces, si dividimos $[a,b]$ en dos subintervalos $I_1, I_2$, para uno de esos dos elementos no hay SRF. Podemos seguir repitiendo ese procedimiento para tener una serie $I_0 ⊇ I_1 ⊇ \dotsb ⊇I_n$ tal que no existe recubrimiento finito para $I_n$. 

	Está claro que la longitud de $I_n$ es $\frac{b-a}{2^n}$ que converge a $0$. Si $I_n = [a_n, b_n]$, siendo $a_n$ creciente y $b_n$ creciente, ambos convergen a un cierto $c = \sup a_n = \inf b_n$.

	Ese límite $c$ está en $[a,b]$, que  a su vez está contenido en $\bigcup_{i∈I}A_i$, así que $∃i_c$ tal que $c∈A_{i_c}$. Por ser abierto, entonces $∃δ>0$ tal que $(c-δ, c+δ) ⊆ A_{i_c}$. Entonces $c-δ < a_n < b_n < c+δ$ si $\frac{b-a}{2^{n+10}} < δ$. Es decir que tendríamos que $I_n⊆A_{i_c}$, y habríamos encontrado un subrecubrimiento finito, contradicción.

	\item Si $K$ es compacto, entonces al ser $ℝ^m$ Hausdorff entonces $K$ es cerrado. $K$ está acotado, así que $K⊆\bola(0, M)$ para algún $M$. La razón es que $\{\bola(0,r)\}_{r>0}$ es un recubrimiento abierto de $K$, así que existe un SRF $\{\bola(0, r_1), \dotsc, \bola(0, r_n)\}$y por ser bolas concéntricas entonces están contenidas todas en $\bola(0, M)$ para $M = \max r_j$.

	Hacia el otro lado, si $K$ es acotado entonces $K⊆[a_1, b_1] × [a_m, b_m]$. Y por ser $K$ cerrado, entonces $K$ es compacto.\footnote{Pofale.}
\end{enumerate}
\end{proof}

Y una observación que al profesor se le había olvidado antes: si $\topl=\topl_\base$ es una topología generada por una base entonces para comprobar compacidad basta considerar recubrimientos por abiertos de la base.

Vamos a seguir con ejemplos. El toro de \href{http://i.imgur.com/WLRV4HT.gif}{revolución} (\ref{figToroRevolucion}) es compacto (tanto si tiene lo de dentro como si no). Es cerrado: dada $f(x,y,z) = \left(\sqrt{x^2+y^2} - 2\right)^2 + z^2$ continua, el toro es la imagen inversa de $\{1\}$, cerrado.

Además es acotado (podrías calcularlo, pero no me da la gana seguir copiando).

\begin{figure}[hbtp]
\centering
\includegraphics[width=0.5\textwidth]{img/ToroRevolucion.jpg}
\caption{Un toro de revolución.}
\label{figToroRevolucion}
\end{figure}


Pero además, el toro es homeomorfo al producto de dos esferas $\mathbb{S}^1 × \mathbb{S}^1$.


\appendix
\chapter{Ejercicios}

\section{Hoja 1}


\begin{problem}[6] Sea $\appl{g}{X}{Y}$ una aplicación entre dos conjuntos.

\ppart Demostrar que si $\topl$ es una topología en $X$ entonces \[ \mathcal{S} = \{ E ⊆ Y \tq \inv{g}(E) ∈ \topl \} \] es una topología en $Y$.
\ppart Demostrar que si $\mathcal{S}$ es una topología en $Y$ entonces \[ \mathcal{U} = \{ \inv{g}(E) \tq E ∈ \mathcal{S} \} \]es una topología en $X$.

\solution
\spart Vamos a demostrar que es una topología, para lo cual tenemos que comprobar las 3 propiedades (ver \ref{defTopología}):

\begin{enumerate}
\item $Y\in \mathcal{S} \dimplies g^{-1}(x)\in \mathcal{T}$, por ser $g^{-1}(y)=x$

El razonamiento de porqué $\emptyset \in \mathcal{S}$ es igual.

\item $A,B \in \mathcal{S} \dimplies g^{-1}(A),g^{-1}(B) \in \mathcal{T}$.

$A\cap B \in\mathcal{S} \dimplies g^{-1}(A\cap B) \in \mathcal{T}$.

Hemos llegado a que para demostrar la segunda propiedad, tenemos que demostrar $g^{-1}(A),g^{-1}(B) \in \mathcal{T} \implies g^{-1}(A\cap B) \in \mathcal{T}$.

Para ello: $g^{-1}(a\cap B) = g^{-1}(A)\cap g^{-1}(B)$. No es difícil convencernos de esta igualdad. En caso de tener dudas, demostrar las 2 inclusiones (una en cada sentido). Esto para imágenes directas no funciona.

\item Demostramos ahora que la unión de abiertos está en la topología. Si $A, B ∈ \mathcal{S}$, entonces $\inv{g}(A), \inv{g}(B) ∈ \topl$. Como $\topl$ es topología, tenemos que $\inv{g}(A) ∪ \inv{g}(B) ∈ \topl$, lo que implica de forma obvia que $\inv{g}(A∪B) ∈ \topl$ y por lo tanto $A∪B ∈ \mathcal{S}$.
\end{enumerate}

\spart

\end{problem}

\begin{problem}[9] Se consideran las siguientes familias de conjuntos en $ℝ$:

\begin{gather*}
\base_{\leftarrow} = \{ (-∞, b) \tq b ∈ ℝ \} \\
\base_{\rightarrow} = \{ (a,∞) \tq  ∈ ℝ \} 
\end{gather*}

\ppart Demostrar que cada familia es una base de una topología sobre $ℝ$.
\ppart Comparar esas topologías.
\ppart Demostrar que la topología generada por $\base_{\leftarrow} ∪ \base_{\rightarrow}$ es la usual.
\solution
\spart Si añadimos $\emptyset$ y el total, entonces tenemos una topología generada por $\base_{\rightarrow}$ y otra generada por $\base_{\leftarrow}$.

\spart  

\spart
\end{problem}

\begin{problem}[11]
Sea $\topl_j$, $j∈J$ una familia de topologías sobre $X$. Demostrar que existe una topología que contiene a todas las $\topl_j$, para $j∈J$ y además es la menos fina de todas las que verifican esta propiedad. 
\solution

Aplicamos directamente la proposición \ref{propTopologiaMinima}: la topología que contiene a todas ellas es \[ \topl = \bigcap_{j∈J} \topl_j \]
\end{problem}

\paragraph{Observación útil para el 5 y el 12:}  
\begin{enumerate}
\item $x \in C(x,\varepsilon)$
\item $\varepsilon_1 > \varepsilon_2 \implies C(x,\varepsilon_2) \subset C(x,\varepsilon_1)$
\end{enumerate}

Y podemos aplicar la propiedad:
\[
A\in\topl \dimplies \forall a\in A \exists \varepsilon > 0 \tlq C(x,\varepsilon)\subseteq A
\]

Haciendo caso al enunciado y haciendo el dibujo vemos que se cumplen las propiedades de base.

Esta topología contiene a la usual pero al revés no, porque para el punto de intersección de las diagonales no existe un abierto de la usual que le contenga.


\paragraph{Pistas para espacios métricos}

(16) Si tengo $d$, una distancia no acotada, puedo definir $d'=\frac{d}{1+d}$, que sigue siendo una distancia, parecida y además acotada.

(17) $\sum \frac{1}{2n} \leq 1$. La clave está en aplicar la desigualdad triangular a cada término del sumatorio. La clave para este problema es el 16.
% -*- root: ../TopologiaI.tex -*-
\chapter{Exámenes}

\section{2013/2014}

\subsection{Parcial I}

\begin{problem} Consideramos $ℝ$ con la topología generada por la base $\base = \{[a,b) \tq a < b\}$. Hallar el interior y adherencia del conjunto \[ A = \left\{1 - \frac{1}{n} \tq n ∈ ℕ \right\} ∪ [4,5) ∪ [7,8) ∪ (9,10) \]
\solution

Recuperemos la definición de interior (\ref{defInterior}): $x ∈ \mop{Int}(W) = \intr{W}$ si existe un entorno $U$ de $x$ tal que $U⊆W$. Por otra parte, la adherencia (\ref{defAdherencia}) se define de la siguiente forma \[ x ∈ \adh{W} \iffdef A∩W ≠ \emptyset \; ∀ A ∈ \topl \tq x∈A \]

Vamos a estudiar por partes la adherencia y el interior de cada uno de los subconjuntos disjuntos de $A$: $A_1 = \left\{1 - \frac{1}{n} \tq n ∈ ℕ \right\}$, $A_2 = [4,5)$, $A_3 = [7,8)$ y $A_4 = (9,10)$. Dado que estamos trabajando con una base, nos bastará comprobar las definiciones para los elementos de la base.

\paragraph{Interior} Para $A_1$ lo tenemos fácil: no hay interior. Si $x = 1 - \frac{1}{n}∈\intr{A_1}$, entonces $∃[a,b) ⊆ A_1$ con $a ≤ x < b$. Ahora bien, podemos encontrar un $ε ∈ (0,1)$ suficientemente pequeño tal que $β = 1 - \frac{1}{1+ε} < b$, por lo que $[a, β) ⊆ [a,b)$. Pero tal y como hemos construido el intervalo, $[a, β) ∩ A_1 = \{ x \}$, pero está claro que $[a,β) ≠ \{x\}$, lo que quiere decir que $[a,β) \nsubseteq A_1$, contradicción.\footnote{No sé si se podría haber hecho más fácil o si me he pasado de riguroso.}

En $A_2$ y $A_3$ también es sencillo: $[4,5)$ y $[7,8)$ son abiertos y por lo tanto para cualquier punto contenido en ellos existe un entorno abierto (ellos mismos). Es decir, que $A_2$ y $A_3$ están en el interior. También podemos usar la propiedad de que $W$ es abierto si y sólo si $W=\intr{W}$.

Y por último, $A_4$ también es todo interior: $∀x ∈ (9,10)$, el elemento de la base $[x, x+ε)$ con $0 < ε < 10 - x (< 1)$ es un abierto contenido en $A_4$ que contiene al punto.

\paragraph{Adherencia} De $A_1$, la adherencia es $A_1$, y lo mismo ocurre con $A_2$ y $A_3$. En $A_4 = (9,10)$, la adherencia es $[9,10)$ porque cualquier abierto de la base $[9, 9+ε)$\footnote{Los abiertos que se extienden más por la izquierda no hace falta considerarlos porque contienen un intervalo del tipo $[9, 9+ε)$.} con $ε>0$ tiene intersección no vacía con $A_4$.
\end{problem}

\begin{problem} Sean $X$ un espacio topológico, $Y$ un espacio topológico de Hausdorff y \stdf continua. Probar que el grafo $\{(x,f(x)) \tq x ∈ X\}$ es cerrado en $X×Y$.
\solution
\end{problem}

\begin{problem} Demostrar que cualquiera de los siguientes subconjuntos de $ℝ^2$, con la topología usual, no son homeomorfos:

\centering \Huge{A, O, H}
\solution
\end{problem}

\begin{problem} Se considera $X = \{x∈ℝ^2 \tq 1 ≤ x_1^2 + x_2^2 ≤ 2\}$ con la topología de subespacio indicida por la del orden lexicográfico en $ℝ^2$.

\ppart Demostrar que $X$ no es conexo.
\ppart Encontrar las componentes conexas de $X$

\solution
\spart
\spart
\end{problem}

\section{2014-2015}

\subsection{Parcial I}

\begin{problem} Sea $\rel = \{( -∞,b] \tq b ∈ ℝ\}$. 
\ppart ¿Es una base de una topología en $ℝ$?
\ppart Dado \[ A = \]
\solution

\spart Tenemos que demostrar que la unión de todos los elementos de la base está en $ℝ$, \[ \bigcup_{b∈ℝ} (-∞, b] = ℝ \], obvio.

Además, dados $B_1, B_2 ∈ \rel$ y $x∈B_1∩B_2$ entonces tiene que existir $B_3∈\rel$ tal que $x∈B_3 ⊆ B_1 ∩ B_2$. Si $B_1 = (-∞, b_1]$ y $B_2 = (-∞, b_2]$, entonces \[ B_1∩B_2 = (-∞, \min (b_1, b_2)] \], que es elemento de la base y entonces ya tenemos lo que necesitábamos.

\spart Para comprobar $x∈\adh{A}$ o $x∈\intr{A}$ basta considerar en la definición elementos de la base.

Si $x≤1$, no está en la adherencia porque $x∈(-∞, x]$, cuya intersección con $A$ es vacío.

Los puntos del intervalo $(1, 2]$ están en la adherencia por ser puntos de $A$.

Si $x>2$, entonces $x∈\adh{A}$ pues $∀B∈\rel$ con $x∈B$ se tiene que $B∩A ≠ ∅$.

Vamos ahora con el interior, que es vacío ya que si $x∈B ∈ \rel$, entonces $B\nsubseteq A$.

\end{problem}

\begin{problem} Sea $\appl{f}{([0,1]×[0,1], \topl_{Lex})}{(ℝ, \topl_{usual})}$ con $f(x,y) = x$. ¿Es continua? ¿Es abierta?
\solution

Sobre la continuidad, $f$ es continua si y sólo si la imagen inversa de un abierto es abierto. Es decir, tenemos que ver que $\inv{f}((a,b))$ es abierto en $([0,1]×[0,1], \topl_{Lex})$.

Si $a≥1$ o $b≤0$, $\inv{f}((a,b)) = ∅$. Tenemos que estudiar el caso de $a<1, b>0$. 

Si todo el intervalo está ($a < 0$ y $b > 1$) entonces la imagen inversa es el total.

Si $b > 1$ y $a ≥ 0$, entonces $\inv{f}((a,b)) = \{ (x_1, x_2) \tq a < x_1 ≤ 1 \}$ que es abierto por ser $((a,1), (1,1)]$, que está en la topología por definición. El caso $a < 0, b ≤ 1$ es similar.

El último caso a estudiar es cuando $a ≥ 0$ y $b ≤ 1$. Aquí, $\inv{f}((a,b)) = \{ (x_1, x_2) \tq a < x_1 < b\} = ((a,1), (b,0))_{Lex} ∈ \topl_{Lex}$ por definición.

Ahora toca ver si es abierta, que no lo es. Por ejemplo $f([(0,0), (0.5, 0)) = [0, 0.5)$ en $ℝ$ que no es abierto. Otro ejemplo sería un intervalo vertical, cuya imagen es un punto que es cerrado.
\end{problem}

\begin{problem} Dado $A⊆X$, 
\ppart Demostrar $\mop{Fr}(\adh{A}) ⊆ \mop{Fr}(A)$ y $\mop{Fr}(\intr{A}) ⊆ \mop{Fr}(A)$.
\ppart Buscar ejemplos en los que se de las inclusiones anteriores estrictas.
\solution
\spart Sabemos que $\mop{Fr}(B) = \adh{B} \setminus \intr{B}$. Luego \[ \mop{Fr}(\adh{A}) = \adh{\adh{A}} \setminus \intr{\adh{A}} \], pero como $A⊆\adh{A}$ entonces $\intr{A} ⊆ \intr{\adh{A}}$, por lo que $\adh{A} \setminus \intr{\adh{A}} ⊆ \adh{A} \setminus \intr{A}$ y entonces $\mop{Fr}(\adh{A}) ⊆ \mop{Fr}(A)$.

Para la frontera del interior, podemos hacerlo usando la dualidad entre complementarios: $\intr{A} = (\adh{A^c})^c$ y $\mop{Fr}(A) = \mop{Fr}(A^c)$, y con la propiedad anterior es inmediato.\footnote{Esto merecería algo más de expansión.}

\spart Buscamos ejemplos extremos que nos faciliten la vida. En el primero, podemos ver que si cogemos el total, su frontera siempre es vacía. De hecho, la frontera de un conjunto es vacía si y sólo si es abierto y cerrado a la vez.

Buscamos entonces un conjunto cuya adherencia sea el total pero que él mismo no sea el total. Por ejemplo, $A=ℚ$ en $ℝ$ con la topología usual. Aquí, $\mop{Fr}(ℚ) = ℝ$, pero $\mop{Fr}(\adh{ℚ}) = \mop{Fr}(ℝ) = ∅$. De hecho, este mismo ejemplo nos vale para la otra inclusión estricta: $\mop{Fr}(\intr{ℚ}) = \mop{Fr}(∅) = ∅$, pero $\mop{Fr}(ℚ) = ℝ$.

\end{problem}

\begin{problem} En $([0,1]×[0,1], \topl_{Lex})$, cogemos el ``borde'' del cuadrado, esto es, \[W = \left(\{0,1\} × [0,1]\right) ∪ \left([0,1] × \{0,1\}\right) \]. Encuentra las componentes conexas de $W$ en $\topl_{Lex}^{sub}$.
\solution

Las componentes conexas han de ser los subconjuntos conexos de $W$ más grandes posibles. ¿Cómo son los conexos en $\topl_{Lex}$ $[0,1]×[0,1]$? Son los intervalos.

En $W$, el conexo más grande que contiene al $(0,0)$ es $[(0,0), (0,1)]_{Lex}$. Si $x = (x_1, x_2) > (0,1)$ entonces $x_1 > 0$ y $[(0,0), x] \nsubseteq W$.

Por el mismo argumento, la componente conexa más grande que contiene al $(1,1)$ es el intervalo $[(1,0), (1,1)]$. 

¿Cuáles son las componentes que contienen a los puntos de los intervalos horizontales? Si $x = (a,0)$ con $0<a<1$, la componente conexa que lo contiene es $\{(a,0)\}$ pues cualquier intervalo $(α,β)$ con $(a,0) ∈ (α,β)$ implica que $α < a$, y entonces $(α,β) \nsubseteq W$.
\end{problem}

\printindex

\end{document}
