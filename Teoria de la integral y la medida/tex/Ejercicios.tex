\section{Ejercicios}
\subsection{Hoja 1}
\begin{problem}[Ejercicio 5]
Dada una sucesión $\lbrace f_n \rbrace \in R([a,b])$ que converge uniformemente a $f$, se pide demostrar que $f$ es ingrable Riemann y que:
\[ lim \int_{a}^{b} f_n = \int_{a}^{b} f \]

\solution
Supongamos que f es integrable Riemann, entonces tenemos que ver que:
\[ \forall \epsilon > 0 \ , \exists N \tq \forall n> N \ \abs{\int_a^b f_n - \int_a^b f} < \epsilon\]

Sabemos que:
\[\abs{\int_a^b f_n - \int_a^b f} \leq \abs{\int_a^b \abs{f_n -f}} dx\]

Recordemos la definición de convergencia uniforme

\begin{defn}[Convergencia\IS uniforme]
\[f_n \xrightarrow{uniforme} f \Leftrightarrow \forall \epsilon < 0, \exists N_{\epsilon} \tq \forall x \in [a,b]  \forall n \geq N_{\epsilon}, \abs{f_n (x) - f(x)} < \epsilon\]
\end{defn}

Si $n \geq N_{\frac{\epsilon}{b-a}}$ entonces, usando la definición de convergencia uniforme:

\[\int_a^b \abs{f_n(x) - f(x) dx} \leq \int_a^b \frac{\epsilon}{b-a}dx = \epsilon\]

Por tanto queda claro que si $f$ es integrable Riemann podemos conmutar el límite con la integral. Ahora queda ver por qué $f$ es integrable Riemann.

$f$ será integrable Riemann sii:
\[\forall \epsilon > 0 \ \exists P \tq \forall P' \prec P \]
\[\overline{J}_{P'}(f) - \underline{J}_{P'}(f) < \epsilon\]

Vamos a probarlo:
\[\overline{J}_P(f) - \underline{J}_P(f) = \sum(\sup_k f - \underset{k}{inf} f)\abs{I_k} \leq\]
\[\leq \sum\left( \abs{\sup_{k}f_n(x) - \sup_{k}f(x)} + \abs{\sup_{k}f_n(x) - \underset{k}{inf}f_n(x)} +  \abs{\underset{k}{inf}f_n(x) - \underset{k}{inf}f_n(x)} \right)\abs{I_k} \leq \]
Puesto que $f_n$ converge uniformemente a $f$ habrá un $n$ a partir del cual la distancia máxima entre $f_n$ y $f$ sea $\frac{\epsilon}{6}$ y por tanto la distancia máxima entre el supremo y el ínfimo de $f_n$ será menor que $\frac{\epsilon}{3}$
\[\leq \sum\left( \sup_{k} \abs{f_n(x) - f(x)} + \frac{\epsilon}{3} + \sup_k \abs{f_n(x) - f(x)} \right)\abs{I_k} \leq\]
Aplicando de nuevo la convergencia uniforme y tomando el máximo entre este $n$ y el calculado en el paso interior nos queda:
\[\leq \sum \left(\frac{\epsilon}{3} + \frac{\epsilon}{3} + \frac{\epsilon}{3} \right)\abs{I_k} = \sum\epsilon\abs{I_k}\]

Como hay convergencia uniforme entre $f_n$ y $f$ podemos hacer que los supremos sean tan pequeños como queramos y hacer así que el interior sumatorio quede menor que $\epsilon \abs{I_k}$

Puesto que $\epsilon$ es un número cualquiera podemos hacerlo tan pequeño como queramos haciendo que el último sumatorio escrito tienda a 0.

\end{problem}

\begin{problem}[Ejercicio 6]

Sea $\lbrace f_n \rbrace$ una sucesión monótona creciente de funciones continuas en un intervalo $I=[a,b]$ que convergen en dicho intervalo a otra función continua $f$. Demuestra que entonces:
\[ lim \int_{a}^{b} f_n(x) = \int_{a}^{b} f(x) \]
\solution
Por ser funciones continuas son integrables Riemann. Si conseguimos demostrar que convergen uniformemente podemos emplear el ejercicio anterior y lo tendríamos hecho.
\end{problem}

\begin{problem}[Ejercicio 7]
Dada la sucesión $I_k = (a_k, b_k)$ tales que $\bigcup_{k=1}^{N}~I_k~=~[a,b]$

Demostrar que:
\[b-a \leq \sum_{k=1}^N (b_k - a_k)\]

\solution
Vamos a utilizar la integral de Riemann como recomienda el ejercicio, utilizando la función indicatriz de cada intervalo $\ind_{I_k}$

Está claro que la función indicatriz del intervalo I es menor o igual que la suma de las funciones indicatrices de los intervalos. Lo cual es obvio, ya que si $x$ está en el intervalo, $x$ estará también en al menos uno de los intervalos $I_k$. Es decir:
\[\ind_{[a,b]} \leq \sum_{k=1}^{N} \ind_{I_k}\]

Utilizando la monotoreidad de la integral de Riemann podemos ``integrar a ambos lados`` obteniendo:

\[\int_{a}^{b} \ind_{[a,b]} \leq \int_{a_k}^{b_k} \sum_{k=1}^{N} \ind_{I_k}\]

Por la linealidad de la integral, podemos incluso meter la integral denro del sumatorio
\[\int_{a}^{b} \ind_{[a,b]} \leq \sum_{k=1}^{N} \int_{a_k}^{b_k} \ind_{I_k}\]


Conociendo la integral de la función indicatriz tenemos el resulado de forma inmediata.
\[b-a \leq \sum_{k=1}^{N} ( b_k - a_k )\]

\end{problem}