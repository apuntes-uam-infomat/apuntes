\section{Ejercicios}
\subsection{Hoja 1}
\begin{problem}[Ejercicio 5]
Dada una sucesión $\lbrace f_n \rbrace \in R([a,b])$ que converge uniformemente a $f$, se pide demostrar que $f$ es ingrable Riemann y que:
\[\lim \int_a^b f_n = \int_a^b f\]

\solution
Supongamos que f es integrable Riemann, entonces tenemos que ver que:
\[ \forall \epsilon > 0 \  \exists N \tq \forall n> N \ \abs{\int_a^b f_n - \int_a^b f} < \epsilon\]

Sabemos que:
\[\abs{\int_a^b f_n - \int_a^b f} \leq \abs{\int_a^b \abs{f_n -f}} dx\]
%TODO Recordar definición de convergencia uniforme

Si $n \geq N_{\frac{\epsilon}{b-a}}$ entonces, usando la definición de convergencia uniforme:

\[\int_a^b \abs{f_n(x) - f(x) dx} \leq \int_a^b \frac{\epsilon}{b-a}dx = \epsilon\]

Por tanto queda claro que si $f$ es integrable Riemann podemos conmutar el límite con la integral. Ahora queda ver por qué $f$ es integrable Riemann.

$f$ será integrable Riemann sii:
\[\forall \epsilon > 0 \ \exists P \tq \forall P' \prec P \ \]
%TODO Diferencia entre suma superior e inferior menor que epsilon

Vamos a probarlo:
\[\overline{J}_p(f) - \underline{J}_p(f) = \sum(\sup_k f - \inf_k f)\abs{I_k} = \sum()\abs{I_k} \leq \sum ()\abs{I_k}\]
%TODO Pa' ti Jorge

Como hay convergencia uniforme entre $f_n$ y $f$ podemos hacer que los supremos sean tan pequeños como queramos y hacer así que el interior sumatorio quede menor que $\epsilon \abs{I_k}$

\end{problem}

\begin{problem}[Ejercicio 6]

Caca 
\solution
Por ser funciones continuas son integrables Riemann. Si conseguimos demostrar que convergen uniformemente podemos emplear el ejercicio anterior y lo tendríamos hecho.
\end{problem}

\begin{problem}[Ejercicio 7]
Dada la sucesión $I_k = (a_k, b_k)$ tales que $\bigcap_{k=1}^NI_k = \a,b]$

Demostrar que:
\[b-a \leq \sum_{k=1}^N (b_k - a_k)\]

\solution
%TODO Apañar esto 
Vamos a utilizar la integral de Riemann como recomienda el ejercicio, utilizando la función indicatriz de cada intervalo $\ind_{I_k}$

Está claro que la función indicatriz del intervalo I es menor o igual que la suma de las funciones indicatrices de los intervalos. Lo cual es obvio, ya que si x está en el intervalo, x estará también en al menos alguno de los intervalos i_k.

Utilizando la monoticidad de la integral de Riemann podemos ``integrar a ambos lados" obteniendo:

%TODO formula

Por la linealidad de la integral, podemos incluso meter la integral denro del sumatorio

%TODO formula

Conociendo la integral de la función indicatriz tenemos el resulado de forma inmediata.

%TODO formula

\end{problem}