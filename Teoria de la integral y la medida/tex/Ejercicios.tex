% -*- root: ../TIM.tex -*-
\section{Hoja 1}
\begin{problem}[1]
Demuestra que el valor absoluto de una función integrable Riemann es también integrable Riemann

\solution

Vamos a definir:
$f^+(x) = \max\lbrace f(x), 0 \rbrace$

$f(x) = f^+(x) - f^-(x)$

De donde podemos deducir:

$f^-(x) = f^+(x)-f(x)=\max \lbrace -f(x), 0 \rbrace$

Y las funciones tienen el siguiente aspecto:

\begin{center}
	% Función f
	\inputtikz{img1-ej1-h1}
	% Función f+
	\inputtikz{img2-ej1-h1}
	% Funcion f-
	\inputtikz{img3-ej1-h1}
\end{center}


Así podemos escribir \[ \abs{f(x)} = f^+(x)+f^-(x) \]

Gracias a esta afirmación el problema se reduce a demostrar que la parte positiva ($f^+$) es integrable Riemann, ya que entonces lo serán  y $f^-$  y $\abs{f(x)}$  por ser la suma/resta de dos funciones integrables Riemann.


Vamos a calcular $f^+(x)$:
\[\overline{J}_P(f^+)-\underline{J}_P(f^+) = \sum_{k=1}^n\left(\sup_{x\in I_k}(f^+(x))\right)\abs{I_k} - \sum_{k=1}^n\left(\inf_{x\in I_k}(f^+(x))\right)\abs{I_k}=\]
\[= \sum_{k=1}^n\left(\sup_{x\in I_k}(f^+(x))-\inf_{x\in I_k}(f^+(x))\right)\abs{I_k} \leq
\sum_{k=1}^n\left(\sup_{x\in I_k}(f(x))-\inf_{x\in I_k}(f(x))\right)\abs{I_k} =\] \[=\overline{J}_P(f)-\underline{J}_P(f) = 0\]

Puesto que $f$ es integrable
\end{problem}

\begin{problem}[2]
Sean $f,g$ funciones integrables Riemann en el intervalo [a,b]. Demuestra que $h(x)=\max\{f(x),g(x)\}$ es integrable Riemann. Este resultado se extiende con facilidad al máximo de un número finito de funciones. Muestra que en general no es cierto para una sucesión

\textbf{Indicación:}
\[\sup_{I_k}\{\max\{f(x),g(x)\}\}-\inf_{I_k}\{\max\{f(x),g(x)\}\} \geq \]
\[\geq \max\{\sup_{I_k}f(x)-\inf_{I_k}f(x), \sup_{I_k}g(x)-\inf_{I_k}g(x)\}\]

\solution
Vamos a comprobar que la diferencia entre las integrales Riemann superior e inferior es 0.
Pero antes fijémonos en la \underline{indicación}:

\[\sup\big\{ \max\{f(x), g(x)\}\big\} - \inf\big\{ \max\{f(x), g(x)\} \big\} \leq \]

\[ \leq \max\big\{ \sup_{x \in I_k} (f(x)) - \inf_{x \in I_k} (f(x)), \sup_{x \in I_k} (g(x)) - \inf_{x \in I_k} (g(x)) \big\} \leq \]

\[ \leq \left(\sup_{x \in I_k} (f(x)) - \inf_{x \in I_k} (f(x))\right) + \left(\sup_{x \in I_k} (g(x)) - \inf_{x \in I_k} (g(x))\right) \]


Ahora podemos probar que:
\[\overline{J}_P(h)-\underline{J}_P(h)
= \sum\left(\sup_{x\in I_k}(h(x))-\inf_{x\in I_k}(h(x))\right)\abs{I_k} \leq\]


\[\leq \sum\left( \sup_{x \in I_k} (f(x)) - \inf_{x \in I_k} (f(x)) \right) \abs{I_k} +
 \sum\left( \sup_{x \in I_k} (g(x)) - \inf_{x \in I_k} (g(x)) \right) \abs{I_k} \]

Sabiendo que $f$ y $g$ son integrables Riemann llegamos a:
\[\sum\left( \sup_{x \in I_k} (f(x)) - \inf_{x \in I_k} (f(x)) \right) \abs{I_k} +
 \sum\left( \sup_{x \in I_k} (g(x)) - \inf_{x \in I_k} (g(x)) \right) \abs{I_k} \leq 0 + 0 = 0\]

 Con lo que queda probado que $h(x) = \max\lbrace f(x), g(x) \rbrace$ es integrable Riemann

\end{problem}

\begin{problem}[3]
Sea $\{q_1,q_2,q_3,...\} = I \cap \rac$ una enumeración de los racionales de un intervalo $I=[a,b]$. Para cada $k=1,2,3,...,$ sea
\[Y_k(x)= \left\{ \begin{array}{lcc}
             1 &   si  & x \in \{q_1,q_2,q_3,...\} \\
             \\ 0 &  si  & x \notin I \setminus \{q_1,q_2,q_3,...\}
             \end{array}
   \right.\]

Demuestra que $Y_k$ es integrable Riemann.

Sea $Y(x)=\lim_kY_k(x)$, demuestra que $Y$ no es integrable Riemann.

\solution

Ya hemos visto en teoría la demostración de que toda $Y_k$ es integrable Riemann. La idea se basa en descomponerla en una suma de funciones, cada una de las cuales vale 1 en un racional y 0 en el resto.

Tomando intervalos suficientemente pequeños podemos probar que estas funciones son integrables Riemann con integral, por lo que $Y_k$ será integrable por ser suma de funciones integrables.

Para el caso $Y(x)=\lim_kY_k(x)$, vamos a ver que no es integrable Riemann viendo que las sumas superior e inferior no coinciden.

Para cualquier intervalo que cojamos, siempre vamos a tener algún racional, de modo que $\sup_{x \in I_k} (f(x)) = 1 \ \forall I_k$, pero también tendremos siempre un irracional por lo que $\inf_{x \in I_k} (f(x)) = 0 \ \forall I_k$.

Así, vemos sencillamente que la suma superior nos dará la longitud del intervalo mientras que la inferior nos dará 0
\end{problem}

\begin{problem}[4]
Con la notación del ejercicio anterior, sea
\[Z(x)= \left\{ \begin{array}{lcc}
             \frac{1}{k} &   si  & x = q_k \\
             \\ 0 &  si  & x \notin I \setminus \rac
             \end{array}
   \right.\]

Demuestra que la función $Z$ así definida es integrable Riemann en $I$ y calcula $\in_a^bZ(x)dx$

\solution

Vamos a construir la sucesión de funciones $Z_n$ similar a las del apartado anterior
\[Z_n(x)= \left\{ \begin{array}{lcc}
             \frac{1}{k} &   si  & x \in \{q_1,q_2,q_3,...,q_n\} \\
             \\ 0 &  si  & x \notin I \setminus \{q_1,q_2,q_3,...,q_n\}
             \end{array}
   \right.\]

Vemos que esta sucesión converge uniformemente a $Z(x)$ puesto que
\[\forall \epsilon, x \exists N  \tq \forall n > N, \ |Z_n(x)-Z(x)|<\epsilon\]
Si nos fijamos, vemos que $|Z_n(x)-Z(x)|<\frac{1}{n+1}$, de modo que si forzamos que $\frac{1}{n+1}<\epsilon$ estará probada la convergencia uniforme.

Basta con tomar $N=\frac{1}{\epsilon} -1$.

Una vez probada la convergencia uniforme, sabemos que $Z(x)$ es integrable y que
\[\int_a^b \lim Z_n(x) =\lim \int_a^b Z_n(x) = \lim 0 = 0\]
Sabemos que $\int_a^b Z_n(x)$ puesto que es un caso equivalente al del ejercicio anterior.
\end{problem}

\begin{problem}[5]
Dada una sucesión $\lbrace f_n \rbrace \in R([a,b])$ que converge uniformemente a $f$, se pide demostrar que $f$ es integrable Riemann y que:
\[ \lim \int_{a}^{b} f_n = \int_{a}^{b} f \]

\solution
Supongamos que f es integrable Riemann, entonces tenemos que ver que:
\[ \forall \epsilon > 0 \ , \exists N \tq \forall n> N \ \abs{\int_a^b f_n - \int_a^b f} < \epsilon\]

Sabemos que:
\[\abs{\int_a^b f_n - \int_a^b f} \leq \abs{\int_a^b \abs{f_n -f}} dx\]

Recordemos la definición de convergencia uniforme

\begin{defn}[Convergencia\IS uniforme]
\[f_n \xrightarrow{uniforme} f \Leftrightarrow \forall \epsilon < 0, \exists N_{\epsilon} \tq \forall x \in [a,b]  \forall n \geq N_{\epsilon}, \abs{f_n (x) - f(x)} < \epsilon\]
\end{defn}

Si $n \geq N_{\frac{\epsilon}{b-a}}$ entonces, usando la definición de convergencia uniforme:

\[\int_a^b \abs{f_n(x) - f(x) dx} \leq \int_a^b \frac{\epsilon}{b-a}dx = \epsilon\]

Por tanto queda claro que si $f$ es integrable Riemann podemos conmutar el límite con la integral. Ahora queda ver por qué $f$ es integrable Riemann.

$f$ será integrable Riemann sii:
\[\forall \epsilon > 0 \ \exists P \tq \forall P' \prec P \]
\[\overline{J}_{P'}(f) - \underline{J}_{P'}(f) < \epsilon\]

Vamos a probarlo:
\[\overline{J}_P(f) - \underline{J}_P(f) = \sum(\sup_k f - \underset{k}{inf} f)\abs{I_k} \leq\]
\[\leq \sum\left( \abs{\sup_{k}f(x) - \sup_{k}f_n(x)} + \abs{\sup_{k}f_n(x) - \underset{k}{inf}f_n(x)} +  \abs{\underset{k}{inf}f_n(x) - \underset{k}{inf}f(x)} \right)\abs{I_k} \leq \]
Puesto que $f_n$ converge uniformemente a $f$ habrá un $n$ a partir del cual la distancia máxima entre $f_n$ y $f$ sea $\frac{\epsilon}{6}$ y por tanto la distancia máxima entre el supremo y el ínfimo de $f_n$ será menor que $\frac{\epsilon}{3}$
\[\leq \sum\left( \sup_{k} \abs{f_n(x) - f(x)} + \frac{\epsilon}{3} + \sup_k \abs{f_n(x) - f(x)} \right)\abs{I_k} \leq\]
Aplicando de nuevo la convergencia uniforme y tomando el máximo entre este $n$ y el calculado en el paso interior nos queda:
\[\leq \sum \left(\frac{\epsilon}{3} + \frac{\epsilon}{3} + \frac{\epsilon}{3} \right)\abs{I_k} = \sum\epsilon\abs{I_k}\]

Como hay convergencia uniforme entre $f_n$ y $f$ podemos hacer que los supremos sean tan pequeños como queramos y hacer así que el interior sumatorio quede menor que $\epsilon \abs{I_k}$

Puesto que $\epsilon$ es un número cualquiera podemos hacerlo tan pequeño como queramos haciendo que el último sumatorio escrito tienda a 0.

\end{problem}

\begin{problem}[6]

Sea $\lbrace f_n \rbrace$ una sucesión monótona creciente de funciones continuas en un intervalo $I=[a,b]$ que convergen en dicho intervalo a otra función continua $f$. Demuestra que entonces:
\[ lim \int_{a}^{b} f_n(x) = \int_{a}^{b} f(x) \]
\solution
Por ser funciones continuas son integrables Riemann. Si conseguimos demostrar que convergen uniformemente podemos emplear el ejercicio anterior y lo tendríamos hecho.

Vamos a ello.

%TODO solución del libro: "Fundamentos de análisis moderno, volumen 1"

Lo que queremos demostrar coincide exactamente con el \textbf{Teorema de Dini}.
Procedamos a demostrarlo.

Para cada $\epsilon > 0$ y cada $x\in[a,b]$ existe un índice $n(x)$ tal que:
\[\forall m\geq n(x), \ f(x)-f_m(x) \leq \frac{\epsilon}{3}\]
Por ser $f$ y $f_{n(x)}$ continuas, existe un intervalo abierto $x\in(a_x,b_x)$, tal que:
\[\forall y \in (a_x,b_x), \ |f(x)-f(y)| \leq \frac{\epsilon}{3} \text{ y } |f_{n(x)}(x)-f_{n(x)}(y)| \leq \frac{\epsilon}{3}\]

Por tanto, para cada $y\in(a_x, b_x)$ se tiene:
\[|f(y)-f_{n(x)}(y)| = |f(y)-f(x)+f(x)-f_{n(x)}(x)+f_{n(x)}(x)-f_{n(x)}(y)|\leq \epsilon\]
Tómese ahora un número finito de puntos $x \in [a,b]$ tales que $(a_x,b_x)$ recubran $I$ y sea $n_0$ el mayor de los enteros $n(x)$. Entonces para cada $y\in I$, $y$ pertenece a uno de estos $(a_x,b_x)$, por tanto:

\[ \forall n\geq n_0, f(y)-f_n(y) \leq \epsilon\]

\end{problem}

\begin{problem}[7]
Dada la sucesión $I_k = (a_k, b_k)$ tales que $\bigcup_{k=1}^{N}~I_k~=~[a,b]$

Demostrar que:
\[b-a \leq \sum_{k=1}^N (b_k - a_k)\]

\solution
Vamos a utilizar la integral de Riemann como recomienda el ejercicio, utilizando la función indicatriz de cada intervalo $\ind_{I_k}$

Está claro que la función indicatriz del intervalo I es menor o igual que la suma de las funciones indicatrices de los intervalos. Lo cual es obvio, ya que si $x$ está en el intervalo, $x$ estará también en al menos uno de los intervalos $I_k$. Es decir:
\[\ind_{[a,b]} \leq \sum_{k=1}^{N} \ind_{I_k}\]

Utilizando la monotoneidad de la integral de Riemann podemos ``integrar a ambos lados'' obteniendo:

\[\int_{a}^{b} \ind_{[a,b]} \leq \int_{a_k}^{b_k} \sum_{k=1}^{N} \ind_{I_k}\]

Por la linealidad de la integral, podemos incluso meter la integral dentro del sumatorio
\[\int_{a}^{b} \ind_{[a,b]} \leq \sum_{k=1}^{N} \int_{a_k}^{b_k} \ind_{I_k}\]


Conociendo la integral de la función indicatriz tenemos el resultado de forma inmediata.
\[b-a \leq \sum_{k=1}^{N} ( b_k - a_k )\]

\end{problem}

\begin{problem}[8]
Demuestra que para cualquier conjunto $C\subset [a,b]$
\[cont^+(C) = \inf \{\sum_{n=1}^N\abs{I_n} \tq C \subset I_1 \cup I_2 \cup ... \cup I_N \text{ siendo } I_n=[a_n,b_n]\}\]

\solution

Recordemos que
\[cont^+(C)=\overline{J_P}\ind_C(x) = \sum_{n=1}^{\infty}\sup\{\ind_C(x)\}|I_k|\]
pero el supremo de la función indicatriz en un intervalo es 1 (en ese intervalo la función indicatriz siempre valdrá 1.)

Por tanto
\[cont^+(C)=\overline{J_P}\ind_C(x) =  \sum_{n=1}^{\infty}|I_k| \geq \inf \{\sum_{n=1}^N\abs{I_n} \tq C \subset I_1 \cup I_2 \cup ... \cup I_N \text{ siendo } I_n=[a_n,b_n]\}\]
por cumplir los $I_k$ la condición dada y tener a la izquierda un ínfimo.

Si logramos probar ahora la desigualdad contraria tendremos el ejercicio resuelto.
Para ello, podemos fijarnos en que
\[\forall N \  \sum_{n=1}^{\infty}|I_k| \geq \sum_{n=1}^{\infty}|I'_k| \]
ya que en el lado de la derecha, los $I'_k$ son disjuntos y su unión es exactamente $C$.

Por tanto, queda claro que la fórmula dada para el contenido de $cont^+(C)$ es válida

\end{problem}

\begin{problem}[9]
Dado $O\subset (a,b)$ unión numerable de intervalos disjuntos, se pide demostrar:
\[m^*(O)=\sum_{n=1}^{\infty} \abs{I_k}\]

\solution
Por definición de medida exterior, está claro que:
\[m^*(O)=\inf\{\sum_{k=1}^{\infty}|I_k| \tq O \subset \bigcup_{k=1}^{\infty}\} \leq \sum_{k=1}^{\infty} \abs{I_k}\]
Vamos a probar ahora que $m^*(O) \geq \sum_{k=1}^{\infty} \abs{I_k}$ y tendremos la igualdad buscada.\\
Puesto que sabemos que la suma infinita tiene resultado finito (no puede superar la longitud del intervalo $(a,b)$ ya que todos los $I_k$ se contienen en él y son disjuntos), sabemos que:
\[\forall \epsilon > 0 \exists N \tq \sum_{k=1}^{N} \abs{I_k} \geq \sum_{k=1}^{\infty} \abs{I_k} - \frac{\epsilon}{2}\]
Definimos ahora unos intervalos $I'_k$ más pequeños de la forma:
\[I'_k=[a_k+\frac{\frac{\epsilon}{2}}{2^{k+1}}, b-\frac{\frac{\epsilon}{2}}{2^{k+1}}]\]
Así:
\[\sum_{k=1}^{N}|I'_k| = \sum_{k=1}^{N}|I_k|-\frac{\epsilon}{2}\sum_{k=1}^{N}\frac{1}{2^k} \geq \sum_{k=1}^{N}|I_k|-\frac{\epsilon}{2}\]
Definimos ahora el compacto formado por la unión de estos $I_k$ compactos:
\[K =\bigcup_{k=1}^{N}I'_k \subset O\]
Sean $J_k$ intervalos abiertos que recubren $O$, también recubren $K$. Por ser $K$ compacto, dado un recubrimiento podemos encontrar un recubrimiento finito, es decir:
\[\exists N' \tq K \subset \bigcup_{k=1}^{N'}J_k = A_N\].
Volviendo a la medida exterior de O, tenemos que:
\[m^*(O) = \inf \{\sum_{k=1}^{\infty}|J_k|\} \geq \inf\{\sum_{k=1}^{N'}|J_k|\} \geq \inf\{\sum_{k=1}^{N'}|I'_k|\} \geq \inf\{\sum_{k=1}^{N'}|I_k|-\frac{\epsilon}{2}\} \geq \inf \{\sum_{k=1}^{\infty} \abs{I_k} -\epsilon\} \forall \epsilon\]
Por ser cierta la desigualdad para todo $\epsilon$ se deduce que:
\[m^*(O) \geq \inf \{\sum_{k=1}^{\infty}|I_k|\}\]
\end{problem}

\begin{problem}[10]
Dado un compacto K contenido en el intervalo (a,b), nos piden demostrar que:
\[m(K)<b-a\]

\solution
Consideremos la sucesión: $(a+\frac{1}{n}, b - \frac{1}{n})$, con $n >\frac{2}{b-a}$. Obviamente:
\[K \subset \bigcup_{n=n_0}^{\infty}(a+\frac{1}{n}, b - \frac{1}{n})\]

Como K es compacto:
\[\exists N \tq K \subset \bigcup_{n=n_0}^{N}(a+\frac{1}{n}, b - \frac{1}{n}) = (a+\frac{1}{N}, b - \frac{1}{N}) \]

Y llegamos a:
\[m(K) \leq b-a-\frac{2}{N} < b-a\]

\end{problem}

\begin{problem}[11]
Sea $C_n$ una sucesión creciente (es decir: $C_1\subset C_2 \subset C_3...$) de subconjuntos medibles de (a,b). Sea $C=\bigcup C_n$. Demuestra que $m(C_n)$ es una sucesión creciente con límite $m(C)$
\solution
Está claro que $m(C_n)$ es una sucesión creciente puesto que, por las propiedades de la medida exterior, se cumple que:
\[C_1 \subset C_2 \Rightarrow m^*(C_1) \leq m^*(C_2)\]
y al ser conjuntos medibles, $m(C) = m^*(C)$, con lo que nos queda:
\[C_1 \subset C_2 \Rightarrow m(C_1) \leq m(C_2)\]

Sabemos además que la sucesión es acotada puesto que nunca puede superar $m((a,b))$, ya que:
\[\forall i, C_i \subset (a,b) \Rightarrow \forall i, m(C_i)\leq m((a,b))\]

La sucesión de las medidas convergerá si:
\[\lim_{n \to \infty}m(C_n) = m(C) \iff \forall \epsilon > 0 \exists n_0 \tq \forall n > n_0, m(C)-m(C_n)<\epsilon\]

Vamos a demostrar que converge por reducción al absurdo.\\
Supongamos que no converge, entonces:
\[\exists ε > 0 \tq \forall n \ m(C)-m(C_n) > \epsilon\]
En ese caso, por ser medibles tanto los $C_n$ como $C$, tendríamos que:
\[m(C\setminus C_n) > \epsilon \Rightarrow \exists x \in C \wedge x \notin C_n \Rightarrow C\neq \bigcup C_n\]
Y llegamos a contradicción.

\end{problem}

\begin{problem}[12]
Sea $C_n$ una sucesión decreciente de subconjuntos medibles contenidos en (a,b) y sea $C$ la intersección de estos subconjuntos. Se pide demostrar que:
\[m(C_n) \searrow m(C)\]
\solution

Tomando los complementarios, que también son medibles, tenemos una sucesión creciente de subconjuntos medibles contenidos en (a,b). Aplicando el ejercicio anterior llegamos a que:
\[(b-a)-m(C_n) \rightarrow (b-a)-m(C)\]
De donde puede deducirse que $m(C_n)$ decrece hacia $m(c)$.
\end{problem}


\begin{problem}[13]
Sea $C_n$ una sucesión creciente de subconjuntos medibles de $\real$. Demuestra que
\[\lim_nm(C_n) \nearrow m(\bigcup C_n)\]

Sea $C_n$ una sucesión decreciente de subconjuntos medibles de $\real$. Demuestra que si existe $n_0$ tal que $m(C_{n_0}) < \infty$ entonces
\[\lim_nm(C_n) \searrow m(\bigcap C_n)\]

Da un contraejemplo que muestre que sin la condiciones de la finitud de al menos una de las medidas el resultado puede ser falso.

\solution

\textcolor{blue}{Hecho por mi, no fiarse al 100\%}

Para la primera parte podemos repetir la prueba del ejercicio 11 considerando que $C$ es un conjunto medible. En caso de no serlo tampoco tendríamos problemas puesto que las medidas convergerían a infinito.

Para el segundo apartado, si consideramos la finitud de una de las medidas, podemos emplear $C_0$ como conjunto respecto al cual calcularemos los complementarios, replicando así la demostración del ejercicio 12.

En este caso, la finitud de al menos uno de los intervalos si es necesaria ya que de lo contrario podríamos tener conjuntos infinitos con intersección vacía, lo que causaría que el límite de las medidas se nos fuese a infinito mientras la intersección permanece vacía y, por tanto, con medida 0.

\end{problem}
\newpage
\begin{problem}[14]
Dada una sucesión $A_n$ contenida en (a,b), se pide demostrar que:
\[\lim_n m(A_0\Delta A_n)=0 \Rightarrow \lim_n m(A_n)=m(A_0) \]

Recordemos que:
\[ A_n \Delta A_0 = (A_n \setminus A_0) \cup (A_0 \setminus A_n) =
(A_n \cup A_0)\cap(A_n^c \cup A_0^c) \]

\solution

\[\lim_n m(A_0\Delta A_n)=0 \Rightarrow \lim_n m(A_0^c \cap A_n)=0 \wedge \lim_n m(A_0\cap A_n^c)=0\]

Escribimos ahora $A_n$ y $A_0$ como:
\[A_n = (A_n \cap A_0) \bigcup (A_n \cap A_0^c)\]
\[A_0 = (A_0 \cap A_n) \bigcup (A_0 \cap A_n^c)\]

De aquí podemos ver que:
\[m(A_n) = m(A_n \cap A_0) + m(A_n \cap A_n^c)\]
\[m(A_0) = m(A_0 \cap A_n) + m(A_0 \cap A_n^c)\]

Y restando llegamos a:
\[m(An) = m(A_0) -m(A_0 \cap A_n^c) + m(A_n \cap A_0^c)\]

Y sabemos que las medidas de estas intersecciones tienden a 0.
\end{problem}

\begin{problem}[15]
\textbf{Conjunto ternario de Cantor}.Construye el conjunto K de la siguiente manera: $K_0=[0,1]$, $K_1$ se obtiene a partir de $K_0$ suprimiendo el intervalo abierto central de longitud $r=\frac{1}{3}$. $K_1$ es entonces la unión de dos intervalos cerrados de longitud $\frac{1}{3}$.. $K_2$ se obtiene suprimiendo de cada uno de los dos intervalos de $K_1$ su intervalo abierto central de longitud $r^2=\frac{1}{9}$ Si se continúa el proceso see obtiene una sucesión de cerrados $K_n$
en la que cada uno de ellos es unión de $2^n$ intervalos cerrados, cada uno de longitud $3^{-n}$:
\[K_0 \supset K_1 \supset ... \searrow K=\bigcap_nK_n\]

Demuestra que $K$ es compacto en $\real$ (y por tanto medible), que no contiene intervalos abiertos, que no tiene puntos aislados, y que $m(K)=0$.

\solution

\textcolor{blue}{Hecho por mi, no fiarse al 100\%}

En cada iteración estamos extrayendo del conjunto $K_0$ una serie de intervalos abiertos y, al final, el conjunto que nos queda es la diferencia entre un cerrado y un abierto contenido en él.

Por tanto el complementario de nuestro conjunto es la unión del complementario de $K_0$ y todos los abiertos que hemos ido extrayendo, es decir, su complementario es una unión de abiertos que es abierta, por lo que tenemos que el conjunto $K$ es cerrado y acotado (sigue contenido en $K_0$).

Por ser un conjunto cerrado y acotado en $\real$ obtenemos que es compacto.

En cada paso, extraemos un intervalo de longitud $\frac{1}{3}$ y evitamos la existencia de intervalos abiertos de longitud mayor que esa. Por tanto, dado un intervalo abierto de longitud $a$, a partir de la iteración $n$ en la que $\sum_{i=1}^{n}2^i\frac{1}{3^{i+1}}>a$ tendremos excluida la posibilidad de que exista dicho intervalo abierto.

La suma anterior tiene límite 1, de modo que puede superar a cualquier $a$.

Por último nos queda ver que no tiene puntos aislados. Todo punto $x$ del conjunto es límite de una sucesión de extremos de intervalos (los que hemos ido quitando) complementarios distintos de $x$, por lo que todo abierto que lo contenga contendrá también a un número infinito de elementos de la sucesión, luego intersecará con el conjunto de Cantor.

Para construir esta sucesión podemos empezar por el 0 y, en cada paso, tomamos el extremo de los nuevo sintervalos que deje el punto a su izquierda (por ejemplo)

Para ver que $m(K)=0$ basta con ver que el complementario (los abiertos que vamos eliminando) constituyen un conjunto de medida 1:
\[\sum_{i=0}^{n}2^i\frac{1}{3^{i+1}} = \frac{1}{2}\sum_{i=1}^{n}\left(\frac{2}{3}\right)^i = \frac{1}{2}\frac{\frac{2}{3}}{1-\frac{2}{3}} = 1\]
por tratarse de ĺa suma de una sucesión geométrica de razón $r=\frac{1}{3}$
\end{problem}

\begin{problem}[16]
Demuestra que si repetimos la construcción anterior con $r^n=\frac{1}{5^n}$ obtenemos otro conjunto de Cantor compacto, que no contiene intervalos abiertos, ni puntos aislados pero cuya medida es mayor que 0.

\solution

\textcolor{blue}{Hecho por mi, no fiarse al 100\%}

Para ver que es compacto podemos repetir la demostración del ejercicio anterior. Teniendo en cuenta que en cada paso impedimos la existencia de intervalos de longitud mayor que $\frac{1}{2\cdot 5^n}$, podemos imitar la demostración del ejercicio anterior para ver que el conjunto de Cantor no contiene ningún intervalo abierto.

Respecto a la medida, podemos ver que la medida del complementario (los abiertos $L_n$ que vamos eliminando del conjunto del partida, tienen medida menor que 1).
\[\sum_{i=0}^{n}2^i\frac{1}{5^{i+1}} = \frac{1}{2}\sum_{i=1}^{n}\left(\frac{2}{5}\right)^i = \frac{1}{2}\frac{\frac{2}{5}}{1-\frac{2}{5}} = \frac{1}{3}\]

De modo que el conjunto $L$ por tratarse de su complementario dentro del intervalo $[0,1]$ debe tener medida $m(L)=\frac{2}{3}$

Para la no existencia de puntos aislados podemos recurrir nuevamente a la prueba ya realizada en el apartado anterior.
\end{problem}

\section{Hoja 2}
\begin{problem}[1]
Sea X=$\{a,b,c,d\}$. Comprueba que la familia de conjuntos
\[A = \{\emptyset, \{a\}, \{b\}, \{a,b\},\{c,d\},\{a,c,d\},\{b,c,d\}, \{a,b,c,d\}\}\]
forma una $\salgb$ en $X$.

\solution
\textcolor{blue}{Hecho por mi, no fiarse al 100\%}

Para comprobar que forma una $\salgb$ debemos comprobar que cumple las condiciones de $\salgb$:

\begin{enumerate}
\item \textbf{Debe contener al total y al vacío.}
Vemos que es cierto ya que contiene $\emptyset$ y $X=\{a,b,c,d,\}$.

\item \textbf{Debe ser cerrada por complementación}
Vemos que tomando cualquier elemento de $A$, su complementario respecto de $X$ también se contiene en $A$.

Los complementarios de los subconjuntos de un sólo elemento son los subconjuntos de tres elementos (y viceversa) y los dos subconjuntos de dos elementos son complementarios el uno del otro.

\item \textbf{Debe ser cerrada por uniones infinitas}
Puesto que el conjunto es finito no tiene sentido hablar de uniones infinitas, por lo que simplemente debemos comprobar que la unión de dos elementos cualesquiera de $A$ se contiene en $A$.

Fácilmente podemos observar que se cumple esta condición puesto que la unión de los subconjuntos de tres elementos con cualquier otro subconjunto nos da el total; la unión de uno de 2 elementos y uno de 2 nos da uno de los de 3; la unión de los de 2 elementos da el total y la unión de los de 1 elemento nos da uno de 2 elementos.
\end{enumerate}

\end{problem}
\begin{problem}[2]
Sea $X=\{a,b,c,d\}$. Se pide construir la $\salgb$ generada por $\algb{E}=\{\{a\},\{b\}\}$ y por $\algb{E}=\{\{a\}\}$

\solution
Sabemos que en un conjunto finito toda álgebra es una $\salgb$, ya que la diferencia entre álgebra y $\salgb$ era el cierre por uniones infinitas numerables. Si un conjunto es finito, $\algb{P}(X)$ será finito y no habrá posibilidad de hacer uniones infinitas.

Vamos a construir la mínima álgebra que contiene a $\algb{E}$ a pelo, forzando que se cumplan las propiedades de un álgebra de conjuntos:
\[\algbM(\{\{a\}\}) = \{\emptyset, \{a\}, \{b,c,d\}, \{a,b,c,d\}\}\]
\[\algb{M}(\{\{a\},\{b\}\})=\{\emptyset, X, \{a\}, \{b\}, \{a,b\}, \{b,c,d\}, \{a,c,d\}, \{c,d\}\}\]

\obs Si tomamos $A_1=\{a\} \ y \ A_2 = \{b\}$ podemos comprobar fácilmente que:
\[\algb{M}(A_1 \cup A_2) \neq \algb{M}(A_1) \cup \algb{M}(A_2)\]
\end{problem}

\begin{problem}
Comprobar que la unión de dos $\salgb$ no tiene por qué ser una $\salgb$. Poner un ejemplo en el que las $\salgb$ de partida tengan infinitos elementos.

\solution
\textcolor{blue}{Hecho por mi, no fiarse al 100\%}

Para comprobarlo podemos apoyarnos en el ejemplo anterior y ver que
\[\algb{M}(A_1) \cup \algb{M}(A_2) = \{\emptyset, \{a\}, \{b\}, \{b,c,d\}, \{a,c,d\}, \{a,b,c,d\}\}\]
es unión de dos $\salgb$ pero no es $\salgb$ ya que $\{a\} \cup \{b\} = \{a,b\} \notin \algbM$

Vamos ahora a buscar un ejemplo de $\salgb$ de infinitos elementos, como pide el enunciado: si tomamos el álgebra formada por los pares (E) y la formada por los impares (O), su unión no es álgebra: eg \{1,2\} $\notin$ E $\cup$ O.
\end{problem}

\begin{problem}[4]
Dada una función $\appl{g}{X}{Y}$ y sea $\algb{A}$ una $\salgb$ de X, demostrar que:
\[\algb{B}=\{E\subset Y \tq g^{-1}(E)\in \algb{A}\}\]
es una $\salgb$

\solution
Debemos comprobar que $\algbB$ cumple las condiciones de $\salgb$. Para ello basta con ver que se cumplen las siguientes dos propiedades sobre $g$
\begin{enumerate}
\item
\[g^{-1}(E_1 \cup E_2)=g^{-1}(E_1) \cup g^{-1}(E_2)\]
\item
\[g^{-1}(Y \setminus E) = X - g^{-1}(E)\]
\end{enumerate}
Vamos a comprobarlas pues
\begin{proof}
\begin{enumerate}
\item
\[x\in g^{-1}(E_1 \cup E_2) \iff g(x) \in E_1 \cup E_2 \iff  \exists i=1,2 \ g(x) \in E_i \iff \]
\[\iff \exists i=1,2 \ x\in g^{-1}(E_i) \iff x\in g^{-1}(E_1) \cup g^{-1}(E_2) \]
\item
\[x\in g^{-1}(Y \setminus E) \iff g(x) \in Y \setminus E \iff g(x) \notin E \iff x \notin g^{-1}(E) \iff x \in X\setminus g^{-1}(E)\]
\end{enumerate}

\end{proof}
Así, por la primera propiedad queda claro que $\algb{B}$ es cerrado por uniones y con la segunda propiedad vemos que es cerrado por complementación. Es decir:
\[\{E_i\}_{i_1}^{\infty} \in \algb{B} \implies \bigcup_{i=1}^{\infty} E_i \in \algb{B}\]
ya que $g^{-1}(\bigcup_{i=1}^{\infty} E_i) = \bigcup_{i=1}^{\infty} g^{-1}(E_i) \in A$

Podemos hacer lo mismo para ver el cierre por complementación

Además, queda claro que tanto el vacío como el total se contienen en $\algb{B}$, ya que $X$ y $\emptyset \in A \implies g^{-1}(Y)=X \in A, g^{-1}(\emptyset)=\emptyset \in A \implies Y, \emptyset \in \algb{B}$ por lo que se trata de una $\salgb$
\end{problem}

\begin{problem}[5]
Dada una función $\appl{g}{X}{Y}$ y sea $\algb{B}$ una $\salgb$ de Y, demostrar que:
\[\algb{A}=\{g^{-1}(E) \tq E \in \algb{B}\}\]
es una $\salgb$

\solution
Vamos a comprobar las propiedades de cierre por unión y por complementación:
\begin{enumerate}
\item. Debemos ver que:
\[A_1, A_2\in \algb{A} \Rightarrow A_1 \cup A_2 \in A\]
Lo cual es cierto ya que, como demostramos en el ejercicio anterior:
\[g^{-1}(E_1 \cup E_2)=g^{-1}(E_1) \cup g^{-1}(E_2)\]
Si tomamos $A_i=g^{-1}(E_i)$ queda claro que:
\[A_1 \cup A_2 = g^{-1}(E_1) \cup g^{-1}(E_2) = g^{-1}(E_1\cup E_2) \Rightarrow g^{-1}(E)\]
para algún $E\in \algb{B}$ por ser esta una $\salgb$.

\item Ahora tenemos que ver que:
\[A \in \algb{A} \Rightarrow X\setminus A \in \algb{A}\]
Con la segunda propiedad del ejercicio anterior queda claro que es cierto
\end{enumerate}

Por tanto, $\algb{A}$ cumple todas las propiedades de $\salgb$.

\obs En este caso resultaba obvio ver que $\algbA$ contenía al vacío y al total por lo que ni nos hemos molestado. Si no estuviese tan claro habría que comprobarlo al igual que las otras propiedades.

\end{problem}

\begin{problem}[4-5Bis]
\textbf{Inventado por el profesor.}

Vamos a ver que la imagen directa de una $\salgb$ no tiene por qué ser una $\salgb$, es decir:
dada una función $\appl{f}{X}{Y}$ y sea $\algb{A}$ una $\salgb$ de X, demostrar que:
\[\algb{B}=\{g(E) \tq E \in \algb{A}\}\]
no es necesariamente una $\salgb$

\solution
Esto es sencillo puesto que si la función $f$ no es suprayectiva, entonces $Y \neq g(E)$ para cualquier $E$, por lo que $\algb{B}$ no contiene al total y, por tanto no sería $\salgb$.

Supongamos ahora que  la función $f$ es suprayectiva. En este caso seguiríamos teniendo problemas si $f$ no es inyectiva.

Tomemos por ejemplo $g(x)=x^2$. En este caso, $g((-\infty, 0] \cup [0, \infty))=g(0)=0 \neq g((-\infty, 0]) \cup g([0, \infty))$.
\end{problem}

\begin{problem}[6]
Demuestra que una álgebra $\algb{A}$ es una $\salgb$ si y sólo si es cerrada para uniones numerables crecientes.

\solution
Vamos a demostrar las dos direcciones de la implicación:
\begin{itemize}
\item $\Rightarrow$
Es obvio ya que una $\salgb$ es un álgebra cerrada por uniones numerables y por tanto, es cerrada para el caso concreto de uniones numerables crecientes.
\item $\Leftarrow$

Dado un conjunto $\{A_i\}_{i=1}^{\infty}$ tal que $A_i \in \algb{A} \quad \forall i$, construyo otro de la siguiente forma:
\[B_n = \bigcup_{i=1}^{n} A_i\]
de tal forma que $B_i \subset B_j \quad \forall i<j$.

Ahora, por hipótesis, sabemos que la unión de los $B_i$ se contiene en la $\salgb$, luego:
\[\bigcup_{n=1}^{\infty} A_i=\bigcup_{n=1}^{\infty} B_i \in \algb{A}\]\qed

\end{itemize}
\end{problem}

\begin{problem}[7]
Determina el álgebra $\algb{A}$ generada por la colección de los subconjuntos finitos de un conjunto X no-numerable. Determina la $\salgb$ generada por $\algb{A}$. Estudiar el mismo problema en caso de que el conjunto X sea infinito numerable

\solution
Dado el conjunto de todos los subconjuntos finitos, para convertirlo en una $\salgb$ debemos asegurarnos de que sea cerrado por uniones numerables y por complementación.

Para hacerlo cerrado por uniones numerables, debemos incluir todos los conjuntos numerables, ya que cualquiera de estos puede obtenerse como unión de conjuntos finitos.

Además, si queremos que sea cerrado por complementación, tendremos que incluir los complementarios de todos los conjuntos mencionados anteriormente, es decir, debemos incluir todos aquellos conjuntos cuyo complementario sea numerable.

Así, vamos a construir de forma directa la $\salgb$ pedida:
\[\algb{M}(\algb{E})=\{ E\subset X \tq E \text{ numerable o finito, } E^c \text{ numerable o finito}\}\]

Podemos comprobar fácilmente que es una $\salgb$ observando que cumple las propiedades necesarias siguiendo el mismo razonamiento que el realizado para construirla (la unión de numerables o finitos sigue siendo numerable o finita y su complementario cumple la propiedad de tener complementario numerable o finito).

Es la mínima por construcción.

Si X es infinito numerable entonces
\[\algb{M}(\algb{E})=\algb{P}(X)\]
aplicando la misma regla de construcción, ya que todos los subconjuntos de $\algb{P}(X)$ son numerables.
\end{problem}

\begin{problem}[8]
La $\salgb$ de (0,1] engendrada por:
\[\algb{E}= \{(0, \frac{1}{n}]: n=1,2,...\}\]
está formada por uniones finitas o numerables de intervalos (a,b]. Estudia cómo son estos intervalos.

\solution
Queremos ver como son los intervalos (a,b] tales que:
\[\algb{M}(\algb{E})=\bigcup_{i=1}^{\infty}\{(a_i, b_i]\}\]

Vamos a ver cuáles son los elementos que tenemos en $\algb{M}(\algb{E})$. Aquí, además de los propios elementos de $\algb{E}$ tenemos:
\begin{itemize}
\item \textbf{Complementarios} Son de la forma $(\frac{1}{n}, 1]$

\item \textbf{Intersecciones} Son de la forma $(\frac{1}{m}, \frac{1}{n}]$ con m>n

\item \textbf{Uniones} Uniendo dos elementos seguimos estando en $\algb{E}$, así que no ganamos nada nuevo.
\end{itemize}

Puesto que las uniones contienen a los complementarios, tenemos que los intervalos que forman la $\salgb$ son de la forma:
\[\algb{M}(\algb{E})=\bigcup_{i=1}^{\infty}\{(a_i, b_i]: a_i =\frac{1}{m}, \ b_i = \frac{1}{n} \ m>n\}\]
\end{problem}

\begin{problem}[9]
Describe la $\salgb$ generada por:
\[\algb{E}= \{N \subset \nat \tq \forall n \in \nat, \ 2n \in N\}\]

\solution
\textcolor{blue}{Hecho por mi. No fiarse al 100\%}

Vemos que el conjunto $\algb{E}$ está formado por todos los conjuntos que contienen a todos los pares.

Esta claro que la unión y la intersección de conjuntos de $\algb{E}$ pertenecen a $\algb{E}$ ya que contendrán a todos los pares.

Los complementarios serán los subconjuntos de $\nat$ que sólo contengan números impares.

Si combinamos todos estos conjuntos formaremos la mínima $\salgb$ que lo contenga todo. Es decir:
\[\algb{M}(\algb{E})=\{N \subset \nat \tq \text{N no contiene ningún par, o N contiene a todos los pares}\}\]
\end{problem}

\begin{problem}[10]
Sea $\algb{M}$ una $\salgb$ de cardinal infinito. Demuestra que tiene cardinal no numerable.
\solution
Si la $\salgb$ es infinita podemos encontrar una colección numerable y disjunta de elementos de la misma. (Lo podemos hacer tomando una colección numerable y haciéndola disjunta mediante la eliminación en cada elemento de la unión de los anteriores).

Denotamos a esta colección como: $\{A_n:n\in \nat\}$.

Ahora vamos a construir una colección infinita de $A_x$ como sigue:
\[\forall x \in (0,1) \text{ obtenemos el desarrollo en base 2 de } x=0,x_1,x_2,x_3,...\]
\[A_x=\bigcup_{n=1}^{\infty}A'_n \text{ donde } A'_n=\left\{ \begin{array}{lcc}
             A_n &   si  & x_n = 1 \\
             \\ \emptyset &  si  & x_n \neq 1
             \end{array}
   \right.\]

Como no puede darse el caso de dos $A_x$ iguales, es necesario tomarlos todos para construir $\algb{M}$ y puesto que el número de $A_x$ es no numerable (uno por cada real en el intervalo (0,1)), tenemos que el número de elementos de $\algb{M}$ es no numerable.

\end{problem}

\begin{problem}[11]
Hallar una cota superior al número de elementos que puede tener una $\salgb$ $\algb{M}$ generada a partir de un conjunto con n elementos:
\solution
El enunciado nos dice que tenemos un conjunto $ε\subset \algb{P}(X) \tq ε=\{A_1,...A_n\}$

Si el conjunto ε es una partición del conjunto $X$ (división de $X$ en subconjuntos disjuntos dos a dos), entonces la mínima $\salgb$ generada tendrá $2^{\#ε}$ elementos.
\begin{proof}
Vamos a considerar todas las posibles uniones de los elementos de ε, que deberán pertenecer a $\algb{M}(ε)$.
Para poder contar estas uniones, vamos a representar cada unión mediante un número binario de longitud \# ε, donde los 1s nos indican qué elementos de ε estamos considerando.

Queda claro así que el conjunto de todas las uniones posibles tiene $2^{\# ε}$ elementos. Estas uniones serán todas disjuntas puesto que así lo son los elementos que las generan. Además, esto implica que las intersecciones son vacías y el complementario de cada elemento está formado por la unión del resto y, por lo tanto, también se contiene en el conjunto de uniones.

Obviamente esta construcción genera la mínima $\salgb$ generada por el conjunto ε y que tiene el cardinal buscado.
\end{proof}

Pero el caso que nos concierne es ligeramente distinto, pues no son disjuntos.

Si tomamos el primer elemento, podemos construir una partición de $X$ a partir de él, considerando los conjuntos: $A_1 \ A_1^c$.

Cogiendo ahora un segundo conjunto $A_2 \in ε$, tenemos entonces que los cuatro conjuntos: $A_1 \cap A_2, \ A_1^c \cap A_2 \ A_1^c \cap A_2^c \ A_1 \cap A_2^c$, se contendrían en la $\salgb$. Estos cuatro conjuntos podrían ser distintos todos ellos, o podrían coincidir. Como mucho darán cuatro conjuntos distintos y como poco 2.

\textbf{Recordemos que la un álgebra (y por tanto una $\salgb$) es cerrada por intersecciones ya que:}
\[A_1, A_2 \in \algb{A} \implies A_1 \cap A_2 = (A_1^c\cup A_2^c)^c \in \algb{A}\]

Para dejarlo claro, los hemos conseguido considerando las posibles intersecciones del segundo conjunto que hemos cogido y su complementario con los conjuntos que teníamos en el paso anterior.

Tomo ahora un tercer elemento de ε distinto de los anteriores, y construyo todas las combinaciones posibles con los 4 elementos anteriores, de forma similar a como construimos esas mismas combinaciones, siendo $A_1$ cada una de esas intersecciones y $A_2$ nuestro tercer conjunto.

En cada paso tenemos $2^i$ conjuntos disjuntos dos a dos que constituyen una partición de $X$. Cuando lleguemos al final, tendremos la partición que contiene a todos los elementos de ε.

Es decir, tenemos $N=2^n$ elementos (o menos) disjuntos que generan nuestra $\salgb$.

Puesto que ahora tenemos un ε como el que comentamos al inicio del ejercicio, tenemos que:
\[\#\algb{M}(ε)=2^N\]
\end{problem}

\begin{problem}[12]
Se llama $\sigma$-anillo de subconjuntos de un conjunto $X$ a toda familia no vacía $\algb{F}$ de subconjuntos de $X$ cerrada para uniones numerables y para las diferencias. Demuestra que todo $\sigma$-anillo es también cerrado para intersecciones numerables. Demuestra que todos $\sigma$-anillo de $X$ es una $\salgb \iff X \in \algb{F}$
\solution
Vamos a probar que un $\sigma$-anillo es cerrado para intersecciones numerables.
Dada una familia infinita numerable $A_i \in \algb{F}$ sabemos que $\bigcup A_i \in \algb{F}$ y lo denotamos por $A$.
Entonces:
\[A \setminus \bigcap A_i = \bigcup(A\setminus A_i) \in \algb{F} \Rightarrow A\setminus (A \setminus \bigcap A_i)=\bigcap A_i \in \algb{F}\]

La condición que le falta a un $\sigma$-anillo para ser una $\salgb$ es contener al total. Por tanto, si añadimos el total ya estamos forzando el cumplimiento de la última condición.
\end{problem}

\begin{problem}[13]
Se llama ``clase monótona'' de un conjunto $X$ a toda familia no vacía $\algb{M}$ de subconjuntos de $X$ que sea cerrada para las uniones crecientes y para las intersecciones decrecientes (es decir, si $\forall i=1,2,.. \ C_i \in \algb{M}$ y $C_i \subset C_{i+1}$ o $C_i \supset C_{i+1}$ entonces $\cup_iC_i \in \algb{M}$ o $\cap_iC_i \in \algb{M}$, respectivamente).

Demuestra que toda $\salgb$ es clase monótona. Da un ejemplo de una clase monótona que no sea $\salgb$
\solution

\begin{defn}[Clase monótona]
$\algb{M} \subset \algb{P}(X)$ es clase monótona si para toda sucesión $\{A_i\}$ de elementos de $\algb{M}$ se cumple que:
\begin{enumerate}
\item \[\forall i A_i \subset A_{i+1} \Rightarrow \bigcup_{n=1}^{\infty}A_i \in \algb{M}\]
\item \[\forall i A_i \supset A_{i+1} \Rightarrow \bigcap_{n=1}^{\infty}A_i \in \algb{M}\]
\end{enumerate}
\end{defn}
Basta con que probemos la primera propiedad ya que la segunda sale por complementación.

Es obvio que toda $\salgb$ es clase monótona, puesto que una $\salgb$ es cerrada para uniones numerables y, en concreto, lo será para uniones numerables crecientes.

La parte interesante del ejercicio es la que sigue.

Vamos a buscar el ejemplo de clase monótona que no sea $\salgb$.
Para el ejemplo basta con encontrar una cadena finita de conjuntos crecientes lo que nos garantiza el cumplimiento de las propiedades de una clase monótona, pero dista mucho de ser una $\salgb$.

Un ejemplo concreto sería:
\[\nat \supset \{2n, \ \forall n \in \nat\}\supset \{4n, \ \forall n \in \nat\}\supset \{8n, \ \forall n \in \nat\}\supset \{16n, \ \forall n \in \nat\}\supset \emptyset\]
La clase monótona sería una subclase de $\algb{P}(\nat)$ formada por todos los conjuntos aquí descritos y no es una $\salgb$, ya que no es cerrada por complementación.
\end{problem}

\begin{problem}[14]
Demuestra que la mínima clase monótona que contiene un álgebra dada $\algb{A}$ es también una $\salgb$

\solution
\textcolor{blue}{Atención, ejercicio clave}
\begin{center}
\includegraphics[scale=0.5]{img/clave.jpg}
\end{center}

Vamos a observar dos cosas:
\begin{enumerate}
\item Existe al menos una clase monótona que contiene a un conjunto $\algb{E}$, que es $\algb{P}(X)$
\item La intersección de clases monótonas es clase monótona. Demostración trivial.
\end{enumerate}

Tras estas observaciones podemos hablar de la mínima clase monótona que contiene a $\algb{E}$ como la intersección de todas las clases monótonas que lo contienen y sabemos que la intersección no es vacía por que hay al menos una.

Vamos a tener que hacer dos observaciones más:
\begin{enumerate}
\item Si una clase de conjuntos es álgebra y clase monótona, entonces es $\salgb$.
\item Sea $\algb{A}$ es un álgebra y $\algb{C}$ la mínima clase monótona que contiene a $\algb{A}$, entonces $\algb{C}$ es un álgebra.
\end{enumerate}

%\textcolor{red}{Hecho por mi. No fiarse al 100\%}
% Rehecho en clase y corregidos errores
\begin{proof}
\begin{enumerate}
\item Si nos apoyamos en el ejercicio 6, vemos clara esta demostración. Por ser clase monótona será cerrada para uniones numerables crecientes. Apoyándonos en el ejercicio 6, tenemos que un álgebra cerrada por uniones crecientes numerables es una $\salgb$

\item Vamos a ver que decir que un conjunto $\algb{C}$ es un álgebra es equivalente a decir:
\[\forall E,F \in \algb{C}, E\cap F, \ E\setminus F, \ F\setminus E \in \algb{C}\]

Vamos a definir:
\[\forall E \in \algb{C}, \ \algb{C}(E)=\{F \in \algb{C}: \  E\cap F, \ E\setminus F, \ F\setminus E \in \algb{C}\} \]

Vamos a comprobar que $\algb{C}(E)$ es una clase monótona comprobando las dos propiedades que definen una clase monótona:
\begin{itemize}
\item Dada una sucesión de conjuntos $\{F_i\} \in \algb{C}(E)$ tales que $F_i \subset F_{i+1}$ vemos que:
\ppart \[(\cup F_i)\setminus E = \cup (F_i \setminus E) \in \algb{C}\]
\ppart \[E \setminus (\cup F_i) = \cap(E \setminus F_i) \in \algb{C}\]
\ppart \[E \cap (\cup F_i) = \cup (E \cap F_i) \in \algb{C}\]

De donde podemos deducir que
\[F_i \in \algb{C}\]

\item Tomemos ahora una sucesión de conjuntos $\{F_i\}\in \algb{C}(E)$ tales que $F_i \supset F_{i+1}$ vemos que:
\ppart \[(\cap F_i)\setminus E = \cap (F_i \setminus E) \in \algb{C}\]
\ppart \[E \setminus (\cap F_i) = \cup(E\setminus F_i) \in \algb{C}\]
\ppart \[E \cap (\cap F_i) = \cap (E \cap F_i)\in \algb{C}\]

De donde podemos deducir que, en este caso:
\[F_i \in \algb{C}\]
\end{itemize}

Además esta claro que:
\[\forall E,F \in \algb{C}, \quad E\in \algb{C}(F) \iff F \in \algb{C}(E)\]

Vamos a ver ahora que $\algb{C} \subset \algb{C}(E) \quad \forall E \in \algb{A}$

Para probarlo, basta con ver que $\forall E \in \algb{A}, \ \algb{C}(E)$ es monótona, $\algb{A} \subseteq \algb{C}(E)$ y que $\algb{C}$ es la mínima clase monótona que contiene a $\algb{A} \implies \algb{C} \subseteq \algb{C}(E) $.

\end{enumerate}
\end{proof}
\end{problem}

\section{Hoja 3}
A lo largo de esta hoja se piden realizar varias demostraciones. La terminología que ha establecido Patricio en estos ejercicios consiste en llamar comprobaciones a los ejercicios mas triviales, pruebas a los del un nivel intermedio y demostraciones a los que realmente requieren esfuerzo.

\begin{problem}
Sea $(X, \algbM, µ)$ un espacio de medida. Si $E,F \in \algbM$ comprueba que
\[µ(E)+µ(F)=µ(E\cup F) + µ (E\cap F)\]

\solution
En el enunciado pide comprobar por lo que, según Patricio, debe ser bastante trivial.

En este caso tiene razón, ya que lo ocurre es conceptualmente muy sencillo. Al sumar $µ(E)+µ(F)$ la intersección la estoy teniendo en cuenta dos veces (una al medir $E$ y otra al medir $F$).

Para escribir una demostración formal observemos que:
\[E \cup F = (E \setminus F) \cup (F\setminus E) \cup (E \cap F)\]
Puesto que se trata de una unión disjunta, y µ es una medida, si tomamos la medida de todo ello obtenemos:
\[µ(E \cup F) = µ(E \setminus F) + µ(F\setminus E) + µ(E \cap F) = µ(E)-µ(F\cap E)+µ(F)-µ(E \cap F)+µ(E \cap F) =\]
\[= µ(E) + µ (F) +µ (E \cap F)\]
\end{problem}

\begin{problem}
Sea $(X, \algbM, µ)$ un espacio de medida y sea $E \in \algbM$. Para cada $A \in \algbM$ sea $µ_E(A)=µ(E\cap A)$.

Comprueba que $µ_E$ es una medida en $\algbM$
\solution
Si recordamos la definición de medida en un espacio medible, vemos que se trata de una aplicación de los subconjuntos del espacio a los reales que cumple dos propiedades básicas:
\begin{enumerate}
\item La medida del vacío es 0
\item La medida de una unión de conjuntos disjuntos es la suma de las medidas de cada uno de los conjuntos
\end{enumerate}

Si queremos probar que $µ_E$ es una medida deberemos comprobar que se satisfacen esas dos condiciones, es decir:
\[µ_E(\emptyset) = 0\]
\[µ_E(\bigcup A_n) = \sum µ_E(A_n)\]

La primera propiedad es trivial, ya que, por ser µ una medida tenemos:
\[µ_E(\emptyset)=µ(E \cap \emptyset)=µ(\emptyset)=0\]

Para la segunda propiedad vemos que:
\[µ_E(\bigcup A_n)=µ(E \cap (\bigcup A_n)) = µ(\bigcup(A_n \cap E)) \text{ con } A_i\cap A_j = \emptyset \ \forall i\neq j\]
Puesto que los $A_n \cap E$ son disjuntos y $µ$ es una medida sobre $\algbM$ tenemos que:
\[µ(\bigcup(A_n \cap E))=\sum µ(E \cap A_n) = \sum µ_E(A_n)\]

\end{problem}

\begin{problem}
\begin{enumerate}
\item Comprueba que una medida $\sfin$ es semifinita
\item Sea $X$ un conjunto no numerable y sea µ la medida discreta en $(X, \algbP (X))$, comprueba que µ es  semifinita pero no $\sfin$
\end{enumerate}

\solution
\begin{enumerate}
\item
Sabemos que, por definición de $\sfin$, el conjunto $X$ (sobre el que está definida la medida) puede expresarse como una unión de elementos de la $\salgb$ de medida finita. Es decir:
\[\exists \{E_n\}_{n \in \nat} \subset \algbM \tq X = \bigcup E_n \text{ con } µ(E_n)< \infty \forall n\]

Y para ver que µ es semifinita, por definición, debemos ver que todo subconjunto $E$ de la $\salgb$ tiene un subconjunto contenido también en la $\salgb$ de medida finita. Es decir:
\[\forall E \subset \algbM \text{ con } µ(E)>0, \ \exists A \in \algbM \text{ con } A\neq \emptyset \ A \subset E \text{ y } 0<µ(A) < \infty\]

Puesto que $X$ puede expresarse como unión de los $E_n$ está claro que:
\[E= \bigcup (E \cap E_n)\]
y puesto que no podemos garantizar que sean disjuntos los elementos de la unión tenemos:
\[µ(E)=µ(\bigcup (E \cap E_n))\leq \sum(µ(E \cap E_n))\]

Entonces
%TODO WTF?
\[\exists n \tq 0<µ(E \cap E_n) < \infty\]
Y basta con tomar $A=E \cap E_n \subset E$

\item
Recordemos que la medida discreta es la que, aplicada sobre un conjunto, nos devuelve el cardinal del conjunto si este es finito o infinito en caso contrario.

Para ver que es semifinita nos fijamos en que
\[µ(E) > 0 \implies E \neq \emptyset \implies \exists a \in E\]
En ese caso
\[\{a\}\subset E \text{ con } µ(\{a\})=1\]
Por lo que es semifinita
%TODO WTF?

Ahora vamos a comprobar que no es $\sfin$ por reducción al absurdo.

Si fuese $\sfin$,
\[\exists \{E_n\}_{n \in \nat} \subset \algbM \tq X = \bigcup E_n \text{ con } µ(E_n)< \infty \forall n\]
es decir, el cardinal de $E_n$ sería finito  y $X= \bigcup_{n \in \nat }E_n$ lo que implicaría que $X$ es numerable, en contradicción la hipótesis inicial.

\end{enumerate}
\end{problem}

\begin{problem}
Sea $X$ un conjunto infinito numerable, para $A \subset X$ se define
\[µ(A)=\left\{ \begin{array}{lcc}
             0 &   si  & A \text{ es finito} \\
             \\ \infty &  si  & A \text{ es infinito}
             \end{array}
   \right.\]

\begin{enumerate}
\item Comprueba que µ es finitamente aditiva (en $(X, \algbP (X))$) pero no numerable aditiva
\item Comprueba que existe una sucesión creciente de conjuntos $\{A_n\}$ tal que:
\[\forall n \in \nat, \ µ(A_n)=0 \ y \ \lim_{n} A_n =X\]
\end{enumerate}
\solution
\begin{enumerate}
\item Comprobar que es numerablemente aditiva implica comprobar que la medida de la unión de conjuntos disjuntos es la suma de las medidas.

Vamos a comprobarlo por casos:
Tomemos una sucesión finita y disjunta de $A_n$.
\begin{enumerate}
\item Si todos son finitos las medidas serán 0, la unión será finita y también tendrá medida 0.

\item Si al menos uno es infinito, entonces la medida sería infinita y la suma de medidas será infinito también.

\item Sin embargo, si tomo una sucesión numerable de conjuntos finitos, la medida de cada uno de ellos sería 0 pero la unión sería infinita por lo que tendría medida infinita.

Para obtener elegantemente estos conjuntos podemos tomar una numeración de los elementos de X (ya que es numerable). Es decir, consideramos:
\[X= \bigcup_{n=0}^{\infty}A_n \text{ donde } A_n=\{a_n\}\subset X\]
En este caso µ(X) es infinita pero a la izquierda tendríamos una suma de 0s ($µ(A_n)$) que nos daría 0
\end{enumerate}
Queda claro que es finitamente aditiva pero no numerablemente.

\item Vamos a definir la sucesión creciente:
\[A_n = \bigcup_{i=0}^{n}\{a_n\} \ a_n \in X\]
Puesto que cada $A_n$ será finito, tendrá medida 0 y la sucesión converge a $X$
\end{enumerate}

\end{problem}

\begin{problem} Sean $(X, \algbM, μ)$ un espacio de medida y $\{A_n\}_{n∈ℕ}$ una sucesión de conjuntos medibles. Si $A= \bigcup_{j=0}^∞ A_j$, prueba que \[ μ(A) = \lim_{n\to∞} μ\left(\bigcup_{j=0}^n A_j\right)\]
\solution

Vamos a demostrarlo viendo primero que \[ μ(A) ≥ \lim_{n\to∞} μ\left(\bigcup_{j=0}^n A_j\right) \] y es que es fácil ver que $\bigcup_{j=0}^n A_j ⊆ A \; ∀n$, luego \[ μ(A) ≥ μ\left(\bigcup_{j=0}^n A_j\right)\quad ∀n \]

Ahora bien, ¿puede ser esa desigualdad estrictamente mayor cuando $n\to∞$? Si lo fuera,
\[ μ(A) > μ\left(\bigcup_{j=0}^∞ A_j\right) \]
entonces $\bigcup_{j=0}^∞ A_j \subsetneq A$, lo que implica una contradicción.
\end{problem}

\begin{problem}[6]
Sea ($X, \algbM, µ$) un espacio de medida y $\{E_n\} $ una sucesión de conjuntos en $\algbM$. Demuestra que si existe un k tal que $µ(\bigcup_{i=k}^{\infty}E_i) < \infty$ entonces:
\[µ(\liminf (E_j))<\liminf (µ(E_j)))\]
\[µ(\limsup (E_j))>\limsup (µ(E_j)))\]

En particular si µ(X) < $\infty$ entonces:
\begin{enumerate}
\item
\[µ(\liminf E_j) \leq \liminf µ(E_j) \leq \limsup µ(E_j) \leq µ(\limsup(E_j))\]
\item Si existe $\lim E_j$ entonces $µ(\lim E_j)=\lim µ(E_j)$.
\end{enumerate}
Indica en qué punto es necesaria la condición $µ(\bigcup_{j=k}^{\infty}E_j)< \infty$ para al menos un k.
\solution
Recordemos algunas 'definiciones' necesarias para resolver el ejercicio:
\[\liminf (E_k) = \bigcup_{j=0}^{\infty} \bigcap_{k=j}^{\infty} E_k\]
\[\limsup (E_k)= \bigcap_{j=0}^{\infty} \bigcup_{k=j}^{\infty} E_k\]
Basándonos en la \href{http://en.wikipedia.org/wiki/Limit_superior_and_limit_inferior}{Wikipedia}, daremos una definición más intuitiva de estos conceptos:

\begin{defn}[Límite\IS inferior]
Es el conjunto formado por los elementos que pertenecen a todos los conjuntos de la sucesión salvo, quizás, a un número finito de ellos.

Es decir, se contienen en todos los conjuntos de la sucesión a partir de un cierto n
\end{defn}

\begin{defn}[Límite\IS superior]
Es el conjunto formado por todos los elementos que pertenecen a infinitos conjuntos en la sucesión
\end{defn}

\textcolor{red}{Parra: Ejemplos:\\
-(1, 2, 3, 1, 2, 3, 1, 2, 3, 1, 2, 3,...) : en esta sucesión existe una subsucesión (1, 1, 1,...) con límite 1, otra (2, 2, 2, 2, 2,...) con limite 2 y otra con limite 3. El límite inferior seria 1 y el superior 3.\\
-(0, 3, $\frac{1}{2}$, $\frac{5}{2}$, $\frac{3}{4}$, $\frac{9}{4}$, $\frac{4}{5}$, $\frac{11}{5}$, $\frac{5}{6}$, $\frac{13}{6}$...) Existe una subsucesión creciente que llega a 1 (0, $\frac{1}{2}$, $\frac{3}{4}$, $\frac{4}{5}$, $\frac{5}{6}$,...) y otra decreciente que llega a 2 (3, $\frac{5}{2}$, $\frac{9}{4}$, $\frac{11}{5}$, $\frac{13}{6}$...). El limite inferior seria 1 y el superior 2.\\
-Si el límite inferior y superior coinciden se dice que la sucesión converge.}

Además, como ya vimos, si una sucesión es creciente el límite coincide con el límite de la unión. Igual ocurre con una sucesión decreciente y la intersección. Es decir:
\[E_i \subset E_{i+1} \implies \lim E_i = \bigcup E_i\]
\[E_i \supset E_{i+1} \implies \lim E_i = \bigcap E_i\]


Vamos ya a por el ejercicio.

Si tomamos un elemento $E_k$ y lo intersecamos con uno o varios elementos de la sucesión obtendremos un subconjunto de $E_k$, es decir:
\[E_k \supset \bigcap_{j=k}^{\infty}E_j\]
Por tanto, si aplicamos medidas a ambos lados tenemos:
\[µ(E_k) < µ(\bigcap_{j=k}^{\infty}E_j) \rightarrow µ(\bigcup_{k=0}^{\infty}\bigcap_{j=k}^{\infty}E_j)\]

La convergencia se debe a que la intersección será más grande (estoy tomando menos conjuntos) por lo que al final me quedarán aquellos elementos que estén en todos los conjuntos salvo en un número finito de ellos. (Está en todos los conjuntos salvo, quizás, en los que he ido quitando)

De esto, y basándonos en las definiciones iniciales, podemos concluir:
\[\lim \inf µ(E_k) \geq \lim \inf µ(\bigcap_{j=k}^{\infty}E_j) = \lim (µ(\bigcap_{j=k}^{\infty}E_k))=µ(\bigcup_{k=0}^{\infty}\bigcap_{j=k}^{\infty}E_j)\]
%TODO No veo claro el por qué

Ahora bien, tenemos una sucesión de conjuntos de la que calculamos el límite superior y el límite inferior mediante las dos definiciones proporcionadas al inicio del ejercicio.

Si tiramos los n primeros conjuntos y nos quedamos con los demás, el límite superior y el límite inferior siguen siendo los mismos.

Analicemos un poco más en detalle esta afirmación:
\[x \in \lim \sup (E_i)_{i=0}^{\infty} \text{ si } \forall k \exists k' \geq k \tq x \in E_{k'}\]
%TODO Ha copiado otro par de definiciones que yo no tengo. apañarlo.

Una vez queda claro que el límite superior y el inferior no dependen de los n primeros elementos de la familia, está claro que, sin pérdida de generalidad, podemos demostrar lo que pide el enunciado suponiendo que k=0.

Tenemos pues:
\[µ(\bigcup_{i=0}^{\infty} E_i)< \infty\]
Basándonos en las definiciones iniciales, vemos que lo que nos pide demostrar el enunciado es equivalente a:
\[µ(\bigcap_{k=0}^{\infty}\bigcup_{j=k}^{\infty}E_j)= µ(E)-µ((\bigcap_{k=0}^{\infty}\bigcup_{j=k}^{\infty}E_j)^c)\]
siendo $E=\bigcup_{j=0}^{\infty}E_j$ y hablando del complementario como el complementario dentro de $E$

Pero si lo analizamos vemos que lo que tenemos es:
\[µ(\bigcup_{k=0}^{\infty}\bigcap_{j=k}^{\infty}E_j)=\]
\[=µ(E)-\liminf µ(\bigcup_{k=0}^{\infty}\bigcap_{j=k}^{\infty}E^c_j) \geq \]
\[\geq µ(E)-\liminf(µ(E)-µ(E_k))=µ(E)-µ(E)+\limsup µ(E_k)\]

Tomando el principio y el final de esta cadena de desigualdades llegamos a lo que queríamos demostrar:
\[µ(\bigcup_{k=0}^{\infty}\bigcap_{j=k}^{\infty}E_j)\geq\limsup µ(E_k)\]
\end{problem}

\begin{problem}[7]
Sea $X=\{a_1, a_2, a_3\}$ y µ una medida definida en $\algbP (X)$ tal que $µ(A_i)=\frac{1}{3} \ \forall i$.

Se define una sucesión $A_n$ tal que $A_{2k}=\{a_1, a_2\}$ y $A_{2k+1}=\{a_3\}$ $\forall k \in \nat$.

Prueba que:
\[µ(\liminf A_n) < \liminf µ(A_n) < \limsup µ(A_n) < µ(\limsup(A_n))\]
\solution
En este caso está bastante claro que:
\[\liminf A_n = \emptyset\]
puesto que este límite consiste en tomar la unión de todos los elementos que están presentes en todos los elementos de la sucesión. No hay ningún elemento que esté en todos los elementos y por ello obtenemos el vacío.
Así mismo
\[\limsup A_n = X\]
ya que este límite implica tomar la intersección de conjuntos formados por los elementos contenidos en algún elemento de la sucesión. Todos los elementos de $X$ están contenidos en algún elemento de la sucesión y la intersección de los $X$ con ellos mismo es $X$.

Por otro lado:
\[\limsup µ(A_n) = \frac{2}{3}\]
\[\liminf µ(A_n) = \frac{1}{3}\]

Puesto que $µ(X) =1$ y $µ(\emptyset)=0$ obtenemos que es claramente correcta la cadena de desigualdades.
\end{problem}


\begin{problem}[8]
Sean $X=\nat, \ \algbM=\algbP(\nat)$ y µ la medida discreta.

Construye una sucesión $(A_n), \ A_n\subset \algbP(\nat) \tq \lim_n A_n=\emptyset$ y $\lim_n µ(A_n) \neq 0$
\solution
\textcolor{blue}{Hecho por mi. No fiarse al 100 \%}

Tomemos la sucesión $(A_n)$ definida como
\[A_i = \{i, i+1, i+2,...\}\]
todos los conjuntos tienen medida infinita por lo que el límite de las medidas es distinto de 0 pero en el infinito el conjunto será vacío.
\end{problem}


\begin{problem}[9]
Sea µ una medida semifinita y sea $E$ tal que $µ(E) = \infty$. Prueba que si c es un número real mayor que 0, existe un conjunto $F \subset E$ tal que c < $µ(F)$ < $\infty$

\solution
Basándonos en la sugerencia del enunciado, tomemos:
\[k=\sup \{µ(F): \ F \subset E, \ µ(F)< \infty \}\]

Por ser k un supremo,
\[\forall n \exists F_n \subset E \tq k-\frac{1}{n} \leq µ(F_n) \leq k\]
pero
\[k - \frac{1}{n} \leq µ(\bigcup_{n=1}^{N} F_n) \leq k\]
por lo que si tomamos un n suficientemente grande llegaremos a la igualdad:
\[k \leq µ(\bigcup_{n=1}^{\infty}F_n) \leq k\]
Ahora podemos construir una sucesión creciente de $F_n$ tales que la unión de todos ellos es F. Es decir:
\[F= \bigcup F_n\]
y como la medida de $F$ es finita sabemos que:
\[µ(E \setminus F)= \infty\]
y por tanto
\[\exists F'\subset (E \setminus F) \tq µ(F')> 0\]

Entonces
\[F \cup F' \subset E \Rightarrow µ(F \cup F')=µ(F)+µ(F') > k\]
Y llegamos a una contradicción porque k era el supremo.


\end{problem}

\begin{problem}[10]
En un espacio de medida $(X, \algbM, µ)$ sea $\{A_i\}$ una familia de conjuntos medibles tales que $\sum µ(A_i) < \infty$.

Prueba que casi todo elemento $x \in X$ pertenece sólo a un número finito de $A_i$

\solution

Recordemos que el conjunto de los $x \in X$ que pertenecían a un número infinito de $A_i \subset X$ lo denominamos límite superior.

\[\limsup = \bigcap_{n=1}^{\infty}\bigcup_{k=n}^{\infty} A_k\]

\textbf{Si casi todo elemento cumple una propiedad, los que no la cumplen constituyen un conjunto de medida 0}

Vemos ahora que pasa con su medida:
\[µ(\limsup) = µ(\bigcap_{n=1}^{\infty}\bigcup_{k=n}^{\infty} A_k) \leq \sum_{k=n}^{\infty}µ(A_n) \rightarrow 0\]

La última convergencia aquí indicada se deduce de la finitud de la serie. Si una serie es finita, su cola converge a 0, ya que en caso contrario la serie crecería hasta el infinito.
\end{problem}

\begin{problem}[11]
Sea $(X_1 \algbM_1,µ_1)$ un espacio de medida completo. Sean $\appl{g}{X_1}{X_2}$ una aplicación, $\algbM_2=\{A \subset X_2 \tq g^{-1}(A) \in \algbM_1\}$, y $µ_2(A)=µ_1(g^{-1}(A)$.

Comprueba que $(X_2 \algbM_2,µ_2)$ es un espacio de medida completo.

\solution
\textcolor{blue}{Hecho por mi. No fiarse al 100\%}

Para comprobar que $(X_2 \algbM_2,µ_2)$ es un espacio de medida completo debemos comprobar que $\algbM_2$ es una $\salgb$ en $X_2$ y que $µ_2$ es una medida. Vamos a ello.

El ejercicio 4 de la hoja 2 nos permite afirmar que $\algbM_2$ es una $\salgb$ así que no hay nada que añadir.

Para comprobar que $µ_2$ es una medida debemos ver que la medida del vacío es 0 y la medida de una unión disjunta de conjuntos es la suma de las medidas.

\textbf{Medida del vacío}
\[µ_2(\emptyset) = µ_1(g^{-1}(\emptyset) = µ_1(\emptyset) = 0\]

\textbf{Numerablemente aditiva}
\[µ_2(\bigcup A_n) =  µ_1(g^{-1}(\bigcup A_n)) = µ_1(\bigcup g^{-1}(A_n)) = \sum µ_1(g^{-1}(A_n)) = \sum µ_2(A_n)\]
\end{problem}

\begin{problem}[12]
Sea $X$ un conjunto cualquiera. Se define $\appl{µ^*}{\algbP(X)}{[0,1]}$ mediante $µ^*(\emptyset)=0$, $µ^*(A)=1$, si $A \neq \emptyset$, $A\subset X$.

Comprueba que $µ^*$ es una medida exterior. Determina la $\salgb$ de los conjuntos medibles.

\solution
\textcolor{blue}{Hecho por mi. No fiarse al 100\%}

Para comprobar que se trata de una medida exterior debemos verificar que se cumplen las siguientes propiedades:
\begin{enumerate}
\item
\[µ^*(\emptyset)=0\]
Cierto por definición.
\item
\[A \subset B \implies µ^*(A)\leq µ^*(B)\]
Cierto porque tendremos las desigualdades 0$\leq$0, 0$\leq$1, 1$\leq$1; según sean ambos el vacío, sólo sea $A$ el vacío o ninguno sea vacío.
\item
\[µ^*(\bigcup A_n) \leq \sum µ^*(A_n)\]
Cierto pues a la izquierda sólo podremos tener, como mucho, 1 y a la derecha tendremos una suma mayor.
\end{enumerate}

Para la segunda parte del ejercicio vamos a basarnos en el teorema de Caratheodory, que nos dice que teniendo una medida exterior, la $\salgb$ de los conjuntos medibles es:
\[\algbM = \{A \subset X: \ \forall E \subset X \ µ^*(E)=µ^*(E \cap A) + µ^*(E \cap A^c)\}\]

Cuando el conjunto $E$ sea vacío no habrá ningún problema. Cuando no lo sea, su medida exterior será 1 y para que se cumpla la igualdad necesitaremos que $A$ se contenga en él, o en su complementario (de lo contrario la suma quedaría 1+1=2).

Por desgracia esto sólo ocurre con el vacío y el total, por lo que
\[\algbM = \{\emptyset, X\}\]
\end{problem}

\begin{problem}[13]
Sea $X$ un conjunto cualquiera. Se define $µ^*(\emptyset)=0, \ µ^*(X)=2, µ^*(A)=1, \forall A \subset X$

Comprueba que $µ^*$ es una medida exterior. Determina la $\salgb$ de los conjuntos medibles.
\solution
\textcolor{blue}{Hecho por mi. No fiarse al 100\%}

Vamos a repetir los pasos del ejercicio anterior. Primero, comprobemos que es una medida exterior
\begin{enumerate}
\item
\[µ^*(\emptyset)=0\]
Cierto por definición.
\item
\[A \subset B \implies µ^*(A)\leq µ^*(B)\]
Cierto porque tendremos las desigualdades 0$\leq$0, 0$\leq$1, 0$\leq$2, 1$\leq$1, 1$\leq$2, 2$\leq$2. Según sean ambos el vacío; sólo sea $A$ el vacío y $B$ un conjunto cualquiera; $A$ el vacío y $B$ el total; ninguno sea vacío ni el total; $A$ sea un conjunto cualquiera y $B$ el total; o ambos sean el total
\item
\[µ^*(\bigcup A_n) \leq \sum µ^*(A_n)\]
Si todos los $A_n$ son vacíos tendremos un 0 a la izquierda y otro a la derecha.

Si alguno no es vacío pero ninguno es el total tendremos a la izquierda un 1 y a la derecha una suma de 1s y 0s con, al menos, un 1.

Si unos de los $A_n$ es el total, a la izquierda tendremos un 2 y a la derecha una suma de 0s, 1s y 2s, con al menos un 2.
\end{enumerate}

Nos apoyamos de nuevo en el teorema de Caratheodory y vemos que la $\salgb$ de los conjuntos medibles es:
\[\algbM = \{A \subset X: \ \forall E \subset X \ µ^*(E)=µ^*(E \cap A) + µ^*(E \cap A^c)\}\]

Cuando $E$ sea el vacío todo va bien así como cuando sea el total (0=0, 2=1+1 respectivamente) para cualquier conjunto $A$.

Veamos que pasa cuando $E$ es un subconjunto cualquiera distinto del vacío y del total.

En este caso, a la izquierda tendremos un 1 y, como en el ejercicio anterior, necesitamos que $A$ se contenga en $E$ o en su complementario y esto sólo ocurre con el vacío y el total.

Por tanto
\[\algbM = \{\emptyset, X\}\]
\end{problem}

\begin{problem}[15]
Sea $X$ un conjunto no numerable. Sea $\algbM$ la $\salgb$ formada por los conjuntos finitos o numerables y los conjuntos con complementario finito o numerable.
\[µ(E)= \left\{ \begin{array}{lcc}
             card(E) &   si  & E \text{ finito } \\
            \\ \infty &  si  & E \text{ infinito }
             \end{array}
   \right.\]
\begin{enumerate}
\item Demuestra que µ es una medida completa en $\algbM$
\item Estudia la medida $µ^*$ construida a partir de µ y de $\algbM$
\end{enumerate}

\solution
Aunque el enunciado no dice mucho al respecto, Patricio insiste en dejar claro que  $\algbM$ es realmente una $\salgb$ lo cual se ve de manera sencilla, observando que se cumplen las tres propiedades necesarias para ello.

Vamos ahora a por lo que pide el ejercicio
\begin{enumerate}
\item Para ver que se una medida completa debemos ver que:
\begin{enumerate}
\item La medida del vacío es 0, cosa que es obvia pues el cardinal del vacío es 0.

\item Todo subconjunto de un conjunto de medida 0 se contiene en $\algbM$.

Puesto que en $\algbM$ el único conjunto de medida 0 es $\emptyset$, no hay nada que demostrar.
\end{enumerate}
\item La medida exterior construida a partir de µ y de $\algbM$ se define como:
\[µ ^*(E)=\inf\{µ(A) \tq E \subset A \text { con } A \in \algbM\}\]

En este caso, si $E\in\algbM \implies µ^*(E)=card(E)$ si $E$ es finito y $µ^*(E)=\infty$ en caso contrario.

Habría que ver ahora qué ocurre con los conjuntos que no pertenecen a  $\algbM$.

\textcolor{blue}{Hecho por mi. No fiarse al 100\%}

Los conjuntos que no pertenecen a $\algbM$ son aquellos que son infinitos no numerables y cuyo complementario también es infinito no numerable.

Estos conjuntos no estarán contenidos en ningún conjunto finito ni numerable, por lo que sólo podremos estudiarlos como conjuntos contenidos en el total. Por tanto
\[\forall E \subset X \ E \notin \algbM \implies µ^*(E)=\infty\]
\end{enumerate}
\end{problem}

\begin{problem}[16]
Se define $µ^*$ sobre $\algbP (\nat)$ como:
\[µ^*(E) = \frac{n}{n+1} \text{ si } E \text{ es finito }\]
\[µ^*(E) = 1 \text{ si } E \text{ es infinito}\]
siendo n=$|E|$

Demuestra que $µ^*$ es una medida exterior y halla la $\salgb$ de los conjuntos medibles
\solution

Para ver que $µ^*$ es una medida exterior debemos que:
\begin{itemize}
\item $µ^*(\emptyset)= 0$

Es obvio puesto que n=$|\emptyset|$=0.

\item $E \subset E' \implies µ^*(E)<µ^*(E')$

Para comprobar esta propiedad vamos a ver las diferentes posibilidades:
\begin{itemize}
\item E y E' finitos.

En este caso tenemos n=$|E|$ y m=$|E'|$ con m>n.

Es obvio entonces que:
\[\frac{n}{n+1} < \frac{m}{m+1}\]

\item E es finito y E' es infinito.

En este caso también vemos a simple vista que 1 es mayor que cualquier fracción de la forma
\[\frac{n}{n+1} \text{ con } n = |E|\]

\item E y E' infinitos
En este caso tendríamos la desigualdad:
\[1 \leq 1\]
que, evidentemente, es cierta
\end{itemize}

\item $µ^*(\bigcup A_n) \leq \sum µ^*(A_n)$

Como en el apartado anterior, podemos verlo por casos.
\end{itemize}

Vamos a ver ahora cuál es la $\salgb$ de los conjuntos medibles.
\[\algbM^*=\{E \subset \nat \ \forall A \subset \nat \ µ^*(A)\geq µ^*(A \cap E) + µ^*(A \cap E^c)\}\]
Tomamos la igualdad porque la desigualdad contraria está garantizada siempre, de modo que forzar que se cumpla la desigualdad implica el cumplimiento de esta igualdad (que es como se define la $\salgb$ de los conjuntos medibles).

Para que un conjunto $E$ pertenezca a esta $\salgb$ es imprescindible que la parte de la derecha de la desigualdad sea menor que 1.

Sin embargo, tanto tomando $E$ finito como infinito nos topamos siempre con contradicción. Por tanto este álgebra sólo contiene el vacío y el total.
\end{problem}

\begin{problem}[17]
Dada una medida exterior $µ^*$ en $X$ y $\{A_j\}_{j \in \nat}$ una familia disjunta de conjuntos $µ^*-$medibles.

Demuestra que para cualquier $E \subset X$, se cumple que:
\[µ^*(E \bigcap (\bigcup_{j=0}^{\infty} A_i))= \sum_{j=0}^{\infty}µ^*(E \bigcap A_j)\]
\solution
\textcolor{blue}{Hecho por mi. No fiarse al 100\%}

Sabemos que, por ser los $A_j$ $µ^*$-medibles, se cumple que
\[\forall B \subset X \ µ^*(B)=µ^*(B \cap A_j) + µ^*(B \cap A_j^c)\]
Si tomamos $B=E \cap \bigcup_{j=0}^{\infty}A_j$ y fijamos $A_j=0$ tenemos
\[µ^*(E \cap \bigcup_{j=0}^{\infty}A_j)=µ^*(E \cap \bigcup_{j=1}^{\infty}A_j \cap A_0) + µ^*(E \cap \bigcup_{j=0}^{\infty}A_j \cap A_0^c)\]
puesto que los $A_j$ son disjuntos tenemos
\[µ^*(E \cap \bigcup_{j=0}^{\infty}A_j)=µ^*(E \cap A_0) + µ^*(E \cap \bigcup_{j=1}^{\infty}A_j)\]
Repetimos el proceso tomando ahora $B=E \cap \bigcup_{j=1}^{\infty}A_j$ y fijando $A_j=1$ llegando a
\[µ^*(E \cap \bigcup_{j=0}^{\infty}A_j)=µ^*(E \cap A_0) + µ^*(E \cap \bigcup_{j=1}^{\infty}A_j) = µ^*(E \cap A_0) + µ^*(E \cap A_1) + µ^*(E \cap \bigcup_{j=2}^{\infty}A_j)\]
y por indución llegamos a
\[µ^*(E \bigcap (\bigcup_{j=0}^{\infty} A_i))= \sum_{j=0}^{\infty}µ^*(E \bigcap A_j)\]
\end{problem}

\includepdf[scale=0.9]{pdf/2014-03-18.pdf}
\includepdf[scale=0.9]{pdf/2014-03-19.pdf}
\includepdf[scale=0.9]{pdf/2014-03-20.pdf}

\section{Hoja 4}

\begin{problem}
Sea µ la medida de Lebesgue-Stieltjes asociada a la función:
\[F(x)=\left\{ \begin{array}{lcc}
             0 &   \text{si}  & x < 1 \\
             \\ x & \text{si} & 1 \leq x < 3 \\
             \\ 4 &  \text{si}  & 3 \leq x
             \end{array}
   \right.\]

Calcula las siguientes medidas:
\solution
\begin{itemize}
\item $µ(\{1\}) = F(1)-\displaystyle\lim_{x \to 1^-} F(x)$ = 1
\item $µ(\{2\}) = 0$
\item $µ((1, 3]) = F(3)-F(1) = 4 - 1 = 3$
\item $µ((1, 3)) = µ((1, 3]) - µ(\{3\}) = 3 - 1 = 2$
\item $µ([1, 3)) = µ((1, 3)) + µ(\{1\}) = 2 + 1 = 3$
\item $µ([1, 3])) = µ((1, 3]) + µ (\{1\}) = 3 + 1 = 4$
\end{itemize}
\end{problem}

\begin{problem}
Halla funciones de distribución $F$, $F_1$, $F_2$ de forma que, en cada caso, existan $a$ y $b$ tales que:
\begin{enumerate}
\item $µ((a,b)) < F(b)-F(a) < µ([a,b])$, donde $µ=µ_F$
\item $µ_1((a,b)) < µ_1((a,b]) < µ_1([a,b)) < µ_1([a,b])$ y

$µ_2((a,b)) < µ_2([a,b)) < µ_2((a,b]) < µ_2([a,b])$ donde $µ_i = µ_F, \ i=1,2$
\end{enumerate}
\solution
\begin{enumerate}
\item Vale con la función del ejercicio anterior
\item
\textbf{Con i = 1}
\[F_1(x)=\left\{ \begin{array}{lcc}
             0 &   si  & x < 1 \\
             \\ x & si & 1 \leq x < 3 \\
             \\ 3.5 &  si  & 3 \leq x
             \end{array}
   \right.\]

\textbf{Con i = 2}
\[F_2(x)=\left\{ \begin{array}{lcc}
             0 &   si  & x < 1 \\
             \\ x & si & 1 \leq x < 3 \\
             \\ 5 &  si  & 3 \leq x
             \end{array}
   \right.\]
\end{enumerate}

La construcción de estas funciones se ha realizado por la cuenta de la vieja. Si repetimos los cálculos del ejercicio anterior con estas funciones podemos ver que se cumplen las condiciones pedidas (Incluso puede que entendamos de qué va esto)
\end{problem}

\begin{problem}
Sea µ la medida de contar en $(\real, \algbP(\real))$. Para un conjunto finito A $\subset \real$ se define $µ_A(B) = µ(B \cap A)$ para todo $B \subset \real$

\ppart Sea $A=\{1,2,...,n,...\}$ ¿Es $µ_A$ una medida de Lebesgue-Stieltjes?. En caso afirmativo halla $F$ tal que $µ_A=µ_F$
\ppart Sea $A=\{\frac{1}{1},\frac{1}{2},...,\frac{1}{n},...\}$ ¿Es $µ_A$ una medida de Lebesgue-Stieltjes?. En caso afirmativo halla $F$ tal que $µ_A=µ_F$
\solution


\spart Esta medida simplemente cuenta el número de enteros positivos que hay en un conjunto $B$.

Por tanto, simplemente tenemos que buscar una función que realice esa misma función, por ejemplo:
\[F(x)=\left\{ \begin{array}{lcc}
             0 &   si  & 0 \leq x < 1 \\
             \\ n &  si  & n \leq x < n+1
             \end{array}
   \right.\]

Imitando las cuentas realizadas en el ejercicio 1 podemos que ver:
\begin{itemize}
\item $µ((1,3])) = F(3) - F(1) = 2$
\item $µ((1,3)) = µ((1,3])) - µ(\{3\}) = 2 - 1 = 1$
\end{itemize}
observamos que, efectivamente, obtenemos el número de enteros de cada intervalo.

\spart Esta medida cuenta el número de racionales de la forma $\frac{1}{n}$ que hay en un conjunto.

Esta medida no es de Lebesgue-Stieltjes. Una medida de Lebesgue-Stieltjes tiene que tener asociada una función $F$ como en el apartado anterior, es decir, tal que $μ((a,b]) = F(b) - F(a)$, y esta no puede tenerla.

Veamos por qué: tomemos el intervalo $B = (0,ε)$. Su medida según $μ_A$ es infinita, ya que para todo $n$ mayor que $\frac{1}{ε}$, $\frac{1}{n} ∈ A$, y entonces $B∩A$ es infinito. Luego tenemos que encontrar un $F$ tal que $F(ε) - F(0) = ∞$, y la única forma de que eso tenga sentido es que $F(ε) = ∞\; ∀ε$, lo cual es absurdo.

\end{problem}

\begin{problem}
Sea $F$ la función de distribución
\[F(x)=\left\{ \begin{array}{lcc}
             0 &   si  & x \in (-\infty, -1) \\
             \\ 1+x & si & x \in [-1, 0) \\
             \\ 2+x^2 & si & x \in [0, 2) \\
             \\ 9 &  si  & x \in [2, \infty)
             \end{array}
   \right.\]

Siendo $µ=µ_F$, hallar las siguientes medidas.
\solution
\begin{itemize}
\item $µ(\{2\}) = 3 $
\item $µ([-\frac{1}{2}, 3)) = 9 - \frac{1}{2}$
\item $µ((-1,0]\cup (1,2)) = 2 + µ((1,2]) - µ(\{2\}) = 2 + 9 -3 - 3 = 5$

\item $µ([0, \frac{1}{2}) \cup (1, 2]) = F(\frac{1}{2}) -F(0) + µ(\{0\}) - µ(\{\frac{1}{2}\}) + F(2) - F(1) =
F(\frac{1}{2}) -F(0) + (F(0) - \lim_{x \to 0^-}F(x)) - (F(\frac{1}{2}) - \lim_{x \to \frac{1}{2}^-}F(x)) + F(2) - F(1)
= - \lim_{x \to 0^-}F(x) + \lim_{x \to \frac{1}{2}^-}F(x) + F(2) - F(1) = 1+\frac{9}{4}+9-3=\frac{37}{4}$
\item $µ(\{x \in \real \tq |x|+2x^2 > 1\})$=$µ((-\infty, -\frac{1}{2})) + µ((\frac{1}{2}, \infty)) =\frac{1}{2} + 9 - 0 - 2- \frac{1}{4} = 7 - \frac{1}{4} + \frac{1}{2} = \frac{29}{4}$

\end{itemize}
\end{problem}

\begin{problem}
Sea $\appl{f}{\real}{\real}$ no negativa, integrable Riemann sobre cada intervalo finito y tal que $\int_{-\infty}^{\infty}f(x)=1$.

Prueba que $F(x)=\int_{-\infty}^x f(y) \dif y$ es una función de distribución de probabilidad y que, además, $F$ es continua ($f$ es la función de densidad de $F$).

Si $f(x)=\ind_{[0,1]}$ hallar $F$

\solution
Simplemente tenemos que ver que $F$ es no decreciente, continua por la derecha y que se cumple:
\[\lim_{n \to - \infty}F(x)=0 \ \ \lim_{n \to \infty}F(x)=1\]

Observando que $F$ es una integral y que integrar una función equivale a calcular el área encerrada bajo ella, vemos que a medida que avanzamos la $x$, cada vez estamos calculando un área mayor, luego los dos límites anteriores se cumplen.

Para ver que es continua por la derecha observamos que:
\[ \lim_{h \to 0^+}F(x+h) = \lim_{h \to 0^+} \int_{-\infty}^{x+h}f(y)\dif y = \int_{-\infty}^xf(y)\dif y = F(x)\]

Y es que por ser $F$ una integral, es obvio que es continua.

Suponemos ahora $f(x)=\ind_{[0,1]}$, para responder a la segunda pregunta del enunciado. Entonces
\[F(x)= \int_{-\infty}^{x} \ind_{[0,1]} = \left\{ \begin{array}{lcc}
             0 &   si  & x < 0 \\
             \\ x & si &  0 \leq x \leq 1 \\
             \\ 1 &  si  & 1 \leq x
             \end{array}
   \right.\]
\end{problem}

\begin{problem}
Halla el valor de $k$ para que $f= kx(1-x)\ind_{[0, 1)}$ sea la función de densidad de una medida de probabilidad. Calcula su función de distribución.

\solution
Para que sea función de densidad necesitamos que:
\[\int_{-\infty}^{\infty}f(x) dx = 1\]

Vamos a ver cuánto vale esa integral.
\begin{gather*}
\int_{-\infty}^{\infty}f(x) \dif x = \int_{-\infty}^{\infty}kx(1-x)\ind_{[0, 1)} \dif x  = \\
 = \int_{-\infty}^{0} kx(1-x)\cdot 0 \dif x + \int_{0}^{1}kx(1-x)\dif x + \int_{1}^{\infty}kx(1-x)\cdot 0 \dif x = \\
= k\left(\frac{1}{2}-\frac{1}{3}\right)
\end{gather*}


De donde obtenemos fácilmente que $k = \frac{1}{\frac{1}{2}-\frac{1}{3}} = 6$

La función de distribución sería:
\[\F(x)= \int_{-\infty}^{x} \ind_{[0,1]} = \left\{ \begin{array}{lcc}
             0 &   si  & x < 0 \\
             \\ \int_{0}^{x}kt(1-t)\dif t = (3x^2-2)x^3)& si &  0 \leq x \leq 1 \\
             \\ 1 &  si  & 1 \leq x
             \end{array}
   \right.\]
\end{problem}

\newpage
\begin{problem}
Dado $k > 0$, sea $f(x)=αe^{-kx}\ind_{[0, \infty)}(x)$
\ppart Halla α para que $f$ sea una función de densidad de probabilidad
\ppart Sea $X$ una variable aleatoria con función de densidad $f$, si $k=\frac{1}{2}$, calcula la probabilidad de que $X \geq 3$
\ppart Si $k=\frac{1}{2}$ calcula la probabilidad de que $3 \leq X \leq 6$
\solution

\spart Repitiendo el proceso del ejercicio anterior, debemos hacer que $\int_{0}^{\infty}f(x)dx =1$.

En este caso obtenemos que $α=k$, es decir, nos encontramos ante una exponencial.

\spart
\[\mathbb{P}(X \geq 3) = \int_{3}^{\infty}e^{-kx}\dif x=e^{\frac{3}{2}}\]

\spart Puesto que la función de distribución es continua, la probabilidad de que $X=3$ ó $X=6$ es 0, de modo que podemos calcular la probabilidad pedida como:
\[\mathbb{P}(3 \leq X \leq 6) = \int_{3}^{6}e^{-kx}\dif x\]


\end{problem}

\begin{problem}
Sea µ la medida de probabilidad definida por la función de distribución:

\[F(x)= \int_{-\infty}^{x} \ind_{[0,1]} = \left\{ \begin{array}{lcc}
             0 &   si  & x \in (- \infty, -1) \\
             \\ \frac{1}{3} & si &  x \in [-1, \sqrt{2}) \\
             \\ \frac{1}{2} + \frac{x-\sqrt{2}}{10} & si &  x \in [\sqrt{2}, 5) \\
             \\ 1 &  si  & x \in [5, \infty)
             \end{array}
   \right.\]

Calcular las siguientes medidas:
\solution

Antes de nada deberíamos comprobar que la función $F(x)$ dada es, efectivamente, una función de distribución. Para ello debemos comprobar que siempre es positiva y que se trata de una función creciente.

En este caso nos fiamos y se deja como ejercicio para el lector desconfiado (Edu) la comprobación de estas propiedades.
\newpage
\begin{enumerate}
\item \[µ((\real \setminus \rac)\cap[\sqrt{2}, 5]) = µ([\sqrt{2}, 5)) = µ((\sqrt{2}, 5]) + µ (\{\sqrt{2}\}) - µ (\{\sqrt{5}\}) =\]
\[ = F(5) - F(\sqrt{2}) +(\frac{1}{2}-\frac{1}{3}) -(1-(1-\frac{\sqrt{2}}{10}))=1-\frac{1}{2}+\frac{1}{6}-\frac{\sqrt{2}}{10}\]

\item \[µ((\real \setminus \rac)\cap [-2, \sqrt{2}]) = µ(\{\sqrt{2}\}) = \frac{1}{2}-\frac{1}{3} = \frac{1}{6}\]

\item \[µ(\rac \cap [1,6]) = µ(\{5\}) = \frac{\sqrt{2}}{10}\]
\end{enumerate}

Vamos ahora a por la parte complicada del ejercicio.

\[A_{3n-2} = \left(\frac{2n}{4n+3}, \frac{4n+5}{3n}\right)\]
\[A_{3n} = \left(\frac{4}{5n+2}, \frac{6n+1}{2n}\right)\]
\[A_{3n-1} = \left(-2, \frac{6n-1}{5n+2}\right)\]

Vemos que:
\begin{enumerate}
	\item $\lim A_{3n-2}= [\frac{1}{2}, \frac{4}{3}]$
	\item $\lim A_n{3n-1} = (-2, \frac{6}{5})$
	\item $\lim A_{3n} = (0^+, 3^+)$
\end{enumerate}

Recordemos que el límite superior de $A_n$ es el conjunto de puntos que están en infinitos conjuntos de la sucesión. Por tanto, todos los puntos contenidos en estos límites se contienen en el límite superior de la sucesión.
\[\limsup A_n = [\frac{1}{2}, \frac{4}{3}] \bigcup  (-2, \frac{6}{5}) \bigcup (0^+, 3^+)\]

Por otro lado, el límite inferior es el conjunto de puntos que se encuentran en todos los elementos de la sucesión a partir de uno dado. Así, el límite inferior será la intersección de los límites calculados anteriormente.
\[\liminf A_n = [\frac{1}{2}, \frac{4}{3}] \bigcap  (-2, \frac{6}{5}) \bigcap (0^+, 3^+)\]

\textcolor{blue}{Completado por mi. No fiarse al 100\%}
\[µ(\limsup A_n) = µ([\frac{1}{2}, \frac{4}{3}]) +  µ((-2, \frac{6}{5})) +µ((0^+, 3^+)) = µ((-2, 3]) = F(3) - F(2) = \frac{8-\sqrt{2}}{10}\]

\[µ(\liminf A_n) = µ([\frac{1}{2}, \frac{6}{5})) = 0\]

\end{problem}

\begin{problem}
Sea $\appl{F}{\real}{\real}$ una función de distribución
\ppart Prueba que el conjunto de puntos de discontinuidad de $F$ es numerable
\ppart Prueba que el conjunto de puntos de continuidad de $F$ es denso en $\real$

\obs $F$ es monótona luego no tiene más discontinuidades que saltos
\solution

\spart Vamos a probar que el número de puntos de discontinuidad en $(n, n+1]$ es numerable.

Para ello tomamos la medida de este intervalo, que es

\[ M = F(n+1)-F(n)\]

La pregunta que nos hacemos ahora es, ¿cuántos puntos $x \in (n, n+1]$ pueden tener $µ_F(\{x\}) = \frac{M}{k}$?

La respuesta es sencilla (la supo hasta Elena en clase) es que no podemos tener más de $k$ puntos con esta condición $∀k ∈ ℕ$. Por tanto no puede haber una cantidad no numerable de puntos de $(n, n+1]$ con $µ_F(\{x\})>0$.

Con más detalle, si tuviésemos una cantidad no numerable de discontinuidades tendríamos una cantidad no no numerable de saltos en un intervalo cerrado.

Así, tendríamos que la medida del intervalo cerrado sería la suma infinita y no numerable de valores positivos, lo que nos daría un resultado finito.

%Leyendo de un artículo de wikipedia:
%\begin{verbatim}
%La suma de los saltos no puede ser mayor que la diferencia de los
%valores de la función en los extremos del intervalo, de modo que
%el conjunto de discontinuidades con salto mayor que 1/n es finito
%y, por tanto, el conjunto de discontinuidades es a lo más numerable
%\end{verbatim}

\spart \textcolor{blue}{Hecho por mí. No fiarse al 100\%}

Recordando lo dado en topología, sabemos que un conjunto es denso si la adherencia del conjunto coincide con el total.

Recordamos también que la adeherencia son aquellos puntos tales que todo abierto que lo contenga corta al conjunto dado.

Puesto que las únicas discontinuidades que puede presentar una función de distribución son discontinuidades de salto, es obvio que para cualquier punto en que la función sea continua todo entorno del punto contiene otros puntos continuos.

\end{problem}

\begin{problem}
Variando si es necesario en cada caso el tamaño de los intervalos, construir un conjunto de tipo Cantor cuya medida de Lebesgue sea mayor que $1-ε$.
\solution
La construcción del conjunto de Cantor consiste en tomar el intervalo $[0,1]$ y los siguientes conjuntos:
\begin{enumerate}
\item Construimos un intervalo $A_1$ de longitud $\frac{ε}{2}$ centrado en el intervalo [0,1].
\item Construimos $A_2 = \bigcup (a_i, b_i)$ tales que cada elemento de la unión tiene longitud igual a $\frac{1}{2}\frac{ε}{4}$.
\item etc
\end{enumerate}
Así, en el paso n-ésimo tenemos:
\[A_n = (a_1^n,b_1^n) \cup (a_2^n,b_2^n) \cup ... \cup (a_n^n,b_n^n)\]
donde cada intervalo de la unión tiene longitud $\frac{1}{2^{n-1}}\frac{ε}{2}$.

El conjunto de Cantor se obtiene restando del intervalo incial todos los intervalos $A_n$ que hemos ido construyendo. Es decir:
\[C = I -\bigcup A_n\]
\[m(C)=m(I)-\sum m(A_n) = 1 - ε\]
\end{problem}

\begin{problem}
Sea $µ_F$ la medida de Lebesgue-Stieltjes correspondiente a una función creciente y continua $\appl{F}{\real}{\real}$
\begin{enumerate}
\item Prueba que si $A$ es numerable entonces $µ_F(A)=0$.
\item Prueba que existen conjuntos $A$ tales que $µ_F(A)> 0$ y $A$ no contiene ningún intervalo abierto.
\item Si $µ(A)\geq 0$ y $µ(\real \setminus A) = 0$, ¿tiene que ser $A$ denso en $\real$?
\end{enumerate}
\obs Se recomienda construir una función $F(x)$ que sea constante en un intervalo
\solution

\begin{enumerate}
\item \textcolor{blue}{Hecho por mí. No fiarse al 100\%}

Si $A$ es numerable podemos escribirlo de la forma:
\[A = \bigcup_{i=1}^{\infty}\{a_i\} \tq a_i \neq a_j \forall i, j \ i \neq j\]
por tanto,
\[µ(A) = \sum_{i=1}^{\infty} µ(\{a_i\}) \]
pero, puesto que la función del enunciado es continua, la medida de un único punto siempre es 0 y por lo que
\[µ(A) = \sum_{i=1}^{\infty} µ(\{a_i\}) = \sum_{i=1}^{\infty}  0 = 0\]

\item Podemos tomar el conjunto $(a, b]$ que tendrá
\[µ((a,b]) = F(b)-F(a) \geq 0\]
ya que $F$ es creciente.

Ahora debemos cuidar que no contenta abiertos. Para ello basta con tomar la intersección de este intervalo con los irracionales.

Con ello eliminamos la posibilidad de que haya abiertos contenidos en el conjunto y mantenemos la medida del conjunto puesto que la medida de los racionales (que son los que estamos extrayendo) es 0.

\item No tiene por qué ser A denso. Pongamos un contraejemplo para probarlo.

Si tomo una función que crece entre el 0 y el 1 y luego se queda constante y tomo $A=(0,1)$ tenemos que se cumplen las condiciones del enunciado pero obviamente el intervalo $(0,1)$ en $\real$ no es denso.
\end{enumerate}
\end{problem}

\begin{problem}
Sea $F(x)=log(1 + |x|)\cdot \ind_{[0, \infty)}(x)$
\begin{enumerate}
\item Comprueba que $\appl{F}{\real}{\real}$ es creciente y continua por la derecha.
\item Calcula $µ_F(\{Cantor\})$
\end{enumerate}
\obs El conjunto de Cantor está contenido en $2^n$ intervalos de longitud $\frac{1}{3^n}$
\solution

\begin{enumerate}
\item \textcolor{blue}{Hecho por mi. No fiarse al 100\%}

Tenemos que ver que $a<b \implies F(a) < F(b)$ lo cual es obvio ya que el logaritmo es creciente y la función indicatriz simplemente vale 0 cuando $x4 < 0$ y 1 en el resto de casos.

Para ver que es contínua por la derecha necesitamos probar que:
\[\lim_{x \to a^+}F(x) = F(a) \ \forall a \in X\]
El único punto donde podemos dudar es en $a=0$ pero en ese caso está claro que:
\[\lim_{x \to 0^+}F(x) = \log(1) = 0 = F(0)\]

\item
\begin{defn}[Conjunto\IS de Cantor]
Es el conjunto de todos los puntos del intervalo real [0,1] que admiten una expresión en base 3 que no utilice el dígito 1
\end{defn}

\[µ_F(\{Cantor\}) = 1 - \frac{1}{3}+2\frac{1}{9}+4\frac{1}{27}+... = \frac{1}{3} \sum_{n=1}^{\infty}\left(\frac{2}{3}\right)^{n-1} = \frac{1}{3} \frac{1}{1-\frac{2}{3}} = 1-1 = 0\]

\end{enumerate}
\end{problem}

\begin{problem}
Sea µ una medida de Borel en $\real$, finita sobre compactos, con $µ((0, 1])=1$

\ppart Prueba que si $\forall s \in \real, µ(s + E)=µ(E)$, entonces µ es la medida de Lebesgue.
\ppart Prueba que si $\forall r \in \real$, se tiene que $µ(rE)=|r|·µ(E)$, entonces µ es la medida de Lebesgue
\solution

\spart
Para ver que es la medida de Lebesgue necesitamos ver que
\[\forall a,b \ a<b \ µ((a,b]) = b-a\]
Por continuidad podemos restringirnos a trabajar con $a,b$ racionales.

Por la invarianza por traslaciones es suficiente ver:
\[\forall b \in \rac \ µ((0,b])=b = \frac{p}{q}\]

Vamos a dividir ese intervalo en $p$ subintervalos disjuntos de longitud $\frac{1}{q}$.
\[µ\left(\left(0, \frac{p}{q}\right]\right) = \sum_{i=1}^{p} µ\left(\left(\frac{i-1}{q}, \frac{i}{q}\right]\right) \]

Al ser los intervalos invariantes por traslaciones tendremos que esa suma es igual a \[ p\cdot µ\left(\left(0, \frac{1}{q}\right]\right)\]

Así, sólo nos queda ver que $µ\left((0, \frac{1}{q}]\right) = \frac{1}{q}$. Para ello vamos a utilizar la premisa del enunciado que nos dice que $μ((0,1]) = 1$. Podemos dividir el intervalo $(0,1]$ en $q$ subintervalos de la forma \[ \left(\frac{i-1}{q}, \frac{i}{q}\right]\quad i = 1,\dotsc, q \] que tienen longitud $\frac{1}{q}$ todos ellos. Luego entonces

\[ 1 = µ((0, 1]) = μ\left(\bigcup_{i=1}^q \left(\frac{i-1}{q}, \frac{i}{q}\right] \right)= \sum_{i=1}^{q} µ\left(\left(\frac{i-1}{q}, \frac{i}{q}\right]\right) = q \cdot µ\left(\left(0, \frac{1}{q}\right]\right)\] luego \[ µ\left(\left(0, \frac{1}{q}\right]\right) = \frac{1}{q} \] que es lo que nos faltaba para comprobar que $μ$ fuese una medida de Lebesgue.

\spart De nuevo, queremos ver que $μ\left((a,b]\right) = b - a$. Podemos escribir \[ (a, b] = (0, b] \setminus (0,a] \] luego \[ μ((a,b]) = μ\left((0,b] \setminus (0, a]\right) = μ((0,b]) - μ((0,a])\] y usando el hecho de que son invariantes por dilataciones, nos queda que

\[ μ((0,b]) - μ((0,a]) = bμ((0,1]) - aμ((0,1]) = b - a \] que es lo que queríamos que demostrar.

\end{problem}

\begin{problem}
Sea µ la medida de Lebesgue de $\real$ y $E \subset \real$ medible Lebesgue tal que $0 < µ(E) < \infty$. Demuestra que para todo α, $0<α<1$ existe un intervalo abierto $I$ tal que $µ(I \cap E) > α μ(I)$
\solution
Sabemos que la medida de Lebesgue se puede aproximar bien por abiertos que contengan al conjunto $E$, es decir:
\[\forall ε > 0 \ \exists A \text{ abierto con } E \subset A \tq µ(E) > µ(A)-ε \]

Tomamos $A= \bigcup_{k=1}^{\infty} I_k$ con los $I_k$ disjuntos.

Suponemos ahora que $\exists α \in (0,1) \tq \forall I \text{ intervalo abierto } µ(E \cap I)\leq αµ(I)$
Entonces
\[µ(E) = \sum_{k=1}^{\infty}µ(E \cap I_k) \leq \sum_{k=1}^{\infty} α µ(I_k) = α µ(A)\]

Si la suposición fuese cierta tendríamos ahora
\[µ(A)-µ(E) \geq µ(A) - α µ(A) = (1-α)µ(A) \geq (1-α)µ(E) > ε\]
y llegamos a contradicción.
\end{problem}

\section{Hoja 5}
\begin{problem}[1]
Sea $\algbM$ la $\salgb$ formada por $\{\emptyset, \real, (-\infty, 0], (0, \infty)\}$ y sea $\appl{f}{(\real,\algbM)}{(\real, \algbB_{\real})}$ definida mediante:
\[ f(x)= \begin{cases}
    0 & x \in  (-\infty, 0] \\
    1 & x \in (0, 1] \\
    2 & x \in (1, \infty)
\end{cases} \]
\ppart
¿Es $f$ medible?
\ppart
¿Cómo son las funciones medibles $\appl{f}{(\real,\algbM)}{(\real, \algbB_{\real})}$

\solution
\spart
Según la definición, para que una función sea medible necesitamos que la imagen inversa de un medible (conjunto contenido en la $\salgb$) sea medible. Es decir:
\[F \text{ es }  (\algbM,\algb{N})\text{-medible } \iff \forall E \in \algb{N}, f^{-1}(E)\in \algbM\]
Siguiendo esta definición, podemos concluir que la función no es medible puesto que \[f^{-1}((0,1])=(0,1]\notin \algbM\]

\spart
Atendiendo nuevamente a la definición, para que una función $\appl{f}{(\real,\algbM)}{(\real, \algbB_{\real})}$ sea medible necesitamos que la imagen inversa de un medible de $(\real, \algbB_{\real})$, sea medible en $(\real,\algbM)$.

Por tanto, $f$ es medible si:
\[\forall x,y \in (-\infty, 0] \implies f(x)=f(y) \text{ y } \forall x,y \in (0, \infty) \implies f(x)=f(y)\]

\end{problem}

\begin{problem}[2]
Sea $(X,\algbM)$, un espacio medible y $\appl{f}{(\real,\algbM)}{(\real, \algbB_{\real})}$ ¿Cuál de las siguientes afirmaciones es cierta?
\ppart $|f|$ medible $\implies$ f medible
\ppart $f_1+f_2$ medible $\implies f_1$ ó $f_2$ medible
\ppart $f_1\cdot f_2$ medible $\implies f_1$ ó $f_2$ medible
\ppart $f_1+f_2$ medible y $f_1-f_2$ medible $\implies f_1$ ó $f_2$ medible
\solution

\spart
\textbf{Falso}

Como ejemplo podemos tomar una $\algbM$ como en el ejercicio anterior y una función
\[ f(x)=  \begin{cases}
    -1 & x \in  (-\infty, 1] \\
    1 & x \in (1, \infty)
\end{cases} \]
de modo que su valor absoluto, que es la función constante $f(x)=1$ es medible mientras $f$ no lo es, puesto que:
\[f^{-1}((0,1]) = (-\infty, 1) \notin \algbM \]
\spart
\textbf{Falso}

Tomando la función anterior y su opuesto tenemos dos funciones no medibles cuya suma es la función nula ($f(x)=0$), que si es medible.

\spart
\textbf{Falso}

Repetimos el ejemplo del apartado anterior, al multiplicar la función del apartado a) con su opuesto obtenemos una función constante $f(x)=-1$ que es medible, sin ser ninguno de los factores medible.

\spart
\textbf{Verdadero}

Puesto que sabemos que la suma de funciones medibles es medible, tenemos que
\[f_1 + f_2 + f_1 -f_2 \text{ medible por ser suma de medibles } = 2\cdot f_1 \text{ medible } \implies f_1 \text{ medible}\]

\end{problem}

\begin{problem}[3]
Sean $\appl{f}{(X, \algbM, µ)}{(\overline{\real}, \algbB_{\overline{\real}})}$ una función medible no negativa y µ una medida $\sfin$ en $\algbM$. Demuestra que $f= \lim t_n$, siendo $t_n$ funciones simples no negativas que verifican
\[t_1 \leq t_2 \leq t_3 ...\leq t_n \leq ...\]
y son tales que $t_n$ toma valores distintos de cero solamente en un conjunto de medida finita.

\obs Construye una sucesión $B_1 \subset B_2 \subset \dotsb \subset B_n \subset ... \subset X$ con $µ(B_n)<\infty$ y toma $t_n=\ind_{B_n}$, siendo $s_n$ una sucesión creciente de funciones simples no negativas con límite $f$.
\solution

Dado que μ es σ-finita (ver \ref{defSigmaFinita}), entonces $X$ se puede expresar como una unión como mucho numerable de elementos $E_n$ de la σ-álgebra de medida finita. Por otra parte, por ser $f$ medible, existe una sucesión $s_n$ de funciones simples crecientes cuyo límite es $f$.

Sólo nos falta construir esa sucesión creciente de intervalos, y podemos hacerlo de la siguiente forma:

\[ B_n = \bigcap_{i=n}^{∞} E_i \]

Es decir, que para $B_1$ tomamos la intersección de todos los $E_n$, para $B_2$ la de todos menos el primero, y así sucesivamente. Es obvio que estos $B_n$ forman una sucesión como la de la indicación, y con eso ya tenemos todo hecho, salvo el hecho de que los $t_n$ sólo toman valores en conjuntos de medida finita.

Pero en realidad sí está hecha: los $E_n$ son de medida finita, luego los $B_n$ también lo serán, y entonces $t_n = s_n \ind_{B_n}$ sólo será distinta de $0$ en $B_n$ que es de medida finita.

\end{problem}

\begin{problem}[4]
Prueba que si $\appl{f}{X}{\overline{\real}}$ verifica que $f^{-1}(r, \infty]$ es medible para todo $r\in\rac$, entonces $f$ es medible.

\solution
Para asegurar que $f$ sea medible tenemos que ver:
\[\forall x \in \real f^{-1}(x, \infty] \text{ es medible}\]

Pero sabemos que
\[(x, \infty] = \bigcup_{r \in \rac \ r \geq x}(r, \infty]\]
por tanto
\[f^{-1}((x, \infty]) = f^{-1}(\bigcup_{r \in \rac \ r \geq x}(r, \infty]) = \bigcup_{r \in \rac \ r \geq x} f^{-1}((r, \infty])\]
que es medible por ser unión de funciones medibles.

\end{problem}

\begin{problem}[5]
Si $\forall n, \appl{f_n}{(X,\algbM)}{(\overline{\real}, \algbB_{\overline{\real}})}$ es medible demuestra que
\[A = \{x \in X : \ \exists  \lim_{n \to \infty} f_n(x)\} \in \algbM\]
\solution

Consideramos las funciones
\[F(x) = \limsup f_n(x) \]
\[G(x) = \liminf f_n(x) \]
tales que su diferencia es medible.

Consideramos $A_c=(F-G)^{-1}(\{c\})$, que es medible, puesto que se trata de la imagen inversa de un conjunto medible por una función medible. Con esto tenemos todos los puntos de $A$ para los que el límite es una constante. Ahora vamos a estudiar aquellos para los que el límite es $\pm \infty$

Vamos a considerar $A=A_{-\infty} \cup \bigcup_{c \in \real} A_c \cup A_{\infty}$ con
\[A_{-\infty} = \{x \tq \limsup f_n = -\infty\}=F^{-1}(\{-\infty\})\]
\[A_{\infty} = \{x \tq \liminf f_n = \infty\}=G^{-1}(\{\infty\})\]

Con lo que ahora tenemos $A$ expresado como unión de medibles, por lo que es medible.

\end{problem}

\begin{problem}[6]
Sean $(\Omega, \algbM, \algb{P})$ un espacio de probabilidad; $\appl{X_1, X_2}{(\Omega, \algbM, \algb{P})}{(\real, \algbB_{\real})}$ funciones medibles y $F_{X_1}, F_{X_2}$ las funciones de distribución inducidas por $X_1$ y $X_2$ respectivamente. Prueba que si $P\{\omega \in \Omega : X_1(\omega)=X_2(\omega)\}=1$, entonces $\forall x \in \real, \ F_{X_1}(x) = F_{X_2}(x).$

¿Es cierta la implicación contraria?

\solution
Tenemos que $P\{\omega \in \Omega : X_1(\omega)=X_2(\omega)\}=1$, entonces
\[F_{X_1}(x) = P(X_1 \leq x) = P(X_1 \leq x, X_1=X_2) + P(X_1 \leq x, X_1 \neq X_2) \overbrace{=}^{\text{por hipótesis}}  \]
\[\overbrace{=}^{\text{por hipótesis}}  P(X_1 \leq x, X_1=X_2) = P(X_2 \leq x, X_1=X_2)=P(X_2  \leq x)=F_{X_2}(x)\]

Para demostrar que la implicación contraria no es cierta consideremos el siguiente contraejemplo:

Tomemos $\Omega = \{c,x\}$ y las variables:
\[X_1(c)=1, X_1(x)=0\]
\[X_2(c)=0, X_2(x)=1\]
Son distintas en todo punto pero tienen la misma función de distribución
\end{problem}

\begin{problem}[7]
Se considera el espacio de probabilidad $(\nat^*, \algbP(\nat^*), \algbP),$ con $\algbP(n) = \frac{1}{2^n}, \ n \in \nat^*$.

Sea $\appl{X}{\nat^*}{\{0,1,...,k-1\}}$ la función definida como $X(n)=n \ mod \ k$, con $k \in \nat^*$ y sea $\algbP^*$ la probabilidad inducida por $X$.

Calcula $\algbP^*(r), \ 0 \leq r \leq k-1$.

\obs El símbolo $N^*$ denota $\nat \setminus \{0\}$

\solution
Puesto que $\algbP$ es la probabilidad inducida por $X$, tenemos que:
\[\algbP^*(i) = \algbP(X^{-1}(i))\]

Vamos a calcularlo
\[\algbP^*(i) = \algbP(\{i+k, i+2k, i+3k, ...\}) = \sum_{n=0}^{\infty} \algbP(\{i+nk\}) = \sum_{n=0}^{\infty} \frac{1}{2^{i+nk}} = \frac{2^k}{2^i}\frac{1}{2^k-1}\]

\end{problem}

\begin{problem}[8]
Se considera el espacio de probabilidad $(\real, \algbB_{\real}, \algbP)$ donde $\algbP$ viene dada por la función de densidad:
\[
f(x)= \left\{ \begin{array}{lcc}
             1 &   si  & x \in  (0, 1] \\
             \\ 0 &  si & x \notin (0, 1]
             \end{array}
   \right.
\]
Sea $\appl{X}{(\real, \algbB_{\real}, \algbP)}{(\real, \algbB_{\real})}$ definida mediante:
\[
X(x)= \left\{ \begin{array}{lcc}
             -2\log x &   si  & x >  0 \\
             \\ 0 &  si & x \leq 0
             \end{array}
   \right.
\]
Halla $F_X$, la función de distribución de la probabilidad inducida por $X$.
\solution
Sin mediar palabra vamos a ello
\[F_X(x_0) = \algbP(X^{-1}(-\infty, x_0]) = \algbP(X \leq x_0) = \algbP(-2\log x \leq x_0) = \algbP(x \geq e^{-\frac{x_0}{2}})\]

Y ahora vemos que:
\[
\algbP(x \geq e^{- \frac{x_0}{2}})= \left\{ \begin{array}{lcc}
             1-e^{-\frac{x_0}{2}} &   si  & 0 < e^{\frac{-x_0}{2}} <1 \\
             \\ 0 &  si & 1 < e^{\frac{-x_0}{2}}
             \end{array}
   \right.
\]
\end{problem}

\begin{problem}[9]
El mismo enunciado que en el ejercicio anterior, siendo $X(x)=x^2$ y
\[
f(x)= \left\{ \begin{array}{lcc}
             \frac{1}{2} &   si  & x \in  [-1, 0) \\
             \\ \frac{1}{4} & si & x \in [0,2) \\
             \\ 0 &  si & x \geq 2
             \end{array}
   \right.
\]

\solution

Nuevamente, para calcular la función de distribución inducida debemos calcular
\[F_X(x_0) = \algbP(X^{-1}(-\infty, x_0)) = \algbP (x^2 \leq x_0) = \algbP (-\sqrt{x} \leq x \leq \sqrt{x_0})\]

Y llegamos a
\[
\algbP (x \leq \sqrt{x_0})= \left\{ \begin{array}{lcc}
				0 & si & x_0 < 0
			 \\ \frac{1}{2} + \frac{1}{4}\sqrt{x_0} - \frac{1}{2}\sqrt{2} & si & 0 \leq x_0 \leq 1 \\
             \\ \frac{1}{2} + \frac{1}{4}\sqrt{x_0} & si & 1 \leq x_0 \leq 4 \\
             \\ \frac{3}{4} &  si & x \geq 2
             \end{array}
   \right.
\]
\end{problem}

\section{Hoja 6}
En los tres primeros ejercicios, $\appl{f}{[0,1]}{\real^+}$

\begin{problem}[1]
Se define $f$ mediante $f(x) = 0 $, si $x$ es racional y $f(x)=n$ si $n$ es el número de ceros inmediatamente después del punto decimal en la representación de $x$ en la escala decimal. Caclula $\int f(x) dm$

\solution

\textcolor{blue}{Completado por mi. No fiarse al 100\%}

Para hacer este ejercicios observamos que la probabilidad de tener un único 0 tras la coma decimal es de $\frac{1}{10}\frac{9}{10}$, de tener dos 0 es $\frac{1}{10}\frac{1}{10}\frac{9}{10}$ y así sucesivamente.

Además, por tratarse de una función discreta podemos calcular la integral como un sumatorio, con lo que obtenemos:

\[\int f(x) dm = \sum_{n=1}^{\infty} n \frac{9}{10^{n+1}}\]

Una forma más física de verlo es percatarse de que la función valdrá 1 para todos los puntos entre 0,01 y 0,1 $\implies$ vale 1 a lo largo de un conjunto de puntos de longitud 0,09; valdrá 2 entre los puntos 0,001 y 0,01 $\implies$ un conjunto de puntos de longitud 0,009 ...

\end{problem}

\begin{problem}[2]
Sea $f(x)$=0 en cada punto del conjunto de Cantor de [0,1] y $f(x)=p$ en cada intervalo del complementario de longitud $\frac{1}{3^p}$. Demuestra que $f$ es medible y calcula $\int_0^1f(x)dm$

\solution
\textcolor{blue}{Completado por mi. No fiarse al 100\%}

Recordemos que conjunto de Cantor es el que se contruye colocando un intervalo de longitud (en este caso) $\frac{1}{3}$ centrado en el intervalo [0,1]. A continuación colocamos un intervalo de longitud $\left(\frac{1}{3}\right)^2$ centrado en el centro de cada uno de los intervalos que quedaban fuera del anterior, y así sucesivamente.

Finalmente, el conjunto de Cantor se el conjunto de puntos que no se contienen en ninguno de los abiertos definidos anteriormente.

Sabemos que es medible puesto que
\[f^{-1}([0, x)) = \bigcup_{k=0}^nf^{-1}(\{k\})\]
y la inversa de $\{k\}$ será la unión de los intervalos abiertos construidos por Cantor (para extraerlos del intervalo unidad) que son abiertos.

\[\int_0^1 f(x)dm= \sum_{n=1}^{\infty}n2^n\frac{1}{3^n} = \frac{1}{3}\sum_{n=1}^{\infty} n \left(\frac{2}{3}\right)^n\]

puesto que la función tendrá el valor $n$ para $2^n$ conjuntos de longitud $\frac{1}{3}$
\end{problem}

\begin{problem}[3]
Sea $f(x) = 0$ si $x$ es racional, $f(x)=[\frac{1}{x}]$ si $x$ es irracional. Calcula $\int f(x)dm$

\solution
\textcolor{blue}{Completado por mi. No fiarse al 100\%}

Podemos ignorar el valor de la función en los racionales, puesto que constituyen un conjunto de medida 0.

Nuevamente, por tratarse de una función discreta (sólo toma valores enteros) podemos calcular la integral como un sumatorio.

\[\int f = \sum_{n=1}^{\infty} n \left( \frac{1}{n}-\frac{1}{n+1}\right)= \sum_{n=1}^{\infty} \frac{1}{n+1} = \infty\]

ya que todo punto en el intervalo $\left( \frac{1}{n}-\frac{1}{n+1}\right)$tendrá el mismo valor de f(x): $n$

\end{problem}

\begin{problem}[4]
Sea $(f_n)$ la sucesión de funciones definida por $f_{2n-1}=\ind_{[0,1]}$; $f_{2n}=\ind_{[1,2]}$; n=1,2,... Comprueba que para esta sucesión se verifica la desigualdad estricta en el Lema de Fatou.

\solution

Recordemos que el lema de Fatou nos decía que
\[\forall n f_n \in L^+ \implies \int \liminf f_n \leq \liminf \int f_n\]

En este caso $\int \liminf f_n = \int 0 = 0$ mientras que $\liminf \int f_n = \liminf \int 1 = 1$

Y obviamente tenemos que 0<1
\end{problem}

\begin{problem}[5]

Comprueba que
\[\int_{[1,\infty)} \frac{1}{x}dm = \infty\]

\solution

Para comprobarlo debemos ver que existe una sucesión creciente de funciones simples qué estén por debajo de $f(x)$ y cuyas integrales converjan a infinito.

Evidentemente, la función que debemos tomar es $\{\frac{1}{n+1}\ind_{[n,n+1)}\}$ de modo que tendríamos:
\[\int_{[1,\infty)} \frac{1}{x}dm \geq \sum_{n=1}^N \{\frac{1}{n+1}\int\ind_{[n,n+1)}\} =  \sum_{n=1}^N \frac{1}{n+1}\rightarrow\infty\]

\end{problem}

\begin{problem}[6]

Sea $(f_n)$ una sucesión de funciones medibles tales que
\[\forall x, \ \forall n, \ 0 \leq f_n(x) \leq \lim_n f_n(x)=f(x)\]
Comprueba que
$\int f\ dµ = \lim_n \int f_n dµ$

\textbf{Sugerencia:} Usa el Lema de Fatou y la condición $f_n < f$

\solution
\textcolor{blue}{Completado por mi. No fiarse al 100\%}

Atendiendo a la sugerencia, vemos que
\[0 \leq \liminf \int (f-f_n) \implies 0 \leq \liminf(\int f - \int f_n) = \int f - \limsup \int f_n\]

Por otro lado, si aplicamos el lema de Fatou sabiendo que el límite inferior de la sucesión de funciones coincide con el límite de la misma, obtenemos
\[\int f \leq \liminf \int f_n \leq \limsup \int f_n \leq \int f \implies \exists \lim \int f_n = \int f \]

Es decir, vemos que el límite inferor y superior de la integral son a la vez mayores y menores que la integral de la función, por lo que han de ser iguales a la misma.

Una vez tenemos que los límites inferior y superior coinciden y conocemos su valor, queda clara la existencia del límite y el valor del mismo.
\end{problem}

\begin{problem}[7]
Sea $\appl{f}{X}{[0,\infty]}$ medible. Se define $f_n(x)=\min\{f(x), n\}$. Demuestra que
\[\int f_n \ dµ \nearrow \int f \ dµ\]

\solution
\textcolor{blue}{Completado por mi. No fiarse al 100\%}

Por la propia definición de $f_n(x)$ queda claro que
\[f_n(x) \leq f(x) \forall x \implies \int f_n(x) \leq \int f(x)\]

Para garantizar la convergencia podemos apoyarnos en el teorema de la integral monótona puesto que las $f_n$ constituyen una sucesión estrictamente creciente de funciones medibles con límite $f$.

Para ver que es una sucesión creciente procedemos a emplear reducción al absurdo. Supongamos que no es creciente, es decir
\[\exists x \tq f_n(x)>f_m(x) \text{ con } n<m\]
Si $f(x)<n \implies f_n(x)=f(x)=f_m(x)$ que implica contradicción.

Si $n<f(x)<m \implies f_n(x)=n \wedge f_m(x)=f(x)>n=f_n(x)$ que contradice de nuevo el hecho de que $f_n(x)>f_m(x)$

Por último, si $m<f(x) \implies f_n(x)=n \wedge f_m(x)=m$ que vuelve a contradecir la idea de que la sucesión no sea creciente.

Por tanto, no queda más remedio que la sucesión sea creciente.

Lo último que nos queda por hacer para completar el ejercicio es ver que las funciones son medibles. Para ello tenemos que ver que la inversa de un conjunto medible es medible.

Sea $A$ medible en $\real$, $f_n^{-1}(A)=f^{-1}(A \cap [n,\infty)) = f_n^{-1}(A)\cap f_n^{-1}([n,\infty))$ que es medible por ser intersección de medibles.


\end{problem}

\begin{problem}[8]
Sean $f\geq 0$, $g\geq 0$ medibles, $f\geq g$ y $\int g \ dµ < \infty$. Prueba que
\[\int f \ dµ - \int g \ dµ = \int (f-g)dµ\]

\solution
\textcolor{blue}{Completado por mi. No fiarse al 100\%}

Vamos a definir $f=g+(f-g)$, sabiendo que $f\geq g \implies f-g\geq 0$

Así, tenemos
\[\int f dµ = \int g+(f-g)dµ = \int gdµ + \int (f-g) dµ\]
por ser $\int g < \infty$ podemos pasarla restando sin problemas, con lo que llegamos a
\[\int f dµ - \int gdµ = \int (f-g) dµ\]

\end{problem}

\begin{problem}[9]

Sea $(f_n)$ una sucesión de funciones medibles no negativas y acotadas. Supongamos que $f_n(x) \searrow f(x)$ y que para algún $k$ se verifica que $\int f_k dµ < \infty$.

Prueba que
\[\lim \int f_n dµ = \int f dµ\]

\textbf{Sugerencia:} Forma la sucesión $g_n = f_k -f_{k+n}$

\solution

Sin pérdida de generalidad podemos suponer que $\int f_0 dµ < \infty$.

Ahora, atendiendo a la sugerencia, vamos a considerar la sucesión
\[g_n = f_0 - f_n\]

Puesto que todas las $f$ son integrables y finitas, tenemos que
\[\int g_n = \int f_0 - \int f_n\]

Sabiendo que las $g_n$ constituyen una sucesión creciente, aplicamos el teorema de la convergencia monótona llegamos a
\[\lim \int g_n = \int \lim g_n = \int \lim(f_0-f_n) = \int (f_0-f)\]
\end{problem}

\begin{problem}[10]
Sea $(a_n)$ una sucesión decreciente de números positivos con límite 0. Sea
\[f_n(x) = \frac{a_n}{x} \ind_{x > a}(x)\]
Comprueba que la sucesión $(f_n)$ decrece uniformemente a 0, pero $\forall n, \int f_n dm = \infty$

\solution

En primer lugar vamos a acotar el valor absoluto de la función $f(x)$.
\[x>a \implies \frac{1}{x} < \frac{1}{a} \implies \abs{f_n(x)} \leq \frac{a_n}{a}\]

Podemos ver que la función $f_n$ converge uniformemente a 0 pues su límite es positivo y menor que $\frac{\lim a_n}{\lim a} = 0$.

Por último, ver que $\forall n, \int f_n dm = \infty$ es algo evidente pues estamos integrando un valor constante a lo largo de un conjunto de medida infinita $(a, \infty)$:
\[\int f_n \ dm = \int \frac{a_n}{a} = \frac{a_n}{a} m((a, \infty)) = cte \cdot \infty = \infty\]

\end{problem}

\begin{problem}[11]
Sea $\appl{f_n}{(0,1]}{[0, infty)}$, definida mediante:
\[f_n(x)= \left\{ \begin{array}{lcc}
             n &   si  & x \in (0, \frac{1}{n}] \\
             \\ 0 &  si  & x \in (\frac{1}{n}, 1]
             \end{array}
   \right.\]
Comprueba que $f_n \rightarrow 0$ puntualmente pero $\int f_n \ dm = 1$

\solution

Para comprobar la convergencia puntual debemos ver que
\[\forall x, \forall \epsilon \exists N \tq \forall n > N \ |f_n(x)-f(x)| < \epsilon\]

En este caso es sencillo ver que para cualquier $x$ y cualquier valor de $\epsilon$ basta con tomar $n>\frac{1}{x}$ para que se cumpla la definición de convergencia puntual.

Cada $f_n$ es un rectángulo de base $\frac{1}{n}$ y altura $n$ por tanto, queda claro que su área sería:
\[\int f_n(x) \ dm= 1\]
\end{problem}

\begin{problem}[12]
Sea $\algbM$ la $\salgb$ de los conjuntos medibles de Lebesgue en [0, $\infty)$. En $\algbM$ de define la medida µ como:
\[µ(E) = \int_E \frac{1}{x+1} dx\]

Halla una función $F$ tal que µ sea la medida de Lebesgue-Stieldjes inducida por $F$. Halla una función $f$ tal que
\[\int f dµ < \infty \text{ pero } \int f dm = \infty\]

\solution

Tomamos la función
\[F(x) = µ((-\infty, x]) = µ([0,x]) = \int_0^x \frac{dt}{1+t}=\log (1+x)\]
\[
F(x) (x \leq \sqrt{x_0})= \left\{ \begin{array}{lcc}
             0 & si & x_0 < 0 \\
             \\ \log(1+x) &  si & x \geq 0
             \end{array}
   \right.
\]

\obs \[µ(E) = \int_E dF(x) = \int_E F'(x) dx = \int \ind_E F(x) dx\]

La función $f$ buscada es
\[f(x) = \frac{1}{1+x}\ind_{[0, \infty)} \implies \int_0^{\infty} f dm = \int_0^{\infty} \frac{dx}{1+x} = \infty\]
Además
\[\int f(x) dF(x) = \int_0^{\infty} \frac{dx}{(1+x)^2} = 1\]

\end{problem}

\begin{problem}[13]
Sea $\{f_n\}$ una sucesión de funciones de $(\real, \algb{L}, m)$ en $(\real, \algbB_{\real})$, medibles y no negativas, tales que $\forall x \in \real$, $\lim_{n \to \infty}f_n(x) = f(x)$, y además $\int_{\real}f_n dm = \int_{\real} fdm=1$ para todo $n \in \nat$. Demuestra que
\[\forall x \in \real, \ \lim_{n}\int_{-\infty}^{\infty}f_n dm = \int_{-\infty}^{\infty} f dm\]

\textbf{Sugerencia:} Usa un teorema de convergencia adecuado
\solution

El teorema de convergencia adecuado es el lema de Fatou, que hay que aplicar 2 veces.

Para demostrar que
\[\lim_n \int_{-∞}^{x}f_n dm = \int_{-∞}^{x}fdm\]

Vamos a demostrar las 2 desigualdades.

$\geq$) la da el lema de Fatou, reescribiendo $f_n(t)\ind_{(-∞,x)} \to f(t)\ind_{(-∞,x)}$, entonces:

\[\int_{-∞}^{x} f_n = \int_{ℝ} f_n(t)\ind_{(-∞,x)}\]

\[\limsup_n \int_{-∞}^{x} f_n dm =\limsup_n \int f_n(t)\ind_{(-∞,x)} \geq \lim inf \int f_n(t)1_{(-∞,x)}dm \overset{Fatou}{≥} \int f\ind_{(-∞,x)} dm\]

$\leq$) \[\int_{-∞}^{x} f_n(t) = 1-\int_{x}^{∞} f_n(t) = 1 - \int f_n(t) 1_{(x,∞)} \]

Ahora vamos a aplicar el lema de Fatou en $\int f_n(t) 1_{(x,∞)}$ para lo que hay que reescribirlo utilizando la hipótesis $\int f_n = 1$

\[
\int f_n(t) \ind_{(x,∞)}  = \int_x^∞ f_n(t) = 1 -\int_{-∞}^{x} f_n(t) = 1 - \int f_n(t) \ind_{(-∞,x)}
\]

Aplicando Fatou
\begin{equation}\label{eq6.13.1}\underbrace{\lim inf \int f_n(t) \ind_{(x,∞)}}_{(1)} \overset{Fatou}{≥} \int f(t) 1_{(x,∞)} = 1 - \int f(t) \ind_{(-∞,x)}
\end{equation}

\begin{equation}\label{eq6.13.2}
(1) = \lim inf \left(1-\int f_n(t) 1_{(-∞,x)}\right) = 1-\lim sup \int f_n \ind_{(-∞,x)}\end{equation}

Utilizando \ref{eq6.13.1} y \ref{eq6.13.2},$ \implies \lim sup \int f_n ≤ \int f \ind_{(-∞,x)}$


Juntando los 2 resultados:
\[\int f \ind_{(-∞,x)} ≤ \lim inf \int f_n(t)1_{(-∞,x)}dm ≤ \lim sup  \int f_n\ind_{(-∞,x)}dm ≤ \int f \ind_{(-∞,x)}\]

Por tanto, el límite inferior y superior coinciden puesto que están acotados superior e inferiormente por el mismo valor con lo que obtenemos la igualdad buscada

\[
\int f(t) \ind_{(-∞,x)}dm = \lim_n \int f_n(t) \ind_{(-∞,x)}dm
\]

\end{problem}

\begin{problem}[14]
Sea $f(x) = \frac{1}{\sqrt{x}}\ind_{(0,1)}(x)$ y sea $\{r_n\}$ una enumeración de los racionales, se define
\[g(x)=\sum_{n=1}^{\infty}\frac{1}{2^n}f(x-r_n)\]
Prueba las siguientes afirmaciones
\ppart $g \in L^1(m)$
\ppart Para casi todo $x \in \real$, $g(x) < \infty$
\ppart $g$ es discontinua en todo punto en $\real$ y además no está acotada en ningún intervalo $I \subset \real$

\solution
\spart
Definimos la sucesión
\[g_n(x) = \sum_{k=1}^{n} \frac{1}{2^k}f(x-r_k)\]
que, obviamente, es creciente (por ser las $g_n(x)$ positivas) y converge a $g(x)$

Sabemos que $f$ es $L^1$ y que, por traslación, también lo es $f(x-r_k)$. Puesto que la suma de funciones de $L^1$ y el producto por constante son funciones de $L^1$, podemos concluir que $g_n \in L^1 \forall n$

Aplicando ahora el teorema de convergencia monótona tenemos que
\[\int f = \lim \int \sum_{k=1}^{\infty} \frac{1}{2^k}f(x-r_k) = \lim_n  \sum_{k=1}^{\infty} \int \frac{1}{2^k}f(x-r_k) =\lim_n  \sum_{k=1}^{\infty}\frac{1}{2^k} 2 = 2\]

Por lo que $g \in L^1$

\spart
Por pertenecer a $L^1$ sabemos que para casi todo punto $g(x) < \infty$

\spart

En un intervalo de cualquier $r_n$, $g(x)$ tiene valores tan grandes como yo quiera por lo que no puede ser continua ya $x\to r_n \implies g(x) \to \infty$

En cualquier intervalo $I$ vamos a tener racionales. Por tanto habrá un racional $r_n$ para el que $g(r_n)=\infty$. Queda claro pues que la función no está acotada para ningún intervalo $I$.
\end{problem}

\begin{problem}[15]
Sea $f\geq 0$ una función de $L^1(\real)$ tal que para cualquier par de enteros positivos n,m se verifica
\[\int_{\real} f^n(x) \ dx=\int_{\real}f^m(x) \ dx\]

Demuestra que $f$ coincide a.e con la función característica de un conjunto medible $A$ de medida finita.

\solution
\textcolor{blue}{Completado por mi. No fiarse ni al 12\%}

Vamos a seguir los pasos indicados en la sugerencia:

\spart
\textbf{Demostrar que $A=\{x \in \real: f(x)=1\}$ es un conjunto medible de medida finita}

Para poder garantizarlo, necesitamos que
\[\int_{\real}f(x)\ind_A(x) dx = \int_A f(x) dx < \infty\]
y, puesto que la integral es el cálculo de un área, tenemos que
\[\int_A f(x) \leq \int_{\real}f(x) = \int_{\real}|f(x)| <\infty \text{ Por ser } f\in L^1\]

\spart
\textbf{Demuestra que a.e. $x, f(x) \leq 1$}

Demostrar que una propiedad se cumple para casi todo punto implica porbar que el conjunto de puntos para los que no se cumple la propiedad tiene medida 0.

Sea $B=\{x \in \real : f(x) > 1\}$, vamos a ver que tiene medida 0, es decir:
\[\int f(x) \ind_B dx = \int_B f(x) = 0\]

Tenemos que:
\[\int_{\real}f^n(x) = \int_{\real}f^m(x) \implies \int_{\real}f^n(x)-f^m(x) = 0\]
Teniendo en cuenta que el conjunto $B$ es el formado por los puntos en los que $f(x)>1$, si tomamos $n>m$ siendo ambos pares (para los que también se cumple la igualdad, puesto que es válida para n,m cualesquiera) nos encontramos con la integral de una función $f^n(x)-f^m(x)$, siempre positiva y con valor 0.

Por tanto, podemos restringirnos al conjunto $B$ donde, puesto que estamos calculando áreas y nos estamos limitando a un subconjunto de $\real$, sabemos que la integral tendrá un valor menor. Pero, puesto que la integral tendrá que ser positiva tenemos que debe valer 0.

Es decir
\[\forall n,m \text{ pares con } n>m \  \int_B f^n(x)-f^m(x) = 0\]

Para que una integral de una función positiva valga siempre 0 tenemos que:
\[\forall n,m \ \int_B f^n(x)-f^m(x) = 0 \implies f^n(x)=f^m(x) \forall n,m \text{ ó } µ(B) = 0\]

La primera condición sólo puede darse si $f(x)$ valiese $\pm$ 1 en todos sus puntos. No puede ser la función constante $f(x) = \pm 1$ ya que entonces no sería $L^1$, pero aún podría ir alternando.

No obstante, puesto que la igualdad inicial es válida para cualquier $n,m$ si fuese alternando podríamos tomar un $n$ par y un $m$ impar, de modo que
\[\int_{\real}f^n(x)=\int_{\real}1=\infty\]
pero
\[\neq\int_{\real}f^m(x)dx=\int_{\real}f(x) < \infty\]
ya que $f(x) \in L^1$.

También podría ocurrir que $f(x)$ fuese la función nula pero, en ese caso, estaría claro que el conjunot $B$ tendría medida 0, puesto que sería el vacío.

Por tanto, sólo puede darse la segunda condición, que el conjunto $B$ tenga medida 0.

\spart
\textbf{Demuestra que $\int_{A^c}dx = 0$}

Esto es sencillo de ver puesto que
\[\int_{A^c}f(x) dx = \int_B f(x)dx + \int_{\{x: f(x)<1\}} f(x)dx = 0 + \int_{\{x: f(x)<1\}} f(x)dx\]

Pero, repitiendo un razonamiento similar al del apartado anterior, tenemos
\[\forall n<m \text{ pares } \ \int_{\{x: f(x)<1\}} f^n(x)-f^m(x) = 0 \implies f^n(x)=f^m(x) \text{ ó } µ(\{x: f(x)<1\}) = 0\]

Por el mismo razonamiento del apartado anterior vemos que la única posibilidad viable es que su cumpla la segunda condición, es decir $µ(\{x: f(x)<1\}) = 0$


\textbf{Demuestra que a.e. $x\in\real, f(x)=\ind_A(x)$}

Para verlo tenemos que probar que, siendo $B=\{x\in\real: f(x) \neq \ind_A(x)\} = \{x\in\real: f(x) \neq 1\} = A^c$ el conjunto de puntos en los que $f(x)$ no coincide con la identidad sobre $A$ pues no vale 1, este conjunto tiene medida 0.

También debemos ver que el conjunto $A$ es un conjunto medible de medida finita.

Ya hemos probado en apartados anteriores estas restricciones por lo que ya lo tenemos.

\end{problem}


\begin{problem}[16]
Compara la conclusión del Lema de Fatou con los resultados que se pide demostrar en el ejercicio 6 de la hoja de ejercicios número 3.

\solution

Puesto que la forma de medir un conjunto es considerar la integral de la función indicatriz sobre el mismo observamos que el Lema de Fatou implica los resultados que se pide demostrar en el ejercicio 6 de la hoja 3.
\end{problem}

\begin{problem}[17]
Sea $µ(X)<\infty$ y $(f_n)$ una sucesión perteneciente a $L^1(µ)$ tal que $f_n\rightarrow f(x)$ uniformemente.

Demuestra que $f\in L^1(µ)$ y que $\int f_n dµ \rightarrow \int f dµ$.

\textbf{Sugerencia:} Estudia la sucesión $g_n(x)=f_n(x)-f(x)$, escribe $f(x)=f_n(x)-(f_n(x)-f(x))$

\solution

$f_n(x) \to f(x)$ uniformemente implica que
\[\forall \epsilon \exists N \tq \forall x |f_n(x)-f(x)|< \epsilon \forall n > N\]

Esto implica que la sucesión $\epsilon_n(x)=f_n(x)-f(x)$ mencionada en la sugerencia converge uniformemente a 0.

Vamos a empezar viendo que $f\in L^1(µ)$. Para ello, debemos probar que
\[\int |f|dµ < \infty\]
Apoyándonos en la sugerencia del enunciado, vemos que
\[\int |f|dµ = \int |f_n(x)-(f_n(x)-f(x))| \leq \int |f_n(x)|dµ +\int |f_n(x)-f(x)| \leq M + \int \epsilon\]
\textbf{Para la última desigualdad nos basamos en la convergencia uniforme de $f_n \to f$}

Nos queda:
\[M + \int \epsilon = M + \epsilon µ(X) < \infty \text{ por hipótesis del enunciado }\]

Por último tenemos que ver que
\[\int f_n dµ \to \int f dµ\]

Para probarlo vemos que
\[\forall x, \epsilon \exists N \tq \forall n > N \ \int f_n dµ - \int f dµ < \epsilon\]
puesto que $\epsilon_n \in L^1$
\[\int f_n dµ - \int f dµ < \epsilon \iff \int f_n dµ - \int f_n dµ - \int (f_n-f)dµ < \epsilon \iff  \int (f_n-f)dµ < \epsilon\]

Sabiendo que el espacio $X$ tiene medida finita y que $f_n \to f$ de manera uniforme vemos que la última desigualdad es trivialmente cierta si escogemos $\epsilon$ y $N$ de manera adecuada

\end{problem}


\begin{problem}[18]
Demuestra que
\[\lim_{n\to\infty}\int_0^{\infty}\frac{dx}{(1+\frac{x}{n})^n x^{\frac{1}{n}}} = 1\]

\solution
\textcolor{blue}{Completado por mi. No fiarse al 100\%}

Si pudiésemos intercambiar el límite con la integral ya tendríamos el problema resuelto pues
\[\lim_{n\to\infty}\int_0^{\infty}\frac{dx}{(1+\frac{x}{n})^n x^{\frac{1}{n}}} = \int_0^{\infty}\lim_{n\to\infty}\frac{dx}{(1+\frac{x}{n})^n x^{\frac{1}{n}}}=\int_0^{\infty}\frac{dx}{e^x}=\int_0^{\infty}e^{-x}=1\]

Por tanto, lo único que tenemos que hacer es probar que podemos intercambiar el límite con la integral.
\end{problem}

\begin{problem}[19]
Demostrar que
\[\int_0^1\frac{x}{1-x}\log\frac{1}{x}dx = \sum_{n=2}^{\infty}\frac{1}{n^2}\]

\textbf{Sugerencia:} Usa el hecho de que para x$\in(0,1)$ se tiene que $\frac{1}{1-x}=\sum_{n=0}^{\infty}x^n$

\solution
\textcolor{blue}{Completado por mi. No fiarse al 100\%}

Atendiendo a la sugerencia del enunciado, tenemos que
\[\int_0^1\frac{x}{1-x}\log\frac{1}{x}dx = -\int_0^1\sum_{n=0}^{\infty}x^{n+1}\log(x) = -\int_0^1 \sum_{n=1}^{\infty}x^n\log(x) =  -\sum_{n=1}^{\infty}\int_0^1 x^n\log(x)\]

Veamos ahora que pasa con la integral indicada. A vista de buen cubero podemos llegar a
\[\int_0^1x^n\log(x)=\frac{x^{n+1}\log(x)}{n+1}-\frac{x^{n+1}}{(n+1)^2}\]
y como estábamos integrando entre 0 y 1, tenemos
\[\int_0^1x^n\log(x)=-\frac{1}{(n+1)^2}\]

Volviendo al sumatorio con el que empezamos, vemos que
\[-\sum_{n=1}^{\infty}\int_0^1 x^n\log(x) = \sum_{n=1}^{\infty}\frac{1}{(n+1)^2} = \sum_{n=2}^{\infty}\frac{1}{n^2}\]
\end{problem}

\begin{problem}[20]
Sea $(f_n)$ la sucesión de funciones medibles definida por
\[f(x)= \left\{ \begin{array}{lcc}
             n\cos(nx) &   si  & x \in [-\frac{\pi}{2n}, \frac{\pi}{2n}] \\
             \\ 0 &  si  & x \notin [-\frac{\pi}{2n}, \frac{\pi}{2n}]
             \end{array}
   \right.\]

Estudia si
\[\lim_{n\to \infty}\int_{-\pi}^{\pi}f_n(x) dx = \int_{-\pi}^{\pi}\lim_{n\to \infty} f_n(x)dx\]

Estudia si pueden aplicarse en este caso los teoremas de convergencia monótona y convergencia dominada de Lebesgue

\solution

Vamos a ver si se da la conmutatividad de integral y límite.
\[\lim_{n\to \infty}\int_{-\pi}^{\pi}f_n(x) dx = \lim_{n\to \infty} \sin(nx)|_{-\frac{\pi}{2n}}^{\frac{\pi}{2n}}=\lim_{n\to \infty} 0 = 0\]
Por otro lado
\[\int_{-\pi}^{\pi}\lim_{n\to \infty} f_n(x)dx = \int_{-\pi}^{\pi} 0 = 0\]

Por tanto, si que se da la igualdad.

Para aplicar el teorema de convergencia monótona necesitamos tener una sucesión de funciones medibles (lo tenemos), crecientes (nos falla).

Cada $f_n$ tiene más puntos en los que vale 0 que $f_{n-1}$ y puesto que son todas positivas, no pueden formar una sucesión creciente.

Para el teorema de la convergencia dominada necesitamos que las funciones sean $L^1$, lo cual es cierto puesto que son positivas y la integral del valor absoluto coinciden con la integral sin valor absoluto que, como acabamos de calcular para el apartado anterior, es finita. Puesto que la sucesión es decreciente, está acotada por $f_1$, que también es $L^1$.

Tenemos todas las condiciones necesarias, por lo que podemos aplicar el teorema obteniendo que el límite también es $L^1$ y que, efectivamente, conmutan límite e integral.
\end{problem}
\section{Hoja 7}
\begin{problem}[1]
Sean $X=Y=\nat$, $\algbM=\algb{N}=\algbP(\nat)$ y $µ,\upsilon$ medidas discretas. Se define
\[f(m,n)= \left\{ \begin{array}{lcc}
             1 &   \text{si m=n} \\
             \\ -1 &  \text{si m=n+1} \\
             \\ 0 &  \text{en otro caso}
             \end{array}
   \right.\]

Comprueba que $\int |f|d(µ\times \upsilon)=\infty$ y que $\int(\int f dµ)d\upsilon$ y $\int(\int f d\upsilon)dµ$ existen y son distintas.
\solution
\textcolor{blue}{Completado por mi. No fiarse al 100\%}

\[\int |f|d(µ\times \upsilon)=\sum_{n=0}^{\infty} \sum_{m=0}^{\infty}|f(n,m)| = \sum_{n=0}^{\infty} 2 = \infty\]

Vamos ahora a calcular las integrales iteradas que pedía el enunciado
\[\int\left(\int f dµ\right)d\upsilon = \sum_{n=0}^{\infty} \sum_{m=0}^{\infty}f(n,m) = \sum_{n=0}^{\infty} 0 = 0 \]
En este caso $\sum_{m=0}^{\infty}f(n,m) = 0$ ya que para cada valor de n fijado sólo hay dos puntos en los que $f(m,n)$ tome un valor distinto de 0 y dichos valores son 1,-1.

Por último
\[\int\left(\int f d\upsilon\right)dµ = \sum_{m=0}^{\infty} \sum_{n=0}^{\infty}f(n,m) = 1 + \sum_{m=1}^{\infty} \sum_{n=0}^{\infty}f(n,m) = 1+ \sum_{m=0}^{\infty} 0 = 1 \]

Para cada valor de $m > 0$ que fijemos, vamos a tener dos puntos donde $f(m,n)$ es no nula, con valores 1 y -1, por lo que obtenemos $\sum_{m=0}^{\infty} 0 $. Sin embargo, cuando la $m=0$ sólo hay un valor posible de $n$ (que siempre tiene que ser positivo) donde la función no sea nula. Este punto es $n=m$ y en él la función vale 1.
\end{problem}

\begin{problem}[2]
Sean $X=Y=[0,1], \ A_1=A_2=\algbB_{[0,1]}$, µ la medida de Lebesgue en $A_1$, $ν$ la medida de contar en $A_2$. En el espacio de medida $(X\times Y, A_1 \otimes A_2, µ \times ν)$ se considera el conjunto $V = \{(x,y): x=y\}$.
Comprueba que $V \in A_1 \otimes A_2$ y sin embargo
\[\int_Y d ν \int_X \ind_V dµ = 0\]
y
\[\int_X d µ \int_Y \ind_V d\upsilon = 1\]

¿Qué hipótesis del teorema de Fubini no se cumple?

\solution

\textcolor{blue}{Completado por mi. No fiarse al 100\%}

$V$ es medible porque es union numerable de conjuntos medibles. Gráficamente se ve fácil. $V$ es la diagonal de un rectángulo y es recubrible por rectángulos producto que son medibles.


Una vez visto que es medible, vamos a calcular las 2 integrales

\[\int_Ydν \underbrace{\int_X \ind_V(x,y)dm(x)}_{h(y)} =  0 \]

\[h(y) = \int_X \ind_V(x,y) = \int0=0\]

\textbf{Razón:} $h(y)$ depende de $y$, y para cada valor de $y$ hay un único punto $(x,y)$ en el que la indicatriz toma valor, que es $(y,y)$. Es por esto, y por ser $m(x)$ la medida de Lebesgue, que la integral vale 0.


Ahora vamos con la otra integral iterada:

\[
\int_Xdµ \underbrace{\int_Y \ind_V(x,y)dν(x)}_{g(y)} = 1
\]

$g(x) = \int_Y \ind_V(x,y) dν(y) = 1$, por ser $ν$ la medida de contar, ya que para cada $x$ hay un punto en $V(x,y)$, con lo que $\ind_V$ sólo toma valor en un único punto ($(x,x)$)

La hipótesis del teorema de Fubini que no se cumple es que los conjuntos sean $\sfin$s, ya que $\upsilon(Y)=\infty$
\end{problem}

\begin{problem}[3]
Sean $X, \algbM, µ)$ y $(Y, \algb{N}, \upsilon)$ espacios de medida $\sfin$s. Sea $\appl{f}{(X, \algbM)}{\cplex}$, $\appl{g}{(Y, \algb{N})}{\cplex}$ medibles y $h$ definida mediante $h(x,y)=f(x)g(y)$.

\ppart
Demuestra que $h$ es $\algbM\otimes\algb{N}$-medible.

\ppart
Demuestra que si $f\in L^1(µ)$ y $g \in L^1(\upsilon)$ entonces $h\in L^1(µ \times \upsilon)$ y además $\int hd(µ\times\upsilon)=\int f dµ \int f d\upsilon$

\solution

\textcolor{blue}{Completado por mi. No fiarse al 100\%}

\spart
Definimos las funciones
\[\appl{f'}{(X,Y)}{\cplex}\]
\[\appl{f'}{(x,y)}{f(x)}\]
\[\appl{g'}{(X,Y)}{\cplex}\]
\[\appl{g'}{(x,y)}{g(x)}\]
ambas $\algbM\otimes\algb{N}$-medibles puesto que, siendo $A$ un conjunto medible en $\cplex$ y sabiendo que $f,g$ son funciones meidbles, tenemos
\[(f')^{-1}(A)=f^{-1}\times Y \text{ medible por ser producto de medibles }\]
\[(g')^{-1}(A)=g^{-1}\times Y \text{ medible por ser producto de medibles }\]

Ahora es sencillo ver que
\[\appl{h}{(X,Y)}{f'(x)\cdot g'(x)}\]
es $\algbM\otimes\algb{N}$-medibles por ser el producto de dos funciones $\algbM\otimes\algb{N}$-medibles.


\spart

Para ver que $h\in L^1$ debemos comprobar que
\[\int |h|d(µ\times\upsilon) < \infty\]

Vamos a ello
\[\int |h|d(µ\times\upsilon) = \int |f||g|d(µ\times \upsilon) \underbrace{=}_{Aplicamos \ Tonelli} \int_X\int_Y |f(x)||g(y)| dµ d\upsilon =\]
\[= \int_X |f(x)| dµ \int_Y |g(y)| d \upsilon = M_X \cdot M_Y = M < \infty \implies |h| \in L^1(µ\times \upsilon)\]

Recordemos que para aplicar Tonelli necesitábamos $|h|\in L^+$, es decir, necesitamos $|h|$ medible y positiva. Obviamente es positiva, puesto que se trata de un valor absoluto para ver que es medible tomamos un conjunto $A$, medible en $\cplex$, cuya inversa será $|h|^{-1}(A)\cup[0,\infty)$ que es intersección de conjuntos medibles (i.e. pertenecientes al álgebra de Borel en $\cplex$)

Teniendo ahora que $h\in L^1$ podemos aplicar el teorema de Fubini, obteniendo:
\[\int h \ d(µ\times\upsilon) = \int f\cdot g \ d(µ\times \upsilon) \underbrace{=}_{Aplicamos \ Fubini} \int_X\int_Y f(x)g(y) dµ d\upsilon =\]
\[= \int_X f(x) dµ \int_Y g(y) d \upsilon \]
\end{problem}

\begin{problem}[4]
Sea $\appl{f}{[-1,1]\times [-1,1]}{\real}$ definida por
\[f(x,y)= \left\{ \begin{array}{lcc}
             \frac{xy}{(x^2+y^2)^2} &  si & (x,y)\neq (0,0) \\
             \\ 0 &  si & x=y=0
             \end{array}
   \right.\]

Demuestra que las integrales iteradas de $f$ coinciden y valen 0 y, sin embargo, f no es integrable en $[-1,1]\times [-1,1]$. ¿Qué hipótesis del teorema de Fubini no se verifica?

\solution
\textcolor{blue}{Completado por mi. No fiarse al 100\%}

Para ver que las integrales iteradas coinciden y valen 0 basta con observar que, una vez fijado x $f(x,y)$ es una función simétrica respecto del origen y lo mismo ocurre si fijamos y.

La hipótesis del teorema de Fubini que no se cumple es que la función $f(x,y)$ pertenezca a $L^1$, ya que
\[\int_{-1}^{1}\abs{\frac{xy}{(x^2+y^2)^2}} dx= 2 \int_0^1\frac{xy}{(x^2+y^2)^2} dx \geq 2 \int_0^1\frac{xy}{(x+y)^2} = 2 \int_0^1\frac{xy}{x^2+2x+1} dx \geq\]
\[\geq 2 \int_0^1\frac{xy}{x^2+2+1} dx = y\log(x^2+3)|_0^1\]

\textcolor{blue}{No termina de salirme pero apostaría a que, de alguna forma, podemos ver que esta integral es infinita, lo que implicaría que las integrales iteradas difieren o son infinitas y en cualquier caso, eso conlleva que la función no es integrable.}
\end{problem}

\begin{problem}[5]
Demuestra que
\[\int_0^{\infty}e^{-y}\frac{\sin^2(y)}{y}dy=\frac{1}{4}\log(5)\]
\textbf{Sugerencia:} Comprueba en primer lugar que $e^{-y}\sin(2xy)$ es integrable en $A=[0,1)\times(0,\infty)$. Luego aplica Fubini tras comprobar que
\[\int_0^1e^{-y}\sin(2xy)dx=e^{-y}\frac{\sin^2(y)}{y} \text{ y que } \int_0^{\infty}e^{-y}\sin(2xy)dy = \frac{2x}{1+4x^2}\]

\solution
\textcolor{blue}{Hecho por mi. No fiarse al 100 \%}

Si damos por cierto todo lo indicado en la sugerencia es fácil ver que
\[\int_0^{\infty}e^{-y}\frac{\sin^2(y)}{y}dy = \int_0^{\infty}\int_0^1e^{-y}\sin(2xy)dxdy = \int_0^1\int_0^{\infty}e^{-y}\sin(2xy)dydx = \]
\[=\int_0^1\frac{2x}{1+4x^2} =\frac{1}{4}\log(5)\]

Para demostrar que los pasos aplicados aquí son válidos debemos ver que las igualdades dadas en la sugerencia son correctas y que la función $e^{-y}\sin(2xy)$ es integrable para poder aplicar Fubini.

Ver que las igualdades son ciertas es un trabajo sencillo pero cansado, por lo que se deja como ejercicio para el lector. Basta con calcular las integrales dadas como hacíamos en bachillerato.

Lo último que nos quedaría por hacer es comprobar que la función $e^{-y}\sin(2xy)$ es integrable. Nuevamente esto es bastante sencillo pues podemos descomponerla según
\[h(x,y)=e^{-y}\sin(2xy)=\underbrace{\underbrace{e^{-y}\cos(y)}_{f_1}\underbrace{\sin(x)}_{g_1}}_{h_1}+\underbrace{\underbrace{e^{-y}\sin(y)}_{f_2}\underbrace{\cos(x)}_{g_2}}_{h_2}\]

Sabiendo que $f_1,g_1,f_2,g_2$ son funciones integrables, tenemos que $h_1$ y $h_2$ también lo son. Por último, tenemos la función $h$ que es suma de funciones integrables y por tanto es integrable.
\end{problem}