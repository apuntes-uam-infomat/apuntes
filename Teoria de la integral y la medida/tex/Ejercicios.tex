\section{Ejercicios}
\subsection{Hoja 1}
\begin{problem}[5]
Dada una sucesión $\lbrace f_n \rbrace \in R([a,b])$ que converge uniformemente a $f$, se pide demostrar que $f$ es ingrable Riemann y que:
\[ lim \int_{a}^{b} f_n = \int_{a}^{b} f \]

\solution
Supongamos que f es integrable Riemann, entonces tenemos que ver que:
\[ \forall \epsilon > 0 \ , \exists N \tq \forall n> N \ \abs{\int_a^b f_n - \int_a^b f} < \epsilon\]

Sabemos que:
\[\abs{\int_a^b f_n - \int_a^b f} \leq \abs{\int_a^b \abs{f_n -f}} dx\]

Recordemos la definición de convergencia uniforme

\begin{defn}[Convergencia\IS uniforme]
\[f_n \xrightarrow{uniforme} f \Leftrightarrow \forall \epsilon < 0, \exists N_{\epsilon} \tq \forall x \in [a,b]  \forall n \geq N_{\epsilon}, \abs{f_n (x) - f(x)} < \epsilon\]
\end{defn}

Si $n \geq N_{\frac{\epsilon}{b-a}}$ entonces, usando la definición de convergencia uniforme:

\[\int_a^b \abs{f_n(x) - f(x) dx} \leq \int_a^b \frac{\epsilon}{b-a}dx = \epsilon\]

Por tanto queda claro que si $f$ es integrable Riemann podemos conmutar el límite con la integral. Ahora queda ver por qué $f$ es integrable Riemann.

$f$ será integrable Riemann sii:
\[\forall \epsilon > 0 \ \exists P \tq \forall P' \prec P \]
\[\overline{J}_{P'}(f) - \underline{J}_{P'}(f) < \epsilon\]

Vamos a probarlo:
\[\overline{J}_P(f) - \underline{J}_P(f) = \sum(\sup_k f - \underset{k}{inf} f)\abs{I_k} \leq\]
\[\leq \sum\left( \abs{\sup_{k}f_n(x) - \sup_{k}f(x)} + \abs{\sup_{k}f_n(x) - \underset{k}{inf}f_n(x)} +  \abs{\underset{k}{inf}f_n(x) - \underset{k}{inf}f_n(x)} \right)\abs{I_k} \leq \]
Puesto que $f_n$ converge uniformemente a $f$ habrá un $n$ a partir del cual la distancia máxima entre $f_n$ y $f$ sea $\frac{\epsilon}{6}$ y por tanto la distancia máxima entre el supremo y el ínfimo de $f_n$ será menor que $\frac{\epsilon}{3}$
\[\leq \sum\left( \sup_{k} \abs{f_n(x) - f(x)} + \frac{\epsilon}{3} + \sup_k \abs{f_n(x) - f(x)} \right)\abs{I_k} \leq\]
Aplicando de nuevo la convergencia uniforme y tomando el máximo entre este $n$ y el calculado en el paso interior nos queda:
\[\leq \sum \left(\frac{\epsilon}{3} + \frac{\epsilon}{3} + \frac{\epsilon}{3} \right)\abs{I_k} = \sum\epsilon\abs{I_k}\]

Como hay convergencia uniforme entre $f_n$ y $f$ podemos hacer que los supremos sean tan pequeños como queramos y hacer así que el interior sumatorio quede menor que $\epsilon \abs{I_k}$

Puesto que $\epsilon$ es un número cualquiera podemos hacerlo tan pequeño como queramos haciendo que el último sumatorio escrito tienda a 0.

\end{problem}

\begin{problem}[6]

Sea $\lbrace f_n \rbrace$ una sucesión monótona creciente de funciones continuas en un intervalo $I=[a,b]$ que convergen en dicho intervalo a otra función continua $f$. Demuestra que entonces:
\[ lim \int_{a}^{b} f_n(x) = \int_{a}^{b} f(x) \]
\solution
Por ser funciones continuas son integrables Riemann. Si conseguimos demostrar que convergen uniformemente podemos emplear el ejercicio anterior y lo tendríamos hecho.
\end{problem}

\begin{problem}[7]
Dada la sucesión $I_k = (a_k, b_k)$ tales que $\bigcup_{k=1}^{N}~I_k~=~[a,b]$

Demostrar que:
\[b-a \leq \sum_{k=1}^N (b_k - a_k)\]

\solution
Vamos a utilizar la integral de Riemann como recomienda el ejercicio, utilizando la función indicatriz de cada intervalo $\ind_{I_k}$

Está claro que la función indicatriz del intervalo I es menor o igual que la suma de las funciones indicatrices de los intervalos. Lo cual es obvio, ya que si $x$ está en el intervalo, $x$ estará también en al menos uno de los intervalos $I_k$. Es decir:
\[\ind_{[a,b]} \leq \sum_{k=1}^{N} \ind_{I_k}\]

Utilizando la monotoreidad de la integral de Riemann podemos ``integrar a ambos lados`` obteniendo:

\[\int_{a}^{b} \ind_{[a,b]} \leq \int_{a_k}^{b_k} \sum_{k=1}^{N} \ind_{I_k}\]

Por la linealidad de la integral, podemos incluso meter la integral denro del sumatorio
\[\int_{a}^{b} \ind_{[a,b]} \leq \sum_{k=1}^{N} \int_{a_k}^{b_k} \ind_{I_k}\]


Conociendo la integral de la función indicatriz tenemos el resulado de forma inmediata.
\[b-a \leq \sum_{k=1}^{N} ( b_k - a_k )\]

\end{problem}

\begin{problem}[8/9]
Dado $O\subset (a,b)$ unión numerable de intervalos disjuntos, se pide demostrar:
\[m^*(O)=\sum_{n=1}^{\infty} \abs{I_k}\]

\solution
Está claro que:
\[m^*(O) \leq \sum_{n=1}^{\infty} \abs{I_k}\]

Tenemos que demostrar la desigualdad contraria para poder concluir la igualdad.
Vamos a probar que:
\[m^*(O) \geq \sum_{n=1}^{\infty} \abs{I_k} - \epsilon\]

Puesto que sabemos que la suma infinita tiene un resultad finito (ya que $O$ es finito)
\[\forall \epsilon > 0 \exists N \tq \sum_{n=1}^N\abs{I_n} \geq \sum_{n=1}^{\infty} \abs{I_k} - \frac{\epsilon}{2}\]

Vamos a definir los intervalos $I'_n$ que serán un poco más pequeños que los $I_n$
\[\forall n=1,2,...,N \ definimos \ I'_n=[a_n+\frac{\frac{\epsilon}{2}}{2^{n+1}}, b-\frac{\frac{\epsilon}{2}}{2^{n+1}}]\]

Ahora tenemos que:
\[\sum_{n=1}^N \abs{I'_n}=\sum_{n=1}^N\abs{I_n} - \frac{\epsilon}{2}\sum_{n=1}^{\infty}\frac{1}{2^n} \geq \sum_{n=1}^N\abs{I_n} - \frac{\epsilon}{2}\]

Definimos ahora el compacto:
\[K_n = \bigcup_{n=1}^N I'_n\]

Sean $J_n$ intervalos abiertos tales que $O \subset \bigcup_{n=1}^{\infty}J_n \Rightarrow K_n \subset \bigcup_{n=1}^{\infty}J_n$.

Por tanto
\[\exists N' \tq K_n \subset \bigcup_{n=1}^{N'}J_n=A_N\]

Vamos a fijarnos ahora en la medida exterior de $O$:

\[ m^*(O) = \inf\lbrace \sum_{n=1}^{\infty}\abs{J_n} \rbrace \geq \inf\lbrace \sum_{n=1}^{N'}\abs{J_n} \rbrace \geq \]
\[ \geq \inf\lbrace \int \ind_{A_N} \rbrace \geq \inf\lbrace \int \ind_{K_N} \rbrace \geq \inf\lbrace \sum_{n=1}^N\abs{I'_n} \rbrace \geq \]
\[ \geq \inf\lbrace \sum_{n=1}^N\abs{I_n} -\frac{\epsilon}{2}\rbrace \geq \inf\lbrace \sum_{n=1}^N\abs{I_n} -\epsilon \rbrace \]

Y obtenemos así la desigualdad buscada.
\end{problem}

\begin{problem}[10/9]
Dado un compacto K contenido en el intervalo (a,b), nos piden demostrar que:
\[m(K)<b-a\]

\solution
Consideremos la sucesión: $(a+\frac{1}{n}, + - \frac{1}{n})$, con $n >\frac{2}{b-a}$. Obviamente:
\[K \subset \bigcup_{n=n_0}^{\infty}(a+\frac{1}{n}, + - \frac{1}{n})\]

Como K es compacto:
\[\exists N \tq K \subset \bigcup_{n=n_0}^{N}(a+\frac{1}{n}, + - \frac{1}{n}) = (a+\frac{1}{N}, + - \frac{1}{N}) \]

Y llegamos a:
\[m(K) \leq b-a-\frac{2}{N} < b-a\]

\end{problem}

\begin{problem}[11/10]
Sea $C_n$ una sucesión creciente de subconjuntos medibles contenidos en (a,b) y sea $C$ la unión de estos subconjuntos. Se pide demostrar que:
\[m(C_n) \rightarrow m(C)\]

\solution
Vamos a definir la sucesión $D_n$ como:
\[D_1=C_1 \ D_2 = C_2 \setminus D_1 \ D_3 = C_3 \setminus (C_1 \bigcup C_2) \ ...\]

Teniendo así C expresado como unión de los $D_n$, que son disjuntos. Así, la medida de $C$ queda expresada como:
\[m(C)=\sum_{n=1}^{\infty}m(D_n)\]
sabiendo que:
\[m(C_N)=\sum_{n=1}^{N}m(D_n)\]
\end{problem}

\begin{problem}[12/11]
Sea $C_n$ una sucesión decreciente de subconjuntos medibles contenidos en (a,b) y sea $C$ la intersección de estos subconjuntos. Se pide demostrar que:
\[m(C) \rightarrow m(C)\]
\solution

Tomando los complementarios, que también son medibles, tenemos una sucesión creciente de subconjuntos medibles contenidos en (a,b). Aplicando el ejercicio anterior llegamos a que:
\[(b-a)-m(C_n) \rightarrow (b-a)-m(C)\]
De donde puede deducirse que $m(C_n)$ decrece hacia $m(c)$.
\end{problem}

\begin{problem}[13/14]
Dada una sucesión $A_n$ contenia en (a,b), se pide demostrar que:
\[\lim_n(A_0\Delta A_n)=0 \Rightarrow \lim_n m(A_n)=m(A_0) \]

Recordemos que:
%TODO copiar definición de la diferencia asimétrica esta asquerosa
\[caca\]
\solution
\[\lim_n(A_0\Delta A_n)=0 \Rightarrow \lim_n m(A_0 \bigcap A_n)=0 \wedge \lim_n m(A_0\bigcap A_n^c)=0\]

Escribimos ahora $A_n$ y $A_0$ como:
\[A_n = (A_n \bigcap A_0) \bigcup (A_n \bigcap A_n^c)\]
\[A_0 = (A_0 \bigcap A_n) \bigcup (A_0 \bigcap A_n^c)\]

De aquí podemos ver que:
\[m(A_n) = m(A_n \bigcap A_0) + m(A_n \bigcap A_n^c)\]
\[m(A_0) = m(A_0 \bigcap A_n) + m(A_0 \bigcap A_n^c)\]

Y restando llegamos a:
\[m(An) = m(A_0) -m(A_0 \bigcap A_n^c) + m(A_n \bigcap A_n^c)\]

Y sabemos que las medidas de estas intersecciones tienden a 0.
\end{problem}
