\documentclass{apuntes}
\author{Guillermo Julián Moreno}
\date{13/14 C1}
\title{Estadística I}

\begin{document}

\pagestyle{plain}
\maketitle

\tableofcontents
\newpage

\section{Estadística descriptiva de datos univariantes}

La estadística descriptiva es el conjunto de técnicas para resumir la información proporcionada por una gran masa de datos. El primer objetivo natural es resumir la información que proporcionan esos datos.

\subsection{Estadísticos de tendencia central}
 
\begin{defn}[Media]

\[ \avg{x} = \frac{\sum_{i=1}^n x_i}{n} \]

Es la medida de tendencia central más utilizada. Es bastante sensible a los valores atípicos (\textit{outliers}), observaciones anormalmente grandes que aparecen en el conjunto de datos por errores de transcripción o medición.

\end{defn}

\begin{defn}[Mediana]
Es el valor que divide a los datos en dos mitades, de tal forma que la mitad son menores y la otra mitad mayores que la mediana. 

La mediana se calcula de la siguiente forma: dado un conjunto de datos $\{x_1,\dotsc, x_n\}$, la mediana es $x_{\frac{n+1}{2}}$ si $n$ es impar y  el promedio entre $x_{\frac{n}{2}}$ y $x_{\frac{n}{2} + 1}$.
\end{defn} 

\subsection{Estadísticos de dispersión}

\begin{defn}[Varianza]
\[ \sigma^2 = \frac{1}{n} \sum_{i=1}^n \left(x_i - \avg{x}\right)^2 = \frac{1}{n} \sum_{i=1}^n x_i^2 - \avg{x}^2 \]
\end{defn}

\begin{defn}[Desviación\IS típica]
\[\sigma = \sqrt{\sigma^2} \]

La desviación típica es la raíz de la varianza.
\end{defn}

\begin{defn}[Cuantil]
Para $p\in (0, 1)$ se llama cuantil $p$ o $q_p$ al valor que deja el $100p \%$ de los datos a la izquierda.
\end{defn}

\begin{defn}[Cuartil]
Los cuartiles son los tres datos que dejan a la izquierda el 25, 50 y 75 por ciento de los datos respectivamente. Es decir:

\begin{itemize}
\item $Q_1 = q_{0.25}$
\item $Q_2 = q_{0.5}$. El cuartil dos es la mediana.
\item $Q_3 = q_{0.75}$
\end{itemize}
\end{defn}

Hay varios métodos para el cálculo de cuantiles. Para hacerlo a mano, podemos usar el siguiente método.

Si el dato en la posición $p(n+1)$ no es un número entero, entonces se interpola entre las observaciones ordenadas que están en la posición $\floor{p(n+1)}$ y $\floor{p(n+1)} + 1$ de la siguiente forma: sea $j$ la parte entera de $p(n+1)$ y $m$ la parte decimal. Entonces, \[ q_p = (1-m)x_j + m x_{j+1} \]


\begin{defn}[Coeficiente\IS de asimetría]
\index{Skewness}
El tercer momento con respecto a la media se define como \[ \frac{1}{n}\sum_{i=1}^n\left(x_i-\avg{x}\right)^3 \] que, en su versión adimensional dividimos por $\sigma^3$.
\end{defn}

Al ser una función cúbica, los valores que se alejen mucho de la media tendrán un valor muy alto en valor absoluto (positivo o negativo según se aleje por la derecha o izquierda, respectivamente). Si la distribución de datos es muy asimétrica, los valores más altos no se cancelan con los valores altos del otro lado (porque no hay) y saldrá un valor más alejado de cero.\footnote{Está explicado como el p. culo, ya.}

\subsection{Representación gráfica de datos}

\begin{defn}[Box-plot]
El diagrama de caja o \textit{box-plot}  (imagen \ref{imgCaja}) nos permite visualizar las medidas de dispersión respecto a la mediana. Hay que añadir una nueva medida, el \textbf{rango intercuartílico}\index{Rango!intercuartílico}, la diferencia entre el primer y el tercer cuartil: \[RI = Q_3 - Q_1 \]

\easyimg{DiagramaCaja.png}{Diagrama de caja}{imgCaja}
\end{defn}

\begin{defn}[Histograma]
El histograma se trata de una aproximación discreta a la función de densidad continua $f(t)$ de la variable que estamos midiendo. Es un diagrama de frecuencias que \textit{mantiene la forma} de esa función de densidad. 

Definimos una serie, las marcas de intervalos $a^n_1, \dotsc, a^n_n$, donde $n$ es el número de intervalos y la longitud de cada intervalo  es $h_n = a^n_{j+1} - a^n_j$. Sea el conjunto $\{x_i\}_{i=0,\dotsc,m}$ los datos de nuestra muestra. Entonces, el estimador, la función $\hat{f}_n$, se define de la siguiente forma:

\[ \hat{f}^n(t) = \frac{\card{i \tq x_i \in \left( a_j^n, a_{j+1}^n \right]}}{n h_n} = \frac{\sum_{i=1}^m \ind_{(a_j^n, a_{j+1}^n]} (x_i)}{n h_n} \]

Recordemos que \[ \ind_A (n) = \begin{cases} 1 & n \in A \\ 0 & n \notin A\end{cases}\]

A grandes rasgos, lo que hace en una función es definir un número de intervalos fijos de ancho $h_n$. Al evaluar $\hat{f}^n(t)$ buscamos en qué intervalo cae $t$ y contamos cuántas de nuestras mediciones caen también en ese intervalo.

\easyimg{DensidadAHistograma.png}{El histograma es una aproximación de la función de densidad real en base a la muestra que hemos obtenido.}{lblDensidad}

\end{defn}

\subsubsection{Estimadores núcleo o kernel}

\begin{defn}[Método de ventana móvil][Ventana móvil]
El método de ventana móvil nos da una estimación de la función de densidad en un punto $t$ midiendo los $x_i$ que están en el intervalo de radio $h_n$ centrado en $t$. Matemáticamente:

\[ \hat{f}_n(t) = \frac{1}{n2h_n}\sum_{i=1}^n \ind_{[t-h_n, t+h_n]}(x_i) = \frac{1}{n2h_n}\sum_{i=1}^n \ind_{[-1,1]}\left(\frac{t-x_i}{h_n}\right) \]
\end{defn}

Podemos reemplazar la función $\frac{1}{2}\ind_{[-1, 1]}$ por otra, llamada la función de densidad $K$, kernel o núcleo:

\begin{defn}[Estimador\IS núcleo]
Dada una función de densidad $K$ simétrica, no necesariamente positiva, definimos el estimador kernel como:

\[ \hat{f}_n(t) = \frac{1}{n}\sum_{i=1}^n K_h (t - x_i)  = \frac{1}{nh_n} \sum_{i=1}^n K\left(\frac{t-x_i}{h_n}\right) \]

con $K_h(x) = \frac{1}{h}K(\frac{x}{h})$.
\end{defn}

La elección del núcleo $K$ no afecta especialmente a lo bien aproximada que esté la función de densidad. Sin embargo, sí que influye la selección de la ventana $h_n$ (figura \ref{lblSuavizado}), también llamada \textit{bandwith} en inglés.  Si escogemos una ventana muy pequeña, damos demasiado peso a los datos de nuestra muestra. Si elegimos una ventana muy grande, nuestra muestra pierde importancia y podemos perder información importante.

La elección del $h_n$ más habitual es el que minimiza la distancia $L^2$ entre $\hat{f}$ y $f$, es decir, el parámetro que minimice $\displaystyle\int\left(\hat{f}_h-f\right)^2$. Sin embargo, hay un problema: no sabemos qué es $f$. Hay trucos que imagino que veremos más tarde.

\easyimgw{Suavizado.png}{Los efectos que causa elegir una ventana más grande o más pequeña en el estimador}{lblSuavizado}{1}

Las funciones kernel más usadas son la uniforme, $\frac{1}{2}\ind_{[-1, 1]}$, la gaussiana $\frac{1}{\sqrt{2 \pi}}e^{-\frac{t^2}{2}}$ y la de Epanechnikov, que matemáticamente es la que mejor aproxima $f$.

El estimador kernel $\hat{f}_n(t)$ es la función de densidad de una medida de probabilidad que es la convolución \footnote{Ya aprenderemos en al algún momento de nuestra vida qué narices es una convolución} de dos medidas de probabilidad: una, $K_h(x)$ (el kernel reescalado) y otra que da probabilidad $\frac{1}{n}$ a cada punto de la muestra $\{x_i\}$ (distribución o medida empírica).

\paragraph{Generación de datos del estimador kernel} Supongamos que $K$ es el núcelo gaussiano. Podemos generar datos artificiales de la densidad así:

\[ x_i^0 = x_i^* + h_n Z_i,\; i=1,\dotsc, k \]

donde $x_i^*$ es una observación elegida al azar entre los datos originales y $Z_i$ una observación aleatoria con probabilidad $N(0,1)$. Es decir, lo que hacemos es añadir un dato aleatorio de la muestra y sumamos una pequeña perturbación aleatoria.

\section{Estadística descriptiva de datos bivariantes}

En esta sección estudiaremos dos variables $(X, Y)$ para explorar la relación entre ambas y tratar de inferir si existe una relación funcional para predecir los valores de una variable en función de los de la otra.

\subsection{Representación gráfica}

\begin{defn}[Diagrama\IS de dispersión]
El diagrama de dispersión representa cada variable en función de la otra para que podamos ver la posible relación entre ambas. Ver figura \ref{lblDispersion}.

\easyimg{Dispersion.png}{Diagrama de dispersión}{lblDispersion}
\end{defn} 

\subsection{Regresión}

\begin{defn}[Recta de regresión]

La recta de regresión de $y$ sobre $x$ es la recta de forma $\hat{y} = \hat{a} + \hat{b}x$ que más se aproxima a los datos, minimizando los cuadrados de la distancia: \[ (\hat{a},\hat{b}) =\argmin_{a, b} \sum_{i=1}^n\left(y_i - a - bx_i)\right)^2 \]
\end{defn}

La recta de regresión se calcula obteniendo primero $\hat{b}$:

\[ \hat{b} = \frac{\sigma_{x,y}}{\sigma^2_x} \]

donde \[ \sigma_{x,y} = \frac{1}{n} \left( \sum_{i=1}^n x_i y_i\right)  - \avg{x}\avg{y} \] y después, sabiendo que la recta pasa por el punto $(\avg{x}, \avg{y})$, obtenemos $\hat{a}$ \[ \hat{a} = \avg{y} - \hat{b}\avg{x} \]

El valor $b$ se denomina \textbf{coeficiente de regresión lineal}\index{Regresión lineal!coeficiente de} o parámetro de la regresión. Cada valor $e_i= y_i - \hat{y}_i$ se denomina \textbf{residuo}\index{Residuo}. Hay que notar que

\begin{gather*}
 \sum_{i=1}^n e_i = \sum_{i=1}^n \left(y_i - \hat{a} -\hat{b}x_i \right)= \sum_{i=1}^n\left( y_i - (\avg{y} - \hat{b}\avg{x}) - \hat{b}x_i \right) = \\
 = \sum_{i=1}^n  \left(y_i - \hat{b}x_i\right) - n\avg{y}  + n\hat{b}\avg{x} = n\avg{y} - n \hat{b}\avg{x}- n\avg{y} + n\hat{b}\avg{x} = 0 \end{gather*}

Esta ecuación ($\sum_{i=1}^n e_i = 0$) junto con \[ \sum_{i=1}^n x_i e_1 = 0 \] son las dos restricciones entre los residuos que nos dan la recta.

\begin{defn}[Varianza\IS residual]
La varianza residual $s_R^2$ o $\hat{\sigma}_e^2$ mide, aproximadamente el \textit{error cuadrático} cometido en la aproximación dada por la recta de regresión:

\[ s_R^2 = \hat{\sigma}_e^2 = \frac{1}{n}\sum_{i=1}^n e_i^2 \]
\end{defn}

\begin{defn}[Coeficiente\IS de correlación lineal]
\index{Coeficiente!de Pearson}
El coeficiente de correlación lineal o coeficiente de Pearson

\[ r = \frac{\hat{\sigma}_{x,y}}{\hat{\sigma}_x \hat{\sigma}_y} \] que cumple las siguientes condiciones:

\begin{gather*}
0 ≤ r^2 ≤ 1 \\
\hat{\sigma}_e^2 = \hat{\sigma}_y^2(1-r^2) \\
r = \hat{b}\frac{\hat{\sigma}_x}{\hat{\sigma}_y} 
\end{gather*}

nos indica el grado de ajuste lineal entre las dos variables. Un valor absoluto más cercano a 1 indica una correlación más fuerte. Un valor absoluto cercano a cero indica una correlación débil. El signo, positivo o negativo, indica si la correlación es creciente o decreciente.
\end{defn}


\section{Muestreo aleatorio}

La muestra aleatoria de una cierta v.a. $X$ se denomina como la \textbf{muestra aleatoria} o simplemente \textbf{muestra}.\index{Muestra}

Durante este tema, usaremos conceptos de Probabilidad, que repasaré aquí brevemente porque no me apetece escribir demasiado.

\subsection{Conceptos de probabilidad}

\begin{defn}[Distribución de una v.a.][Distribución]
\[ P_X(B) = P(X \in B) \]
\end{defn}

\begin{defn}[Función\IS de distribución]
\[F(t) = P(X ≤ t) \]
\end{defn}

\begin{defn}[Media\IS de una distribución] \index{Esperanza} También llamada esperanza de X:
\[ E(X) = \int_{-\infty}^\infty F(t)\,dt \]
\end{defn}

\begin{theorem}[Teorema\IS de cambio de espacio de integración] Sea $g$ una función real medible tal que $E(g(X))$ es finita, entonces 

\[ E(g(X)) = \int_\real g(x) \, dF(x) = \int_\real g(x)\, dP(x) \]. 

En particular \[ µ =\int_\real x\, dF(x)  \] y \[ \sigma^2 = \int_\real \left(x - µ\right)^2 \, dF(x) \]
\end{theorem}

\begin{defn}[Momento] El momento $µ_k$ es la esperanza de X elevado a una potencia de orden $k$. Es el valor esperado de la distancia de orden $k$ con respecto a la media

\[ µ_k = E\left((X-µ)^k\right) \]
\end{defn}

\subsubsection{Distribuciones aleatorias}

Ver apéndice \ref{secDistr} (página \pageref{secDistr}).

\subsubsection{Criterios de convergencia}

Queremos buscar convergencias entre variables aleatorias.

\begin{defn}[Convergencia\IS en distribución]\index{Convergencia!débil} Se dice que $X_n$ converge débilmente o en distribución a $X$ si la función de distribución de $X_n$ $F_n(x)$ tiende a $F(x)$ para todo $x$ punto de continuidad de $F$, donde $F$ y $F_n$ son las funciones de distribución de $X$ y $X_n$ respectivamente.

Esto es equivalente a decir que  \[ \lim_{n\to\infty} P(X_n\in (-\infty, x]) = P(X\in (-\infty, x]) \]

Notación:
\[ X_n  \convdist X \text{ ó }  X_n \convs[w] X \] 
\end{defn}

\begin{defn}[Convergencia\IS en probabilidad] 
Se dice que $X_n$ converge en probabilidad a $X$ si $\forall \epsilon > 0$ se tiene que \[ P(\abs{X_n-X} > \epsilon) \convs 0 \]. Es decir, que para cualquier error que tomemos el error cometido en la aproximación va a tender a cero siempre que tome un $X_n$ suficientemente grande.

Notación: \[ X_n \convprob X \]
\end{defn}

\begin{defn}[Convergencia\IS casi segura] También denotada c.s o a.s en inglés, convergencia en casi todo punto (c.t.p) o convergencia con probabilidad 1. Se dice que $X_n$ converge a $X$ casi seguro si el conjunto de puntos que no son convergentes tiende a ser vacío. Es decir \[ \prob{X_n \convs X} = 1\]

Más estrictamente, la condición se expresa como \[\prob{\omega \in \Omega\tq X_n(\omega) \convs X(\omega)} = 1\]

Notación \[ X_n\convcs X \]
\end{defn}


\begin{theorem}Se puede probar que si $\{X_n\}$ es una sucesión de variables aleatorias y $X$ es variable aleatoria, 

\[ X_n\convcs X \implies X_n \convprob X \implies X_n \convdist X \]
Al contrario no tiene por qué darse.
\end{theorem}


\begin{theorem}[Teorema\IS de Slutsky] Sean $\{X_n\}$, $\{Y_n\}$ sucesiones de variables aleatorias tales que $X_n\convdist X$, $Y_n\convprob c$ con $c\in\real$ constante. Entonces

\begin{enumerate}
\item $X_n + Y_n \convdist X + c$
\item $X_n \cdot Y_n \convdist X \cdot c$
\item $\dfrac{X_n}{Y_n}\convdist \dfrac{X}{c}$ si $c≠0$.
\end{enumerate}
\end{theorem}

\subsubsection{Desigualdades básicas}

\begin{theorem}[Desigualdad\IS de Markov] Sea $X$ v.a. Entonces, $\forall \epsilon > 0$, \[ \prob{\abs{X} > \epsilon} ≤ \frac{\esp{X}}{\epsilon} \]
\end{theorem}

\begin{theorem}[Desigualdad\IS de Chebichev] En las mismas condiciones del teorema anterior, se cumple que  \[ \prob{\abs{X - \esp{X}} > \epsilon} ≤ \frac{V(X)}{\epsilon^2} \]
\end{theorem}
\subsection{Problema de inferencia}
\subsubsection{Interpretación estadística de la ley de los grandes números}

\begin{theorem}[Ley\IS de los grandes números] Sea $\{x_k\}$ una sucesión de v.a.i.i.d con media finita $µ$. Se verifica entonces que 
\label{thmGrandes}
\[ \avg{X} = \frac{\sum_{i=1}^n x_i}{n} \convcs µ \]

\end{theorem}

\subsubsection{Función de distribución empírica}

\begin{defn}[Función\IS de distribución empírica] La función de distribución empírica asociada a la muestra $\{x_n\}$ se define mediante

\[ \prob{X ≤ t} =  \fd_n(t) = \frac{1}{n}\sum_{i=1}^n \ind_{(-\infty, t]} (x_i) \]

Es decir, $\fd_n(t)$ es la proporción de puntos de la muestra que caen en el intervalo $(-\infty, t]$.
\end{defn}

Sin embargo, surge una duda: ¿converge la función de distribución empírica a la función de distribución original?

Intuitivamente, podemos pensar que cuantos más puntos cojamos más se aproximará a la función de distribución original. De hecho, eso es lo que demuestra el siguiente teorema:

\begin{theorem}[Teorema\IS de Glivenko-Cantelli] Sean $\{x_n\}$ v.a.i.i.d con función de distribución $F$. Se verifica que
\label{thmGlivenko}
\[ \md{\fd_n - F}_\infty=\sup_{t\in\real} \abs{\fd_n(t) - F(t)} \convcs 0 \]

donde $\md{\fd_n - F}_\infty$ es el \index{Estadístico! de Kolmogorov-Smirnov} \textbf{estadístico de Kolmogorov-Smirnov}.

\end{theorem}

\begin{proof}
Empezamos demostrando la convergencia de los términos intermedios. Es decir, queremos demostrar que 

\begin{equation}\label{eqConvCsGC}
\fd_n(t) \convcs F(t)
\end{equation} 

Tenemos que \[ \fd_n(t) = \frac{1}{n}\sum_{i=1}^n \ind_{(-\infty, t]} (x_i) \]

A cada uno de los términos de los términos de la suma $\ind_{(-\infty, t]}(x_i)$ los podemos llamar $y_i$. Estos valores son una muestra de la distribución \[ Y = \ind_{(-\infty, t]}(X) \]. Por lo tanto y por la LGN (\ref{thmGrandes}) \[ \fd_n(t) = \frac{1}{n}\sum_{i=1}^n Y_i = \avg{Y} \convcs \esp{Y} \]

pero

\[ \esp{Y} = \esp{\ind_{(-\infty, t]}(X)} = \prob{X\in (-\infty, t]} = F(t) \] por lo tanto hemos demostrado (\ref{eqConvCsGC}).

Ahora tenemos que demostrar que el límite por la izquierda converge. Es decir, hay que demostrar que \begin{equation}
 \fd_n(t^-) \convcs F(t^-)  \label{eqConvIzq}
\end{equation}. Esa convergencia se da si y sólo si en un conjunto de probabilidad $1$ se tiene que $ \fd_n(t^-) \convs F(t^-) $. Según la definición de límite, esto se da si y sólo si \begin{equation}
 \forall \epsilon > 0\; \exists N \tq n ≥ N \implies \abs{\fd_n(t^-) - F(t^-) } < \epsilon \label{eqLim1} \end{equation}

Sabemos que 
\begin{equation}
	\exists\epsilon >0\tq \fd_n (t^-) = \fd_n (x)\; \forall x \in (t-\delta, t+\delta) \label{eqLim2}
\end{equation}. Seguimos:

\begin{equation}
 F(t^-) = \lim_{x\to t^-} F(x) \dimplies \forall \epsilon > 0 \; \exists \delta > 0 \tq x \in (t - \delta, t) \implies \abs{F(x) - F(t^-)} < \frac{\epsilon}{2}\label{eqLim3} 
\end{equation}

Tomamos $x\in(t-\delta, t)$ con un delta que cumpla tanto la condición en (\ref{eqLim2}) como en (\ref{eqLim3}). Entonces

\[ \abs{\fd_n(t^-) - F(t^-)} =  \abs{\fd_n(x) - F(x) + F(x) - F(t^-)} ≤ \underbrace{\abs{\fd_n(x) - F(x)}}_{(a)} + \underbrace{\abs{F(x) - F(t^-)}}_{(b)} \]

Sabemos que $(a)$ es menor que $\frac{\epsilon}{2}$ por (\ref{eqLim1}) y que $(b)$ también es menor que $\frac{\epsilon}{2}$ por (\ref{eqLim3}), por lo tanto 

\[ \abs{\fd_n(t^-) - F(t^-)}  < \epsilon \]

Buscamos ahora una partición finita de $\real$ dada por $t_0 = -\infty ≤ t_1 ≤ \dotsb ≤ t_k = \infty$ tal que para todo $\epsilon > 0$ se cumpla que $\abs{F(t_i^-) - F(t_{i-1})} ≤ \epsilon$. Lo construimos de forma recursiva: dado $t_{i-1}$ tomamos

\[ t_i =\sup_{z\in\real} \{ F(z) ≤ F(t_{i-1} + \epsilon \} \]

El siguiente paso: para todo $t_{i-1} ≤ t ≤ t_i$ se tiene que 

\[ \fd_n(t) - F(t) ≤ \fd_n(t_i^-) - F(t_i^-) + \epsilon \]

Como $\fd_n$ es no decreciente (es una función de distribución), tenemos también que 

\[ \fd_n(t) - F(t) ≥ \fd_n(t_{i-1}) - F(t_{i-1}) - \epsilon \]

Con estas dos últimas ecuaciones, llegamos a que 

\[ \sup_{t\in\real} \abs{\fd_n(t) - F(t)} ≤ \max\left\lbrace \max_{i=1,\dotsc ,k} \abs{\fd_n(t_i) - F(t_i)},\; \max_{i=1,\dotsc ,k} \abs{\fd_n(t_i^-) - F(t_i^n)} \right\rbrace + \epsilon \]

Por (\ref{eqConvCsGC}), sabemos que $\abs{\fd_n(t_i) - F(t_i)} \convcs 0$, y por lo tanto \[ \max_{i=1,\dotsc ,k} \abs{\fd_n(t_i) - F(t_i)} \convcs 0 \].

De la misma forma, usando (\ref{eqConvIzq}) tenemos que \[ \max_{i=1,\dotsc ,k} \abs{\fd_n(t_i^-) - F(t_i^n)} \convcs 0 \]. Por lo tanto, todo ese máximo enorme vale 0, de tal forma que 

\[ \lim_{n\to\infty} \sup_{t\in\real} \abs{\fd_n(t) - F(t)}  =  \lim_{n\to\infty} \md{\fd_n - F}_\infty ≤ \epsilon \]

para cualquier $\epsilon > 0$ arbitrario que cojamos. Es decir, que \[ \md{\fd_n - F}_\infty=\sup_{t\in\real} \abs{\fd_n(t) - F(t)} \convcs 0 \]
\end{proof}

\subsection{Estadísticos}

Cuando extraemos una muestra $\{x_n\}$ de $X$ se pueden calcular algunas \textit{medidas resumen}. Cualquiera de ellas se puede expresar matemáticamente como una función $T(x_1,\dotsc,x_n)$ de la muestra. 

\begin{defn}[Estadístico]
Sea $T(x_1,\dotsc,x_n)$ una función cuyo dominio incluye el espacio muestral del vector aleatorio $(X_1, \dotsc, X_n)$. Entonces la variable aleatoria $T$ se denomina \textbf{estadístico}. La única restricción es que un estadístico no puede ser función de un parámetro.
\end{defn}

Como la distribución de $T$ se calcula a partir de la distribución de las variables $X_i$ que constituyen la muestra, la denominaremos distribución de $T$ en el muestreo (\textit{sampling distribution).}

\begin{defn}[Error\IS típico]\index{Error!estándar}
El error estándar o error típico $\sigma$ de un estadístico $T$ es la desviación típica de su distribución en el muestreo. Como en ocasiones depende de alguna cantidad desconocida, también se denomina error típico a una estimación de ese valor.
\end{defn}

En ocasiones, se cumple que $\dfrac{T}{\sigma}$ sigue una distribución t de Student, lo que nos permitirá definir intervalos de confianza.

\subsubsection{Media muestral y poblacional}

\begin{defn}[Media\IS muestral] La media muestral \[ \avg{X} = \frac{\sum_{i=1}^n X_i}{n} \] se puede expresar de la siguiente forma

\[ \avg{X} = \int_\real x\,d\fd_n(x) \]
\end{defn}

\index{Media!poblacional}
La definición es análoga con la de la \textbf{media poblacional}

\[ µ = \int_\real x \,dF(x) \]

Esto nos da una clave de la estadística: sustituir todo lo que desconozco de la población con su análogo muestral (en este caso, pasamos de la función de distribución teórica a la función de distribución empírica). Sólo quedaría ver si los estimadores que resultan son adecuados.

La media muestral tiene otras relaciones muy importantes con $µ$:

\begin{enumerate}
\item $\avg{X}$ es \index{Estimador!insesgado}\index{Estimador!centrado} \textbf{estimador insesgado o centrado} de µ: $ \esp{\avg{X}} = µ$
\item $\var{\avg{X}} = \dfrac{\sigma^2}{n}$. Como es inversamente proporcional, está claro que cuantos más datos haya mejor nos aproximaremos a lo que queremos estimar.
\end{enumerate}

\begin{theorem}[Teorema\IS central del límite] Suponemos que $\{X_n\}$ son v.a.i.i.d. con media $µ$ y desviación típica $\sigma$ finitas. Entonces
\label{thmCentral}
\[ \sqrt{n}\frac{\avg{X}-µ}{\sigma} \convdist N(0,1) \]

Si denotamos la función de distribución de la normal como \[ \Phi(x) = \int_{-\infty}^x \frac{1}{\sqrt{2\pi}} e^{-\frac{t^2}{2}}\] entonces

\[ \forall t\in\real\quad \prob{ \sqrt{n}\frac{\avg{X}-µ}{\sigma} ≤ t } \convs \Phi(t) \]

Por tanto, para $n$ grande se tiene

\[ \prob{\sqrt{n} \left(\avg{X} - µ\right) ≤ x} ≈ \Phi(\frac{x}{\sigma}) \]

\textbf{aunque las $X_i$ no tengan distribución normal.}

\end{theorem}

\subsubsection{Varianza muestral y poblacional}

Una medida importante de dispersión de una variable aleatoria es la varianza \begin{equation}
 \mathbb{V}(X)=\sigma^2 = \int_\real(x-µ)^2 \,dF(x) \label{eqVarianza}
\end{equation}

\begin{defn}[Varianza\IS muestral]El análogo muestral de $\sigma^2$ es la \textbf{varianza muestral.}. Utilizando el criterio \textit{plugin} en (\ref{eqVarianza})

\[\hat{\sigma}^2_n = \int_\real (x-\avg{X})^2\,d\fd_n(x) = \frac{1}{n}\sum_{i=1}^n(X_i - \avg{X})^2 \]
\end{defn}

\begin{theorem} La varianza muestral cumple lo siguiente

\begin{gather*}
\esp{\hat{\sigma}^2_n} = \frac{n-1}{n}\sigma^2\\
\hat{\sigma}^2_n \convcs \sigma^2
\end{gather*}
\end{theorem}

Por lo tanto, la varianza muestral es un estimador sesgado. No es un problema grande ya que cuando $n\to\infty$ acaba convergiendo a $\sigma^2$ y el sesgo 

\[ \esp{\hat{\sigma}^2_n} - \sigma^2= \frac{n-1}{n}\sigma^2 - \sigma^2 = \frac{-1}{n}\sigma^2 \]

también tiende a cero. Es decir, es \index{Asintóticamente!insesgado} \textbf{asintóticamente insesgado}.

\begin{defn}[Cuasivarianza\IS muestral] En lugar de usar $\hat{\sigma}^2_n$ usamos la cuasivarianza muestral, definida como

\[ S^2 = \frac{n}{n-1}\hat{\sigma}^2_n \] de tal forma que se tiene

\begin{gather*}
\esp{S^2} = \sigma^2 \\
S^2 \convcs \sigma^2 
\end{gather*}
\end{defn}

\subsubsection{Estadísticos de orden}

\begin{defn}[Estadístico\IS de orden]Dada una muestra $\{X_n\}$, se denotan como \[ X_{(1)} ≤ \dotsb ≤ X_{(n)} \] las observaciones de la muestra ordenadas de menor a mayor, llamados \textbf{estadísticos de orden}. Cuando la distribución de las v.a. es continua, la probabilidad de coincidencia en valores es $0$ y con probabilidad $1$ se tiene que \[ X_{(1)} < \dotsb < X_{(n)} \]
\end{defn}

Los estadísticos de orden pueden utilizarse para definir la mediana o los cuartiles. Sin embargo, podemos usar la función cuantílica para definir mejor estos conceptos.

\begin{defn}[Función\IS cuantílica] La función cuantílica en $p$ es el punto que deja una probabilidad $p$ a la izquierda, de tal forma que una proporción $p$ de los individuos de la población $X$ sería menor que el cuantil poblacional de orden $p$.

La función cuantílica correspondiente a la función de distribución $F$ como la función 

\begin{gather*}
\appl{\inv{F}}{\real}{(0,1)} \\
\inv{F}(p) = \inf \left\lbrace x \tq F(x) ≥ p \right\rbrace 
\end{gather*}
\end{defn}

La función cuantílica nos permite obtener los \textbf{cuantiles poblacionales de orden $p$} \index{Cuantil!poblacional} al valor $\inv{F}(p)$. El análogo es el \textbf{cuantil muestral de orden $p$}, \index{Cuantil!muestral} se define a partir de la función de distribución empírica como $\inv{\fd_n}(p)$.

\section{Estimación puntual paramétrica}

En este tema, supondremos que la muestra, absolutamente continua o discreta, con función de densidad o probabilidad $f(·;\theta)$ que es totalmente conocida salvo el valor de un parámetro $\theta$ del cuál sólo se conoce su rango de posibles valores $\Theta$, al que se llama el \textbf{espacio paramétrico.}\index{Espacio!paramétrico}

\subsection{Estimadores}

\begin{defn}[Estimador] Sean $\{X_n\}$ v.a.i.i.d. con distribución común caracterizada por la función de densidad/masa $f(\cdot;\theta)$, con $\theta$ un parámetro desconocido del que sólo se sabe que pertenece al espacio paramétrico $\Theta \subset \real$.

El \textbf{estimador} es una función medible $\hat{\theta}_n = T_n(X_1,\dotsc, X_n)$ que se utiliza para estimar o aproximar el valor de $\theta$.
\end{defn}

Cuando tenemos una muestra aleatoria $\{X_n\}$, cada $T_n(X_1, \dotsc, X_n)$ es un estimador de $\theta$, una variable aleatoria. Si por el contrario tenemos una serie de observaciones de una muestra $\{x_n\}$ entonces $T_n(x_1,\dotsc,x_n)$ es una \textbf{estimación} de $\theta$.

Podemos evaluar la calidad de un estimador con el \textbf{error cuadrático medio} (ECM):

\[ ECM(T_n) = \esp{(T_n - \theta)^2}\]

Si sumamos y restamos $\esp{T_n}$, nos queda que 

\[ ECM(T_n) = \var{T_n} + (\text{sesgo}\, T_n)^2 \]

que nos describe el error cuadrático medio en función de la varianza y del sesgo de $T_n$.

\subsubsection{Propiedades interesantes de los estimadores}
Buscaremos varias propiedades interesantes de los estimadores:

\index{Estimador!insesgado}
\paragraph{Ausencia de sesgo} Se dice que un estimador $T_n$ es \textbf{insesgado} \index{Estimador!insesgado} si, siempre que $X_i \sim f(\cdot;\theta)$ se tiene que \[\esp{T_n} = \theta\; \forall \theta \in \Theta \]

\appendix
\section{Distribuciones notables}
\label{secDistr}
\includepdf[pages={2-last}, nup=1x3]{_Distribuciones.pdf}

\section{Ejercicios}
\section{Tema 1 - Estadística descriptiva}

\begin{problem}[2] Demostrar que \[ \sum_{i=1}^n \left(x_i-\avg{x}\right)^2 = \min_{a\in \real} \sum_{i=1}^n(x_i-a)^2 \]

\solution

Definimos una función \[ g(a) = \sum_{i=1}^n(x_i-a)^2 \], buscamos su derivada \[ g'(a) = -2 \sum_{i=1}^n(x_i-a) \] e igualamos a cero:

\begin{gather*}
-2 \sum_{i=1}^n(x_i-a) = 0 \\
\sum_{i=1}^n x_i - \sum_{i=1}^n a = 0 \\
n \avg{x} = n a \\
\avg{x} = a 
\end{gather*}

Esto quiere decir que la media muestral es el valor que minimiza la distancia con cada uno de los datos de la muestra.
\end{problem}

\begin{problem}[5]Determina si es verdadero o falso:

\ppart Si añadimos 7 a todos los datos de un conjunto, el primer cuartil aumenta en 7 unidades y el rango intercuartílico no cambia.

\ppart Si todos los datos de un conjunto se multiplican por -2, la desviación típica se dobla.
\solution 

\spart Añadir siete a todos los datos es una traslación, así que la distribución de los datos no cambia.

\spart Teniendo en cuenta que si multiplicamos todos los datos del conjunto por $-2$ la media también se multiplica por $-2$, y sustituyendo en la fórmula de la varianza:

\[ \sigma' = \sqrt{\frac{1}{n} \sum_{i=1}n (-2x_i)^2 - (-2\avg{x})^2} = \sqrt{\frac{1}{n} \sum_{i=1}4\left(n x_i^2 - \avg{x}^2\right)} = \sqrt{4\sigma^2} = 2\sigma \]

Por lo tanto, la desviación típica sí se dobla.

\spart Usando los cálculos del apartado anterior vemos que la varianza se multiplica por cuatro.

\spart Efectivamente: cambiar el signo haría una reflexión de los datos sobre el eje Y y la asimetría estaría orientada hacia el lado contrario. 

\end{problem}

\section{Tema 2 - Muestreo aleatorio}

\begin{problem}[1] Se desea estimar el momento de orden 4, $\alpha_3 = \esp{X^3}$ en una v.a. $X$ con distirbución exponencial de parámetro 2, es decir, la función de distribución de $X$ es $F(t) = \prob{X ≤ t} = 1 - e^{-2t}$ para $t≥0$. Definir un estimador natural para $\alpha_3$ y calcular su error cuadrático medio.

\solution

Usando el criterio de \textit{plugin}, podríamos definir el estimador \[ \hat{\alpha}_3 = \int_\real x^3\,d\fd_n(x) \]. 

Calculamos ahora el error cuadrático medio:

\begin{gather*}
ECM(\hat{\alpha}_3) = \esp{\hat{\alpha}_3 - \alpha_3}^2 = \esp{(\hat{\alpha}_3 - \esp{\hat{\alpha}_3} + \esp{\hat{\alpha}_3} - \alpha_3) ^2} = \\
= \esp{(\hat{\alpha}_3 - \esp{\hat{\alpha_3}})^2 +  (\esp{\hat{\alpha_3}}- \alpha_3)^2 + 2(\hat{\alpha}_3 - \esp{\hat{\alpha_3}}) (\esp{\hat{\alpha_3}}- \alpha_3)} = \\
= \underbrace{\esp{(\hat{\alpha_3} - \esp{\hat{\alpha_3}})^2}}_{(a)}+ \underbrace{\left(\esp{\hat{\alpha_3}} - \alpha_3\right)^2}_{(b)} + \underbrace{2 \cdot \esp{ (\esp{\hat{\alpha_3}}- \alpha_3)^2 + 2(\hat{\alpha}_3 - \esp{\hat{\alpha_3}})}}_{(c)} 
\end{gather*}

Aquí ya hay cosas raras. (c) es cero por alguna razón, luego hay que calcular la varianza y el sesgo.

\[ \text{sesgo}(\hat{\alpha}_3) = \esp{\hat{\alpha}_3} - \alpha_3 = \alpha_3 - \alpha_3 = 0 \]

\[ \var{\hat{\alpha}_3} = \var{\frac{1}{n}\sum X_i^3 } = \frac{1}{n^2}\var{\sum X_i^3} = \frac{1}{n^2}\sum \var{X_i^3} = \frac{\var{X^3}}{n} \]

y, teniendo en cuenta el enunciado,

\[ \var{X^3} = \esp{X^6} - \esp{X^3}^2 = \frac{6!}{2^6} - \left(\frac{3!}{2^3}\right)^2 = \frac{171}{16} \]

y por lo tanto

\[ \text{ECM}(\hat{\alpha}_3) = \frac{171}{16n} = O(\frac{1}{n)} \convs 0 \]

donde lo que más nos importa es la convergencia a cero, que indica que cuanto más muestras tenemos mejor será el estimador.

\end{problem}

\begin{problem}[2] Supongamos que la muestra tiene tamaño $n=50$ y que la distribución de las $X_i$ es una $N(4,1)$. 

\ppart Obtener, utilizando la desigualdad de Chebichev, una cota superior para la probabilidad $\prob{\abs{\avg{X} - 4} > 0.3}$.

\ppart Calcula exactamente $\prob{\abs{\avg{X} - 4} > 0.3}$ utilizando la distribución de $X_i$. 

\solution
\spart

Como la media es cuatro, la desigualdad de Checbichev nos da una cota de 

\[ \frac{\var{\avg{x}}}{0.3^2} = \frac{\var{X}}{n \cdot 0.3^2} \simeq 0.22 \]

\spart

Normalizamos

\[ Z = \frac{\avg{X} - 4}{\frac{1}{\sqrt{50}}} ~ N(0,1) \]

y calculamos.

\[ \prob{\abs{\avg{X} - 4} > 0.3} = \prob{\abs{Z} > \frac{0.3}{\frac{1}{\sqrt{50}}}} = 2 \cdot \prob{Z > 2.12} = 0.038 \]

\end{problem}

\begin{problem}[4] Denotemos por 

\[ C_n = \int_\real \left(\fd_n(t) - F(t)\right)^2 \, dF(t) \]

la llamada discrepancia de Cramer-Von Mises entre $\fd_n$ y $F$. ¿Converge a cero casi seguro esta discrepancia?

Calcular la distribución asintótica de la sucesión $D_n = \sqrt{n}\left(\fd_n(t) - F(t)\right)$ para un valor fijo $t\in\real$.

\solution

\[ C_n = \int_\real \left(\fd_n(t) - F(t)\right)^2 \, dF(t) = \int_\real \left(\fd_n(t) - F(t)\right)^2 f(t) \, dt \]

Como por el teorema de Glivenko-Cantelli (\ref{thmGlivenko}) tenemos que 

\[ \fd_n(t) - F(t) ≤ \sup_t \abs{\fd_n(t) - F(t)} = \md{\fd_n - F}_\infty \]

entonces 

\[ \int_\real \left(\fd_n(t) - F(t)\right)^2 f(t) \, dt ≤  \md{\fd_n - F}_\infty^2 \int_\real f(t) \,dt = \md{\fd_n - F}_\infty^2 \]

Igualmente por Glivenko-Cantelli, 

\[ \md{\fd_n - F}_\infty^2 \convcs 0  \qed \]

\spart

Para calcular la distirbución asintótica de \[ D_n = \sqrt{n}\left(\fd_n(t) - F(t)\right) \] usamos el Teorema Central del Límite (\ref{thmCentral}). Necesitamos algo que se asemeje a una media muestral, y de hecho

\[ \fd_n(t) = \frac{1}{n} \sum_{i=1}^n \ind_{(-\infty, t]} (X_i) = \frac{1}{n} \sum_{i=1}^n Y_i = \avg{Y} \]

Por otra parte, $Y = \ind_{(-\infty, t]}(X)$ y por lo tanto \[ \esp{Y} = \esp{\ind_{(-\infty, t]}(X)} = \prob{X ≤ t} = F(t) \]

Ya podemos aplicar el TCL, pero nos falta saber cuál es la desviación típica de $Y$. Como es una distribución de Bernoulli 

\[ \mathbb{V}(Y) = p(1-p) = F(t)(1-F(t)) \]

y por lo tanto 

\[ D_n \convdist N\left(0, \sqrt{F(t)(1-F(t))}\right) \]
\end{problem}

\begin{problem}[5] Sea $X$ una v.a. cuya función de densidad depende de un parámetro desconocido $\theta \in \real$, concretamente \[ f(x;\theta) = \frac{1}{\pi}\frac{1}{1+(x-\theta)^2} \] para $x\in \real$. Comprobar que $\theta$ coincide con la mediana y la moda de $X$ pero que la media $\esp{X}$ no está definida.

Diseñar un experimento de simulación en R, tomando algún valor concreto de $\theta$, orientado a comprobar cómo se comportan la mediana muestral y la media muestral como estimadores de $\theta$: mientras la mediana muestral se acerca al verdadero valor de $\theta$ al aumentar $N$, la media muestral oscila fuertemente y no se acerca a $\theta$ aunque se aumente el tamaño muestral $n$.

\solution Viendo la función, vemos que es simétrica con respecto al eje $x= \theta$. Por lo tanto, el punto que deja a izquierda y derecha la misma probablidad, la mediana, es precisamente $\theta$. 

De la misma forma, la moda es el valor máximo de la distribución, que se ve claramente que ocurre cuando $x=\theta$.
\end{problem}

\begin{problem}[7] Sea $X$ una v.a con distribución absolutamente continua. Sea $F$ la correspondiente función de distribución y $f = F'$ continua en todo punto la función de densidad. para $r\in \{1,\dotsc,n\}$, denotemos por $X_{(r)}$ el $r$-simo estadístico ordenado de una muestra de tamaño $n$ extraída de $X$. Calcular la función de distirbución y la de densidad de la v.a. $X_{(r)}$.

\solution

Por definición

\[ F_{X_{(r)}} (x) = \prob{X_{(r)} ≤ x }\]

que es la probabilidad que al menos $r$ elementos de la muestra sean menores o iguales que $x$. Luego la probabilidad es igual a

\begin{gather*}
\sum_{j=r}^n \prob{\text{exactamente j observaciones de la muestra son ≤ x}} =  \\
= \sum_{j=r}^n \prob{B(n, F(x)) = j} = \sum_{j=r}^n \comb{n}{j}F(x)^j \left(1 - F(x)\right)^{n-j}
\end{gather*}

Ahora sólo falta calcular la densidad de $X_{(r)}$, y la obtenemos derivando

\begin{gather*}
 f_{X_{(r)}} (x) = \\
 = \sum_{j=r}^n \left(\comb{n}{j}j(F(x)^{j-1}(1-F(x))^{n-j}f(x) - (F(x))^j(n-j)(1-F(x))^{n-j-1} f(x)\right) = \\
 = \sum_{j=r}^n \comb{n}{j} j(F(x)^{j-1}(1-F(x))^{n-j}f(x)  - \sum_{j=r}^n\comb{n}{j} (F(x))^j(n-j)(1-F(x))^{n-j-1} f(x) = \\
 = \comb{n}{r} r(F(x))^{r-1} (1-F(x))^{n-1}f(x) + \sum_{j=r+1}^n \comb{n}{j}j(F(x))^{j-1} f(x) (1-F(x))^{n-j} \\
 \quad \quad - \sum_{j=r}^n\comb{n}{j}(n-j)(F(x))^j (1-F(x))^{n-j-1}f(x) = \\
 n\comb{n-1}{r-1}(F(x))^{r-1} (1-F(x))^{n-r} f(x)    +   \sum_{l=r}^{n-1}n\comb{n-1}{l}(F(x))^l (1-F(x))^{n-l-1} f(x) \\
 \quad\quad -  \sum_{j=r}^{n-1}n\comb{n-1}{j}(F(x))^j (1-F(x))^{n-j-1} f(x)
\end{gather*} 

Los dos últimos términos se cancelan y nos queda que 

\[ f_{X_{(r)}} (x) = n\comb{n-1}{r-1}(F(x))^{r-1} (1-F(x))^{n-r} f(x) \]

Consideremos los dos casos particulares del mínimo y máximo de la muestra. Con el mínimo, $r=1$ y entonces

\[ F_{X_{(1)}} (x)= \prob{X_{(1)} ≤ x} = \sum_{j=1}^n\comb{n}{j}(F(x))^j(1-F(x))^{n-j} = 1 - (1-F(x))^n \]

En el caso del máximo:

\[ F_{X_{(n)}} (x) = \prob{X_{(n)} ≤ x } = (F(x))^n \]

\end{problem}

\begin{problem}[8] Sea $\hat{f}_n$ un estimador kernel de la densidad basado en un núcleo $K$ que es una función de densidad con media finita. Comprobar que, en general, $\hat{f}_n(t)$ es un estimador sesgado de $f(t)$ en el sentido de que \textbf{no} se tiene $\esp{\hat{f}_n(t)} = f(t)$ para todo $t$ y para toda densidad $f$.

\solution

\begin{gather*}
\esp{\hat{f}_n(t)} = \esp{\frac{1}{nh}\sum_{i=1}^n K \left(\frac{t-X_i}{h}\right)} = \\
= \frac{1}{nh}\sum_{i=1}^n \esp{K\left(\frac{t-X_i}{h}\right)} = \frac{1}{h} \esp{K\left(\frac{t-X}{h}\right)} = \\
= \frac{1}{h} \int_\real K \left(\frac{t-x}{h}\right) f(x) \,dx = 
\end{gather*}

Haciendo un cambio de variable $x = t-hz$, $dx = -h\,dz$, los límites se invierten,

\[ = \frac{1}{h} \int_{-\infty}^\infty K \left(\frac{t-x}{h}\right) f(x) \,d(x)  = \frac{-1}{h} \int_\infty^{-\infty} K(z) f(t-hz) h \,dz  = \int_{-\infty}^\infty Kz f(t-hz)\,dz \]

Ahora buscamos calcular el sesgo:

\[ \text{sesgo}\,(\hat{f}_n(t)) = \esp{\hat{f}_n(t)} - f(t) = \]

Usando que $K$ es función de densidad y que $\int K = 1$, nos queda

\begin{gather*}
 = \int_{-\infty}^\infty K(z) f(t-hz)\,dz - \int_{-\infty}^\infty K(z) f(t)\, dz = \\
 = \int_{-\infty}^\infty K(z) \left[f(t-hz)-f(t)\right]\,dz =\\
 = hf'(t)\int_{-\infty}^\infty zK(z)\,dz + \frac{1}{2} h^2 f''(t) \int_{-\infty}^\infty z^2K(z)\,dz + \frac{1}{6}h^3 f'''(t) \int_{-\infty}^\infty z^3K(z)\,dz + \dotsb  
\end{gather*}

al hacer el desarrollo de Taylor. Como $K$ es una función simétrica, las integrales con índice impar (con $z=1, 3,\dotsc$) se anulan. Sin embargo, el segundo término no lo hace. Por lo tanto, el sesgo de un estimador kernel \textbf{no es nunca cero}. 

El sesgo del estimador kernel depende de $h$ (el parámetro de suavizado o \textit{bandwith}) en potencias pares. Por eso, se toma de manera tal que $h\convs 0$ y entonces $\text{sesgo}\,\hat{f}_n(t) \convs 0$ pero manteniendo un equilibrio para que la varianza también sea pequeña y no tengamos picos en el histograma (ver sección \ref{secEst}).

\end{problem}
\section{Tema 3 - Estimación puntual paramétrica}

\begin{problem}[3] Se disponeb de un gran lote de piezas producidas en una cadena de montaje. Denotemos por $p$ la proporción de piezas defectuosas en ese lote. Supongamos que se seleccionan al azar sucesivamente (con reemplazamiento) piezas del lote hasta que se encuentra una defectuosa. Sea $X$ la variable aleatoria que indica el número de la extracción en la que aparece la primera pieza defectuosa.

\ppart Calcular $\prob{X=k}$ para $k=1,2,\dotsc$ Obtener el estimador de $p$ por el método de los momentos, a partir de una muestra $X_1,\dotsc , X_n$.

\ppart Obtener el estimador de $p$ por el método de máxima verosimilitud. Calcular su distribución asintótica.
\solution
\spart
La probabilidad sigue una distribución geométrica de parámetro $p$:

\[ \prob{X=k} = (1-p)^{k-1}p \]

\spart Calculamos la función de verosimilitud:

\[ L(p;x_1,\dotsc,x_n) = \prod_{i=1}^n f(x_i;p) = \prod_{i=1}^n (1-p)^{x_i -1}p = (1-p)^{\sum_{i=1}^n x_i -n} p^n \]

Tomamos logaritmos

\[ \log L(p) = \log(1-p) \left(\sum_{i=1}^n x_i -n\right) + n\log p \]

y derivando

\[ \deriv{}{p} \log L(p) = \frac{-1}{1-p} \left(\sum_{i=1}^n x_i -n\right)  + \frac{n}{p} \] 

y derivas tú lo que queda, majo.
\end{problem}

\begin{problem}[5]
Distribución de Rayleigh, cuya función de densidad es:
\[f(x;\theta) = \frac{x}{\theta^2} e^{\frac{-x^2}{2\theta^2}} \mathbb{I}_{[0,\infty)} (x), \theta > 0\]

\begin{itemize}
\item[a]Calcular el estimador de máxima verosimilitud (e.m.v.)
\item[b]Calcular la consistencia.
\item[c] ¿Es asintóticamente normal?
\end{itemize}

\solution

\paragraph{a)}

\[L_n(\theta;x_1,...,x_n) = \frac{x_1 \cdot ... \cdot x_n}{\theta^2} e^{\frac{-1}{2\theta^2} \sum_{i=1}^n x_i^2}\]
\[log L_n(\theta) = \sum log x_i - 2nlog\theta -\frac{1}{2\theta^2}\sum x_i^2\]
\[\dpa log L_n(\theta) = \frac{1}{\theta} \left(-2n+\frac{1}{\theta^2}\sum x_i^2\right) = 0\]
\[\implies \hat{\theta}^2 = \frac{\sum x_i^2}{2n} \implies \hat{\theta} emv(\theta) = (\frac{\sum x_i^2}{2n}^2\]

Estimador razonable porque $E(x^2) = V(x) + E(x) = 2\theta^2 \dimplies \theta^2 = \frac{1}{2} E(x^2)$

\paragraph{b)}
\textbf{Consistencia:} $\hat{\theta}^2 = \frac{1}{2} \gor{Y}, Y_i = X_i^2$

Por la ley fuerte de los grandes números (\ref{thmGrandes}) sabemos que: $\gor{Y} \convs[cs] E_{\theta}(Y) = E_{\theta}(X^2) = 2\theta^2$

Vamos a aplicar el teorema de Snouschky?.

Sea $g(x) = \sqrt{\frac{1}{2}x}$ definida sobre $[0,\infty)$.

Teorema de Snoopy $\implies g\left(\gor{Y}\right) = \sqrt{\frac{1}{2} \frac{\sum x_i^2}{n}} \convs[c.s.] g(E_{\theta}) = \sqrt{\frac{1}{2}\theta^2} = \theta \implies $ El e.m.v. de $\theta$, $\hat{\theta}$ es consistente c.s.


\paragraph{c)}

Queremos aplicar el método delta:

\[\sqrt{n}(\hat{\theta} - \theta) = \sqrt{n}\left(g\left(\gor{Y}\right) - g\left(E(Y)\right)\right) \convs[d]N(0,\abs{g'(E(Y))}\sqrt{V(Y)}\]

\[E_{\theta}(Y) = E_{\theta} (X^2) = 2\theta^2\]
\[V_{\theta}(Y) = E(X^4) - E^2(X^2) = 8\theta^4-4\theta^4 = 4\theta^4\]

Entonces tenemos que $g'(E(Y)) = \displaystyle \frac{1}{2\sqrt{2E(Y)}} = \frac{1}{4\theta}$.

Con esta información completamos:  

\[\sqrt{n}(\hat{\theta} - \theta) \convs[d] N\left(0,\sqrt{\frac{1}{2\theta}}\right)\]

\end{problem}

\begin{problem}[11]
\footnote{Este ejercicio es del parcial del año pasado}

ashkjdf
\solution

$X\leadsto Unif[0,\theta]$
Con \[ f(x) = \displaystyle\left\{\begin{array}{cc}
\frac{1}{\theta} & 0\leq x \leq \theta\\
0 & x \notin [0,\theta]
\end{array}\right.\]

Vamos a calcular la función de distribución:

\[F_{\theta} (x) = \mathbb{P}_{\theta}\{X\leq x\} = \int_{-infty}^x f_{\theta}(t)dt = \int_0^x \frac{1}{\theta} dt = \frac{x}{\theta} \ si 0\leq x \leq \theta\]

\[F_{\theta} = \left\{\begin{array}{cc}
\frac{x}{\theta} & 0\leq x \leq \theta\\
0 & x \notin [0,\theta]
\end{array}\right.\]

Nos piden dibujar las funciones... GUILLEEEE xD

Vamos a calcular \[L_n(\theta;x_i) = \prod_{i=1}^n f_{\theta} (x_i) = \left\{\begin{array}{cc}
\left(\frac{1}{\theta}\right)^n & \forall x_i \in [0,\theta]\\
0 & \exists x_i\notin [0,\theta]
\end{array}\right.\]

Calculamos la $logL_n$ que nos piden dibujarla:

\[logL_n(\theta) = \left\{\begin{array}{cc}
-nlog(\theta) & si \ max(\{x_i\})\leq \theta\\
0 & si \ no
\end{array}\right.\]
Dibujoo!

\[\hat{\theta_n} = e.m.v.(\theta) = max\left(L_n(\theta)\right)\]

También vale tomando el logaritmo:

\[\hat{\theta}_n = e.m.v. (\theta) = arg\ mas logL_n(\theta) = max\{x_i\}\]
porque \[ logLn(\theta) = \displaystyle\left\{\begin{array}{cc}
-nlog(\theta) & max\{x_i\} \leq \theta\\
-\infty & si \ no
\end{array}\right.\]
\end{problem}

\begin{problem}[5]
Distribución de Rayleigh, cuya función de densidad es:
\[f(x;\theta) = \frac{x}{\theta^2} e^{\frac{-x^2}{2\theta^2}} \mathbb{I}_{[0,\infty)} (x), \theta > 0\]

\ppart Calcular el estimador de máxima verosimilitud (e.m.v.)

\ppart Calcular la consistencia.

\ppart ¿Es asintóticamente normal?

\solution

\spart

\[L_n(\theta;x_1,...,x_n) = \frac{x_1 \cdot ... \cdot x_n}{\theta^2} e^{\frac{-1}{2\theta^2} \sum_{i=1}^n x_i^2}\]
\[log L_n(\theta) = \sum log x_i - 2nlog\theta -\frac{1}{2\theta^2}\sum x_i^2\]
\[\dpa log L_n(\theta) = \frac{1}{\theta} \left(-2n+\frac{1}{\theta^2}\sum x_i^2\right) = 0\]
\[\implies \hat{\theta}^2 = \frac{\sum x_i^2}{2n} \implies \hat{\theta} emv(\theta) = (\frac{\sum x_i^2}{2n}^2\]

Estimador razonable porque $E(x^2) = V(x) + E(x) = 2\theta^2 \dimplies \theta^2 = \frac{1}{2} E(x^2)$

Buscamos ahora el estimador $\tilde\theta$ por el \textbf{método de los momentos}

\[ \esp[\theta]{X}= \theta\sqrt{\frac{\pi}{2}} = \avg{X} \] 

y entonces el estimador es \[\tilde{\theta} = \avg{X}\frac{2}{\pi} \]

\spart

\textbf{Consistencia:} $\hat{\theta}^2 = \frac{1}{2} \gor{Y}, Y_i = X_i^2$

Por la ley fuerte de los grandes números (\ref{thmGrandes}) sabemos que: $\gor{Y} \convs[cs] E_{\theta}(Y) = E_{\theta}(X^2) = 2\theta^2$

Vamos a aplicar el teorema de Slutsky.

Sea $g(x) = \sqrt{\frac{1}{2}x}$ definida sobre $[0,\infty)$.

Teorema de Slutsky (\ref{thmSlutsky}) $\implies g\left(\gor{Y}\right) = \sqrt{\frac{1}{2} \frac{\sum x_i^2}{n}} \convcs g(E_{\theta}) = \sqrt{\frac{1}{2}\theta^2} = \theta \implies $ El e.m.v. de $\theta$, $\hat{\theta}$ es consistente c.s.


\spart

Queremos aplicar el método delta:

\[\sqrt{n}(\hat{\theta} - \theta) = \sqrt{n}\left(g\left(\gor{Y}\right) - g\left(E(Y)\right)\right) \convs[d]N(0,\abs{g'(E(Y))}\sqrt{V(Y)}\]

\[E_{\theta}(Y) = E_{\theta} (X^2) = 2\theta^2\]
\[V_{\theta}(Y) = E(X^4) - E^2(X^2) = 8\theta^4-4\theta^4 = 4\theta^4\]

Entonces tenemos que $g'(E(Y)) = \displaystyle \frac{1}{2\sqrt{2E(Y)}} = \frac{1}{4\theta}$.

Con esta información completamos:  

\[\sqrt{n}(\hat{\theta} - \theta) \convs[d] N\left(0,\sqrt{\frac{1}{2\theta}}\right)\]

Buscamos ahora la convergencia asintótica del estimador por el método de los momentos:

\[ \sqrt{n}(\tilde\theta-\theta) = \sqrt{n}\left(\avg{X}\frac{2}{\pi}  - \esp{X}\frac{2}{\pi}\right) = \sqrt{\frac{2}{\pi}}\sqrt{n}(\avg{X}-\esp{X}) \]

que, por el TCL (\ref{thmCentral})

\[ \sqrt{\frac{2}{\pi}}\sqrt{n}(\avg{X}-\esp{X})  \convdist  \sqrt{\frac{2}{\pi}}N\left(0,\theta\sqrt{\frac{4-\pi}{2}}\right) = N\left(0,\theta\sqrt{\frac{4-\pi}{\pi}}\right) \]

y por lo tanto es efectivamente asintóticamente normal.

\end{problem}

\begin{problem}[11]
\footnote{Este ejercicio es del parcial del año pasado}

asdasdf
\solution

$X\leadsto Unif[0,\theta]$
Con $f(x) = \displaystyle\left\{\begin{array}{cc}
\frac{1}{\theta} & 0\leq x \leq \theta\\
0 & x \notin [0,\theta]
\end{array}\right.$

Vamos a calcular la función de distribución:

\[F_{\theta} (x) = P_{\theta}\{X\leq x\} = \int_{-infty}^x f_{\theta}(t)dt = \int_0^x \frac{1}{\theta} dt = \frac{x}{\theta} \ si 0\leq x \leq \theta\]

\[F_{\theta} = \left\{\begin{array}{cc}
\frac{x}{\theta} & 0\leq x \leq \theta\\
0 & x \notin [0,\theta]
\end{array}\right.\]

Nos piden dibujar las funciones... GUILLEEEE xD

Vamos a calcular \[L_n(\theta;x_i) = \prod_{i=1}^n f_{\theta} (x_i) = \left\{\begin{array}{cc}
\left(\frac{1}{\theta}\right)^n & \forall x_i \in [0,\theta]\\
0 & \exists x_i\notin [0,\theta]
\end{array}\right.\]

Calculamos la $logL_n$ que nos piden dibujarla:

\[logL_n(\theta) = \left\{\begin{array}{cc}
-nlog(\theta) & si \ max(\{x_i\})\leq \theta\\
0 & si \ no
\end{array}\right.\]
Dibujoo!

\[\hat{\theta_n} = e.m.v.(\theta) = max\left(L_n(\theta)\right)\]

También vale tomando el logaritmo:

\[\hat{\theta}_n = e.m.v. (\theta) = arg\ mas logL_n(\theta) = max\{x_i\}\]
porque \[ logLn(\theta) = \displaystyle\left\{\begin{array}{cc}
-nlog(\theta) & max\{x_i\} \leq \theta\\
-\infty & si \ no
\end{array}\right.\]
\end{problem}

\begin{problem}[2]
\[X\leadsto N(0,\sqrt{\theta}), \theta>0, Espacioparametrico = (0,\infty)\]
\solution
\paragraph{a)}
\[L_n(\theta;X_1,...,X_n) = \prod_{i=1}^n f(x_i;\theta) = \frac{1}{\sqrt{2\pi}^{\frac{n}{2}}\theta^{\frac{n}{2}}} e ^ {-\frac{1}{2\theta} \sum x_i^2}\]

\[logL_n(\theta) = \frac{n}{2}\cdot log(2\pi) - \frac{n}{2}log(\theta) - \frac{1}{2\theta} \sum x_i^2\]

\[\dpa{}{\theta} logL_n(\theta) = ... \implies T_n = e.m.v.(\theta) = \frac{1}{n}\sum x_i^2\]


b) $\esp[\theta]{T_n} = \esp[\theta]{\frac{1}{n}\sum x_i^2} = \esp[\theta]{X^2} = \theta$

Vamos a calcular la información de fisher para comprobar si el estimador es eficiente o no.

\[ log f(x;\theta) = \frac{-1}{2}log(2\pi)-\frac{1}{2}log(\theta) - \frac{1}{2\theta}X^2\]
Derivamos:
\[\dpa{}{\theta} log f(x;\theta) = -\frac{1}{2\theta} + \frac{1}{2\theta^2}X^2\]
Elegimos derivar otra vez o elevar al cuadrado (2 alternativas para calcularlo).

En este caso vamos a elevar al cuadrado:

\[\dpa{}{\theta}logf(X;\theta) = \frac{1}{4\theta} \left( 1+\frac{X^4}{\theta} - 2\frac{X^2}{\theta}\right)\]

Entonces la información de fisher será:

\[I(\theta) = \esp[\theta]{\frac{1}{4\theta} \left( 1+\frac{X^4}{\theta} - 2\frac{X^2}{\theta}\right)} = \frac{1}{4\theta} \left( 1+\frac{\esp[\theta]{X^4}}{\theta} - 2\frac{\esp[\theta]{X^2}}{\theta}\right)\]

Aplicamos por hipótesis: $\esp[\theta]{X^4} = 3\theta^2$

\[I(\theta) = \frac{1}{4\theta^4} \left(1+\frac{3\theta^3}{\theta^2} - 2 \frac{\theta}{\theta}\right) = \frac{1}{2\theta}\]

Vamos a calcular \[\var[\theta]{T_n} = \var[\theta]{\frac{1}{n}\sum x_i^2} = \frac{1}{n^2}\sum \var[\theta]{x_i^2} = \frac{n}{n^2} \var[\theta]{X^2} = \frac{1}{n}\left(\esp[\theta]{X^4} - \esp[\theta]{X^2}\right) = \frac{1}{n}(3\theta^2-\theta^2) = \frac{2\theta^2}{n} = \frac{1}{nI(\theta)} \implies EFICIENTEEE!\]

Los siguientes pasos para comprobar lo bueno que es el estimador son: \begin{itemize}
\item $T_n$ asintóticamente normal.
\item $T_n$ es consistente casi seguro.
\end{itemize}

\paragraph{c)} Vamos a estudiar la distribución asintótica:

\[\sqrt{n}(T_n-\theta) \convs[d] N(0,\sigma(\theta))\]

Llamando $Y_i = X_i^2 \implies \esp[\theta]{Y} = \esp[\theta]{X^2} = \theta$

Entonces: $\displaystyle \sqrt{n}(\hat{Y} - \esp[\theta]{Y}) \convs[TCL \ d] N(0,\sqrt{\var{Y}})$

Donde $\var{Y} = \var[\theta]{X^2} = 2\theta^2$
\end{problem}

\begin{problem}[8] Sea $X \sim N(µ,\sqrt{\theta})$. Estamos interesados en la estimación de $\theta$ basados en muestras $X_1,\dotsc,X_n$ de tamaño $n$. Calcular la cota de Fréchet-Cramer-Rao (\ref{thmCotaFCR}) para estimadores insesgados.

\solution

La cota FCR es \[ \frac{1}{n I(\theta)} \]

Podíamos calcular la información de Fisher como

\[ I(\theta) = \esp{\left(\dpa{}{\theta}\log f(X;\theta)\right)^2} = - \esp{\frac{∂^2}{∂\theta^2}\log f(X;\theta)} \]

Usaremos la segunda expresión. Calculamos primero el logaritmo:

\[ \log f(X;\theta) = \frac{-1}{2}\log 2\pi - \frac{1}{2}\log \theta - \frac{1}{2\theta}(x-µ)^2 \]

y derivamos dos veces

\begin{gather*}
 \dpa{}{\theta} \log f(X;\theta) = \log f(X;\theta) = -\frac{1}{2\theta} + \frac{1}{2\theta^2}(x-µ)^2 \\
 \frac{∂^2}{∂\theta^2} \log f(X;\theta) = \frac{1}{2\theta^2} - \frac{2}{2\theta^3} (x-µ)^2 = \frac{1}{\theta^2} \left(\frac{1}{2} - \frac{1}{\theta}(x-µ)^2\right) 
 \end{gather*}
 
 Calculamos ahora la esperanza:
 
 \[ \esp{\frac{1}{\theta^2} \left(\frac{1}{2} - \frac{1}{\theta}(x-µ)^2\right) } = -\frac{1}{\theta^2}\left(\frac{1}{2} - \frac{1}{\theta} \underbrace{\esp{X-µ}^2}_{\theta}\right) = \frac{1}{2\theta^2} \]
 
 y por lo tanto la cota FCR vale $\dfrac{2\theta^2}{n}$, el valor mínimo.

\end{problem}

\begin{problem}[9] Sea $X_1,\dotsc,X_n$ una muestra de una v.a. con función de densidad 

\[ f(x;\theta) = \theta x^{\theta - 1} \]

Sea  \[ T_n(X_1,\dotsc,X_n) = \frac{-1}{n}\sum_{i=1}^n\log X_i \]

\ppart Probar que \[\esp[\theta]{T_n} = \frac{1}{\theta};\; \var[\theta]{T_n} = \frac{1}{n\theta^2} \]
\ppart ¿Es eficiente $T_n$ como estimador de $\frac{1}{\theta}$?

\solution

\spart

\[ \esp[\theta]{T_n} = -\esp[\theta]{\log X} = - \int_0^1 \log x \theta x ^{\theta-1}\,dx = \frac{1}{\theta} \]

Calculamos ahora la varianza:

\begin{gather*}
\var[\theta]{T_n} = \frac{1}{n\theta^2} = \esp[\theta]{T_n^2} - \esp[\theta]{T_n}^2 = \frac{\var[\theta]{\log X}}{n} = \\
= \esp[\theta]{\log^2 X} - \esp[\theta]{\log X}^2 = \frac{1}{\theta^2}
\end{gather*}

\end{problem}

\section{Tema 4 - Intervalos de confianza}

\begin{problem}[1 y 2]
\textbf{a)}Representa un estimador de la función de densidad de la v.a. X = cantidad de contaminación por mercurio (en p.p.m.) en los peces capturados en los ríos norteamericanos Lumber y Wacamaw (ver fichero Datos-mercurio.txt). Comparar esta densidad estimada con la densidad normal de igual media y desviación típica (representada en la misma gráfica). En vista de las dos funciones dirías que la función de densidad de X es aproximadamente normal?

\textbf{b)} Obtener un intervalo de confianza de nivel 0.95 para la media de X.

\textbf{c)} Se puede considerar fiable este intervalo a pesar de la posible no-normalidad de X?

\textbf{d)} Qué tamaño muestral habrá que tomar para estimar la contaminación media con un error máximo de 0.06?
\solution
Solucionado por Amprao, descargable 
\href{http://www.uam.es/personal_pdi/ciencias/abaillo/MatEstI/T4DatosMercurio.pdf}{aqui}

\end{problem}

\begin{problem}[3]
\ppart Representa en un mismo gráficp las densidades de las distribuciones $\chi^2_k $ con k = 4,8,20,30.
\ppart $X \sim \gamma(5,10)$. Calcular $\mathbb{P}\{X\leq 3\}$
\ppart Sea $Y \sim \chi_{200}^2$. Calcular $\mathbb{P}\{Y\leq 3\}$

\solution
\spart
El código R utilizado para generar las gráficas es:
> x = seq(0,20,length.out=1000)

> d1=dchisq(x,df=4)

> d2=dchisq(x,df=8)

> d3=dchisq(x,df=10)

> d4=dchisq(x,df=20)

> plot(x,d1,type='l')

> lines(x,d2,type='l',col='blue')

> lines(x,d3,type='l',col='green')

> lines(x,d4,type='l',col='red')


\begin{center}
\includegraphics[width=1\textwidth]{Chicuadrado.png}
\label{Ejercicio 4}
\end{center}

\spart
Vamos a usar el resultado visto en clase:
Si $X\sim \gamma(a,p)$ entonces tenemos que 
\[cX \sim \gamma(c\cdot a, p)\]

En este caso, tomando $c=10$ tenemos:

\[\mathbb{P}\{10X\leq 30\} = \mathbb{P}\{\chi^2_{20 }\leq 30\}  \]

Tenemos varias opciontes. Una de ellas es ir a R y calcularlo con el comando \emph{pchisq(30,20)} = 0.93

Y la otra es irse a las tablas y vemos que $\mathbb{P}\{\chi^2_{20} \leq 30\} \simeq 0.93$

\spart Sea $Y \sim \chi_{200}^2$ 

Podemos hacerlo en R directamente y nos da $\mathbb{P}\{Y\leq 3\} = 10 ^{-141}$

A mano, aplicamos el T.C.L, que dice:
\[\sqrt{n}(\gor{X} - \mu) \convs[d] N(0,\sigma)  \]

Entonces tenemos: $\gor{X} \sim N\left(\esp{X},\displaystyle \sqrt{\frac{\var{X}}{n}}\right)$

Donde $\esp{X} = \esp{Z^2} = \var{Z} = 1$ y $\var{X} = \var{Z^2} = \var{\chi_1^2} = 2$

Con lo que:
\[\gor{X} \sim N\left(1,\frac{1}{10}\right)\]

Sustituyendo y estandarizando:

\[
\mathbb{P}\{\gor{X}\leq\frac{3}{20} \} \simeq \mathbb{P} \{Z\leq \frac{\frac{3}{200} - 1}{\frac{1}{10}} \} = \mathbb{P} \{Z\leq -9.85\} = 3 \cdot 10^{-23}
\]

Una diferencia bastante distinta a lo que decía R. Tras un debate entre Miguel y Amparo de 10 minutos no se ha llegado a ninguna conclusión.
\end{problem}

\begin{problem}[4]
\ppart Utilizando el fichero Datos-lipidos.txt, estima, mediante un intervalo de confianza de nivel
0.95, la proporción de pacientes que tienen una concentración de colesterol superior o igual a
220 mg/dl. ¿Qué tamaço muestral habrá que usar para tener una probabilidad aproximada de 0.95 de no cometer un error mayor que 0.01 en la estimación de esta proporción?

\ppart
\solution
Solucionado por Amprao, descargable 
\href{http://www.uam.es/personal_pdi/ciencias/abaillo/MatEstI/T4DatosLipidos.pdf}{aqui}
\end{problem}
\begin{problem}[5] Sea una v.a. con función de densidad $f(x;\theta) = \theta x^{-(\theta) + 1}\ind_{[1,\infty)} $

\ppart Obtener el e.m.v.

\ppart Obtener su distribución asintótica

\ppart Calcular la cantidad pivotal aproximada y, a partir de ella, un intervalo de confianza de nivel aproximada $1-\alpha$ para $\theta$
\solution
\spart \[\dpa{logL(\theta)}{\theta} = 0 \implies e.m.v.(\theta) = \frac{1}{\gor{Y}}\]
donde $Y = log X_i$

\spart Posibles caminos:

a) $\hat{\theta} \convs[d] $¿?

b) \[\sqrt{n}(\hat{\theta} - \theta) \convs[d] N\left(0,?\right)\]

La primera opción es algo difusa y la segunda es mucho más concreta y mejor.

Tenemos que examinar la expresión $\sqrt{n}(\hat{\theta} - \theta)$
Tenemos 2 posibilidades con las que calcular este tipo de cosas (T.C.L) y método delta (que es el que emplearemos a continuación)

\[\mu = \esp{X}; \sigma = \var{X}\]
\[\sqrt{n}\left(g(\gor{X}) - g(u)\right) \convs N(0,\abs{g'(u)} \sigma\]

Aplicando el método delta:

\[
\sqrt{n}(\hat{\theta} - \theta) = \sqrt{ n}\left(g(\gor{y})-g(\esp{Y})\right)\convs[d] N\left(0,\underbrace{\abs{g'\left(\frac{1}{\theta}\right)}}_{\theta^2} \sqrt{\var{Y}}\right) = N(0,\theta)
\]

Peeero... hay que tener cuidado con que $\theta = g(\esp{Y})$ porque sino no podemos aplicar el método delta.

\[
\var{Y} = \esp{Y^2} - \esp{^2 Y} = \underbrace{\int_1^2 (log\,x)^2 \theta x^{-(\theta + 1)}dx}_{\displaystyle\frac{2}{\theta}} - \frac{1}{\theta^2} = \theta{1}{\theta^2}
\]

\spart
La cantidad pivotal les un estadístico que depende de la muestra y del parámetro desconocido (del que estamos calculando el intervalo) y cuya distribución, al menos asintóticamente) es totalmente conocida.

En el apartado b) hemos encontrado la distribución asintótica para poder construir la cantidad pivotal.

Tipificamos el resultado anterior para evitar que la distribución depende del parámetro desconocido.

\[
\frac{1}{\theta} \sqrt{n}(\hat{\theta} - \theta)  = 
\sqrt{n} \left(\frac{\hat{\theta}}{\theta} - 1 \right) = \mathbb{Q}(\theta;X_1,...,X_N)
\]

Esta es nuestra cantidad pivotal, que depende de la muestra (por el $\hat{\theta}$) y depende del parámetro.

\[1-\alpha  = \mathbb{P} = \{q_1(\alpha) \leq \mathbb{Q}(\theta;X_1,...,X_N) \leq q_2 (\alpha)\}\]


El despejar se deja como ejercicio para el lector.

\end{problem}

\begin{problem}[6]
Sea $\sample$ una muestra de una v.a. uniforme en el interalo $[0,θ]$ con $0 < θ < 1$. Obtener una cantidad pivotal para $θ$ a partir del emv. Usando esta cantidad pivotal construye un intervalo de confianza para $θ$ de nivel prefijado $1-α$.

\solution

El e.m.v es \[ emv (θ) = \hat{θ} = \max X_i \] La cantidad pivotal para $θ = Q(θ; \sample)$

\[ F_{X_{(n)}} (x) = \prob{\hat{θ}_n ≤ x} = \prob{X_{(n)} ≤ x} = \prod_{i=1}^{n} \prob{X_i ≤ x} = \begin{cases}
0& x<0 \\
\left( \frac{x}{θ} \right)^n & 0≤x≤θ \\
1 & x > 1
\end{cases}\]

Tomo $Q(θ; \sample ) = \dfrac{X_{(n)}}{θ} = \dfrac{\hat{\theta}}{n}$, que es válido como cantidad pivotal porque \[ \prob{Q≤x} = \prob{\frac{X_{(n)}}{θ} ≤ x} = \begin{cases}
0 & x<0 \\
x^n & 0≤x≤θ \\
1 & x > 1
\end{cases} \]

Tenemos que elegir dos valores $q_1, q_2$ de tal forma que 

\[ 1- α = \prob{q_1(α) ≤ Q(θ;\sample) ≤ q_2(α)} \]

¿Cómo elegirlos? Queremos buscar que la longitud del intervalo de confianza $IC_{1-α}(θ) = \left(\dfrac{\hat{θ}_n}{q_2},\dfrac{\hat{θ}_n}{q_1}\right)$ sea mínima. Calculamos esa longitud:

\[ \text{len IC} = \hat{θ}_n\left(\frac{1}{q_1}-\frac{1}{q_2}\right)=\hat{θ}_n \left(\frac{q_2-q_1}{q_1q_2}\right) \]

Es decir, tenemos que buscar que $q_1-q_2$ sea más pequeño y además tienen que ser lo mayores posible. Por lo tanto, la elección óptima es 

\[ q_2 = 1,\;q_1=α^{1/n} \]

\end{problem}

\begin{problem}[7] Construye tres intervalos de confianza asintóticos diferentes para el parámetro $λ$ de una distribución de Poisson usando los tres métodos siguientes:

\ppart Utiliza el comportamiento asintótico de la media muestral, estima de forma consistente la varianza y aplica el teorema de Slutsky.

\ppart Igual que el anterior, pero sin estimar la varianza

\ppart Aplicando el método delta para \textit{estabilizar la varianza}, es decir, buscando una función $g$ tal que $\sqrt{n}(g(\avg{X}) - g(λ))\convdist N(0,1)$.

\solution

\spart El TCL (\ref{thmCentral}) nos dice que

\[ \sqrt{n}\frac{\avg{X} - λ}{\sqrt{λ}} \convdist N(0,1) \]

Entonces tenemos que 
\begin{equation}
 1-α = \prob{-z_{α/2}≤\sqrt{n}\frac{\avg{X} - λ}{\sqrt{λ}} ≤ z_{α/2}} \label{eqEj7}
 \end{equation}

Sustituyo $λ$ en el denominador por una estimación consistente $\hat{λ}\convs[P, c.s]λ$:

\[ \sqrt{n}\frac{\avg{X} - λ}{\sqrt{\hat{λ}}} \convdist N(0,1) \]

Como sabemos que $λ=\esp{X}$, tomamos la media muestral como el estimador: $\hat{λ} = \avg{X}$. La convergencia nos queda entonces como


\[ \sqrt{n}\frac{\avg{X} - λ}{\sqrt{\avg{X}}} \convdist N(0,1) \]

y por lo tanto tomamos $ \sqrt{n}\dfrac{\avg{X} - λ}{\sqrt{\avg{X}}}$ como nuestra cantidad pivotal. Despejamos ahora en (\ref{eqEj7}):

\[ \prob{\avg{X} - z_{α/2} \sqrt{\frac{\avg{X}}{n}} 
	≤ λ
	≤ \avg{X} + z_{α/2} \sqrt{\frac{\avg{X}}{n}}}
	\]
	
\spart Partimos de nuevo de (\ref{eqEj7}), pero no tenemos que estimar $λ$. Esta ecuación es equivalente a 

\[ \prob{n\frac{(\avg{X}-λ)^2}{λ} ≤ z_{α/2}^2} \]

De ahí sólo tenemos que despejar $λ$ para hallar nuestro intervalo de confianza.

\spart Tenemos que buscar que se satisfaga la ecuación \[ \sqrt{n}(g(\avg{X}) - g(λ))\convdist N(0,1) \]

Sin embargo, el método delta (\ref{defMetDelta}) nos dice algo distinto:

\[ \sqrt{n}(g(\avg{X}) - g(λ))\convdist N(0,\abs{g'(μ)}\sqrt{\var{X}}) \]

Entonces tenemos que 

\[ \abs{g'(λ)}\sqrt{λ} = 1 \implies g'(λ) = \frac{1}{\sqrt{λ}} \]

e integrando vemos que $g(λ) = 2\sqrt{λ} $.
\end{problem}

\begin{problem}[8]
\ppart Se desea evaluar aproximadamente, por el \textit{método de Montecarlo}, la integral 

\[ p = \int_0^1f(x)\,dx \] 

de una función continua $\appl{f}{[0,1]}{[0,1]}$. Para ello se generan 500 observaciones independientes $(X_i,Y_i)$ con $i=1,\dotsc,500$ con distribución uniforme en el cuadrado $[0,1]×[0,1]$ y se estima $p$ mediante

\[ \hat{p} = \sum_{i=1}^{500} \frac{Z_i}{500} \]

donde la v.a. $Z_i$ vale 1 si $Y_i≤f(X_i)$ y $0$ en caso contrario. ¿Qué distribución tienen las $Z_i$? Suponiendo que, en una muestra concreta hemos obtenido $\sum_{i=1}^{500} z_i = 255$, obtener un intervalo de confianza de nivel $0.99$ para la correspondiente estimación de $p$.

\solution

\spart La v.a. sigue una distribución de Bernoulli, de tal forma que

\begin{equation} \prob{Z=1}=\prob{Y ≤ f(X)} \label{eqEj8} \end{equation}

La distribución de densidad de la v.a. $(X_i, Y_i)$ es 

\[ f(x,y) = \begin{cases}
1 & (x,y) ∈ [0,1]×[0,1] \\
0 & \text{en otro caso}
\end{cases} \]

Aplicando esto en $(\ref{eqEj8})$

\[ \prob{Z=1} = \prob{(X,Y) ∈ \{(x,y)\tq y ≤ f(x) \}} = \int_0^1\int_0^{f(x)} \,dy\,dx = \int_0^1f(x)\,dx = p \]

y llegamos a la forma de estimar la integral que queríamos. 

Vamos a contruir el intervalo de confianza de nuvel $0.99$.

\[IC_{0.99} (p) = \left(\gor{z} \pm Z_{0.005}\sqrt{\frac{\gor{z}(1-\gor{\gz})}{500}}\right) = \left(\hat{p} \pm 2575 \sqrt{\frac{\hat{p}(1-\hat{p})}{500}}) \right) = (0.45\pm 0.057)\]


\spart En este caso sabemos el valor de \[p = \int_0^1 x^2dx = \frac{1}{3}\]
Buscamos un $n$ que cumpla: \[z_{0.005} \sqrt{\frac{\frac{1}{3}\cdot\frac{2}{3}}{n}} \implies n > 14734.72\]

\end{problem}

\begin{problem}[9]
Sea X una v.a. con distribución normal de media $\mu$ y variandza $\theta$. Estamos interesados en la estimación de $\theta$ basados en muestras $X_1,...,X_n$. Si $s^2$ denota la cuasivarianza muestras, calcular $\var{s^2}$ y compararla con la cota de Fréchet-Cramer-Rao obtenida en la relación 3 de problemas.
\solution

Comentarios previos: Sabemos que $s^2$ es un estimador insesgado de \[\var{X} = \frac{1}{n-1} \sum_{i=1}^n (X_i - \gor{X})^2\]

Vamos a calcular $\var{s^2}$

Posibilidades:
\begin{itemize}
\item Aunque es un poco largo\[
\var{s^2} = \esp{s^2}-\left[\esp{s^2}\right]^2
\]

\item Si $X\sim N(\mu,\sigma)$ entonces \[\frac{(n-1)s^2}{\sigma^2} \sim \chi_{n-1}^2\]
\end{itemize}

Vamos a utilizar la segunda opción (es un resultado que pondría una referencia pero no se donde está)

\[
\var{s^2} = \var{\frac{n-1}{\sigma^2}s^2\cdot\frac{\sigma^2}{n-1}} = \frac{\sigma^4}{(n-1)^2} \var{\frac{n-1}{s^2}} s^2 = \frac{\theta^2}{(n-1)^2}2(n-1) = \frac{2\theta^2}{n-1} \]

$s^2$ por lo tanto no es eficiente $\left( \text{porque la Cota de FCR es: } \displaystyle\frac{2\theta}{n}\right)$ Por ser $\theta$ la varianza de una $N(\mu,\sigma)$, de la que nos sabemos de memoria la cota de FCR.


\end{problem}

\section{Tema 5 - Contraste de hipótesis}

\begin{problem}[1] En octubre de 2007 el periódico \textit{The New York Times} realizó un muestreo en 20 restaurantes y tiendas de Nueva York con objeto de analizar la variable $X$, que representa el contenido en ppm de metilmercurio en el sushi de atún que se pone a la venta. La media y la cuasi-desviación típica muestrales obtenidas con estas 20 observaciones de $X$ fueron $\avg{x} = 0.794,\, s=0.2953$. Supongamos que $X$ tiene distribución aproximadamente normal.

\ppart ¿Proporcionan estos datos suficiente evidencia estadística a nivel $0.05$ a favor de la hipótesis de que la concentración media de metilmercurio en las raciones de sushi de atún en la población considerada es superior a 0.6 ppm? El p-valor, ¿es menor o mayor que 0.01?

\ppart Obtener, a partir de estos datos, un intervalo de confianza de nivel 0.95 para la concentración media de metilmercurio $μ$ en toda la población. Calcular el mínimo tamaño muestral mínimo que habría que utilizar para, con una probabilidad de 0.95, estimar la concentración media de metilmercurio con un error máximo de 0.06 ppm.

\solution

\spart Empezamos definiendo la hipótesis nula, que será que $μ≤0.6$ ya que queremos una evidencia muy fuerte para rechazar que la concentración suba del nivel mínimo.

La región de rechazo en este caso es 

\[ R = \{ T > t_{19;α} \}\]

donde \[ T = \frac{\avg{x} - 0.6}{0.2953/\sqrt{20}} = 2.938 \]

Por otra parte, $t_{19;α} = 1.729$. Se cumple la condición de la región de rechazo, por lo tanto rechazamos $H_0$. El p-valor del contraste tendrá que ser menor entonces que $0.05$.

Para saber si el p-valor es menor que $0.01$ calculamos $t_{19;0.01}=2.53$. Como sigue siendo menor que $T$, seguimos rechazando $H_0$ y por lo tanto el p-valor del contraste será menor que $0.01$.

Si quisiésemos obtener el p-valor concreto del contraste, buscaríamos el valor de $α$ tal que $ t_{19;α} = 2.938$. En R, obtendríamos este valor con la orden

\begin{verbatim}
> pt(2.938, 19, lower.tail=FALSE)
[1] 0.004221168
\end{verbatim}

El p-valor es por lo tanto $0.004$. Esto quiere decir que la probabilidad de obtener la muestra que hemos conseguido suponiendo que $H_0$ es cierta (esto es, suponiendo que la media de ppm de metilmercurio en el atún es menor que $0.6$) es extremadamente baja, y o bien hemos obtenido una muestra muy, muy extraña o $H_0$ es falsa. Por lo tanto, lo razonable sería rechazar la hipótesis nula y decir que, de media, la concentración de metilmercurio media es mayor que $0.6$.

\spart El intervalo de confianza sería 

\[ IC_{0.95} (μ) = \left(\avg{x}\pm t_{n-1;\frac{α}{2}}\frac{s}{\sqrt{n}} \right) = (0.656, 0.932) \]

Como además $0.6\notin IC_{0.95}(μ)$, rechazaríamos $H_0:\,μ=0.06$ a nivel $α=0.05$.

Para hallar el tamñao muestral mínimo buscamos que 

\[ IC_{0.95}(μ) = (\avg{x} \pm 0.06)\]

Despejando, tenemos que resolver

\[ t_{n-1;0.025}\frac{s}{\sqrt{n}} < 0.06\]

Como no conocemos $s$, lo sustituimos por una aproximación, la cuasivarianza muestral de los 20 restaurantes que teníamos al principio. Además, intuimos que $n$ va a ser grande y por lo tanto $t$ se aproximaría a una distribución normal $Z = N(0,1)$, y por lo tanto

\[ t_{n-1;0.025} ≈ z_{0.025} = 1.96 \]

y entonces $n > 93$.
\end{problem}

\begin{problem}[8] \ppart Supongamos que en una determinada población de referencia, formada por adultos sanos, el nivel en sangre de la enzima hepática GGT (gamma-glutamil-transpeptidasa) sigue aproximadamente una distribución normal con media polacional $42 IU/L$ y desviación típica poblacional 13. Calcular aproximadamente el porcentaje de personas en la población que tienen un nivel de GGT superior a 80.
\ppart Supongamos ahora que se selecciona una muestra de 61 personas en otra población formada por bebedores habituales no diagnosticados de alcoholismo y se obtiene una media muestra de 58 IU/L con una desviación típica de 21. ¿Hay suficiente evidencia estadística, al nivel 0.05, para afirmar que la concentración media de GGT en la población de bebedores es mayor que 42?

\solution

Sí.

\end{problem}

\newpage
\printindex
\end{document}